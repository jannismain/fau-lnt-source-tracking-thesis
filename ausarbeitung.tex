\documentclass[12pt,twoside,a4paper]{report} % Die Option a4paper kann n�tig sein, um das PDF drucken zu k�nnen.

%\makeindex
\usepackage{german,a4}
\usepackage{color}
\usepackage[dvips]{graphicx}
\usepackage{psfrag}
\usepackage{epsfig}
\usepackage[latin1]{inputenc}
\usepackage{fancyhdr}
\usepackage{biblatex}
\usepackage{caption}
\usepackage{tikz}
\usepackage{iflang}

% set image-path
\graphicspath{{images/}}
\renewcommand{\baselinestretch}{1.5}
\renewcommand{\captionfont}{\linespread{1}\normalsize}
\renewcommand{\textfraction}{0}
\newcommand{\matlab}{Matlab\textsuperscript{\textregistered} }
\newcommand{\alt}[1]{}

\newcommand{\mathgloss}[2]{%
    \newglossaryentry{#1}{name={#1},description={#2}}%
    \begin{description}[labelwidth=3em]%
      \item[\gls{#1}]#2%
    \end{description}%
}

%% MATH MAKROS
\counterwithout{equation}{chapter}  % to have consecutive numbers for equations
\DeclareMathOperator*{\argmin}{arg\,min}
\DeclareMathOperator*{\argmax}{arg\,max}
\newcommand{\given}[1]{\vert #1} % for more expressive parameters
\newcommand\numberthis{\addtocounter{equation}{1}\tag{\theequation}}

%% CODE MAKROS
\newcommand{\code}[1]{\lstinline[basicstyle=\ttfamily]{#1}}  % used to display code fragments inline

%% TEXT MAKROS
\setlist{noitemsep,topsep=0pt,parsep=0pt,partopsep=0pt}  % to have bullet points in a more condensed form

% Dimensions of...
% ...site
\setlength{\textwidth}{15.5 true cm}
\setlength{\textheight}{22 true cm}
% ...margin
\oddsidemargin  0.5 cm
\evensidemargin 0.5 cm
\topmargin      0 cm
\parindent 		0 cm

% Language
\selectlanguage{english}

% Bibliography
\bibliography{bibliography/source_tracking.bib}


\renewcommand{\baselinestretch}{1.5}
\renewcommand{\captionfont}{\linespread{1}\normalsize}
\renewcommand{\textfraction}{0}

% Fancyhdr
\setlength{\headheight}{0.75cm}
\fancypagestyle{plain}{
	\fancyhf{}
	\fancyhead[EL,OR]{\thepage}
	\fancyhead[ER]{\leftmark}
	\fancyhead[OL]{\leftmark}
}
\fancypagestyle{newfancy}{
	\fancyhf{}
	\fancyhead[EL,OR]{\thepage}
%	\fancyhead[ER]{\rightmark}
	\fancyhead[OL]{\rightmark}
	}
\pagestyle{newfancy}

\renewcommand{\chaptermark}[1]{\markboth{\uppercase{\chaptername \ \thechapter.\ #1}}{}}
\renewcommand{\sectionmark}[1]{ \markright{ \uppercase{\thesection.\ #1}}{}}

%%%%%%%%%%%%%%%%%%%%%%%%%%%%%%%%%%    C O N F I G    %%%%%%%%%%%%%%%%%%%%%%%%%%%%%%%%%%%
%%%%%%%%%%%%%%%%%%%%%%%%%%%%%%%%%%%%%%%%%%%%%%%%%%%%%%%%%%%%%%%%%%%%%%%%%%%%%%%%%%%%%%%%
%%%%%%%%%%%%%%%%%%%%%%%%%%%%%%%%%%  D O C U M E N T  %%%%%%%%%%%%%%%%%%%%%%%%%%%%%%%%%%%

\begin{document}

% TITLE
\begin{titlepage}
\begin{center}
\vspace*{-1cm}
 {\LARGE Friedrich-Alexander-University Erlangen-Nuremberg}\\
\vspace{1cm}
 {\Large \textbf{Chair for Multimedia Communication und Signal Processing}}\\
\vspace{1cm}
 {\Large Prof. Dr.-Ing. Walter Kellermann}\\
\vspace{3cm}
 {\LARGE Master Thesis}\\
\vspace{2cm}
 {\LARGE \textbf{Source Tracking in Acoustical Sensor Networks}}\\
\vspace{2cm}
{\LARGE by Jannis Mainczyk}\\
\vspace{3cm}
{\Large November 2017}\\
\vspace{1cm}
{\Large Advisor: Andreas Brendel M.Sc.}
\end{center}
\end{titlepage}

\thispagestyle{empty}
\section*{}
\newpage



\addtocounter{page}{-1}

% Declaration
\newpage
\chapter*{Declaration of Authorship}
\thispagestyle{empty}
% !TEX encoding = UTF-8 Unicode
\vspace*{1cm}

\large
\noindent
\IfLanguageName{ngerman}
{Ich versichere, dass ich die vorliegende Arbeit ohne fremde Hilfe und ohne Benutzung anderer als der angegebenen Quellen angefertigt \nobreak habe, und dass die Arbeit in gleicher oder \"ahnlicher Form noch keiner anderen Pr\"ufungsbeh\"orde vorgelegen hat und von dieser als Teil einer Pr\"ufungsleistung angenommen wurde. Alle Ausf\"uhrungen, die w\"ortlich oder sinngem\"a\s{} übernommen wurden, sind als solche gekennzeichnet.}
{I assure that I have produced the present work without the help of others and without using any sources other than those specified and that the work has not been submitted in the same or similar form to any other examination body and has been accepted as part of an examination. All statements, which have been taken literally or meaningfully, are marked as such.}
\vspace{3cm}

\begin{minipage}{0.45\textwidth}
	------------------------------------
	
	\IfLanguageName{ngerman}{Ort, Datum}{Location, Date}
	
\end{minipage}
\begin{minipage}{0.45\textwidth}
	------------------------------------
	
	\IfLanguageName{ngerman}{Unterschrift}{Signature}

	
\end{minipage}

\normalsize
\pagenumbering{Roman}

\thispagestyle{empty}
\section*{}
\newpage



\thispagestyle{empty}
\cleardoublepage

% Table of Contents
\addtocounter{page}{-2}
\tableofcontents
\clearpage


% Summary
\renewcommand{\chaptermark}[1]{ \markboth{ \uppercase{#1}}{}}

\thispagestyle{empty}
\section*{}
\newpage



\chaptermark{Abstract}
\renewcommand{\chaptermark}[1]{\markboth{\uppercase{\chaptername \ \thechapter.\ #1}}{}}

\chapter*{Abstract}
\addcontentsline{toc}{chapter}{Abstract}
Kurzfassung
\clearpage

% Abbreviations
\renewcommand{\chaptermark}[1]{ \markboth{ \uppercase{#1}}{}}
\chaptermark{List of Abbreviations}
\renewcommand{\chaptermark}[1]{\markboth{\uppercase{\chaptername \ \thechapter.\ #1}}{}}
\chapter*{List of Abbreviations}
\addcontentsline{toc}{chapter}{List of Abbreviations}
\begin{tabular}{p{6cm}l}
DOA			&	degree of arrival\\
GMM         &   gaussian mixture model\\
STFT        &   short-time fourier transformation\\
\end{tabular}

\clearpage

% List of Symbols
\renewcommand{\chaptermark}[1]{ \markboth{ \uppercase{#1}}{}}
\renewcommand{\sectionmark}[1]{ \markright{ \uppercase{#1}}{}}
\chaptermark{List of Symbols}
\sectionmark{List of Symbols}
\renewcommand{\chaptermark}[1]{\markboth{\uppercase{\chaptername \ \thechapter.\ #1}}{}}
\renewcommand{\sectionmark}[1]{ \markright{ \uppercase{\thesection.\ #1}}{}}

\chapter*{List of Symbols}
\addcontentsline{toc}{chapter}{List of Symbols}
\renewcommand{\chaptermark}[1]{ \markboth{ \uppercase{#1}}{}} 
\renewcommand{\sectionmark}[1]{ \markright{ \uppercase{#1}}{}} 
\chaptermark{List of Symbols}
\sectionmark{List of Symbols}
\renewcommand{\chaptermark}[1]{\markboth{\uppercase{\chaptername \ \thechapter.\ #1}}{}} 
\renewcommand{\sectionmark}[1]{ \markright{ \uppercase{\thesection.\ #1}}{}} 

\chapter*{List of Symbols}
\addcontentsline{toc}{chapter}{List of Symbols}

\subsubsection*{Notational Remarks}
The following notational conventions prevail throughout this thesis: A regular lowercase letter indicates a scalar ($x$), a bold lowercase letter indicates a vector ($\bm x$), and a bold capital letter indicates a matrix ($\bm X$). A mathematical set is denoted by an uppercase calligraphic letter ($\mathcal{X}$). An exception to this is a calligraphic letter with function parameters ($\mathcal{L}(x)$), which remains a mathematical function. An estimated parameter is indicated by a hat ($\hat{\bm x}$)

%Positions in the cartesian coordinate system are described by their location vector $\bm p_{\text{index}}=[x, y, z]$. These are used for microphone locations $\bm p_m^i$ and source locations $\bm p_s$. The set of all possible positions is described by $\pall$.

%Sources are addressed by their index $s=1,\dots,S$ and microphones are addressed by their microphone pair and microphone index $(m,i)$, where $m\in[1, \dots, M]$ and $i\in[1, 2]$. A braced superscript $l$, like $\psi^{(l)}$, denotes the iteration the \gls{em} algorithm is currently in, so $\psi^{(l-1)}$ indicates the value of $\psi$ of the preceding iteration. As recursions will be iterated with respects to the time-index, superscript $(t)$ (e.g. $\psi^{(t)}$) is used respectively. 

%% DEFINITIONS %%

% Single Symbols
\newcommand{\prp}{\bm\phi}

\newcommand{\p}{\bm p}
\newcommand{\ps}{\p_s}
\newcommand{\psest}{\hat\p_s}
\newcommand{\pr}{\p_m^i}
\newcommand{\pall}{\mathcal{P}}
\newcommand{\pinp}{\p \in \pall}

\newcommand{\psip}{\psi_{\bm p}}
\newcommand{\psiRlast}{\bm\psi_R^{(t-1)}}
\newcommand{\psiRPlast}{\psi_{\bm p,R}^{(t-1)}}
\newcommand{\psiRnow}{\bm\psi_R^{(t)}}
\newcommand{\psiRPnow}{\psi_{\bm p,R}^{(t)}}
\newcommand{\psips}{\psi_{s\bm p}}  

% Functions
\newcommand{\gaussian}[1]{\mathcal{N}\left (#1\right )}
\newcommand{\pdffunc}{\psi_{s\bm p}^{(l-1)}\prod_{m}\mathcal{N}^c\left (\phi^k_m(t,k);\tilde\phi^k_m(\bm p),\sigma_s^{2,(l-1)}\right )}
\newcommand{\pdffuncR}{\psi_{\bm p,R}^{(t-1)}\prod_{m}\mathcal{N}^c\left (\phi_m(t,k);\tilde\phi^k_m(\bm p),\sigma_R^{2,(t-1)}\right )}
\newcommand{\gauss}{\mathcal{N}^c\big(\phi_m(t,k);\tilde\phi^k_m,\sigma_s^2\big)}
\newcommand{\Q}{\mathcal{Q}\left (\theta\vert\theta^{(l-1)}\right )}
\newcommand{\mulast}{h^{(l)}(t,k,s,\bm p)}
\newcommand{\muRlast}{h(t,k,\bm p)}
\newcommand{\z}{z(t,k,s,\bm p)}
\newcommand{\lcompl}{\prod_{t,k}\sum_{s,\bm p}\psips\cdot\z\prod_{m}\gauss}

% Other
\newcommand{\vect}[1]{\mathbf{#1}}
%\newcommand{\norm}[1]{|{#1}|_2}
\DeclarePairedDelimiter{\abs}{\lvert}{\rvert}
\DeclarePairedDelimiter{\norm}{\lVert}{\rVert}
%\newcommand{\abs}[1]{|{#1}|_1}
\newcommand{\Tsixty}{T$_{60}$\ }

%% TABLE %%
\begin{longtable*}{lp{13cm}}
\multicolumn{2}{@{}l@{}}{\textbf{Mathematical Operators}} \\[2pt]
	$\abs{\cdot}$    & Absolute value (when applied to scalar)\\
	$\abs{\cdot}$    & Cardinality (when applied to set)\\
	$\norm{\cdot}_{_p}$  & $p$-Norm, where $p\in [1,\infty )$\\
	$(\cdot)^{\text{T}}$  & Transpose of vector or matrix \\
	det$(\cdot)$  & Determinant of matrix\\
	vec$_a(\cdot)$ & Vector concatenation of a scalar or vector across all indices $a$\\
	&e.g., $\bm x = \text{vec}_a(x_a)=[x_1~x_2~\dots~x_n]^{\text{T}}$, $a\in[1,\dots,n] $.\\
	diag$(\cdot)$ & Diagonal matrix of vector\\
	&e.g., $\text{diag}(\bm x)=
    \begin{bmatrix}
    x_1 &     0     & \dots  & 0 \\
       0      & x_2 & \dots  & 0 \\
       \vdots &     \vdots     & \ddots & \vdots\\
    0         &     0     & \dots  &  x_n
\end{bmatrix}$.\\ \pagebreak

\multicolumn{2}{@{}l@{}}{\textbf{Scalars}} \\[2pt]
    $c$         & Speed of sound ($c=343\frac{m}{s}$)\\
    $d_m$       & Distance of microphones within microphone pair\\
    $d^w_m$     & Wall distance of microphone pairs\\
    $d^w_{s,\text{min}}$     & Minimum wall distance of sources\\
    $d_{s,\text{min}}$     & Minimum distance of sources\\
	$K$         & Number of frequency bins \\
	$L$         & Maximum number of EM iterations \\
	$M$         & Number of microphone pairs\\
	$P$         & Number of grid points\\
	$S$         & Number of sources\\
	$T$         & Number of time-steps \\
	$T_s$       & Sampling period\\
	$\epsilon_s$ & Localisation error of source $s$\\
	$\gamma_t$    & Step size of recursive update\\
    $\iota$     & Imaginary unit ($\iota=\sqrt{-1}$)\\[6pt]
	$\psips$    & Weight of Gaussian mixture component corresponding to position $\bm p$ and source $s$ \\

\multicolumn{2}{@{}l@{}}{\textbf{Indices}} \\[2pt]
    $i$         & Microphone index ($i\in\{1,2\}$) \\
    $j$         & Gaussian component index \\
    $k$         & Frequency bin index ($k\in\{1,\dots,K\}$)\\
    $l$         & EM-Algorithm iteration index ($l\in\{1,\dots,L\}$)\\
    $m$         & Microphone pair index ($m\in\{1,\dots,M\}$)\\
    $s$         & Source index ($s\in\{1,\dots,S\}$)\\
    $t$         & Time-bin index ($t\in\{1,\dots,T\}$)\\[6pt]

\multicolumn{2}{@{}l@{}}{\textbf{Vectors}} \\[2pt]
	$\p $      & Location vector of grid point\\
	$\ps $      & Location vector of source $s$ \\
	$\pr $      & Location vector of microphone $(m,i)$\\
	$\bm\beta$ & Vector of wall reflection coefficients ($\bm\beta = [\beta_1~\beta_2~\dots~\beta_6]$)\\
	$\prp$      & Concatenated vector of \acrshortpl{prp} across all microphone pairs $m$ and time-frequency-bins $(t,k)$ \\
	$\prp_m$      & Pair-wise relative phase ratio at microphone pair $m$ \\
	$\tilde\prp_m^k$ & Expected \acrshort{prp} for frequency bin $k$ at microphone pair $m$ \\
	$\bm\psi$      & Vector of weights of Gaussian mixture components across all positions $\bm p$ and sources $s$ \\[6pt]
	
\multicolumn{2}{@{}l@{}}{\textbf{Sets}} \\[2pt]
	$\pall$    & Set of all grid points\\
	$\mathcal{P}_s$ & Set of all source positions ($\mathcal{P}_s \subseteq \pall$)\\[6pt]
	
\multicolumn{2}{@{}l@{}}{\textbf{Signals}} \\[2pt]
	$v_s$      & Source signal of source $s$ \\
	$x_m^i $      & Received signal at microphone $(m,i)$\\
	$a^i_{sm} $      & Transfer function of source $s$ and microphone $(m,i)$ \\
	$n^i_m$      & Additive spectral and temporal white Gaussian noise at microphone $(m,i)$ \\[6pt]
\end{longtable*}

\clearpage

\thispagestyle{empty}
\section*{}
\newpage





%%%%%%%%%%%%%%%%%%%  Main Text  %%%%%%%%%%%%%%%%%%%

% Formula Margins
\setlength{\belowdisplayskip}{0.5cm}
\setlength{\belowdisplayshortskip}{0.5cm}
\pagenumbering{arabic}

\renewcommand{\sectionmark}[1]{ \markright{ \uppercase{\chaptername \ \thechapter.\ #1}}{}}
\chapter{Introduction}
\chaptermark{Introduction}
\sectionmark{Introduction}
\label{chap:intro}
\chapter{Introduction}
\chaptermark{Introduction}
\sectionmark{Introduction}
\label{chap:intro}
Here comes the introduction...

\renewcommand{\sectionmark}[1]{ \markright{ \uppercase{\thesection.\ #1}}{}}
\chapter[Theoretical Background]{Theoretical Background}
\chaptermark{Theoretical Background}
\label{chap:Theoretical Background}
\chapter[Theoretical Background]{Theoretical Background}
\chaptermark{Theoretical Background}
\label{chap:Theoretical Background}
To understand the source tracking algorithm introduced in the subsequent chapters, a firm understanding of the Expectation-Maximization-Algorithm (hereafter called EM-Algorithm), as well as Gaussian Mixture Models (GMM) is required. Lastly, also the basic signal processing concepts are revised, which will be focused on the properties of the system at hand (multiple sources in a reverberant and noisy environment) and the application of the short-time fourier transformation (STFT) as a way to solve the problem at hand in the frequency-domain.

\section{EM-Algorithmus}
\label{sec:em}
\section{EM-Algorithmus}
\label{sec:em}
The Expectation-Maximization-Algorithm (EM-Algorithm) is an important algorithm in probabilistic theory \cite{Schwartz2014}.

%%%%%%%%%%%%%%%%%%%%%%%%%%%%%%%%%%  D O C U M E N T  %%%%%%%%%%%%%%%%%%%%%%%%%%%%%%%%%%%
%%%%%%%%%%%%%%%%%%%%%%%%%%%%%%%%%%%%%%%%%%%%%%%%%%%%%%%%%%%%%%%%%%%%%%%%%%%%%%%%%%%%%%%%
%%%%%%%%%%%%%%%%%%%%%%%%%%%%%%%%%%  A P P E N D I X  %%%%%%%%%%%%%%%%%%%%%%%%%%%%%%%%%%%

% ANHANG
\begin{appendix}
	\renewcommand{\chaptermark}[1]{\markboth{\uppercase{Anhang \ \thechapter.\ #1}}{}}
	
	\chapter{Anhang Kapitel}
	\chaptermark{Anhang Kapitel}
	
	\section{Anhang Abschnitt}
	\label{sec:AAbschnitt}
	% \input{}
	\clearpage
	\newpage
	\renewcommand{\chaptermark}[1]{\markboth{\uppercase{\chaptername \ \thechapter.\ #1}}{}}
\end{appendix}

% ABBILDUNGSVERZEICHNIS
\addcontentsline{toc}{chapter}{Abbildungsverzeichnis}
\listoffigures

% TABELLENVERZIECHNIS
\addcontentsline{toc}{chapter}{Tabellenverzeichnis}
\listoftables

% LITERATURVERZEICHNIS
%nicht referenzierte Literaturstellen
\nocite{}

% Literaturverzeichnis einbinden, alpha, plain, unsrt, abbrv
\newpage
%Eintrag im Inhaltsverzeichnis
\addtocounter{page}{1}
\addcontentsline{toc}{chapter}{Literaturverzeichnis}
\addtocounter{page}{-1}
\bibstyle{plain}
\cite{Schwartz2014}
\citation{Schwartz2014}


\end{document}
