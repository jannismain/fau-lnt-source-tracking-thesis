\chapter{Description of Algorithms}
\label{chap:algorithms}
The following algorithms are mainly adopted from \cite{Schwartz2014}. Different from the implementation outlined there, there is only one single probability matrix $\psi$, instead of one matrix per source $\psi_{s}$
\begin{itemize}
	\item Summarize sources of Schwartz2014
	\item Describe alterations to the algorithms
\end{itemize}

\section{Location Estimation}
\label{sec:algLocEst}
\begin{itemize}
	\item Hidden Data
	\item Observations
	\item Target Parameters
\end{itemize}

The \gls{em} algorithm requires three datasets: the \emph{target parameters}, the \emph{observations} and the \emph{hidden data}. For the source localisation algorithm, the hidden data is defined as an indicator-variable $x$, that a certain time-frequency bin $(t,f)$ belongs to an active source at the position $\vect{p}$. 
\begin{equation}
	x=(t,f,\vect{p})
\end{equation}

The target parameters are the gaussian's standard deviation $\sigma^2$ and it's probability of being assigned to an active source $\psi_{s,vect{p}}$, and will henceforth be referenced as:
\begin{equation}
	\theta=[\psi, \sigma^2]
\end{equation}

Localisation in this context simply means estimating, which $n$ gaussians best resemble the active sources $s$.

\section{Source Tracking}
\label{sec:algSrcTrack}

\newacronym{rem}{REM}{Recursive \acrfull{em}}
\newacronym{crem}{CREM}{\citeauthor{Cappe2009}'s Recursive Expectation-Maximisation}
\newacronym{trem}{TREM}{Titterington Recursive Expectation-Maximisation}

To integrate data from prior steps into each \gls{em} iteration, recursive variations of the \gls{em} algorithm have been developed. These will be called \gls{rem} algorithms in the upcoming sections.

In \cite{Schwartz2014}, two different \gls{rem} algorithms for speaker tracking are proposed. The first one is based on \citeauthor{Cappe2009}'s \gls{rem} and will be referenced as \acrshort{crem}, whereas the second one is based on \citeauthor{Titterington1984}'s \gls{rem} algorithm and will be referenced as \acrshort{trem}.

\subsection{CREM Algorithm}
\label{sec:crem}
\import{equations/}{alg-crem}

\subsection{TREM Algorithm}
\label{sec:trem}
\import{equations/}{alg-trem}
