\section{Simulation of Room Acoustics}
\label{sec:simulation}

\begin{figure}[!b]
\centering
    \includegraphics[width=\textwidth]{data/figures/image-method3}
    \caption[Second order image expansion for a single source and receiver]{Second order image expansion for a single source and receiver: \itshape The black rectangle represents the room that is being simulated, containing the source (black dot) and the receiver (black circle). The red rectangles represent the first-order image expansion, where every signal wave arriving at the receiver after being reflected one time (dotted red arrows), is simulated as a direct arrival wave (red arrows) to the virtual source (red dot) in the image of the room, mirrored at the reflection wall. The grey rectangles represent the second-order images for signal waves that are subject to two reflections (dotted grey arrow).}
    \label{fig:imageMethod}
\end{figure}
%TODO: Change orientation of arrows in figure! (sources are mirrored, not receivers)

\paragraph{Image-Source Method}
To conduct the experimental part this thesis, simulation as means of data collection has been chosen over real recordings, as simulated acoustic environments are more flexible compared to a laboratory environment. To simulate the experimental setup \alt{in it's different configurations} described in Section \ref{sec:setup}, the image-source method for small-room acoustics is used \cite{Allen1979}. The basic idea of this method is to simulate an arriving signal wave reflected from the walls as a wave arriving directly from a virtual sources, mirrored at the reflecting wall.\alt{direct arrival waves of a mirror image of the room on the reflecting wall.} Two parameters determine, how many images are created during simulation with everything else being equal: \Tsixty, the time it takes the source signal to have decreased by 60dB after the exciting source is switched off, and the reflection order $r$, which is equal to the maximum order of the image expansion. Instead of explicitly stating \Tsixty, the wall reflection coefficients $\beta = [\beta_1, \beta_2,\dots,\beta_6]$ could also be provided, simulating different types walls, like concrete walls or walls with sound proofing. There is one coefficient per wall, including the ceiling and the floor, for a total of six coefficients. Whenever the signal crosses a wall into another image (i.e., being reflected), it is attenuated by the $\beta$-coefficient of that wall. 

%Figure~\ref{fig:imageMethod} shows the second-order image expansion, meaning only received signals of up to two reflections are displayed. The direct propagation path is shown in black, the first order images (including real and virtual propagation paths) are shown in orange and translucent orange respectively, and the second order images (propagation paths omitted) are shown in gray.

This method allowed for experiments, where a controlled acoustic environment in form of a small, rectangular room is required, that can be easily adjusted. For many experiments\alt{experiments with budget- and time-constraints}, constructing such an environment in a laboratory is prohibitive, which is why the image method has gained widespread popularity since its inception in \citeyear{Allen1979}~\cite{Allen1979} and has been used in a wide range of studies \cite{Champagne1996}.
%TODO: Cite more studies using the image method

\paragraph{Simulation Parameters}
The microphones are simulated to be omnidirectional, meaning they record sound equally well for all directions, mounted at 1m height and directed towards the middle of the room.