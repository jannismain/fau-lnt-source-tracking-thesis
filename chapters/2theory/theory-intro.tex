\chapter{Theoretical Background}
\label{chap:2theory}

In this section, the main theoretical concepts needed for speech source localisation and tracking, as it is implemented here, are laid out and put into context. First, the signal model is defined and it is shown, which features of the signal are exploited to estimate the locations of their origin. As we are simulating the received signal, the image-source model for simulating a received signal influenced by room acoustics and noise, is shown and it's limitations are discussed. Next, the \gls{gmm} is introduced as a probabilistic model of the possible locations of the audio sources, that are to be estimated. Last, the \gls{em} algorithm is explained and it is shown, how it can be used to estimate the parameters of a \gls{gmm} to yield the most likely source locations, with respect to the underlying probabilistic model.