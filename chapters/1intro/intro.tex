\chapter{Introduction}

\section{Motivation}
% Application Scenarios
Interest in the acoustic source localisation and tracking research field is driven by the wide variety of possible application scenarios. With the recent proliferation of smart-home devices, speech has become an important interface for users to interact with these devices in a multitude of different environments. For products like the Amazon Echo or Google Home Speakers, voice is the main interface through which users access the products functionality, making the task of reliable speech recognition detrimental to the market success of these products. Among many other factors, the accuracy of these speech recognition schemes depend on the quality of the received signal. Spatial filtering can increase the quality of the received signal by excluding some of the noise, reverberation and interfering speakers. Therefore, source localisation and source tracking can be seen as important preprocessing steps for reliable speech recognition in adverse environments. Other application scenarios, where speaker localisation and tracking could aid speech recognition, include robot audition \cite{Frechette2012}, \cite{Evers2015} and automatic speech transcription in meeting room \cite{Busso2005} or lecture hall \cite{Parviainen2006} environments. In \cite{Frechette2012}, the word recognition rate of a robot was improved by incorporating source seperation, localisation and tracking schemes into the robot's audition module.  Apart from speech recognition, source localisation and tracking can also be useful in and of itself, one example being automated steering for surveillance or teleconference cameras \cite{Wang1997}.

\section{Overview of Existing Literature}
\label{chap:1literature_review}

A preliminary exploration of the acoustic source localisation and tracking literature shall provide an understanding of the progress that has been made so far, the current state-of-the-art and the most promising ways forward in this diverse research field. \alt{This branch of research has its roots in the...} 
% Research fields
Historically, the speech enhancement research field consisted of two branches, namely \textit{microphone array processing} and \textit{blind source separation} \cite[p. 693]{Gannot2017}. The former was focused on the localization and enhancement of speech signals collected by sensor arrays in noisy and reverberant environments, whereas the later was concerned with the "cocktail party scenario" that involved several sound sources mixed together. From today's perspective, one can easily see, how these two areas in part rely on the same methods and techniques to solve very similar problems. So naturally, these branched have merged over the last decades and are "hardly distinguishable today". 





\section{Setting Constraints}
\label{chap:1constraints}

This thesis is approaching the problem of source localisation and tracking using the \gls{em} algorithm to estimate the parameters of a \gls{gmm}. Other methods, like particle filters or Kalman filters, that are often used in this context and provide promising results (see \cite{Lehmann2007} for source tracking using a particle filter and \cite{Gannot2012} for source tracking using a Kalman filter), are not subject of this thesis. In addition, as this thesis mostly reimplements the algorithms for location estimation and tracking found in \cite{Schwartz2014}, the focus of this thesis will be on the evaluation of these algorithms performance in varying environments, rather than the algorithm derivation itself. For the derivation itself, only the core elements shall be reiterated.
\begin{itemize}
	\item Number of speakers a-priori known
	\item Source Tracking: CREM or TREM or both?
\end{itemize}

\section{Structure of the Thesis}
\label{chap:1structure}

% Categorization of own approach
This thesis adapts the \gls{em} algorithm for the purpose of acoustic source localisation and uses a recursive extension of the \gls{em} algorithm for acoustic source tracking. The \gls{em} algorithm is used to estimate the parameters of a \gls{gmm}. The algorithm, that will be presented in this thesis, was proposed by \citeauthor{Schwartz2014} \cite{Schwartz2014}. The derivation of the algorithm follows the original derivation outlined in \cite{Schwartz2014}. Therefore, apart from a minor adjustment to the dimensionality of the estimated parameters, the main focus of this thesis lies on the algorithm implementation as well as the performance evaluation for varying parameter sets.

The structure of the subsequent chapters is as follows: First, the theoretical concepts that build the foundation of the algorithms source localisation and tracking are presented in \autoref{chap:2theory}. Then, the algorithms in question are derived in \autoref{chap:algorithms}. The experimental part of this thesis is presented in \autoref{chap:experiments}, where the methodology is explained, key implementation details are highlighted and then results are presented and discussed thoroughly. This thesis concludes with \autoref{chap:concl} and a summary of the main findings as well as a critical review and possible directions for future research.


%Beginning with \autoref{chap:2theory}, the theoretical concepts that the algorithms for source localisation and tracking reviewed in this thesis are based upon will be introduced. Subsequently, in Chapter \ref{chap:algorithms} these algorithms will then be formally defined. In Chapter \ref{chap:implementation} the implementation of these algorithms in \matlab is shown and test scenarios to evaluate these implementations are defined. The results of these tests are presented and discussed in Chapter \ref{chap:results}. Finally, the findings of this thesis are summarized in Chapter \ref{chap:concl} and a conclusion towards the applicability of these findings to the general performance of the reviewed algorithms is drawn.
	