\section{Notation}
\label{chap:1notation}

For easier understanding of the mathematical terms included within this thesis, the norms of mathematical notation are followed. A regular font indicates scalars and a bold font indicates vectors ($\bm\psi$, $\bm\phi$,~\dots), if not stated differently.

%For all signals in the STFT domain, $t=1,2,\dots, T$ denotes the time-index, whereas $k=1,2,\dots,K$ denotes the frequency index.
Positions in the cartesian coordinate system are described by their location vector $\bm p_{\text{index}}=[x, y, z]$ and are used for receiver locations $\bm p_m^i$ and source locations $\bm p_s$. The set of all possible positions is described by $\pall$. 

%Sources are addressed by their index $s=1,\dots,S$ and receivers are addressed by their receiver pair $m\in[1, \dots, M]$ and pair index $i\in[1, 2]$. A braced superscript $l$, like $\psi^{(l)}$, denotes the iteration the \gls{em} algorithm is currently in, so $\psi^{(l-1)}$ indicates the value of $\psi$ of the preceding iteration. As recursions will be iterated with respects to the time-index, superscript $(t)$ (e.g. $\psi^{(t)}$) is used respectively. 
The diag$(\cdot)$ operator describes a diagonal matrix, where the elements in braces are placed on the diagonal, whereas all other entries of the matrix are equal to 0.
