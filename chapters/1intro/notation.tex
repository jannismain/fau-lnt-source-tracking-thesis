\section{Notation}
\label{chap:1notation}

For easier understanding of the mathematical terms included within this thesis, the norms of mathematical notation are followed. A bold font indicates vectors ($\bm\psi$, $\bm v$,~\dots).

For all signals in the STFT domain, $t=1,2,\dots, T$ denotes the time-index, whereas $k=1,2,\dots,K$ denotes the frequency index.
Positions in the cartesian coordinate system are described by their location vector $\bm p_{\text{index}}=[x, y, z]$ and are used for receiver locations $\bm p_m^i$ and source locations $\bm p_s$. Note, that $\bm p$ without index is reserved for the set of all possible positions. Sources are addressed by their index $s=1,\dots,S$ and receivers are addressed by their receiver pair $m\in[1, \dots, M]$ and pair index $i\in[1, 2]$. A braced superscript $l$, like $\psi^{(l)}$, denotes the iteration the \gls{em} algorithm is currently in, so $\psi^{(l-1)}$ indicates the value of $\psi$ of the preceding iteration. As recursions will be iterated with respects to the time-index, superscript $(t)$ (e.g. $\psi^{(t)}$) is used respectively. The diag$(\cdot)$ operator describes a diagonal matrix, where the elements in braces are placed on the diagonal, whereas all other entries of the matrix are equal to 0
\begin{equation}
    \text{diag}(\lambda_1, \lambda_2, \cdots, \lambda_N)=
    \begin{bmatrix}
    \lambda_1 &     0     & \dots  & 0 \\
       0      & \lambda_2 & \dots  & 0 \\
       \vdots &     \vdots     & \ddots & \vdots\\
    0         &     0     & \dots  &  \lambda_N
\end{bmatrix}.    
\end{equation}
