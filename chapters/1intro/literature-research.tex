\section{Overview of Existing Literature}
\label{chap:1literature_review}

A preliminary exploration of the acoustic source localisation and tracking literature shall provide an understanding of the progress that has been made so far, the current state-of-the-art and the most promising ways forward in this diverse research field. \alt{This branch of research has its roots in the...} 
% Research fields
Historically, the speech enhancement research field consisted of two branches, namely \textit{microphone array processing} and \textit{blind source separation} \cite[p. 693]{Gannot2017}. The former was focused on the localization and enhancement of speech signals collected by sensor arrays in noisy and reverberant environments, whereas the later was concerned with the "cocktail party scenario" that involved several sound sources mixed together. From today's perspective, one can easily see, how these two areas in part rely on the same methods and techniques to solve very similar problems. So naturally, these branched have merged over the last decades and are "hardly distinguishable today". 




