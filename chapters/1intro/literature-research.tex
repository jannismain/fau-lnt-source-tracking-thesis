\section{Literature Overview}

A preliminary exploration of the speaker localisation and tracking literature shall provide an understanding of the progress that has been made so far, the current state-of-the-art and the most promising ways forward in this diverse research field.
\paragraph{Main Research Fields}
Historically, the speech enhancement research field consisted of two branches, namely \textit{microphone array processing} and \textit{blind source separation} \cite[p.~693]{Gannot2017}. The former is focused on the localization and enhancement of speech signals collected by microphone arrays in noisy and reverberant environments, whereas the latter is concerned with the "cocktail party scenario" that involves several sound sources mixed together. From today's perspective one can easily see, how these two areas in part rely on the same methods and techniques to solve very similar problems. In practice, identifying individuals in the cocktail party scenario is often addressed with a range of different microphone array configurations (see \cite{Wang2013} for a blind source separation approach using microphone arrays and beamforming), and classical microphone array processing applications, like the one presented in this thesis, have to deal with an unknown number of target sources and interference by other speakers or noise. So naturally, these branches have merged over the last decades and, according to \cite[p.~693]{Gannot2017}, they are hardly distinguishable anymore. Focusing on the task of source localisation within the microphone array processing research field, many of the early solutions for source localisation, most notably sonar \cite{Hackmann1986} and its airborn counterpart radar, are based on \gls{tde}. This has spawned a wide array of studies that examine the opportunities and challenges of time-delay based signal processing. Those concerned with the issue of acoustic source localisation are reviewed next.

\paragraph{Time-Delay Estimation}
% TDE
\alt{In its basic form, the problem of source localisation using microphone arrays can be reduced to \gls{tde} of the received signals} Reducing the problem of location estimation through microphone arrays to only two microphones, one way how source localisation can be achieved is \gls{tde} between the received signals at the microphone pair. Once the time-delay has been determined, the \gls{doa} can be inferred and, combining multiple \glspl{doa}, the source location can be estimated. The intuitive approach for \gls{tde} is to calculate the cross-correlation function of the received signals to determine the time delay one signal has relative to the other one. \citeauthor{Knapp1976} used this to develop the \gls{gcc} method in \cite{Knapp1976}, where they applied different filters to the received signal prior to computing the cross-correlation. Among these prefilters, the phase transform filter is the one that has been used in most applications, which is why the procedure is commonly referred to as \gls{gcc-phat}. While this method works reasonably well for a single source in ideal conditions, its performance quickly deteriorates in adverse conditions, i.e., in the presence of multiple sources, noise or reverberation \cite{Bedard1994,Champagne1996}. Therefore, many extensions to \gls{gcc-phat}, like incorporating parts of the human audition characteristics \cite{Wilson2006}, have been proposed since to provide \glspl{tde} that is more robust within real-life environments.
 
% Categories of Improvements for noisy, reverberant environments
According to \cite{Chen2006}, the approaches to alleviate the shortcomings of the above mentioned schemes for source localisation can be categorised into three groups. The first group consists of improvements that incorporate a-priori knowledge about the distortions into the \gls{gcc} framework to improve its resilience against them. The second type of improvements add more microphones and use the redundant information to increase the localisation performance \cite{Brandstein1996,Silverman1997}. The third approach is to model the reverberant parts of the signal and apply advanced system identification techniques to enhance \gls{tde} performance \cite{Huang1999}.

% Localisation Schemes apart from TDE
Besides localisation methods that rely on \gls{tde}, there are also other methods that enable location estimation, like the \gls{srp-phat}, which use steered beamformers to acquire a \gls{doa} estimate. According to \cite{DiBiase2001}, there is a third category of localisation methods that use high-resolution spectral estimation. An example for these type of solutions is the \gls{music} algorithm \cite{Schmidt1986}.

\paragraph{Existing Solutions for Speaker Tracking}
% Bayesian and non-Bayesian Solutions
Current solutions to the problem of acoustic source tracking can be grouped into Bayesian and non-Bayesian approaches. Bayesian approaches are characterised by the usage of Bayesian inference to derive a posterior probability incorporating prior knowledge and a statistical model that describes the observations. Examples of solutions to the source tracking problem, that are part of the Bayesian family of algorithms, are Kalman filters \cite{Kalman1960} and particle filters, both of which have been successfully applied to the problem of acoustic source tracking \cite{Gannot2012,Lehmann2007}. This thesis however focuses on \gls{ml} estimation, which is part of the non-Bayesian approaches, and how it is used to acquire location estimates for multiple sources in adverse conditions as described in \cite{Schwartz2014}. The main challenges for \gls{ml} estimation to obtain source locations are the computational complexity involved as well as the non-existence of a closed form solution due to incomplete information \cite[p.~1692]{Dorfan2015}. This is why the \gls{em} algorithm, an iterative procedure to compute \gls{ml} estimates, is often used in this context.