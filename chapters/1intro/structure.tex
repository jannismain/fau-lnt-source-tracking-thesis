\section{Structure of the Thesis}
\label{chap:1structure}

% Categorization of own approach
This thesis adapts the \gls{em} algorithm for the purpose of acoustic source localisation and uses a recursive extension of the \gls{em} algorithm for acoustic source tracking. The \gls{em} algorithm is used to estimate the parameters of a \gls{gmm}. The algorithm, that will be presented in this thesis, was proposed by \citeauthor{Schwartz2014} \cite{Schwartz2014}. The derivation of the algorithm follows the original derivation outlined in \cite{Schwartz2014}. Therefore, apart from a minor adjustment to the dimensionality of the estimated parameters, the main focus of this thesis lies on the algorithm implementation as well as the performance evaluation for varying parameter sets.

The structure of the subsequent chapters is as follows: First, the theoretical concepts that build the foundation of the algorithms source localisation and tracking are presented in \autoref{chap:2theory}. Then, the algorithms in question are derived in \autoref{chap:algorithms}. The experimental part of this thesis is presented in \autoref{chap:experiments}, where the methodology is explained, key implementation details are highlighted and then results are presented and discussed thoroughly. This thesis concludes with \autoref{chap:concl} and a summary of the main findings as well as a critical review and possible directions for future research.


%Beginning with \autoref{chap:2theory}, the theoretical concepts that the algorithms for source localisation and tracking reviewed in this thesis are based upon will be introduced. Subsequently, in Chapter \ref{chap:algorithms} these algorithms will then be formally defined. In Chapter \ref{chap:implementation} the implementation of these algorithms in \matlab is shown and test scenarios to evaluate these implementations are defined. The results of these tests are presented and discussed in Chapter \ref{chap:results}. Finally, the findings of this thesis are summarized in Chapter \ref{chap:concl} and a conclusion towards the applicability of these findings to the general performance of the reviewed algorithms is drawn.
