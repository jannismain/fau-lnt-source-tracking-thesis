\chapter{Implementation Details}
\label{chap:implementation}
Before any results can be produced, the described algorithms and setup have to be implemented and tested within a simulation framework, that allows for easy parameter customisation and evaluation. Therefore, what follows is the description of the different parts of the implementation that allows for the evaluation of the algorithms described in \ref{chap:algorithms}. First, the setup is defined and the simulation framework is laid out. Then, the implementation of both source localisation and source tracking are discussed. Lastly, evaluation scenarios are established to analyse the possibilities and limitations of the implementation. This will assist in drawing conclusions about the strengths and shortcomings of the algorithm in this particular setup and may allow for more general findings to be deducted.

\begin{figure}[H]
	\centering
	\begin{tikzpicture}
  \tikzset{
    block/.style={
      draw,
      minimum height=1cm,
      inner sep=1em,
    },
    arrow/.style={thick, ->, >=stealth}
    }
  \begin{scope}[start chain=transition going right,node distance=1cm]
    \node (sim)[block,on chain, minimum width=2cm] {simulate};
    \node (stft)[block,on chain, minimum width=2cm] {stft};
    \node (em)[block,on chain,minimum width=2cm] {em-algorithm};
    \node (loc)[block,on chain,minimum width=2cm] {estimate location};
    
    \draw [arrow] (sim) -- node[anchor=west] {} (stft);
    \draw [arrow] (stft) -- node[anchor=west] {} (em);
    \draw [arrow] (em) -- node[anchor=west] {} (loc);
  \end{scope}
\end{tikzpicture}
	\caption{Block diagram of execution steps}
	\label{diag:execBlocks}
\end{figure}


\section{Setup}
\label{sec:setup}
\import{chapters/4.implementation/}{setup}

\newacronym{rir}{RIR}{room impulse response}
\section{Simulation Framework}


\paragraph{Simulation of RIRs}
For the simulation of the \gls{rir} for the static location estimation case, the \emph{\gls{rir}-Generator} by \citeauthor{Habets2014} is used \cite{Habets2014}.

For the source trajectory simulation, execution time of the \gls{rir}-Generator increases drastically. Therefore, the \emph{fastISM} package by \citeauthor{Lehmann2010} is used \cite{Lehmann2010}.
\begin{figure}[H]
	\centering
	\tikzstyle{every node}=[draw=black,thick,anchor=west]
\tikzstyle{selected}=[draw=black,fill=yellow!20]
\tikzstyle{optional}=[dashed,fill=gray!50]
\begin{tikzpicture}[%
  grow via three points={one child at (0.5,-0.7) and
  two children at (0.5,-0.7) and (0.5,-1.4)},
  edge from parent path={(\tikzparentnode.south) |- (\tikzchildnode.west)}]
  \node [selected]{src} 
    child { node {config-update}}		
    child { node {simulate}}
    child { node {stft}}
    child { node {estimate-location}}
    child { node {estimation-error}}		
%    child { node {example}
%      child { node [selected] {generic}}
%      child { node [optional] {latex}}
%      child [missing] {}	
%      child { node {plain}}
%    }
;
\end{tikzpicture}
	\caption{Functions of Simulation Framework}
	\label{diag:simulationFramework}
\end{figure}

%\import{chapters/4.implementation/simulation-framework}

\section{Location Estimation}
\begin{figure}[H]
	\centering
	\tikzstyle{every node}=[draw=black,thick,anchor=west]
\tikzstyle{selected}=[draw=black,fill=yellow!20]
\tikzstyle{optional}=[dashed,fill=gray!50]
\begin{tikzpicture}[%
  grow via three points={one child at (0.5,-0.7) and
  two children at (0.5,-0.7) and (0.5,-1.4)},
  edge from parent path={(\tikzparentnode.south) |- (\tikzchildnode.west)}]
  \node [selected]{location-estimation} 
    child { node {em-algorithm}}		
    child { node {estimate-location}}
    child { node {estimation-error}}
;
\end{tikzpicture}
	\caption{Functions of the location estimation routine}
	\label{diag:locEstFunc}
\end{figure}



\section{Source Tracking}
\section{Evaluation Scenarios}
There are many parameters that each could have an effect on the performance of the localisation algorithm.

