\chapter{Conclusions}
\label{chap:concl}

\section{Critical review}
\section{Further research}