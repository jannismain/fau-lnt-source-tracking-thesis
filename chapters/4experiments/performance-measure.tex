\subsection{Localisation Performance Measure}
\label{sec:performanceMeasure}

Intuitively, the localisation error for a given source $s$ is given by the eucledian distance between the original source position and the location estimate

\begin{equation}
	\epsilon_s = \norm{\p_s-\hat\p_s}.
	\label{eq:error}
\end{equation}

\paragraph{Aggregate Measure} Whenever the aggregation of data into a single measure is pursued, the loss of information about some of the original information is inevitable. Nevertheless, to evaluate and compare the results for different parameter sets, the raw data has to be aggregated in a way that allows for straight-forward comparison of the results. For this kind of error aggregation, there are several different measures used in the literature, the two most common of them being the \gls{mae} and the \gls{rmse}
\begin{align}
	\text{\glsentryshort{mae}}  & =\frac{1}{S}\sum_{s=1}^{S}\abs{\epsilon_s},    \\
	\text{\glsentryshort{rmse}} & =\sqrt{\frac{1}{S}\sum_{s=1}^{S}\epsilon_s^2}.
	\label{eq:mae}
\end{align}

The underlying assumption of the \gls{rmse} is an unbiased error that follows a normal distribution. This might be the case for trials, where localisation is severely affected by noise and reverberation, but does not fit trials, where localisation is generally good, meaning most errors are zero, and only few but large outliers occur. The \gls{rmse} overemphasises these outliers, whereas the \gls{mae} weights every error equal. Therefore, the \gls{mae} is a more accurate performance measure for the static localisation trials.

\paragraph{Assigning Estimates} Note, that neither the localisation nor the tracking algorithm create an identity between original source location and location estimate, as the original source location is inherently unknown to the algorithm. Therefore, assigning this identity to compute a sensible localisation error introduces a bias, that depends on the assignment strategy. One way of assigning estimates to their respective true location would be to simply minimise the overall \gls{mae}. The downside with this approach is, that estimates close to one source might be assigned to another one, so that the overall error is minimised. In other words, one correctly and one incorrectly estimated source locations could result in an assignment, that states two incorrectly estimates sources, which is suboptimal if we want to evaluate, what the percentage of correct location estimates is across different trials. The solution is to assign estimates that are closer to one of the source positions first. This can be done by calculating the matrix of distances $\bm{D}_{ij}$ between the location estimates $\hat\p_s$ and the actual source positions $\p_s$

\begin{equation}
	\bm{D}_{ij}=
	\begin{bmatrix}
		d_{11} & d_{12} & \dots  & d_{1S} \\
		d_{21} & d_{22} & \dots  & 0        \\
		\vdots   & \vdots   & \ddots & \vdots   \\
		d_{S1} & 0        & \dots  & d_{SS}
	\end{bmatrix},
\end{equation}

where $i,j\ \in \{1,\dots,S\}$ and $d_{ij}$ denotes the distance between the $i$-th source position $\p_{s=i}$ and the $j$-th location estimate $\hat\p_{s=j}$. In this context, assigning estimates to source locations means selecting a combination of $S$ different $d_{ij}$, so that no column and no row is taken from twice. The Algorithm for this is described in \autoref{alg:assignment}. The result of this algorithm is a set of distances $\mathcal{D}$ (a possible result for $S=3$ would be $\mathcal{D}=\{d_{13}, d_{21}, d_{32}\}$, where the third location estimate is assigned to the first source location, the first estimate is assigned to the second source position and the second estimate is assigned to the third source position), which can then be used to calculate the \gls{mae} according to \eqref{eq:mae}
\begin{equation}
    \text{\glsentryshort{mae}}=\frac{1}{S}\sum_{d_{ij}\in\mathcal{D}}\abs{d_{ij}}
\end{equation}

When there are two minimum distances $d_{ij}$ of equal value, one is chosen that minimises the overall \gls{mae}. This can be done by carrying out the algorithm once for each possibility and choosing the assignments with the lower \gls{mae}.
\begin{algorithm}[!b]
\caption{Assigning Location Estimates to Source Positions}
\label{alg:assignment}
\begin{algorithmic}
    \While{$\bm{D}_{ij}$ has elements}
    \State find smallest $d_{ij}\ \in\ \bm{D}_{ij}$ 
    \State eliminate $i$-th row and $j$-th column of $\bm{D}_{ij}$
    \EndWhile
\end{algorithmic}
\end{algorithm}

\begin{figure}[H]
% three plots
%	\begin{subfigure}{0.32\textwidth}
%	\centering
%%		\includegraphics[width=\textwidth]{data/plots/reference/assignment-problematic-MANUAL2}  % PLACEHOLDER
%        \footnotesize
%        \setlength{\figurewidth}{0.95\textwidth}
%        \setlength{\figureheight}{5cm}
%        % This file was created by matlab2tikz.
%
\definecolor{mycolor1}{rgb}{1.00000,1.00000,0.00000}%
%
\begin{tikzpicture}

\begin{axis}[%
width=0.951\figurewidth,
height=\figureheight,
at={(0\figurewidth,0\figureheight)},
scale only axis,
xmin=-0,
xmax=6,
ymin=-0,
ymax=6,
axis background/.style={fill=white},
axis x line*=bottom,
axis y line*=left
]

\addplot[%
surf,
shader=interp, colormap={mymap}{[1pt] rgb(0pt)=(0.239216,0.14902,0.658824); rgb(1pt)=(0.239216,0.14902,0.658824)}, mesh/rows=6]
table[row sep=crcr, point meta=\thisrow{c}] {%
%
x	y	c\\
0	0	0\\
0	1.2	0\\
0	2.4	0\\
0	3.6	0\\
0	4.8	0\\
0	6	0\\
1.2	0	0\\
1.2	1.2	0\\
1.2	2.4	0\\
1.2	3.6	0\\
1.2	4.8	0\\
1.2	6	0\\
2.4	0	0\\
2.4	1.2	0\\
2.4	2.4	0\\
2.4	3.6	0\\
2.4	4.8	0\\
2.4	6	0\\
3.6	0	0\\
3.6	1.2	0\\
3.6	2.4	0\\
3.6	3.6	0\\
3.6	4.8	0\\
3.6	6	0\\
4.8	0	0\\
4.8	1.2	0\\
4.8	2.4	0\\
4.8	3.6	0\\
4.8	4.8	0\\
4.8	6	0\\
6	0	0\\
6	1.2	0\\
6	2.4	0\\
6	3.6	0\\
6	4.8	0\\
6	6	0\\
};
\addplot [color=green, line width=1.0pt, draw=none, mark size=4.0pt, mark=o, mark options={solid, green}, forget plot]
  table[row sep=crcr]{%
2.1	1\\
2.3	1\\
2.7	1\\
2.9	1\\
3.7	1\\
3.9	1\\
5	2.2\\
5	2.4\\
5	2.8\\
5	3\\
5	3.8\\
5	4\\
2.2	5\\
2.4	5\\
3	5\\
3.2	5\\
3.8	5\\
4	5\\
1	2.1\\
1	2.3\\
1	2.9\\
1	3.1\\
1	3.7\\
1	3.9\\
};
\addplot [color=red, line width=1.0pt, draw=none, mark size=6.0pt, mark=x, mark options={solid, red}, forget plot]
  table[row sep=crcr]{%
3.3	3\\
2.7	3\\
};
\addplot [color=black, draw=none, mark size=0.2pt, mark=*, mark options={solid, black}, forget plot]
  table[row sep=crcr]{%
0.1	0.1\\
0.1	0.2\\
0.1	0.3\\
0.1	0.4\\
0.1	0.5\\
0.1	0.6\\
0.1	0.7\\
0.1	0.8\\
0.1	0.9\\
0.1	1\\
0.1	1.1\\
0.1	1.2\\
0.1	1.3\\
0.1	1.4\\
0.1	1.5\\
0.1	1.6\\
0.1	1.7\\
0.1	1.8\\
0.1	1.9\\
0.1	2\\
0.1	2.1\\
0.1	2.2\\
0.1	2.3\\
0.1	2.4\\
0.1	2.5\\
0.1	2.6\\
0.1	2.7\\
0.1	2.8\\
0.1	2.9\\
0.1	3\\
0.1	3.1\\
0.1	3.2\\
0.1	3.3\\
0.1	3.4\\
0.1	3.5\\
0.1	3.6\\
0.1	3.7\\
0.1	3.8\\
0.1	3.9\\
0.1	4\\
0.1	4.1\\
0.1	4.2\\
0.1	4.3\\
0.1	4.4\\
0.1	4.5\\
0.1	4.6\\
0.1	4.7\\
0.1	4.8\\
0.1	4.9\\
0.1	5\\
0.1	5.1\\
0.1	5.2\\
0.1	5.3\\
0.1	5.4\\
0.1	5.5\\
0.1	5.6\\
0.1	5.7\\
0.1	5.8\\
0.1	5.9\\
};
\addplot [color=black, draw=none, mark size=0.2pt, mark=*, mark options={solid, black}, forget plot]
  table[row sep=crcr]{%
0.2	0.1\\
0.2	0.2\\
0.2	0.3\\
0.2	0.4\\
0.2	0.5\\
0.2	0.6\\
0.2	0.7\\
0.2	0.8\\
0.2	0.9\\
0.2	1\\
0.2	1.1\\
0.2	1.2\\
0.2	1.3\\
0.2	1.4\\
0.2	1.5\\
0.2	1.6\\
0.2	1.7\\
0.2	1.8\\
0.2	1.9\\
0.2	2\\
0.2	2.1\\
0.2	2.2\\
0.2	2.3\\
0.2	2.4\\
0.2	2.5\\
0.2	2.6\\
0.2	2.7\\
0.2	2.8\\
0.2	2.9\\
0.2	3\\
0.2	3.1\\
0.2	3.2\\
0.2	3.3\\
0.2	3.4\\
0.2	3.5\\
0.2	3.6\\
0.2	3.7\\
0.2	3.8\\
0.2	3.9\\
0.2	4\\
0.2	4.1\\
0.2	4.2\\
0.2	4.3\\
0.2	4.4\\
0.2	4.5\\
0.2	4.6\\
0.2	4.7\\
0.2	4.8\\
0.2	4.9\\
0.2	5\\
0.2	5.1\\
0.2	5.2\\
0.2	5.3\\
0.2	5.4\\
0.2	5.5\\
0.2	5.6\\
0.2	5.7\\
0.2	5.8\\
0.2	5.9\\
};
\addplot [color=black, draw=none, mark size=0.2pt, mark=*, mark options={solid, black}, forget plot]
  table[row sep=crcr]{%
0.3	0.1\\
0.3	0.2\\
0.3	0.3\\
0.3	0.4\\
0.3	0.5\\
0.3	0.6\\
0.3	0.7\\
0.3	0.8\\
0.3	0.9\\
0.3	1\\
0.3	1.1\\
0.3	1.2\\
0.3	1.3\\
0.3	1.4\\
0.3	1.5\\
0.3	1.6\\
0.3	1.7\\
0.3	1.8\\
0.3	1.9\\
0.3	2\\
0.3	2.1\\
0.3	2.2\\
0.3	2.3\\
0.3	2.4\\
0.3	2.5\\
0.3	2.6\\
0.3	2.7\\
0.3	2.8\\
0.3	2.9\\
0.3	3\\
0.3	3.1\\
0.3	3.2\\
0.3	3.3\\
0.3	3.4\\
0.3	3.5\\
0.3	3.6\\
0.3	3.7\\
0.3	3.8\\
0.3	3.9\\
0.3	4\\
0.3	4.1\\
0.3	4.2\\
0.3	4.3\\
0.3	4.4\\
0.3	4.5\\
0.3	4.6\\
0.3	4.7\\
0.3	4.8\\
0.3	4.9\\
0.3	5\\
0.3	5.1\\
0.3	5.2\\
0.3	5.3\\
0.3	5.4\\
0.3	5.5\\
0.3	5.6\\
0.3	5.7\\
0.3	5.8\\
0.3	5.9\\
};
\addplot [color=black, draw=none, mark size=0.2pt, mark=*, mark options={solid, black}, forget plot]
  table[row sep=crcr]{%
0.4	0.1\\
0.4	0.2\\
0.4	0.3\\
0.4	0.4\\
0.4	0.5\\
0.4	0.6\\
0.4	0.7\\
0.4	0.8\\
0.4	0.9\\
0.4	1\\
0.4	1.1\\
0.4	1.2\\
0.4	1.3\\
0.4	1.4\\
0.4	1.5\\
0.4	1.6\\
0.4	1.7\\
0.4	1.8\\
0.4	1.9\\
0.4	2\\
0.4	2.1\\
0.4	2.2\\
0.4	2.3\\
0.4	2.4\\
0.4	2.5\\
0.4	2.6\\
0.4	2.7\\
0.4	2.8\\
0.4	2.9\\
0.4	3\\
0.4	3.1\\
0.4	3.2\\
0.4	3.3\\
0.4	3.4\\
0.4	3.5\\
0.4	3.6\\
0.4	3.7\\
0.4	3.8\\
0.4	3.9\\
0.4	4\\
0.4	4.1\\
0.4	4.2\\
0.4	4.3\\
0.4	4.4\\
0.4	4.5\\
0.4	4.6\\
0.4	4.7\\
0.4	4.8\\
0.4	4.9\\
0.4	5\\
0.4	5.1\\
0.4	5.2\\
0.4	5.3\\
0.4	5.4\\
0.4	5.5\\
0.4	5.6\\
0.4	5.7\\
0.4	5.8\\
0.4	5.9\\
};
\addplot [color=black, draw=none, mark size=0.2pt, mark=*, mark options={solid, black}, forget plot]
  table[row sep=crcr]{%
0.5	0.1\\
0.5	0.2\\
0.5	0.3\\
0.5	0.4\\
0.5	0.5\\
0.5	0.6\\
0.5	0.7\\
0.5	0.8\\
0.5	0.9\\
0.5	1\\
0.5	1.1\\
0.5	1.2\\
0.5	1.3\\
0.5	1.4\\
0.5	1.5\\
0.5	1.6\\
0.5	1.7\\
0.5	1.8\\
0.5	1.9\\
0.5	2\\
0.5	2.1\\
0.5	2.2\\
0.5	2.3\\
0.5	2.4\\
0.5	2.5\\
0.5	2.6\\
0.5	2.7\\
0.5	2.8\\
0.5	2.9\\
0.5	3\\
0.5	3.1\\
0.5	3.2\\
0.5	3.3\\
0.5	3.4\\
0.5	3.5\\
0.5	3.6\\
0.5	3.7\\
0.5	3.8\\
0.5	3.9\\
0.5	4\\
0.5	4.1\\
0.5	4.2\\
0.5	4.3\\
0.5	4.4\\
0.5	4.5\\
0.5	4.6\\
0.5	4.7\\
0.5	4.8\\
0.5	4.9\\
0.5	5\\
0.5	5.1\\
0.5	5.2\\
0.5	5.3\\
0.5	5.4\\
0.5	5.5\\
0.5	5.6\\
0.5	5.7\\
0.5	5.8\\
0.5	5.9\\
};
\addplot [color=black, draw=none, mark size=0.2pt, mark=*, mark options={solid, black}, forget plot]
  table[row sep=crcr]{%
0.6	0.1\\
0.6	0.2\\
0.6	0.3\\
0.6	0.4\\
0.6	0.5\\
0.6	0.6\\
0.6	0.7\\
0.6	0.8\\
0.6	0.9\\
0.6	1\\
0.6	1.1\\
0.6	1.2\\
0.6	1.3\\
0.6	1.4\\
0.6	1.5\\
0.6	1.6\\
0.6	1.7\\
0.6	1.8\\
0.6	1.9\\
0.6	2\\
0.6	2.1\\
0.6	2.2\\
0.6	2.3\\
0.6	2.4\\
0.6	2.5\\
0.6	2.6\\
0.6	2.7\\
0.6	2.8\\
0.6	2.9\\
0.6	3\\
0.6	3.1\\
0.6	3.2\\
0.6	3.3\\
0.6	3.4\\
0.6	3.5\\
0.6	3.6\\
0.6	3.7\\
0.6	3.8\\
0.6	3.9\\
0.6	4\\
0.6	4.1\\
0.6	4.2\\
0.6	4.3\\
0.6	4.4\\
0.6	4.5\\
0.6	4.6\\
0.6	4.7\\
0.6	4.8\\
0.6	4.9\\
0.6	5\\
0.6	5.1\\
0.6	5.2\\
0.6	5.3\\
0.6	5.4\\
0.6	5.5\\
0.6	5.6\\
0.6	5.7\\
0.6	5.8\\
0.6	5.9\\
};
\addplot [color=black, draw=none, mark size=0.2pt, mark=*, mark options={solid, black}, forget plot]
  table[row sep=crcr]{%
0.7	0.1\\
0.7	0.2\\
0.7	0.3\\
0.7	0.4\\
0.7	0.5\\
0.7	0.6\\
0.7	0.7\\
0.7	0.8\\
0.7	0.9\\
0.7	1\\
0.7	1.1\\
0.7	1.2\\
0.7	1.3\\
0.7	1.4\\
0.7	1.5\\
0.7	1.6\\
0.7	1.7\\
0.7	1.8\\
0.7	1.9\\
0.7	2\\
0.7	2.1\\
0.7	2.2\\
0.7	2.3\\
0.7	2.4\\
0.7	2.5\\
0.7	2.6\\
0.7	2.7\\
0.7	2.8\\
0.7	2.9\\
0.7	3\\
0.7	3.1\\
0.7	3.2\\
0.7	3.3\\
0.7	3.4\\
0.7	3.5\\
0.7	3.6\\
0.7	3.7\\
0.7	3.8\\
0.7	3.9\\
0.7	4\\
0.7	4.1\\
0.7	4.2\\
0.7	4.3\\
0.7	4.4\\
0.7	4.5\\
0.7	4.6\\
0.7	4.7\\
0.7	4.8\\
0.7	4.9\\
0.7	5\\
0.7	5.1\\
0.7	5.2\\
0.7	5.3\\
0.7	5.4\\
0.7	5.5\\
0.7	5.6\\
0.7	5.7\\
0.7	5.8\\
0.7	5.9\\
};
\addplot [color=black, draw=none, mark size=0.2pt, mark=*, mark options={solid, black}, forget plot]
  table[row sep=crcr]{%
0.8	0.1\\
0.8	0.2\\
0.8	0.3\\
0.8	0.4\\
0.8	0.5\\
0.8	0.6\\
0.8	0.7\\
0.8	0.8\\
0.8	0.9\\
0.8	1\\
0.8	1.1\\
0.8	1.2\\
0.8	1.3\\
0.8	1.4\\
0.8	1.5\\
0.8	1.6\\
0.8	1.7\\
0.8	1.8\\
0.8	1.9\\
0.8	2\\
0.8	2.1\\
0.8	2.2\\
0.8	2.3\\
0.8	2.4\\
0.8	2.5\\
0.8	2.6\\
0.8	2.7\\
0.8	2.8\\
0.8	2.9\\
0.8	3\\
0.8	3.1\\
0.8	3.2\\
0.8	3.3\\
0.8	3.4\\
0.8	3.5\\
0.8	3.6\\
0.8	3.7\\
0.8	3.8\\
0.8	3.9\\
0.8	4\\
0.8	4.1\\
0.8	4.2\\
0.8	4.3\\
0.8	4.4\\
0.8	4.5\\
0.8	4.6\\
0.8	4.7\\
0.8	4.8\\
0.8	4.9\\
0.8	5\\
0.8	5.1\\
0.8	5.2\\
0.8	5.3\\
0.8	5.4\\
0.8	5.5\\
0.8	5.6\\
0.8	5.7\\
0.8	5.8\\
0.8	5.9\\
};
\addplot [color=black, draw=none, mark size=0.2pt, mark=*, mark options={solid, black}, forget plot]
  table[row sep=crcr]{%
0.9	0.1\\
0.9	0.2\\
0.9	0.3\\
0.9	0.4\\
0.9	0.5\\
0.9	0.6\\
0.9	0.7\\
0.9	0.8\\
0.9	0.9\\
0.9	1\\
0.9	1.1\\
0.9	1.2\\
0.9	1.3\\
0.9	1.4\\
0.9	1.5\\
0.9	1.6\\
0.9	1.7\\
0.9	1.8\\
0.9	1.9\\
0.9	2\\
0.9	2.1\\
0.9	2.2\\
0.9	2.3\\
0.9	2.4\\
0.9	2.5\\
0.9	2.6\\
0.9	2.7\\
0.9	2.8\\
0.9	2.9\\
0.9	3\\
0.9	3.1\\
0.9	3.2\\
0.9	3.3\\
0.9	3.4\\
0.9	3.5\\
0.9	3.6\\
0.9	3.7\\
0.9	3.8\\
0.9	3.9\\
0.9	4\\
0.9	4.1\\
0.9	4.2\\
0.9	4.3\\
0.9	4.4\\
0.9	4.5\\
0.9	4.6\\
0.9	4.7\\
0.9	4.8\\
0.9	4.9\\
0.9	5\\
0.9	5.1\\
0.9	5.2\\
0.9	5.3\\
0.9	5.4\\
0.9	5.5\\
0.9	5.6\\
0.9	5.7\\
0.9	5.8\\
0.9	5.9\\
};
\addplot [color=black, draw=none, mark size=0.2pt, mark=*, mark options={solid, black}, forget plot]
  table[row sep=crcr]{%
1	0.1\\
1	0.2\\
1	0.3\\
1	0.4\\
1	0.5\\
1	0.6\\
1	0.7\\
1	0.8\\
1	0.9\\
1	1\\
1	1.1\\
1	1.2\\
1	1.3\\
1	1.4\\
1	1.5\\
1	1.6\\
1	1.7\\
1	1.8\\
1	1.9\\
1	2\\
1	2.1\\
1	2.2\\
1	2.3\\
1	2.4\\
1	2.5\\
1	2.6\\
1	2.7\\
1	2.8\\
1	2.9\\
1	3\\
1	3.1\\
1	3.2\\
1	3.3\\
1	3.4\\
1	3.5\\
1	3.6\\
1	3.7\\
1	3.8\\
1	3.9\\
1	4\\
1	4.1\\
1	4.2\\
1	4.3\\
1	4.4\\
1	4.5\\
1	4.6\\
1	4.7\\
1	4.8\\
1	4.9\\
1	5\\
1	5.1\\
1	5.2\\
1	5.3\\
1	5.4\\
1	5.5\\
1	5.6\\
1	5.7\\
1	5.8\\
1	5.9\\
};
\addplot [color=black, draw=none, mark size=0.2pt, mark=*, mark options={solid, black}, forget plot]
  table[row sep=crcr]{%
1.1	0.1\\
1.1	0.2\\
1.1	0.3\\
1.1	0.4\\
1.1	0.5\\
1.1	0.6\\
1.1	0.7\\
1.1	0.8\\
1.1	0.9\\
1.1	1\\
1.1	1.1\\
1.1	1.2\\
1.1	1.3\\
1.1	1.4\\
1.1	1.5\\
1.1	1.6\\
1.1	1.7\\
1.1	1.8\\
1.1	1.9\\
1.1	2\\
1.1	2.1\\
1.1	2.2\\
1.1	2.3\\
1.1	2.4\\
1.1	2.5\\
1.1	2.6\\
1.1	2.7\\
1.1	2.8\\
1.1	2.9\\
1.1	3\\
1.1	3.1\\
1.1	3.2\\
1.1	3.3\\
1.1	3.4\\
1.1	3.5\\
1.1	3.6\\
1.1	3.7\\
1.1	3.8\\
1.1	3.9\\
1.1	4\\
1.1	4.1\\
1.1	4.2\\
1.1	4.3\\
1.1	4.4\\
1.1	4.5\\
1.1	4.6\\
1.1	4.7\\
1.1	4.8\\
1.1	4.9\\
1.1	5\\
1.1	5.1\\
1.1	5.2\\
1.1	5.3\\
1.1	5.4\\
1.1	5.5\\
1.1	5.6\\
1.1	5.7\\
1.1	5.8\\
1.1	5.9\\
};
\addplot [color=black, draw=none, mark size=0.2pt, mark=*, mark options={solid, black}, forget plot]
  table[row sep=crcr]{%
1.2	0.1\\
1.2	0.2\\
1.2	0.3\\
1.2	0.4\\
1.2	0.5\\
1.2	0.6\\
1.2	0.7\\
1.2	0.8\\
1.2	0.9\\
1.2	1\\
1.2	1.1\\
1.2	1.2\\
1.2	1.3\\
1.2	1.4\\
1.2	1.5\\
1.2	1.6\\
1.2	1.7\\
1.2	1.8\\
1.2	1.9\\
1.2	2\\
1.2	2.1\\
1.2	2.2\\
1.2	2.3\\
1.2	2.4\\
1.2	2.5\\
1.2	2.6\\
1.2	2.7\\
1.2	2.8\\
1.2	2.9\\
1.2	3\\
1.2	3.1\\
1.2	3.2\\
1.2	3.3\\
1.2	3.4\\
1.2	3.5\\
1.2	3.6\\
1.2	3.7\\
1.2	3.8\\
1.2	3.9\\
1.2	4\\
1.2	4.1\\
1.2	4.2\\
1.2	4.3\\
1.2	4.4\\
1.2	4.5\\
1.2	4.6\\
1.2	4.7\\
1.2	4.8\\
1.2	4.9\\
1.2	5\\
1.2	5.1\\
1.2	5.2\\
1.2	5.3\\
1.2	5.4\\
1.2	5.5\\
1.2	5.6\\
1.2	5.7\\
1.2	5.8\\
1.2	5.9\\
};
\addplot [color=black, draw=none, mark size=0.2pt, mark=*, mark options={solid, black}, forget plot]
  table[row sep=crcr]{%
1.3	0.1\\
1.3	0.2\\
1.3	0.3\\
1.3	0.4\\
1.3	0.5\\
1.3	0.6\\
1.3	0.7\\
1.3	0.8\\
1.3	0.9\\
1.3	1\\
1.3	1.1\\
1.3	1.2\\
1.3	1.3\\
1.3	1.4\\
1.3	1.5\\
1.3	1.6\\
1.3	1.7\\
1.3	1.8\\
1.3	1.9\\
1.3	2\\
1.3	2.1\\
1.3	2.2\\
1.3	2.3\\
1.3	2.4\\
1.3	2.5\\
1.3	2.6\\
1.3	2.7\\
1.3	2.8\\
1.3	2.9\\
1.3	3\\
1.3	3.1\\
1.3	3.2\\
1.3	3.3\\
1.3	3.4\\
1.3	3.5\\
1.3	3.6\\
1.3	3.7\\
1.3	3.8\\
1.3	3.9\\
1.3	4\\
1.3	4.1\\
1.3	4.2\\
1.3	4.3\\
1.3	4.4\\
1.3	4.5\\
1.3	4.6\\
1.3	4.7\\
1.3	4.8\\
1.3	4.9\\
1.3	5\\
1.3	5.1\\
1.3	5.2\\
1.3	5.3\\
1.3	5.4\\
1.3	5.5\\
1.3	5.6\\
1.3	5.7\\
1.3	5.8\\
1.3	5.9\\
};
\addplot [color=black, draw=none, mark size=0.2pt, mark=*, mark options={solid, black}, forget plot]
  table[row sep=crcr]{%
1.4	0.1\\
1.4	0.2\\
1.4	0.3\\
1.4	0.4\\
1.4	0.5\\
1.4	0.6\\
1.4	0.7\\
1.4	0.8\\
1.4	0.9\\
1.4	1\\
1.4	1.1\\
1.4	1.2\\
1.4	1.3\\
1.4	1.4\\
1.4	1.5\\
1.4	1.6\\
1.4	1.7\\
1.4	1.8\\
1.4	1.9\\
1.4	2\\
1.4	2.1\\
1.4	2.2\\
1.4	2.3\\
1.4	2.4\\
1.4	2.5\\
1.4	2.6\\
1.4	2.7\\
1.4	2.8\\
1.4	2.9\\
1.4	3\\
1.4	3.1\\
1.4	3.2\\
1.4	3.3\\
1.4	3.4\\
1.4	3.5\\
1.4	3.6\\
1.4	3.7\\
1.4	3.8\\
1.4	3.9\\
1.4	4\\
1.4	4.1\\
1.4	4.2\\
1.4	4.3\\
1.4	4.4\\
1.4	4.5\\
1.4	4.6\\
1.4	4.7\\
1.4	4.8\\
1.4	4.9\\
1.4	5\\
1.4	5.1\\
1.4	5.2\\
1.4	5.3\\
1.4	5.4\\
1.4	5.5\\
1.4	5.6\\
1.4	5.7\\
1.4	5.8\\
1.4	5.9\\
};
\addplot [color=black, draw=none, mark size=0.2pt, mark=*, mark options={solid, black}, forget plot]
  table[row sep=crcr]{%
1.5	0.1\\
1.5	0.2\\
1.5	0.3\\
1.5	0.4\\
1.5	0.5\\
1.5	0.6\\
1.5	0.7\\
1.5	0.8\\
1.5	0.9\\
1.5	1\\
1.5	1.1\\
1.5	1.2\\
1.5	1.3\\
1.5	1.4\\
1.5	1.5\\
1.5	1.6\\
1.5	1.7\\
1.5	1.8\\
1.5	1.9\\
1.5	2\\
1.5	2.1\\
1.5	2.2\\
1.5	2.3\\
1.5	2.4\\
1.5	2.5\\
1.5	2.6\\
1.5	2.7\\
1.5	2.8\\
1.5	2.9\\
1.5	3\\
1.5	3.1\\
1.5	3.2\\
1.5	3.3\\
1.5	3.4\\
1.5	3.5\\
1.5	3.6\\
1.5	3.7\\
1.5	3.8\\
1.5	3.9\\
1.5	4\\
1.5	4.1\\
1.5	4.2\\
1.5	4.3\\
1.5	4.4\\
1.5	4.5\\
1.5	4.6\\
1.5	4.7\\
1.5	4.8\\
1.5	4.9\\
1.5	5\\
1.5	5.1\\
1.5	5.2\\
1.5	5.3\\
1.5	5.4\\
1.5	5.5\\
1.5	5.6\\
1.5	5.7\\
1.5	5.8\\
1.5	5.9\\
};
\addplot [color=black, draw=none, mark size=0.2pt, mark=*, mark options={solid, black}, forget plot]
  table[row sep=crcr]{%
1.6	0.1\\
1.6	0.2\\
1.6	0.3\\
1.6	0.4\\
1.6	0.5\\
1.6	0.6\\
1.6	0.7\\
1.6	0.8\\
1.6	0.9\\
1.6	1\\
1.6	1.1\\
1.6	1.2\\
1.6	1.3\\
1.6	1.4\\
1.6	1.5\\
1.6	1.6\\
1.6	1.7\\
1.6	1.8\\
1.6	1.9\\
1.6	2\\
1.6	2.1\\
1.6	2.2\\
1.6	2.3\\
1.6	2.4\\
1.6	2.5\\
1.6	2.6\\
1.6	2.7\\
1.6	2.8\\
1.6	2.9\\
1.6	3\\
1.6	3.1\\
1.6	3.2\\
1.6	3.3\\
1.6	3.4\\
1.6	3.5\\
1.6	3.6\\
1.6	3.7\\
1.6	3.8\\
1.6	3.9\\
1.6	4\\
1.6	4.1\\
1.6	4.2\\
1.6	4.3\\
1.6	4.4\\
1.6	4.5\\
1.6	4.6\\
1.6	4.7\\
1.6	4.8\\
1.6	4.9\\
1.6	5\\
1.6	5.1\\
1.6	5.2\\
1.6	5.3\\
1.6	5.4\\
1.6	5.5\\
1.6	5.6\\
1.6	5.7\\
1.6	5.8\\
1.6	5.9\\
};
\addplot [color=black, draw=none, mark size=0.2pt, mark=*, mark options={solid, black}, forget plot]
  table[row sep=crcr]{%
1.7	0.1\\
1.7	0.2\\
1.7	0.3\\
1.7	0.4\\
1.7	0.5\\
1.7	0.6\\
1.7	0.7\\
1.7	0.8\\
1.7	0.9\\
1.7	1\\
1.7	1.1\\
1.7	1.2\\
1.7	1.3\\
1.7	1.4\\
1.7	1.5\\
1.7	1.6\\
1.7	1.7\\
1.7	1.8\\
1.7	1.9\\
1.7	2\\
1.7	2.1\\
1.7	2.2\\
1.7	2.3\\
1.7	2.4\\
1.7	2.5\\
1.7	2.6\\
1.7	2.7\\
1.7	2.8\\
1.7	2.9\\
1.7	3\\
1.7	3.1\\
1.7	3.2\\
1.7	3.3\\
1.7	3.4\\
1.7	3.5\\
1.7	3.6\\
1.7	3.7\\
1.7	3.8\\
1.7	3.9\\
1.7	4\\
1.7	4.1\\
1.7	4.2\\
1.7	4.3\\
1.7	4.4\\
1.7	4.5\\
1.7	4.6\\
1.7	4.7\\
1.7	4.8\\
1.7	4.9\\
1.7	5\\
1.7	5.1\\
1.7	5.2\\
1.7	5.3\\
1.7	5.4\\
1.7	5.5\\
1.7	5.6\\
1.7	5.7\\
1.7	5.8\\
1.7	5.9\\
};
\addplot [color=black, draw=none, mark size=0.2pt, mark=*, mark options={solid, black}, forget plot]
  table[row sep=crcr]{%
1.8	0.1\\
1.8	0.2\\
1.8	0.3\\
1.8	0.4\\
1.8	0.5\\
1.8	0.6\\
1.8	0.7\\
1.8	0.8\\
1.8	0.9\\
1.8	1\\
1.8	1.1\\
1.8	1.2\\
1.8	1.3\\
1.8	1.4\\
1.8	1.5\\
1.8	1.6\\
1.8	1.7\\
1.8	1.8\\
1.8	1.9\\
1.8	2\\
1.8	2.1\\
1.8	2.2\\
1.8	2.3\\
1.8	2.4\\
1.8	2.5\\
1.8	2.6\\
1.8	2.7\\
1.8	2.8\\
1.8	2.9\\
1.8	3\\
1.8	3.1\\
1.8	3.2\\
1.8	3.3\\
1.8	3.4\\
1.8	3.5\\
1.8	3.6\\
1.8	3.7\\
1.8	3.8\\
1.8	3.9\\
1.8	4\\
1.8	4.1\\
1.8	4.2\\
1.8	4.3\\
1.8	4.4\\
1.8	4.5\\
1.8	4.6\\
1.8	4.7\\
1.8	4.8\\
1.8	4.9\\
1.8	5\\
1.8	5.1\\
1.8	5.2\\
1.8	5.3\\
1.8	5.4\\
1.8	5.5\\
1.8	5.6\\
1.8	5.7\\
1.8	5.8\\
1.8	5.9\\
};
\addplot [color=black, draw=none, mark size=0.2pt, mark=*, mark options={solid, black}, forget plot]
  table[row sep=crcr]{%
1.9	0.1\\
1.9	0.2\\
1.9	0.3\\
1.9	0.4\\
1.9	0.5\\
1.9	0.6\\
1.9	0.7\\
1.9	0.8\\
1.9	0.9\\
1.9	1\\
1.9	1.1\\
1.9	1.2\\
1.9	1.3\\
1.9	1.4\\
1.9	1.5\\
1.9	1.6\\
1.9	1.7\\
1.9	1.8\\
1.9	1.9\\
1.9	2\\
1.9	2.1\\
1.9	2.2\\
1.9	2.3\\
1.9	2.4\\
1.9	2.5\\
1.9	2.6\\
1.9	2.7\\
1.9	2.8\\
1.9	2.9\\
1.9	3\\
1.9	3.1\\
1.9	3.2\\
1.9	3.3\\
1.9	3.4\\
1.9	3.5\\
1.9	3.6\\
1.9	3.7\\
1.9	3.8\\
1.9	3.9\\
1.9	4\\
1.9	4.1\\
1.9	4.2\\
1.9	4.3\\
1.9	4.4\\
1.9	4.5\\
1.9	4.6\\
1.9	4.7\\
1.9	4.8\\
1.9	4.9\\
1.9	5\\
1.9	5.1\\
1.9	5.2\\
1.9	5.3\\
1.9	5.4\\
1.9	5.5\\
1.9	5.6\\
1.9	5.7\\
1.9	5.8\\
1.9	5.9\\
};
\addplot [color=black, draw=none, mark size=0.2pt, mark=*, mark options={solid, black}, forget plot]
  table[row sep=crcr]{%
2	0.1\\
2	0.2\\
2	0.3\\
2	0.4\\
2	0.5\\
2	0.6\\
2	0.7\\
2	0.8\\
2	0.9\\
2	1\\
2	1.1\\
2	1.2\\
2	1.3\\
2	1.4\\
2	1.5\\
2	1.6\\
2	1.7\\
2	1.8\\
2	1.9\\
2	2\\
2	2.1\\
2	2.2\\
2	2.3\\
2	2.4\\
2	2.5\\
2	2.6\\
2	2.7\\
2	2.8\\
2	2.9\\
2	3\\
2	3.1\\
2	3.2\\
2	3.3\\
2	3.4\\
2	3.5\\
2	3.6\\
2	3.7\\
2	3.8\\
2	3.9\\
2	4\\
2	4.1\\
2	4.2\\
2	4.3\\
2	4.4\\
2	4.5\\
2	4.6\\
2	4.7\\
2	4.8\\
2	4.9\\
2	5\\
2	5.1\\
2	5.2\\
2	5.3\\
2	5.4\\
2	5.5\\
2	5.6\\
2	5.7\\
2	5.8\\
2	5.9\\
};
\addplot [color=black, draw=none, mark size=0.2pt, mark=*, mark options={solid, black}, forget plot]
  table[row sep=crcr]{%
2.1	0.1\\
2.1	0.2\\
2.1	0.3\\
2.1	0.4\\
2.1	0.5\\
2.1	0.6\\
2.1	0.7\\
2.1	0.8\\
2.1	0.9\\
2.1	1\\
2.1	1.1\\
2.1	1.2\\
2.1	1.3\\
2.1	1.4\\
2.1	1.5\\
2.1	1.6\\
2.1	1.7\\
2.1	1.8\\
2.1	1.9\\
2.1	2\\
2.1	2.1\\
2.1	2.2\\
2.1	2.3\\
2.1	2.4\\
2.1	2.5\\
2.1	2.6\\
2.1	2.7\\
2.1	2.8\\
2.1	2.9\\
2.1	3\\
2.1	3.1\\
2.1	3.2\\
2.1	3.3\\
2.1	3.4\\
2.1	3.5\\
2.1	3.6\\
2.1	3.7\\
2.1	3.8\\
2.1	3.9\\
2.1	4\\
2.1	4.1\\
2.1	4.2\\
2.1	4.3\\
2.1	4.4\\
2.1	4.5\\
2.1	4.6\\
2.1	4.7\\
2.1	4.8\\
2.1	4.9\\
2.1	5\\
2.1	5.1\\
2.1	5.2\\
2.1	5.3\\
2.1	5.4\\
2.1	5.5\\
2.1	5.6\\
2.1	5.7\\
2.1	5.8\\
2.1	5.9\\
};
\addplot [color=black, draw=none, mark size=0.2pt, mark=*, mark options={solid, black}, forget plot]
  table[row sep=crcr]{%
2.2	0.1\\
2.2	0.2\\
2.2	0.3\\
2.2	0.4\\
2.2	0.5\\
2.2	0.6\\
2.2	0.7\\
2.2	0.8\\
2.2	0.9\\
2.2	1\\
2.2	1.1\\
2.2	1.2\\
2.2	1.3\\
2.2	1.4\\
2.2	1.5\\
2.2	1.6\\
2.2	1.7\\
2.2	1.8\\
2.2	1.9\\
2.2	2\\
2.2	2.1\\
2.2	2.2\\
2.2	2.3\\
2.2	2.4\\
2.2	2.5\\
2.2	2.6\\
2.2	2.7\\
2.2	2.8\\
2.2	2.9\\
2.2	3\\
2.2	3.1\\
2.2	3.2\\
2.2	3.3\\
2.2	3.4\\
2.2	3.5\\
2.2	3.6\\
2.2	3.7\\
2.2	3.8\\
2.2	3.9\\
2.2	4\\
2.2	4.1\\
2.2	4.2\\
2.2	4.3\\
2.2	4.4\\
2.2	4.5\\
2.2	4.6\\
2.2	4.7\\
2.2	4.8\\
2.2	4.9\\
2.2	5\\
2.2	5.1\\
2.2	5.2\\
2.2	5.3\\
2.2	5.4\\
2.2	5.5\\
2.2	5.6\\
2.2	5.7\\
2.2	5.8\\
2.2	5.9\\
};
\addplot [color=black, draw=none, mark size=0.2pt, mark=*, mark options={solid, black}, forget plot]
  table[row sep=crcr]{%
2.3	0.1\\
2.3	0.2\\
2.3	0.3\\
2.3	0.4\\
2.3	0.5\\
2.3	0.6\\
2.3	0.7\\
2.3	0.8\\
2.3	0.9\\
2.3	1\\
2.3	1.1\\
2.3	1.2\\
2.3	1.3\\
2.3	1.4\\
2.3	1.5\\
2.3	1.6\\
2.3	1.7\\
2.3	1.8\\
2.3	1.9\\
2.3	2\\
2.3	2.1\\
2.3	2.2\\
2.3	2.3\\
2.3	2.4\\
2.3	2.5\\
2.3	2.6\\
2.3	2.7\\
2.3	2.8\\
2.3	2.9\\
2.3	3\\
2.3	3.1\\
2.3	3.2\\
2.3	3.3\\
2.3	3.4\\
2.3	3.5\\
2.3	3.6\\
2.3	3.7\\
2.3	3.8\\
2.3	3.9\\
2.3	4\\
2.3	4.1\\
2.3	4.2\\
2.3	4.3\\
2.3	4.4\\
2.3	4.5\\
2.3	4.6\\
2.3	4.7\\
2.3	4.8\\
2.3	4.9\\
2.3	5\\
2.3	5.1\\
2.3	5.2\\
2.3	5.3\\
2.3	5.4\\
2.3	5.5\\
2.3	5.6\\
2.3	5.7\\
2.3	5.8\\
2.3	5.9\\
};
\addplot [color=black, draw=none, mark size=0.2pt, mark=*, mark options={solid, black}, forget plot]
  table[row sep=crcr]{%
2.4	0.1\\
2.4	0.2\\
2.4	0.3\\
2.4	0.4\\
2.4	0.5\\
2.4	0.6\\
2.4	0.7\\
2.4	0.8\\
2.4	0.9\\
2.4	1\\
2.4	1.1\\
2.4	1.2\\
2.4	1.3\\
2.4	1.4\\
2.4	1.5\\
2.4	1.6\\
2.4	1.7\\
2.4	1.8\\
2.4	1.9\\
2.4	2\\
2.4	2.1\\
2.4	2.2\\
2.4	2.3\\
2.4	2.4\\
2.4	2.5\\
2.4	2.6\\
2.4	2.7\\
2.4	2.8\\
2.4	2.9\\
2.4	3\\
2.4	3.1\\
2.4	3.2\\
2.4	3.3\\
2.4	3.4\\
2.4	3.5\\
2.4	3.6\\
2.4	3.7\\
2.4	3.8\\
2.4	3.9\\
2.4	4\\
2.4	4.1\\
2.4	4.2\\
2.4	4.3\\
2.4	4.4\\
2.4	4.5\\
2.4	4.6\\
2.4	4.7\\
2.4	4.8\\
2.4	4.9\\
2.4	5\\
2.4	5.1\\
2.4	5.2\\
2.4	5.3\\
2.4	5.4\\
2.4	5.5\\
2.4	5.6\\
2.4	5.7\\
2.4	5.8\\
2.4	5.9\\
};
\addplot [color=black, draw=none, mark size=0.2pt, mark=*, mark options={solid, black}, forget plot]
  table[row sep=crcr]{%
2.5	0.1\\
2.5	0.2\\
2.5	0.3\\
2.5	0.4\\
2.5	0.5\\
2.5	0.6\\
2.5	0.7\\
2.5	0.8\\
2.5	0.9\\
2.5	1\\
2.5	1.1\\
2.5	1.2\\
2.5	1.3\\
2.5	1.4\\
2.5	1.5\\
2.5	1.6\\
2.5	1.7\\
2.5	1.8\\
2.5	1.9\\
2.5	2\\
2.5	2.1\\
2.5	2.2\\
2.5	2.3\\
2.5	2.4\\
2.5	2.5\\
2.5	2.6\\
2.5	2.7\\
2.5	2.8\\
2.5	2.9\\
2.5	3\\
2.5	3.1\\
2.5	3.2\\
2.5	3.3\\
2.5	3.4\\
2.5	3.5\\
2.5	3.6\\
2.5	3.7\\
2.5	3.8\\
2.5	3.9\\
2.5	4\\
2.5	4.1\\
2.5	4.2\\
2.5	4.3\\
2.5	4.4\\
2.5	4.5\\
2.5	4.6\\
2.5	4.7\\
2.5	4.8\\
2.5	4.9\\
2.5	5\\
2.5	5.1\\
2.5	5.2\\
2.5	5.3\\
2.5	5.4\\
2.5	5.5\\
2.5	5.6\\
2.5	5.7\\
2.5	5.8\\
2.5	5.9\\
};
\addplot [color=black, draw=none, mark size=0.2pt, mark=*, mark options={solid, black}, forget plot]
  table[row sep=crcr]{%
2.6	0.1\\
2.6	0.2\\
2.6	0.3\\
2.6	0.4\\
2.6	0.5\\
2.6	0.6\\
2.6	0.7\\
2.6	0.8\\
2.6	0.9\\
2.6	1\\
2.6	1.1\\
2.6	1.2\\
2.6	1.3\\
2.6	1.4\\
2.6	1.5\\
2.6	1.6\\
2.6	1.7\\
2.6	1.8\\
2.6	1.9\\
2.6	2\\
2.6	2.1\\
2.6	2.2\\
2.6	2.3\\
2.6	2.4\\
2.6	2.5\\
2.6	2.6\\
2.6	2.7\\
2.6	2.8\\
2.6	2.9\\
2.6	3\\
2.6	3.1\\
2.6	3.2\\
2.6	3.3\\
2.6	3.4\\
2.6	3.5\\
2.6	3.6\\
2.6	3.7\\
2.6	3.8\\
2.6	3.9\\
2.6	4\\
2.6	4.1\\
2.6	4.2\\
2.6	4.3\\
2.6	4.4\\
2.6	4.5\\
2.6	4.6\\
2.6	4.7\\
2.6	4.8\\
2.6	4.9\\
2.6	5\\
2.6	5.1\\
2.6	5.2\\
2.6	5.3\\
2.6	5.4\\
2.6	5.5\\
2.6	5.6\\
2.6	5.7\\
2.6	5.8\\
2.6	5.9\\
};
\addplot [color=black, draw=none, mark size=0.2pt, mark=*, mark options={solid, black}, forget plot]
  table[row sep=crcr]{%
2.7	0.1\\
2.7	0.2\\
2.7	0.3\\
2.7	0.4\\
2.7	0.5\\
2.7	0.6\\
2.7	0.7\\
2.7	0.8\\
2.7	0.9\\
2.7	1\\
2.7	1.1\\
2.7	1.2\\
2.7	1.3\\
2.7	1.4\\
2.7	1.5\\
2.7	1.6\\
2.7	1.7\\
2.7	1.8\\
2.7	1.9\\
2.7	2\\
2.7	2.1\\
2.7	2.2\\
2.7	2.3\\
2.7	2.4\\
2.7	2.5\\
2.7	2.6\\
2.7	2.7\\
2.7	2.8\\
2.7	2.9\\
2.7	3\\
2.7	3.1\\
2.7	3.2\\
2.7	3.3\\
2.7	3.4\\
2.7	3.5\\
2.7	3.6\\
2.7	3.7\\
2.7	3.8\\
2.7	3.9\\
2.7	4\\
2.7	4.1\\
2.7	4.2\\
2.7	4.3\\
2.7	4.4\\
2.7	4.5\\
2.7	4.6\\
2.7	4.7\\
2.7	4.8\\
2.7	4.9\\
2.7	5\\
2.7	5.1\\
2.7	5.2\\
2.7	5.3\\
2.7	5.4\\
2.7	5.5\\
2.7	5.6\\
2.7	5.7\\
2.7	5.8\\
2.7	5.9\\
};
\addplot [color=black, draw=none, mark size=0.2pt, mark=*, mark options={solid, black}, forget plot]
  table[row sep=crcr]{%
2.8	0.1\\
2.8	0.2\\
2.8	0.3\\
2.8	0.4\\
2.8	0.5\\
2.8	0.6\\
2.8	0.7\\
2.8	0.8\\
2.8	0.9\\
2.8	1\\
2.8	1.1\\
2.8	1.2\\
2.8	1.3\\
2.8	1.4\\
2.8	1.5\\
2.8	1.6\\
2.8	1.7\\
2.8	1.8\\
2.8	1.9\\
2.8	2\\
2.8	2.1\\
2.8	2.2\\
2.8	2.3\\
2.8	2.4\\
2.8	2.5\\
2.8	2.6\\
2.8	2.7\\
2.8	2.8\\
2.8	2.9\\
2.8	3\\
2.8	3.1\\
2.8	3.2\\
2.8	3.3\\
2.8	3.4\\
2.8	3.5\\
2.8	3.6\\
2.8	3.7\\
2.8	3.8\\
2.8	3.9\\
2.8	4\\
2.8	4.1\\
2.8	4.2\\
2.8	4.3\\
2.8	4.4\\
2.8	4.5\\
2.8	4.6\\
2.8	4.7\\
2.8	4.8\\
2.8	4.9\\
2.8	5\\
2.8	5.1\\
2.8	5.2\\
2.8	5.3\\
2.8	5.4\\
2.8	5.5\\
2.8	5.6\\
2.8	5.7\\
2.8	5.8\\
2.8	5.9\\
};
\addplot [color=black, draw=none, mark size=0.2pt, mark=*, mark options={solid, black}, forget plot]
  table[row sep=crcr]{%
2.9	0.1\\
2.9	0.2\\
2.9	0.3\\
2.9	0.4\\
2.9	0.5\\
2.9	0.6\\
2.9	0.7\\
2.9	0.8\\
2.9	0.9\\
2.9	1\\
2.9	1.1\\
2.9	1.2\\
2.9	1.3\\
2.9	1.4\\
2.9	1.5\\
2.9	1.6\\
2.9	1.7\\
2.9	1.8\\
2.9	1.9\\
2.9	2\\
2.9	2.1\\
2.9	2.2\\
2.9	2.3\\
2.9	2.4\\
2.9	2.5\\
2.9	2.6\\
2.9	2.7\\
2.9	2.8\\
2.9	2.9\\
2.9	3\\
2.9	3.1\\
2.9	3.2\\
2.9	3.3\\
2.9	3.4\\
2.9	3.5\\
2.9	3.6\\
2.9	3.7\\
2.9	3.8\\
2.9	3.9\\
2.9	4\\
2.9	4.1\\
2.9	4.2\\
2.9	4.3\\
2.9	4.4\\
2.9	4.5\\
2.9	4.6\\
2.9	4.7\\
2.9	4.8\\
2.9	4.9\\
2.9	5\\
2.9	5.1\\
2.9	5.2\\
2.9	5.3\\
2.9	5.4\\
2.9	5.5\\
2.9	5.6\\
2.9	5.7\\
2.9	5.8\\
2.9	5.9\\
};
\addplot [color=black, draw=none, mark size=0.2pt, mark=*, mark options={solid, black}, forget plot]
  table[row sep=crcr]{%
3	0.1\\
3	0.2\\
3	0.3\\
3	0.4\\
3	0.5\\
3	0.6\\
3	0.7\\
3	0.8\\
3	0.9\\
3	1\\
3	1.1\\
3	1.2\\
3	1.3\\
3	1.4\\
3	1.5\\
3	1.6\\
3	1.7\\
3	1.8\\
3	1.9\\
3	2\\
3	2.1\\
3	2.2\\
3	2.3\\
3	2.4\\
3	2.5\\
3	2.6\\
3	2.7\\
3	2.8\\
3	2.9\\
3	3\\
3	3.1\\
3	3.2\\
3	3.3\\
3	3.4\\
3	3.5\\
3	3.6\\
3	3.7\\
3	3.8\\
3	3.9\\
3	4\\
3	4.1\\
3	4.2\\
3	4.3\\
3	4.4\\
3	4.5\\
3	4.6\\
3	4.7\\
3	4.8\\
3	4.9\\
3	5\\
3	5.1\\
3	5.2\\
3	5.3\\
3	5.4\\
3	5.5\\
3	5.6\\
3	5.7\\
3	5.8\\
3	5.9\\
};
\addplot [color=black, draw=none, mark size=0.2pt, mark=*, mark options={solid, black}, forget plot]
  table[row sep=crcr]{%
3.1	0.1\\
3.1	0.2\\
3.1	0.3\\
3.1	0.4\\
3.1	0.5\\
3.1	0.6\\
3.1	0.7\\
3.1	0.8\\
3.1	0.9\\
3.1	1\\
3.1	1.1\\
3.1	1.2\\
3.1	1.3\\
3.1	1.4\\
3.1	1.5\\
3.1	1.6\\
3.1	1.7\\
3.1	1.8\\
3.1	1.9\\
3.1	2\\
3.1	2.1\\
3.1	2.2\\
3.1	2.3\\
3.1	2.4\\
3.1	2.5\\
3.1	2.6\\
3.1	2.7\\
3.1	2.8\\
3.1	2.9\\
3.1	3\\
3.1	3.1\\
3.1	3.2\\
3.1	3.3\\
3.1	3.4\\
3.1	3.5\\
3.1	3.6\\
3.1	3.7\\
3.1	3.8\\
3.1	3.9\\
3.1	4\\
3.1	4.1\\
3.1	4.2\\
3.1	4.3\\
3.1	4.4\\
3.1	4.5\\
3.1	4.6\\
3.1	4.7\\
3.1	4.8\\
3.1	4.9\\
3.1	5\\
3.1	5.1\\
3.1	5.2\\
3.1	5.3\\
3.1	5.4\\
3.1	5.5\\
3.1	5.6\\
3.1	5.7\\
3.1	5.8\\
3.1	5.9\\
};
\addplot [color=black, draw=none, mark size=0.2pt, mark=*, mark options={solid, black}, forget plot]
  table[row sep=crcr]{%
3.2	0.1\\
3.2	0.2\\
3.2	0.3\\
3.2	0.4\\
3.2	0.5\\
3.2	0.6\\
3.2	0.7\\
3.2	0.8\\
3.2	0.9\\
3.2	1\\
3.2	1.1\\
3.2	1.2\\
3.2	1.3\\
3.2	1.4\\
3.2	1.5\\
3.2	1.6\\
3.2	1.7\\
3.2	1.8\\
3.2	1.9\\
3.2	2\\
3.2	2.1\\
3.2	2.2\\
3.2	2.3\\
3.2	2.4\\
3.2	2.5\\
3.2	2.6\\
3.2	2.7\\
3.2	2.8\\
3.2	2.9\\
3.2	3\\
3.2	3.1\\
3.2	3.2\\
3.2	3.3\\
3.2	3.4\\
3.2	3.5\\
3.2	3.6\\
3.2	3.7\\
3.2	3.8\\
3.2	3.9\\
3.2	4\\
3.2	4.1\\
3.2	4.2\\
3.2	4.3\\
3.2	4.4\\
3.2	4.5\\
3.2	4.6\\
3.2	4.7\\
3.2	4.8\\
3.2	4.9\\
3.2	5\\
3.2	5.1\\
3.2	5.2\\
3.2	5.3\\
3.2	5.4\\
3.2	5.5\\
3.2	5.6\\
3.2	5.7\\
3.2	5.8\\
3.2	5.9\\
};
\addplot [color=black, draw=none, mark size=0.2pt, mark=*, mark options={solid, black}, forget plot]
  table[row sep=crcr]{%
3.3	0.1\\
3.3	0.2\\
3.3	0.3\\
3.3	0.4\\
3.3	0.5\\
3.3	0.6\\
3.3	0.7\\
3.3	0.8\\
3.3	0.9\\
3.3	1\\
3.3	1.1\\
3.3	1.2\\
3.3	1.3\\
3.3	1.4\\
3.3	1.5\\
3.3	1.6\\
3.3	1.7\\
3.3	1.8\\
3.3	1.9\\
3.3	2\\
3.3	2.1\\
3.3	2.2\\
3.3	2.3\\
3.3	2.4\\
3.3	2.5\\
3.3	2.6\\
3.3	2.7\\
3.3	2.8\\
3.3	2.9\\
3.3	3\\
3.3	3.1\\
3.3	3.2\\
3.3	3.3\\
3.3	3.4\\
3.3	3.5\\
3.3	3.6\\
3.3	3.7\\
3.3	3.8\\
3.3	3.9\\
3.3	4\\
3.3	4.1\\
3.3	4.2\\
3.3	4.3\\
3.3	4.4\\
3.3	4.5\\
3.3	4.6\\
3.3	4.7\\
3.3	4.8\\
3.3	4.9\\
3.3	5\\
3.3	5.1\\
3.3	5.2\\
3.3	5.3\\
3.3	5.4\\
3.3	5.5\\
3.3	5.6\\
3.3	5.7\\
3.3	5.8\\
3.3	5.9\\
};
\addplot [color=black, draw=none, mark size=0.2pt, mark=*, mark options={solid, black}, forget plot]
  table[row sep=crcr]{%
3.4	0.1\\
3.4	0.2\\
3.4	0.3\\
3.4	0.4\\
3.4	0.5\\
3.4	0.6\\
3.4	0.7\\
3.4	0.8\\
3.4	0.9\\
3.4	1\\
3.4	1.1\\
3.4	1.2\\
3.4	1.3\\
3.4	1.4\\
3.4	1.5\\
3.4	1.6\\
3.4	1.7\\
3.4	1.8\\
3.4	1.9\\
3.4	2\\
3.4	2.1\\
3.4	2.2\\
3.4	2.3\\
3.4	2.4\\
3.4	2.5\\
3.4	2.6\\
3.4	2.7\\
3.4	2.8\\
3.4	2.9\\
3.4	3\\
3.4	3.1\\
3.4	3.2\\
3.4	3.3\\
3.4	3.4\\
3.4	3.5\\
3.4	3.6\\
3.4	3.7\\
3.4	3.8\\
3.4	3.9\\
3.4	4\\
3.4	4.1\\
3.4	4.2\\
3.4	4.3\\
3.4	4.4\\
3.4	4.5\\
3.4	4.6\\
3.4	4.7\\
3.4	4.8\\
3.4	4.9\\
3.4	5\\
3.4	5.1\\
3.4	5.2\\
3.4	5.3\\
3.4	5.4\\
3.4	5.5\\
3.4	5.6\\
3.4	5.7\\
3.4	5.8\\
3.4	5.9\\
};
\addplot [color=black, draw=none, mark size=0.2pt, mark=*, mark options={solid, black}, forget plot]
  table[row sep=crcr]{%
3.5	0.1\\
3.5	0.2\\
3.5	0.3\\
3.5	0.4\\
3.5	0.5\\
3.5	0.6\\
3.5	0.7\\
3.5	0.8\\
3.5	0.9\\
3.5	1\\
3.5	1.1\\
3.5	1.2\\
3.5	1.3\\
3.5	1.4\\
3.5	1.5\\
3.5	1.6\\
3.5	1.7\\
3.5	1.8\\
3.5	1.9\\
3.5	2\\
3.5	2.1\\
3.5	2.2\\
3.5	2.3\\
3.5	2.4\\
3.5	2.5\\
3.5	2.6\\
3.5	2.7\\
3.5	2.8\\
3.5	2.9\\
3.5	3\\
3.5	3.1\\
3.5	3.2\\
3.5	3.3\\
3.5	3.4\\
3.5	3.5\\
3.5	3.6\\
3.5	3.7\\
3.5	3.8\\
3.5	3.9\\
3.5	4\\
3.5	4.1\\
3.5	4.2\\
3.5	4.3\\
3.5	4.4\\
3.5	4.5\\
3.5	4.6\\
3.5	4.7\\
3.5	4.8\\
3.5	4.9\\
3.5	5\\
3.5	5.1\\
3.5	5.2\\
3.5	5.3\\
3.5	5.4\\
3.5	5.5\\
3.5	5.6\\
3.5	5.7\\
3.5	5.8\\
3.5	5.9\\
};
\addplot [color=black, draw=none, mark size=0.2pt, mark=*, mark options={solid, black}, forget plot]
  table[row sep=crcr]{%
3.6	0.1\\
3.6	0.2\\
3.6	0.3\\
3.6	0.4\\
3.6	0.5\\
3.6	0.6\\
3.6	0.7\\
3.6	0.8\\
3.6	0.9\\
3.6	1\\
3.6	1.1\\
3.6	1.2\\
3.6	1.3\\
3.6	1.4\\
3.6	1.5\\
3.6	1.6\\
3.6	1.7\\
3.6	1.8\\
3.6	1.9\\
3.6	2\\
3.6	2.1\\
3.6	2.2\\
3.6	2.3\\
3.6	2.4\\
3.6	2.5\\
3.6	2.6\\
3.6	2.7\\
3.6	2.8\\
3.6	2.9\\
3.6	3\\
3.6	3.1\\
3.6	3.2\\
3.6	3.3\\
3.6	3.4\\
3.6	3.5\\
3.6	3.6\\
3.6	3.7\\
3.6	3.8\\
3.6	3.9\\
3.6	4\\
3.6	4.1\\
3.6	4.2\\
3.6	4.3\\
3.6	4.4\\
3.6	4.5\\
3.6	4.6\\
3.6	4.7\\
3.6	4.8\\
3.6	4.9\\
3.6	5\\
3.6	5.1\\
3.6	5.2\\
3.6	5.3\\
3.6	5.4\\
3.6	5.5\\
3.6	5.6\\
3.6	5.7\\
3.6	5.8\\
3.6	5.9\\
};
\addplot [color=black, draw=none, mark size=0.2pt, mark=*, mark options={solid, black}, forget plot]
  table[row sep=crcr]{%
3.7	0.1\\
3.7	0.2\\
3.7	0.3\\
3.7	0.4\\
3.7	0.5\\
3.7	0.6\\
3.7	0.7\\
3.7	0.8\\
3.7	0.9\\
3.7	1\\
3.7	1.1\\
3.7	1.2\\
3.7	1.3\\
3.7	1.4\\
3.7	1.5\\
3.7	1.6\\
3.7	1.7\\
3.7	1.8\\
3.7	1.9\\
3.7	2\\
3.7	2.1\\
3.7	2.2\\
3.7	2.3\\
3.7	2.4\\
3.7	2.5\\
3.7	2.6\\
3.7	2.7\\
3.7	2.8\\
3.7	2.9\\
3.7	3\\
3.7	3.1\\
3.7	3.2\\
3.7	3.3\\
3.7	3.4\\
3.7	3.5\\
3.7	3.6\\
3.7	3.7\\
3.7	3.8\\
3.7	3.9\\
3.7	4\\
3.7	4.1\\
3.7	4.2\\
3.7	4.3\\
3.7	4.4\\
3.7	4.5\\
3.7	4.6\\
3.7	4.7\\
3.7	4.8\\
3.7	4.9\\
3.7	5\\
3.7	5.1\\
3.7	5.2\\
3.7	5.3\\
3.7	5.4\\
3.7	5.5\\
3.7	5.6\\
3.7	5.7\\
3.7	5.8\\
3.7	5.9\\
};
\addplot [color=black, draw=none, mark size=0.2pt, mark=*, mark options={solid, black}, forget plot]
  table[row sep=crcr]{%
3.8	0.1\\
3.8	0.2\\
3.8	0.3\\
3.8	0.4\\
3.8	0.5\\
3.8	0.6\\
3.8	0.7\\
3.8	0.8\\
3.8	0.9\\
3.8	1\\
3.8	1.1\\
3.8	1.2\\
3.8	1.3\\
3.8	1.4\\
3.8	1.5\\
3.8	1.6\\
3.8	1.7\\
3.8	1.8\\
3.8	1.9\\
3.8	2\\
3.8	2.1\\
3.8	2.2\\
3.8	2.3\\
3.8	2.4\\
3.8	2.5\\
3.8	2.6\\
3.8	2.7\\
3.8	2.8\\
3.8	2.9\\
3.8	3\\
3.8	3.1\\
3.8	3.2\\
3.8	3.3\\
3.8	3.4\\
3.8	3.5\\
3.8	3.6\\
3.8	3.7\\
3.8	3.8\\
3.8	3.9\\
3.8	4\\
3.8	4.1\\
3.8	4.2\\
3.8	4.3\\
3.8	4.4\\
3.8	4.5\\
3.8	4.6\\
3.8	4.7\\
3.8	4.8\\
3.8	4.9\\
3.8	5\\
3.8	5.1\\
3.8	5.2\\
3.8	5.3\\
3.8	5.4\\
3.8	5.5\\
3.8	5.6\\
3.8	5.7\\
3.8	5.8\\
3.8	5.9\\
};
\addplot [color=black, draw=none, mark size=0.2pt, mark=*, mark options={solid, black}, forget plot]
  table[row sep=crcr]{%
3.9	0.1\\
3.9	0.2\\
3.9	0.3\\
3.9	0.4\\
3.9	0.5\\
3.9	0.6\\
3.9	0.7\\
3.9	0.8\\
3.9	0.9\\
3.9	1\\
3.9	1.1\\
3.9	1.2\\
3.9	1.3\\
3.9	1.4\\
3.9	1.5\\
3.9	1.6\\
3.9	1.7\\
3.9	1.8\\
3.9	1.9\\
3.9	2\\
3.9	2.1\\
3.9	2.2\\
3.9	2.3\\
3.9	2.4\\
3.9	2.5\\
3.9	2.6\\
3.9	2.7\\
3.9	2.8\\
3.9	2.9\\
3.9	3\\
3.9	3.1\\
3.9	3.2\\
3.9	3.3\\
3.9	3.4\\
3.9	3.5\\
3.9	3.6\\
3.9	3.7\\
3.9	3.8\\
3.9	3.9\\
3.9	4\\
3.9	4.1\\
3.9	4.2\\
3.9	4.3\\
3.9	4.4\\
3.9	4.5\\
3.9	4.6\\
3.9	4.7\\
3.9	4.8\\
3.9	4.9\\
3.9	5\\
3.9	5.1\\
3.9	5.2\\
3.9	5.3\\
3.9	5.4\\
3.9	5.5\\
3.9	5.6\\
3.9	5.7\\
3.9	5.8\\
3.9	5.9\\
};
\addplot [color=black, draw=none, mark size=0.2pt, mark=*, mark options={solid, black}, forget plot]
  table[row sep=crcr]{%
4	0.1\\
4	0.2\\
4	0.3\\
4	0.4\\
4	0.5\\
4	0.6\\
4	0.7\\
4	0.8\\
4	0.9\\
4	1\\
4	1.1\\
4	1.2\\
4	1.3\\
4	1.4\\
4	1.5\\
4	1.6\\
4	1.7\\
4	1.8\\
4	1.9\\
4	2\\
4	2.1\\
4	2.2\\
4	2.3\\
4	2.4\\
4	2.5\\
4	2.6\\
4	2.7\\
4	2.8\\
4	2.9\\
4	3\\
4	3.1\\
4	3.2\\
4	3.3\\
4	3.4\\
4	3.5\\
4	3.6\\
4	3.7\\
4	3.8\\
4	3.9\\
4	4\\
4	4.1\\
4	4.2\\
4	4.3\\
4	4.4\\
4	4.5\\
4	4.6\\
4	4.7\\
4	4.8\\
4	4.9\\
4	5\\
4	5.1\\
4	5.2\\
4	5.3\\
4	5.4\\
4	5.5\\
4	5.6\\
4	5.7\\
4	5.8\\
4	5.9\\
};
\addplot [color=black, draw=none, mark size=0.2pt, mark=*, mark options={solid, black}, forget plot]
  table[row sep=crcr]{%
4.1	0.1\\
4.1	0.2\\
4.1	0.3\\
4.1	0.4\\
4.1	0.5\\
4.1	0.6\\
4.1	0.7\\
4.1	0.8\\
4.1	0.9\\
4.1	1\\
4.1	1.1\\
4.1	1.2\\
4.1	1.3\\
4.1	1.4\\
4.1	1.5\\
4.1	1.6\\
4.1	1.7\\
4.1	1.8\\
4.1	1.9\\
4.1	2\\
4.1	2.1\\
4.1	2.2\\
4.1	2.3\\
4.1	2.4\\
4.1	2.5\\
4.1	2.6\\
4.1	2.7\\
4.1	2.8\\
4.1	2.9\\
4.1	3\\
4.1	3.1\\
4.1	3.2\\
4.1	3.3\\
4.1	3.4\\
4.1	3.5\\
4.1	3.6\\
4.1	3.7\\
4.1	3.8\\
4.1	3.9\\
4.1	4\\
4.1	4.1\\
4.1	4.2\\
4.1	4.3\\
4.1	4.4\\
4.1	4.5\\
4.1	4.6\\
4.1	4.7\\
4.1	4.8\\
4.1	4.9\\
4.1	5\\
4.1	5.1\\
4.1	5.2\\
4.1	5.3\\
4.1	5.4\\
4.1	5.5\\
4.1	5.6\\
4.1	5.7\\
4.1	5.8\\
4.1	5.9\\
};
\addplot [color=black, draw=none, mark size=0.2pt, mark=*, mark options={solid, black}, forget plot]
  table[row sep=crcr]{%
4.2	0.1\\
4.2	0.2\\
4.2	0.3\\
4.2	0.4\\
4.2	0.5\\
4.2	0.6\\
4.2	0.7\\
4.2	0.8\\
4.2	0.9\\
4.2	1\\
4.2	1.1\\
4.2	1.2\\
4.2	1.3\\
4.2	1.4\\
4.2	1.5\\
4.2	1.6\\
4.2	1.7\\
4.2	1.8\\
4.2	1.9\\
4.2	2\\
4.2	2.1\\
4.2	2.2\\
4.2	2.3\\
4.2	2.4\\
4.2	2.5\\
4.2	2.6\\
4.2	2.7\\
4.2	2.8\\
4.2	2.9\\
4.2	3\\
4.2	3.1\\
4.2	3.2\\
4.2	3.3\\
4.2	3.4\\
4.2	3.5\\
4.2	3.6\\
4.2	3.7\\
4.2	3.8\\
4.2	3.9\\
4.2	4\\
4.2	4.1\\
4.2	4.2\\
4.2	4.3\\
4.2	4.4\\
4.2	4.5\\
4.2	4.6\\
4.2	4.7\\
4.2	4.8\\
4.2	4.9\\
4.2	5\\
4.2	5.1\\
4.2	5.2\\
4.2	5.3\\
4.2	5.4\\
4.2	5.5\\
4.2	5.6\\
4.2	5.7\\
4.2	5.8\\
4.2	5.9\\
};
\addplot [color=black, draw=none, mark size=0.2pt, mark=*, mark options={solid, black}, forget plot]
  table[row sep=crcr]{%
4.3	0.1\\
4.3	0.2\\
4.3	0.3\\
4.3	0.4\\
4.3	0.5\\
4.3	0.6\\
4.3	0.7\\
4.3	0.8\\
4.3	0.9\\
4.3	1\\
4.3	1.1\\
4.3	1.2\\
4.3	1.3\\
4.3	1.4\\
4.3	1.5\\
4.3	1.6\\
4.3	1.7\\
4.3	1.8\\
4.3	1.9\\
4.3	2\\
4.3	2.1\\
4.3	2.2\\
4.3	2.3\\
4.3	2.4\\
4.3	2.5\\
4.3	2.6\\
4.3	2.7\\
4.3	2.8\\
4.3	2.9\\
4.3	3\\
4.3	3.1\\
4.3	3.2\\
4.3	3.3\\
4.3	3.4\\
4.3	3.5\\
4.3	3.6\\
4.3	3.7\\
4.3	3.8\\
4.3	3.9\\
4.3	4\\
4.3	4.1\\
4.3	4.2\\
4.3	4.3\\
4.3	4.4\\
4.3	4.5\\
4.3	4.6\\
4.3	4.7\\
4.3	4.8\\
4.3	4.9\\
4.3	5\\
4.3	5.1\\
4.3	5.2\\
4.3	5.3\\
4.3	5.4\\
4.3	5.5\\
4.3	5.6\\
4.3	5.7\\
4.3	5.8\\
4.3	5.9\\
};
\addplot [color=black, draw=none, mark size=0.2pt, mark=*, mark options={solid, black}, forget plot]
  table[row sep=crcr]{%
4.4	0.1\\
4.4	0.2\\
4.4	0.3\\
4.4	0.4\\
4.4	0.5\\
4.4	0.6\\
4.4	0.7\\
4.4	0.8\\
4.4	0.9\\
4.4	1\\
4.4	1.1\\
4.4	1.2\\
4.4	1.3\\
4.4	1.4\\
4.4	1.5\\
4.4	1.6\\
4.4	1.7\\
4.4	1.8\\
4.4	1.9\\
4.4	2\\
4.4	2.1\\
4.4	2.2\\
4.4	2.3\\
4.4	2.4\\
4.4	2.5\\
4.4	2.6\\
4.4	2.7\\
4.4	2.8\\
4.4	2.9\\
4.4	3\\
4.4	3.1\\
4.4	3.2\\
4.4	3.3\\
4.4	3.4\\
4.4	3.5\\
4.4	3.6\\
4.4	3.7\\
4.4	3.8\\
4.4	3.9\\
4.4	4\\
4.4	4.1\\
4.4	4.2\\
4.4	4.3\\
4.4	4.4\\
4.4	4.5\\
4.4	4.6\\
4.4	4.7\\
4.4	4.8\\
4.4	4.9\\
4.4	5\\
4.4	5.1\\
4.4	5.2\\
4.4	5.3\\
4.4	5.4\\
4.4	5.5\\
4.4	5.6\\
4.4	5.7\\
4.4	5.8\\
4.4	5.9\\
};
\addplot [color=black, draw=none, mark size=0.2pt, mark=*, mark options={solid, black}, forget plot]
  table[row sep=crcr]{%
4.5	0.1\\
4.5	0.2\\
4.5	0.3\\
4.5	0.4\\
4.5	0.5\\
4.5	0.6\\
4.5	0.7\\
4.5	0.8\\
4.5	0.9\\
4.5	1\\
4.5	1.1\\
4.5	1.2\\
4.5	1.3\\
4.5	1.4\\
4.5	1.5\\
4.5	1.6\\
4.5	1.7\\
4.5	1.8\\
4.5	1.9\\
4.5	2\\
4.5	2.1\\
4.5	2.2\\
4.5	2.3\\
4.5	2.4\\
4.5	2.5\\
4.5	2.6\\
4.5	2.7\\
4.5	2.8\\
4.5	2.9\\
4.5	3\\
4.5	3.1\\
4.5	3.2\\
4.5	3.3\\
4.5	3.4\\
4.5	3.5\\
4.5	3.6\\
4.5	3.7\\
4.5	3.8\\
4.5	3.9\\
4.5	4\\
4.5	4.1\\
4.5	4.2\\
4.5	4.3\\
4.5	4.4\\
4.5	4.5\\
4.5	4.6\\
4.5	4.7\\
4.5	4.8\\
4.5	4.9\\
4.5	5\\
4.5	5.1\\
4.5	5.2\\
4.5	5.3\\
4.5	5.4\\
4.5	5.5\\
4.5	5.6\\
4.5	5.7\\
4.5	5.8\\
4.5	5.9\\
};
\addplot [color=black, draw=none, mark size=0.2pt, mark=*, mark options={solid, black}, forget plot]
  table[row sep=crcr]{%
4.6	0.1\\
4.6	0.2\\
4.6	0.3\\
4.6	0.4\\
4.6	0.5\\
4.6	0.6\\
4.6	0.7\\
4.6	0.8\\
4.6	0.9\\
4.6	1\\
4.6	1.1\\
4.6	1.2\\
4.6	1.3\\
4.6	1.4\\
4.6	1.5\\
4.6	1.6\\
4.6	1.7\\
4.6	1.8\\
4.6	1.9\\
4.6	2\\
4.6	2.1\\
4.6	2.2\\
4.6	2.3\\
4.6	2.4\\
4.6	2.5\\
4.6	2.6\\
4.6	2.7\\
4.6	2.8\\
4.6	2.9\\
4.6	3\\
4.6	3.1\\
4.6	3.2\\
4.6	3.3\\
4.6	3.4\\
4.6	3.5\\
4.6	3.6\\
4.6	3.7\\
4.6	3.8\\
4.6	3.9\\
4.6	4\\
4.6	4.1\\
4.6	4.2\\
4.6	4.3\\
4.6	4.4\\
4.6	4.5\\
4.6	4.6\\
4.6	4.7\\
4.6	4.8\\
4.6	4.9\\
4.6	5\\
4.6	5.1\\
4.6	5.2\\
4.6	5.3\\
4.6	5.4\\
4.6	5.5\\
4.6	5.6\\
4.6	5.7\\
4.6	5.8\\
4.6	5.9\\
};
\addplot [color=black, draw=none, mark size=0.2pt, mark=*, mark options={solid, black}, forget plot]
  table[row sep=crcr]{%
4.7	0.1\\
4.7	0.2\\
4.7	0.3\\
4.7	0.4\\
4.7	0.5\\
4.7	0.6\\
4.7	0.7\\
4.7	0.8\\
4.7	0.9\\
4.7	1\\
4.7	1.1\\
4.7	1.2\\
4.7	1.3\\
4.7	1.4\\
4.7	1.5\\
4.7	1.6\\
4.7	1.7\\
4.7	1.8\\
4.7	1.9\\
4.7	2\\
4.7	2.1\\
4.7	2.2\\
4.7	2.3\\
4.7	2.4\\
4.7	2.5\\
4.7	2.6\\
4.7	2.7\\
4.7	2.8\\
4.7	2.9\\
4.7	3\\
4.7	3.1\\
4.7	3.2\\
4.7	3.3\\
4.7	3.4\\
4.7	3.5\\
4.7	3.6\\
4.7	3.7\\
4.7	3.8\\
4.7	3.9\\
4.7	4\\
4.7	4.1\\
4.7	4.2\\
4.7	4.3\\
4.7	4.4\\
4.7	4.5\\
4.7	4.6\\
4.7	4.7\\
4.7	4.8\\
4.7	4.9\\
4.7	5\\
4.7	5.1\\
4.7	5.2\\
4.7	5.3\\
4.7	5.4\\
4.7	5.5\\
4.7	5.6\\
4.7	5.7\\
4.7	5.8\\
4.7	5.9\\
};
\addplot [color=black, draw=none, mark size=0.2pt, mark=*, mark options={solid, black}, forget plot]
  table[row sep=crcr]{%
4.8	0.1\\
4.8	0.2\\
4.8	0.3\\
4.8	0.4\\
4.8	0.5\\
4.8	0.6\\
4.8	0.7\\
4.8	0.8\\
4.8	0.9\\
4.8	1\\
4.8	1.1\\
4.8	1.2\\
4.8	1.3\\
4.8	1.4\\
4.8	1.5\\
4.8	1.6\\
4.8	1.7\\
4.8	1.8\\
4.8	1.9\\
4.8	2\\
4.8	2.1\\
4.8	2.2\\
4.8	2.3\\
4.8	2.4\\
4.8	2.5\\
4.8	2.6\\
4.8	2.7\\
4.8	2.8\\
4.8	2.9\\
4.8	3\\
4.8	3.1\\
4.8	3.2\\
4.8	3.3\\
4.8	3.4\\
4.8	3.5\\
4.8	3.6\\
4.8	3.7\\
4.8	3.8\\
4.8	3.9\\
4.8	4\\
4.8	4.1\\
4.8	4.2\\
4.8	4.3\\
4.8	4.4\\
4.8	4.5\\
4.8	4.6\\
4.8	4.7\\
4.8	4.8\\
4.8	4.9\\
4.8	5\\
4.8	5.1\\
4.8	5.2\\
4.8	5.3\\
4.8	5.4\\
4.8	5.5\\
4.8	5.6\\
4.8	5.7\\
4.8	5.8\\
4.8	5.9\\
};
\addplot [color=black, draw=none, mark size=0.2pt, mark=*, mark options={solid, black}, forget plot]
  table[row sep=crcr]{%
4.9	0.1\\
4.9	0.2\\
4.9	0.3\\
4.9	0.4\\
4.9	0.5\\
4.9	0.6\\
4.9	0.7\\
4.9	0.8\\
4.9	0.9\\
4.9	1\\
4.9	1.1\\
4.9	1.2\\
4.9	1.3\\
4.9	1.4\\
4.9	1.5\\
4.9	1.6\\
4.9	1.7\\
4.9	1.8\\
4.9	1.9\\
4.9	2\\
4.9	2.1\\
4.9	2.2\\
4.9	2.3\\
4.9	2.4\\
4.9	2.5\\
4.9	2.6\\
4.9	2.7\\
4.9	2.8\\
4.9	2.9\\
4.9	3\\
4.9	3.1\\
4.9	3.2\\
4.9	3.3\\
4.9	3.4\\
4.9	3.5\\
4.9	3.6\\
4.9	3.7\\
4.9	3.8\\
4.9	3.9\\
4.9	4\\
4.9	4.1\\
4.9	4.2\\
4.9	4.3\\
4.9	4.4\\
4.9	4.5\\
4.9	4.6\\
4.9	4.7\\
4.9	4.8\\
4.9	4.9\\
4.9	5\\
4.9	5.1\\
4.9	5.2\\
4.9	5.3\\
4.9	5.4\\
4.9	5.5\\
4.9	5.6\\
4.9	5.7\\
4.9	5.8\\
4.9	5.9\\
};
\addplot [color=black, draw=none, mark size=0.2pt, mark=*, mark options={solid, black}, forget plot]
  table[row sep=crcr]{%
5	0.1\\
5	0.2\\
5	0.3\\
5	0.4\\
5	0.5\\
5	0.6\\
5	0.7\\
5	0.8\\
5	0.9\\
5	1\\
5	1.1\\
5	1.2\\
5	1.3\\
5	1.4\\
5	1.5\\
5	1.6\\
5	1.7\\
5	1.8\\
5	1.9\\
5	2\\
5	2.1\\
5	2.2\\
5	2.3\\
5	2.4\\
5	2.5\\
5	2.6\\
5	2.7\\
5	2.8\\
5	2.9\\
5	3\\
5	3.1\\
5	3.2\\
5	3.3\\
5	3.4\\
5	3.5\\
5	3.6\\
5	3.7\\
5	3.8\\
5	3.9\\
5	4\\
5	4.1\\
5	4.2\\
5	4.3\\
5	4.4\\
5	4.5\\
5	4.6\\
5	4.7\\
5	4.8\\
5	4.9\\
5	5\\
5	5.1\\
5	5.2\\
5	5.3\\
5	5.4\\
5	5.5\\
5	5.6\\
5	5.7\\
5	5.8\\
5	5.9\\
};
\addplot [color=black, draw=none, mark size=0.2pt, mark=*, mark options={solid, black}, forget plot]
  table[row sep=crcr]{%
5.1	0.1\\
5.1	0.2\\
5.1	0.3\\
5.1	0.4\\
5.1	0.5\\
5.1	0.6\\
5.1	0.7\\
5.1	0.8\\
5.1	0.9\\
5.1	1\\
5.1	1.1\\
5.1	1.2\\
5.1	1.3\\
5.1	1.4\\
5.1	1.5\\
5.1	1.6\\
5.1	1.7\\
5.1	1.8\\
5.1	1.9\\
5.1	2\\
5.1	2.1\\
5.1	2.2\\
5.1	2.3\\
5.1	2.4\\
5.1	2.5\\
5.1	2.6\\
5.1	2.7\\
5.1	2.8\\
5.1	2.9\\
5.1	3\\
5.1	3.1\\
5.1	3.2\\
5.1	3.3\\
5.1	3.4\\
5.1	3.5\\
5.1	3.6\\
5.1	3.7\\
5.1	3.8\\
5.1	3.9\\
5.1	4\\
5.1	4.1\\
5.1	4.2\\
5.1	4.3\\
5.1	4.4\\
5.1	4.5\\
5.1	4.6\\
5.1	4.7\\
5.1	4.8\\
5.1	4.9\\
5.1	5\\
5.1	5.1\\
5.1	5.2\\
5.1	5.3\\
5.1	5.4\\
5.1	5.5\\
5.1	5.6\\
5.1	5.7\\
5.1	5.8\\
5.1	5.9\\
};
\addplot [color=black, draw=none, mark size=0.2pt, mark=*, mark options={solid, black}, forget plot]
  table[row sep=crcr]{%
5.2	0.1\\
5.2	0.2\\
5.2	0.3\\
5.2	0.4\\
5.2	0.5\\
5.2	0.6\\
5.2	0.7\\
5.2	0.8\\
5.2	0.9\\
5.2	1\\
5.2	1.1\\
5.2	1.2\\
5.2	1.3\\
5.2	1.4\\
5.2	1.5\\
5.2	1.6\\
5.2	1.7\\
5.2	1.8\\
5.2	1.9\\
5.2	2\\
5.2	2.1\\
5.2	2.2\\
5.2	2.3\\
5.2	2.4\\
5.2	2.5\\
5.2	2.6\\
5.2	2.7\\
5.2	2.8\\
5.2	2.9\\
5.2	3\\
5.2	3.1\\
5.2	3.2\\
5.2	3.3\\
5.2	3.4\\
5.2	3.5\\
5.2	3.6\\
5.2	3.7\\
5.2	3.8\\
5.2	3.9\\
5.2	4\\
5.2	4.1\\
5.2	4.2\\
5.2	4.3\\
5.2	4.4\\
5.2	4.5\\
5.2	4.6\\
5.2	4.7\\
5.2	4.8\\
5.2	4.9\\
5.2	5\\
5.2	5.1\\
5.2	5.2\\
5.2	5.3\\
5.2	5.4\\
5.2	5.5\\
5.2	5.6\\
5.2	5.7\\
5.2	5.8\\
5.2	5.9\\
};
\addplot [color=black, draw=none, mark size=0.2pt, mark=*, mark options={solid, black}, forget plot]
  table[row sep=crcr]{%
5.3	0.1\\
5.3	0.2\\
5.3	0.3\\
5.3	0.4\\
5.3	0.5\\
5.3	0.6\\
5.3	0.7\\
5.3	0.8\\
5.3	0.9\\
5.3	1\\
5.3	1.1\\
5.3	1.2\\
5.3	1.3\\
5.3	1.4\\
5.3	1.5\\
5.3	1.6\\
5.3	1.7\\
5.3	1.8\\
5.3	1.9\\
5.3	2\\
5.3	2.1\\
5.3	2.2\\
5.3	2.3\\
5.3	2.4\\
5.3	2.5\\
5.3	2.6\\
5.3	2.7\\
5.3	2.8\\
5.3	2.9\\
5.3	3\\
5.3	3.1\\
5.3	3.2\\
5.3	3.3\\
5.3	3.4\\
5.3	3.5\\
5.3	3.6\\
5.3	3.7\\
5.3	3.8\\
5.3	3.9\\
5.3	4\\
5.3	4.1\\
5.3	4.2\\
5.3	4.3\\
5.3	4.4\\
5.3	4.5\\
5.3	4.6\\
5.3	4.7\\
5.3	4.8\\
5.3	4.9\\
5.3	5\\
5.3	5.1\\
5.3	5.2\\
5.3	5.3\\
5.3	5.4\\
5.3	5.5\\
5.3	5.6\\
5.3	5.7\\
5.3	5.8\\
5.3	5.9\\
};
\addplot [color=black, draw=none, mark size=0.2pt, mark=*, mark options={solid, black}, forget plot]
  table[row sep=crcr]{%
5.4	0.1\\
5.4	0.2\\
5.4	0.3\\
5.4	0.4\\
5.4	0.5\\
5.4	0.6\\
5.4	0.7\\
5.4	0.8\\
5.4	0.9\\
5.4	1\\
5.4	1.1\\
5.4	1.2\\
5.4	1.3\\
5.4	1.4\\
5.4	1.5\\
5.4	1.6\\
5.4	1.7\\
5.4	1.8\\
5.4	1.9\\
5.4	2\\
5.4	2.1\\
5.4	2.2\\
5.4	2.3\\
5.4	2.4\\
5.4	2.5\\
5.4	2.6\\
5.4	2.7\\
5.4	2.8\\
5.4	2.9\\
5.4	3\\
5.4	3.1\\
5.4	3.2\\
5.4	3.3\\
5.4	3.4\\
5.4	3.5\\
5.4	3.6\\
5.4	3.7\\
5.4	3.8\\
5.4	3.9\\
5.4	4\\
5.4	4.1\\
5.4	4.2\\
5.4	4.3\\
5.4	4.4\\
5.4	4.5\\
5.4	4.6\\
5.4	4.7\\
5.4	4.8\\
5.4	4.9\\
5.4	5\\
5.4	5.1\\
5.4	5.2\\
5.4	5.3\\
5.4	5.4\\
5.4	5.5\\
5.4	5.6\\
5.4	5.7\\
5.4	5.8\\
5.4	5.9\\
};
\addplot [color=black, draw=none, mark size=0.2pt, mark=*, mark options={solid, black}, forget plot]
  table[row sep=crcr]{%
5.5	0.1\\
5.5	0.2\\
5.5	0.3\\
5.5	0.4\\
5.5	0.5\\
5.5	0.6\\
5.5	0.7\\
5.5	0.8\\
5.5	0.9\\
5.5	1\\
5.5	1.1\\
5.5	1.2\\
5.5	1.3\\
5.5	1.4\\
5.5	1.5\\
5.5	1.6\\
5.5	1.7\\
5.5	1.8\\
5.5	1.9\\
5.5	2\\
5.5	2.1\\
5.5	2.2\\
5.5	2.3\\
5.5	2.4\\
5.5	2.5\\
5.5	2.6\\
5.5	2.7\\
5.5	2.8\\
5.5	2.9\\
5.5	3\\
5.5	3.1\\
5.5	3.2\\
5.5	3.3\\
5.5	3.4\\
5.5	3.5\\
5.5	3.6\\
5.5	3.7\\
5.5	3.8\\
5.5	3.9\\
5.5	4\\
5.5	4.1\\
5.5	4.2\\
5.5	4.3\\
5.5	4.4\\
5.5	4.5\\
5.5	4.6\\
5.5	4.7\\
5.5	4.8\\
5.5	4.9\\
5.5	5\\
5.5	5.1\\
5.5	5.2\\
5.5	5.3\\
5.5	5.4\\
5.5	5.5\\
5.5	5.6\\
5.5	5.7\\
5.5	5.8\\
5.5	5.9\\
};
\addplot [color=black, draw=none, mark size=0.2pt, mark=*, mark options={solid, black}, forget plot]
  table[row sep=crcr]{%
5.6	0.1\\
5.6	0.2\\
5.6	0.3\\
5.6	0.4\\
5.6	0.5\\
5.6	0.6\\
5.6	0.7\\
5.6	0.8\\
5.6	0.9\\
5.6	1\\
5.6	1.1\\
5.6	1.2\\
5.6	1.3\\
5.6	1.4\\
5.6	1.5\\
5.6	1.6\\
5.6	1.7\\
5.6	1.8\\
5.6	1.9\\
5.6	2\\
5.6	2.1\\
5.6	2.2\\
5.6	2.3\\
5.6	2.4\\
5.6	2.5\\
5.6	2.6\\
5.6	2.7\\
5.6	2.8\\
5.6	2.9\\
5.6	3\\
5.6	3.1\\
5.6	3.2\\
5.6	3.3\\
5.6	3.4\\
5.6	3.5\\
5.6	3.6\\
5.6	3.7\\
5.6	3.8\\
5.6	3.9\\
5.6	4\\
5.6	4.1\\
5.6	4.2\\
5.6	4.3\\
5.6	4.4\\
5.6	4.5\\
5.6	4.6\\
5.6	4.7\\
5.6	4.8\\
5.6	4.9\\
5.6	5\\
5.6	5.1\\
5.6	5.2\\
5.6	5.3\\
5.6	5.4\\
5.6	5.5\\
5.6	5.6\\
5.6	5.7\\
5.6	5.8\\
5.6	5.9\\
};
\addplot [color=black, draw=none, mark size=0.2pt, mark=*, mark options={solid, black}, forget plot]
  table[row sep=crcr]{%
5.7	0.1\\
5.7	0.2\\
5.7	0.3\\
5.7	0.4\\
5.7	0.5\\
5.7	0.6\\
5.7	0.7\\
5.7	0.8\\
5.7	0.9\\
5.7	1\\
5.7	1.1\\
5.7	1.2\\
5.7	1.3\\
5.7	1.4\\
5.7	1.5\\
5.7	1.6\\
5.7	1.7\\
5.7	1.8\\
5.7	1.9\\
5.7	2\\
5.7	2.1\\
5.7	2.2\\
5.7	2.3\\
5.7	2.4\\
5.7	2.5\\
5.7	2.6\\
5.7	2.7\\
5.7	2.8\\
5.7	2.9\\
5.7	3\\
5.7	3.1\\
5.7	3.2\\
5.7	3.3\\
5.7	3.4\\
5.7	3.5\\
5.7	3.6\\
5.7	3.7\\
5.7	3.8\\
5.7	3.9\\
5.7	4\\
5.7	4.1\\
5.7	4.2\\
5.7	4.3\\
5.7	4.4\\
5.7	4.5\\
5.7	4.6\\
5.7	4.7\\
5.7	4.8\\
5.7	4.9\\
5.7	5\\
5.7	5.1\\
5.7	5.2\\
5.7	5.3\\
5.7	5.4\\
5.7	5.5\\
5.7	5.6\\
5.7	5.7\\
5.7	5.8\\
5.7	5.9\\
};
\addplot [color=black, draw=none, mark size=0.2pt, mark=*, mark options={solid, black}, forget plot]
  table[row sep=crcr]{%
5.8	0.1\\
5.8	0.2\\
5.8	0.3\\
5.8	0.4\\
5.8	0.5\\
5.8	0.6\\
5.8	0.7\\
5.8	0.8\\
5.8	0.9\\
5.8	1\\
5.8	1.1\\
5.8	1.2\\
5.8	1.3\\
5.8	1.4\\
5.8	1.5\\
5.8	1.6\\
5.8	1.7\\
5.8	1.8\\
5.8	1.9\\
5.8	2\\
5.8	2.1\\
5.8	2.2\\
5.8	2.3\\
5.8	2.4\\
5.8	2.5\\
5.8	2.6\\
5.8	2.7\\
5.8	2.8\\
5.8	2.9\\
5.8	3\\
5.8	3.1\\
5.8	3.2\\
5.8	3.3\\
5.8	3.4\\
5.8	3.5\\
5.8	3.6\\
5.8	3.7\\
5.8	3.8\\
5.8	3.9\\
5.8	4\\
5.8	4.1\\
5.8	4.2\\
5.8	4.3\\
5.8	4.4\\
5.8	4.5\\
5.8	4.6\\
5.8	4.7\\
5.8	4.8\\
5.8	4.9\\
5.8	5\\
5.8	5.1\\
5.8	5.2\\
5.8	5.3\\
5.8	5.4\\
5.8	5.5\\
5.8	5.6\\
5.8	5.7\\
5.8	5.8\\
5.8	5.9\\
};
\addplot [color=black, draw=none, mark size=0.2pt, mark=*, mark options={solid, black}, forget plot]
  table[row sep=crcr]{%
5.9	0.1\\
5.9	0.2\\
5.9	0.3\\
5.9	0.4\\
5.9	0.5\\
5.9	0.6\\
5.9	0.7\\
5.9	0.8\\
5.9	0.9\\
5.9	1\\
5.9	1.1\\
5.9	1.2\\
5.9	1.3\\
5.9	1.4\\
5.9	1.5\\
5.9	1.6\\
5.9	1.7\\
5.9	1.8\\
5.9	1.9\\
5.9	2\\
5.9	2.1\\
5.9	2.2\\
5.9	2.3\\
5.9	2.4\\
5.9	2.5\\
5.9	2.6\\
5.9	2.7\\
5.9	2.8\\
5.9	2.9\\
5.9	3\\
5.9	3.1\\
5.9	3.2\\
5.9	3.3\\
5.9	3.4\\
5.9	3.5\\
5.9	3.6\\
5.9	3.7\\
5.9	3.8\\
5.9	3.9\\
5.9	4\\
5.9	4.1\\
5.9	4.2\\
5.9	4.3\\
5.9	4.4\\
5.9	4.5\\
5.9	4.6\\
5.9	4.7\\
5.9	4.8\\
5.9	4.9\\
5.9	5\\
5.9	5.1\\
5.9	5.2\\
5.9	5.3\\
5.9	5.4\\
5.9	5.5\\
5.9	5.6\\
5.9	5.7\\
5.9	5.8\\
5.9	5.9\\
};
\addplot [color=white, draw=none, mark size=0.2pt, mark=*, mark options={solid, white}, forget plot]
  table[row sep=crcr]{%
1.2	1.2\\
1.2	1.3\\
1.2	1.4\\
1.2	1.5\\
1.2	1.6\\
1.2	1.7\\
1.2	1.8\\
1.2	1.9\\
1.2	2\\
1.2	2.1\\
1.2	2.2\\
1.2	2.3\\
1.2	2.4\\
1.2	2.5\\
1.2	2.6\\
1.2	2.7\\
1.2	2.8\\
1.2	2.9\\
1.2	3\\
1.2	3.1\\
1.2	3.2\\
1.2	3.3\\
1.2	3.4\\
1.2	3.5\\
1.2	3.6\\
1.2	3.7\\
1.2	3.8\\
1.2	3.9\\
1.2	4\\
1.2	4.1\\
1.2	4.2\\
1.2	4.3\\
1.2	4.4\\
1.2	4.5\\
1.2	4.6\\
1.2	4.7\\
1.2	4.8\\
};
\addplot [color=white, draw=none, mark size=0.2pt, mark=*, mark options={solid, white}, forget plot]
  table[row sep=crcr]{%
1.3	1.2\\
1.3	1.3\\
1.3	1.4\\
1.3	1.5\\
1.3	1.6\\
1.3	1.7\\
1.3	1.8\\
1.3	1.9\\
1.3	2\\
1.3	2.1\\
1.3	2.2\\
1.3	2.3\\
1.3	2.4\\
1.3	2.5\\
1.3	2.6\\
1.3	2.7\\
1.3	2.8\\
1.3	2.9\\
1.3	3\\
1.3	3.1\\
1.3	3.2\\
1.3	3.3\\
1.3	3.4\\
1.3	3.5\\
1.3	3.6\\
1.3	3.7\\
1.3	3.8\\
1.3	3.9\\
1.3	4\\
1.3	4.1\\
1.3	4.2\\
1.3	4.3\\
1.3	4.4\\
1.3	4.5\\
1.3	4.6\\
1.3	4.7\\
1.3	4.8\\
};
\addplot [color=white, draw=none, mark size=0.2pt, mark=*, mark options={solid, white}, forget plot]
  table[row sep=crcr]{%
1.4	1.2\\
1.4	1.3\\
1.4	1.4\\
1.4	1.5\\
1.4	1.6\\
1.4	1.7\\
1.4	1.8\\
1.4	1.9\\
1.4	2\\
1.4	2.1\\
1.4	2.2\\
1.4	2.3\\
1.4	2.4\\
1.4	2.5\\
1.4	2.6\\
1.4	2.7\\
1.4	2.8\\
1.4	2.9\\
1.4	3\\
1.4	3.1\\
1.4	3.2\\
1.4	3.3\\
1.4	3.4\\
1.4	3.5\\
1.4	3.6\\
1.4	3.7\\
1.4	3.8\\
1.4	3.9\\
1.4	4\\
1.4	4.1\\
1.4	4.2\\
1.4	4.3\\
1.4	4.4\\
1.4	4.5\\
1.4	4.6\\
1.4	4.7\\
1.4	4.8\\
};
\addplot [color=white, draw=none, mark size=0.2pt, mark=*, mark options={solid, white}, forget plot]
  table[row sep=crcr]{%
1.5	1.2\\
1.5	1.3\\
1.5	1.4\\
1.5	1.5\\
1.5	1.6\\
1.5	1.7\\
1.5	1.8\\
1.5	1.9\\
1.5	2\\
1.5	2.1\\
1.5	2.2\\
1.5	2.3\\
1.5	2.4\\
1.5	2.5\\
1.5	2.6\\
1.5	2.7\\
1.5	2.8\\
1.5	2.9\\
1.5	3\\
1.5	3.1\\
1.5	3.2\\
1.5	3.3\\
1.5	3.4\\
1.5	3.5\\
1.5	3.6\\
1.5	3.7\\
1.5	3.8\\
1.5	3.9\\
1.5	4\\
1.5	4.1\\
1.5	4.2\\
1.5	4.3\\
1.5	4.4\\
1.5	4.5\\
1.5	4.6\\
1.5	4.7\\
1.5	4.8\\
};
\addplot [color=white, draw=none, mark size=0.2pt, mark=*, mark options={solid, white}, forget plot]
  table[row sep=crcr]{%
1.6	1.2\\
1.6	1.3\\
1.6	1.4\\
1.6	1.5\\
1.6	1.6\\
1.6	1.7\\
1.6	1.8\\
1.6	1.9\\
1.6	2\\
1.6	2.1\\
1.6	2.2\\
1.6	2.3\\
1.6	2.4\\
1.6	2.5\\
1.6	2.6\\
1.6	2.7\\
1.6	2.8\\
1.6	2.9\\
1.6	3\\
1.6	3.1\\
1.6	3.2\\
1.6	3.3\\
1.6	3.4\\
1.6	3.5\\
1.6	3.6\\
1.6	3.7\\
1.6	3.8\\
1.6	3.9\\
1.6	4\\
1.6	4.1\\
1.6	4.2\\
1.6	4.3\\
1.6	4.4\\
1.6	4.5\\
1.6	4.6\\
1.6	4.7\\
1.6	4.8\\
};
\addplot [color=white, draw=none, mark size=0.2pt, mark=*, mark options={solid, white}, forget plot]
  table[row sep=crcr]{%
1.7	1.2\\
1.7	1.3\\
1.7	1.4\\
1.7	1.5\\
1.7	1.6\\
1.7	1.7\\
1.7	1.8\\
1.7	1.9\\
1.7	2\\
1.7	2.1\\
1.7	2.2\\
1.7	2.3\\
1.7	2.4\\
1.7	2.5\\
1.7	2.6\\
1.7	2.7\\
1.7	2.8\\
1.7	2.9\\
1.7	3\\
1.7	3.1\\
1.7	3.2\\
1.7	3.3\\
1.7	3.4\\
1.7	3.5\\
1.7	3.6\\
1.7	3.7\\
1.7	3.8\\
1.7	3.9\\
1.7	4\\
1.7	4.1\\
1.7	4.2\\
1.7	4.3\\
1.7	4.4\\
1.7	4.5\\
1.7	4.6\\
1.7	4.7\\
1.7	4.8\\
};
\addplot [color=white, draw=none, mark size=0.2pt, mark=*, mark options={solid, white}, forget plot]
  table[row sep=crcr]{%
1.8	1.2\\
1.8	1.3\\
1.8	1.4\\
1.8	1.5\\
1.8	1.6\\
1.8	1.7\\
1.8	1.8\\
1.8	1.9\\
1.8	2\\
1.8	2.1\\
1.8	2.2\\
1.8	2.3\\
1.8	2.4\\
1.8	2.5\\
1.8	2.6\\
1.8	2.7\\
1.8	2.8\\
1.8	2.9\\
1.8	3\\
1.8	3.1\\
1.8	3.2\\
1.8	3.3\\
1.8	3.4\\
1.8	3.5\\
1.8	3.6\\
1.8	3.7\\
1.8	3.8\\
1.8	3.9\\
1.8	4\\
1.8	4.1\\
1.8	4.2\\
1.8	4.3\\
1.8	4.4\\
1.8	4.5\\
1.8	4.6\\
1.8	4.7\\
1.8	4.8\\
};
\addplot [color=white, draw=none, mark size=0.2pt, mark=*, mark options={solid, white}, forget plot]
  table[row sep=crcr]{%
1.9	1.2\\
1.9	1.3\\
1.9	1.4\\
1.9	1.5\\
1.9	1.6\\
1.9	1.7\\
1.9	1.8\\
1.9	1.9\\
1.9	2\\
1.9	2.1\\
1.9	2.2\\
1.9	2.3\\
1.9	2.4\\
1.9	2.5\\
1.9	2.6\\
1.9	2.7\\
1.9	2.8\\
1.9	2.9\\
1.9	3\\
1.9	3.1\\
1.9	3.2\\
1.9	3.3\\
1.9	3.4\\
1.9	3.5\\
1.9	3.6\\
1.9	3.7\\
1.9	3.8\\
1.9	3.9\\
1.9	4\\
1.9	4.1\\
1.9	4.2\\
1.9	4.3\\
1.9	4.4\\
1.9	4.5\\
1.9	4.6\\
1.9	4.7\\
1.9	4.8\\
};
\addplot [color=white, draw=none, mark size=0.2pt, mark=*, mark options={solid, white}, forget plot]
  table[row sep=crcr]{%
2	1.2\\
2	1.3\\
2	1.4\\
2	1.5\\
2	1.6\\
2	1.7\\
2	1.8\\
2	1.9\\
2	2\\
2	2.1\\
2	2.2\\
2	2.3\\
2	2.4\\
2	2.5\\
2	2.6\\
2	2.7\\
2	2.8\\
2	2.9\\
2	3\\
2	3.1\\
2	3.2\\
2	3.3\\
2	3.4\\
2	3.5\\
2	3.6\\
2	3.7\\
2	3.8\\
2	3.9\\
2	4\\
2	4.1\\
2	4.2\\
2	4.3\\
2	4.4\\
2	4.5\\
2	4.6\\
2	4.7\\
2	4.8\\
};
\addplot [color=white, draw=none, mark size=0.2pt, mark=*, mark options={solid, white}, forget plot]
  table[row sep=crcr]{%
2.1	1.2\\
2.1	1.3\\
2.1	1.4\\
2.1	1.5\\
2.1	1.6\\
2.1	1.7\\
2.1	1.8\\
2.1	1.9\\
2.1	2\\
2.1	2.1\\
2.1	2.2\\
2.1	2.3\\
2.1	2.4\\
2.1	2.5\\
2.1	2.6\\
2.1	2.7\\
2.1	2.8\\
2.1	2.9\\
2.1	3\\
2.1	3.1\\
2.1	3.2\\
2.1	3.3\\
2.1	3.4\\
2.1	3.5\\
2.1	3.6\\
2.1	3.7\\
2.1	3.8\\
2.1	3.9\\
2.1	4\\
2.1	4.1\\
2.1	4.2\\
2.1	4.3\\
2.1	4.4\\
2.1	4.5\\
2.1	4.6\\
2.1	4.7\\
2.1	4.8\\
};
\addplot [color=white, draw=none, mark size=0.2pt, mark=*, mark options={solid, white}, forget plot]
  table[row sep=crcr]{%
2.2	1.2\\
2.2	1.3\\
2.2	1.4\\
2.2	1.5\\
2.2	1.6\\
2.2	1.7\\
2.2	1.8\\
2.2	1.9\\
2.2	2\\
2.2	2.1\\
2.2	2.2\\
2.2	2.3\\
2.2	2.4\\
2.2	2.5\\
2.2	2.6\\
2.2	2.7\\
2.2	2.8\\
2.2	2.9\\
2.2	3\\
2.2	3.1\\
2.2	3.2\\
2.2	3.3\\
2.2	3.4\\
2.2	3.5\\
2.2	3.6\\
2.2	3.7\\
2.2	3.8\\
2.2	3.9\\
2.2	4\\
2.2	4.1\\
2.2	4.2\\
2.2	4.3\\
2.2	4.4\\
2.2	4.5\\
2.2	4.6\\
2.2	4.7\\
2.2	4.8\\
};
\addplot [color=white, draw=none, mark size=0.2pt, mark=*, mark options={solid, white}, forget plot]
  table[row sep=crcr]{%
2.3	1.2\\
2.3	1.3\\
2.3	1.4\\
2.3	1.5\\
2.3	1.6\\
2.3	1.7\\
2.3	1.8\\
2.3	1.9\\
2.3	2\\
2.3	2.1\\
2.3	2.2\\
2.3	2.3\\
2.3	2.4\\
2.3	2.5\\
2.3	2.6\\
2.3	2.7\\
2.3	2.8\\
2.3	2.9\\
2.3	3\\
2.3	3.1\\
2.3	3.2\\
2.3	3.3\\
2.3	3.4\\
2.3	3.5\\
2.3	3.6\\
2.3	3.7\\
2.3	3.8\\
2.3	3.9\\
2.3	4\\
2.3	4.1\\
2.3	4.2\\
2.3	4.3\\
2.3	4.4\\
2.3	4.5\\
2.3	4.6\\
2.3	4.7\\
2.3	4.8\\
};
\addplot [color=white, draw=none, mark size=0.2pt, mark=*, mark options={solid, white}, forget plot]
  table[row sep=crcr]{%
2.4	1.2\\
2.4	1.3\\
2.4	1.4\\
2.4	1.5\\
2.4	1.6\\
2.4	1.7\\
2.4	1.8\\
2.4	1.9\\
2.4	2\\
2.4	2.1\\
2.4	2.2\\
2.4	2.3\\
2.4	2.4\\
2.4	2.5\\
2.4	2.6\\
2.4	2.7\\
2.4	2.8\\
2.4	2.9\\
2.4	3\\
2.4	3.1\\
2.4	3.2\\
2.4	3.3\\
2.4	3.4\\
2.4	3.5\\
2.4	3.6\\
2.4	3.7\\
2.4	3.8\\
2.4	3.9\\
2.4	4\\
2.4	4.1\\
2.4	4.2\\
2.4	4.3\\
2.4	4.4\\
2.4	4.5\\
2.4	4.6\\
2.4	4.7\\
2.4	4.8\\
};
\addplot [color=white, draw=none, mark size=0.2pt, mark=*, mark options={solid, white}, forget plot]
  table[row sep=crcr]{%
2.5	1.2\\
2.5	1.3\\
2.5	1.4\\
2.5	1.5\\
2.5	1.6\\
2.5	1.7\\
2.5	1.8\\
2.5	1.9\\
2.5	2\\
2.5	2.1\\
2.5	2.2\\
2.5	2.3\\
2.5	2.4\\
2.5	2.5\\
2.5	2.6\\
2.5	2.7\\
2.5	2.8\\
2.5	2.9\\
2.5	3\\
2.5	3.1\\
2.5	3.2\\
2.5	3.3\\
2.5	3.4\\
2.5	3.5\\
2.5	3.6\\
2.5	3.7\\
2.5	3.8\\
2.5	3.9\\
2.5	4\\
2.5	4.1\\
2.5	4.2\\
2.5	4.3\\
2.5	4.4\\
2.5	4.5\\
2.5	4.6\\
2.5	4.7\\
2.5	4.8\\
};
\addplot [color=white, draw=none, mark size=0.2pt, mark=*, mark options={solid, white}, forget plot]
  table[row sep=crcr]{%
2.6	1.2\\
2.6	1.3\\
2.6	1.4\\
2.6	1.5\\
2.6	1.6\\
2.6	1.7\\
2.6	1.8\\
2.6	1.9\\
2.6	2\\
2.6	2.1\\
2.6	2.2\\
2.6	2.3\\
2.6	2.4\\
2.6	2.5\\
2.6	2.6\\
2.6	2.7\\
2.6	2.8\\
2.6	2.9\\
2.6	3\\
2.6	3.1\\
2.6	3.2\\
2.6	3.3\\
2.6	3.4\\
2.6	3.5\\
2.6	3.6\\
2.6	3.7\\
2.6	3.8\\
2.6	3.9\\
2.6	4\\
2.6	4.1\\
2.6	4.2\\
2.6	4.3\\
2.6	4.4\\
2.6	4.5\\
2.6	4.6\\
2.6	4.7\\
2.6	4.8\\
};
\addplot [color=white, draw=none, mark size=0.2pt, mark=*, mark options={solid, white}, forget plot]
  table[row sep=crcr]{%
2.7	1.2\\
2.7	1.3\\
2.7	1.4\\
2.7	1.5\\
2.7	1.6\\
2.7	1.7\\
2.7	1.8\\
2.7	1.9\\
2.7	2\\
2.7	2.1\\
2.7	2.2\\
2.7	2.3\\
2.7	2.4\\
2.7	2.5\\
2.7	2.6\\
2.7	2.7\\
2.7	2.8\\
2.7	2.9\\
2.7	3\\
2.7	3.1\\
2.7	3.2\\
2.7	3.3\\
2.7	3.4\\
2.7	3.5\\
2.7	3.6\\
2.7	3.7\\
2.7	3.8\\
2.7	3.9\\
2.7	4\\
2.7	4.1\\
2.7	4.2\\
2.7	4.3\\
2.7	4.4\\
2.7	4.5\\
2.7	4.6\\
2.7	4.7\\
2.7	4.8\\
};
\addplot [color=white, draw=none, mark size=0.2pt, mark=*, mark options={solid, white}, forget plot]
  table[row sep=crcr]{%
2.8	1.2\\
2.8	1.3\\
2.8	1.4\\
2.8	1.5\\
2.8	1.6\\
2.8	1.7\\
2.8	1.8\\
2.8	1.9\\
2.8	2\\
2.8	2.1\\
2.8	2.2\\
2.8	2.3\\
2.8	2.4\\
2.8	2.5\\
2.8	2.6\\
2.8	2.7\\
2.8	2.8\\
2.8	2.9\\
2.8	3\\
2.8	3.1\\
2.8	3.2\\
2.8	3.3\\
2.8	3.4\\
2.8	3.5\\
2.8	3.6\\
2.8	3.7\\
2.8	3.8\\
2.8	3.9\\
2.8	4\\
2.8	4.1\\
2.8	4.2\\
2.8	4.3\\
2.8	4.4\\
2.8	4.5\\
2.8	4.6\\
2.8	4.7\\
2.8	4.8\\
};
\addplot [color=white, draw=none, mark size=0.2pt, mark=*, mark options={solid, white}, forget plot]
  table[row sep=crcr]{%
2.9	1.2\\
2.9	1.3\\
2.9	1.4\\
2.9	1.5\\
2.9	1.6\\
2.9	1.7\\
2.9	1.8\\
2.9	1.9\\
2.9	2\\
2.9	2.1\\
2.9	2.2\\
2.9	2.3\\
2.9	2.4\\
2.9	2.5\\
2.9	2.6\\
2.9	2.7\\
2.9	2.8\\
2.9	2.9\\
2.9	3\\
2.9	3.1\\
2.9	3.2\\
2.9	3.3\\
2.9	3.4\\
2.9	3.5\\
2.9	3.6\\
2.9	3.7\\
2.9	3.8\\
2.9	3.9\\
2.9	4\\
2.9	4.1\\
2.9	4.2\\
2.9	4.3\\
2.9	4.4\\
2.9	4.5\\
2.9	4.6\\
2.9	4.7\\
2.9	4.8\\
};
\addplot [color=white, draw=none, mark size=0.2pt, mark=*, mark options={solid, white}, forget plot]
  table[row sep=crcr]{%
3	1.2\\
3	1.3\\
3	1.4\\
3	1.5\\
3	1.6\\
3	1.7\\
3	1.8\\
3	1.9\\
3	2\\
3	2.1\\
3	2.2\\
3	2.3\\
3	2.4\\
3	2.5\\
3	2.6\\
3	2.7\\
3	2.8\\
3	2.9\\
3	3\\
3	3.1\\
3	3.2\\
3	3.3\\
3	3.4\\
3	3.5\\
3	3.6\\
3	3.7\\
3	3.8\\
3	3.9\\
3	4\\
3	4.1\\
3	4.2\\
3	4.3\\
3	4.4\\
3	4.5\\
3	4.6\\
3	4.7\\
3	4.8\\
};
\addplot [color=white, draw=none, mark size=0.2pt, mark=*, mark options={solid, white}, forget plot]
  table[row sep=crcr]{%
3.1	1.2\\
3.1	1.3\\
3.1	1.4\\
3.1	1.5\\
3.1	1.6\\
3.1	1.7\\
3.1	1.8\\
3.1	1.9\\
3.1	2\\
3.1	2.1\\
3.1	2.2\\
3.1	2.3\\
3.1	2.4\\
3.1	2.5\\
3.1	2.6\\
3.1	2.7\\
3.1	2.8\\
3.1	2.9\\
3.1	3\\
3.1	3.1\\
3.1	3.2\\
3.1	3.3\\
3.1	3.4\\
3.1	3.5\\
3.1	3.6\\
3.1	3.7\\
3.1	3.8\\
3.1	3.9\\
3.1	4\\
3.1	4.1\\
3.1	4.2\\
3.1	4.3\\
3.1	4.4\\
3.1	4.5\\
3.1	4.6\\
3.1	4.7\\
3.1	4.8\\
};
\addplot [color=white, draw=none, mark size=0.2pt, mark=*, mark options={solid, white}, forget plot]
  table[row sep=crcr]{%
3.2	1.2\\
3.2	1.3\\
3.2	1.4\\
3.2	1.5\\
3.2	1.6\\
3.2	1.7\\
3.2	1.8\\
3.2	1.9\\
3.2	2\\
3.2	2.1\\
3.2	2.2\\
3.2	2.3\\
3.2	2.4\\
3.2	2.5\\
3.2	2.6\\
3.2	2.7\\
3.2	2.8\\
3.2	2.9\\
3.2	3\\
3.2	3.1\\
3.2	3.2\\
3.2	3.3\\
3.2	3.4\\
3.2	3.5\\
3.2	3.6\\
3.2	3.7\\
3.2	3.8\\
3.2	3.9\\
3.2	4\\
3.2	4.1\\
3.2	4.2\\
3.2	4.3\\
3.2	4.4\\
3.2	4.5\\
3.2	4.6\\
3.2	4.7\\
3.2	4.8\\
};
\addplot [color=white, draw=none, mark size=0.2pt, mark=*, mark options={solid, white}, forget plot]
  table[row sep=crcr]{%
3.3	1.2\\
3.3	1.3\\
3.3	1.4\\
3.3	1.5\\
3.3	1.6\\
3.3	1.7\\
3.3	1.8\\
3.3	1.9\\
3.3	2\\
3.3	2.1\\
3.3	2.2\\
3.3	2.3\\
3.3	2.4\\
3.3	2.5\\
3.3	2.6\\
3.3	2.7\\
3.3	2.8\\
3.3	2.9\\
3.3	3\\
3.3	3.1\\
3.3	3.2\\
3.3	3.3\\
3.3	3.4\\
3.3	3.5\\
3.3	3.6\\
3.3	3.7\\
3.3	3.8\\
3.3	3.9\\
3.3	4\\
3.3	4.1\\
3.3	4.2\\
3.3	4.3\\
3.3	4.4\\
3.3	4.5\\
3.3	4.6\\
3.3	4.7\\
3.3	4.8\\
};
\addplot [color=white, draw=none, mark size=0.2pt, mark=*, mark options={solid, white}, forget plot]
  table[row sep=crcr]{%
3.4	1.2\\
3.4	1.3\\
3.4	1.4\\
3.4	1.5\\
3.4	1.6\\
3.4	1.7\\
3.4	1.8\\
3.4	1.9\\
3.4	2\\
3.4	2.1\\
3.4	2.2\\
3.4	2.3\\
3.4	2.4\\
3.4	2.5\\
3.4	2.6\\
3.4	2.7\\
3.4	2.8\\
3.4	2.9\\
3.4	3\\
3.4	3.1\\
3.4	3.2\\
3.4	3.3\\
3.4	3.4\\
3.4	3.5\\
3.4	3.6\\
3.4	3.7\\
3.4	3.8\\
3.4	3.9\\
3.4	4\\
3.4	4.1\\
3.4	4.2\\
3.4	4.3\\
3.4	4.4\\
3.4	4.5\\
3.4	4.6\\
3.4	4.7\\
3.4	4.8\\
};
\addplot [color=white, draw=none, mark size=0.2pt, mark=*, mark options={solid, white}, forget plot]
  table[row sep=crcr]{%
3.5	1.2\\
3.5	1.3\\
3.5	1.4\\
3.5	1.5\\
3.5	1.6\\
3.5	1.7\\
3.5	1.8\\
3.5	1.9\\
3.5	2\\
3.5	2.1\\
3.5	2.2\\
3.5	2.3\\
3.5	2.4\\
3.5	2.5\\
3.5	2.6\\
3.5	2.7\\
3.5	2.8\\
3.5	2.9\\
3.5	3\\
3.5	3.1\\
3.5	3.2\\
3.5	3.3\\
3.5	3.4\\
3.5	3.5\\
3.5	3.6\\
3.5	3.7\\
3.5	3.8\\
3.5	3.9\\
3.5	4\\
3.5	4.1\\
3.5	4.2\\
3.5	4.3\\
3.5	4.4\\
3.5	4.5\\
3.5	4.6\\
3.5	4.7\\
3.5	4.8\\
};
\addplot [color=white, draw=none, mark size=0.2pt, mark=*, mark options={solid, white}, forget plot]
  table[row sep=crcr]{%
3.6	1.2\\
3.6	1.3\\
3.6	1.4\\
3.6	1.5\\
3.6	1.6\\
3.6	1.7\\
3.6	1.8\\
3.6	1.9\\
3.6	2\\
3.6	2.1\\
3.6	2.2\\
3.6	2.3\\
3.6	2.4\\
3.6	2.5\\
3.6	2.6\\
3.6	2.7\\
3.6	2.8\\
3.6	2.9\\
3.6	3\\
3.6	3.1\\
3.6	3.2\\
3.6	3.3\\
3.6	3.4\\
3.6	3.5\\
3.6	3.6\\
3.6	3.7\\
3.6	3.8\\
3.6	3.9\\
3.6	4\\
3.6	4.1\\
3.6	4.2\\
3.6	4.3\\
3.6	4.4\\
3.6	4.5\\
3.6	4.6\\
3.6	4.7\\
3.6	4.8\\
};
\addplot [color=white, draw=none, mark size=0.2pt, mark=*, mark options={solid, white}, forget plot]
  table[row sep=crcr]{%
3.7	1.2\\
3.7	1.3\\
3.7	1.4\\
3.7	1.5\\
3.7	1.6\\
3.7	1.7\\
3.7	1.8\\
3.7	1.9\\
3.7	2\\
3.7	2.1\\
3.7	2.2\\
3.7	2.3\\
3.7	2.4\\
3.7	2.5\\
3.7	2.6\\
3.7	2.7\\
3.7	2.8\\
3.7	2.9\\
3.7	3\\
3.7	3.1\\
3.7	3.2\\
3.7	3.3\\
3.7	3.4\\
3.7	3.5\\
3.7	3.6\\
3.7	3.7\\
3.7	3.8\\
3.7	3.9\\
3.7	4\\
3.7	4.1\\
3.7	4.2\\
3.7	4.3\\
3.7	4.4\\
3.7	4.5\\
3.7	4.6\\
3.7	4.7\\
3.7	4.8\\
};
\addplot [color=white, draw=none, mark size=0.2pt, mark=*, mark options={solid, white}, forget plot]
  table[row sep=crcr]{%
3.8	1.2\\
3.8	1.3\\
3.8	1.4\\
3.8	1.5\\
3.8	1.6\\
3.8	1.7\\
3.8	1.8\\
3.8	1.9\\
3.8	2\\
3.8	2.1\\
3.8	2.2\\
3.8	2.3\\
3.8	2.4\\
3.8	2.5\\
3.8	2.6\\
3.8	2.7\\
3.8	2.8\\
3.8	2.9\\
3.8	3\\
3.8	3.1\\
3.8	3.2\\
3.8	3.3\\
3.8	3.4\\
3.8	3.5\\
3.8	3.6\\
3.8	3.7\\
3.8	3.8\\
3.8	3.9\\
3.8	4\\
3.8	4.1\\
3.8	4.2\\
3.8	4.3\\
3.8	4.4\\
3.8	4.5\\
3.8	4.6\\
3.8	4.7\\
3.8	4.8\\
};
\addplot [color=white, draw=none, mark size=0.2pt, mark=*, mark options={solid, white}, forget plot]
  table[row sep=crcr]{%
3.9	1.2\\
3.9	1.3\\
3.9	1.4\\
3.9	1.5\\
3.9	1.6\\
3.9	1.7\\
3.9	1.8\\
3.9	1.9\\
3.9	2\\
3.9	2.1\\
3.9	2.2\\
3.9	2.3\\
3.9	2.4\\
3.9	2.5\\
3.9	2.6\\
3.9	2.7\\
3.9	2.8\\
3.9	2.9\\
3.9	3\\
3.9	3.1\\
3.9	3.2\\
3.9	3.3\\
3.9	3.4\\
3.9	3.5\\
3.9	3.6\\
3.9	3.7\\
3.9	3.8\\
3.9	3.9\\
3.9	4\\
3.9	4.1\\
3.9	4.2\\
3.9	4.3\\
3.9	4.4\\
3.9	4.5\\
3.9	4.6\\
3.9	4.7\\
3.9	4.8\\
};
\addplot [color=white, draw=none, mark size=0.2pt, mark=*, mark options={solid, white}, forget plot]
  table[row sep=crcr]{%
4	1.2\\
4	1.3\\
4	1.4\\
4	1.5\\
4	1.6\\
4	1.7\\
4	1.8\\
4	1.9\\
4	2\\
4	2.1\\
4	2.2\\
4	2.3\\
4	2.4\\
4	2.5\\
4	2.6\\
4	2.7\\
4	2.8\\
4	2.9\\
4	3\\
4	3.1\\
4	3.2\\
4	3.3\\
4	3.4\\
4	3.5\\
4	3.6\\
4	3.7\\
4	3.8\\
4	3.9\\
4	4\\
4	4.1\\
4	4.2\\
4	4.3\\
4	4.4\\
4	4.5\\
4	4.6\\
4	4.7\\
4	4.8\\
};
\addplot [color=white, draw=none, mark size=0.2pt, mark=*, mark options={solid, white}, forget plot]
  table[row sep=crcr]{%
4.1	1.2\\
4.1	1.3\\
4.1	1.4\\
4.1	1.5\\
4.1	1.6\\
4.1	1.7\\
4.1	1.8\\
4.1	1.9\\
4.1	2\\
4.1	2.1\\
4.1	2.2\\
4.1	2.3\\
4.1	2.4\\
4.1	2.5\\
4.1	2.6\\
4.1	2.7\\
4.1	2.8\\
4.1	2.9\\
4.1	3\\
4.1	3.1\\
4.1	3.2\\
4.1	3.3\\
4.1	3.4\\
4.1	3.5\\
4.1	3.6\\
4.1	3.7\\
4.1	3.8\\
4.1	3.9\\
4.1	4\\
4.1	4.1\\
4.1	4.2\\
4.1	4.3\\
4.1	4.4\\
4.1	4.5\\
4.1	4.6\\
4.1	4.7\\
4.1	4.8\\
};
\addplot [color=white, draw=none, mark size=0.2pt, mark=*, mark options={solid, white}, forget plot]
  table[row sep=crcr]{%
4.2	1.2\\
4.2	1.3\\
4.2	1.4\\
4.2	1.5\\
4.2	1.6\\
4.2	1.7\\
4.2	1.8\\
4.2	1.9\\
4.2	2\\
4.2	2.1\\
4.2	2.2\\
4.2	2.3\\
4.2	2.4\\
4.2	2.5\\
4.2	2.6\\
4.2	2.7\\
4.2	2.8\\
4.2	2.9\\
4.2	3\\
4.2	3.1\\
4.2	3.2\\
4.2	3.3\\
4.2	3.4\\
4.2	3.5\\
4.2	3.6\\
4.2	3.7\\
4.2	3.8\\
4.2	3.9\\
4.2	4\\
4.2	4.1\\
4.2	4.2\\
4.2	4.3\\
4.2	4.4\\
4.2	4.5\\
4.2	4.6\\
4.2	4.7\\
4.2	4.8\\
};
\addplot [color=white, draw=none, mark size=0.2pt, mark=*, mark options={solid, white}, forget plot]
  table[row sep=crcr]{%
4.3	1.2\\
4.3	1.3\\
4.3	1.4\\
4.3	1.5\\
4.3	1.6\\
4.3	1.7\\
4.3	1.8\\
4.3	1.9\\
4.3	2\\
4.3	2.1\\
4.3	2.2\\
4.3	2.3\\
4.3	2.4\\
4.3	2.5\\
4.3	2.6\\
4.3	2.7\\
4.3	2.8\\
4.3	2.9\\
4.3	3\\
4.3	3.1\\
4.3	3.2\\
4.3	3.3\\
4.3	3.4\\
4.3	3.5\\
4.3	3.6\\
4.3	3.7\\
4.3	3.8\\
4.3	3.9\\
4.3	4\\
4.3	4.1\\
4.3	4.2\\
4.3	4.3\\
4.3	4.4\\
4.3	4.5\\
4.3	4.6\\
4.3	4.7\\
4.3	4.8\\
};
\addplot [color=white, draw=none, mark size=0.2pt, mark=*, mark options={solid, white}, forget plot]
  table[row sep=crcr]{%
4.4	1.2\\
4.4	1.3\\
4.4	1.4\\
4.4	1.5\\
4.4	1.6\\
4.4	1.7\\
4.4	1.8\\
4.4	1.9\\
4.4	2\\
4.4	2.1\\
4.4	2.2\\
4.4	2.3\\
4.4	2.4\\
4.4	2.5\\
4.4	2.6\\
4.4	2.7\\
4.4	2.8\\
4.4	2.9\\
4.4	3\\
4.4	3.1\\
4.4	3.2\\
4.4	3.3\\
4.4	3.4\\
4.4	3.5\\
4.4	3.6\\
4.4	3.7\\
4.4	3.8\\
4.4	3.9\\
4.4	4\\
4.4	4.1\\
4.4	4.2\\
4.4	4.3\\
4.4	4.4\\
4.4	4.5\\
4.4	4.6\\
4.4	4.7\\
4.4	4.8\\
};
\addplot [color=white, draw=none, mark size=0.2pt, mark=*, mark options={solid, white}, forget plot]
  table[row sep=crcr]{%
4.5	1.2\\
4.5	1.3\\
4.5	1.4\\
4.5	1.5\\
4.5	1.6\\
4.5	1.7\\
4.5	1.8\\
4.5	1.9\\
4.5	2\\
4.5	2.1\\
4.5	2.2\\
4.5	2.3\\
4.5	2.4\\
4.5	2.5\\
4.5	2.6\\
4.5	2.7\\
4.5	2.8\\
4.5	2.9\\
4.5	3\\
4.5	3.1\\
4.5	3.2\\
4.5	3.3\\
4.5	3.4\\
4.5	3.5\\
4.5	3.6\\
4.5	3.7\\
4.5	3.8\\
4.5	3.9\\
4.5	4\\
4.5	4.1\\
4.5	4.2\\
4.5	4.3\\
4.5	4.4\\
4.5	4.5\\
4.5	4.6\\
4.5	4.7\\
4.5	4.8\\
};
\addplot [color=white, draw=none, mark size=0.2pt, mark=*, mark options={solid, white}, forget plot]
  table[row sep=crcr]{%
4.6	1.2\\
4.6	1.3\\
4.6	1.4\\
4.6	1.5\\
4.6	1.6\\
4.6	1.7\\
4.6	1.8\\
4.6	1.9\\
4.6	2\\
4.6	2.1\\
4.6	2.2\\
4.6	2.3\\
4.6	2.4\\
4.6	2.5\\
4.6	2.6\\
4.6	2.7\\
4.6	2.8\\
4.6	2.9\\
4.6	3\\
4.6	3.1\\
4.6	3.2\\
4.6	3.3\\
4.6	3.4\\
4.6	3.5\\
4.6	3.6\\
4.6	3.7\\
4.6	3.8\\
4.6	3.9\\
4.6	4\\
4.6	4.1\\
4.6	4.2\\
4.6	4.3\\
4.6	4.4\\
4.6	4.5\\
4.6	4.6\\
4.6	4.7\\
4.6	4.8\\
};
\addplot [color=white, draw=none, mark size=0.2pt, mark=*, mark options={solid, white}, forget plot]
  table[row sep=crcr]{%
4.7	1.2\\
4.7	1.3\\
4.7	1.4\\
4.7	1.5\\
4.7	1.6\\
4.7	1.7\\
4.7	1.8\\
4.7	1.9\\
4.7	2\\
4.7	2.1\\
4.7	2.2\\
4.7	2.3\\
4.7	2.4\\
4.7	2.5\\
4.7	2.6\\
4.7	2.7\\
4.7	2.8\\
4.7	2.9\\
4.7	3\\
4.7	3.1\\
4.7	3.2\\
4.7	3.3\\
4.7	3.4\\
4.7	3.5\\
4.7	3.6\\
4.7	3.7\\
4.7	3.8\\
4.7	3.9\\
4.7	4\\
4.7	4.1\\
4.7	4.2\\
4.7	4.3\\
4.7	4.4\\
4.7	4.5\\
4.7	4.6\\
4.7	4.7\\
4.7	4.8\\
};
\addplot [color=white, draw=none, mark size=0.2pt, mark=*, mark options={solid, white}, forget plot]
  table[row sep=crcr]{%
4.8	1.2\\
4.8	1.3\\
4.8	1.4\\
4.8	1.5\\
4.8	1.6\\
4.8	1.7\\
4.8	1.8\\
4.8	1.9\\
4.8	2\\
4.8	2.1\\
4.8	2.2\\
4.8	2.3\\
4.8	2.4\\
4.8	2.5\\
4.8	2.6\\
4.8	2.7\\
4.8	2.8\\
4.8	2.9\\
4.8	3\\
4.8	3.1\\
4.8	3.2\\
4.8	3.3\\
4.8	3.4\\
4.8	3.5\\
4.8	3.6\\
4.8	3.7\\
4.8	3.8\\
4.8	3.9\\
4.8	4\\
4.8	4.1\\
4.8	4.2\\
4.8	4.3\\
4.8	4.4\\
4.8	4.5\\
4.8	4.6\\
4.8	4.7\\
4.8	4.8\\
};
\addplot [color=mycolor1, line width=1.0pt, draw=none, mark size=6.0pt, mark=x, mark options={solid, mycolor1}, forget plot]
  table[row sep=crcr]{%
2.5	3\\
3.5	3.2\\
};
\end{axis}
\end{tikzpicture}%
%		\caption{Easy Assignment}
%	\end{subfigure}
	\begin{subfigure}{0.49\textwidth}
	\centering
	     \footnotesize
        \setlength{\figurewidth}{0.8\textwidth}
        \setlength{\figureheight}{6cm}
        % This file was created by matlab2tikz.
%
\definecolor{mycolor1}{rgb}{1.00000,1.00000,0.00000}%
%
\begin{tikzpicture}

\begin{axis}[%
width=0.951\figurewidth,
height=\figureheight,
at={(0\figurewidth,0\figureheight)},
scale only axis,
xmin=-0,
xmax=6,
ymin=-0,
ymax=6,
axis background/.style={fill=white},
axis x line*=bottom,
axis y line*=left
]

\addplot[%
surf,
shader=interp, colormap={mymap}{[1pt] rgb(0pt)=(0.239216,0.14902,0.658824); rgb(1pt)=(0.239216,0.14902,0.658824)}, mesh/rows=6]
table[row sep=crcr, point meta=\thisrow{c}] {%
%
x	y	c\\
0	0	0\\
0	1.2	0\\
0	2.4	0\\
0	3.6	0\\
0	4.8	0\\
0	6	0\\
1.2	0	0\\
1.2	1.2	0\\
1.2	2.4	0\\
1.2	3.6	0\\
1.2	4.8	0\\
1.2	6	0\\
2.4	0	0\\
2.4	1.2	0\\
2.4	2.4	0\\
2.4	3.6	0\\
2.4	4.8	0\\
2.4	6	0\\
3.6	0	0\\
3.6	1.2	0\\
3.6	2.4	0\\
3.6	3.6	0\\
3.6	4.8	0\\
3.6	6	0\\
4.8	0	0\\
4.8	1.2	0\\
4.8	2.4	0\\
4.8	3.6	0\\
4.8	4.8	0\\
4.8	6	0\\
6	0	0\\
6	1.2	0\\
6	2.4	0\\
6	3.6	0\\
6	4.8	0\\
6	6	0\\
};
\addplot [color=green, line width=1.0pt, draw=none, mark size=4.0pt, mark=o, mark options={solid, green}, forget plot]
  table[row sep=crcr]{%
2.1	1\\
2.3	1\\
2.7	1\\
2.9	1\\
3.7	1\\
3.9	1\\
5	2.2\\
5	2.4\\
5	2.8\\
5	3\\
5	3.8\\
5	4\\
2.2	5\\
2.4	5\\
3	5\\
3.2	5\\
3.8	5\\
4	5\\
1	2.1\\
1	2.3\\
1	2.9\\
1	3.1\\
1	3.7\\
1	3.9\\
};
\addplot [color=red, line width=1.0pt, draw=none, mark size=6.0pt, mark=x, mark options={solid, red}, forget plot]
  table[row sep=crcr]{%
3.3	3\\
2.7	3\\
};
\addplot [color=black, draw=none, mark size=0.2pt, mark=*, mark options={solid, black}, forget plot]
  table[row sep=crcr]{%
0.1	0.1\\
0.1	0.2\\
0.1	0.3\\
0.1	0.4\\
0.1	0.5\\
0.1	0.6\\
0.1	0.7\\
0.1	0.8\\
0.1	0.9\\
0.1	1\\
0.1	1.1\\
0.1	1.2\\
0.1	1.3\\
0.1	1.4\\
0.1	1.5\\
0.1	1.6\\
0.1	1.7\\
0.1	1.8\\
0.1	1.9\\
0.1	2\\
0.1	2.1\\
0.1	2.2\\
0.1	2.3\\
0.1	2.4\\
0.1	2.5\\
0.1	2.6\\
0.1	2.7\\
0.1	2.8\\
0.1	2.9\\
0.1	3\\
0.1	3.1\\
0.1	3.2\\
0.1	3.3\\
0.1	3.4\\
0.1	3.5\\
0.1	3.6\\
0.1	3.7\\
0.1	3.8\\
0.1	3.9\\
0.1	4\\
0.1	4.1\\
0.1	4.2\\
0.1	4.3\\
0.1	4.4\\
0.1	4.5\\
0.1	4.6\\
0.1	4.7\\
0.1	4.8\\
0.1	4.9\\
0.1	5\\
0.1	5.1\\
0.1	5.2\\
0.1	5.3\\
0.1	5.4\\
0.1	5.5\\
0.1	5.6\\
0.1	5.7\\
0.1	5.8\\
0.1	5.9\\
};
\addplot [color=black, draw=none, mark size=0.2pt, mark=*, mark options={solid, black}, forget plot]
  table[row sep=crcr]{%
0.2	0.1\\
0.2	0.2\\
0.2	0.3\\
0.2	0.4\\
0.2	0.5\\
0.2	0.6\\
0.2	0.7\\
0.2	0.8\\
0.2	0.9\\
0.2	1\\
0.2	1.1\\
0.2	1.2\\
0.2	1.3\\
0.2	1.4\\
0.2	1.5\\
0.2	1.6\\
0.2	1.7\\
0.2	1.8\\
0.2	1.9\\
0.2	2\\
0.2	2.1\\
0.2	2.2\\
0.2	2.3\\
0.2	2.4\\
0.2	2.5\\
0.2	2.6\\
0.2	2.7\\
0.2	2.8\\
0.2	2.9\\
0.2	3\\
0.2	3.1\\
0.2	3.2\\
0.2	3.3\\
0.2	3.4\\
0.2	3.5\\
0.2	3.6\\
0.2	3.7\\
0.2	3.8\\
0.2	3.9\\
0.2	4\\
0.2	4.1\\
0.2	4.2\\
0.2	4.3\\
0.2	4.4\\
0.2	4.5\\
0.2	4.6\\
0.2	4.7\\
0.2	4.8\\
0.2	4.9\\
0.2	5\\
0.2	5.1\\
0.2	5.2\\
0.2	5.3\\
0.2	5.4\\
0.2	5.5\\
0.2	5.6\\
0.2	5.7\\
0.2	5.8\\
0.2	5.9\\
};
\addplot [color=black, draw=none, mark size=0.2pt, mark=*, mark options={solid, black}, forget plot]
  table[row sep=crcr]{%
0.3	0.1\\
0.3	0.2\\
0.3	0.3\\
0.3	0.4\\
0.3	0.5\\
0.3	0.6\\
0.3	0.7\\
0.3	0.8\\
0.3	0.9\\
0.3	1\\
0.3	1.1\\
0.3	1.2\\
0.3	1.3\\
0.3	1.4\\
0.3	1.5\\
0.3	1.6\\
0.3	1.7\\
0.3	1.8\\
0.3	1.9\\
0.3	2\\
0.3	2.1\\
0.3	2.2\\
0.3	2.3\\
0.3	2.4\\
0.3	2.5\\
0.3	2.6\\
0.3	2.7\\
0.3	2.8\\
0.3	2.9\\
0.3	3\\
0.3	3.1\\
0.3	3.2\\
0.3	3.3\\
0.3	3.4\\
0.3	3.5\\
0.3	3.6\\
0.3	3.7\\
0.3	3.8\\
0.3	3.9\\
0.3	4\\
0.3	4.1\\
0.3	4.2\\
0.3	4.3\\
0.3	4.4\\
0.3	4.5\\
0.3	4.6\\
0.3	4.7\\
0.3	4.8\\
0.3	4.9\\
0.3	5\\
0.3	5.1\\
0.3	5.2\\
0.3	5.3\\
0.3	5.4\\
0.3	5.5\\
0.3	5.6\\
0.3	5.7\\
0.3	5.8\\
0.3	5.9\\
};
\addplot [color=black, draw=none, mark size=0.2pt, mark=*, mark options={solid, black}, forget plot]
  table[row sep=crcr]{%
0.4	0.1\\
0.4	0.2\\
0.4	0.3\\
0.4	0.4\\
0.4	0.5\\
0.4	0.6\\
0.4	0.7\\
0.4	0.8\\
0.4	0.9\\
0.4	1\\
0.4	1.1\\
0.4	1.2\\
0.4	1.3\\
0.4	1.4\\
0.4	1.5\\
0.4	1.6\\
0.4	1.7\\
0.4	1.8\\
0.4	1.9\\
0.4	2\\
0.4	2.1\\
0.4	2.2\\
0.4	2.3\\
0.4	2.4\\
0.4	2.5\\
0.4	2.6\\
0.4	2.7\\
0.4	2.8\\
0.4	2.9\\
0.4	3\\
0.4	3.1\\
0.4	3.2\\
0.4	3.3\\
0.4	3.4\\
0.4	3.5\\
0.4	3.6\\
0.4	3.7\\
0.4	3.8\\
0.4	3.9\\
0.4	4\\
0.4	4.1\\
0.4	4.2\\
0.4	4.3\\
0.4	4.4\\
0.4	4.5\\
0.4	4.6\\
0.4	4.7\\
0.4	4.8\\
0.4	4.9\\
0.4	5\\
0.4	5.1\\
0.4	5.2\\
0.4	5.3\\
0.4	5.4\\
0.4	5.5\\
0.4	5.6\\
0.4	5.7\\
0.4	5.8\\
0.4	5.9\\
};
\addplot [color=black, draw=none, mark size=0.2pt, mark=*, mark options={solid, black}, forget plot]
  table[row sep=crcr]{%
0.5	0.1\\
0.5	0.2\\
0.5	0.3\\
0.5	0.4\\
0.5	0.5\\
0.5	0.6\\
0.5	0.7\\
0.5	0.8\\
0.5	0.9\\
0.5	1\\
0.5	1.1\\
0.5	1.2\\
0.5	1.3\\
0.5	1.4\\
0.5	1.5\\
0.5	1.6\\
0.5	1.7\\
0.5	1.8\\
0.5	1.9\\
0.5	2\\
0.5	2.1\\
0.5	2.2\\
0.5	2.3\\
0.5	2.4\\
0.5	2.5\\
0.5	2.6\\
0.5	2.7\\
0.5	2.8\\
0.5	2.9\\
0.5	3\\
0.5	3.1\\
0.5	3.2\\
0.5	3.3\\
0.5	3.4\\
0.5	3.5\\
0.5	3.6\\
0.5	3.7\\
0.5	3.8\\
0.5	3.9\\
0.5	4\\
0.5	4.1\\
0.5	4.2\\
0.5	4.3\\
0.5	4.4\\
0.5	4.5\\
0.5	4.6\\
0.5	4.7\\
0.5	4.8\\
0.5	4.9\\
0.5	5\\
0.5	5.1\\
0.5	5.2\\
0.5	5.3\\
0.5	5.4\\
0.5	5.5\\
0.5	5.6\\
0.5	5.7\\
0.5	5.8\\
0.5	5.9\\
};
\addplot [color=black, draw=none, mark size=0.2pt, mark=*, mark options={solid, black}, forget plot]
  table[row sep=crcr]{%
0.6	0.1\\
0.6	0.2\\
0.6	0.3\\
0.6	0.4\\
0.6	0.5\\
0.6	0.6\\
0.6	0.7\\
0.6	0.8\\
0.6	0.9\\
0.6	1\\
0.6	1.1\\
0.6	1.2\\
0.6	1.3\\
0.6	1.4\\
0.6	1.5\\
0.6	1.6\\
0.6	1.7\\
0.6	1.8\\
0.6	1.9\\
0.6	2\\
0.6	2.1\\
0.6	2.2\\
0.6	2.3\\
0.6	2.4\\
0.6	2.5\\
0.6	2.6\\
0.6	2.7\\
0.6	2.8\\
0.6	2.9\\
0.6	3\\
0.6	3.1\\
0.6	3.2\\
0.6	3.3\\
0.6	3.4\\
0.6	3.5\\
0.6	3.6\\
0.6	3.7\\
0.6	3.8\\
0.6	3.9\\
0.6	4\\
0.6	4.1\\
0.6	4.2\\
0.6	4.3\\
0.6	4.4\\
0.6	4.5\\
0.6	4.6\\
0.6	4.7\\
0.6	4.8\\
0.6	4.9\\
0.6	5\\
0.6	5.1\\
0.6	5.2\\
0.6	5.3\\
0.6	5.4\\
0.6	5.5\\
0.6	5.6\\
0.6	5.7\\
0.6	5.8\\
0.6	5.9\\
};
\addplot [color=black, draw=none, mark size=0.2pt, mark=*, mark options={solid, black}, forget plot]
  table[row sep=crcr]{%
0.7	0.1\\
0.7	0.2\\
0.7	0.3\\
0.7	0.4\\
0.7	0.5\\
0.7	0.6\\
0.7	0.7\\
0.7	0.8\\
0.7	0.9\\
0.7	1\\
0.7	1.1\\
0.7	1.2\\
0.7	1.3\\
0.7	1.4\\
0.7	1.5\\
0.7	1.6\\
0.7	1.7\\
0.7	1.8\\
0.7	1.9\\
0.7	2\\
0.7	2.1\\
0.7	2.2\\
0.7	2.3\\
0.7	2.4\\
0.7	2.5\\
0.7	2.6\\
0.7	2.7\\
0.7	2.8\\
0.7	2.9\\
0.7	3\\
0.7	3.1\\
0.7	3.2\\
0.7	3.3\\
0.7	3.4\\
0.7	3.5\\
0.7	3.6\\
0.7	3.7\\
0.7	3.8\\
0.7	3.9\\
0.7	4\\
0.7	4.1\\
0.7	4.2\\
0.7	4.3\\
0.7	4.4\\
0.7	4.5\\
0.7	4.6\\
0.7	4.7\\
0.7	4.8\\
0.7	4.9\\
0.7	5\\
0.7	5.1\\
0.7	5.2\\
0.7	5.3\\
0.7	5.4\\
0.7	5.5\\
0.7	5.6\\
0.7	5.7\\
0.7	5.8\\
0.7	5.9\\
};
\addplot [color=black, draw=none, mark size=0.2pt, mark=*, mark options={solid, black}, forget plot]
  table[row sep=crcr]{%
0.8	0.1\\
0.8	0.2\\
0.8	0.3\\
0.8	0.4\\
0.8	0.5\\
0.8	0.6\\
0.8	0.7\\
0.8	0.8\\
0.8	0.9\\
0.8	1\\
0.8	1.1\\
0.8	1.2\\
0.8	1.3\\
0.8	1.4\\
0.8	1.5\\
0.8	1.6\\
0.8	1.7\\
0.8	1.8\\
0.8	1.9\\
0.8	2\\
0.8	2.1\\
0.8	2.2\\
0.8	2.3\\
0.8	2.4\\
0.8	2.5\\
0.8	2.6\\
0.8	2.7\\
0.8	2.8\\
0.8	2.9\\
0.8	3\\
0.8	3.1\\
0.8	3.2\\
0.8	3.3\\
0.8	3.4\\
0.8	3.5\\
0.8	3.6\\
0.8	3.7\\
0.8	3.8\\
0.8	3.9\\
0.8	4\\
0.8	4.1\\
0.8	4.2\\
0.8	4.3\\
0.8	4.4\\
0.8	4.5\\
0.8	4.6\\
0.8	4.7\\
0.8	4.8\\
0.8	4.9\\
0.8	5\\
0.8	5.1\\
0.8	5.2\\
0.8	5.3\\
0.8	5.4\\
0.8	5.5\\
0.8	5.6\\
0.8	5.7\\
0.8	5.8\\
0.8	5.9\\
};
\addplot [color=black, draw=none, mark size=0.2pt, mark=*, mark options={solid, black}, forget plot]
  table[row sep=crcr]{%
0.9	0.1\\
0.9	0.2\\
0.9	0.3\\
0.9	0.4\\
0.9	0.5\\
0.9	0.6\\
0.9	0.7\\
0.9	0.8\\
0.9	0.9\\
0.9	1\\
0.9	1.1\\
0.9	1.2\\
0.9	1.3\\
0.9	1.4\\
0.9	1.5\\
0.9	1.6\\
0.9	1.7\\
0.9	1.8\\
0.9	1.9\\
0.9	2\\
0.9	2.1\\
0.9	2.2\\
0.9	2.3\\
0.9	2.4\\
0.9	2.5\\
0.9	2.6\\
0.9	2.7\\
0.9	2.8\\
0.9	2.9\\
0.9	3\\
0.9	3.1\\
0.9	3.2\\
0.9	3.3\\
0.9	3.4\\
0.9	3.5\\
0.9	3.6\\
0.9	3.7\\
0.9	3.8\\
0.9	3.9\\
0.9	4\\
0.9	4.1\\
0.9	4.2\\
0.9	4.3\\
0.9	4.4\\
0.9	4.5\\
0.9	4.6\\
0.9	4.7\\
0.9	4.8\\
0.9	4.9\\
0.9	5\\
0.9	5.1\\
0.9	5.2\\
0.9	5.3\\
0.9	5.4\\
0.9	5.5\\
0.9	5.6\\
0.9	5.7\\
0.9	5.8\\
0.9	5.9\\
};
\addplot [color=black, draw=none, mark size=0.2pt, mark=*, mark options={solid, black}, forget plot]
  table[row sep=crcr]{%
1	0.1\\
1	0.2\\
1	0.3\\
1	0.4\\
1	0.5\\
1	0.6\\
1	0.7\\
1	0.8\\
1	0.9\\
1	1\\
1	1.1\\
1	1.2\\
1	1.3\\
1	1.4\\
1	1.5\\
1	1.6\\
1	1.7\\
1	1.8\\
1	1.9\\
1	2\\
1	2.1\\
1	2.2\\
1	2.3\\
1	2.4\\
1	2.5\\
1	2.6\\
1	2.7\\
1	2.8\\
1	2.9\\
1	3\\
1	3.1\\
1	3.2\\
1	3.3\\
1	3.4\\
1	3.5\\
1	3.6\\
1	3.7\\
1	3.8\\
1	3.9\\
1	4\\
1	4.1\\
1	4.2\\
1	4.3\\
1	4.4\\
1	4.5\\
1	4.6\\
1	4.7\\
1	4.8\\
1	4.9\\
1	5\\
1	5.1\\
1	5.2\\
1	5.3\\
1	5.4\\
1	5.5\\
1	5.6\\
1	5.7\\
1	5.8\\
1	5.9\\
};
\addplot [color=black, draw=none, mark size=0.2pt, mark=*, mark options={solid, black}, forget plot]
  table[row sep=crcr]{%
1.1	0.1\\
1.1	0.2\\
1.1	0.3\\
1.1	0.4\\
1.1	0.5\\
1.1	0.6\\
1.1	0.7\\
1.1	0.8\\
1.1	0.9\\
1.1	1\\
1.1	1.1\\
1.1	1.2\\
1.1	1.3\\
1.1	1.4\\
1.1	1.5\\
1.1	1.6\\
1.1	1.7\\
1.1	1.8\\
1.1	1.9\\
1.1	2\\
1.1	2.1\\
1.1	2.2\\
1.1	2.3\\
1.1	2.4\\
1.1	2.5\\
1.1	2.6\\
1.1	2.7\\
1.1	2.8\\
1.1	2.9\\
1.1	3\\
1.1	3.1\\
1.1	3.2\\
1.1	3.3\\
1.1	3.4\\
1.1	3.5\\
1.1	3.6\\
1.1	3.7\\
1.1	3.8\\
1.1	3.9\\
1.1	4\\
1.1	4.1\\
1.1	4.2\\
1.1	4.3\\
1.1	4.4\\
1.1	4.5\\
1.1	4.6\\
1.1	4.7\\
1.1	4.8\\
1.1	4.9\\
1.1	5\\
1.1	5.1\\
1.1	5.2\\
1.1	5.3\\
1.1	5.4\\
1.1	5.5\\
1.1	5.6\\
1.1	5.7\\
1.1	5.8\\
1.1	5.9\\
};
\addplot [color=black, draw=none, mark size=0.2pt, mark=*, mark options={solid, black}, forget plot]
  table[row sep=crcr]{%
1.2	0.1\\
1.2	0.2\\
1.2	0.3\\
1.2	0.4\\
1.2	0.5\\
1.2	0.6\\
1.2	0.7\\
1.2	0.8\\
1.2	0.9\\
1.2	1\\
1.2	1.1\\
1.2	1.2\\
1.2	1.3\\
1.2	1.4\\
1.2	1.5\\
1.2	1.6\\
1.2	1.7\\
1.2	1.8\\
1.2	1.9\\
1.2	2\\
1.2	2.1\\
1.2	2.2\\
1.2	2.3\\
1.2	2.4\\
1.2	2.5\\
1.2	2.6\\
1.2	2.7\\
1.2	2.8\\
1.2	2.9\\
1.2	3\\
1.2	3.1\\
1.2	3.2\\
1.2	3.3\\
1.2	3.4\\
1.2	3.5\\
1.2	3.6\\
1.2	3.7\\
1.2	3.8\\
1.2	3.9\\
1.2	4\\
1.2	4.1\\
1.2	4.2\\
1.2	4.3\\
1.2	4.4\\
1.2	4.5\\
1.2	4.6\\
1.2	4.7\\
1.2	4.8\\
1.2	4.9\\
1.2	5\\
1.2	5.1\\
1.2	5.2\\
1.2	5.3\\
1.2	5.4\\
1.2	5.5\\
1.2	5.6\\
1.2	5.7\\
1.2	5.8\\
1.2	5.9\\
};
\addplot [color=black, draw=none, mark size=0.2pt, mark=*, mark options={solid, black}, forget plot]
  table[row sep=crcr]{%
1.3	0.1\\
1.3	0.2\\
1.3	0.3\\
1.3	0.4\\
1.3	0.5\\
1.3	0.6\\
1.3	0.7\\
1.3	0.8\\
1.3	0.9\\
1.3	1\\
1.3	1.1\\
1.3	1.2\\
1.3	1.3\\
1.3	1.4\\
1.3	1.5\\
1.3	1.6\\
1.3	1.7\\
1.3	1.8\\
1.3	1.9\\
1.3	2\\
1.3	2.1\\
1.3	2.2\\
1.3	2.3\\
1.3	2.4\\
1.3	2.5\\
1.3	2.6\\
1.3	2.7\\
1.3	2.8\\
1.3	2.9\\
1.3	3\\
1.3	3.1\\
1.3	3.2\\
1.3	3.3\\
1.3	3.4\\
1.3	3.5\\
1.3	3.6\\
1.3	3.7\\
1.3	3.8\\
1.3	3.9\\
1.3	4\\
1.3	4.1\\
1.3	4.2\\
1.3	4.3\\
1.3	4.4\\
1.3	4.5\\
1.3	4.6\\
1.3	4.7\\
1.3	4.8\\
1.3	4.9\\
1.3	5\\
1.3	5.1\\
1.3	5.2\\
1.3	5.3\\
1.3	5.4\\
1.3	5.5\\
1.3	5.6\\
1.3	5.7\\
1.3	5.8\\
1.3	5.9\\
};
\addplot [color=black, draw=none, mark size=0.2pt, mark=*, mark options={solid, black}, forget plot]
  table[row sep=crcr]{%
1.4	0.1\\
1.4	0.2\\
1.4	0.3\\
1.4	0.4\\
1.4	0.5\\
1.4	0.6\\
1.4	0.7\\
1.4	0.8\\
1.4	0.9\\
1.4	1\\
1.4	1.1\\
1.4	1.2\\
1.4	1.3\\
1.4	1.4\\
1.4	1.5\\
1.4	1.6\\
1.4	1.7\\
1.4	1.8\\
1.4	1.9\\
1.4	2\\
1.4	2.1\\
1.4	2.2\\
1.4	2.3\\
1.4	2.4\\
1.4	2.5\\
1.4	2.6\\
1.4	2.7\\
1.4	2.8\\
1.4	2.9\\
1.4	3\\
1.4	3.1\\
1.4	3.2\\
1.4	3.3\\
1.4	3.4\\
1.4	3.5\\
1.4	3.6\\
1.4	3.7\\
1.4	3.8\\
1.4	3.9\\
1.4	4\\
1.4	4.1\\
1.4	4.2\\
1.4	4.3\\
1.4	4.4\\
1.4	4.5\\
1.4	4.6\\
1.4	4.7\\
1.4	4.8\\
1.4	4.9\\
1.4	5\\
1.4	5.1\\
1.4	5.2\\
1.4	5.3\\
1.4	5.4\\
1.4	5.5\\
1.4	5.6\\
1.4	5.7\\
1.4	5.8\\
1.4	5.9\\
};
\addplot [color=black, draw=none, mark size=0.2pt, mark=*, mark options={solid, black}, forget plot]
  table[row sep=crcr]{%
1.5	0.1\\
1.5	0.2\\
1.5	0.3\\
1.5	0.4\\
1.5	0.5\\
1.5	0.6\\
1.5	0.7\\
1.5	0.8\\
1.5	0.9\\
1.5	1\\
1.5	1.1\\
1.5	1.2\\
1.5	1.3\\
1.5	1.4\\
1.5	1.5\\
1.5	1.6\\
1.5	1.7\\
1.5	1.8\\
1.5	1.9\\
1.5	2\\
1.5	2.1\\
1.5	2.2\\
1.5	2.3\\
1.5	2.4\\
1.5	2.5\\
1.5	2.6\\
1.5	2.7\\
1.5	2.8\\
1.5	2.9\\
1.5	3\\
1.5	3.1\\
1.5	3.2\\
1.5	3.3\\
1.5	3.4\\
1.5	3.5\\
1.5	3.6\\
1.5	3.7\\
1.5	3.8\\
1.5	3.9\\
1.5	4\\
1.5	4.1\\
1.5	4.2\\
1.5	4.3\\
1.5	4.4\\
1.5	4.5\\
1.5	4.6\\
1.5	4.7\\
1.5	4.8\\
1.5	4.9\\
1.5	5\\
1.5	5.1\\
1.5	5.2\\
1.5	5.3\\
1.5	5.4\\
1.5	5.5\\
1.5	5.6\\
1.5	5.7\\
1.5	5.8\\
1.5	5.9\\
};
\addplot [color=black, draw=none, mark size=0.2pt, mark=*, mark options={solid, black}, forget plot]
  table[row sep=crcr]{%
1.6	0.1\\
1.6	0.2\\
1.6	0.3\\
1.6	0.4\\
1.6	0.5\\
1.6	0.6\\
1.6	0.7\\
1.6	0.8\\
1.6	0.9\\
1.6	1\\
1.6	1.1\\
1.6	1.2\\
1.6	1.3\\
1.6	1.4\\
1.6	1.5\\
1.6	1.6\\
1.6	1.7\\
1.6	1.8\\
1.6	1.9\\
1.6	2\\
1.6	2.1\\
1.6	2.2\\
1.6	2.3\\
1.6	2.4\\
1.6	2.5\\
1.6	2.6\\
1.6	2.7\\
1.6	2.8\\
1.6	2.9\\
1.6	3\\
1.6	3.1\\
1.6	3.2\\
1.6	3.3\\
1.6	3.4\\
1.6	3.5\\
1.6	3.6\\
1.6	3.7\\
1.6	3.8\\
1.6	3.9\\
1.6	4\\
1.6	4.1\\
1.6	4.2\\
1.6	4.3\\
1.6	4.4\\
1.6	4.5\\
1.6	4.6\\
1.6	4.7\\
1.6	4.8\\
1.6	4.9\\
1.6	5\\
1.6	5.1\\
1.6	5.2\\
1.6	5.3\\
1.6	5.4\\
1.6	5.5\\
1.6	5.6\\
1.6	5.7\\
1.6	5.8\\
1.6	5.9\\
};
\addplot [color=black, draw=none, mark size=0.2pt, mark=*, mark options={solid, black}, forget plot]
  table[row sep=crcr]{%
1.7	0.1\\
1.7	0.2\\
1.7	0.3\\
1.7	0.4\\
1.7	0.5\\
1.7	0.6\\
1.7	0.7\\
1.7	0.8\\
1.7	0.9\\
1.7	1\\
1.7	1.1\\
1.7	1.2\\
1.7	1.3\\
1.7	1.4\\
1.7	1.5\\
1.7	1.6\\
1.7	1.7\\
1.7	1.8\\
1.7	1.9\\
1.7	2\\
1.7	2.1\\
1.7	2.2\\
1.7	2.3\\
1.7	2.4\\
1.7	2.5\\
1.7	2.6\\
1.7	2.7\\
1.7	2.8\\
1.7	2.9\\
1.7	3\\
1.7	3.1\\
1.7	3.2\\
1.7	3.3\\
1.7	3.4\\
1.7	3.5\\
1.7	3.6\\
1.7	3.7\\
1.7	3.8\\
1.7	3.9\\
1.7	4\\
1.7	4.1\\
1.7	4.2\\
1.7	4.3\\
1.7	4.4\\
1.7	4.5\\
1.7	4.6\\
1.7	4.7\\
1.7	4.8\\
1.7	4.9\\
1.7	5\\
1.7	5.1\\
1.7	5.2\\
1.7	5.3\\
1.7	5.4\\
1.7	5.5\\
1.7	5.6\\
1.7	5.7\\
1.7	5.8\\
1.7	5.9\\
};
\addplot [color=black, draw=none, mark size=0.2pt, mark=*, mark options={solid, black}, forget plot]
  table[row sep=crcr]{%
1.8	0.1\\
1.8	0.2\\
1.8	0.3\\
1.8	0.4\\
1.8	0.5\\
1.8	0.6\\
1.8	0.7\\
1.8	0.8\\
1.8	0.9\\
1.8	1\\
1.8	1.1\\
1.8	1.2\\
1.8	1.3\\
1.8	1.4\\
1.8	1.5\\
1.8	1.6\\
1.8	1.7\\
1.8	1.8\\
1.8	1.9\\
1.8	2\\
1.8	2.1\\
1.8	2.2\\
1.8	2.3\\
1.8	2.4\\
1.8	2.5\\
1.8	2.6\\
1.8	2.7\\
1.8	2.8\\
1.8	2.9\\
1.8	3\\
1.8	3.1\\
1.8	3.2\\
1.8	3.3\\
1.8	3.4\\
1.8	3.5\\
1.8	3.6\\
1.8	3.7\\
1.8	3.8\\
1.8	3.9\\
1.8	4\\
1.8	4.1\\
1.8	4.2\\
1.8	4.3\\
1.8	4.4\\
1.8	4.5\\
1.8	4.6\\
1.8	4.7\\
1.8	4.8\\
1.8	4.9\\
1.8	5\\
1.8	5.1\\
1.8	5.2\\
1.8	5.3\\
1.8	5.4\\
1.8	5.5\\
1.8	5.6\\
1.8	5.7\\
1.8	5.8\\
1.8	5.9\\
};
\addplot [color=black, draw=none, mark size=0.2pt, mark=*, mark options={solid, black}, forget plot]
  table[row sep=crcr]{%
1.9	0.1\\
1.9	0.2\\
1.9	0.3\\
1.9	0.4\\
1.9	0.5\\
1.9	0.6\\
1.9	0.7\\
1.9	0.8\\
1.9	0.9\\
1.9	1\\
1.9	1.1\\
1.9	1.2\\
1.9	1.3\\
1.9	1.4\\
1.9	1.5\\
1.9	1.6\\
1.9	1.7\\
1.9	1.8\\
1.9	1.9\\
1.9	2\\
1.9	2.1\\
1.9	2.2\\
1.9	2.3\\
1.9	2.4\\
1.9	2.5\\
1.9	2.6\\
1.9	2.7\\
1.9	2.8\\
1.9	2.9\\
1.9	3\\
1.9	3.1\\
1.9	3.2\\
1.9	3.3\\
1.9	3.4\\
1.9	3.5\\
1.9	3.6\\
1.9	3.7\\
1.9	3.8\\
1.9	3.9\\
1.9	4\\
1.9	4.1\\
1.9	4.2\\
1.9	4.3\\
1.9	4.4\\
1.9	4.5\\
1.9	4.6\\
1.9	4.7\\
1.9	4.8\\
1.9	4.9\\
1.9	5\\
1.9	5.1\\
1.9	5.2\\
1.9	5.3\\
1.9	5.4\\
1.9	5.5\\
1.9	5.6\\
1.9	5.7\\
1.9	5.8\\
1.9	5.9\\
};
\addplot [color=black, draw=none, mark size=0.2pt, mark=*, mark options={solid, black}, forget plot]
  table[row sep=crcr]{%
2	0.1\\
2	0.2\\
2	0.3\\
2	0.4\\
2	0.5\\
2	0.6\\
2	0.7\\
2	0.8\\
2	0.9\\
2	1\\
2	1.1\\
2	1.2\\
2	1.3\\
2	1.4\\
2	1.5\\
2	1.6\\
2	1.7\\
2	1.8\\
2	1.9\\
2	2\\
2	2.1\\
2	2.2\\
2	2.3\\
2	2.4\\
2	2.5\\
2	2.6\\
2	2.7\\
2	2.8\\
2	2.9\\
2	3\\
2	3.1\\
2	3.2\\
2	3.3\\
2	3.4\\
2	3.5\\
2	3.6\\
2	3.7\\
2	3.8\\
2	3.9\\
2	4\\
2	4.1\\
2	4.2\\
2	4.3\\
2	4.4\\
2	4.5\\
2	4.6\\
2	4.7\\
2	4.8\\
2	4.9\\
2	5\\
2	5.1\\
2	5.2\\
2	5.3\\
2	5.4\\
2	5.5\\
2	5.6\\
2	5.7\\
2	5.8\\
2	5.9\\
};
\addplot [color=black, draw=none, mark size=0.2pt, mark=*, mark options={solid, black}, forget plot]
  table[row sep=crcr]{%
2.1	0.1\\
2.1	0.2\\
2.1	0.3\\
2.1	0.4\\
2.1	0.5\\
2.1	0.6\\
2.1	0.7\\
2.1	0.8\\
2.1	0.9\\
2.1	1\\
2.1	1.1\\
2.1	1.2\\
2.1	1.3\\
2.1	1.4\\
2.1	1.5\\
2.1	1.6\\
2.1	1.7\\
2.1	1.8\\
2.1	1.9\\
2.1	2\\
2.1	2.1\\
2.1	2.2\\
2.1	2.3\\
2.1	2.4\\
2.1	2.5\\
2.1	2.6\\
2.1	2.7\\
2.1	2.8\\
2.1	2.9\\
2.1	3\\
2.1	3.1\\
2.1	3.2\\
2.1	3.3\\
2.1	3.4\\
2.1	3.5\\
2.1	3.6\\
2.1	3.7\\
2.1	3.8\\
2.1	3.9\\
2.1	4\\
2.1	4.1\\
2.1	4.2\\
2.1	4.3\\
2.1	4.4\\
2.1	4.5\\
2.1	4.6\\
2.1	4.7\\
2.1	4.8\\
2.1	4.9\\
2.1	5\\
2.1	5.1\\
2.1	5.2\\
2.1	5.3\\
2.1	5.4\\
2.1	5.5\\
2.1	5.6\\
2.1	5.7\\
2.1	5.8\\
2.1	5.9\\
};
\addplot [color=black, draw=none, mark size=0.2pt, mark=*, mark options={solid, black}, forget plot]
  table[row sep=crcr]{%
2.2	0.1\\
2.2	0.2\\
2.2	0.3\\
2.2	0.4\\
2.2	0.5\\
2.2	0.6\\
2.2	0.7\\
2.2	0.8\\
2.2	0.9\\
2.2	1\\
2.2	1.1\\
2.2	1.2\\
2.2	1.3\\
2.2	1.4\\
2.2	1.5\\
2.2	1.6\\
2.2	1.7\\
2.2	1.8\\
2.2	1.9\\
2.2	2\\
2.2	2.1\\
2.2	2.2\\
2.2	2.3\\
2.2	2.4\\
2.2	2.5\\
2.2	2.6\\
2.2	2.7\\
2.2	2.8\\
2.2	2.9\\
2.2	3\\
2.2	3.1\\
2.2	3.2\\
2.2	3.3\\
2.2	3.4\\
2.2	3.5\\
2.2	3.6\\
2.2	3.7\\
2.2	3.8\\
2.2	3.9\\
2.2	4\\
2.2	4.1\\
2.2	4.2\\
2.2	4.3\\
2.2	4.4\\
2.2	4.5\\
2.2	4.6\\
2.2	4.7\\
2.2	4.8\\
2.2	4.9\\
2.2	5\\
2.2	5.1\\
2.2	5.2\\
2.2	5.3\\
2.2	5.4\\
2.2	5.5\\
2.2	5.6\\
2.2	5.7\\
2.2	5.8\\
2.2	5.9\\
};
\addplot [color=black, draw=none, mark size=0.2pt, mark=*, mark options={solid, black}, forget plot]
  table[row sep=crcr]{%
2.3	0.1\\
2.3	0.2\\
2.3	0.3\\
2.3	0.4\\
2.3	0.5\\
2.3	0.6\\
2.3	0.7\\
2.3	0.8\\
2.3	0.9\\
2.3	1\\
2.3	1.1\\
2.3	1.2\\
2.3	1.3\\
2.3	1.4\\
2.3	1.5\\
2.3	1.6\\
2.3	1.7\\
2.3	1.8\\
2.3	1.9\\
2.3	2\\
2.3	2.1\\
2.3	2.2\\
2.3	2.3\\
2.3	2.4\\
2.3	2.5\\
2.3	2.6\\
2.3	2.7\\
2.3	2.8\\
2.3	2.9\\
2.3	3\\
2.3	3.1\\
2.3	3.2\\
2.3	3.3\\
2.3	3.4\\
2.3	3.5\\
2.3	3.6\\
2.3	3.7\\
2.3	3.8\\
2.3	3.9\\
2.3	4\\
2.3	4.1\\
2.3	4.2\\
2.3	4.3\\
2.3	4.4\\
2.3	4.5\\
2.3	4.6\\
2.3	4.7\\
2.3	4.8\\
2.3	4.9\\
2.3	5\\
2.3	5.1\\
2.3	5.2\\
2.3	5.3\\
2.3	5.4\\
2.3	5.5\\
2.3	5.6\\
2.3	5.7\\
2.3	5.8\\
2.3	5.9\\
};
\addplot [color=black, draw=none, mark size=0.2pt, mark=*, mark options={solid, black}, forget plot]
  table[row sep=crcr]{%
2.4	0.1\\
2.4	0.2\\
2.4	0.3\\
2.4	0.4\\
2.4	0.5\\
2.4	0.6\\
2.4	0.7\\
2.4	0.8\\
2.4	0.9\\
2.4	1\\
2.4	1.1\\
2.4	1.2\\
2.4	1.3\\
2.4	1.4\\
2.4	1.5\\
2.4	1.6\\
2.4	1.7\\
2.4	1.8\\
2.4	1.9\\
2.4	2\\
2.4	2.1\\
2.4	2.2\\
2.4	2.3\\
2.4	2.4\\
2.4	2.5\\
2.4	2.6\\
2.4	2.7\\
2.4	2.8\\
2.4	2.9\\
2.4	3\\
2.4	3.1\\
2.4	3.2\\
2.4	3.3\\
2.4	3.4\\
2.4	3.5\\
2.4	3.6\\
2.4	3.7\\
2.4	3.8\\
2.4	3.9\\
2.4	4\\
2.4	4.1\\
2.4	4.2\\
2.4	4.3\\
2.4	4.4\\
2.4	4.5\\
2.4	4.6\\
2.4	4.7\\
2.4	4.8\\
2.4	4.9\\
2.4	5\\
2.4	5.1\\
2.4	5.2\\
2.4	5.3\\
2.4	5.4\\
2.4	5.5\\
2.4	5.6\\
2.4	5.7\\
2.4	5.8\\
2.4	5.9\\
};
\addplot [color=black, draw=none, mark size=0.2pt, mark=*, mark options={solid, black}, forget plot]
  table[row sep=crcr]{%
2.5	0.1\\
2.5	0.2\\
2.5	0.3\\
2.5	0.4\\
2.5	0.5\\
2.5	0.6\\
2.5	0.7\\
2.5	0.8\\
2.5	0.9\\
2.5	1\\
2.5	1.1\\
2.5	1.2\\
2.5	1.3\\
2.5	1.4\\
2.5	1.5\\
2.5	1.6\\
2.5	1.7\\
2.5	1.8\\
2.5	1.9\\
2.5	2\\
2.5	2.1\\
2.5	2.2\\
2.5	2.3\\
2.5	2.4\\
2.5	2.5\\
2.5	2.6\\
2.5	2.7\\
2.5	2.8\\
2.5	2.9\\
2.5	3\\
2.5	3.1\\
2.5	3.2\\
2.5	3.3\\
2.5	3.4\\
2.5	3.5\\
2.5	3.6\\
2.5	3.7\\
2.5	3.8\\
2.5	3.9\\
2.5	4\\
2.5	4.1\\
2.5	4.2\\
2.5	4.3\\
2.5	4.4\\
2.5	4.5\\
2.5	4.6\\
2.5	4.7\\
2.5	4.8\\
2.5	4.9\\
2.5	5\\
2.5	5.1\\
2.5	5.2\\
2.5	5.3\\
2.5	5.4\\
2.5	5.5\\
2.5	5.6\\
2.5	5.7\\
2.5	5.8\\
2.5	5.9\\
};
\addplot [color=black, draw=none, mark size=0.2pt, mark=*, mark options={solid, black}, forget plot]
  table[row sep=crcr]{%
2.6	0.1\\
2.6	0.2\\
2.6	0.3\\
2.6	0.4\\
2.6	0.5\\
2.6	0.6\\
2.6	0.7\\
2.6	0.8\\
2.6	0.9\\
2.6	1\\
2.6	1.1\\
2.6	1.2\\
2.6	1.3\\
2.6	1.4\\
2.6	1.5\\
2.6	1.6\\
2.6	1.7\\
2.6	1.8\\
2.6	1.9\\
2.6	2\\
2.6	2.1\\
2.6	2.2\\
2.6	2.3\\
2.6	2.4\\
2.6	2.5\\
2.6	2.6\\
2.6	2.7\\
2.6	2.8\\
2.6	2.9\\
2.6	3\\
2.6	3.1\\
2.6	3.2\\
2.6	3.3\\
2.6	3.4\\
2.6	3.5\\
2.6	3.6\\
2.6	3.7\\
2.6	3.8\\
2.6	3.9\\
2.6	4\\
2.6	4.1\\
2.6	4.2\\
2.6	4.3\\
2.6	4.4\\
2.6	4.5\\
2.6	4.6\\
2.6	4.7\\
2.6	4.8\\
2.6	4.9\\
2.6	5\\
2.6	5.1\\
2.6	5.2\\
2.6	5.3\\
2.6	5.4\\
2.6	5.5\\
2.6	5.6\\
2.6	5.7\\
2.6	5.8\\
2.6	5.9\\
};
\addplot [color=black, draw=none, mark size=0.2pt, mark=*, mark options={solid, black}, forget plot]
  table[row sep=crcr]{%
2.7	0.1\\
2.7	0.2\\
2.7	0.3\\
2.7	0.4\\
2.7	0.5\\
2.7	0.6\\
2.7	0.7\\
2.7	0.8\\
2.7	0.9\\
2.7	1\\
2.7	1.1\\
2.7	1.2\\
2.7	1.3\\
2.7	1.4\\
2.7	1.5\\
2.7	1.6\\
2.7	1.7\\
2.7	1.8\\
2.7	1.9\\
2.7	2\\
2.7	2.1\\
2.7	2.2\\
2.7	2.3\\
2.7	2.4\\
2.7	2.5\\
2.7	2.6\\
2.7	2.7\\
2.7	2.8\\
2.7	2.9\\
2.7	3\\
2.7	3.1\\
2.7	3.2\\
2.7	3.3\\
2.7	3.4\\
2.7	3.5\\
2.7	3.6\\
2.7	3.7\\
2.7	3.8\\
2.7	3.9\\
2.7	4\\
2.7	4.1\\
2.7	4.2\\
2.7	4.3\\
2.7	4.4\\
2.7	4.5\\
2.7	4.6\\
2.7	4.7\\
2.7	4.8\\
2.7	4.9\\
2.7	5\\
2.7	5.1\\
2.7	5.2\\
2.7	5.3\\
2.7	5.4\\
2.7	5.5\\
2.7	5.6\\
2.7	5.7\\
2.7	5.8\\
2.7	5.9\\
};
\addplot [color=black, draw=none, mark size=0.2pt, mark=*, mark options={solid, black}, forget plot]
  table[row sep=crcr]{%
2.8	0.1\\
2.8	0.2\\
2.8	0.3\\
2.8	0.4\\
2.8	0.5\\
2.8	0.6\\
2.8	0.7\\
2.8	0.8\\
2.8	0.9\\
2.8	1\\
2.8	1.1\\
2.8	1.2\\
2.8	1.3\\
2.8	1.4\\
2.8	1.5\\
2.8	1.6\\
2.8	1.7\\
2.8	1.8\\
2.8	1.9\\
2.8	2\\
2.8	2.1\\
2.8	2.2\\
2.8	2.3\\
2.8	2.4\\
2.8	2.5\\
2.8	2.6\\
2.8	2.7\\
2.8	2.8\\
2.8	2.9\\
2.8	3\\
2.8	3.1\\
2.8	3.2\\
2.8	3.3\\
2.8	3.4\\
2.8	3.5\\
2.8	3.6\\
2.8	3.7\\
2.8	3.8\\
2.8	3.9\\
2.8	4\\
2.8	4.1\\
2.8	4.2\\
2.8	4.3\\
2.8	4.4\\
2.8	4.5\\
2.8	4.6\\
2.8	4.7\\
2.8	4.8\\
2.8	4.9\\
2.8	5\\
2.8	5.1\\
2.8	5.2\\
2.8	5.3\\
2.8	5.4\\
2.8	5.5\\
2.8	5.6\\
2.8	5.7\\
2.8	5.8\\
2.8	5.9\\
};
\addplot [color=black, draw=none, mark size=0.2pt, mark=*, mark options={solid, black}, forget plot]
  table[row sep=crcr]{%
2.9	0.1\\
2.9	0.2\\
2.9	0.3\\
2.9	0.4\\
2.9	0.5\\
2.9	0.6\\
2.9	0.7\\
2.9	0.8\\
2.9	0.9\\
2.9	1\\
2.9	1.1\\
2.9	1.2\\
2.9	1.3\\
2.9	1.4\\
2.9	1.5\\
2.9	1.6\\
2.9	1.7\\
2.9	1.8\\
2.9	1.9\\
2.9	2\\
2.9	2.1\\
2.9	2.2\\
2.9	2.3\\
2.9	2.4\\
2.9	2.5\\
2.9	2.6\\
2.9	2.7\\
2.9	2.8\\
2.9	2.9\\
2.9	3\\
2.9	3.1\\
2.9	3.2\\
2.9	3.3\\
2.9	3.4\\
2.9	3.5\\
2.9	3.6\\
2.9	3.7\\
2.9	3.8\\
2.9	3.9\\
2.9	4\\
2.9	4.1\\
2.9	4.2\\
2.9	4.3\\
2.9	4.4\\
2.9	4.5\\
2.9	4.6\\
2.9	4.7\\
2.9	4.8\\
2.9	4.9\\
2.9	5\\
2.9	5.1\\
2.9	5.2\\
2.9	5.3\\
2.9	5.4\\
2.9	5.5\\
2.9	5.6\\
2.9	5.7\\
2.9	5.8\\
2.9	5.9\\
};
\addplot [color=black, draw=none, mark size=0.2pt, mark=*, mark options={solid, black}, forget plot]
  table[row sep=crcr]{%
3	0.1\\
3	0.2\\
3	0.3\\
3	0.4\\
3	0.5\\
3	0.6\\
3	0.7\\
3	0.8\\
3	0.9\\
3	1\\
3	1.1\\
3	1.2\\
3	1.3\\
3	1.4\\
3	1.5\\
3	1.6\\
3	1.7\\
3	1.8\\
3	1.9\\
3	2\\
3	2.1\\
3	2.2\\
3	2.3\\
3	2.4\\
3	2.5\\
3	2.6\\
3	2.7\\
3	2.8\\
3	2.9\\
3	3\\
3	3.1\\
3	3.2\\
3	3.3\\
3	3.4\\
3	3.5\\
3	3.6\\
3	3.7\\
3	3.8\\
3	3.9\\
3	4\\
3	4.1\\
3	4.2\\
3	4.3\\
3	4.4\\
3	4.5\\
3	4.6\\
3	4.7\\
3	4.8\\
3	4.9\\
3	5\\
3	5.1\\
3	5.2\\
3	5.3\\
3	5.4\\
3	5.5\\
3	5.6\\
3	5.7\\
3	5.8\\
3	5.9\\
};
\addplot [color=black, draw=none, mark size=0.2pt, mark=*, mark options={solid, black}, forget plot]
  table[row sep=crcr]{%
3.1	0.1\\
3.1	0.2\\
3.1	0.3\\
3.1	0.4\\
3.1	0.5\\
3.1	0.6\\
3.1	0.7\\
3.1	0.8\\
3.1	0.9\\
3.1	1\\
3.1	1.1\\
3.1	1.2\\
3.1	1.3\\
3.1	1.4\\
3.1	1.5\\
3.1	1.6\\
3.1	1.7\\
3.1	1.8\\
3.1	1.9\\
3.1	2\\
3.1	2.1\\
3.1	2.2\\
3.1	2.3\\
3.1	2.4\\
3.1	2.5\\
3.1	2.6\\
3.1	2.7\\
3.1	2.8\\
3.1	2.9\\
3.1	3\\
3.1	3.1\\
3.1	3.2\\
3.1	3.3\\
3.1	3.4\\
3.1	3.5\\
3.1	3.6\\
3.1	3.7\\
3.1	3.8\\
3.1	3.9\\
3.1	4\\
3.1	4.1\\
3.1	4.2\\
3.1	4.3\\
3.1	4.4\\
3.1	4.5\\
3.1	4.6\\
3.1	4.7\\
3.1	4.8\\
3.1	4.9\\
3.1	5\\
3.1	5.1\\
3.1	5.2\\
3.1	5.3\\
3.1	5.4\\
3.1	5.5\\
3.1	5.6\\
3.1	5.7\\
3.1	5.8\\
3.1	5.9\\
};
\addplot [color=black, draw=none, mark size=0.2pt, mark=*, mark options={solid, black}, forget plot]
  table[row sep=crcr]{%
3.2	0.1\\
3.2	0.2\\
3.2	0.3\\
3.2	0.4\\
3.2	0.5\\
3.2	0.6\\
3.2	0.7\\
3.2	0.8\\
3.2	0.9\\
3.2	1\\
3.2	1.1\\
3.2	1.2\\
3.2	1.3\\
3.2	1.4\\
3.2	1.5\\
3.2	1.6\\
3.2	1.7\\
3.2	1.8\\
3.2	1.9\\
3.2	2\\
3.2	2.1\\
3.2	2.2\\
3.2	2.3\\
3.2	2.4\\
3.2	2.5\\
3.2	2.6\\
3.2	2.7\\
3.2	2.8\\
3.2	2.9\\
3.2	3\\
3.2	3.1\\
3.2	3.2\\
3.2	3.3\\
3.2	3.4\\
3.2	3.5\\
3.2	3.6\\
3.2	3.7\\
3.2	3.8\\
3.2	3.9\\
3.2	4\\
3.2	4.1\\
3.2	4.2\\
3.2	4.3\\
3.2	4.4\\
3.2	4.5\\
3.2	4.6\\
3.2	4.7\\
3.2	4.8\\
3.2	4.9\\
3.2	5\\
3.2	5.1\\
3.2	5.2\\
3.2	5.3\\
3.2	5.4\\
3.2	5.5\\
3.2	5.6\\
3.2	5.7\\
3.2	5.8\\
3.2	5.9\\
};
\addplot [color=black, draw=none, mark size=0.2pt, mark=*, mark options={solid, black}, forget plot]
  table[row sep=crcr]{%
3.3	0.1\\
3.3	0.2\\
3.3	0.3\\
3.3	0.4\\
3.3	0.5\\
3.3	0.6\\
3.3	0.7\\
3.3	0.8\\
3.3	0.9\\
3.3	1\\
3.3	1.1\\
3.3	1.2\\
3.3	1.3\\
3.3	1.4\\
3.3	1.5\\
3.3	1.6\\
3.3	1.7\\
3.3	1.8\\
3.3	1.9\\
3.3	2\\
3.3	2.1\\
3.3	2.2\\
3.3	2.3\\
3.3	2.4\\
3.3	2.5\\
3.3	2.6\\
3.3	2.7\\
3.3	2.8\\
3.3	2.9\\
3.3	3\\
3.3	3.1\\
3.3	3.2\\
3.3	3.3\\
3.3	3.4\\
3.3	3.5\\
3.3	3.6\\
3.3	3.7\\
3.3	3.8\\
3.3	3.9\\
3.3	4\\
3.3	4.1\\
3.3	4.2\\
3.3	4.3\\
3.3	4.4\\
3.3	4.5\\
3.3	4.6\\
3.3	4.7\\
3.3	4.8\\
3.3	4.9\\
3.3	5\\
3.3	5.1\\
3.3	5.2\\
3.3	5.3\\
3.3	5.4\\
3.3	5.5\\
3.3	5.6\\
3.3	5.7\\
3.3	5.8\\
3.3	5.9\\
};
\addplot [color=black, draw=none, mark size=0.2pt, mark=*, mark options={solid, black}, forget plot]
  table[row sep=crcr]{%
3.4	0.1\\
3.4	0.2\\
3.4	0.3\\
3.4	0.4\\
3.4	0.5\\
3.4	0.6\\
3.4	0.7\\
3.4	0.8\\
3.4	0.9\\
3.4	1\\
3.4	1.1\\
3.4	1.2\\
3.4	1.3\\
3.4	1.4\\
3.4	1.5\\
3.4	1.6\\
3.4	1.7\\
3.4	1.8\\
3.4	1.9\\
3.4	2\\
3.4	2.1\\
3.4	2.2\\
3.4	2.3\\
3.4	2.4\\
3.4	2.5\\
3.4	2.6\\
3.4	2.7\\
3.4	2.8\\
3.4	2.9\\
3.4	3\\
3.4	3.1\\
3.4	3.2\\
3.4	3.3\\
3.4	3.4\\
3.4	3.5\\
3.4	3.6\\
3.4	3.7\\
3.4	3.8\\
3.4	3.9\\
3.4	4\\
3.4	4.1\\
3.4	4.2\\
3.4	4.3\\
3.4	4.4\\
3.4	4.5\\
3.4	4.6\\
3.4	4.7\\
3.4	4.8\\
3.4	4.9\\
3.4	5\\
3.4	5.1\\
3.4	5.2\\
3.4	5.3\\
3.4	5.4\\
3.4	5.5\\
3.4	5.6\\
3.4	5.7\\
3.4	5.8\\
3.4	5.9\\
};
\addplot [color=black, draw=none, mark size=0.2pt, mark=*, mark options={solid, black}, forget plot]
  table[row sep=crcr]{%
3.5	0.1\\
3.5	0.2\\
3.5	0.3\\
3.5	0.4\\
3.5	0.5\\
3.5	0.6\\
3.5	0.7\\
3.5	0.8\\
3.5	0.9\\
3.5	1\\
3.5	1.1\\
3.5	1.2\\
3.5	1.3\\
3.5	1.4\\
3.5	1.5\\
3.5	1.6\\
3.5	1.7\\
3.5	1.8\\
3.5	1.9\\
3.5	2\\
3.5	2.1\\
3.5	2.2\\
3.5	2.3\\
3.5	2.4\\
3.5	2.5\\
3.5	2.6\\
3.5	2.7\\
3.5	2.8\\
3.5	2.9\\
3.5	3\\
3.5	3.1\\
3.5	3.2\\
3.5	3.3\\
3.5	3.4\\
3.5	3.5\\
3.5	3.6\\
3.5	3.7\\
3.5	3.8\\
3.5	3.9\\
3.5	4\\
3.5	4.1\\
3.5	4.2\\
3.5	4.3\\
3.5	4.4\\
3.5	4.5\\
3.5	4.6\\
3.5	4.7\\
3.5	4.8\\
3.5	4.9\\
3.5	5\\
3.5	5.1\\
3.5	5.2\\
3.5	5.3\\
3.5	5.4\\
3.5	5.5\\
3.5	5.6\\
3.5	5.7\\
3.5	5.8\\
3.5	5.9\\
};
\addplot [color=black, draw=none, mark size=0.2pt, mark=*, mark options={solid, black}, forget plot]
  table[row sep=crcr]{%
3.6	0.1\\
3.6	0.2\\
3.6	0.3\\
3.6	0.4\\
3.6	0.5\\
3.6	0.6\\
3.6	0.7\\
3.6	0.8\\
3.6	0.9\\
3.6	1\\
3.6	1.1\\
3.6	1.2\\
3.6	1.3\\
3.6	1.4\\
3.6	1.5\\
3.6	1.6\\
3.6	1.7\\
3.6	1.8\\
3.6	1.9\\
3.6	2\\
3.6	2.1\\
3.6	2.2\\
3.6	2.3\\
3.6	2.4\\
3.6	2.5\\
3.6	2.6\\
3.6	2.7\\
3.6	2.8\\
3.6	2.9\\
3.6	3\\
3.6	3.1\\
3.6	3.2\\
3.6	3.3\\
3.6	3.4\\
3.6	3.5\\
3.6	3.6\\
3.6	3.7\\
3.6	3.8\\
3.6	3.9\\
3.6	4\\
3.6	4.1\\
3.6	4.2\\
3.6	4.3\\
3.6	4.4\\
3.6	4.5\\
3.6	4.6\\
3.6	4.7\\
3.6	4.8\\
3.6	4.9\\
3.6	5\\
3.6	5.1\\
3.6	5.2\\
3.6	5.3\\
3.6	5.4\\
3.6	5.5\\
3.6	5.6\\
3.6	5.7\\
3.6	5.8\\
3.6	5.9\\
};
\addplot [color=black, draw=none, mark size=0.2pt, mark=*, mark options={solid, black}, forget plot]
  table[row sep=crcr]{%
3.7	0.1\\
3.7	0.2\\
3.7	0.3\\
3.7	0.4\\
3.7	0.5\\
3.7	0.6\\
3.7	0.7\\
3.7	0.8\\
3.7	0.9\\
3.7	1\\
3.7	1.1\\
3.7	1.2\\
3.7	1.3\\
3.7	1.4\\
3.7	1.5\\
3.7	1.6\\
3.7	1.7\\
3.7	1.8\\
3.7	1.9\\
3.7	2\\
3.7	2.1\\
3.7	2.2\\
3.7	2.3\\
3.7	2.4\\
3.7	2.5\\
3.7	2.6\\
3.7	2.7\\
3.7	2.8\\
3.7	2.9\\
3.7	3\\
3.7	3.1\\
3.7	3.2\\
3.7	3.3\\
3.7	3.4\\
3.7	3.5\\
3.7	3.6\\
3.7	3.7\\
3.7	3.8\\
3.7	3.9\\
3.7	4\\
3.7	4.1\\
3.7	4.2\\
3.7	4.3\\
3.7	4.4\\
3.7	4.5\\
3.7	4.6\\
3.7	4.7\\
3.7	4.8\\
3.7	4.9\\
3.7	5\\
3.7	5.1\\
3.7	5.2\\
3.7	5.3\\
3.7	5.4\\
3.7	5.5\\
3.7	5.6\\
3.7	5.7\\
3.7	5.8\\
3.7	5.9\\
};
\addplot [color=black, draw=none, mark size=0.2pt, mark=*, mark options={solid, black}, forget plot]
  table[row sep=crcr]{%
3.8	0.1\\
3.8	0.2\\
3.8	0.3\\
3.8	0.4\\
3.8	0.5\\
3.8	0.6\\
3.8	0.7\\
3.8	0.8\\
3.8	0.9\\
3.8	1\\
3.8	1.1\\
3.8	1.2\\
3.8	1.3\\
3.8	1.4\\
3.8	1.5\\
3.8	1.6\\
3.8	1.7\\
3.8	1.8\\
3.8	1.9\\
3.8	2\\
3.8	2.1\\
3.8	2.2\\
3.8	2.3\\
3.8	2.4\\
3.8	2.5\\
3.8	2.6\\
3.8	2.7\\
3.8	2.8\\
3.8	2.9\\
3.8	3\\
3.8	3.1\\
3.8	3.2\\
3.8	3.3\\
3.8	3.4\\
3.8	3.5\\
3.8	3.6\\
3.8	3.7\\
3.8	3.8\\
3.8	3.9\\
3.8	4\\
3.8	4.1\\
3.8	4.2\\
3.8	4.3\\
3.8	4.4\\
3.8	4.5\\
3.8	4.6\\
3.8	4.7\\
3.8	4.8\\
3.8	4.9\\
3.8	5\\
3.8	5.1\\
3.8	5.2\\
3.8	5.3\\
3.8	5.4\\
3.8	5.5\\
3.8	5.6\\
3.8	5.7\\
3.8	5.8\\
3.8	5.9\\
};
\addplot [color=black, draw=none, mark size=0.2pt, mark=*, mark options={solid, black}, forget plot]
  table[row sep=crcr]{%
3.9	0.1\\
3.9	0.2\\
3.9	0.3\\
3.9	0.4\\
3.9	0.5\\
3.9	0.6\\
3.9	0.7\\
3.9	0.8\\
3.9	0.9\\
3.9	1\\
3.9	1.1\\
3.9	1.2\\
3.9	1.3\\
3.9	1.4\\
3.9	1.5\\
3.9	1.6\\
3.9	1.7\\
3.9	1.8\\
3.9	1.9\\
3.9	2\\
3.9	2.1\\
3.9	2.2\\
3.9	2.3\\
3.9	2.4\\
3.9	2.5\\
3.9	2.6\\
3.9	2.7\\
3.9	2.8\\
3.9	2.9\\
3.9	3\\
3.9	3.1\\
3.9	3.2\\
3.9	3.3\\
3.9	3.4\\
3.9	3.5\\
3.9	3.6\\
3.9	3.7\\
3.9	3.8\\
3.9	3.9\\
3.9	4\\
3.9	4.1\\
3.9	4.2\\
3.9	4.3\\
3.9	4.4\\
3.9	4.5\\
3.9	4.6\\
3.9	4.7\\
3.9	4.8\\
3.9	4.9\\
3.9	5\\
3.9	5.1\\
3.9	5.2\\
3.9	5.3\\
3.9	5.4\\
3.9	5.5\\
3.9	5.6\\
3.9	5.7\\
3.9	5.8\\
3.9	5.9\\
};
\addplot [color=black, draw=none, mark size=0.2pt, mark=*, mark options={solid, black}, forget plot]
  table[row sep=crcr]{%
4	0.1\\
4	0.2\\
4	0.3\\
4	0.4\\
4	0.5\\
4	0.6\\
4	0.7\\
4	0.8\\
4	0.9\\
4	1\\
4	1.1\\
4	1.2\\
4	1.3\\
4	1.4\\
4	1.5\\
4	1.6\\
4	1.7\\
4	1.8\\
4	1.9\\
4	2\\
4	2.1\\
4	2.2\\
4	2.3\\
4	2.4\\
4	2.5\\
4	2.6\\
4	2.7\\
4	2.8\\
4	2.9\\
4	3\\
4	3.1\\
4	3.2\\
4	3.3\\
4	3.4\\
4	3.5\\
4	3.6\\
4	3.7\\
4	3.8\\
4	3.9\\
4	4\\
4	4.1\\
4	4.2\\
4	4.3\\
4	4.4\\
4	4.5\\
4	4.6\\
4	4.7\\
4	4.8\\
4	4.9\\
4	5\\
4	5.1\\
4	5.2\\
4	5.3\\
4	5.4\\
4	5.5\\
4	5.6\\
4	5.7\\
4	5.8\\
4	5.9\\
};
\addplot [color=black, draw=none, mark size=0.2pt, mark=*, mark options={solid, black}, forget plot]
  table[row sep=crcr]{%
4.1	0.1\\
4.1	0.2\\
4.1	0.3\\
4.1	0.4\\
4.1	0.5\\
4.1	0.6\\
4.1	0.7\\
4.1	0.8\\
4.1	0.9\\
4.1	1\\
4.1	1.1\\
4.1	1.2\\
4.1	1.3\\
4.1	1.4\\
4.1	1.5\\
4.1	1.6\\
4.1	1.7\\
4.1	1.8\\
4.1	1.9\\
4.1	2\\
4.1	2.1\\
4.1	2.2\\
4.1	2.3\\
4.1	2.4\\
4.1	2.5\\
4.1	2.6\\
4.1	2.7\\
4.1	2.8\\
4.1	2.9\\
4.1	3\\
4.1	3.1\\
4.1	3.2\\
4.1	3.3\\
4.1	3.4\\
4.1	3.5\\
4.1	3.6\\
4.1	3.7\\
4.1	3.8\\
4.1	3.9\\
4.1	4\\
4.1	4.1\\
4.1	4.2\\
4.1	4.3\\
4.1	4.4\\
4.1	4.5\\
4.1	4.6\\
4.1	4.7\\
4.1	4.8\\
4.1	4.9\\
4.1	5\\
4.1	5.1\\
4.1	5.2\\
4.1	5.3\\
4.1	5.4\\
4.1	5.5\\
4.1	5.6\\
4.1	5.7\\
4.1	5.8\\
4.1	5.9\\
};
\addplot [color=black, draw=none, mark size=0.2pt, mark=*, mark options={solid, black}, forget plot]
  table[row sep=crcr]{%
4.2	0.1\\
4.2	0.2\\
4.2	0.3\\
4.2	0.4\\
4.2	0.5\\
4.2	0.6\\
4.2	0.7\\
4.2	0.8\\
4.2	0.9\\
4.2	1\\
4.2	1.1\\
4.2	1.2\\
4.2	1.3\\
4.2	1.4\\
4.2	1.5\\
4.2	1.6\\
4.2	1.7\\
4.2	1.8\\
4.2	1.9\\
4.2	2\\
4.2	2.1\\
4.2	2.2\\
4.2	2.3\\
4.2	2.4\\
4.2	2.5\\
4.2	2.6\\
4.2	2.7\\
4.2	2.8\\
4.2	2.9\\
4.2	3\\
4.2	3.1\\
4.2	3.2\\
4.2	3.3\\
4.2	3.4\\
4.2	3.5\\
4.2	3.6\\
4.2	3.7\\
4.2	3.8\\
4.2	3.9\\
4.2	4\\
4.2	4.1\\
4.2	4.2\\
4.2	4.3\\
4.2	4.4\\
4.2	4.5\\
4.2	4.6\\
4.2	4.7\\
4.2	4.8\\
4.2	4.9\\
4.2	5\\
4.2	5.1\\
4.2	5.2\\
4.2	5.3\\
4.2	5.4\\
4.2	5.5\\
4.2	5.6\\
4.2	5.7\\
4.2	5.8\\
4.2	5.9\\
};
\addplot [color=black, draw=none, mark size=0.2pt, mark=*, mark options={solid, black}, forget plot]
  table[row sep=crcr]{%
4.3	0.1\\
4.3	0.2\\
4.3	0.3\\
4.3	0.4\\
4.3	0.5\\
4.3	0.6\\
4.3	0.7\\
4.3	0.8\\
4.3	0.9\\
4.3	1\\
4.3	1.1\\
4.3	1.2\\
4.3	1.3\\
4.3	1.4\\
4.3	1.5\\
4.3	1.6\\
4.3	1.7\\
4.3	1.8\\
4.3	1.9\\
4.3	2\\
4.3	2.1\\
4.3	2.2\\
4.3	2.3\\
4.3	2.4\\
4.3	2.5\\
4.3	2.6\\
4.3	2.7\\
4.3	2.8\\
4.3	2.9\\
4.3	3\\
4.3	3.1\\
4.3	3.2\\
4.3	3.3\\
4.3	3.4\\
4.3	3.5\\
4.3	3.6\\
4.3	3.7\\
4.3	3.8\\
4.3	3.9\\
4.3	4\\
4.3	4.1\\
4.3	4.2\\
4.3	4.3\\
4.3	4.4\\
4.3	4.5\\
4.3	4.6\\
4.3	4.7\\
4.3	4.8\\
4.3	4.9\\
4.3	5\\
4.3	5.1\\
4.3	5.2\\
4.3	5.3\\
4.3	5.4\\
4.3	5.5\\
4.3	5.6\\
4.3	5.7\\
4.3	5.8\\
4.3	5.9\\
};
\addplot [color=black, draw=none, mark size=0.2pt, mark=*, mark options={solid, black}, forget plot]
  table[row sep=crcr]{%
4.4	0.1\\
4.4	0.2\\
4.4	0.3\\
4.4	0.4\\
4.4	0.5\\
4.4	0.6\\
4.4	0.7\\
4.4	0.8\\
4.4	0.9\\
4.4	1\\
4.4	1.1\\
4.4	1.2\\
4.4	1.3\\
4.4	1.4\\
4.4	1.5\\
4.4	1.6\\
4.4	1.7\\
4.4	1.8\\
4.4	1.9\\
4.4	2\\
4.4	2.1\\
4.4	2.2\\
4.4	2.3\\
4.4	2.4\\
4.4	2.5\\
4.4	2.6\\
4.4	2.7\\
4.4	2.8\\
4.4	2.9\\
4.4	3\\
4.4	3.1\\
4.4	3.2\\
4.4	3.3\\
4.4	3.4\\
4.4	3.5\\
4.4	3.6\\
4.4	3.7\\
4.4	3.8\\
4.4	3.9\\
4.4	4\\
4.4	4.1\\
4.4	4.2\\
4.4	4.3\\
4.4	4.4\\
4.4	4.5\\
4.4	4.6\\
4.4	4.7\\
4.4	4.8\\
4.4	4.9\\
4.4	5\\
4.4	5.1\\
4.4	5.2\\
4.4	5.3\\
4.4	5.4\\
4.4	5.5\\
4.4	5.6\\
4.4	5.7\\
4.4	5.8\\
4.4	5.9\\
};
\addplot [color=black, draw=none, mark size=0.2pt, mark=*, mark options={solid, black}, forget plot]
  table[row sep=crcr]{%
4.5	0.1\\
4.5	0.2\\
4.5	0.3\\
4.5	0.4\\
4.5	0.5\\
4.5	0.6\\
4.5	0.7\\
4.5	0.8\\
4.5	0.9\\
4.5	1\\
4.5	1.1\\
4.5	1.2\\
4.5	1.3\\
4.5	1.4\\
4.5	1.5\\
4.5	1.6\\
4.5	1.7\\
4.5	1.8\\
4.5	1.9\\
4.5	2\\
4.5	2.1\\
4.5	2.2\\
4.5	2.3\\
4.5	2.4\\
4.5	2.5\\
4.5	2.6\\
4.5	2.7\\
4.5	2.8\\
4.5	2.9\\
4.5	3\\
4.5	3.1\\
4.5	3.2\\
4.5	3.3\\
4.5	3.4\\
4.5	3.5\\
4.5	3.6\\
4.5	3.7\\
4.5	3.8\\
4.5	3.9\\
4.5	4\\
4.5	4.1\\
4.5	4.2\\
4.5	4.3\\
4.5	4.4\\
4.5	4.5\\
4.5	4.6\\
4.5	4.7\\
4.5	4.8\\
4.5	4.9\\
4.5	5\\
4.5	5.1\\
4.5	5.2\\
4.5	5.3\\
4.5	5.4\\
4.5	5.5\\
4.5	5.6\\
4.5	5.7\\
4.5	5.8\\
4.5	5.9\\
};
\addplot [color=black, draw=none, mark size=0.2pt, mark=*, mark options={solid, black}, forget plot]
  table[row sep=crcr]{%
4.6	0.1\\
4.6	0.2\\
4.6	0.3\\
4.6	0.4\\
4.6	0.5\\
4.6	0.6\\
4.6	0.7\\
4.6	0.8\\
4.6	0.9\\
4.6	1\\
4.6	1.1\\
4.6	1.2\\
4.6	1.3\\
4.6	1.4\\
4.6	1.5\\
4.6	1.6\\
4.6	1.7\\
4.6	1.8\\
4.6	1.9\\
4.6	2\\
4.6	2.1\\
4.6	2.2\\
4.6	2.3\\
4.6	2.4\\
4.6	2.5\\
4.6	2.6\\
4.6	2.7\\
4.6	2.8\\
4.6	2.9\\
4.6	3\\
4.6	3.1\\
4.6	3.2\\
4.6	3.3\\
4.6	3.4\\
4.6	3.5\\
4.6	3.6\\
4.6	3.7\\
4.6	3.8\\
4.6	3.9\\
4.6	4\\
4.6	4.1\\
4.6	4.2\\
4.6	4.3\\
4.6	4.4\\
4.6	4.5\\
4.6	4.6\\
4.6	4.7\\
4.6	4.8\\
4.6	4.9\\
4.6	5\\
4.6	5.1\\
4.6	5.2\\
4.6	5.3\\
4.6	5.4\\
4.6	5.5\\
4.6	5.6\\
4.6	5.7\\
4.6	5.8\\
4.6	5.9\\
};
\addplot [color=black, draw=none, mark size=0.2pt, mark=*, mark options={solid, black}, forget plot]
  table[row sep=crcr]{%
4.7	0.1\\
4.7	0.2\\
4.7	0.3\\
4.7	0.4\\
4.7	0.5\\
4.7	0.6\\
4.7	0.7\\
4.7	0.8\\
4.7	0.9\\
4.7	1\\
4.7	1.1\\
4.7	1.2\\
4.7	1.3\\
4.7	1.4\\
4.7	1.5\\
4.7	1.6\\
4.7	1.7\\
4.7	1.8\\
4.7	1.9\\
4.7	2\\
4.7	2.1\\
4.7	2.2\\
4.7	2.3\\
4.7	2.4\\
4.7	2.5\\
4.7	2.6\\
4.7	2.7\\
4.7	2.8\\
4.7	2.9\\
4.7	3\\
4.7	3.1\\
4.7	3.2\\
4.7	3.3\\
4.7	3.4\\
4.7	3.5\\
4.7	3.6\\
4.7	3.7\\
4.7	3.8\\
4.7	3.9\\
4.7	4\\
4.7	4.1\\
4.7	4.2\\
4.7	4.3\\
4.7	4.4\\
4.7	4.5\\
4.7	4.6\\
4.7	4.7\\
4.7	4.8\\
4.7	4.9\\
4.7	5\\
4.7	5.1\\
4.7	5.2\\
4.7	5.3\\
4.7	5.4\\
4.7	5.5\\
4.7	5.6\\
4.7	5.7\\
4.7	5.8\\
4.7	5.9\\
};
\addplot [color=black, draw=none, mark size=0.2pt, mark=*, mark options={solid, black}, forget plot]
  table[row sep=crcr]{%
4.8	0.1\\
4.8	0.2\\
4.8	0.3\\
4.8	0.4\\
4.8	0.5\\
4.8	0.6\\
4.8	0.7\\
4.8	0.8\\
4.8	0.9\\
4.8	1\\
4.8	1.1\\
4.8	1.2\\
4.8	1.3\\
4.8	1.4\\
4.8	1.5\\
4.8	1.6\\
4.8	1.7\\
4.8	1.8\\
4.8	1.9\\
4.8	2\\
4.8	2.1\\
4.8	2.2\\
4.8	2.3\\
4.8	2.4\\
4.8	2.5\\
4.8	2.6\\
4.8	2.7\\
4.8	2.8\\
4.8	2.9\\
4.8	3\\
4.8	3.1\\
4.8	3.2\\
4.8	3.3\\
4.8	3.4\\
4.8	3.5\\
4.8	3.6\\
4.8	3.7\\
4.8	3.8\\
4.8	3.9\\
4.8	4\\
4.8	4.1\\
4.8	4.2\\
4.8	4.3\\
4.8	4.4\\
4.8	4.5\\
4.8	4.6\\
4.8	4.7\\
4.8	4.8\\
4.8	4.9\\
4.8	5\\
4.8	5.1\\
4.8	5.2\\
4.8	5.3\\
4.8	5.4\\
4.8	5.5\\
4.8	5.6\\
4.8	5.7\\
4.8	5.8\\
4.8	5.9\\
};
\addplot [color=black, draw=none, mark size=0.2pt, mark=*, mark options={solid, black}, forget plot]
  table[row sep=crcr]{%
4.9	0.1\\
4.9	0.2\\
4.9	0.3\\
4.9	0.4\\
4.9	0.5\\
4.9	0.6\\
4.9	0.7\\
4.9	0.8\\
4.9	0.9\\
4.9	1\\
4.9	1.1\\
4.9	1.2\\
4.9	1.3\\
4.9	1.4\\
4.9	1.5\\
4.9	1.6\\
4.9	1.7\\
4.9	1.8\\
4.9	1.9\\
4.9	2\\
4.9	2.1\\
4.9	2.2\\
4.9	2.3\\
4.9	2.4\\
4.9	2.5\\
4.9	2.6\\
4.9	2.7\\
4.9	2.8\\
4.9	2.9\\
4.9	3\\
4.9	3.1\\
4.9	3.2\\
4.9	3.3\\
4.9	3.4\\
4.9	3.5\\
4.9	3.6\\
4.9	3.7\\
4.9	3.8\\
4.9	3.9\\
4.9	4\\
4.9	4.1\\
4.9	4.2\\
4.9	4.3\\
4.9	4.4\\
4.9	4.5\\
4.9	4.6\\
4.9	4.7\\
4.9	4.8\\
4.9	4.9\\
4.9	5\\
4.9	5.1\\
4.9	5.2\\
4.9	5.3\\
4.9	5.4\\
4.9	5.5\\
4.9	5.6\\
4.9	5.7\\
4.9	5.8\\
4.9	5.9\\
};
\addplot [color=black, draw=none, mark size=0.2pt, mark=*, mark options={solid, black}, forget plot]
  table[row sep=crcr]{%
5	0.1\\
5	0.2\\
5	0.3\\
5	0.4\\
5	0.5\\
5	0.6\\
5	0.7\\
5	0.8\\
5	0.9\\
5	1\\
5	1.1\\
5	1.2\\
5	1.3\\
5	1.4\\
5	1.5\\
5	1.6\\
5	1.7\\
5	1.8\\
5	1.9\\
5	2\\
5	2.1\\
5	2.2\\
5	2.3\\
5	2.4\\
5	2.5\\
5	2.6\\
5	2.7\\
5	2.8\\
5	2.9\\
5	3\\
5	3.1\\
5	3.2\\
5	3.3\\
5	3.4\\
5	3.5\\
5	3.6\\
5	3.7\\
5	3.8\\
5	3.9\\
5	4\\
5	4.1\\
5	4.2\\
5	4.3\\
5	4.4\\
5	4.5\\
5	4.6\\
5	4.7\\
5	4.8\\
5	4.9\\
5	5\\
5	5.1\\
5	5.2\\
5	5.3\\
5	5.4\\
5	5.5\\
5	5.6\\
5	5.7\\
5	5.8\\
5	5.9\\
};
\addplot [color=black, draw=none, mark size=0.2pt, mark=*, mark options={solid, black}, forget plot]
  table[row sep=crcr]{%
5.1	0.1\\
5.1	0.2\\
5.1	0.3\\
5.1	0.4\\
5.1	0.5\\
5.1	0.6\\
5.1	0.7\\
5.1	0.8\\
5.1	0.9\\
5.1	1\\
5.1	1.1\\
5.1	1.2\\
5.1	1.3\\
5.1	1.4\\
5.1	1.5\\
5.1	1.6\\
5.1	1.7\\
5.1	1.8\\
5.1	1.9\\
5.1	2\\
5.1	2.1\\
5.1	2.2\\
5.1	2.3\\
5.1	2.4\\
5.1	2.5\\
5.1	2.6\\
5.1	2.7\\
5.1	2.8\\
5.1	2.9\\
5.1	3\\
5.1	3.1\\
5.1	3.2\\
5.1	3.3\\
5.1	3.4\\
5.1	3.5\\
5.1	3.6\\
5.1	3.7\\
5.1	3.8\\
5.1	3.9\\
5.1	4\\
5.1	4.1\\
5.1	4.2\\
5.1	4.3\\
5.1	4.4\\
5.1	4.5\\
5.1	4.6\\
5.1	4.7\\
5.1	4.8\\
5.1	4.9\\
5.1	5\\
5.1	5.1\\
5.1	5.2\\
5.1	5.3\\
5.1	5.4\\
5.1	5.5\\
5.1	5.6\\
5.1	5.7\\
5.1	5.8\\
5.1	5.9\\
};
\addplot [color=black, draw=none, mark size=0.2pt, mark=*, mark options={solid, black}, forget plot]
  table[row sep=crcr]{%
5.2	0.1\\
5.2	0.2\\
5.2	0.3\\
5.2	0.4\\
5.2	0.5\\
5.2	0.6\\
5.2	0.7\\
5.2	0.8\\
5.2	0.9\\
5.2	1\\
5.2	1.1\\
5.2	1.2\\
5.2	1.3\\
5.2	1.4\\
5.2	1.5\\
5.2	1.6\\
5.2	1.7\\
5.2	1.8\\
5.2	1.9\\
5.2	2\\
5.2	2.1\\
5.2	2.2\\
5.2	2.3\\
5.2	2.4\\
5.2	2.5\\
5.2	2.6\\
5.2	2.7\\
5.2	2.8\\
5.2	2.9\\
5.2	3\\
5.2	3.1\\
5.2	3.2\\
5.2	3.3\\
5.2	3.4\\
5.2	3.5\\
5.2	3.6\\
5.2	3.7\\
5.2	3.8\\
5.2	3.9\\
5.2	4\\
5.2	4.1\\
5.2	4.2\\
5.2	4.3\\
5.2	4.4\\
5.2	4.5\\
5.2	4.6\\
5.2	4.7\\
5.2	4.8\\
5.2	4.9\\
5.2	5\\
5.2	5.1\\
5.2	5.2\\
5.2	5.3\\
5.2	5.4\\
5.2	5.5\\
5.2	5.6\\
5.2	5.7\\
5.2	5.8\\
5.2	5.9\\
};
\addplot [color=black, draw=none, mark size=0.2pt, mark=*, mark options={solid, black}, forget plot]
  table[row sep=crcr]{%
5.3	0.1\\
5.3	0.2\\
5.3	0.3\\
5.3	0.4\\
5.3	0.5\\
5.3	0.6\\
5.3	0.7\\
5.3	0.8\\
5.3	0.9\\
5.3	1\\
5.3	1.1\\
5.3	1.2\\
5.3	1.3\\
5.3	1.4\\
5.3	1.5\\
5.3	1.6\\
5.3	1.7\\
5.3	1.8\\
5.3	1.9\\
5.3	2\\
5.3	2.1\\
5.3	2.2\\
5.3	2.3\\
5.3	2.4\\
5.3	2.5\\
5.3	2.6\\
5.3	2.7\\
5.3	2.8\\
5.3	2.9\\
5.3	3\\
5.3	3.1\\
5.3	3.2\\
5.3	3.3\\
5.3	3.4\\
5.3	3.5\\
5.3	3.6\\
5.3	3.7\\
5.3	3.8\\
5.3	3.9\\
5.3	4\\
5.3	4.1\\
5.3	4.2\\
5.3	4.3\\
5.3	4.4\\
5.3	4.5\\
5.3	4.6\\
5.3	4.7\\
5.3	4.8\\
5.3	4.9\\
5.3	5\\
5.3	5.1\\
5.3	5.2\\
5.3	5.3\\
5.3	5.4\\
5.3	5.5\\
5.3	5.6\\
5.3	5.7\\
5.3	5.8\\
5.3	5.9\\
};
\addplot [color=black, draw=none, mark size=0.2pt, mark=*, mark options={solid, black}, forget plot]
  table[row sep=crcr]{%
5.4	0.1\\
5.4	0.2\\
5.4	0.3\\
5.4	0.4\\
5.4	0.5\\
5.4	0.6\\
5.4	0.7\\
5.4	0.8\\
5.4	0.9\\
5.4	1\\
5.4	1.1\\
5.4	1.2\\
5.4	1.3\\
5.4	1.4\\
5.4	1.5\\
5.4	1.6\\
5.4	1.7\\
5.4	1.8\\
5.4	1.9\\
5.4	2\\
5.4	2.1\\
5.4	2.2\\
5.4	2.3\\
5.4	2.4\\
5.4	2.5\\
5.4	2.6\\
5.4	2.7\\
5.4	2.8\\
5.4	2.9\\
5.4	3\\
5.4	3.1\\
5.4	3.2\\
5.4	3.3\\
5.4	3.4\\
5.4	3.5\\
5.4	3.6\\
5.4	3.7\\
5.4	3.8\\
5.4	3.9\\
5.4	4\\
5.4	4.1\\
5.4	4.2\\
5.4	4.3\\
5.4	4.4\\
5.4	4.5\\
5.4	4.6\\
5.4	4.7\\
5.4	4.8\\
5.4	4.9\\
5.4	5\\
5.4	5.1\\
5.4	5.2\\
5.4	5.3\\
5.4	5.4\\
5.4	5.5\\
5.4	5.6\\
5.4	5.7\\
5.4	5.8\\
5.4	5.9\\
};
\addplot [color=black, draw=none, mark size=0.2pt, mark=*, mark options={solid, black}, forget plot]
  table[row sep=crcr]{%
5.5	0.1\\
5.5	0.2\\
5.5	0.3\\
5.5	0.4\\
5.5	0.5\\
5.5	0.6\\
5.5	0.7\\
5.5	0.8\\
5.5	0.9\\
5.5	1\\
5.5	1.1\\
5.5	1.2\\
5.5	1.3\\
5.5	1.4\\
5.5	1.5\\
5.5	1.6\\
5.5	1.7\\
5.5	1.8\\
5.5	1.9\\
5.5	2\\
5.5	2.1\\
5.5	2.2\\
5.5	2.3\\
5.5	2.4\\
5.5	2.5\\
5.5	2.6\\
5.5	2.7\\
5.5	2.8\\
5.5	2.9\\
5.5	3\\
5.5	3.1\\
5.5	3.2\\
5.5	3.3\\
5.5	3.4\\
5.5	3.5\\
5.5	3.6\\
5.5	3.7\\
5.5	3.8\\
5.5	3.9\\
5.5	4\\
5.5	4.1\\
5.5	4.2\\
5.5	4.3\\
5.5	4.4\\
5.5	4.5\\
5.5	4.6\\
5.5	4.7\\
5.5	4.8\\
5.5	4.9\\
5.5	5\\
5.5	5.1\\
5.5	5.2\\
5.5	5.3\\
5.5	5.4\\
5.5	5.5\\
5.5	5.6\\
5.5	5.7\\
5.5	5.8\\
5.5	5.9\\
};
\addplot [color=black, draw=none, mark size=0.2pt, mark=*, mark options={solid, black}, forget plot]
  table[row sep=crcr]{%
5.6	0.1\\
5.6	0.2\\
5.6	0.3\\
5.6	0.4\\
5.6	0.5\\
5.6	0.6\\
5.6	0.7\\
5.6	0.8\\
5.6	0.9\\
5.6	1\\
5.6	1.1\\
5.6	1.2\\
5.6	1.3\\
5.6	1.4\\
5.6	1.5\\
5.6	1.6\\
5.6	1.7\\
5.6	1.8\\
5.6	1.9\\
5.6	2\\
5.6	2.1\\
5.6	2.2\\
5.6	2.3\\
5.6	2.4\\
5.6	2.5\\
5.6	2.6\\
5.6	2.7\\
5.6	2.8\\
5.6	2.9\\
5.6	3\\
5.6	3.1\\
5.6	3.2\\
5.6	3.3\\
5.6	3.4\\
5.6	3.5\\
5.6	3.6\\
5.6	3.7\\
5.6	3.8\\
5.6	3.9\\
5.6	4\\
5.6	4.1\\
5.6	4.2\\
5.6	4.3\\
5.6	4.4\\
5.6	4.5\\
5.6	4.6\\
5.6	4.7\\
5.6	4.8\\
5.6	4.9\\
5.6	5\\
5.6	5.1\\
5.6	5.2\\
5.6	5.3\\
5.6	5.4\\
5.6	5.5\\
5.6	5.6\\
5.6	5.7\\
5.6	5.8\\
5.6	5.9\\
};
\addplot [color=black, draw=none, mark size=0.2pt, mark=*, mark options={solid, black}, forget plot]
  table[row sep=crcr]{%
5.7	0.1\\
5.7	0.2\\
5.7	0.3\\
5.7	0.4\\
5.7	0.5\\
5.7	0.6\\
5.7	0.7\\
5.7	0.8\\
5.7	0.9\\
5.7	1\\
5.7	1.1\\
5.7	1.2\\
5.7	1.3\\
5.7	1.4\\
5.7	1.5\\
5.7	1.6\\
5.7	1.7\\
5.7	1.8\\
5.7	1.9\\
5.7	2\\
5.7	2.1\\
5.7	2.2\\
5.7	2.3\\
5.7	2.4\\
5.7	2.5\\
5.7	2.6\\
5.7	2.7\\
5.7	2.8\\
5.7	2.9\\
5.7	3\\
5.7	3.1\\
5.7	3.2\\
5.7	3.3\\
5.7	3.4\\
5.7	3.5\\
5.7	3.6\\
5.7	3.7\\
5.7	3.8\\
5.7	3.9\\
5.7	4\\
5.7	4.1\\
5.7	4.2\\
5.7	4.3\\
5.7	4.4\\
5.7	4.5\\
5.7	4.6\\
5.7	4.7\\
5.7	4.8\\
5.7	4.9\\
5.7	5\\
5.7	5.1\\
5.7	5.2\\
5.7	5.3\\
5.7	5.4\\
5.7	5.5\\
5.7	5.6\\
5.7	5.7\\
5.7	5.8\\
5.7	5.9\\
};
\addplot [color=black, draw=none, mark size=0.2pt, mark=*, mark options={solid, black}, forget plot]
  table[row sep=crcr]{%
5.8	0.1\\
5.8	0.2\\
5.8	0.3\\
5.8	0.4\\
5.8	0.5\\
5.8	0.6\\
5.8	0.7\\
5.8	0.8\\
5.8	0.9\\
5.8	1\\
5.8	1.1\\
5.8	1.2\\
5.8	1.3\\
5.8	1.4\\
5.8	1.5\\
5.8	1.6\\
5.8	1.7\\
5.8	1.8\\
5.8	1.9\\
5.8	2\\
5.8	2.1\\
5.8	2.2\\
5.8	2.3\\
5.8	2.4\\
5.8	2.5\\
5.8	2.6\\
5.8	2.7\\
5.8	2.8\\
5.8	2.9\\
5.8	3\\
5.8	3.1\\
5.8	3.2\\
5.8	3.3\\
5.8	3.4\\
5.8	3.5\\
5.8	3.6\\
5.8	3.7\\
5.8	3.8\\
5.8	3.9\\
5.8	4\\
5.8	4.1\\
5.8	4.2\\
5.8	4.3\\
5.8	4.4\\
5.8	4.5\\
5.8	4.6\\
5.8	4.7\\
5.8	4.8\\
5.8	4.9\\
5.8	5\\
5.8	5.1\\
5.8	5.2\\
5.8	5.3\\
5.8	5.4\\
5.8	5.5\\
5.8	5.6\\
5.8	5.7\\
5.8	5.8\\
5.8	5.9\\
};
\addplot [color=black, draw=none, mark size=0.2pt, mark=*, mark options={solid, black}, forget plot]
  table[row sep=crcr]{%
5.9	0.1\\
5.9	0.2\\
5.9	0.3\\
5.9	0.4\\
5.9	0.5\\
5.9	0.6\\
5.9	0.7\\
5.9	0.8\\
5.9	0.9\\
5.9	1\\
5.9	1.1\\
5.9	1.2\\
5.9	1.3\\
5.9	1.4\\
5.9	1.5\\
5.9	1.6\\
5.9	1.7\\
5.9	1.8\\
5.9	1.9\\
5.9	2\\
5.9	2.1\\
5.9	2.2\\
5.9	2.3\\
5.9	2.4\\
5.9	2.5\\
5.9	2.6\\
5.9	2.7\\
5.9	2.8\\
5.9	2.9\\
5.9	3\\
5.9	3.1\\
5.9	3.2\\
5.9	3.3\\
5.9	3.4\\
5.9	3.5\\
5.9	3.6\\
5.9	3.7\\
5.9	3.8\\
5.9	3.9\\
5.9	4\\
5.9	4.1\\
5.9	4.2\\
5.9	4.3\\
5.9	4.4\\
5.9	4.5\\
5.9	4.6\\
5.9	4.7\\
5.9	4.8\\
5.9	4.9\\
5.9	5\\
5.9	5.1\\
5.9	5.2\\
5.9	5.3\\
5.9	5.4\\
5.9	5.5\\
5.9	5.6\\
5.9	5.7\\
5.9	5.8\\
5.9	5.9\\
};
\addplot [color=white, draw=none, mark size=0.2pt, mark=*, mark options={solid, white}, forget plot]
  table[row sep=crcr]{%
1.2	1.2\\
1.2	1.3\\
1.2	1.4\\
1.2	1.5\\
1.2	1.6\\
1.2	1.7\\
1.2	1.8\\
1.2	1.9\\
1.2	2\\
1.2	2.1\\
1.2	2.2\\
1.2	2.3\\
1.2	2.4\\
1.2	2.5\\
1.2	2.6\\
1.2	2.7\\
1.2	2.8\\
1.2	2.9\\
1.2	3\\
1.2	3.1\\
1.2	3.2\\
1.2	3.3\\
1.2	3.4\\
1.2	3.5\\
1.2	3.6\\
1.2	3.7\\
1.2	3.8\\
1.2	3.9\\
1.2	4\\
1.2	4.1\\
1.2	4.2\\
1.2	4.3\\
1.2	4.4\\
1.2	4.5\\
1.2	4.6\\
1.2	4.7\\
1.2	4.8\\
};
\addplot [color=white, draw=none, mark size=0.2pt, mark=*, mark options={solid, white}, forget plot]
  table[row sep=crcr]{%
1.3	1.2\\
1.3	1.3\\
1.3	1.4\\
1.3	1.5\\
1.3	1.6\\
1.3	1.7\\
1.3	1.8\\
1.3	1.9\\
1.3	2\\
1.3	2.1\\
1.3	2.2\\
1.3	2.3\\
1.3	2.4\\
1.3	2.5\\
1.3	2.6\\
1.3	2.7\\
1.3	2.8\\
1.3	2.9\\
1.3	3\\
1.3	3.1\\
1.3	3.2\\
1.3	3.3\\
1.3	3.4\\
1.3	3.5\\
1.3	3.6\\
1.3	3.7\\
1.3	3.8\\
1.3	3.9\\
1.3	4\\
1.3	4.1\\
1.3	4.2\\
1.3	4.3\\
1.3	4.4\\
1.3	4.5\\
1.3	4.6\\
1.3	4.7\\
1.3	4.8\\
};
\addplot [color=white, draw=none, mark size=0.2pt, mark=*, mark options={solid, white}, forget plot]
  table[row sep=crcr]{%
1.4	1.2\\
1.4	1.3\\
1.4	1.4\\
1.4	1.5\\
1.4	1.6\\
1.4	1.7\\
1.4	1.8\\
1.4	1.9\\
1.4	2\\
1.4	2.1\\
1.4	2.2\\
1.4	2.3\\
1.4	2.4\\
1.4	2.5\\
1.4	2.6\\
1.4	2.7\\
1.4	2.8\\
1.4	2.9\\
1.4	3\\
1.4	3.1\\
1.4	3.2\\
1.4	3.3\\
1.4	3.4\\
1.4	3.5\\
1.4	3.6\\
1.4	3.7\\
1.4	3.8\\
1.4	3.9\\
1.4	4\\
1.4	4.1\\
1.4	4.2\\
1.4	4.3\\
1.4	4.4\\
1.4	4.5\\
1.4	4.6\\
1.4	4.7\\
1.4	4.8\\
};
\addplot [color=white, draw=none, mark size=0.2pt, mark=*, mark options={solid, white}, forget plot]
  table[row sep=crcr]{%
1.5	1.2\\
1.5	1.3\\
1.5	1.4\\
1.5	1.5\\
1.5	1.6\\
1.5	1.7\\
1.5	1.8\\
1.5	1.9\\
1.5	2\\
1.5	2.1\\
1.5	2.2\\
1.5	2.3\\
1.5	2.4\\
1.5	2.5\\
1.5	2.6\\
1.5	2.7\\
1.5	2.8\\
1.5	2.9\\
1.5	3\\
1.5	3.1\\
1.5	3.2\\
1.5	3.3\\
1.5	3.4\\
1.5	3.5\\
1.5	3.6\\
1.5	3.7\\
1.5	3.8\\
1.5	3.9\\
1.5	4\\
1.5	4.1\\
1.5	4.2\\
1.5	4.3\\
1.5	4.4\\
1.5	4.5\\
1.5	4.6\\
1.5	4.7\\
1.5	4.8\\
};
\addplot [color=white, draw=none, mark size=0.2pt, mark=*, mark options={solid, white}, forget plot]
  table[row sep=crcr]{%
1.6	1.2\\
1.6	1.3\\
1.6	1.4\\
1.6	1.5\\
1.6	1.6\\
1.6	1.7\\
1.6	1.8\\
1.6	1.9\\
1.6	2\\
1.6	2.1\\
1.6	2.2\\
1.6	2.3\\
1.6	2.4\\
1.6	2.5\\
1.6	2.6\\
1.6	2.7\\
1.6	2.8\\
1.6	2.9\\
1.6	3\\
1.6	3.1\\
1.6	3.2\\
1.6	3.3\\
1.6	3.4\\
1.6	3.5\\
1.6	3.6\\
1.6	3.7\\
1.6	3.8\\
1.6	3.9\\
1.6	4\\
1.6	4.1\\
1.6	4.2\\
1.6	4.3\\
1.6	4.4\\
1.6	4.5\\
1.6	4.6\\
1.6	4.7\\
1.6	4.8\\
};
\addplot [color=white, draw=none, mark size=0.2pt, mark=*, mark options={solid, white}, forget plot]
  table[row sep=crcr]{%
1.7	1.2\\
1.7	1.3\\
1.7	1.4\\
1.7	1.5\\
1.7	1.6\\
1.7	1.7\\
1.7	1.8\\
1.7	1.9\\
1.7	2\\
1.7	2.1\\
1.7	2.2\\
1.7	2.3\\
1.7	2.4\\
1.7	2.5\\
1.7	2.6\\
1.7	2.7\\
1.7	2.8\\
1.7	2.9\\
1.7	3\\
1.7	3.1\\
1.7	3.2\\
1.7	3.3\\
1.7	3.4\\
1.7	3.5\\
1.7	3.6\\
1.7	3.7\\
1.7	3.8\\
1.7	3.9\\
1.7	4\\
1.7	4.1\\
1.7	4.2\\
1.7	4.3\\
1.7	4.4\\
1.7	4.5\\
1.7	4.6\\
1.7	4.7\\
1.7	4.8\\
};
\addplot [color=white, draw=none, mark size=0.2pt, mark=*, mark options={solid, white}, forget plot]
  table[row sep=crcr]{%
1.8	1.2\\
1.8	1.3\\
1.8	1.4\\
1.8	1.5\\
1.8	1.6\\
1.8	1.7\\
1.8	1.8\\
1.8	1.9\\
1.8	2\\
1.8	2.1\\
1.8	2.2\\
1.8	2.3\\
1.8	2.4\\
1.8	2.5\\
1.8	2.6\\
1.8	2.7\\
1.8	2.8\\
1.8	2.9\\
1.8	3\\
1.8	3.1\\
1.8	3.2\\
1.8	3.3\\
1.8	3.4\\
1.8	3.5\\
1.8	3.6\\
1.8	3.7\\
1.8	3.8\\
1.8	3.9\\
1.8	4\\
1.8	4.1\\
1.8	4.2\\
1.8	4.3\\
1.8	4.4\\
1.8	4.5\\
1.8	4.6\\
1.8	4.7\\
1.8	4.8\\
};
\addplot [color=white, draw=none, mark size=0.2pt, mark=*, mark options={solid, white}, forget plot]
  table[row sep=crcr]{%
1.9	1.2\\
1.9	1.3\\
1.9	1.4\\
1.9	1.5\\
1.9	1.6\\
1.9	1.7\\
1.9	1.8\\
1.9	1.9\\
1.9	2\\
1.9	2.1\\
1.9	2.2\\
1.9	2.3\\
1.9	2.4\\
1.9	2.5\\
1.9	2.6\\
1.9	2.7\\
1.9	2.8\\
1.9	2.9\\
1.9	3\\
1.9	3.1\\
1.9	3.2\\
1.9	3.3\\
1.9	3.4\\
1.9	3.5\\
1.9	3.6\\
1.9	3.7\\
1.9	3.8\\
1.9	3.9\\
1.9	4\\
1.9	4.1\\
1.9	4.2\\
1.9	4.3\\
1.9	4.4\\
1.9	4.5\\
1.9	4.6\\
1.9	4.7\\
1.9	4.8\\
};
\addplot [color=white, draw=none, mark size=0.2pt, mark=*, mark options={solid, white}, forget plot]
  table[row sep=crcr]{%
2	1.2\\
2	1.3\\
2	1.4\\
2	1.5\\
2	1.6\\
2	1.7\\
2	1.8\\
2	1.9\\
2	2\\
2	2.1\\
2	2.2\\
2	2.3\\
2	2.4\\
2	2.5\\
2	2.6\\
2	2.7\\
2	2.8\\
2	2.9\\
2	3\\
2	3.1\\
2	3.2\\
2	3.3\\
2	3.4\\
2	3.5\\
2	3.6\\
2	3.7\\
2	3.8\\
2	3.9\\
2	4\\
2	4.1\\
2	4.2\\
2	4.3\\
2	4.4\\
2	4.5\\
2	4.6\\
2	4.7\\
2	4.8\\
};
\addplot [color=white, draw=none, mark size=0.2pt, mark=*, mark options={solid, white}, forget plot]
  table[row sep=crcr]{%
2.1	1.2\\
2.1	1.3\\
2.1	1.4\\
2.1	1.5\\
2.1	1.6\\
2.1	1.7\\
2.1	1.8\\
2.1	1.9\\
2.1	2\\
2.1	2.1\\
2.1	2.2\\
2.1	2.3\\
2.1	2.4\\
2.1	2.5\\
2.1	2.6\\
2.1	2.7\\
2.1	2.8\\
2.1	2.9\\
2.1	3\\
2.1	3.1\\
2.1	3.2\\
2.1	3.3\\
2.1	3.4\\
2.1	3.5\\
2.1	3.6\\
2.1	3.7\\
2.1	3.8\\
2.1	3.9\\
2.1	4\\
2.1	4.1\\
2.1	4.2\\
2.1	4.3\\
2.1	4.4\\
2.1	4.5\\
2.1	4.6\\
2.1	4.7\\
2.1	4.8\\
};
\addplot [color=white, draw=none, mark size=0.2pt, mark=*, mark options={solid, white}, forget plot]
  table[row sep=crcr]{%
2.2	1.2\\
2.2	1.3\\
2.2	1.4\\
2.2	1.5\\
2.2	1.6\\
2.2	1.7\\
2.2	1.8\\
2.2	1.9\\
2.2	2\\
2.2	2.1\\
2.2	2.2\\
2.2	2.3\\
2.2	2.4\\
2.2	2.5\\
2.2	2.6\\
2.2	2.7\\
2.2	2.8\\
2.2	2.9\\
2.2	3\\
2.2	3.1\\
2.2	3.2\\
2.2	3.3\\
2.2	3.4\\
2.2	3.5\\
2.2	3.6\\
2.2	3.7\\
2.2	3.8\\
2.2	3.9\\
2.2	4\\
2.2	4.1\\
2.2	4.2\\
2.2	4.3\\
2.2	4.4\\
2.2	4.5\\
2.2	4.6\\
2.2	4.7\\
2.2	4.8\\
};
\addplot [color=white, draw=none, mark size=0.2pt, mark=*, mark options={solid, white}, forget plot]
  table[row sep=crcr]{%
2.3	1.2\\
2.3	1.3\\
2.3	1.4\\
2.3	1.5\\
2.3	1.6\\
2.3	1.7\\
2.3	1.8\\
2.3	1.9\\
2.3	2\\
2.3	2.1\\
2.3	2.2\\
2.3	2.3\\
2.3	2.4\\
2.3	2.5\\
2.3	2.6\\
2.3	2.7\\
2.3	2.8\\
2.3	2.9\\
2.3	3\\
2.3	3.1\\
2.3	3.2\\
2.3	3.3\\
2.3	3.4\\
2.3	3.5\\
2.3	3.6\\
2.3	3.7\\
2.3	3.8\\
2.3	3.9\\
2.3	4\\
2.3	4.1\\
2.3	4.2\\
2.3	4.3\\
2.3	4.4\\
2.3	4.5\\
2.3	4.6\\
2.3	4.7\\
2.3	4.8\\
};
\addplot [color=white, draw=none, mark size=0.2pt, mark=*, mark options={solid, white}, forget plot]
  table[row sep=crcr]{%
2.4	1.2\\
2.4	1.3\\
2.4	1.4\\
2.4	1.5\\
2.4	1.6\\
2.4	1.7\\
2.4	1.8\\
2.4	1.9\\
2.4	2\\
2.4	2.1\\
2.4	2.2\\
2.4	2.3\\
2.4	2.4\\
2.4	2.5\\
2.4	2.6\\
2.4	2.7\\
2.4	2.8\\
2.4	2.9\\
2.4	3\\
2.4	3.1\\
2.4	3.2\\
2.4	3.3\\
2.4	3.4\\
2.4	3.5\\
2.4	3.6\\
2.4	3.7\\
2.4	3.8\\
2.4	3.9\\
2.4	4\\
2.4	4.1\\
2.4	4.2\\
2.4	4.3\\
2.4	4.4\\
2.4	4.5\\
2.4	4.6\\
2.4	4.7\\
2.4	4.8\\
};
\addplot [color=white, draw=none, mark size=0.2pt, mark=*, mark options={solid, white}, forget plot]
  table[row sep=crcr]{%
2.5	1.2\\
2.5	1.3\\
2.5	1.4\\
2.5	1.5\\
2.5	1.6\\
2.5	1.7\\
2.5	1.8\\
2.5	1.9\\
2.5	2\\
2.5	2.1\\
2.5	2.2\\
2.5	2.3\\
2.5	2.4\\
2.5	2.5\\
2.5	2.6\\
2.5	2.7\\
2.5	2.8\\
2.5	2.9\\
2.5	3\\
2.5	3.1\\
2.5	3.2\\
2.5	3.3\\
2.5	3.4\\
2.5	3.5\\
2.5	3.6\\
2.5	3.7\\
2.5	3.8\\
2.5	3.9\\
2.5	4\\
2.5	4.1\\
2.5	4.2\\
2.5	4.3\\
2.5	4.4\\
2.5	4.5\\
2.5	4.6\\
2.5	4.7\\
2.5	4.8\\
};
\addplot [color=white, draw=none, mark size=0.2pt, mark=*, mark options={solid, white}, forget plot]
  table[row sep=crcr]{%
2.6	1.2\\
2.6	1.3\\
2.6	1.4\\
2.6	1.5\\
2.6	1.6\\
2.6	1.7\\
2.6	1.8\\
2.6	1.9\\
2.6	2\\
2.6	2.1\\
2.6	2.2\\
2.6	2.3\\
2.6	2.4\\
2.6	2.5\\
2.6	2.6\\
2.6	2.7\\
2.6	2.8\\
2.6	2.9\\
2.6	3\\
2.6	3.1\\
2.6	3.2\\
2.6	3.3\\
2.6	3.4\\
2.6	3.5\\
2.6	3.6\\
2.6	3.7\\
2.6	3.8\\
2.6	3.9\\
2.6	4\\
2.6	4.1\\
2.6	4.2\\
2.6	4.3\\
2.6	4.4\\
2.6	4.5\\
2.6	4.6\\
2.6	4.7\\
2.6	4.8\\
};
\addplot [color=white, draw=none, mark size=0.2pt, mark=*, mark options={solid, white}, forget plot]
  table[row sep=crcr]{%
2.7	1.2\\
2.7	1.3\\
2.7	1.4\\
2.7	1.5\\
2.7	1.6\\
2.7	1.7\\
2.7	1.8\\
2.7	1.9\\
2.7	2\\
2.7	2.1\\
2.7	2.2\\
2.7	2.3\\
2.7	2.4\\
2.7	2.5\\
2.7	2.6\\
2.7	2.7\\
2.7	2.8\\
2.7	2.9\\
2.7	3\\
2.7	3.1\\
2.7	3.2\\
2.7	3.3\\
2.7	3.4\\
2.7	3.5\\
2.7	3.6\\
2.7	3.7\\
2.7	3.8\\
2.7	3.9\\
2.7	4\\
2.7	4.1\\
2.7	4.2\\
2.7	4.3\\
2.7	4.4\\
2.7	4.5\\
2.7	4.6\\
2.7	4.7\\
2.7	4.8\\
};
\addplot [color=white, draw=none, mark size=0.2pt, mark=*, mark options={solid, white}, forget plot]
  table[row sep=crcr]{%
2.8	1.2\\
2.8	1.3\\
2.8	1.4\\
2.8	1.5\\
2.8	1.6\\
2.8	1.7\\
2.8	1.8\\
2.8	1.9\\
2.8	2\\
2.8	2.1\\
2.8	2.2\\
2.8	2.3\\
2.8	2.4\\
2.8	2.5\\
2.8	2.6\\
2.8	2.7\\
2.8	2.8\\
2.8	2.9\\
2.8	3\\
2.8	3.1\\
2.8	3.2\\
2.8	3.3\\
2.8	3.4\\
2.8	3.5\\
2.8	3.6\\
2.8	3.7\\
2.8	3.8\\
2.8	3.9\\
2.8	4\\
2.8	4.1\\
2.8	4.2\\
2.8	4.3\\
2.8	4.4\\
2.8	4.5\\
2.8	4.6\\
2.8	4.7\\
2.8	4.8\\
};
\addplot [color=white, draw=none, mark size=0.2pt, mark=*, mark options={solid, white}, forget plot]
  table[row sep=crcr]{%
2.9	1.2\\
2.9	1.3\\
2.9	1.4\\
2.9	1.5\\
2.9	1.6\\
2.9	1.7\\
2.9	1.8\\
2.9	1.9\\
2.9	2\\
2.9	2.1\\
2.9	2.2\\
2.9	2.3\\
2.9	2.4\\
2.9	2.5\\
2.9	2.6\\
2.9	2.7\\
2.9	2.8\\
2.9	2.9\\
2.9	3\\
2.9	3.1\\
2.9	3.2\\
2.9	3.3\\
2.9	3.4\\
2.9	3.5\\
2.9	3.6\\
2.9	3.7\\
2.9	3.8\\
2.9	3.9\\
2.9	4\\
2.9	4.1\\
2.9	4.2\\
2.9	4.3\\
2.9	4.4\\
2.9	4.5\\
2.9	4.6\\
2.9	4.7\\
2.9	4.8\\
};
\addplot [color=white, draw=none, mark size=0.2pt, mark=*, mark options={solid, white}, forget plot]
  table[row sep=crcr]{%
3	1.2\\
3	1.3\\
3	1.4\\
3	1.5\\
3	1.6\\
3	1.7\\
3	1.8\\
3	1.9\\
3	2\\
3	2.1\\
3	2.2\\
3	2.3\\
3	2.4\\
3	2.5\\
3	2.6\\
3	2.7\\
3	2.8\\
3	2.9\\
3	3\\
3	3.1\\
3	3.2\\
3	3.3\\
3	3.4\\
3	3.5\\
3	3.6\\
3	3.7\\
3	3.8\\
3	3.9\\
3	4\\
3	4.1\\
3	4.2\\
3	4.3\\
3	4.4\\
3	4.5\\
3	4.6\\
3	4.7\\
3	4.8\\
};
\addplot [color=white, draw=none, mark size=0.2pt, mark=*, mark options={solid, white}, forget plot]
  table[row sep=crcr]{%
3.1	1.2\\
3.1	1.3\\
3.1	1.4\\
3.1	1.5\\
3.1	1.6\\
3.1	1.7\\
3.1	1.8\\
3.1	1.9\\
3.1	2\\
3.1	2.1\\
3.1	2.2\\
3.1	2.3\\
3.1	2.4\\
3.1	2.5\\
3.1	2.6\\
3.1	2.7\\
3.1	2.8\\
3.1	2.9\\
3.1	3\\
3.1	3.1\\
3.1	3.2\\
3.1	3.3\\
3.1	3.4\\
3.1	3.5\\
3.1	3.6\\
3.1	3.7\\
3.1	3.8\\
3.1	3.9\\
3.1	4\\
3.1	4.1\\
3.1	4.2\\
3.1	4.3\\
3.1	4.4\\
3.1	4.5\\
3.1	4.6\\
3.1	4.7\\
3.1	4.8\\
};
\addplot [color=white, draw=none, mark size=0.2pt, mark=*, mark options={solid, white}, forget plot]
  table[row sep=crcr]{%
3.2	1.2\\
3.2	1.3\\
3.2	1.4\\
3.2	1.5\\
3.2	1.6\\
3.2	1.7\\
3.2	1.8\\
3.2	1.9\\
3.2	2\\
3.2	2.1\\
3.2	2.2\\
3.2	2.3\\
3.2	2.4\\
3.2	2.5\\
3.2	2.6\\
3.2	2.7\\
3.2	2.8\\
3.2	2.9\\
3.2	3\\
3.2	3.1\\
3.2	3.2\\
3.2	3.3\\
3.2	3.4\\
3.2	3.5\\
3.2	3.6\\
3.2	3.7\\
3.2	3.8\\
3.2	3.9\\
3.2	4\\
3.2	4.1\\
3.2	4.2\\
3.2	4.3\\
3.2	4.4\\
3.2	4.5\\
3.2	4.6\\
3.2	4.7\\
3.2	4.8\\
};
\addplot [color=white, draw=none, mark size=0.2pt, mark=*, mark options={solid, white}, forget plot]
  table[row sep=crcr]{%
3.3	1.2\\
3.3	1.3\\
3.3	1.4\\
3.3	1.5\\
3.3	1.6\\
3.3	1.7\\
3.3	1.8\\
3.3	1.9\\
3.3	2\\
3.3	2.1\\
3.3	2.2\\
3.3	2.3\\
3.3	2.4\\
3.3	2.5\\
3.3	2.6\\
3.3	2.7\\
3.3	2.8\\
3.3	2.9\\
3.3	3\\
3.3	3.1\\
3.3	3.2\\
3.3	3.3\\
3.3	3.4\\
3.3	3.5\\
3.3	3.6\\
3.3	3.7\\
3.3	3.8\\
3.3	3.9\\
3.3	4\\
3.3	4.1\\
3.3	4.2\\
3.3	4.3\\
3.3	4.4\\
3.3	4.5\\
3.3	4.6\\
3.3	4.7\\
3.3	4.8\\
};
\addplot [color=white, draw=none, mark size=0.2pt, mark=*, mark options={solid, white}, forget plot]
  table[row sep=crcr]{%
3.4	1.2\\
3.4	1.3\\
3.4	1.4\\
3.4	1.5\\
3.4	1.6\\
3.4	1.7\\
3.4	1.8\\
3.4	1.9\\
3.4	2\\
3.4	2.1\\
3.4	2.2\\
3.4	2.3\\
3.4	2.4\\
3.4	2.5\\
3.4	2.6\\
3.4	2.7\\
3.4	2.8\\
3.4	2.9\\
3.4	3\\
3.4	3.1\\
3.4	3.2\\
3.4	3.3\\
3.4	3.4\\
3.4	3.5\\
3.4	3.6\\
3.4	3.7\\
3.4	3.8\\
3.4	3.9\\
3.4	4\\
3.4	4.1\\
3.4	4.2\\
3.4	4.3\\
3.4	4.4\\
3.4	4.5\\
3.4	4.6\\
3.4	4.7\\
3.4	4.8\\
};
\addplot [color=white, draw=none, mark size=0.2pt, mark=*, mark options={solid, white}, forget plot]
  table[row sep=crcr]{%
3.5	1.2\\
3.5	1.3\\
3.5	1.4\\
3.5	1.5\\
3.5	1.6\\
3.5	1.7\\
3.5	1.8\\
3.5	1.9\\
3.5	2\\
3.5	2.1\\
3.5	2.2\\
3.5	2.3\\
3.5	2.4\\
3.5	2.5\\
3.5	2.6\\
3.5	2.7\\
3.5	2.8\\
3.5	2.9\\
3.5	3\\
3.5	3.1\\
3.5	3.2\\
3.5	3.3\\
3.5	3.4\\
3.5	3.5\\
3.5	3.6\\
3.5	3.7\\
3.5	3.8\\
3.5	3.9\\
3.5	4\\
3.5	4.1\\
3.5	4.2\\
3.5	4.3\\
3.5	4.4\\
3.5	4.5\\
3.5	4.6\\
3.5	4.7\\
3.5	4.8\\
};
\addplot [color=white, draw=none, mark size=0.2pt, mark=*, mark options={solid, white}, forget plot]
  table[row sep=crcr]{%
3.6	1.2\\
3.6	1.3\\
3.6	1.4\\
3.6	1.5\\
3.6	1.6\\
3.6	1.7\\
3.6	1.8\\
3.6	1.9\\
3.6	2\\
3.6	2.1\\
3.6	2.2\\
3.6	2.3\\
3.6	2.4\\
3.6	2.5\\
3.6	2.6\\
3.6	2.7\\
3.6	2.8\\
3.6	2.9\\
3.6	3\\
3.6	3.1\\
3.6	3.2\\
3.6	3.3\\
3.6	3.4\\
3.6	3.5\\
3.6	3.6\\
3.6	3.7\\
3.6	3.8\\
3.6	3.9\\
3.6	4\\
3.6	4.1\\
3.6	4.2\\
3.6	4.3\\
3.6	4.4\\
3.6	4.5\\
3.6	4.6\\
3.6	4.7\\
3.6	4.8\\
};
\addplot [color=white, draw=none, mark size=0.2pt, mark=*, mark options={solid, white}, forget plot]
  table[row sep=crcr]{%
3.7	1.2\\
3.7	1.3\\
3.7	1.4\\
3.7	1.5\\
3.7	1.6\\
3.7	1.7\\
3.7	1.8\\
3.7	1.9\\
3.7	2\\
3.7	2.1\\
3.7	2.2\\
3.7	2.3\\
3.7	2.4\\
3.7	2.5\\
3.7	2.6\\
3.7	2.7\\
3.7	2.8\\
3.7	2.9\\
3.7	3\\
3.7	3.1\\
3.7	3.2\\
3.7	3.3\\
3.7	3.4\\
3.7	3.5\\
3.7	3.6\\
3.7	3.7\\
3.7	3.8\\
3.7	3.9\\
3.7	4\\
3.7	4.1\\
3.7	4.2\\
3.7	4.3\\
3.7	4.4\\
3.7	4.5\\
3.7	4.6\\
3.7	4.7\\
3.7	4.8\\
};
\addplot [color=white, draw=none, mark size=0.2pt, mark=*, mark options={solid, white}, forget plot]
  table[row sep=crcr]{%
3.8	1.2\\
3.8	1.3\\
3.8	1.4\\
3.8	1.5\\
3.8	1.6\\
3.8	1.7\\
3.8	1.8\\
3.8	1.9\\
3.8	2\\
3.8	2.1\\
3.8	2.2\\
3.8	2.3\\
3.8	2.4\\
3.8	2.5\\
3.8	2.6\\
3.8	2.7\\
3.8	2.8\\
3.8	2.9\\
3.8	3\\
3.8	3.1\\
3.8	3.2\\
3.8	3.3\\
3.8	3.4\\
3.8	3.5\\
3.8	3.6\\
3.8	3.7\\
3.8	3.8\\
3.8	3.9\\
3.8	4\\
3.8	4.1\\
3.8	4.2\\
3.8	4.3\\
3.8	4.4\\
3.8	4.5\\
3.8	4.6\\
3.8	4.7\\
3.8	4.8\\
};
\addplot [color=white, draw=none, mark size=0.2pt, mark=*, mark options={solid, white}, forget plot]
  table[row sep=crcr]{%
3.9	1.2\\
3.9	1.3\\
3.9	1.4\\
3.9	1.5\\
3.9	1.6\\
3.9	1.7\\
3.9	1.8\\
3.9	1.9\\
3.9	2\\
3.9	2.1\\
3.9	2.2\\
3.9	2.3\\
3.9	2.4\\
3.9	2.5\\
3.9	2.6\\
3.9	2.7\\
3.9	2.8\\
3.9	2.9\\
3.9	3\\
3.9	3.1\\
3.9	3.2\\
3.9	3.3\\
3.9	3.4\\
3.9	3.5\\
3.9	3.6\\
3.9	3.7\\
3.9	3.8\\
3.9	3.9\\
3.9	4\\
3.9	4.1\\
3.9	4.2\\
3.9	4.3\\
3.9	4.4\\
3.9	4.5\\
3.9	4.6\\
3.9	4.7\\
3.9	4.8\\
};
\addplot [color=white, draw=none, mark size=0.2pt, mark=*, mark options={solid, white}, forget plot]
  table[row sep=crcr]{%
4	1.2\\
4	1.3\\
4	1.4\\
4	1.5\\
4	1.6\\
4	1.7\\
4	1.8\\
4	1.9\\
4	2\\
4	2.1\\
4	2.2\\
4	2.3\\
4	2.4\\
4	2.5\\
4	2.6\\
4	2.7\\
4	2.8\\
4	2.9\\
4	3\\
4	3.1\\
4	3.2\\
4	3.3\\
4	3.4\\
4	3.5\\
4	3.6\\
4	3.7\\
4	3.8\\
4	3.9\\
4	4\\
4	4.1\\
4	4.2\\
4	4.3\\
4	4.4\\
4	4.5\\
4	4.6\\
4	4.7\\
4	4.8\\
};
\addplot [color=white, draw=none, mark size=0.2pt, mark=*, mark options={solid, white}, forget plot]
  table[row sep=crcr]{%
4.1	1.2\\
4.1	1.3\\
4.1	1.4\\
4.1	1.5\\
4.1	1.6\\
4.1	1.7\\
4.1	1.8\\
4.1	1.9\\
4.1	2\\
4.1	2.1\\
4.1	2.2\\
4.1	2.3\\
4.1	2.4\\
4.1	2.5\\
4.1	2.6\\
4.1	2.7\\
4.1	2.8\\
4.1	2.9\\
4.1	3\\
4.1	3.1\\
4.1	3.2\\
4.1	3.3\\
4.1	3.4\\
4.1	3.5\\
4.1	3.6\\
4.1	3.7\\
4.1	3.8\\
4.1	3.9\\
4.1	4\\
4.1	4.1\\
4.1	4.2\\
4.1	4.3\\
4.1	4.4\\
4.1	4.5\\
4.1	4.6\\
4.1	4.7\\
4.1	4.8\\
};
\addplot [color=white, draw=none, mark size=0.2pt, mark=*, mark options={solid, white}, forget plot]
  table[row sep=crcr]{%
4.2	1.2\\
4.2	1.3\\
4.2	1.4\\
4.2	1.5\\
4.2	1.6\\
4.2	1.7\\
4.2	1.8\\
4.2	1.9\\
4.2	2\\
4.2	2.1\\
4.2	2.2\\
4.2	2.3\\
4.2	2.4\\
4.2	2.5\\
4.2	2.6\\
4.2	2.7\\
4.2	2.8\\
4.2	2.9\\
4.2	3\\
4.2	3.1\\
4.2	3.2\\
4.2	3.3\\
4.2	3.4\\
4.2	3.5\\
4.2	3.6\\
4.2	3.7\\
4.2	3.8\\
4.2	3.9\\
4.2	4\\
4.2	4.1\\
4.2	4.2\\
4.2	4.3\\
4.2	4.4\\
4.2	4.5\\
4.2	4.6\\
4.2	4.7\\
4.2	4.8\\
};
\addplot [color=white, draw=none, mark size=0.2pt, mark=*, mark options={solid, white}, forget plot]
  table[row sep=crcr]{%
4.3	1.2\\
4.3	1.3\\
4.3	1.4\\
4.3	1.5\\
4.3	1.6\\
4.3	1.7\\
4.3	1.8\\
4.3	1.9\\
4.3	2\\
4.3	2.1\\
4.3	2.2\\
4.3	2.3\\
4.3	2.4\\
4.3	2.5\\
4.3	2.6\\
4.3	2.7\\
4.3	2.8\\
4.3	2.9\\
4.3	3\\
4.3	3.1\\
4.3	3.2\\
4.3	3.3\\
4.3	3.4\\
4.3	3.5\\
4.3	3.6\\
4.3	3.7\\
4.3	3.8\\
4.3	3.9\\
4.3	4\\
4.3	4.1\\
4.3	4.2\\
4.3	4.3\\
4.3	4.4\\
4.3	4.5\\
4.3	4.6\\
4.3	4.7\\
4.3	4.8\\
};
\addplot [color=white, draw=none, mark size=0.2pt, mark=*, mark options={solid, white}, forget plot]
  table[row sep=crcr]{%
4.4	1.2\\
4.4	1.3\\
4.4	1.4\\
4.4	1.5\\
4.4	1.6\\
4.4	1.7\\
4.4	1.8\\
4.4	1.9\\
4.4	2\\
4.4	2.1\\
4.4	2.2\\
4.4	2.3\\
4.4	2.4\\
4.4	2.5\\
4.4	2.6\\
4.4	2.7\\
4.4	2.8\\
4.4	2.9\\
4.4	3\\
4.4	3.1\\
4.4	3.2\\
4.4	3.3\\
4.4	3.4\\
4.4	3.5\\
4.4	3.6\\
4.4	3.7\\
4.4	3.8\\
4.4	3.9\\
4.4	4\\
4.4	4.1\\
4.4	4.2\\
4.4	4.3\\
4.4	4.4\\
4.4	4.5\\
4.4	4.6\\
4.4	4.7\\
4.4	4.8\\
};
\addplot [color=white, draw=none, mark size=0.2pt, mark=*, mark options={solid, white}, forget plot]
  table[row sep=crcr]{%
4.5	1.2\\
4.5	1.3\\
4.5	1.4\\
4.5	1.5\\
4.5	1.6\\
4.5	1.7\\
4.5	1.8\\
4.5	1.9\\
4.5	2\\
4.5	2.1\\
4.5	2.2\\
4.5	2.3\\
4.5	2.4\\
4.5	2.5\\
4.5	2.6\\
4.5	2.7\\
4.5	2.8\\
4.5	2.9\\
4.5	3\\
4.5	3.1\\
4.5	3.2\\
4.5	3.3\\
4.5	3.4\\
4.5	3.5\\
4.5	3.6\\
4.5	3.7\\
4.5	3.8\\
4.5	3.9\\
4.5	4\\
4.5	4.1\\
4.5	4.2\\
4.5	4.3\\
4.5	4.4\\
4.5	4.5\\
4.5	4.6\\
4.5	4.7\\
4.5	4.8\\
};
\addplot [color=white, draw=none, mark size=0.2pt, mark=*, mark options={solid, white}, forget plot]
  table[row sep=crcr]{%
4.6	1.2\\
4.6	1.3\\
4.6	1.4\\
4.6	1.5\\
4.6	1.6\\
4.6	1.7\\
4.6	1.8\\
4.6	1.9\\
4.6	2\\
4.6	2.1\\
4.6	2.2\\
4.6	2.3\\
4.6	2.4\\
4.6	2.5\\
4.6	2.6\\
4.6	2.7\\
4.6	2.8\\
4.6	2.9\\
4.6	3\\
4.6	3.1\\
4.6	3.2\\
4.6	3.3\\
4.6	3.4\\
4.6	3.5\\
4.6	3.6\\
4.6	3.7\\
4.6	3.8\\
4.6	3.9\\
4.6	4\\
4.6	4.1\\
4.6	4.2\\
4.6	4.3\\
4.6	4.4\\
4.6	4.5\\
4.6	4.6\\
4.6	4.7\\
4.6	4.8\\
};
\addplot [color=white, draw=none, mark size=0.2pt, mark=*, mark options={solid, white}, forget plot]
  table[row sep=crcr]{%
4.7	1.2\\
4.7	1.3\\
4.7	1.4\\
4.7	1.5\\
4.7	1.6\\
4.7	1.7\\
4.7	1.8\\
4.7	1.9\\
4.7	2\\
4.7	2.1\\
4.7	2.2\\
4.7	2.3\\
4.7	2.4\\
4.7	2.5\\
4.7	2.6\\
4.7	2.7\\
4.7	2.8\\
4.7	2.9\\
4.7	3\\
4.7	3.1\\
4.7	3.2\\
4.7	3.3\\
4.7	3.4\\
4.7	3.5\\
4.7	3.6\\
4.7	3.7\\
4.7	3.8\\
4.7	3.9\\
4.7	4\\
4.7	4.1\\
4.7	4.2\\
4.7	4.3\\
4.7	4.4\\
4.7	4.5\\
4.7	4.6\\
4.7	4.7\\
4.7	4.8\\
};
\addplot [color=white, draw=none, mark size=0.2pt, mark=*, mark options={solid, white}, forget plot]
  table[row sep=crcr]{%
4.8	1.2\\
4.8	1.3\\
4.8	1.4\\
4.8	1.5\\
4.8	1.6\\
4.8	1.7\\
4.8	1.8\\
4.8	1.9\\
4.8	2\\
4.8	2.1\\
4.8	2.2\\
4.8	2.3\\
4.8	2.4\\
4.8	2.5\\
4.8	2.6\\
4.8	2.7\\
4.8	2.8\\
4.8	2.9\\
4.8	3\\
4.8	3.1\\
4.8	3.2\\
4.8	3.3\\
4.8	3.4\\
4.8	3.5\\
4.8	3.6\\
4.8	3.7\\
4.8	3.8\\
4.8	3.9\\
4.8	4\\
4.8	4.1\\
4.8	4.2\\
4.8	4.3\\
4.8	4.4\\
4.8	4.5\\
4.8	4.6\\
4.8	4.7\\
4.8	4.8\\
};
\addplot [color=mycolor1, line width=1.0pt, draw=none, mark size=6.0pt, mark=x, mark options={solid, mycolor1}, forget plot]
  table[row sep=crcr]{%
1.3	4.2\\
2.8	3\\
};
\end{axis}
\end{tikzpicture}%
        %		\includegraphics[width=\textwidth]{data/plots/reference/assignment-problematic-MANUAL}  % PLACEHOLDER
		\caption{Hard Assignment}
	\end{subfigure}
	\begin{subfigure}{0.49\textwidth}
	\centering
%		\includegraphics[width=\textwidth]{data/plots/reference/assignment-problematic-ambivalent}  % PLACEHOLDER
		 \footnotesize
		 \setlength{\figurewidth}{0.8\textwidth}
        \setlength{\figureheight}{6cm}
        % This file was created by matlab2tikz.
%
\definecolor{mycolor1}{rgb}{1.00000,1.00000,0.00000}%
%
\begin{tikzpicture}

\begin{axis}[%
width=0.951\figurewidth,
height=\figureheight,
at={(0\figurewidth,0\figureheight)},
scale only axis,
xmin=-0,
xmax=6,
ymin=-0,
ymax=6,
axis background/.style={fill=white},
axis x line*=bottom,
axis y line*=left
]

\addplot[%
surf,
shader=interp, colormap={mymap}{[1pt] rgb(0pt)=(0.239216,0.14902,0.658824); rgb(1pt)=(0.239216,0.14902,0.658824)}, mesh/rows=6]
table[row sep=crcr, point meta=\thisrow{c}] {%
%
x	y	c\\
0	0	0\\
0	1.2	0\\
0	2.4	0\\
0	3.6	0\\
0	4.8	0\\
0	6	0\\
1.2	0	0\\
1.2	1.2	0\\
1.2	2.4	0\\
1.2	3.6	0\\
1.2	4.8	0\\
1.2	6	0\\
2.4	0	0\\
2.4	1.2	0\\
2.4	2.4	0\\
2.4	3.6	0\\
2.4	4.8	0\\
2.4	6	0\\
3.6	0	0\\
3.6	1.2	0\\
3.6	2.4	0\\
3.6	3.6	0\\
3.6	4.8	0\\
3.6	6	0\\
4.8	0	0\\
4.8	1.2	0\\
4.8	2.4	0\\
4.8	3.6	0\\
4.8	4.8	0\\
4.8	6	0\\
6	0	0\\
6	1.2	0\\
6	2.4	0\\
6	3.6	0\\
6	4.8	0\\
6	6	0\\
};
\addplot [color=green, line width=1.0pt, draw=none, mark size=4.0pt, mark=o, mark options={solid, green}, forget plot]
  table[row sep=crcr]{%
2.1	1\\
2.3	1\\
2.7	1\\
2.9	1\\
3.7	1\\
3.9	1\\
5	2.2\\
5	2.4\\
5	2.8\\
5	3\\
5	3.8\\
5	4\\
2.2	5\\
2.4	5\\
3	5\\
3.2	5\\
3.8	5\\
4	5\\
1	2.1\\
1	2.3\\
1	2.9\\
1	3.1\\
1	3.7\\
1	3.9\\
};
\addplot [color=red, line width=1.0pt, draw=none, mark size=6.0pt, mark=x, mark options={solid, red}, forget plot]
  table[row sep=crcr]{%
3.3	3\\
2.7	3\\
};
\addplot [color=black, draw=none, mark size=0.2pt, mark=*, mark options={solid, black}, forget plot]
  table[row sep=crcr]{%
0.1	0.1\\
0.1	0.2\\
0.1	0.3\\
0.1	0.4\\
0.1	0.5\\
0.1	0.6\\
0.1	0.7\\
0.1	0.8\\
0.1	0.9\\
0.1	1\\
0.1	1.1\\
0.1	1.2\\
0.1	1.3\\
0.1	1.4\\
0.1	1.5\\
0.1	1.6\\
0.1	1.7\\
0.1	1.8\\
0.1	1.9\\
0.1	2\\
0.1	2.1\\
0.1	2.2\\
0.1	2.3\\
0.1	2.4\\
0.1	2.5\\
0.1	2.6\\
0.1	2.7\\
0.1	2.8\\
0.1	2.9\\
0.1	3\\
0.1	3.1\\
0.1	3.2\\
0.1	3.3\\
0.1	3.4\\
0.1	3.5\\
0.1	3.6\\
0.1	3.7\\
0.1	3.8\\
0.1	3.9\\
0.1	4\\
0.1	4.1\\
0.1	4.2\\
0.1	4.3\\
0.1	4.4\\
0.1	4.5\\
0.1	4.6\\
0.1	4.7\\
0.1	4.8\\
0.1	4.9\\
0.1	5\\
0.1	5.1\\
0.1	5.2\\
0.1	5.3\\
0.1	5.4\\
0.1	5.5\\
0.1	5.6\\
0.1	5.7\\
0.1	5.8\\
0.1	5.9\\
};
\addplot [color=black, draw=none, mark size=0.2pt, mark=*, mark options={solid, black}, forget plot]
  table[row sep=crcr]{%
0.2	0.1\\
0.2	0.2\\
0.2	0.3\\
0.2	0.4\\
0.2	0.5\\
0.2	0.6\\
0.2	0.7\\
0.2	0.8\\
0.2	0.9\\
0.2	1\\
0.2	1.1\\
0.2	1.2\\
0.2	1.3\\
0.2	1.4\\
0.2	1.5\\
0.2	1.6\\
0.2	1.7\\
0.2	1.8\\
0.2	1.9\\
0.2	2\\
0.2	2.1\\
0.2	2.2\\
0.2	2.3\\
0.2	2.4\\
0.2	2.5\\
0.2	2.6\\
0.2	2.7\\
0.2	2.8\\
0.2	2.9\\
0.2	3\\
0.2	3.1\\
0.2	3.2\\
0.2	3.3\\
0.2	3.4\\
0.2	3.5\\
0.2	3.6\\
0.2	3.7\\
0.2	3.8\\
0.2	3.9\\
0.2	4\\
0.2	4.1\\
0.2	4.2\\
0.2	4.3\\
0.2	4.4\\
0.2	4.5\\
0.2	4.6\\
0.2	4.7\\
0.2	4.8\\
0.2	4.9\\
0.2	5\\
0.2	5.1\\
0.2	5.2\\
0.2	5.3\\
0.2	5.4\\
0.2	5.5\\
0.2	5.6\\
0.2	5.7\\
0.2	5.8\\
0.2	5.9\\
};
\addplot [color=black, draw=none, mark size=0.2pt, mark=*, mark options={solid, black}, forget plot]
  table[row sep=crcr]{%
0.3	0.1\\
0.3	0.2\\
0.3	0.3\\
0.3	0.4\\
0.3	0.5\\
0.3	0.6\\
0.3	0.7\\
0.3	0.8\\
0.3	0.9\\
0.3	1\\
0.3	1.1\\
0.3	1.2\\
0.3	1.3\\
0.3	1.4\\
0.3	1.5\\
0.3	1.6\\
0.3	1.7\\
0.3	1.8\\
0.3	1.9\\
0.3	2\\
0.3	2.1\\
0.3	2.2\\
0.3	2.3\\
0.3	2.4\\
0.3	2.5\\
0.3	2.6\\
0.3	2.7\\
0.3	2.8\\
0.3	2.9\\
0.3	3\\
0.3	3.1\\
0.3	3.2\\
0.3	3.3\\
0.3	3.4\\
0.3	3.5\\
0.3	3.6\\
0.3	3.7\\
0.3	3.8\\
0.3	3.9\\
0.3	4\\
0.3	4.1\\
0.3	4.2\\
0.3	4.3\\
0.3	4.4\\
0.3	4.5\\
0.3	4.6\\
0.3	4.7\\
0.3	4.8\\
0.3	4.9\\
0.3	5\\
0.3	5.1\\
0.3	5.2\\
0.3	5.3\\
0.3	5.4\\
0.3	5.5\\
0.3	5.6\\
0.3	5.7\\
0.3	5.8\\
0.3	5.9\\
};
\addplot [color=black, draw=none, mark size=0.2pt, mark=*, mark options={solid, black}, forget plot]
  table[row sep=crcr]{%
0.4	0.1\\
0.4	0.2\\
0.4	0.3\\
0.4	0.4\\
0.4	0.5\\
0.4	0.6\\
0.4	0.7\\
0.4	0.8\\
0.4	0.9\\
0.4	1\\
0.4	1.1\\
0.4	1.2\\
0.4	1.3\\
0.4	1.4\\
0.4	1.5\\
0.4	1.6\\
0.4	1.7\\
0.4	1.8\\
0.4	1.9\\
0.4	2\\
0.4	2.1\\
0.4	2.2\\
0.4	2.3\\
0.4	2.4\\
0.4	2.5\\
0.4	2.6\\
0.4	2.7\\
0.4	2.8\\
0.4	2.9\\
0.4	3\\
0.4	3.1\\
0.4	3.2\\
0.4	3.3\\
0.4	3.4\\
0.4	3.5\\
0.4	3.6\\
0.4	3.7\\
0.4	3.8\\
0.4	3.9\\
0.4	4\\
0.4	4.1\\
0.4	4.2\\
0.4	4.3\\
0.4	4.4\\
0.4	4.5\\
0.4	4.6\\
0.4	4.7\\
0.4	4.8\\
0.4	4.9\\
0.4	5\\
0.4	5.1\\
0.4	5.2\\
0.4	5.3\\
0.4	5.4\\
0.4	5.5\\
0.4	5.6\\
0.4	5.7\\
0.4	5.8\\
0.4	5.9\\
};
\addplot [color=black, draw=none, mark size=0.2pt, mark=*, mark options={solid, black}, forget plot]
  table[row sep=crcr]{%
0.5	0.1\\
0.5	0.2\\
0.5	0.3\\
0.5	0.4\\
0.5	0.5\\
0.5	0.6\\
0.5	0.7\\
0.5	0.8\\
0.5	0.9\\
0.5	1\\
0.5	1.1\\
0.5	1.2\\
0.5	1.3\\
0.5	1.4\\
0.5	1.5\\
0.5	1.6\\
0.5	1.7\\
0.5	1.8\\
0.5	1.9\\
0.5	2\\
0.5	2.1\\
0.5	2.2\\
0.5	2.3\\
0.5	2.4\\
0.5	2.5\\
0.5	2.6\\
0.5	2.7\\
0.5	2.8\\
0.5	2.9\\
0.5	3\\
0.5	3.1\\
0.5	3.2\\
0.5	3.3\\
0.5	3.4\\
0.5	3.5\\
0.5	3.6\\
0.5	3.7\\
0.5	3.8\\
0.5	3.9\\
0.5	4\\
0.5	4.1\\
0.5	4.2\\
0.5	4.3\\
0.5	4.4\\
0.5	4.5\\
0.5	4.6\\
0.5	4.7\\
0.5	4.8\\
0.5	4.9\\
0.5	5\\
0.5	5.1\\
0.5	5.2\\
0.5	5.3\\
0.5	5.4\\
0.5	5.5\\
0.5	5.6\\
0.5	5.7\\
0.5	5.8\\
0.5	5.9\\
};
\addplot [color=black, draw=none, mark size=0.2pt, mark=*, mark options={solid, black}, forget plot]
  table[row sep=crcr]{%
0.6	0.1\\
0.6	0.2\\
0.6	0.3\\
0.6	0.4\\
0.6	0.5\\
0.6	0.6\\
0.6	0.7\\
0.6	0.8\\
0.6	0.9\\
0.6	1\\
0.6	1.1\\
0.6	1.2\\
0.6	1.3\\
0.6	1.4\\
0.6	1.5\\
0.6	1.6\\
0.6	1.7\\
0.6	1.8\\
0.6	1.9\\
0.6	2\\
0.6	2.1\\
0.6	2.2\\
0.6	2.3\\
0.6	2.4\\
0.6	2.5\\
0.6	2.6\\
0.6	2.7\\
0.6	2.8\\
0.6	2.9\\
0.6	3\\
0.6	3.1\\
0.6	3.2\\
0.6	3.3\\
0.6	3.4\\
0.6	3.5\\
0.6	3.6\\
0.6	3.7\\
0.6	3.8\\
0.6	3.9\\
0.6	4\\
0.6	4.1\\
0.6	4.2\\
0.6	4.3\\
0.6	4.4\\
0.6	4.5\\
0.6	4.6\\
0.6	4.7\\
0.6	4.8\\
0.6	4.9\\
0.6	5\\
0.6	5.1\\
0.6	5.2\\
0.6	5.3\\
0.6	5.4\\
0.6	5.5\\
0.6	5.6\\
0.6	5.7\\
0.6	5.8\\
0.6	5.9\\
};
\addplot [color=black, draw=none, mark size=0.2pt, mark=*, mark options={solid, black}, forget plot]
  table[row sep=crcr]{%
0.7	0.1\\
0.7	0.2\\
0.7	0.3\\
0.7	0.4\\
0.7	0.5\\
0.7	0.6\\
0.7	0.7\\
0.7	0.8\\
0.7	0.9\\
0.7	1\\
0.7	1.1\\
0.7	1.2\\
0.7	1.3\\
0.7	1.4\\
0.7	1.5\\
0.7	1.6\\
0.7	1.7\\
0.7	1.8\\
0.7	1.9\\
0.7	2\\
0.7	2.1\\
0.7	2.2\\
0.7	2.3\\
0.7	2.4\\
0.7	2.5\\
0.7	2.6\\
0.7	2.7\\
0.7	2.8\\
0.7	2.9\\
0.7	3\\
0.7	3.1\\
0.7	3.2\\
0.7	3.3\\
0.7	3.4\\
0.7	3.5\\
0.7	3.6\\
0.7	3.7\\
0.7	3.8\\
0.7	3.9\\
0.7	4\\
0.7	4.1\\
0.7	4.2\\
0.7	4.3\\
0.7	4.4\\
0.7	4.5\\
0.7	4.6\\
0.7	4.7\\
0.7	4.8\\
0.7	4.9\\
0.7	5\\
0.7	5.1\\
0.7	5.2\\
0.7	5.3\\
0.7	5.4\\
0.7	5.5\\
0.7	5.6\\
0.7	5.7\\
0.7	5.8\\
0.7	5.9\\
};
\addplot [color=black, draw=none, mark size=0.2pt, mark=*, mark options={solid, black}, forget plot]
  table[row sep=crcr]{%
0.8	0.1\\
0.8	0.2\\
0.8	0.3\\
0.8	0.4\\
0.8	0.5\\
0.8	0.6\\
0.8	0.7\\
0.8	0.8\\
0.8	0.9\\
0.8	1\\
0.8	1.1\\
0.8	1.2\\
0.8	1.3\\
0.8	1.4\\
0.8	1.5\\
0.8	1.6\\
0.8	1.7\\
0.8	1.8\\
0.8	1.9\\
0.8	2\\
0.8	2.1\\
0.8	2.2\\
0.8	2.3\\
0.8	2.4\\
0.8	2.5\\
0.8	2.6\\
0.8	2.7\\
0.8	2.8\\
0.8	2.9\\
0.8	3\\
0.8	3.1\\
0.8	3.2\\
0.8	3.3\\
0.8	3.4\\
0.8	3.5\\
0.8	3.6\\
0.8	3.7\\
0.8	3.8\\
0.8	3.9\\
0.8	4\\
0.8	4.1\\
0.8	4.2\\
0.8	4.3\\
0.8	4.4\\
0.8	4.5\\
0.8	4.6\\
0.8	4.7\\
0.8	4.8\\
0.8	4.9\\
0.8	5\\
0.8	5.1\\
0.8	5.2\\
0.8	5.3\\
0.8	5.4\\
0.8	5.5\\
0.8	5.6\\
0.8	5.7\\
0.8	5.8\\
0.8	5.9\\
};
\addplot [color=black, draw=none, mark size=0.2pt, mark=*, mark options={solid, black}, forget plot]
  table[row sep=crcr]{%
0.9	0.1\\
0.9	0.2\\
0.9	0.3\\
0.9	0.4\\
0.9	0.5\\
0.9	0.6\\
0.9	0.7\\
0.9	0.8\\
0.9	0.9\\
0.9	1\\
0.9	1.1\\
0.9	1.2\\
0.9	1.3\\
0.9	1.4\\
0.9	1.5\\
0.9	1.6\\
0.9	1.7\\
0.9	1.8\\
0.9	1.9\\
0.9	2\\
0.9	2.1\\
0.9	2.2\\
0.9	2.3\\
0.9	2.4\\
0.9	2.5\\
0.9	2.6\\
0.9	2.7\\
0.9	2.8\\
0.9	2.9\\
0.9	3\\
0.9	3.1\\
0.9	3.2\\
0.9	3.3\\
0.9	3.4\\
0.9	3.5\\
0.9	3.6\\
0.9	3.7\\
0.9	3.8\\
0.9	3.9\\
0.9	4\\
0.9	4.1\\
0.9	4.2\\
0.9	4.3\\
0.9	4.4\\
0.9	4.5\\
0.9	4.6\\
0.9	4.7\\
0.9	4.8\\
0.9	4.9\\
0.9	5\\
0.9	5.1\\
0.9	5.2\\
0.9	5.3\\
0.9	5.4\\
0.9	5.5\\
0.9	5.6\\
0.9	5.7\\
0.9	5.8\\
0.9	5.9\\
};
\addplot [color=black, draw=none, mark size=0.2pt, mark=*, mark options={solid, black}, forget plot]
  table[row sep=crcr]{%
1	0.1\\
1	0.2\\
1	0.3\\
1	0.4\\
1	0.5\\
1	0.6\\
1	0.7\\
1	0.8\\
1	0.9\\
1	1\\
1	1.1\\
1	1.2\\
1	1.3\\
1	1.4\\
1	1.5\\
1	1.6\\
1	1.7\\
1	1.8\\
1	1.9\\
1	2\\
1	2.1\\
1	2.2\\
1	2.3\\
1	2.4\\
1	2.5\\
1	2.6\\
1	2.7\\
1	2.8\\
1	2.9\\
1	3\\
1	3.1\\
1	3.2\\
1	3.3\\
1	3.4\\
1	3.5\\
1	3.6\\
1	3.7\\
1	3.8\\
1	3.9\\
1	4\\
1	4.1\\
1	4.2\\
1	4.3\\
1	4.4\\
1	4.5\\
1	4.6\\
1	4.7\\
1	4.8\\
1	4.9\\
1	5\\
1	5.1\\
1	5.2\\
1	5.3\\
1	5.4\\
1	5.5\\
1	5.6\\
1	5.7\\
1	5.8\\
1	5.9\\
};
\addplot [color=black, draw=none, mark size=0.2pt, mark=*, mark options={solid, black}, forget plot]
  table[row sep=crcr]{%
1.1	0.1\\
1.1	0.2\\
1.1	0.3\\
1.1	0.4\\
1.1	0.5\\
1.1	0.6\\
1.1	0.7\\
1.1	0.8\\
1.1	0.9\\
1.1	1\\
1.1	1.1\\
1.1	1.2\\
1.1	1.3\\
1.1	1.4\\
1.1	1.5\\
1.1	1.6\\
1.1	1.7\\
1.1	1.8\\
1.1	1.9\\
1.1	2\\
1.1	2.1\\
1.1	2.2\\
1.1	2.3\\
1.1	2.4\\
1.1	2.5\\
1.1	2.6\\
1.1	2.7\\
1.1	2.8\\
1.1	2.9\\
1.1	3\\
1.1	3.1\\
1.1	3.2\\
1.1	3.3\\
1.1	3.4\\
1.1	3.5\\
1.1	3.6\\
1.1	3.7\\
1.1	3.8\\
1.1	3.9\\
1.1	4\\
1.1	4.1\\
1.1	4.2\\
1.1	4.3\\
1.1	4.4\\
1.1	4.5\\
1.1	4.6\\
1.1	4.7\\
1.1	4.8\\
1.1	4.9\\
1.1	5\\
1.1	5.1\\
1.1	5.2\\
1.1	5.3\\
1.1	5.4\\
1.1	5.5\\
1.1	5.6\\
1.1	5.7\\
1.1	5.8\\
1.1	5.9\\
};
\addplot [color=black, draw=none, mark size=0.2pt, mark=*, mark options={solid, black}, forget plot]
  table[row sep=crcr]{%
1.2	0.1\\
1.2	0.2\\
1.2	0.3\\
1.2	0.4\\
1.2	0.5\\
1.2	0.6\\
1.2	0.7\\
1.2	0.8\\
1.2	0.9\\
1.2	1\\
1.2	1.1\\
1.2	1.2\\
1.2	1.3\\
1.2	1.4\\
1.2	1.5\\
1.2	1.6\\
1.2	1.7\\
1.2	1.8\\
1.2	1.9\\
1.2	2\\
1.2	2.1\\
1.2	2.2\\
1.2	2.3\\
1.2	2.4\\
1.2	2.5\\
1.2	2.6\\
1.2	2.7\\
1.2	2.8\\
1.2	2.9\\
1.2	3\\
1.2	3.1\\
1.2	3.2\\
1.2	3.3\\
1.2	3.4\\
1.2	3.5\\
1.2	3.6\\
1.2	3.7\\
1.2	3.8\\
1.2	3.9\\
1.2	4\\
1.2	4.1\\
1.2	4.2\\
1.2	4.3\\
1.2	4.4\\
1.2	4.5\\
1.2	4.6\\
1.2	4.7\\
1.2	4.8\\
1.2	4.9\\
1.2	5\\
1.2	5.1\\
1.2	5.2\\
1.2	5.3\\
1.2	5.4\\
1.2	5.5\\
1.2	5.6\\
1.2	5.7\\
1.2	5.8\\
1.2	5.9\\
};
\addplot [color=black, draw=none, mark size=0.2pt, mark=*, mark options={solid, black}, forget plot]
  table[row sep=crcr]{%
1.3	0.1\\
1.3	0.2\\
1.3	0.3\\
1.3	0.4\\
1.3	0.5\\
1.3	0.6\\
1.3	0.7\\
1.3	0.8\\
1.3	0.9\\
1.3	1\\
1.3	1.1\\
1.3	1.2\\
1.3	1.3\\
1.3	1.4\\
1.3	1.5\\
1.3	1.6\\
1.3	1.7\\
1.3	1.8\\
1.3	1.9\\
1.3	2\\
1.3	2.1\\
1.3	2.2\\
1.3	2.3\\
1.3	2.4\\
1.3	2.5\\
1.3	2.6\\
1.3	2.7\\
1.3	2.8\\
1.3	2.9\\
1.3	3\\
1.3	3.1\\
1.3	3.2\\
1.3	3.3\\
1.3	3.4\\
1.3	3.5\\
1.3	3.6\\
1.3	3.7\\
1.3	3.8\\
1.3	3.9\\
1.3	4\\
1.3	4.1\\
1.3	4.2\\
1.3	4.3\\
1.3	4.4\\
1.3	4.5\\
1.3	4.6\\
1.3	4.7\\
1.3	4.8\\
1.3	4.9\\
1.3	5\\
1.3	5.1\\
1.3	5.2\\
1.3	5.3\\
1.3	5.4\\
1.3	5.5\\
1.3	5.6\\
1.3	5.7\\
1.3	5.8\\
1.3	5.9\\
};
\addplot [color=black, draw=none, mark size=0.2pt, mark=*, mark options={solid, black}, forget plot]
  table[row sep=crcr]{%
1.4	0.1\\
1.4	0.2\\
1.4	0.3\\
1.4	0.4\\
1.4	0.5\\
1.4	0.6\\
1.4	0.7\\
1.4	0.8\\
1.4	0.9\\
1.4	1\\
1.4	1.1\\
1.4	1.2\\
1.4	1.3\\
1.4	1.4\\
1.4	1.5\\
1.4	1.6\\
1.4	1.7\\
1.4	1.8\\
1.4	1.9\\
1.4	2\\
1.4	2.1\\
1.4	2.2\\
1.4	2.3\\
1.4	2.4\\
1.4	2.5\\
1.4	2.6\\
1.4	2.7\\
1.4	2.8\\
1.4	2.9\\
1.4	3\\
1.4	3.1\\
1.4	3.2\\
1.4	3.3\\
1.4	3.4\\
1.4	3.5\\
1.4	3.6\\
1.4	3.7\\
1.4	3.8\\
1.4	3.9\\
1.4	4\\
1.4	4.1\\
1.4	4.2\\
1.4	4.3\\
1.4	4.4\\
1.4	4.5\\
1.4	4.6\\
1.4	4.7\\
1.4	4.8\\
1.4	4.9\\
1.4	5\\
1.4	5.1\\
1.4	5.2\\
1.4	5.3\\
1.4	5.4\\
1.4	5.5\\
1.4	5.6\\
1.4	5.7\\
1.4	5.8\\
1.4	5.9\\
};
\addplot [color=black, draw=none, mark size=0.2pt, mark=*, mark options={solid, black}, forget plot]
  table[row sep=crcr]{%
1.5	0.1\\
1.5	0.2\\
1.5	0.3\\
1.5	0.4\\
1.5	0.5\\
1.5	0.6\\
1.5	0.7\\
1.5	0.8\\
1.5	0.9\\
1.5	1\\
1.5	1.1\\
1.5	1.2\\
1.5	1.3\\
1.5	1.4\\
1.5	1.5\\
1.5	1.6\\
1.5	1.7\\
1.5	1.8\\
1.5	1.9\\
1.5	2\\
1.5	2.1\\
1.5	2.2\\
1.5	2.3\\
1.5	2.4\\
1.5	2.5\\
1.5	2.6\\
1.5	2.7\\
1.5	2.8\\
1.5	2.9\\
1.5	3\\
1.5	3.1\\
1.5	3.2\\
1.5	3.3\\
1.5	3.4\\
1.5	3.5\\
1.5	3.6\\
1.5	3.7\\
1.5	3.8\\
1.5	3.9\\
1.5	4\\
1.5	4.1\\
1.5	4.2\\
1.5	4.3\\
1.5	4.4\\
1.5	4.5\\
1.5	4.6\\
1.5	4.7\\
1.5	4.8\\
1.5	4.9\\
1.5	5\\
1.5	5.1\\
1.5	5.2\\
1.5	5.3\\
1.5	5.4\\
1.5	5.5\\
1.5	5.6\\
1.5	5.7\\
1.5	5.8\\
1.5	5.9\\
};
\addplot [color=black, draw=none, mark size=0.2pt, mark=*, mark options={solid, black}, forget plot]
  table[row sep=crcr]{%
1.6	0.1\\
1.6	0.2\\
1.6	0.3\\
1.6	0.4\\
1.6	0.5\\
1.6	0.6\\
1.6	0.7\\
1.6	0.8\\
1.6	0.9\\
1.6	1\\
1.6	1.1\\
1.6	1.2\\
1.6	1.3\\
1.6	1.4\\
1.6	1.5\\
1.6	1.6\\
1.6	1.7\\
1.6	1.8\\
1.6	1.9\\
1.6	2\\
1.6	2.1\\
1.6	2.2\\
1.6	2.3\\
1.6	2.4\\
1.6	2.5\\
1.6	2.6\\
1.6	2.7\\
1.6	2.8\\
1.6	2.9\\
1.6	3\\
1.6	3.1\\
1.6	3.2\\
1.6	3.3\\
1.6	3.4\\
1.6	3.5\\
1.6	3.6\\
1.6	3.7\\
1.6	3.8\\
1.6	3.9\\
1.6	4\\
1.6	4.1\\
1.6	4.2\\
1.6	4.3\\
1.6	4.4\\
1.6	4.5\\
1.6	4.6\\
1.6	4.7\\
1.6	4.8\\
1.6	4.9\\
1.6	5\\
1.6	5.1\\
1.6	5.2\\
1.6	5.3\\
1.6	5.4\\
1.6	5.5\\
1.6	5.6\\
1.6	5.7\\
1.6	5.8\\
1.6	5.9\\
};
\addplot [color=black, draw=none, mark size=0.2pt, mark=*, mark options={solid, black}, forget plot]
  table[row sep=crcr]{%
1.7	0.1\\
1.7	0.2\\
1.7	0.3\\
1.7	0.4\\
1.7	0.5\\
1.7	0.6\\
1.7	0.7\\
1.7	0.8\\
1.7	0.9\\
1.7	1\\
1.7	1.1\\
1.7	1.2\\
1.7	1.3\\
1.7	1.4\\
1.7	1.5\\
1.7	1.6\\
1.7	1.7\\
1.7	1.8\\
1.7	1.9\\
1.7	2\\
1.7	2.1\\
1.7	2.2\\
1.7	2.3\\
1.7	2.4\\
1.7	2.5\\
1.7	2.6\\
1.7	2.7\\
1.7	2.8\\
1.7	2.9\\
1.7	3\\
1.7	3.1\\
1.7	3.2\\
1.7	3.3\\
1.7	3.4\\
1.7	3.5\\
1.7	3.6\\
1.7	3.7\\
1.7	3.8\\
1.7	3.9\\
1.7	4\\
1.7	4.1\\
1.7	4.2\\
1.7	4.3\\
1.7	4.4\\
1.7	4.5\\
1.7	4.6\\
1.7	4.7\\
1.7	4.8\\
1.7	4.9\\
1.7	5\\
1.7	5.1\\
1.7	5.2\\
1.7	5.3\\
1.7	5.4\\
1.7	5.5\\
1.7	5.6\\
1.7	5.7\\
1.7	5.8\\
1.7	5.9\\
};
\addplot [color=black, draw=none, mark size=0.2pt, mark=*, mark options={solid, black}, forget plot]
  table[row sep=crcr]{%
1.8	0.1\\
1.8	0.2\\
1.8	0.3\\
1.8	0.4\\
1.8	0.5\\
1.8	0.6\\
1.8	0.7\\
1.8	0.8\\
1.8	0.9\\
1.8	1\\
1.8	1.1\\
1.8	1.2\\
1.8	1.3\\
1.8	1.4\\
1.8	1.5\\
1.8	1.6\\
1.8	1.7\\
1.8	1.8\\
1.8	1.9\\
1.8	2\\
1.8	2.1\\
1.8	2.2\\
1.8	2.3\\
1.8	2.4\\
1.8	2.5\\
1.8	2.6\\
1.8	2.7\\
1.8	2.8\\
1.8	2.9\\
1.8	3\\
1.8	3.1\\
1.8	3.2\\
1.8	3.3\\
1.8	3.4\\
1.8	3.5\\
1.8	3.6\\
1.8	3.7\\
1.8	3.8\\
1.8	3.9\\
1.8	4\\
1.8	4.1\\
1.8	4.2\\
1.8	4.3\\
1.8	4.4\\
1.8	4.5\\
1.8	4.6\\
1.8	4.7\\
1.8	4.8\\
1.8	4.9\\
1.8	5\\
1.8	5.1\\
1.8	5.2\\
1.8	5.3\\
1.8	5.4\\
1.8	5.5\\
1.8	5.6\\
1.8	5.7\\
1.8	5.8\\
1.8	5.9\\
};
\addplot [color=black, draw=none, mark size=0.2pt, mark=*, mark options={solid, black}, forget plot]
  table[row sep=crcr]{%
1.9	0.1\\
1.9	0.2\\
1.9	0.3\\
1.9	0.4\\
1.9	0.5\\
1.9	0.6\\
1.9	0.7\\
1.9	0.8\\
1.9	0.9\\
1.9	1\\
1.9	1.1\\
1.9	1.2\\
1.9	1.3\\
1.9	1.4\\
1.9	1.5\\
1.9	1.6\\
1.9	1.7\\
1.9	1.8\\
1.9	1.9\\
1.9	2\\
1.9	2.1\\
1.9	2.2\\
1.9	2.3\\
1.9	2.4\\
1.9	2.5\\
1.9	2.6\\
1.9	2.7\\
1.9	2.8\\
1.9	2.9\\
1.9	3\\
1.9	3.1\\
1.9	3.2\\
1.9	3.3\\
1.9	3.4\\
1.9	3.5\\
1.9	3.6\\
1.9	3.7\\
1.9	3.8\\
1.9	3.9\\
1.9	4\\
1.9	4.1\\
1.9	4.2\\
1.9	4.3\\
1.9	4.4\\
1.9	4.5\\
1.9	4.6\\
1.9	4.7\\
1.9	4.8\\
1.9	4.9\\
1.9	5\\
1.9	5.1\\
1.9	5.2\\
1.9	5.3\\
1.9	5.4\\
1.9	5.5\\
1.9	5.6\\
1.9	5.7\\
1.9	5.8\\
1.9	5.9\\
};
\addplot [color=black, draw=none, mark size=0.2pt, mark=*, mark options={solid, black}, forget plot]
  table[row sep=crcr]{%
2	0.1\\
2	0.2\\
2	0.3\\
2	0.4\\
2	0.5\\
2	0.6\\
2	0.7\\
2	0.8\\
2	0.9\\
2	1\\
2	1.1\\
2	1.2\\
2	1.3\\
2	1.4\\
2	1.5\\
2	1.6\\
2	1.7\\
2	1.8\\
2	1.9\\
2	2\\
2	2.1\\
2	2.2\\
2	2.3\\
2	2.4\\
2	2.5\\
2	2.6\\
2	2.7\\
2	2.8\\
2	2.9\\
2	3\\
2	3.1\\
2	3.2\\
2	3.3\\
2	3.4\\
2	3.5\\
2	3.6\\
2	3.7\\
2	3.8\\
2	3.9\\
2	4\\
2	4.1\\
2	4.2\\
2	4.3\\
2	4.4\\
2	4.5\\
2	4.6\\
2	4.7\\
2	4.8\\
2	4.9\\
2	5\\
2	5.1\\
2	5.2\\
2	5.3\\
2	5.4\\
2	5.5\\
2	5.6\\
2	5.7\\
2	5.8\\
2	5.9\\
};
\addplot [color=black, draw=none, mark size=0.2pt, mark=*, mark options={solid, black}, forget plot]
  table[row sep=crcr]{%
2.1	0.1\\
2.1	0.2\\
2.1	0.3\\
2.1	0.4\\
2.1	0.5\\
2.1	0.6\\
2.1	0.7\\
2.1	0.8\\
2.1	0.9\\
2.1	1\\
2.1	1.1\\
2.1	1.2\\
2.1	1.3\\
2.1	1.4\\
2.1	1.5\\
2.1	1.6\\
2.1	1.7\\
2.1	1.8\\
2.1	1.9\\
2.1	2\\
2.1	2.1\\
2.1	2.2\\
2.1	2.3\\
2.1	2.4\\
2.1	2.5\\
2.1	2.6\\
2.1	2.7\\
2.1	2.8\\
2.1	2.9\\
2.1	3\\
2.1	3.1\\
2.1	3.2\\
2.1	3.3\\
2.1	3.4\\
2.1	3.5\\
2.1	3.6\\
2.1	3.7\\
2.1	3.8\\
2.1	3.9\\
2.1	4\\
2.1	4.1\\
2.1	4.2\\
2.1	4.3\\
2.1	4.4\\
2.1	4.5\\
2.1	4.6\\
2.1	4.7\\
2.1	4.8\\
2.1	4.9\\
2.1	5\\
2.1	5.1\\
2.1	5.2\\
2.1	5.3\\
2.1	5.4\\
2.1	5.5\\
2.1	5.6\\
2.1	5.7\\
2.1	5.8\\
2.1	5.9\\
};
\addplot [color=black, draw=none, mark size=0.2pt, mark=*, mark options={solid, black}, forget plot]
  table[row sep=crcr]{%
2.2	0.1\\
2.2	0.2\\
2.2	0.3\\
2.2	0.4\\
2.2	0.5\\
2.2	0.6\\
2.2	0.7\\
2.2	0.8\\
2.2	0.9\\
2.2	1\\
2.2	1.1\\
2.2	1.2\\
2.2	1.3\\
2.2	1.4\\
2.2	1.5\\
2.2	1.6\\
2.2	1.7\\
2.2	1.8\\
2.2	1.9\\
2.2	2\\
2.2	2.1\\
2.2	2.2\\
2.2	2.3\\
2.2	2.4\\
2.2	2.5\\
2.2	2.6\\
2.2	2.7\\
2.2	2.8\\
2.2	2.9\\
2.2	3\\
2.2	3.1\\
2.2	3.2\\
2.2	3.3\\
2.2	3.4\\
2.2	3.5\\
2.2	3.6\\
2.2	3.7\\
2.2	3.8\\
2.2	3.9\\
2.2	4\\
2.2	4.1\\
2.2	4.2\\
2.2	4.3\\
2.2	4.4\\
2.2	4.5\\
2.2	4.6\\
2.2	4.7\\
2.2	4.8\\
2.2	4.9\\
2.2	5\\
2.2	5.1\\
2.2	5.2\\
2.2	5.3\\
2.2	5.4\\
2.2	5.5\\
2.2	5.6\\
2.2	5.7\\
2.2	5.8\\
2.2	5.9\\
};
\addplot [color=black, draw=none, mark size=0.2pt, mark=*, mark options={solid, black}, forget plot]
  table[row sep=crcr]{%
2.3	0.1\\
2.3	0.2\\
2.3	0.3\\
2.3	0.4\\
2.3	0.5\\
2.3	0.6\\
2.3	0.7\\
2.3	0.8\\
2.3	0.9\\
2.3	1\\
2.3	1.1\\
2.3	1.2\\
2.3	1.3\\
2.3	1.4\\
2.3	1.5\\
2.3	1.6\\
2.3	1.7\\
2.3	1.8\\
2.3	1.9\\
2.3	2\\
2.3	2.1\\
2.3	2.2\\
2.3	2.3\\
2.3	2.4\\
2.3	2.5\\
2.3	2.6\\
2.3	2.7\\
2.3	2.8\\
2.3	2.9\\
2.3	3\\
2.3	3.1\\
2.3	3.2\\
2.3	3.3\\
2.3	3.4\\
2.3	3.5\\
2.3	3.6\\
2.3	3.7\\
2.3	3.8\\
2.3	3.9\\
2.3	4\\
2.3	4.1\\
2.3	4.2\\
2.3	4.3\\
2.3	4.4\\
2.3	4.5\\
2.3	4.6\\
2.3	4.7\\
2.3	4.8\\
2.3	4.9\\
2.3	5\\
2.3	5.1\\
2.3	5.2\\
2.3	5.3\\
2.3	5.4\\
2.3	5.5\\
2.3	5.6\\
2.3	5.7\\
2.3	5.8\\
2.3	5.9\\
};
\addplot [color=black, draw=none, mark size=0.2pt, mark=*, mark options={solid, black}, forget plot]
  table[row sep=crcr]{%
2.4	0.1\\
2.4	0.2\\
2.4	0.3\\
2.4	0.4\\
2.4	0.5\\
2.4	0.6\\
2.4	0.7\\
2.4	0.8\\
2.4	0.9\\
2.4	1\\
2.4	1.1\\
2.4	1.2\\
2.4	1.3\\
2.4	1.4\\
2.4	1.5\\
2.4	1.6\\
2.4	1.7\\
2.4	1.8\\
2.4	1.9\\
2.4	2\\
2.4	2.1\\
2.4	2.2\\
2.4	2.3\\
2.4	2.4\\
2.4	2.5\\
2.4	2.6\\
2.4	2.7\\
2.4	2.8\\
2.4	2.9\\
2.4	3\\
2.4	3.1\\
2.4	3.2\\
2.4	3.3\\
2.4	3.4\\
2.4	3.5\\
2.4	3.6\\
2.4	3.7\\
2.4	3.8\\
2.4	3.9\\
2.4	4\\
2.4	4.1\\
2.4	4.2\\
2.4	4.3\\
2.4	4.4\\
2.4	4.5\\
2.4	4.6\\
2.4	4.7\\
2.4	4.8\\
2.4	4.9\\
2.4	5\\
2.4	5.1\\
2.4	5.2\\
2.4	5.3\\
2.4	5.4\\
2.4	5.5\\
2.4	5.6\\
2.4	5.7\\
2.4	5.8\\
2.4	5.9\\
};
\addplot [color=black, draw=none, mark size=0.2pt, mark=*, mark options={solid, black}, forget plot]
  table[row sep=crcr]{%
2.5	0.1\\
2.5	0.2\\
2.5	0.3\\
2.5	0.4\\
2.5	0.5\\
2.5	0.6\\
2.5	0.7\\
2.5	0.8\\
2.5	0.9\\
2.5	1\\
2.5	1.1\\
2.5	1.2\\
2.5	1.3\\
2.5	1.4\\
2.5	1.5\\
2.5	1.6\\
2.5	1.7\\
2.5	1.8\\
2.5	1.9\\
2.5	2\\
2.5	2.1\\
2.5	2.2\\
2.5	2.3\\
2.5	2.4\\
2.5	2.5\\
2.5	2.6\\
2.5	2.7\\
2.5	2.8\\
2.5	2.9\\
2.5	3\\
2.5	3.1\\
2.5	3.2\\
2.5	3.3\\
2.5	3.4\\
2.5	3.5\\
2.5	3.6\\
2.5	3.7\\
2.5	3.8\\
2.5	3.9\\
2.5	4\\
2.5	4.1\\
2.5	4.2\\
2.5	4.3\\
2.5	4.4\\
2.5	4.5\\
2.5	4.6\\
2.5	4.7\\
2.5	4.8\\
2.5	4.9\\
2.5	5\\
2.5	5.1\\
2.5	5.2\\
2.5	5.3\\
2.5	5.4\\
2.5	5.5\\
2.5	5.6\\
2.5	5.7\\
2.5	5.8\\
2.5	5.9\\
};
\addplot [color=black, draw=none, mark size=0.2pt, mark=*, mark options={solid, black}, forget plot]
  table[row sep=crcr]{%
2.6	0.1\\
2.6	0.2\\
2.6	0.3\\
2.6	0.4\\
2.6	0.5\\
2.6	0.6\\
2.6	0.7\\
2.6	0.8\\
2.6	0.9\\
2.6	1\\
2.6	1.1\\
2.6	1.2\\
2.6	1.3\\
2.6	1.4\\
2.6	1.5\\
2.6	1.6\\
2.6	1.7\\
2.6	1.8\\
2.6	1.9\\
2.6	2\\
2.6	2.1\\
2.6	2.2\\
2.6	2.3\\
2.6	2.4\\
2.6	2.5\\
2.6	2.6\\
2.6	2.7\\
2.6	2.8\\
2.6	2.9\\
2.6	3\\
2.6	3.1\\
2.6	3.2\\
2.6	3.3\\
2.6	3.4\\
2.6	3.5\\
2.6	3.6\\
2.6	3.7\\
2.6	3.8\\
2.6	3.9\\
2.6	4\\
2.6	4.1\\
2.6	4.2\\
2.6	4.3\\
2.6	4.4\\
2.6	4.5\\
2.6	4.6\\
2.6	4.7\\
2.6	4.8\\
2.6	4.9\\
2.6	5\\
2.6	5.1\\
2.6	5.2\\
2.6	5.3\\
2.6	5.4\\
2.6	5.5\\
2.6	5.6\\
2.6	5.7\\
2.6	5.8\\
2.6	5.9\\
};
\addplot [color=black, draw=none, mark size=0.2pt, mark=*, mark options={solid, black}, forget plot]
  table[row sep=crcr]{%
2.7	0.1\\
2.7	0.2\\
2.7	0.3\\
2.7	0.4\\
2.7	0.5\\
2.7	0.6\\
2.7	0.7\\
2.7	0.8\\
2.7	0.9\\
2.7	1\\
2.7	1.1\\
2.7	1.2\\
2.7	1.3\\
2.7	1.4\\
2.7	1.5\\
2.7	1.6\\
2.7	1.7\\
2.7	1.8\\
2.7	1.9\\
2.7	2\\
2.7	2.1\\
2.7	2.2\\
2.7	2.3\\
2.7	2.4\\
2.7	2.5\\
2.7	2.6\\
2.7	2.7\\
2.7	2.8\\
2.7	2.9\\
2.7	3\\
2.7	3.1\\
2.7	3.2\\
2.7	3.3\\
2.7	3.4\\
2.7	3.5\\
2.7	3.6\\
2.7	3.7\\
2.7	3.8\\
2.7	3.9\\
2.7	4\\
2.7	4.1\\
2.7	4.2\\
2.7	4.3\\
2.7	4.4\\
2.7	4.5\\
2.7	4.6\\
2.7	4.7\\
2.7	4.8\\
2.7	4.9\\
2.7	5\\
2.7	5.1\\
2.7	5.2\\
2.7	5.3\\
2.7	5.4\\
2.7	5.5\\
2.7	5.6\\
2.7	5.7\\
2.7	5.8\\
2.7	5.9\\
};
\addplot [color=black, draw=none, mark size=0.2pt, mark=*, mark options={solid, black}, forget plot]
  table[row sep=crcr]{%
2.8	0.1\\
2.8	0.2\\
2.8	0.3\\
2.8	0.4\\
2.8	0.5\\
2.8	0.6\\
2.8	0.7\\
2.8	0.8\\
2.8	0.9\\
2.8	1\\
2.8	1.1\\
2.8	1.2\\
2.8	1.3\\
2.8	1.4\\
2.8	1.5\\
2.8	1.6\\
2.8	1.7\\
2.8	1.8\\
2.8	1.9\\
2.8	2\\
2.8	2.1\\
2.8	2.2\\
2.8	2.3\\
2.8	2.4\\
2.8	2.5\\
2.8	2.6\\
2.8	2.7\\
2.8	2.8\\
2.8	2.9\\
2.8	3\\
2.8	3.1\\
2.8	3.2\\
2.8	3.3\\
2.8	3.4\\
2.8	3.5\\
2.8	3.6\\
2.8	3.7\\
2.8	3.8\\
2.8	3.9\\
2.8	4\\
2.8	4.1\\
2.8	4.2\\
2.8	4.3\\
2.8	4.4\\
2.8	4.5\\
2.8	4.6\\
2.8	4.7\\
2.8	4.8\\
2.8	4.9\\
2.8	5\\
2.8	5.1\\
2.8	5.2\\
2.8	5.3\\
2.8	5.4\\
2.8	5.5\\
2.8	5.6\\
2.8	5.7\\
2.8	5.8\\
2.8	5.9\\
};
\addplot [color=black, draw=none, mark size=0.2pt, mark=*, mark options={solid, black}, forget plot]
  table[row sep=crcr]{%
2.9	0.1\\
2.9	0.2\\
2.9	0.3\\
2.9	0.4\\
2.9	0.5\\
2.9	0.6\\
2.9	0.7\\
2.9	0.8\\
2.9	0.9\\
2.9	1\\
2.9	1.1\\
2.9	1.2\\
2.9	1.3\\
2.9	1.4\\
2.9	1.5\\
2.9	1.6\\
2.9	1.7\\
2.9	1.8\\
2.9	1.9\\
2.9	2\\
2.9	2.1\\
2.9	2.2\\
2.9	2.3\\
2.9	2.4\\
2.9	2.5\\
2.9	2.6\\
2.9	2.7\\
2.9	2.8\\
2.9	2.9\\
2.9	3\\
2.9	3.1\\
2.9	3.2\\
2.9	3.3\\
2.9	3.4\\
2.9	3.5\\
2.9	3.6\\
2.9	3.7\\
2.9	3.8\\
2.9	3.9\\
2.9	4\\
2.9	4.1\\
2.9	4.2\\
2.9	4.3\\
2.9	4.4\\
2.9	4.5\\
2.9	4.6\\
2.9	4.7\\
2.9	4.8\\
2.9	4.9\\
2.9	5\\
2.9	5.1\\
2.9	5.2\\
2.9	5.3\\
2.9	5.4\\
2.9	5.5\\
2.9	5.6\\
2.9	5.7\\
2.9	5.8\\
2.9	5.9\\
};
\addplot [color=black, draw=none, mark size=0.2pt, mark=*, mark options={solid, black}, forget plot]
  table[row sep=crcr]{%
3	0.1\\
3	0.2\\
3	0.3\\
3	0.4\\
3	0.5\\
3	0.6\\
3	0.7\\
3	0.8\\
3	0.9\\
3	1\\
3	1.1\\
3	1.2\\
3	1.3\\
3	1.4\\
3	1.5\\
3	1.6\\
3	1.7\\
3	1.8\\
3	1.9\\
3	2\\
3	2.1\\
3	2.2\\
3	2.3\\
3	2.4\\
3	2.5\\
3	2.6\\
3	2.7\\
3	2.8\\
3	2.9\\
3	3\\
3	3.1\\
3	3.2\\
3	3.3\\
3	3.4\\
3	3.5\\
3	3.6\\
3	3.7\\
3	3.8\\
3	3.9\\
3	4\\
3	4.1\\
3	4.2\\
3	4.3\\
3	4.4\\
3	4.5\\
3	4.6\\
3	4.7\\
3	4.8\\
3	4.9\\
3	5\\
3	5.1\\
3	5.2\\
3	5.3\\
3	5.4\\
3	5.5\\
3	5.6\\
3	5.7\\
3	5.8\\
3	5.9\\
};
\addplot [color=black, draw=none, mark size=0.2pt, mark=*, mark options={solid, black}, forget plot]
  table[row sep=crcr]{%
3.1	0.1\\
3.1	0.2\\
3.1	0.3\\
3.1	0.4\\
3.1	0.5\\
3.1	0.6\\
3.1	0.7\\
3.1	0.8\\
3.1	0.9\\
3.1	1\\
3.1	1.1\\
3.1	1.2\\
3.1	1.3\\
3.1	1.4\\
3.1	1.5\\
3.1	1.6\\
3.1	1.7\\
3.1	1.8\\
3.1	1.9\\
3.1	2\\
3.1	2.1\\
3.1	2.2\\
3.1	2.3\\
3.1	2.4\\
3.1	2.5\\
3.1	2.6\\
3.1	2.7\\
3.1	2.8\\
3.1	2.9\\
3.1	3\\
3.1	3.1\\
3.1	3.2\\
3.1	3.3\\
3.1	3.4\\
3.1	3.5\\
3.1	3.6\\
3.1	3.7\\
3.1	3.8\\
3.1	3.9\\
3.1	4\\
3.1	4.1\\
3.1	4.2\\
3.1	4.3\\
3.1	4.4\\
3.1	4.5\\
3.1	4.6\\
3.1	4.7\\
3.1	4.8\\
3.1	4.9\\
3.1	5\\
3.1	5.1\\
3.1	5.2\\
3.1	5.3\\
3.1	5.4\\
3.1	5.5\\
3.1	5.6\\
3.1	5.7\\
3.1	5.8\\
3.1	5.9\\
};
\addplot [color=black, draw=none, mark size=0.2pt, mark=*, mark options={solid, black}, forget plot]
  table[row sep=crcr]{%
3.2	0.1\\
3.2	0.2\\
3.2	0.3\\
3.2	0.4\\
3.2	0.5\\
3.2	0.6\\
3.2	0.7\\
3.2	0.8\\
3.2	0.9\\
3.2	1\\
3.2	1.1\\
3.2	1.2\\
3.2	1.3\\
3.2	1.4\\
3.2	1.5\\
3.2	1.6\\
3.2	1.7\\
3.2	1.8\\
3.2	1.9\\
3.2	2\\
3.2	2.1\\
3.2	2.2\\
3.2	2.3\\
3.2	2.4\\
3.2	2.5\\
3.2	2.6\\
3.2	2.7\\
3.2	2.8\\
3.2	2.9\\
3.2	3\\
3.2	3.1\\
3.2	3.2\\
3.2	3.3\\
3.2	3.4\\
3.2	3.5\\
3.2	3.6\\
3.2	3.7\\
3.2	3.8\\
3.2	3.9\\
3.2	4\\
3.2	4.1\\
3.2	4.2\\
3.2	4.3\\
3.2	4.4\\
3.2	4.5\\
3.2	4.6\\
3.2	4.7\\
3.2	4.8\\
3.2	4.9\\
3.2	5\\
3.2	5.1\\
3.2	5.2\\
3.2	5.3\\
3.2	5.4\\
3.2	5.5\\
3.2	5.6\\
3.2	5.7\\
3.2	5.8\\
3.2	5.9\\
};
\addplot [color=black, draw=none, mark size=0.2pt, mark=*, mark options={solid, black}, forget plot]
  table[row sep=crcr]{%
3.3	0.1\\
3.3	0.2\\
3.3	0.3\\
3.3	0.4\\
3.3	0.5\\
3.3	0.6\\
3.3	0.7\\
3.3	0.8\\
3.3	0.9\\
3.3	1\\
3.3	1.1\\
3.3	1.2\\
3.3	1.3\\
3.3	1.4\\
3.3	1.5\\
3.3	1.6\\
3.3	1.7\\
3.3	1.8\\
3.3	1.9\\
3.3	2\\
3.3	2.1\\
3.3	2.2\\
3.3	2.3\\
3.3	2.4\\
3.3	2.5\\
3.3	2.6\\
3.3	2.7\\
3.3	2.8\\
3.3	2.9\\
3.3	3\\
3.3	3.1\\
3.3	3.2\\
3.3	3.3\\
3.3	3.4\\
3.3	3.5\\
3.3	3.6\\
3.3	3.7\\
3.3	3.8\\
3.3	3.9\\
3.3	4\\
3.3	4.1\\
3.3	4.2\\
3.3	4.3\\
3.3	4.4\\
3.3	4.5\\
3.3	4.6\\
3.3	4.7\\
3.3	4.8\\
3.3	4.9\\
3.3	5\\
3.3	5.1\\
3.3	5.2\\
3.3	5.3\\
3.3	5.4\\
3.3	5.5\\
3.3	5.6\\
3.3	5.7\\
3.3	5.8\\
3.3	5.9\\
};
\addplot [color=black, draw=none, mark size=0.2pt, mark=*, mark options={solid, black}, forget plot]
  table[row sep=crcr]{%
3.4	0.1\\
3.4	0.2\\
3.4	0.3\\
3.4	0.4\\
3.4	0.5\\
3.4	0.6\\
3.4	0.7\\
3.4	0.8\\
3.4	0.9\\
3.4	1\\
3.4	1.1\\
3.4	1.2\\
3.4	1.3\\
3.4	1.4\\
3.4	1.5\\
3.4	1.6\\
3.4	1.7\\
3.4	1.8\\
3.4	1.9\\
3.4	2\\
3.4	2.1\\
3.4	2.2\\
3.4	2.3\\
3.4	2.4\\
3.4	2.5\\
3.4	2.6\\
3.4	2.7\\
3.4	2.8\\
3.4	2.9\\
3.4	3\\
3.4	3.1\\
3.4	3.2\\
3.4	3.3\\
3.4	3.4\\
3.4	3.5\\
3.4	3.6\\
3.4	3.7\\
3.4	3.8\\
3.4	3.9\\
3.4	4\\
3.4	4.1\\
3.4	4.2\\
3.4	4.3\\
3.4	4.4\\
3.4	4.5\\
3.4	4.6\\
3.4	4.7\\
3.4	4.8\\
3.4	4.9\\
3.4	5\\
3.4	5.1\\
3.4	5.2\\
3.4	5.3\\
3.4	5.4\\
3.4	5.5\\
3.4	5.6\\
3.4	5.7\\
3.4	5.8\\
3.4	5.9\\
};
\addplot [color=black, draw=none, mark size=0.2pt, mark=*, mark options={solid, black}, forget plot]
  table[row sep=crcr]{%
3.5	0.1\\
3.5	0.2\\
3.5	0.3\\
3.5	0.4\\
3.5	0.5\\
3.5	0.6\\
3.5	0.7\\
3.5	0.8\\
3.5	0.9\\
3.5	1\\
3.5	1.1\\
3.5	1.2\\
3.5	1.3\\
3.5	1.4\\
3.5	1.5\\
3.5	1.6\\
3.5	1.7\\
3.5	1.8\\
3.5	1.9\\
3.5	2\\
3.5	2.1\\
3.5	2.2\\
3.5	2.3\\
3.5	2.4\\
3.5	2.5\\
3.5	2.6\\
3.5	2.7\\
3.5	2.8\\
3.5	2.9\\
3.5	3\\
3.5	3.1\\
3.5	3.2\\
3.5	3.3\\
3.5	3.4\\
3.5	3.5\\
3.5	3.6\\
3.5	3.7\\
3.5	3.8\\
3.5	3.9\\
3.5	4\\
3.5	4.1\\
3.5	4.2\\
3.5	4.3\\
3.5	4.4\\
3.5	4.5\\
3.5	4.6\\
3.5	4.7\\
3.5	4.8\\
3.5	4.9\\
3.5	5\\
3.5	5.1\\
3.5	5.2\\
3.5	5.3\\
3.5	5.4\\
3.5	5.5\\
3.5	5.6\\
3.5	5.7\\
3.5	5.8\\
3.5	5.9\\
};
\addplot [color=black, draw=none, mark size=0.2pt, mark=*, mark options={solid, black}, forget plot]
  table[row sep=crcr]{%
3.6	0.1\\
3.6	0.2\\
3.6	0.3\\
3.6	0.4\\
3.6	0.5\\
3.6	0.6\\
3.6	0.7\\
3.6	0.8\\
3.6	0.9\\
3.6	1\\
3.6	1.1\\
3.6	1.2\\
3.6	1.3\\
3.6	1.4\\
3.6	1.5\\
3.6	1.6\\
3.6	1.7\\
3.6	1.8\\
3.6	1.9\\
3.6	2\\
3.6	2.1\\
3.6	2.2\\
3.6	2.3\\
3.6	2.4\\
3.6	2.5\\
3.6	2.6\\
3.6	2.7\\
3.6	2.8\\
3.6	2.9\\
3.6	3\\
3.6	3.1\\
3.6	3.2\\
3.6	3.3\\
3.6	3.4\\
3.6	3.5\\
3.6	3.6\\
3.6	3.7\\
3.6	3.8\\
3.6	3.9\\
3.6	4\\
3.6	4.1\\
3.6	4.2\\
3.6	4.3\\
3.6	4.4\\
3.6	4.5\\
3.6	4.6\\
3.6	4.7\\
3.6	4.8\\
3.6	4.9\\
3.6	5\\
3.6	5.1\\
3.6	5.2\\
3.6	5.3\\
3.6	5.4\\
3.6	5.5\\
3.6	5.6\\
3.6	5.7\\
3.6	5.8\\
3.6	5.9\\
};
\addplot [color=black, draw=none, mark size=0.2pt, mark=*, mark options={solid, black}, forget plot]
  table[row sep=crcr]{%
3.7	0.1\\
3.7	0.2\\
3.7	0.3\\
3.7	0.4\\
3.7	0.5\\
3.7	0.6\\
3.7	0.7\\
3.7	0.8\\
3.7	0.9\\
3.7	1\\
3.7	1.1\\
3.7	1.2\\
3.7	1.3\\
3.7	1.4\\
3.7	1.5\\
3.7	1.6\\
3.7	1.7\\
3.7	1.8\\
3.7	1.9\\
3.7	2\\
3.7	2.1\\
3.7	2.2\\
3.7	2.3\\
3.7	2.4\\
3.7	2.5\\
3.7	2.6\\
3.7	2.7\\
3.7	2.8\\
3.7	2.9\\
3.7	3\\
3.7	3.1\\
3.7	3.2\\
3.7	3.3\\
3.7	3.4\\
3.7	3.5\\
3.7	3.6\\
3.7	3.7\\
3.7	3.8\\
3.7	3.9\\
3.7	4\\
3.7	4.1\\
3.7	4.2\\
3.7	4.3\\
3.7	4.4\\
3.7	4.5\\
3.7	4.6\\
3.7	4.7\\
3.7	4.8\\
3.7	4.9\\
3.7	5\\
3.7	5.1\\
3.7	5.2\\
3.7	5.3\\
3.7	5.4\\
3.7	5.5\\
3.7	5.6\\
3.7	5.7\\
3.7	5.8\\
3.7	5.9\\
};
\addplot [color=black, draw=none, mark size=0.2pt, mark=*, mark options={solid, black}, forget plot]
  table[row sep=crcr]{%
3.8	0.1\\
3.8	0.2\\
3.8	0.3\\
3.8	0.4\\
3.8	0.5\\
3.8	0.6\\
3.8	0.7\\
3.8	0.8\\
3.8	0.9\\
3.8	1\\
3.8	1.1\\
3.8	1.2\\
3.8	1.3\\
3.8	1.4\\
3.8	1.5\\
3.8	1.6\\
3.8	1.7\\
3.8	1.8\\
3.8	1.9\\
3.8	2\\
3.8	2.1\\
3.8	2.2\\
3.8	2.3\\
3.8	2.4\\
3.8	2.5\\
3.8	2.6\\
3.8	2.7\\
3.8	2.8\\
3.8	2.9\\
3.8	3\\
3.8	3.1\\
3.8	3.2\\
3.8	3.3\\
3.8	3.4\\
3.8	3.5\\
3.8	3.6\\
3.8	3.7\\
3.8	3.8\\
3.8	3.9\\
3.8	4\\
3.8	4.1\\
3.8	4.2\\
3.8	4.3\\
3.8	4.4\\
3.8	4.5\\
3.8	4.6\\
3.8	4.7\\
3.8	4.8\\
3.8	4.9\\
3.8	5\\
3.8	5.1\\
3.8	5.2\\
3.8	5.3\\
3.8	5.4\\
3.8	5.5\\
3.8	5.6\\
3.8	5.7\\
3.8	5.8\\
3.8	5.9\\
};
\addplot [color=black, draw=none, mark size=0.2pt, mark=*, mark options={solid, black}, forget plot]
  table[row sep=crcr]{%
3.9	0.1\\
3.9	0.2\\
3.9	0.3\\
3.9	0.4\\
3.9	0.5\\
3.9	0.6\\
3.9	0.7\\
3.9	0.8\\
3.9	0.9\\
3.9	1\\
3.9	1.1\\
3.9	1.2\\
3.9	1.3\\
3.9	1.4\\
3.9	1.5\\
3.9	1.6\\
3.9	1.7\\
3.9	1.8\\
3.9	1.9\\
3.9	2\\
3.9	2.1\\
3.9	2.2\\
3.9	2.3\\
3.9	2.4\\
3.9	2.5\\
3.9	2.6\\
3.9	2.7\\
3.9	2.8\\
3.9	2.9\\
3.9	3\\
3.9	3.1\\
3.9	3.2\\
3.9	3.3\\
3.9	3.4\\
3.9	3.5\\
3.9	3.6\\
3.9	3.7\\
3.9	3.8\\
3.9	3.9\\
3.9	4\\
3.9	4.1\\
3.9	4.2\\
3.9	4.3\\
3.9	4.4\\
3.9	4.5\\
3.9	4.6\\
3.9	4.7\\
3.9	4.8\\
3.9	4.9\\
3.9	5\\
3.9	5.1\\
3.9	5.2\\
3.9	5.3\\
3.9	5.4\\
3.9	5.5\\
3.9	5.6\\
3.9	5.7\\
3.9	5.8\\
3.9	5.9\\
};
\addplot [color=black, draw=none, mark size=0.2pt, mark=*, mark options={solid, black}, forget plot]
  table[row sep=crcr]{%
4	0.1\\
4	0.2\\
4	0.3\\
4	0.4\\
4	0.5\\
4	0.6\\
4	0.7\\
4	0.8\\
4	0.9\\
4	1\\
4	1.1\\
4	1.2\\
4	1.3\\
4	1.4\\
4	1.5\\
4	1.6\\
4	1.7\\
4	1.8\\
4	1.9\\
4	2\\
4	2.1\\
4	2.2\\
4	2.3\\
4	2.4\\
4	2.5\\
4	2.6\\
4	2.7\\
4	2.8\\
4	2.9\\
4	3\\
4	3.1\\
4	3.2\\
4	3.3\\
4	3.4\\
4	3.5\\
4	3.6\\
4	3.7\\
4	3.8\\
4	3.9\\
4	4\\
4	4.1\\
4	4.2\\
4	4.3\\
4	4.4\\
4	4.5\\
4	4.6\\
4	4.7\\
4	4.8\\
4	4.9\\
4	5\\
4	5.1\\
4	5.2\\
4	5.3\\
4	5.4\\
4	5.5\\
4	5.6\\
4	5.7\\
4	5.8\\
4	5.9\\
};
\addplot [color=black, draw=none, mark size=0.2pt, mark=*, mark options={solid, black}, forget plot]
  table[row sep=crcr]{%
4.1	0.1\\
4.1	0.2\\
4.1	0.3\\
4.1	0.4\\
4.1	0.5\\
4.1	0.6\\
4.1	0.7\\
4.1	0.8\\
4.1	0.9\\
4.1	1\\
4.1	1.1\\
4.1	1.2\\
4.1	1.3\\
4.1	1.4\\
4.1	1.5\\
4.1	1.6\\
4.1	1.7\\
4.1	1.8\\
4.1	1.9\\
4.1	2\\
4.1	2.1\\
4.1	2.2\\
4.1	2.3\\
4.1	2.4\\
4.1	2.5\\
4.1	2.6\\
4.1	2.7\\
4.1	2.8\\
4.1	2.9\\
4.1	3\\
4.1	3.1\\
4.1	3.2\\
4.1	3.3\\
4.1	3.4\\
4.1	3.5\\
4.1	3.6\\
4.1	3.7\\
4.1	3.8\\
4.1	3.9\\
4.1	4\\
4.1	4.1\\
4.1	4.2\\
4.1	4.3\\
4.1	4.4\\
4.1	4.5\\
4.1	4.6\\
4.1	4.7\\
4.1	4.8\\
4.1	4.9\\
4.1	5\\
4.1	5.1\\
4.1	5.2\\
4.1	5.3\\
4.1	5.4\\
4.1	5.5\\
4.1	5.6\\
4.1	5.7\\
4.1	5.8\\
4.1	5.9\\
};
\addplot [color=black, draw=none, mark size=0.2pt, mark=*, mark options={solid, black}, forget plot]
  table[row sep=crcr]{%
4.2	0.1\\
4.2	0.2\\
4.2	0.3\\
4.2	0.4\\
4.2	0.5\\
4.2	0.6\\
4.2	0.7\\
4.2	0.8\\
4.2	0.9\\
4.2	1\\
4.2	1.1\\
4.2	1.2\\
4.2	1.3\\
4.2	1.4\\
4.2	1.5\\
4.2	1.6\\
4.2	1.7\\
4.2	1.8\\
4.2	1.9\\
4.2	2\\
4.2	2.1\\
4.2	2.2\\
4.2	2.3\\
4.2	2.4\\
4.2	2.5\\
4.2	2.6\\
4.2	2.7\\
4.2	2.8\\
4.2	2.9\\
4.2	3\\
4.2	3.1\\
4.2	3.2\\
4.2	3.3\\
4.2	3.4\\
4.2	3.5\\
4.2	3.6\\
4.2	3.7\\
4.2	3.8\\
4.2	3.9\\
4.2	4\\
4.2	4.1\\
4.2	4.2\\
4.2	4.3\\
4.2	4.4\\
4.2	4.5\\
4.2	4.6\\
4.2	4.7\\
4.2	4.8\\
4.2	4.9\\
4.2	5\\
4.2	5.1\\
4.2	5.2\\
4.2	5.3\\
4.2	5.4\\
4.2	5.5\\
4.2	5.6\\
4.2	5.7\\
4.2	5.8\\
4.2	5.9\\
};
\addplot [color=black, draw=none, mark size=0.2pt, mark=*, mark options={solid, black}, forget plot]
  table[row sep=crcr]{%
4.3	0.1\\
4.3	0.2\\
4.3	0.3\\
4.3	0.4\\
4.3	0.5\\
4.3	0.6\\
4.3	0.7\\
4.3	0.8\\
4.3	0.9\\
4.3	1\\
4.3	1.1\\
4.3	1.2\\
4.3	1.3\\
4.3	1.4\\
4.3	1.5\\
4.3	1.6\\
4.3	1.7\\
4.3	1.8\\
4.3	1.9\\
4.3	2\\
4.3	2.1\\
4.3	2.2\\
4.3	2.3\\
4.3	2.4\\
4.3	2.5\\
4.3	2.6\\
4.3	2.7\\
4.3	2.8\\
4.3	2.9\\
4.3	3\\
4.3	3.1\\
4.3	3.2\\
4.3	3.3\\
4.3	3.4\\
4.3	3.5\\
4.3	3.6\\
4.3	3.7\\
4.3	3.8\\
4.3	3.9\\
4.3	4\\
4.3	4.1\\
4.3	4.2\\
4.3	4.3\\
4.3	4.4\\
4.3	4.5\\
4.3	4.6\\
4.3	4.7\\
4.3	4.8\\
4.3	4.9\\
4.3	5\\
4.3	5.1\\
4.3	5.2\\
4.3	5.3\\
4.3	5.4\\
4.3	5.5\\
4.3	5.6\\
4.3	5.7\\
4.3	5.8\\
4.3	5.9\\
};
\addplot [color=black, draw=none, mark size=0.2pt, mark=*, mark options={solid, black}, forget plot]
  table[row sep=crcr]{%
4.4	0.1\\
4.4	0.2\\
4.4	0.3\\
4.4	0.4\\
4.4	0.5\\
4.4	0.6\\
4.4	0.7\\
4.4	0.8\\
4.4	0.9\\
4.4	1\\
4.4	1.1\\
4.4	1.2\\
4.4	1.3\\
4.4	1.4\\
4.4	1.5\\
4.4	1.6\\
4.4	1.7\\
4.4	1.8\\
4.4	1.9\\
4.4	2\\
4.4	2.1\\
4.4	2.2\\
4.4	2.3\\
4.4	2.4\\
4.4	2.5\\
4.4	2.6\\
4.4	2.7\\
4.4	2.8\\
4.4	2.9\\
4.4	3\\
4.4	3.1\\
4.4	3.2\\
4.4	3.3\\
4.4	3.4\\
4.4	3.5\\
4.4	3.6\\
4.4	3.7\\
4.4	3.8\\
4.4	3.9\\
4.4	4\\
4.4	4.1\\
4.4	4.2\\
4.4	4.3\\
4.4	4.4\\
4.4	4.5\\
4.4	4.6\\
4.4	4.7\\
4.4	4.8\\
4.4	4.9\\
4.4	5\\
4.4	5.1\\
4.4	5.2\\
4.4	5.3\\
4.4	5.4\\
4.4	5.5\\
4.4	5.6\\
4.4	5.7\\
4.4	5.8\\
4.4	5.9\\
};
\addplot [color=black, draw=none, mark size=0.2pt, mark=*, mark options={solid, black}, forget plot]
  table[row sep=crcr]{%
4.5	0.1\\
4.5	0.2\\
4.5	0.3\\
4.5	0.4\\
4.5	0.5\\
4.5	0.6\\
4.5	0.7\\
4.5	0.8\\
4.5	0.9\\
4.5	1\\
4.5	1.1\\
4.5	1.2\\
4.5	1.3\\
4.5	1.4\\
4.5	1.5\\
4.5	1.6\\
4.5	1.7\\
4.5	1.8\\
4.5	1.9\\
4.5	2\\
4.5	2.1\\
4.5	2.2\\
4.5	2.3\\
4.5	2.4\\
4.5	2.5\\
4.5	2.6\\
4.5	2.7\\
4.5	2.8\\
4.5	2.9\\
4.5	3\\
4.5	3.1\\
4.5	3.2\\
4.5	3.3\\
4.5	3.4\\
4.5	3.5\\
4.5	3.6\\
4.5	3.7\\
4.5	3.8\\
4.5	3.9\\
4.5	4\\
4.5	4.1\\
4.5	4.2\\
4.5	4.3\\
4.5	4.4\\
4.5	4.5\\
4.5	4.6\\
4.5	4.7\\
4.5	4.8\\
4.5	4.9\\
4.5	5\\
4.5	5.1\\
4.5	5.2\\
4.5	5.3\\
4.5	5.4\\
4.5	5.5\\
4.5	5.6\\
4.5	5.7\\
4.5	5.8\\
4.5	5.9\\
};
\addplot [color=black, draw=none, mark size=0.2pt, mark=*, mark options={solid, black}, forget plot]
  table[row sep=crcr]{%
4.6	0.1\\
4.6	0.2\\
4.6	0.3\\
4.6	0.4\\
4.6	0.5\\
4.6	0.6\\
4.6	0.7\\
4.6	0.8\\
4.6	0.9\\
4.6	1\\
4.6	1.1\\
4.6	1.2\\
4.6	1.3\\
4.6	1.4\\
4.6	1.5\\
4.6	1.6\\
4.6	1.7\\
4.6	1.8\\
4.6	1.9\\
4.6	2\\
4.6	2.1\\
4.6	2.2\\
4.6	2.3\\
4.6	2.4\\
4.6	2.5\\
4.6	2.6\\
4.6	2.7\\
4.6	2.8\\
4.6	2.9\\
4.6	3\\
4.6	3.1\\
4.6	3.2\\
4.6	3.3\\
4.6	3.4\\
4.6	3.5\\
4.6	3.6\\
4.6	3.7\\
4.6	3.8\\
4.6	3.9\\
4.6	4\\
4.6	4.1\\
4.6	4.2\\
4.6	4.3\\
4.6	4.4\\
4.6	4.5\\
4.6	4.6\\
4.6	4.7\\
4.6	4.8\\
4.6	4.9\\
4.6	5\\
4.6	5.1\\
4.6	5.2\\
4.6	5.3\\
4.6	5.4\\
4.6	5.5\\
4.6	5.6\\
4.6	5.7\\
4.6	5.8\\
4.6	5.9\\
};
\addplot [color=black, draw=none, mark size=0.2pt, mark=*, mark options={solid, black}, forget plot]
  table[row sep=crcr]{%
4.7	0.1\\
4.7	0.2\\
4.7	0.3\\
4.7	0.4\\
4.7	0.5\\
4.7	0.6\\
4.7	0.7\\
4.7	0.8\\
4.7	0.9\\
4.7	1\\
4.7	1.1\\
4.7	1.2\\
4.7	1.3\\
4.7	1.4\\
4.7	1.5\\
4.7	1.6\\
4.7	1.7\\
4.7	1.8\\
4.7	1.9\\
4.7	2\\
4.7	2.1\\
4.7	2.2\\
4.7	2.3\\
4.7	2.4\\
4.7	2.5\\
4.7	2.6\\
4.7	2.7\\
4.7	2.8\\
4.7	2.9\\
4.7	3\\
4.7	3.1\\
4.7	3.2\\
4.7	3.3\\
4.7	3.4\\
4.7	3.5\\
4.7	3.6\\
4.7	3.7\\
4.7	3.8\\
4.7	3.9\\
4.7	4\\
4.7	4.1\\
4.7	4.2\\
4.7	4.3\\
4.7	4.4\\
4.7	4.5\\
4.7	4.6\\
4.7	4.7\\
4.7	4.8\\
4.7	4.9\\
4.7	5\\
4.7	5.1\\
4.7	5.2\\
4.7	5.3\\
4.7	5.4\\
4.7	5.5\\
4.7	5.6\\
4.7	5.7\\
4.7	5.8\\
4.7	5.9\\
};
\addplot [color=black, draw=none, mark size=0.2pt, mark=*, mark options={solid, black}, forget plot]
  table[row sep=crcr]{%
4.8	0.1\\
4.8	0.2\\
4.8	0.3\\
4.8	0.4\\
4.8	0.5\\
4.8	0.6\\
4.8	0.7\\
4.8	0.8\\
4.8	0.9\\
4.8	1\\
4.8	1.1\\
4.8	1.2\\
4.8	1.3\\
4.8	1.4\\
4.8	1.5\\
4.8	1.6\\
4.8	1.7\\
4.8	1.8\\
4.8	1.9\\
4.8	2\\
4.8	2.1\\
4.8	2.2\\
4.8	2.3\\
4.8	2.4\\
4.8	2.5\\
4.8	2.6\\
4.8	2.7\\
4.8	2.8\\
4.8	2.9\\
4.8	3\\
4.8	3.1\\
4.8	3.2\\
4.8	3.3\\
4.8	3.4\\
4.8	3.5\\
4.8	3.6\\
4.8	3.7\\
4.8	3.8\\
4.8	3.9\\
4.8	4\\
4.8	4.1\\
4.8	4.2\\
4.8	4.3\\
4.8	4.4\\
4.8	4.5\\
4.8	4.6\\
4.8	4.7\\
4.8	4.8\\
4.8	4.9\\
4.8	5\\
4.8	5.1\\
4.8	5.2\\
4.8	5.3\\
4.8	5.4\\
4.8	5.5\\
4.8	5.6\\
4.8	5.7\\
4.8	5.8\\
4.8	5.9\\
};
\addplot [color=black, draw=none, mark size=0.2pt, mark=*, mark options={solid, black}, forget plot]
  table[row sep=crcr]{%
4.9	0.1\\
4.9	0.2\\
4.9	0.3\\
4.9	0.4\\
4.9	0.5\\
4.9	0.6\\
4.9	0.7\\
4.9	0.8\\
4.9	0.9\\
4.9	1\\
4.9	1.1\\
4.9	1.2\\
4.9	1.3\\
4.9	1.4\\
4.9	1.5\\
4.9	1.6\\
4.9	1.7\\
4.9	1.8\\
4.9	1.9\\
4.9	2\\
4.9	2.1\\
4.9	2.2\\
4.9	2.3\\
4.9	2.4\\
4.9	2.5\\
4.9	2.6\\
4.9	2.7\\
4.9	2.8\\
4.9	2.9\\
4.9	3\\
4.9	3.1\\
4.9	3.2\\
4.9	3.3\\
4.9	3.4\\
4.9	3.5\\
4.9	3.6\\
4.9	3.7\\
4.9	3.8\\
4.9	3.9\\
4.9	4\\
4.9	4.1\\
4.9	4.2\\
4.9	4.3\\
4.9	4.4\\
4.9	4.5\\
4.9	4.6\\
4.9	4.7\\
4.9	4.8\\
4.9	4.9\\
4.9	5\\
4.9	5.1\\
4.9	5.2\\
4.9	5.3\\
4.9	5.4\\
4.9	5.5\\
4.9	5.6\\
4.9	5.7\\
4.9	5.8\\
4.9	5.9\\
};
\addplot [color=black, draw=none, mark size=0.2pt, mark=*, mark options={solid, black}, forget plot]
  table[row sep=crcr]{%
5	0.1\\
5	0.2\\
5	0.3\\
5	0.4\\
5	0.5\\
5	0.6\\
5	0.7\\
5	0.8\\
5	0.9\\
5	1\\
5	1.1\\
5	1.2\\
5	1.3\\
5	1.4\\
5	1.5\\
5	1.6\\
5	1.7\\
5	1.8\\
5	1.9\\
5	2\\
5	2.1\\
5	2.2\\
5	2.3\\
5	2.4\\
5	2.5\\
5	2.6\\
5	2.7\\
5	2.8\\
5	2.9\\
5	3\\
5	3.1\\
5	3.2\\
5	3.3\\
5	3.4\\
5	3.5\\
5	3.6\\
5	3.7\\
5	3.8\\
5	3.9\\
5	4\\
5	4.1\\
5	4.2\\
5	4.3\\
5	4.4\\
5	4.5\\
5	4.6\\
5	4.7\\
5	4.8\\
5	4.9\\
5	5\\
5	5.1\\
5	5.2\\
5	5.3\\
5	5.4\\
5	5.5\\
5	5.6\\
5	5.7\\
5	5.8\\
5	5.9\\
};
\addplot [color=black, draw=none, mark size=0.2pt, mark=*, mark options={solid, black}, forget plot]
  table[row sep=crcr]{%
5.1	0.1\\
5.1	0.2\\
5.1	0.3\\
5.1	0.4\\
5.1	0.5\\
5.1	0.6\\
5.1	0.7\\
5.1	0.8\\
5.1	0.9\\
5.1	1\\
5.1	1.1\\
5.1	1.2\\
5.1	1.3\\
5.1	1.4\\
5.1	1.5\\
5.1	1.6\\
5.1	1.7\\
5.1	1.8\\
5.1	1.9\\
5.1	2\\
5.1	2.1\\
5.1	2.2\\
5.1	2.3\\
5.1	2.4\\
5.1	2.5\\
5.1	2.6\\
5.1	2.7\\
5.1	2.8\\
5.1	2.9\\
5.1	3\\
5.1	3.1\\
5.1	3.2\\
5.1	3.3\\
5.1	3.4\\
5.1	3.5\\
5.1	3.6\\
5.1	3.7\\
5.1	3.8\\
5.1	3.9\\
5.1	4\\
5.1	4.1\\
5.1	4.2\\
5.1	4.3\\
5.1	4.4\\
5.1	4.5\\
5.1	4.6\\
5.1	4.7\\
5.1	4.8\\
5.1	4.9\\
5.1	5\\
5.1	5.1\\
5.1	5.2\\
5.1	5.3\\
5.1	5.4\\
5.1	5.5\\
5.1	5.6\\
5.1	5.7\\
5.1	5.8\\
5.1	5.9\\
};
\addplot [color=black, draw=none, mark size=0.2pt, mark=*, mark options={solid, black}, forget plot]
  table[row sep=crcr]{%
5.2	0.1\\
5.2	0.2\\
5.2	0.3\\
5.2	0.4\\
5.2	0.5\\
5.2	0.6\\
5.2	0.7\\
5.2	0.8\\
5.2	0.9\\
5.2	1\\
5.2	1.1\\
5.2	1.2\\
5.2	1.3\\
5.2	1.4\\
5.2	1.5\\
5.2	1.6\\
5.2	1.7\\
5.2	1.8\\
5.2	1.9\\
5.2	2\\
5.2	2.1\\
5.2	2.2\\
5.2	2.3\\
5.2	2.4\\
5.2	2.5\\
5.2	2.6\\
5.2	2.7\\
5.2	2.8\\
5.2	2.9\\
5.2	3\\
5.2	3.1\\
5.2	3.2\\
5.2	3.3\\
5.2	3.4\\
5.2	3.5\\
5.2	3.6\\
5.2	3.7\\
5.2	3.8\\
5.2	3.9\\
5.2	4\\
5.2	4.1\\
5.2	4.2\\
5.2	4.3\\
5.2	4.4\\
5.2	4.5\\
5.2	4.6\\
5.2	4.7\\
5.2	4.8\\
5.2	4.9\\
5.2	5\\
5.2	5.1\\
5.2	5.2\\
5.2	5.3\\
5.2	5.4\\
5.2	5.5\\
5.2	5.6\\
5.2	5.7\\
5.2	5.8\\
5.2	5.9\\
};
\addplot [color=black, draw=none, mark size=0.2pt, mark=*, mark options={solid, black}, forget plot]
  table[row sep=crcr]{%
5.3	0.1\\
5.3	0.2\\
5.3	0.3\\
5.3	0.4\\
5.3	0.5\\
5.3	0.6\\
5.3	0.7\\
5.3	0.8\\
5.3	0.9\\
5.3	1\\
5.3	1.1\\
5.3	1.2\\
5.3	1.3\\
5.3	1.4\\
5.3	1.5\\
5.3	1.6\\
5.3	1.7\\
5.3	1.8\\
5.3	1.9\\
5.3	2\\
5.3	2.1\\
5.3	2.2\\
5.3	2.3\\
5.3	2.4\\
5.3	2.5\\
5.3	2.6\\
5.3	2.7\\
5.3	2.8\\
5.3	2.9\\
5.3	3\\
5.3	3.1\\
5.3	3.2\\
5.3	3.3\\
5.3	3.4\\
5.3	3.5\\
5.3	3.6\\
5.3	3.7\\
5.3	3.8\\
5.3	3.9\\
5.3	4\\
5.3	4.1\\
5.3	4.2\\
5.3	4.3\\
5.3	4.4\\
5.3	4.5\\
5.3	4.6\\
5.3	4.7\\
5.3	4.8\\
5.3	4.9\\
5.3	5\\
5.3	5.1\\
5.3	5.2\\
5.3	5.3\\
5.3	5.4\\
5.3	5.5\\
5.3	5.6\\
5.3	5.7\\
5.3	5.8\\
5.3	5.9\\
};
\addplot [color=black, draw=none, mark size=0.2pt, mark=*, mark options={solid, black}, forget plot]
  table[row sep=crcr]{%
5.4	0.1\\
5.4	0.2\\
5.4	0.3\\
5.4	0.4\\
5.4	0.5\\
5.4	0.6\\
5.4	0.7\\
5.4	0.8\\
5.4	0.9\\
5.4	1\\
5.4	1.1\\
5.4	1.2\\
5.4	1.3\\
5.4	1.4\\
5.4	1.5\\
5.4	1.6\\
5.4	1.7\\
5.4	1.8\\
5.4	1.9\\
5.4	2\\
5.4	2.1\\
5.4	2.2\\
5.4	2.3\\
5.4	2.4\\
5.4	2.5\\
5.4	2.6\\
5.4	2.7\\
5.4	2.8\\
5.4	2.9\\
5.4	3\\
5.4	3.1\\
5.4	3.2\\
5.4	3.3\\
5.4	3.4\\
5.4	3.5\\
5.4	3.6\\
5.4	3.7\\
5.4	3.8\\
5.4	3.9\\
5.4	4\\
5.4	4.1\\
5.4	4.2\\
5.4	4.3\\
5.4	4.4\\
5.4	4.5\\
5.4	4.6\\
5.4	4.7\\
5.4	4.8\\
5.4	4.9\\
5.4	5\\
5.4	5.1\\
5.4	5.2\\
5.4	5.3\\
5.4	5.4\\
5.4	5.5\\
5.4	5.6\\
5.4	5.7\\
5.4	5.8\\
5.4	5.9\\
};
\addplot [color=black, draw=none, mark size=0.2pt, mark=*, mark options={solid, black}, forget plot]
  table[row sep=crcr]{%
5.5	0.1\\
5.5	0.2\\
5.5	0.3\\
5.5	0.4\\
5.5	0.5\\
5.5	0.6\\
5.5	0.7\\
5.5	0.8\\
5.5	0.9\\
5.5	1\\
5.5	1.1\\
5.5	1.2\\
5.5	1.3\\
5.5	1.4\\
5.5	1.5\\
5.5	1.6\\
5.5	1.7\\
5.5	1.8\\
5.5	1.9\\
5.5	2\\
5.5	2.1\\
5.5	2.2\\
5.5	2.3\\
5.5	2.4\\
5.5	2.5\\
5.5	2.6\\
5.5	2.7\\
5.5	2.8\\
5.5	2.9\\
5.5	3\\
5.5	3.1\\
5.5	3.2\\
5.5	3.3\\
5.5	3.4\\
5.5	3.5\\
5.5	3.6\\
5.5	3.7\\
5.5	3.8\\
5.5	3.9\\
5.5	4\\
5.5	4.1\\
5.5	4.2\\
5.5	4.3\\
5.5	4.4\\
5.5	4.5\\
5.5	4.6\\
5.5	4.7\\
5.5	4.8\\
5.5	4.9\\
5.5	5\\
5.5	5.1\\
5.5	5.2\\
5.5	5.3\\
5.5	5.4\\
5.5	5.5\\
5.5	5.6\\
5.5	5.7\\
5.5	5.8\\
5.5	5.9\\
};
\addplot [color=black, draw=none, mark size=0.2pt, mark=*, mark options={solid, black}, forget plot]
  table[row sep=crcr]{%
5.6	0.1\\
5.6	0.2\\
5.6	0.3\\
5.6	0.4\\
5.6	0.5\\
5.6	0.6\\
5.6	0.7\\
5.6	0.8\\
5.6	0.9\\
5.6	1\\
5.6	1.1\\
5.6	1.2\\
5.6	1.3\\
5.6	1.4\\
5.6	1.5\\
5.6	1.6\\
5.6	1.7\\
5.6	1.8\\
5.6	1.9\\
5.6	2\\
5.6	2.1\\
5.6	2.2\\
5.6	2.3\\
5.6	2.4\\
5.6	2.5\\
5.6	2.6\\
5.6	2.7\\
5.6	2.8\\
5.6	2.9\\
5.6	3\\
5.6	3.1\\
5.6	3.2\\
5.6	3.3\\
5.6	3.4\\
5.6	3.5\\
5.6	3.6\\
5.6	3.7\\
5.6	3.8\\
5.6	3.9\\
5.6	4\\
5.6	4.1\\
5.6	4.2\\
5.6	4.3\\
5.6	4.4\\
5.6	4.5\\
5.6	4.6\\
5.6	4.7\\
5.6	4.8\\
5.6	4.9\\
5.6	5\\
5.6	5.1\\
5.6	5.2\\
5.6	5.3\\
5.6	5.4\\
5.6	5.5\\
5.6	5.6\\
5.6	5.7\\
5.6	5.8\\
5.6	5.9\\
};
\addplot [color=black, draw=none, mark size=0.2pt, mark=*, mark options={solid, black}, forget plot]
  table[row sep=crcr]{%
5.7	0.1\\
5.7	0.2\\
5.7	0.3\\
5.7	0.4\\
5.7	0.5\\
5.7	0.6\\
5.7	0.7\\
5.7	0.8\\
5.7	0.9\\
5.7	1\\
5.7	1.1\\
5.7	1.2\\
5.7	1.3\\
5.7	1.4\\
5.7	1.5\\
5.7	1.6\\
5.7	1.7\\
5.7	1.8\\
5.7	1.9\\
5.7	2\\
5.7	2.1\\
5.7	2.2\\
5.7	2.3\\
5.7	2.4\\
5.7	2.5\\
5.7	2.6\\
5.7	2.7\\
5.7	2.8\\
5.7	2.9\\
5.7	3\\
5.7	3.1\\
5.7	3.2\\
5.7	3.3\\
5.7	3.4\\
5.7	3.5\\
5.7	3.6\\
5.7	3.7\\
5.7	3.8\\
5.7	3.9\\
5.7	4\\
5.7	4.1\\
5.7	4.2\\
5.7	4.3\\
5.7	4.4\\
5.7	4.5\\
5.7	4.6\\
5.7	4.7\\
5.7	4.8\\
5.7	4.9\\
5.7	5\\
5.7	5.1\\
5.7	5.2\\
5.7	5.3\\
5.7	5.4\\
5.7	5.5\\
5.7	5.6\\
5.7	5.7\\
5.7	5.8\\
5.7	5.9\\
};
\addplot [color=black, draw=none, mark size=0.2pt, mark=*, mark options={solid, black}, forget plot]
  table[row sep=crcr]{%
5.8	0.1\\
5.8	0.2\\
5.8	0.3\\
5.8	0.4\\
5.8	0.5\\
5.8	0.6\\
5.8	0.7\\
5.8	0.8\\
5.8	0.9\\
5.8	1\\
5.8	1.1\\
5.8	1.2\\
5.8	1.3\\
5.8	1.4\\
5.8	1.5\\
5.8	1.6\\
5.8	1.7\\
5.8	1.8\\
5.8	1.9\\
5.8	2\\
5.8	2.1\\
5.8	2.2\\
5.8	2.3\\
5.8	2.4\\
5.8	2.5\\
5.8	2.6\\
5.8	2.7\\
5.8	2.8\\
5.8	2.9\\
5.8	3\\
5.8	3.1\\
5.8	3.2\\
5.8	3.3\\
5.8	3.4\\
5.8	3.5\\
5.8	3.6\\
5.8	3.7\\
5.8	3.8\\
5.8	3.9\\
5.8	4\\
5.8	4.1\\
5.8	4.2\\
5.8	4.3\\
5.8	4.4\\
5.8	4.5\\
5.8	4.6\\
5.8	4.7\\
5.8	4.8\\
5.8	4.9\\
5.8	5\\
5.8	5.1\\
5.8	5.2\\
5.8	5.3\\
5.8	5.4\\
5.8	5.5\\
5.8	5.6\\
5.8	5.7\\
5.8	5.8\\
5.8	5.9\\
};
\addplot [color=black, draw=none, mark size=0.2pt, mark=*, mark options={solid, black}, forget plot]
  table[row sep=crcr]{%
5.9	0.1\\
5.9	0.2\\
5.9	0.3\\
5.9	0.4\\
5.9	0.5\\
5.9	0.6\\
5.9	0.7\\
5.9	0.8\\
5.9	0.9\\
5.9	1\\
5.9	1.1\\
5.9	1.2\\
5.9	1.3\\
5.9	1.4\\
5.9	1.5\\
5.9	1.6\\
5.9	1.7\\
5.9	1.8\\
5.9	1.9\\
5.9	2\\
5.9	2.1\\
5.9	2.2\\
5.9	2.3\\
5.9	2.4\\
5.9	2.5\\
5.9	2.6\\
5.9	2.7\\
5.9	2.8\\
5.9	2.9\\
5.9	3\\
5.9	3.1\\
5.9	3.2\\
5.9	3.3\\
5.9	3.4\\
5.9	3.5\\
5.9	3.6\\
5.9	3.7\\
5.9	3.8\\
5.9	3.9\\
5.9	4\\
5.9	4.1\\
5.9	4.2\\
5.9	4.3\\
5.9	4.4\\
5.9	4.5\\
5.9	4.6\\
5.9	4.7\\
5.9	4.8\\
5.9	4.9\\
5.9	5\\
5.9	5.1\\
5.9	5.2\\
5.9	5.3\\
5.9	5.4\\
5.9	5.5\\
5.9	5.6\\
5.9	5.7\\
5.9	5.8\\
5.9	5.9\\
};
\addplot [color=white, draw=none, mark size=0.2pt, mark=*, mark options={solid, white}, forget plot]
  table[row sep=crcr]{%
1.2	1.2\\
1.2	1.3\\
1.2	1.4\\
1.2	1.5\\
1.2	1.6\\
1.2	1.7\\
1.2	1.8\\
1.2	1.9\\
1.2	2\\
1.2	2.1\\
1.2	2.2\\
1.2	2.3\\
1.2	2.4\\
1.2	2.5\\
1.2	2.6\\
1.2	2.7\\
1.2	2.8\\
1.2	2.9\\
1.2	3\\
1.2	3.1\\
1.2	3.2\\
1.2	3.3\\
1.2	3.4\\
1.2	3.5\\
1.2	3.6\\
1.2	3.7\\
1.2	3.8\\
1.2	3.9\\
1.2	4\\
1.2	4.1\\
1.2	4.2\\
1.2	4.3\\
1.2	4.4\\
1.2	4.5\\
1.2	4.6\\
1.2	4.7\\
1.2	4.8\\
};
\addplot [color=white, draw=none, mark size=0.2pt, mark=*, mark options={solid, white}, forget plot]
  table[row sep=crcr]{%
1.3	1.2\\
1.3	1.3\\
1.3	1.4\\
1.3	1.5\\
1.3	1.6\\
1.3	1.7\\
1.3	1.8\\
1.3	1.9\\
1.3	2\\
1.3	2.1\\
1.3	2.2\\
1.3	2.3\\
1.3	2.4\\
1.3	2.5\\
1.3	2.6\\
1.3	2.7\\
1.3	2.8\\
1.3	2.9\\
1.3	3\\
1.3	3.1\\
1.3	3.2\\
1.3	3.3\\
1.3	3.4\\
1.3	3.5\\
1.3	3.6\\
1.3	3.7\\
1.3	3.8\\
1.3	3.9\\
1.3	4\\
1.3	4.1\\
1.3	4.2\\
1.3	4.3\\
1.3	4.4\\
1.3	4.5\\
1.3	4.6\\
1.3	4.7\\
1.3	4.8\\
};
\addplot [color=white, draw=none, mark size=0.2pt, mark=*, mark options={solid, white}, forget plot]
  table[row sep=crcr]{%
1.4	1.2\\
1.4	1.3\\
1.4	1.4\\
1.4	1.5\\
1.4	1.6\\
1.4	1.7\\
1.4	1.8\\
1.4	1.9\\
1.4	2\\
1.4	2.1\\
1.4	2.2\\
1.4	2.3\\
1.4	2.4\\
1.4	2.5\\
1.4	2.6\\
1.4	2.7\\
1.4	2.8\\
1.4	2.9\\
1.4	3\\
1.4	3.1\\
1.4	3.2\\
1.4	3.3\\
1.4	3.4\\
1.4	3.5\\
1.4	3.6\\
1.4	3.7\\
1.4	3.8\\
1.4	3.9\\
1.4	4\\
1.4	4.1\\
1.4	4.2\\
1.4	4.3\\
1.4	4.4\\
1.4	4.5\\
1.4	4.6\\
1.4	4.7\\
1.4	4.8\\
};
\addplot [color=white, draw=none, mark size=0.2pt, mark=*, mark options={solid, white}, forget plot]
  table[row sep=crcr]{%
1.5	1.2\\
1.5	1.3\\
1.5	1.4\\
1.5	1.5\\
1.5	1.6\\
1.5	1.7\\
1.5	1.8\\
1.5	1.9\\
1.5	2\\
1.5	2.1\\
1.5	2.2\\
1.5	2.3\\
1.5	2.4\\
1.5	2.5\\
1.5	2.6\\
1.5	2.7\\
1.5	2.8\\
1.5	2.9\\
1.5	3\\
1.5	3.1\\
1.5	3.2\\
1.5	3.3\\
1.5	3.4\\
1.5	3.5\\
1.5	3.6\\
1.5	3.7\\
1.5	3.8\\
1.5	3.9\\
1.5	4\\
1.5	4.1\\
1.5	4.2\\
1.5	4.3\\
1.5	4.4\\
1.5	4.5\\
1.5	4.6\\
1.5	4.7\\
1.5	4.8\\
};
\addplot [color=white, draw=none, mark size=0.2pt, mark=*, mark options={solid, white}, forget plot]
  table[row sep=crcr]{%
1.6	1.2\\
1.6	1.3\\
1.6	1.4\\
1.6	1.5\\
1.6	1.6\\
1.6	1.7\\
1.6	1.8\\
1.6	1.9\\
1.6	2\\
1.6	2.1\\
1.6	2.2\\
1.6	2.3\\
1.6	2.4\\
1.6	2.5\\
1.6	2.6\\
1.6	2.7\\
1.6	2.8\\
1.6	2.9\\
1.6	3\\
1.6	3.1\\
1.6	3.2\\
1.6	3.3\\
1.6	3.4\\
1.6	3.5\\
1.6	3.6\\
1.6	3.7\\
1.6	3.8\\
1.6	3.9\\
1.6	4\\
1.6	4.1\\
1.6	4.2\\
1.6	4.3\\
1.6	4.4\\
1.6	4.5\\
1.6	4.6\\
1.6	4.7\\
1.6	4.8\\
};
\addplot [color=white, draw=none, mark size=0.2pt, mark=*, mark options={solid, white}, forget plot]
  table[row sep=crcr]{%
1.7	1.2\\
1.7	1.3\\
1.7	1.4\\
1.7	1.5\\
1.7	1.6\\
1.7	1.7\\
1.7	1.8\\
1.7	1.9\\
1.7	2\\
1.7	2.1\\
1.7	2.2\\
1.7	2.3\\
1.7	2.4\\
1.7	2.5\\
1.7	2.6\\
1.7	2.7\\
1.7	2.8\\
1.7	2.9\\
1.7	3\\
1.7	3.1\\
1.7	3.2\\
1.7	3.3\\
1.7	3.4\\
1.7	3.5\\
1.7	3.6\\
1.7	3.7\\
1.7	3.8\\
1.7	3.9\\
1.7	4\\
1.7	4.1\\
1.7	4.2\\
1.7	4.3\\
1.7	4.4\\
1.7	4.5\\
1.7	4.6\\
1.7	4.7\\
1.7	4.8\\
};
\addplot [color=white, draw=none, mark size=0.2pt, mark=*, mark options={solid, white}, forget plot]
  table[row sep=crcr]{%
1.8	1.2\\
1.8	1.3\\
1.8	1.4\\
1.8	1.5\\
1.8	1.6\\
1.8	1.7\\
1.8	1.8\\
1.8	1.9\\
1.8	2\\
1.8	2.1\\
1.8	2.2\\
1.8	2.3\\
1.8	2.4\\
1.8	2.5\\
1.8	2.6\\
1.8	2.7\\
1.8	2.8\\
1.8	2.9\\
1.8	3\\
1.8	3.1\\
1.8	3.2\\
1.8	3.3\\
1.8	3.4\\
1.8	3.5\\
1.8	3.6\\
1.8	3.7\\
1.8	3.8\\
1.8	3.9\\
1.8	4\\
1.8	4.1\\
1.8	4.2\\
1.8	4.3\\
1.8	4.4\\
1.8	4.5\\
1.8	4.6\\
1.8	4.7\\
1.8	4.8\\
};
\addplot [color=white, draw=none, mark size=0.2pt, mark=*, mark options={solid, white}, forget plot]
  table[row sep=crcr]{%
1.9	1.2\\
1.9	1.3\\
1.9	1.4\\
1.9	1.5\\
1.9	1.6\\
1.9	1.7\\
1.9	1.8\\
1.9	1.9\\
1.9	2\\
1.9	2.1\\
1.9	2.2\\
1.9	2.3\\
1.9	2.4\\
1.9	2.5\\
1.9	2.6\\
1.9	2.7\\
1.9	2.8\\
1.9	2.9\\
1.9	3\\
1.9	3.1\\
1.9	3.2\\
1.9	3.3\\
1.9	3.4\\
1.9	3.5\\
1.9	3.6\\
1.9	3.7\\
1.9	3.8\\
1.9	3.9\\
1.9	4\\
1.9	4.1\\
1.9	4.2\\
1.9	4.3\\
1.9	4.4\\
1.9	4.5\\
1.9	4.6\\
1.9	4.7\\
1.9	4.8\\
};
\addplot [color=white, draw=none, mark size=0.2pt, mark=*, mark options={solid, white}, forget plot]
  table[row sep=crcr]{%
2	1.2\\
2	1.3\\
2	1.4\\
2	1.5\\
2	1.6\\
2	1.7\\
2	1.8\\
2	1.9\\
2	2\\
2	2.1\\
2	2.2\\
2	2.3\\
2	2.4\\
2	2.5\\
2	2.6\\
2	2.7\\
2	2.8\\
2	2.9\\
2	3\\
2	3.1\\
2	3.2\\
2	3.3\\
2	3.4\\
2	3.5\\
2	3.6\\
2	3.7\\
2	3.8\\
2	3.9\\
2	4\\
2	4.1\\
2	4.2\\
2	4.3\\
2	4.4\\
2	4.5\\
2	4.6\\
2	4.7\\
2	4.8\\
};
\addplot [color=white, draw=none, mark size=0.2pt, mark=*, mark options={solid, white}, forget plot]
  table[row sep=crcr]{%
2.1	1.2\\
2.1	1.3\\
2.1	1.4\\
2.1	1.5\\
2.1	1.6\\
2.1	1.7\\
2.1	1.8\\
2.1	1.9\\
2.1	2\\
2.1	2.1\\
2.1	2.2\\
2.1	2.3\\
2.1	2.4\\
2.1	2.5\\
2.1	2.6\\
2.1	2.7\\
2.1	2.8\\
2.1	2.9\\
2.1	3\\
2.1	3.1\\
2.1	3.2\\
2.1	3.3\\
2.1	3.4\\
2.1	3.5\\
2.1	3.6\\
2.1	3.7\\
2.1	3.8\\
2.1	3.9\\
2.1	4\\
2.1	4.1\\
2.1	4.2\\
2.1	4.3\\
2.1	4.4\\
2.1	4.5\\
2.1	4.6\\
2.1	4.7\\
2.1	4.8\\
};
\addplot [color=white, draw=none, mark size=0.2pt, mark=*, mark options={solid, white}, forget plot]
  table[row sep=crcr]{%
2.2	1.2\\
2.2	1.3\\
2.2	1.4\\
2.2	1.5\\
2.2	1.6\\
2.2	1.7\\
2.2	1.8\\
2.2	1.9\\
2.2	2\\
2.2	2.1\\
2.2	2.2\\
2.2	2.3\\
2.2	2.4\\
2.2	2.5\\
2.2	2.6\\
2.2	2.7\\
2.2	2.8\\
2.2	2.9\\
2.2	3\\
2.2	3.1\\
2.2	3.2\\
2.2	3.3\\
2.2	3.4\\
2.2	3.5\\
2.2	3.6\\
2.2	3.7\\
2.2	3.8\\
2.2	3.9\\
2.2	4\\
2.2	4.1\\
2.2	4.2\\
2.2	4.3\\
2.2	4.4\\
2.2	4.5\\
2.2	4.6\\
2.2	4.7\\
2.2	4.8\\
};
\addplot [color=white, draw=none, mark size=0.2pt, mark=*, mark options={solid, white}, forget plot]
  table[row sep=crcr]{%
2.3	1.2\\
2.3	1.3\\
2.3	1.4\\
2.3	1.5\\
2.3	1.6\\
2.3	1.7\\
2.3	1.8\\
2.3	1.9\\
2.3	2\\
2.3	2.1\\
2.3	2.2\\
2.3	2.3\\
2.3	2.4\\
2.3	2.5\\
2.3	2.6\\
2.3	2.7\\
2.3	2.8\\
2.3	2.9\\
2.3	3\\
2.3	3.1\\
2.3	3.2\\
2.3	3.3\\
2.3	3.4\\
2.3	3.5\\
2.3	3.6\\
2.3	3.7\\
2.3	3.8\\
2.3	3.9\\
2.3	4\\
2.3	4.1\\
2.3	4.2\\
2.3	4.3\\
2.3	4.4\\
2.3	4.5\\
2.3	4.6\\
2.3	4.7\\
2.3	4.8\\
};
\addplot [color=white, draw=none, mark size=0.2pt, mark=*, mark options={solid, white}, forget plot]
  table[row sep=crcr]{%
2.4	1.2\\
2.4	1.3\\
2.4	1.4\\
2.4	1.5\\
2.4	1.6\\
2.4	1.7\\
2.4	1.8\\
2.4	1.9\\
2.4	2\\
2.4	2.1\\
2.4	2.2\\
2.4	2.3\\
2.4	2.4\\
2.4	2.5\\
2.4	2.6\\
2.4	2.7\\
2.4	2.8\\
2.4	2.9\\
2.4	3\\
2.4	3.1\\
2.4	3.2\\
2.4	3.3\\
2.4	3.4\\
2.4	3.5\\
2.4	3.6\\
2.4	3.7\\
2.4	3.8\\
2.4	3.9\\
2.4	4\\
2.4	4.1\\
2.4	4.2\\
2.4	4.3\\
2.4	4.4\\
2.4	4.5\\
2.4	4.6\\
2.4	4.7\\
2.4	4.8\\
};
\addplot [color=white, draw=none, mark size=0.2pt, mark=*, mark options={solid, white}, forget plot]
  table[row sep=crcr]{%
2.5	1.2\\
2.5	1.3\\
2.5	1.4\\
2.5	1.5\\
2.5	1.6\\
2.5	1.7\\
2.5	1.8\\
2.5	1.9\\
2.5	2\\
2.5	2.1\\
2.5	2.2\\
2.5	2.3\\
2.5	2.4\\
2.5	2.5\\
2.5	2.6\\
2.5	2.7\\
2.5	2.8\\
2.5	2.9\\
2.5	3\\
2.5	3.1\\
2.5	3.2\\
2.5	3.3\\
2.5	3.4\\
2.5	3.5\\
2.5	3.6\\
2.5	3.7\\
2.5	3.8\\
2.5	3.9\\
2.5	4\\
2.5	4.1\\
2.5	4.2\\
2.5	4.3\\
2.5	4.4\\
2.5	4.5\\
2.5	4.6\\
2.5	4.7\\
2.5	4.8\\
};
\addplot [color=white, draw=none, mark size=0.2pt, mark=*, mark options={solid, white}, forget plot]
  table[row sep=crcr]{%
2.6	1.2\\
2.6	1.3\\
2.6	1.4\\
2.6	1.5\\
2.6	1.6\\
2.6	1.7\\
2.6	1.8\\
2.6	1.9\\
2.6	2\\
2.6	2.1\\
2.6	2.2\\
2.6	2.3\\
2.6	2.4\\
2.6	2.5\\
2.6	2.6\\
2.6	2.7\\
2.6	2.8\\
2.6	2.9\\
2.6	3\\
2.6	3.1\\
2.6	3.2\\
2.6	3.3\\
2.6	3.4\\
2.6	3.5\\
2.6	3.6\\
2.6	3.7\\
2.6	3.8\\
2.6	3.9\\
2.6	4\\
2.6	4.1\\
2.6	4.2\\
2.6	4.3\\
2.6	4.4\\
2.6	4.5\\
2.6	4.6\\
2.6	4.7\\
2.6	4.8\\
};
\addplot [color=white, draw=none, mark size=0.2pt, mark=*, mark options={solid, white}, forget plot]
  table[row sep=crcr]{%
2.7	1.2\\
2.7	1.3\\
2.7	1.4\\
2.7	1.5\\
2.7	1.6\\
2.7	1.7\\
2.7	1.8\\
2.7	1.9\\
2.7	2\\
2.7	2.1\\
2.7	2.2\\
2.7	2.3\\
2.7	2.4\\
2.7	2.5\\
2.7	2.6\\
2.7	2.7\\
2.7	2.8\\
2.7	2.9\\
2.7	3\\
2.7	3.1\\
2.7	3.2\\
2.7	3.3\\
2.7	3.4\\
2.7	3.5\\
2.7	3.6\\
2.7	3.7\\
2.7	3.8\\
2.7	3.9\\
2.7	4\\
2.7	4.1\\
2.7	4.2\\
2.7	4.3\\
2.7	4.4\\
2.7	4.5\\
2.7	4.6\\
2.7	4.7\\
2.7	4.8\\
};
\addplot [color=white, draw=none, mark size=0.2pt, mark=*, mark options={solid, white}, forget plot]
  table[row sep=crcr]{%
2.8	1.2\\
2.8	1.3\\
2.8	1.4\\
2.8	1.5\\
2.8	1.6\\
2.8	1.7\\
2.8	1.8\\
2.8	1.9\\
2.8	2\\
2.8	2.1\\
2.8	2.2\\
2.8	2.3\\
2.8	2.4\\
2.8	2.5\\
2.8	2.6\\
2.8	2.7\\
2.8	2.8\\
2.8	2.9\\
2.8	3\\
2.8	3.1\\
2.8	3.2\\
2.8	3.3\\
2.8	3.4\\
2.8	3.5\\
2.8	3.6\\
2.8	3.7\\
2.8	3.8\\
2.8	3.9\\
2.8	4\\
2.8	4.1\\
2.8	4.2\\
2.8	4.3\\
2.8	4.4\\
2.8	4.5\\
2.8	4.6\\
2.8	4.7\\
2.8	4.8\\
};
\addplot [color=white, draw=none, mark size=0.2pt, mark=*, mark options={solid, white}, forget plot]
  table[row sep=crcr]{%
2.9	1.2\\
2.9	1.3\\
2.9	1.4\\
2.9	1.5\\
2.9	1.6\\
2.9	1.7\\
2.9	1.8\\
2.9	1.9\\
2.9	2\\
2.9	2.1\\
2.9	2.2\\
2.9	2.3\\
2.9	2.4\\
2.9	2.5\\
2.9	2.6\\
2.9	2.7\\
2.9	2.8\\
2.9	2.9\\
2.9	3\\
2.9	3.1\\
2.9	3.2\\
2.9	3.3\\
2.9	3.4\\
2.9	3.5\\
2.9	3.6\\
2.9	3.7\\
2.9	3.8\\
2.9	3.9\\
2.9	4\\
2.9	4.1\\
2.9	4.2\\
2.9	4.3\\
2.9	4.4\\
2.9	4.5\\
2.9	4.6\\
2.9	4.7\\
2.9	4.8\\
};
\addplot [color=white, draw=none, mark size=0.2pt, mark=*, mark options={solid, white}, forget plot]
  table[row sep=crcr]{%
3	1.2\\
3	1.3\\
3	1.4\\
3	1.5\\
3	1.6\\
3	1.7\\
3	1.8\\
3	1.9\\
3	2\\
3	2.1\\
3	2.2\\
3	2.3\\
3	2.4\\
3	2.5\\
3	2.6\\
3	2.7\\
3	2.8\\
3	2.9\\
3	3\\
3	3.1\\
3	3.2\\
3	3.3\\
3	3.4\\
3	3.5\\
3	3.6\\
3	3.7\\
3	3.8\\
3	3.9\\
3	4\\
3	4.1\\
3	4.2\\
3	4.3\\
3	4.4\\
3	4.5\\
3	4.6\\
3	4.7\\
3	4.8\\
};
\addplot [color=white, draw=none, mark size=0.2pt, mark=*, mark options={solid, white}, forget plot]
  table[row sep=crcr]{%
3.1	1.2\\
3.1	1.3\\
3.1	1.4\\
3.1	1.5\\
3.1	1.6\\
3.1	1.7\\
3.1	1.8\\
3.1	1.9\\
3.1	2\\
3.1	2.1\\
3.1	2.2\\
3.1	2.3\\
3.1	2.4\\
3.1	2.5\\
3.1	2.6\\
3.1	2.7\\
3.1	2.8\\
3.1	2.9\\
3.1	3\\
3.1	3.1\\
3.1	3.2\\
3.1	3.3\\
3.1	3.4\\
3.1	3.5\\
3.1	3.6\\
3.1	3.7\\
3.1	3.8\\
3.1	3.9\\
3.1	4\\
3.1	4.1\\
3.1	4.2\\
3.1	4.3\\
3.1	4.4\\
3.1	4.5\\
3.1	4.6\\
3.1	4.7\\
3.1	4.8\\
};
\addplot [color=white, draw=none, mark size=0.2pt, mark=*, mark options={solid, white}, forget plot]
  table[row sep=crcr]{%
3.2	1.2\\
3.2	1.3\\
3.2	1.4\\
3.2	1.5\\
3.2	1.6\\
3.2	1.7\\
3.2	1.8\\
3.2	1.9\\
3.2	2\\
3.2	2.1\\
3.2	2.2\\
3.2	2.3\\
3.2	2.4\\
3.2	2.5\\
3.2	2.6\\
3.2	2.7\\
3.2	2.8\\
3.2	2.9\\
3.2	3\\
3.2	3.1\\
3.2	3.2\\
3.2	3.3\\
3.2	3.4\\
3.2	3.5\\
3.2	3.6\\
3.2	3.7\\
3.2	3.8\\
3.2	3.9\\
3.2	4\\
3.2	4.1\\
3.2	4.2\\
3.2	4.3\\
3.2	4.4\\
3.2	4.5\\
3.2	4.6\\
3.2	4.7\\
3.2	4.8\\
};
\addplot [color=white, draw=none, mark size=0.2pt, mark=*, mark options={solid, white}, forget plot]
  table[row sep=crcr]{%
3.3	1.2\\
3.3	1.3\\
3.3	1.4\\
3.3	1.5\\
3.3	1.6\\
3.3	1.7\\
3.3	1.8\\
3.3	1.9\\
3.3	2\\
3.3	2.1\\
3.3	2.2\\
3.3	2.3\\
3.3	2.4\\
3.3	2.5\\
3.3	2.6\\
3.3	2.7\\
3.3	2.8\\
3.3	2.9\\
3.3	3\\
3.3	3.1\\
3.3	3.2\\
3.3	3.3\\
3.3	3.4\\
3.3	3.5\\
3.3	3.6\\
3.3	3.7\\
3.3	3.8\\
3.3	3.9\\
3.3	4\\
3.3	4.1\\
3.3	4.2\\
3.3	4.3\\
3.3	4.4\\
3.3	4.5\\
3.3	4.6\\
3.3	4.7\\
3.3	4.8\\
};
\addplot [color=white, draw=none, mark size=0.2pt, mark=*, mark options={solid, white}, forget plot]
  table[row sep=crcr]{%
3.4	1.2\\
3.4	1.3\\
3.4	1.4\\
3.4	1.5\\
3.4	1.6\\
3.4	1.7\\
3.4	1.8\\
3.4	1.9\\
3.4	2\\
3.4	2.1\\
3.4	2.2\\
3.4	2.3\\
3.4	2.4\\
3.4	2.5\\
3.4	2.6\\
3.4	2.7\\
3.4	2.8\\
3.4	2.9\\
3.4	3\\
3.4	3.1\\
3.4	3.2\\
3.4	3.3\\
3.4	3.4\\
3.4	3.5\\
3.4	3.6\\
3.4	3.7\\
3.4	3.8\\
3.4	3.9\\
3.4	4\\
3.4	4.1\\
3.4	4.2\\
3.4	4.3\\
3.4	4.4\\
3.4	4.5\\
3.4	4.6\\
3.4	4.7\\
3.4	4.8\\
};
\addplot [color=white, draw=none, mark size=0.2pt, mark=*, mark options={solid, white}, forget plot]
  table[row sep=crcr]{%
3.5	1.2\\
3.5	1.3\\
3.5	1.4\\
3.5	1.5\\
3.5	1.6\\
3.5	1.7\\
3.5	1.8\\
3.5	1.9\\
3.5	2\\
3.5	2.1\\
3.5	2.2\\
3.5	2.3\\
3.5	2.4\\
3.5	2.5\\
3.5	2.6\\
3.5	2.7\\
3.5	2.8\\
3.5	2.9\\
3.5	3\\
3.5	3.1\\
3.5	3.2\\
3.5	3.3\\
3.5	3.4\\
3.5	3.5\\
3.5	3.6\\
3.5	3.7\\
3.5	3.8\\
3.5	3.9\\
3.5	4\\
3.5	4.1\\
3.5	4.2\\
3.5	4.3\\
3.5	4.4\\
3.5	4.5\\
3.5	4.6\\
3.5	4.7\\
3.5	4.8\\
};
\addplot [color=white, draw=none, mark size=0.2pt, mark=*, mark options={solid, white}, forget plot]
  table[row sep=crcr]{%
3.6	1.2\\
3.6	1.3\\
3.6	1.4\\
3.6	1.5\\
3.6	1.6\\
3.6	1.7\\
3.6	1.8\\
3.6	1.9\\
3.6	2\\
3.6	2.1\\
3.6	2.2\\
3.6	2.3\\
3.6	2.4\\
3.6	2.5\\
3.6	2.6\\
3.6	2.7\\
3.6	2.8\\
3.6	2.9\\
3.6	3\\
3.6	3.1\\
3.6	3.2\\
3.6	3.3\\
3.6	3.4\\
3.6	3.5\\
3.6	3.6\\
3.6	3.7\\
3.6	3.8\\
3.6	3.9\\
3.6	4\\
3.6	4.1\\
3.6	4.2\\
3.6	4.3\\
3.6	4.4\\
3.6	4.5\\
3.6	4.6\\
3.6	4.7\\
3.6	4.8\\
};
\addplot [color=white, draw=none, mark size=0.2pt, mark=*, mark options={solid, white}, forget plot]
  table[row sep=crcr]{%
3.7	1.2\\
3.7	1.3\\
3.7	1.4\\
3.7	1.5\\
3.7	1.6\\
3.7	1.7\\
3.7	1.8\\
3.7	1.9\\
3.7	2\\
3.7	2.1\\
3.7	2.2\\
3.7	2.3\\
3.7	2.4\\
3.7	2.5\\
3.7	2.6\\
3.7	2.7\\
3.7	2.8\\
3.7	2.9\\
3.7	3\\
3.7	3.1\\
3.7	3.2\\
3.7	3.3\\
3.7	3.4\\
3.7	3.5\\
3.7	3.6\\
3.7	3.7\\
3.7	3.8\\
3.7	3.9\\
3.7	4\\
3.7	4.1\\
3.7	4.2\\
3.7	4.3\\
3.7	4.4\\
3.7	4.5\\
3.7	4.6\\
3.7	4.7\\
3.7	4.8\\
};
\addplot [color=white, draw=none, mark size=0.2pt, mark=*, mark options={solid, white}, forget plot]
  table[row sep=crcr]{%
3.8	1.2\\
3.8	1.3\\
3.8	1.4\\
3.8	1.5\\
3.8	1.6\\
3.8	1.7\\
3.8	1.8\\
3.8	1.9\\
3.8	2\\
3.8	2.1\\
3.8	2.2\\
3.8	2.3\\
3.8	2.4\\
3.8	2.5\\
3.8	2.6\\
3.8	2.7\\
3.8	2.8\\
3.8	2.9\\
3.8	3\\
3.8	3.1\\
3.8	3.2\\
3.8	3.3\\
3.8	3.4\\
3.8	3.5\\
3.8	3.6\\
3.8	3.7\\
3.8	3.8\\
3.8	3.9\\
3.8	4\\
3.8	4.1\\
3.8	4.2\\
3.8	4.3\\
3.8	4.4\\
3.8	4.5\\
3.8	4.6\\
3.8	4.7\\
3.8	4.8\\
};
\addplot [color=white, draw=none, mark size=0.2pt, mark=*, mark options={solid, white}, forget plot]
  table[row sep=crcr]{%
3.9	1.2\\
3.9	1.3\\
3.9	1.4\\
3.9	1.5\\
3.9	1.6\\
3.9	1.7\\
3.9	1.8\\
3.9	1.9\\
3.9	2\\
3.9	2.1\\
3.9	2.2\\
3.9	2.3\\
3.9	2.4\\
3.9	2.5\\
3.9	2.6\\
3.9	2.7\\
3.9	2.8\\
3.9	2.9\\
3.9	3\\
3.9	3.1\\
3.9	3.2\\
3.9	3.3\\
3.9	3.4\\
3.9	3.5\\
3.9	3.6\\
3.9	3.7\\
3.9	3.8\\
3.9	3.9\\
3.9	4\\
3.9	4.1\\
3.9	4.2\\
3.9	4.3\\
3.9	4.4\\
3.9	4.5\\
3.9	4.6\\
3.9	4.7\\
3.9	4.8\\
};
\addplot [color=white, draw=none, mark size=0.2pt, mark=*, mark options={solid, white}, forget plot]
  table[row sep=crcr]{%
4	1.2\\
4	1.3\\
4	1.4\\
4	1.5\\
4	1.6\\
4	1.7\\
4	1.8\\
4	1.9\\
4	2\\
4	2.1\\
4	2.2\\
4	2.3\\
4	2.4\\
4	2.5\\
4	2.6\\
4	2.7\\
4	2.8\\
4	2.9\\
4	3\\
4	3.1\\
4	3.2\\
4	3.3\\
4	3.4\\
4	3.5\\
4	3.6\\
4	3.7\\
4	3.8\\
4	3.9\\
4	4\\
4	4.1\\
4	4.2\\
4	4.3\\
4	4.4\\
4	4.5\\
4	4.6\\
4	4.7\\
4	4.8\\
};
\addplot [color=white, draw=none, mark size=0.2pt, mark=*, mark options={solid, white}, forget plot]
  table[row sep=crcr]{%
4.1	1.2\\
4.1	1.3\\
4.1	1.4\\
4.1	1.5\\
4.1	1.6\\
4.1	1.7\\
4.1	1.8\\
4.1	1.9\\
4.1	2\\
4.1	2.1\\
4.1	2.2\\
4.1	2.3\\
4.1	2.4\\
4.1	2.5\\
4.1	2.6\\
4.1	2.7\\
4.1	2.8\\
4.1	2.9\\
4.1	3\\
4.1	3.1\\
4.1	3.2\\
4.1	3.3\\
4.1	3.4\\
4.1	3.5\\
4.1	3.6\\
4.1	3.7\\
4.1	3.8\\
4.1	3.9\\
4.1	4\\
4.1	4.1\\
4.1	4.2\\
4.1	4.3\\
4.1	4.4\\
4.1	4.5\\
4.1	4.6\\
4.1	4.7\\
4.1	4.8\\
};
\addplot [color=white, draw=none, mark size=0.2pt, mark=*, mark options={solid, white}, forget plot]
  table[row sep=crcr]{%
4.2	1.2\\
4.2	1.3\\
4.2	1.4\\
4.2	1.5\\
4.2	1.6\\
4.2	1.7\\
4.2	1.8\\
4.2	1.9\\
4.2	2\\
4.2	2.1\\
4.2	2.2\\
4.2	2.3\\
4.2	2.4\\
4.2	2.5\\
4.2	2.6\\
4.2	2.7\\
4.2	2.8\\
4.2	2.9\\
4.2	3\\
4.2	3.1\\
4.2	3.2\\
4.2	3.3\\
4.2	3.4\\
4.2	3.5\\
4.2	3.6\\
4.2	3.7\\
4.2	3.8\\
4.2	3.9\\
4.2	4\\
4.2	4.1\\
4.2	4.2\\
4.2	4.3\\
4.2	4.4\\
4.2	4.5\\
4.2	4.6\\
4.2	4.7\\
4.2	4.8\\
};
\addplot [color=white, draw=none, mark size=0.2pt, mark=*, mark options={solid, white}, forget plot]
  table[row sep=crcr]{%
4.3	1.2\\
4.3	1.3\\
4.3	1.4\\
4.3	1.5\\
4.3	1.6\\
4.3	1.7\\
4.3	1.8\\
4.3	1.9\\
4.3	2\\
4.3	2.1\\
4.3	2.2\\
4.3	2.3\\
4.3	2.4\\
4.3	2.5\\
4.3	2.6\\
4.3	2.7\\
4.3	2.8\\
4.3	2.9\\
4.3	3\\
4.3	3.1\\
4.3	3.2\\
4.3	3.3\\
4.3	3.4\\
4.3	3.5\\
4.3	3.6\\
4.3	3.7\\
4.3	3.8\\
4.3	3.9\\
4.3	4\\
4.3	4.1\\
4.3	4.2\\
4.3	4.3\\
4.3	4.4\\
4.3	4.5\\
4.3	4.6\\
4.3	4.7\\
4.3	4.8\\
};
\addplot [color=white, draw=none, mark size=0.2pt, mark=*, mark options={solid, white}, forget plot]
  table[row sep=crcr]{%
4.4	1.2\\
4.4	1.3\\
4.4	1.4\\
4.4	1.5\\
4.4	1.6\\
4.4	1.7\\
4.4	1.8\\
4.4	1.9\\
4.4	2\\
4.4	2.1\\
4.4	2.2\\
4.4	2.3\\
4.4	2.4\\
4.4	2.5\\
4.4	2.6\\
4.4	2.7\\
4.4	2.8\\
4.4	2.9\\
4.4	3\\
4.4	3.1\\
4.4	3.2\\
4.4	3.3\\
4.4	3.4\\
4.4	3.5\\
4.4	3.6\\
4.4	3.7\\
4.4	3.8\\
4.4	3.9\\
4.4	4\\
4.4	4.1\\
4.4	4.2\\
4.4	4.3\\
4.4	4.4\\
4.4	4.5\\
4.4	4.6\\
4.4	4.7\\
4.4	4.8\\
};
\addplot [color=white, draw=none, mark size=0.2pt, mark=*, mark options={solid, white}, forget plot]
  table[row sep=crcr]{%
4.5	1.2\\
4.5	1.3\\
4.5	1.4\\
4.5	1.5\\
4.5	1.6\\
4.5	1.7\\
4.5	1.8\\
4.5	1.9\\
4.5	2\\
4.5	2.1\\
4.5	2.2\\
4.5	2.3\\
4.5	2.4\\
4.5	2.5\\
4.5	2.6\\
4.5	2.7\\
4.5	2.8\\
4.5	2.9\\
4.5	3\\
4.5	3.1\\
4.5	3.2\\
4.5	3.3\\
4.5	3.4\\
4.5	3.5\\
4.5	3.6\\
4.5	3.7\\
4.5	3.8\\
4.5	3.9\\
4.5	4\\
4.5	4.1\\
4.5	4.2\\
4.5	4.3\\
4.5	4.4\\
4.5	4.5\\
4.5	4.6\\
4.5	4.7\\
4.5	4.8\\
};
\addplot [color=white, draw=none, mark size=0.2pt, mark=*, mark options={solid, white}, forget plot]
  table[row sep=crcr]{%
4.6	1.2\\
4.6	1.3\\
4.6	1.4\\
4.6	1.5\\
4.6	1.6\\
4.6	1.7\\
4.6	1.8\\
4.6	1.9\\
4.6	2\\
4.6	2.1\\
4.6	2.2\\
4.6	2.3\\
4.6	2.4\\
4.6	2.5\\
4.6	2.6\\
4.6	2.7\\
4.6	2.8\\
4.6	2.9\\
4.6	3\\
4.6	3.1\\
4.6	3.2\\
4.6	3.3\\
4.6	3.4\\
4.6	3.5\\
4.6	3.6\\
4.6	3.7\\
4.6	3.8\\
4.6	3.9\\
4.6	4\\
4.6	4.1\\
4.6	4.2\\
4.6	4.3\\
4.6	4.4\\
4.6	4.5\\
4.6	4.6\\
4.6	4.7\\
4.6	4.8\\
};
\addplot [color=white, draw=none, mark size=0.2pt, mark=*, mark options={solid, white}, forget plot]
  table[row sep=crcr]{%
4.7	1.2\\
4.7	1.3\\
4.7	1.4\\
4.7	1.5\\
4.7	1.6\\
4.7	1.7\\
4.7	1.8\\
4.7	1.9\\
4.7	2\\
4.7	2.1\\
4.7	2.2\\
4.7	2.3\\
4.7	2.4\\
4.7	2.5\\
4.7	2.6\\
4.7	2.7\\
4.7	2.8\\
4.7	2.9\\
4.7	3\\
4.7	3.1\\
4.7	3.2\\
4.7	3.3\\
4.7	3.4\\
4.7	3.5\\
4.7	3.6\\
4.7	3.7\\
4.7	3.8\\
4.7	3.9\\
4.7	4\\
4.7	4.1\\
4.7	4.2\\
4.7	4.3\\
4.7	4.4\\
4.7	4.5\\
4.7	4.6\\
4.7	4.7\\
4.7	4.8\\
};
\addplot [color=white, draw=none, mark size=0.2pt, mark=*, mark options={solid, white}, forget plot]
  table[row sep=crcr]{%
4.8	1.2\\
4.8	1.3\\
4.8	1.4\\
4.8	1.5\\
4.8	1.6\\
4.8	1.7\\
4.8	1.8\\
4.8	1.9\\
4.8	2\\
4.8	2.1\\
4.8	2.2\\
4.8	2.3\\
4.8	2.4\\
4.8	2.5\\
4.8	2.6\\
4.8	2.7\\
4.8	2.8\\
4.8	2.9\\
4.8	3\\
4.8	3.1\\
4.8	3.2\\
4.8	3.3\\
4.8	3.4\\
4.8	3.5\\
4.8	3.6\\
4.8	3.7\\
4.8	3.8\\
4.8	3.9\\
4.8	4\\
4.8	4.1\\
4.8	4.2\\
4.8	4.3\\
4.8	4.4\\
4.8	4.5\\
4.8	4.6\\
4.8	4.7\\
4.8	4.8\\
};
\addplot [color=mycolor1, line width=1.0pt, draw=none, mark size=6.0pt, mark=x, mark options={solid, mycolor1}, forget plot]
  table[row sep=crcr]{%
3	1.7\\
3	4\\
};
\end{axis}
\end{tikzpicture}%
		\caption{Ambiguous}
	\end{subfigure}
	\caption[Examples for Ambiguous Location Estimate Assignment Situations]{Examples for Ambivalent Location Estimate Assignment Situations: \itshape Source locations are shown in red, location estimates are shown in yellow. In Example 1, minimising the \glsentryshort{mae} will assign the yellow location estimate besides the left source to the right source. For Example 2, iteratively assigning closest estimates per source and starting with the right source will assign the left estimate to the right source. Both assignments strategies have scenarios in which the resulting assignment does not match the intuitive assignment.}
	\label{fig:assignmentExample}
\end{figure}

%ALTERNATIVE:
%There are two solutions to this problem. First, each estimate could be matched to it's closest original position, going from closest to farthest. The problem with this strategy is, that it depends on the order the estimates are assigned. The example presented in \autoref{fig:assignmentExample} illustrates, how assigning the estimate closest to $S_1$ first yields a different solution, than assigning the estimate for $S_2$ first. Therefore, a second option would be to optimise for the total mean localisation error by choosing an assignment, that minimises the overall error. This strategy determines an assignment that is optimal in an \gls{mmse} sense, but might be removed from the true origins of the location estimates. In \autoref{fig:assignmentExample}, the assignment with the minimum overall error would assign the location estimate close to $S_1$ to $S_2$, despite the fact that it almost correctly identified $S_1$. Therefore, both ways of assigning estimates to their assumed original location to calculate the mean localisation error are flawed. To be able to compare the results of trials with different parameter sets across many, randomly generated source configurations, one of these measures has to be chosen.