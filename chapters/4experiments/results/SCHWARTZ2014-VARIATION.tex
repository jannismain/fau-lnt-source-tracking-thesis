\paragraph{Variations on Experiments in \citeauthor{Schwartz2014} \citeyearpar{Schwartz2014}}
For these variations on the replication trials above, the results of which are presented in \autoref{fig:resultsReplicationAlternativeRotated} and \autoref{fig:resultsReplicationAlternativeEqual}, the original initialisation has been modified. In the first trial, it has been rotated by $90^{\circ}$, now horizontally dividing the room along $p_y=3$. With this initialisation, both sources are located in the non-zero area of a single $\psips$, which should render the seperation into two different $\psips$ meaningless. In the second trial, $\psips$ is initialised with equal values across the whole room for both sources. This shows, how the localisation performs when only a single $\psip$ is used, as both estimates w will be identical. It is also the only trial so far, where no prior knowledge about the sources is introduced before starting the \gls{em} iterations.

\begin{figure}[!htbp]
\centering
    \iftoggle{quick}{%
		\includegraphics[width=\textwidth]{plots/schwartz2014-variation/s=2-sloc=schwartz2014-T60=0.7-prior=schwartz2014-unlucky-results}
		}{%
			\setlength{\figurewidth}{\textwidth}
    % This file was created by matlab2tikz.
%
\begin{tikzpicture}

\begin{axis}[%
width=4.462in,
height=4.075in,
at={(1.733in,0.55in)},
scale only axis,
axis on top,
xmin=0,
xmax=6,
ymin=0,
ymax=6,
axis background/.style={fill=white},
axis x line*=bottom,
axis y line*=left
]
\addplot [forget plot] graphics [xmin=-0.05, xmax=6.05, ymin=-0.05, ymax=6.05] {s=2-sloc=schwartz2014-T60=0.7-prior=schwartz2014-unlucky-results-1.png};
\addplot [color=green, line width=2.0pt, draw=none, mark size=6.0pt, mark=x, mark options={solid, green}, forget plot]
  table[row sep=crcr]{%
1.8	1\\
2	1\\
2.4	1\\
2.6	1\\
3.6	1\\
3.8	1\\
5	1.8\\
5	2\\
5	2.4\\
5	2.6\\
5	3.6\\
5	3.8\\
1.8	5\\
2	5\\
2.4	5\\
2.6	5\\
3.6	5\\
3.8	5\\
1	1.8\\
1	2\\
1	2.4\\
1	2.6\\
1	3.6\\
1	3.8\\
};
\addplot [color=white, line width=2.0pt, draw=none, mark size=8.0pt, mark=x, mark options={solid, white}, forget plot]
  table[row sep=crcr]{%
2.6	2.3\\
3.4	2.3\\
};
\addplot [color=red, line width=2.0pt, draw=none, mark size=8.0pt, mark=x, mark options={solid, red}, forget plot]
  table[row sep=crcr]{%
2.6	2.3\\
3.4	2.3\\
};
\end{axis}

\begin{axis}[%
width=5.867in,
height=4.6in,
at={(6.933in,0.2in)},
scale only axis,
xmin=0,
xmax=6,
tick align=outside,
ymin=0,
ymax=6,
zmin=0,
zmax=0.02,
view={-65}{25},
axis background/.style={fill=white},
axis x line*=bottom,
axis y line*=left,
axis z line*=left,
xmajorgrids,
ymajorgrids,
zmajorgrids
]

\addplot3[%
surf,
shader=interp, colormap={mymap}{[1pt] rgb(0pt)=(0.2422,0.1504,0.6603); rgb(1pt)=(0.25039,0.164995,0.707614); rgb(2pt)=(0.257771,0.181781,0.751138); rgb(3pt)=(0.264729,0.197757,0.795214); rgb(4pt)=(0.270648,0.214676,0.836371); rgb(5pt)=(0.275114,0.234238,0.870986); rgb(6pt)=(0.2783,0.255871,0.899071); rgb(7pt)=(0.280333,0.278233,0.9221); rgb(8pt)=(0.281338,0.300595,0.941376); rgb(9pt)=(0.281014,0.322757,0.957886); rgb(10pt)=(0.279467,0.344671,0.971676); rgb(11pt)=(0.275971,0.366681,0.982905); rgb(12pt)=(0.269914,0.3892,0.9906); rgb(13pt)=(0.260243,0.412329,0.995157); rgb(14pt)=(0.244033,0.435833,0.998833); rgb(15pt)=(0.220643,0.460257,0.997286); rgb(16pt)=(0.196333,0.484719,0.989152); rgb(17pt)=(0.183405,0.507371,0.979795); rgb(18pt)=(0.178643,0.528857,0.968157); rgb(19pt)=(0.176438,0.549905,0.952019); rgb(20pt)=(0.168743,0.570262,0.935871); rgb(21pt)=(0.154,0.5902,0.9218); rgb(22pt)=(0.146029,0.609119,0.907857); rgb(23pt)=(0.138024,0.627629,0.89729); rgb(24pt)=(0.124814,0.645929,0.888343); rgb(25pt)=(0.111252,0.6635,0.876314); rgb(26pt)=(0.0952095,0.679829,0.859781); rgb(27pt)=(0.0688714,0.694771,0.839357); rgb(28pt)=(0.0296667,0.708167,0.816333); rgb(29pt)=(0.00357143,0.720267,0.7917); rgb(30pt)=(0.00665714,0.731214,0.766014); rgb(31pt)=(0.0433286,0.741095,0.73941); rgb(32pt)=(0.0963952,0.75,0.712038); rgb(33pt)=(0.140771,0.7584,0.684157); rgb(34pt)=(0.1717,0.766962,0.655443); rgb(35pt)=(0.193767,0.775767,0.6251); rgb(36pt)=(0.216086,0.7843,0.5923); rgb(37pt)=(0.246957,0.791795,0.556743); rgb(38pt)=(0.290614,0.79729,0.518829); rgb(39pt)=(0.340643,0.8008,0.478857); rgb(40pt)=(0.3909,0.802871,0.435448); rgb(41pt)=(0.445629,0.802419,0.390919); rgb(42pt)=(0.5044,0.7993,0.348); rgb(43pt)=(0.561562,0.794233,0.304481); rgb(44pt)=(0.617395,0.787619,0.261238); rgb(45pt)=(0.671986,0.779271,0.2227); rgb(46pt)=(0.7242,0.769843,0.191029); rgb(47pt)=(0.773833,0.759805,0.16461); rgb(48pt)=(0.820314,0.749814,0.153529); rgb(49pt)=(0.863433,0.7406,0.159633); rgb(50pt)=(0.903543,0.733029,0.177414); rgb(51pt)=(0.939257,0.728786,0.209957); rgb(52pt)=(0.972757,0.729771,0.239443); rgb(53pt)=(0.995648,0.743371,0.237148); rgb(54pt)=(0.996986,0.765857,0.219943); rgb(55pt)=(0.995205,0.789252,0.202762); rgb(56pt)=(0.9892,0.813567,0.188533); rgb(57pt)=(0.978629,0.838629,0.176557); rgb(58pt)=(0.967648,0.8639,0.16429); rgb(59pt)=(0.96101,0.889019,0.153676); rgb(60pt)=(0.959671,0.913457,0.142257); rgb(61pt)=(0.962795,0.937338,0.12651); rgb(62pt)=(0.969114,0.960629,0.106362); rgb(63pt)=(0.9769,0.9839,0.0805)}, mesh/rows=61]
table[row sep=crcr, point meta=\thisrow{c}] {%
%
x	y	z	c\\
0	0	9.43843064247796e-06	9.43843064247796e-06\\
0	0.1	9.68356961873115e-06	9.68356961873115e-06\\
0	0.2	1.03834291716724e-05	1.03834291716724e-05\\
0	0.3	1.15798382802518e-05	1.15798382802518e-05\\
0	0.4	1.33734613434716e-05	1.33734613434716e-05\\
0	0.5	1.59111949546116e-05	1.59111949546116e-05\\
0	0.6	1.93624020932706e-05	1.93624020932706e-05\\
0	0.7	2.3867362233146e-05	2.3867362233146e-05\\
0	0.8	2.94499687671005e-05	2.94499687671005e-05\\
0	0.9	3.59488600155591e-05	3.59488600155591e-05\\
0	1	4.31707030367093e-05	4.31707030367093e-05\\
0	1.1	5.15864779264119e-05	5.15864779264119e-05\\
0	1.2	6.37232614130106e-05	6.37232614130106e-05\\
0	1.3	8.63656794924337e-05	8.63656794924337e-05\\
0	1.4	0.000134617493431872	0.000134617493431872\\
0	1.5	0.000236469397598809	0.000236469397598809\\
0	1.6	0.000419406529591979	0.000419406529591979\\
0	1.7	0.000668249230664501	0.000668249230664501\\
0	1.8	0.000941456218510808	0.000941456218510808\\
0	1.9	0.0012913545051314	0.0012913545051314\\
0	2	0.00209461979792122	0.00209461979792122\\
0	2.1	0.00387258496792457	0.00387258496792457\\
0	2.2	0.00520121927097699	0.00520121927097699\\
0	2.3	0.00379511138912461	0.00379511138912461\\
0	2.4	0.00174297323679537	0.00174297323679537\\
0	2.5	0.000797009458474492	0.000797009458474492\\
0	2.6	0.000485802785001361	0.000485802785001361\\
0	2.7	0.000335004779912165	0.000335004779912165\\
0	2.8	0.000258854369160377	0.000258854369160377\\
0	2.9	0.000262720649974829	0.000262720649974829\\
0	3	0.000500994206747563	0.000500994206747563\\
0	3.1	0.000728817066309015	0.000728817066309015\\
0	3.2	0.00117818840061852	0.00117818840061852\\
0	3.3	0.00186529600756428	0.00186529600756428\\
0	3.4	0.00262760998237929	0.00262760998237929\\
0	3.5	0.00325966270471688	0.00325966270471688\\
0	3.6	0.00363427480555714	0.00363427480555714\\
0	3.7	0.00342930208018208	0.00342930208018208\\
0	3.8	0.00250314617770906	0.00250314617770906\\
0	3.9	0.00140059172691623	0.00140059172691623\\
0	4	0.000646479494090203	0.000646479494090203\\
0	4.1	0.000278138456482299	0.000278138456482299\\
0	4.2	0.000130969321764102	0.000130969321764102\\
0	4.3	7.48982852462055e-05	7.48982852462055e-05\\
0	4.4	5.17941345633223e-05	5.17941345633223e-05\\
0	4.5	4.0873303061202e-05	4.0873303061202e-05\\
0	4.6	3.48495426504882e-05	3.48495426504882e-05\\
0	4.7	3.09093885231262e-05	3.09093885231262e-05\\
0	4.8	2.77713315821049e-05	2.77713315821049e-05\\
0	4.9	2.47940477825688e-05	2.47940477825688e-05\\
0	5	2.1725558322858e-05	2.1725558322858e-05\\
0	5.1	1.85875813279289e-05	1.85875813279289e-05\\
0	5.2	1.55480852055711e-05	1.55480852055711e-05\\
0	5.3	1.27949858418327e-05	1.27949858418327e-05\\
0	5.4	1.0456241081936e-05	1.0456241081936e-05\\
0	5.5	8.57836784697314e-06	8.57836784697314e-06\\
0	5.6	7.14342252463276e-06	7.14342252463276e-06\\
0	5.7	6.09850250239428e-06	6.09850250239428e-06\\
0	5.8	5.38151649620273e-06	5.38151649620273e-06\\
0	5.9	4.93789629354531e-06	4.93789629354531e-06\\
0	6	4.72930137859564e-06	4.72930137859564e-06\\
0.1	0	7.83295684309469e-06	7.83295684309469e-06\\
0.1	0.1	7.85368509942217e-06	7.85368509942217e-06\\
0.1	0.2	8.26032785208831e-06	8.26032785208831e-06\\
0.1	0.3	9.04645474877635e-06	9.04645474877635e-06\\
0.1	0.4	1.02557163546598e-05	1.02557163546598e-05\\
0.1	0.5	1.19673158892867e-05	1.19673158892867e-05\\
0.1	0.6	1.42786640777942e-05	1.42786640777942e-05\\
0.1	0.7	1.72755985754804e-05	1.72755985754804e-05\\
0.1	0.8	2.0976877119085e-05	2.0976877119085e-05\\
0.1	0.9	2.52530341974397e-05	2.52530341974397e-05\\
0.1	1	2.97997623298964e-05	2.97997623298964e-05\\
0.1	1.1	3.44149306460955e-05	3.44149306460955e-05\\
0.1	1.2	3.98626484266157e-05	3.98626484266157e-05\\
0.1	1.3	4.92408039152422e-05	4.92408039152422e-05\\
0.1	1.4	6.9853682845252e-05	6.9853682845252e-05\\
0.1	1.5	0.000116054210896283	0.000116054210896283\\
0.1	1.6	0.000206461779687006	0.000206461779687006\\
0.1	1.7	0.000336473816725589	0.000336473816725589\\
0.1	1.8	0.000467327197900873	0.000467327197900873\\
0.1	1.9	0.000605640611537638	0.000605640611537638\\
0.1	2	0.000894567411849287	0.000894567411849287\\
0.1	2.1	0.0015081901428068	0.0015081901428068\\
0.1	2.2	0.00178142452783382	0.00178142452783382\\
0.1	2.3	0.00113340684124413	0.00113340684124413\\
0.1	2.4	0.000518262557900504	0.000518262557900504\\
0.1	2.5	0.000244115155177182	0.000244115155177182\\
0.1	2.6	0.000128389196197365	0.000128389196197365\\
0.1	2.7	7.15891051879521e-05	7.15891051879521e-05\\
0.1	2.8	4.74284219650958e-05	4.74284219650958e-05\\
0.1	2.9	4.56661684755266e-05	4.56661684755266e-05\\
0.1	3	0.000123724234347189	0.000123724234347189\\
0.1	3.1	0.000183526312779898	0.000183526312779898\\
0.1	3.2	0.00032307244026694	0.00032307244026694\\
0.1	3.3	0.00060062389040513	0.00060062389040513\\
0.1	3.4	0.00105181615768515	0.00105181615768515\\
0.1	3.5	0.00165163512273237	0.00165163512273237\\
0.1	3.6	0.00225620352609612	0.00225620352609612\\
0.1	3.7	0.00235145786493949	0.00235145786493949\\
0.1	3.8	0.00167625673848524	0.00167625673848524\\
0.1	3.9	0.000855377174159716	0.000855377174159716\\
0.1	4	0.000353651381774656	0.000353651381774656\\
0.1	4.1	0.000140053592974096	0.000140053592974096\\
0.1	4.2	6.42382814786637e-05	6.42382814786637e-05\\
0.1	4.3	3.72377880684188e-05	3.72377880684188e-05\\
0.1	4.4	2.66585940037269e-05	2.66585940037269e-05\\
0.1	4.5	2.22214647261996e-05	2.22214647261996e-05\\
0.1	4.6	2.04052161094207e-05	2.04052161094207e-05\\
0.1	4.7	1.97053279601968e-05	1.97053279601968e-05\\
0.1	4.8	1.92702153657434e-05	1.92702153657434e-05\\
0.1	4.9	1.85629861102675e-05	1.85629861102675e-05\\
0.1	5	1.73297976950682e-05	1.73297976950682e-05\\
0.1	5.1	1.55899804383327e-05	1.55899804383327e-05\\
0.1	5.2	1.35503863819891e-05	1.35503863819891e-05\\
0.1	5.3	1.14750094555601e-05	1.14750094555601e-05\\
0.1	5.4	9.58029631258645e-06	9.58029631258645e-06\\
0.1	5.5	7.99124903958003e-06	7.99124903958003e-06\\
0.1	5.6	6.74837559063546e-06	6.74837559063546e-06\\
0.1	5.7	5.83788349122651e-06	5.83788349122651e-06\\
0.1	5.8	5.22300378499745e-06	5.22300378499745e-06\\
0.1	5.9	4.86661312918866e-06	4.86661312918866e-06\\
0.1	6	4.74444791921901e-06	4.74444791921901e-06\\
0.2	0	6.98260043284286e-06	6.98260043284286e-06\\
0.2	0.1	6.81460562913235e-06	6.81460562913235e-06\\
0.2	0.2	7.02019532802732e-06	7.02019532802732e-06\\
0.2	0.3	7.55681218101388e-06	7.55681218101388e-06\\
0.2	0.4	8.42999109834806e-06	8.42999109834806e-06\\
0.2	0.5	9.67404001712341e-06	9.67404001712341e-06\\
0.2	0.6	1.13334637032303e-05	1.13334637032303e-05\\
0.2	0.7	1.344235503265e-05	1.344235503265e-05\\
0.2	0.8	1.59977144969443e-05	1.59977144969443e-05\\
0.2	0.9	1.89175019762964e-05	1.89175019762964e-05\\
0.2	1	2.19867544241649e-05	2.19867544241649e-05\\
0.2	1.1	2.48900369485691e-05	2.48900369485691e-05\\
0.2	1.2	2.76188493635645e-05	2.76188493635645e-05\\
0.2	1.3	3.14884171213754e-05	3.14884171213754e-05\\
0.2	1.4	4.05183049326868e-05	4.05183049326868e-05\\
0.2	1.5	6.31015160750183e-05	6.31015160750183e-05\\
0.2	1.6	0.000111737431307686	0.000111737431307686\\
0.2	1.7	0.000188915891837588	0.000188915891837588\\
0.2	1.8	0.000270091865808213	0.000270091865808213\\
0.2	1.9	0.000351137786618885	0.000351137786618885\\
0.2	2	0.000481179169806092	0.000481179169806092\\
0.2	2.1	0.000727859264180479	0.000727859264180479\\
0.2	2.2	0.00076672684194218	0.00076672684194218\\
0.2	2.3	0.000443380838926088	0.000443380838926088\\
0.2	2.4	0.00022402978522012	0.00022402978522012\\
0.2	2.5	0.000118791326242531	0.000118791326242531\\
0.2	2.6	5.26701788509161e-05	5.26701788509161e-05\\
0.2	2.7	2.30468257315532e-05	2.30468257315532e-05\\
0.2	2.8	1.34140426434737e-05	1.34140426434737e-05\\
0.2	2.9	1.29108259575504e-05	1.29108259575504e-05\\
0.2	3	3.77560811946569e-05	3.77560811946569e-05\\
0.2	3.1	5.41589364658103e-05	5.41589364658103e-05\\
0.2	3.2	9.83615363894636e-05	9.83615363894636e-05\\
0.2	3.3	0.000207967812612669	0.000207967812612669\\
0.2	3.4	0.000456874617955638	0.000456874617955638\\
0.2	3.5	0.000940196118127863	0.000940196118127863\\
0.2	3.6	0.00162177946651671	0.00162177946651671\\
0.2	3.7	0.00184831750579955	0.00184831750579955\\
0.2	3.8	0.00122357495744916	0.00122357495744916\\
0.2	3.9	0.000538209293458867	0.000538209293458867\\
0.2	4	0.000190777743379134	0.000190777743379134\\
0.2	4.1	6.86990573486398e-05	6.86990573486398e-05\\
0.2	4.2	3.1027283469997e-05	3.1027283469997e-05\\
0.2	4.3	1.8664427918339e-05	1.8664427918339e-05\\
0.2	4.4	1.43552008234929e-05	1.43552008234929e-05\\
0.2	4.5	1.31876877935717e-05	1.31876877935717e-05\\
0.2	4.6	1.35029229231383e-05	1.35029229231383e-05\\
0.2	4.7	1.44843061819529e-05	1.44843061819529e-05\\
0.2	4.8	1.55054332381324e-05	1.55054332381324e-05\\
0.2	4.9	1.60416450818682e-05	1.60416450818682e-05\\
0.2	5	1.57815631619092e-05	1.57815631619092e-05\\
0.2	5.1	1.47140540590669e-05	1.47140540590669e-05\\
0.2	5.2	1.30782811009186e-05	1.30782811009186e-05\\
0.2	5.3	1.12143180647889e-05	1.12143180647889e-05\\
0.2	5.4	9.42008184145783e-06	9.42008184145783e-06\\
0.2	5.5	7.88131031767051e-06	7.88131031767051e-06\\
0.2	5.6	6.67323581595562e-06	6.67323581595562e-06\\
0.2	5.7	5.79825179993408e-06	5.79825179993408e-06\\
0.2	5.8	5.22707293948362e-06	5.22707293948362e-06\\
0.2	5.9	4.92837533831494e-06	4.92837533831494e-06\\
0.2	6	4.88582084398016e-06	4.88582084398016e-06\\
0.3	0	6.71393628894671e-06	6.71393628894671e-06\\
0.3	0.1	6.35600878010601e-06	6.35600878010601e-06\\
0.3	0.2	6.40026005895913e-06	6.40026005895913e-06\\
0.3	0.3	6.77362535757224e-06	6.77362535757224e-06\\
0.3	0.4	7.45563539719875e-06	7.45563539719875e-06\\
0.3	0.5	8.4526533224519e-06	8.4526533224519e-06\\
0.3	0.6	9.77272181647281e-06	9.77272181647281e-06\\
0.3	0.7	1.14014821305903e-05	1.14014821305903e-05\\
0.3	0.8	1.32865001817335e-05	1.32865001817335e-05\\
0.3	0.9	1.53342990715681e-05	1.53342990715681e-05\\
0.3	1	1.74073955697893e-05	1.74073955697893e-05\\
0.3	1.1	1.93098035516316e-05	1.93098035516316e-05\\
0.3	1.2	2.08646704755145e-05	2.08646704755145e-05\\
0.3	1.3	2.24345606386765e-05	2.24345606386765e-05\\
0.3	1.4	2.61701618493925e-05	2.61701618493925e-05\\
0.3	1.5	3.78727914278049e-05	3.78727914278049e-05\\
0.3	1.6	6.79965044280684e-05	6.79965044280684e-05\\
0.3	1.7	0.000120551114556849	0.000120551114556849\\
0.3	1.8	0.000180716901691477	0.000180716901691477\\
0.3	1.9	0.000254078852923902	0.000254078852923902\\
0.3	2	0.000337430280639886	0.000337430280639886\\
0.3	2.1	0.000448630382132738	0.000448630382132738\\
0.3	2.2	0.000435434291164847	0.000435434291164847\\
0.3	2.3	0.000247985603573126	0.000247985603573126\\
0.3	2.4	0.000147709720240129	0.000147709720240129\\
0.3	2.5	8.77830669624175e-05	8.77830669624175e-05\\
0.3	2.6	3.05653169179666e-05	3.05653169179666e-05\\
0.3	2.7	1.08866392966503e-05	1.08866392966503e-05\\
0.3	2.8	6.36155933372568e-06	6.36155933372568e-06\\
0.3	2.9	6.33826960904463e-06	6.33826960904463e-06\\
0.3	3	1.47415767209392e-05	1.47415767209392e-05\\
0.3	3.1	1.91424053412715e-05	1.91424053412715e-05\\
0.3	3.2	3.32605062217974e-05	3.32605062217974e-05\\
0.3	3.3	7.46567779665962e-05	7.46567779665962e-05\\
0.3	3.4	0.000202290431651853	0.000202290431651853\\
0.3	3.5	0.000569217775415625	0.000569217775415625\\
0.3	3.6	0.00130868072825162	0.00130868072825162\\
0.3	3.7	0.00163286356161677	0.00163286356161677\\
0.3	3.8	0.000949214643720958	0.000949214643720958\\
0.3	3.9	0.000337934245156612	0.000337934245156612\\
0.3	4	9.87145454901065e-05	9.87145454901065e-05\\
0.3	4.1	3.24935978415521e-05	3.24935978415521e-05\\
0.3	4.2	1.49120690216888e-05	1.49120690216888e-05\\
0.3	4.3	9.76862677376277e-06	9.76862677376277e-06\\
0.3	4.4	8.53346657840478e-06	8.53346657840478e-06\\
0.3	4.5	9.03783526880732e-06	9.03783526880732e-06\\
0.3	4.6	1.05641816373534e-05	1.05641816373534e-05\\
0.3	4.7	1.26262052588322e-05	1.26262052588322e-05\\
0.3	4.8	1.46415849078409e-05	1.46415849078409e-05\\
0.3	4.9	1.59890196460243e-05	1.59890196460243e-05\\
0.3	5	1.62498920212617e-05	1.62498920212617e-05\\
0.3	5.1	1.53901458915513e-05	1.53901458915513e-05\\
0.3	5.2	1.37241354845513e-05	1.37241354845513e-05\\
0.3	5.3	1.17114360630496e-05	1.17114360630496e-05\\
0.3	5.4	9.75127942375011e-06	9.75127942375011e-06\\
0.3	5.5	8.08435330128605e-06	8.08435330128605e-06\\
0.3	5.6	6.80119568928198e-06	6.80119568928198e-06\\
0.3	5.7	5.89987338945458e-06	5.89987338945458e-06\\
0.3	5.8	5.34346100118988e-06	5.34346100118988e-06\\
0.3	5.9	5.09789068012326e-06	5.09789068012326e-06\\
0.3	6	5.15176451444081e-06	5.15176451444081e-06\\
0.4	0	6.92434050095214e-06	6.92434050095214e-06\\
0.4	0.1	6.3587584103271e-06	6.3587584103271e-06\\
0.4	0.2	6.25402893155773e-06	6.25402893155773e-06\\
0.4	0.3	6.509123184902e-06	6.509123184902e-06\\
0.4	0.4	7.08636389148668e-06	7.08636389148668e-06\\
0.4	0.5	7.97919321696769e-06	7.97919321696769e-06\\
0.4	0.6	9.17765581012481e-06	9.17765581012481e-06\\
0.4	0.7	1.0630551727648e-05	1.0630551727648e-05\\
0.4	0.8	1.22206511675551e-05	1.22206511675551e-05\\
0.4	0.9	1.37795516879625e-05	1.37795516879625e-05\\
0.4	1	1.51472307081802e-05	1.51472307081802e-05\\
0.4	1.1	1.62289298601285e-05	1.62289298601285e-05\\
0.4	1.2	1.69948687620866e-05	1.69948687620866e-05\\
0.4	1.3	1.75464196892206e-05	1.75464196892206e-05\\
0.4	1.4	1.87750146044532e-05	1.87750146044532e-05\\
0.4	1.5	2.43791002283833e-05	2.43791002283833e-05\\
0.4	1.6	4.50338441341778e-05	4.50338441341778e-05\\
0.4	1.7	9.07004412864867e-05	9.07004412864867e-05\\
0.4	1.8	0.000140940390366009	0.000140940390366009\\
0.4	1.9	0.000227545755021716	0.000227545755021716\\
0.4	2	0.000310888821748015	0.000310888821748015\\
0.4	2.1	0.000348927905807519	0.000348927905807519\\
0.4	2.2	0.000323803954923778	0.000323803954923778\\
0.4	2.3	0.000216456352391604	0.000216456352391604\\
0.4	2.4	0.000148410946400225	0.000148410946400225\\
0.4	2.5	8.66291536930399e-05	8.66291536930399e-05\\
0.4	2.6	2.19408639489795e-05	2.19408639489795e-05\\
0.4	2.7	7.37037615045297e-06	7.37037615045297e-06\\
0.4	2.8	5.04643320798048e-06	5.04643320798048e-06\\
0.4	2.9	5.0653888654164e-06	5.0653888654164e-06\\
0.4	3	7.53894149332402e-06	7.53894149332402e-06\\
0.4	3.1	8.42174977906322e-06	8.42174977906322e-06\\
0.4	3.2	1.29587359331514e-05	1.29587359331514e-05\\
0.4	3.3	2.78929184205042e-05	2.78929184205042e-05\\
0.4	3.4	8.58637255411117e-05	8.58637255411117e-05\\
0.4	3.5	0.000337523273034783	0.000337523273034783\\
0.4	3.6	0.00113176654952682	0.00113176654952682\\
0.4	3.7	0.00158527451650255	0.00158527451650255\\
0.4	3.8	0.000758683517909812	0.000758683517909812\\
0.4	3.9	0.000203012200541763	0.000203012200541763\\
0.4	4	4.75459214844091e-05	4.75459214844091e-05\\
0.4	4.1	1.48006764698482e-05	1.48006764698482e-05\\
0.4	4.2	7.37752007913747e-06	7.37752007913747e-06\\
0.4	4.3	5.69980839074223e-06	5.69980839074223e-06\\
0.4	4.4	6.02802838966706e-06	6.02802838966706e-06\\
0.4	4.5	7.5758091340074e-06	7.5758091340074e-06\\
0.4	4.6	1.00814979596196e-05	1.00814979596196e-05\\
0.4	4.7	1.3150648560891e-05	1.3150648560891e-05\\
0.4	4.8	1.60809899568034e-05	1.60809899568034e-05\\
0.4	4.9	1.80461127593346e-05	1.80461127593346e-05\\
0.4	5	1.84905604045776e-05	1.84905604045776e-05\\
0.4	5.1	1.74098763525114e-05	1.74098763525114e-05\\
0.4	5.2	1.5287497828493e-05	1.5287497828493e-05\\
0.4	5.3	1.27794790554577e-05	1.27794790554577e-05\\
0.4	5.4	1.04149528792411e-05	1.04149528792411e-05\\
0.4	5.5	8.47754941063737e-06	8.47754941063737e-06\\
0.4	5.6	7.04545149134656e-06	7.04545149134656e-06\\
0.4	5.7	6.08778803766363e-06	6.08778803766363e-06\\
0.4	5.8	5.54488520573438e-06	5.54488520573438e-06\\
0.4	5.9	5.37362035387635e-06	5.37362035387635e-06\\
0.4	6	5.56771798351968e-06	5.56771798351968e-06\\
0.5	0	7.51624945571336e-06	7.51624945571336e-06\\
0.5	0.1	6.73522040924708e-06	6.73522040924708e-06\\
0.5	0.2	6.49059853720551e-06	6.49059853720551e-06\\
0.5	0.3	6.65611016528954e-06	6.65611016528954e-06\\
0.5	0.4	7.1848400785419e-06	7.1848400785419e-06\\
0.5	0.5	8.07285908898258e-06	8.07285908898258e-06\\
0.5	0.6	9.31609430804826e-06	9.31609430804826e-06\\
0.5	0.7	1.08513545793379e-05	1.08513545793379e-05\\
0.5	0.8	1.2500467308722e-05	1.2500467308722e-05\\
0.5	0.9	1.39720486530891e-05	1.39720486530891e-05\\
0.5	1	1.49700001712351e-05	1.49700001712351e-05\\
0.5	1.1	1.53637146739256e-05	1.53637146739256e-05\\
0.5	1.2	1.52729531206598e-05	1.52729531206598e-05\\
0.5	1.3	1.49854850247078e-05	1.49854850247078e-05\\
0.5	1.4	1.49601434122955e-05	1.49601434122955e-05\\
0.5	1.5	1.69220216780981e-05	1.69220216780981e-05\\
0.5	1.6	2.91352308564834e-05	2.91352308564834e-05\\
0.5	1.7	7.72037132131707e-05	7.72037132131707e-05\\
0.5	1.8	0.00013305936051018	0.00013305936051018\\
0.5	1.9	0.000248083643039217	0.000248083643039217\\
0.5	2	0.000363624208003966	0.000363624208003966\\
0.5	2.1	0.000303724134623102	0.000303724134623102\\
0.5	2.2	0.000277764144396937	0.000277764144396937\\
0.5	2.3	0.00028376618149106	0.00028376618149106\\
0.5	2.4	0.000196221424402599	0.000196221424402599\\
0.5	2.5	9.48431746751041e-05	9.48431746751041e-05\\
0.5	2.6	1.769938159768e-05	1.769938159768e-05\\
0.5	2.7	7.01422277084208e-06	7.01422277084208e-06\\
0.5	2.8	5.81347287410033e-06	5.81347287410033e-06\\
0.5	2.9	5.75593169373595e-06	5.75593169373595e-06\\
0.5	3	5.1865779720097e-06	5.1865779720097e-06\\
0.5	3.1	4.93749221000027e-06	4.93749221000027e-06\\
0.5	3.2	6.32565091577227e-06	6.32565091577227e-06\\
0.5	3.3	1.1709078770239e-05	1.1709078770239e-05\\
0.5	3.4	3.45768352845287e-05	3.45768352845287e-05\\
0.5	3.5	0.000176045206728534	0.000176045206728534\\
0.5	3.6	0.000977590133840211	0.000977590133840211\\
0.5	3.7	0.00165448087204105	0.00165448087204105\\
0.5	3.8	0.000599159827108028	0.000599159827108028\\
0.5	3.9	0.000109975877194278	0.000109975877194278\\
0.5	4	2.07026623310344e-05	2.07026623310344e-05\\
0.5	4.1	6.66398300560119e-06	6.66398300560119e-06\\
0.5	4.2	4.08579365785337e-06	4.08579365785337e-06\\
0.5	4.3	4.12012975513276e-06	4.12012975513276e-06\\
0.5	4.4	5.47706015483269e-06	5.47706015483269e-06\\
0.5	4.5	8.04012300597937e-06	8.04012300597937e-06\\
0.5	4.6	1.17116098269923e-05	1.17116098269923e-05\\
0.5	4.7	1.59908428233675e-05	1.59908428233675e-05\\
0.5	4.8	1.98943044223251e-05	1.98943044223251e-05\\
0.5	4.9	2.22936585846166e-05	2.22936585846166e-05\\
0.5	5	2.25072069808695e-05	2.25072069808695e-05\\
0.5	5.1	2.06762996273822e-05	2.06762996273822e-05\\
0.5	5.2	1.76036865045724e-05	1.76036865045724e-05\\
0.5	5.3	1.42393025396186e-05	1.42393025396186e-05\\
0.5	5.4	1.12573721421512e-05	1.12573721421512e-05\\
0.5	5.5	8.94863250626591e-06	8.94863250626591e-06\\
0.5	5.6	7.33496337184712e-06	7.33496337184712e-06\\
0.5	5.7	6.32644050673229e-06	6.32644050673229e-06\\
0.5	5.8	5.82664850389226e-06	5.82664850389226e-06\\
0.5	5.9	5.7808703594046e-06	5.7808703594046e-06\\
0.5	6	6.19270999415931e-06	6.19270999415931e-06\\
0.6	0	8.32341663128522e-06	8.32341663128522e-06\\
0.6	0.1	7.36014233170691e-06	7.36014233170691e-06\\
0.6	0.2	7.01105018755246e-06	7.01105018755246e-06\\
0.6	0.3	7.12690336760719e-06	7.12690336760719e-06\\
0.6	0.4	7.65858537669018e-06	7.65858537669018e-06\\
0.6	0.5	8.62065224482576e-06	8.62065224482576e-06\\
0.6	0.6	1.00453348940926e-05	1.00453348940926e-05\\
0.6	0.7	1.19015062357419e-05	1.19015062357419e-05\\
0.6	0.8	1.3985989469151e-05	1.3985989469151e-05\\
0.6	0.9	1.58648498925911e-05	1.58648498925911e-05\\
0.6	1	1.69936772529074e-05	1.69936772529074e-05\\
0.6	1.1	1.7028875719254e-05	1.7028875719254e-05\\
0.6	1.2	1.60877286713981e-05	1.60877286713981e-05\\
0.6	1.3	1.46943574438549e-05	1.46943574438549e-05\\
0.6	1.4	1.34977495479368e-05	1.34977495479368e-05\\
0.6	1.5	1.3349520086177e-05	1.3349520086177e-05\\
0.6	1.6	1.81295773446014e-05	1.81295773446014e-05\\
0.6	1.7	5.73594333211517e-05	5.73594333211517e-05\\
0.6	1.8	0.000156359293195887	0.000156359293195887\\
0.6	1.9	0.000323476946579077	0.000323476946579077\\
0.6	2	0.000473934961590745	0.000473934961590745\\
0.6	2.1	0.00021239942683144	0.00021239942683144\\
0.6	2.2	0.000187736129025418	0.000187736129025418\\
0.6	2.3	0.000345457562100122	0.000345457562100122\\
0.6	2.4	0.000266188862639863	0.000266188862639863\\
0.6	2.5	0.000105372986691971	0.000105372986691971\\
0.6	2.6	1.64845688663737e-05	1.64845688663737e-05\\
0.6	2.7	9.159365520761e-06	9.159365520761e-06\\
0.6	2.8	8.51226100208497e-06	8.51226100208497e-06\\
0.6	2.9	8.13402120342501e-06	8.13402120342501e-06\\
0.6	3	5.05763604327573e-06	5.05763604327573e-06\\
0.6	3.1	4.28076746395876e-06	4.28076746395876e-06\\
0.6	3.2	4.38905141324809e-06	4.38905141324809e-06\\
0.6	3.3	6.24082855670268e-06	6.24082855670268e-06\\
0.6	3.4	1.45101264172704e-05	1.45101264172704e-05\\
0.6	3.5	7.35224719537145e-05	7.35224719537145e-05\\
0.6	3.6	0.000746686358802596	0.000746686358802596\\
0.6	3.7	0.00181502952115343	0.00181502952115343\\
0.6	3.8	0.000436361017087366	0.000436361017087366\\
0.6	3.9	4.92977673215276e-05	4.92977673215276e-05\\
0.6	4	8.10609416559473e-06	8.10609416559473e-06\\
0.6	4.1	3.3051464150064e-06	3.3051464150064e-06\\
0.6	4.2	2.99066927873739e-06	2.99066927873739e-06\\
0.6	4.3	4.14793665938672e-06	4.14793665938672e-06\\
0.6	4.4	6.61964326968815e-06	6.61964326968815e-06\\
0.6	4.5	1.05433899966198e-05	1.05433899966198e-05\\
0.6	4.6	1.57601113555198e-05	1.57601113555198e-05\\
0.6	4.7	2.14929351688884e-05	2.14929351688884e-05\\
0.6	4.8	2.63477486206282e-05	2.63477486206282e-05\\
0.6	4.9	2.88374118342235e-05	2.88374118342235e-05\\
0.6	5	2.82216617146751e-05	2.82216617146751e-05\\
0.6	5.1	2.49720861339349e-05	2.49720861339349e-05\\
0.6	5.2	2.04035692580368e-05	2.04035692580368e-05\\
0.6	5.3	1.58485373968896e-05	1.58485373968896e-05\\
0.6	5.4	1.21010110907316e-05	1.21010110907316e-05\\
0.6	5.5	9.38738296245788e-06	9.38738296245788e-06\\
0.6	5.6	7.61418928333463e-06	7.61418928333463e-06\\
0.6	5.7	6.60248106133485e-06	6.60248106133485e-06\\
0.6	5.8	6.21125207842626e-06	6.21125207842626e-06\\
0.6	5.9	6.37880362655639e-06	6.37880362655639e-06\\
0.6	6	7.12968469519782e-06	7.12968469519782e-06\\
0.7	0	9.08335526320454e-06	9.08335526320454e-06\\
0.7	0.1	8.01083442055549e-06	8.01083442055549e-06\\
0.7	0.2	7.63658818478829e-06	7.63658818478829e-06\\
0.7	0.3	7.78071146863893e-06	7.78071146863893e-06\\
0.7	0.4	8.39024212026091e-06	8.39024212026091e-06\\
0.7	0.5	9.50619783547684e-06	9.50619783547684e-06\\
0.7	0.6	1.12261122433403e-05	1.12261122433403e-05\\
0.7	0.7	1.36105247584139e-05	1.36105247584139e-05\\
0.7	0.8	1.65110368408192e-05	1.65110368408192e-05\\
0.7	0.9	1.9395596028017e-05	1.9395596028017e-05\\
0.7	1	2.13998946548863e-05	2.13998946548863e-05\\
0.7	1.1	2.17742790530803e-05	2.17742790530803e-05\\
0.7	1.2	2.04235477858431e-05	2.04235477858431e-05\\
0.7	1.3	1.79649885229886e-05	1.79649885229886e-05\\
0.7	1.4	1.52831350535192e-05	1.52831350535192e-05\\
0.7	1.5	1.31863392120958e-05	1.31863392120958e-05\\
0.7	1.6	1.32127467520527e-05	1.32127467520527e-05\\
0.7	1.7	2.88751351562895e-05	2.88751351562895e-05\\
0.7	1.8	0.000193578825995173	0.000193578825995173\\
0.7	1.9	0.000495614840872982	0.000495614840872982\\
0.7	2	0.000515370927077516	0.000515370927077516\\
0.7	2.1	7.51477947043382e-05	7.51477947043382e-05\\
0.7	2.2	5.23865172510831e-05	5.23865172510831e-05\\
0.7	2.3	0.000151550521421228	0.000151550521421228\\
0.7	2.4	0.000333206854309078	0.000333206854309078\\
0.7	2.5	0.000133303728192905	0.000133303728192905\\
0.7	2.6	2.06981127999152e-05	2.06981127999152e-05\\
0.7	2.7	1.47561835957253e-05	1.47561835957253e-05\\
0.7	2.8	1.39299332995174e-05	1.39299332995174e-05\\
0.7	2.9	1.24324431755296e-05	1.24324431755296e-05\\
0.7	3	7.37071001392894e-06	7.37071001392894e-06\\
0.7	3.1	6.03755348985598e-06	6.03755348985598e-06\\
0.7	3.2	5.13554063865437e-06	5.13554063865437e-06\\
0.7	3.3	5.09338346095992e-06	5.09338346095992e-06\\
0.7	3.4	7.59742369483315e-06	7.59742369483315e-06\\
0.7	3.5	2.5789273589058e-05	2.5789273589058e-05\\
0.7	3.6	0.000402833248308753	0.000402833248308753\\
0.7	3.7	0.00203954778472845	0.00203954778472845\\
0.7	3.8	0.000256387655831486	0.000256387655831486\\
0.7	3.9	1.64414787670828e-05	1.64414787670828e-05\\
0.7	4	3.21627233317574e-06	3.21627233317574e-06\\
0.7	4.1	2.40658941301872e-06	2.40658941301872e-06\\
0.7	4.2	3.47867887276956e-06	3.47867887276956e-06\\
0.7	4.3	5.8959548991575e-06	5.8959548991575e-06\\
0.7	4.4	9.85725035978508e-06	9.85725035978508e-06\\
0.7	4.5	1.5480160249889e-05	1.5480160249889e-05\\
0.7	4.6	2.24299831766838e-05	2.24299831766838e-05\\
0.7	4.7	2.96012310232235e-05	2.96012310232235e-05\\
0.7	4.8	3.5146990606444e-05	3.5146990606444e-05\\
0.7	4.9	3.72054533265085e-05	3.72054533265085e-05\\
0.7	5	3.50653625973413e-05	3.50653625973413e-05\\
0.7	5.1	2.97311354987898e-05	2.97311354987898e-05\\
0.7	5.2	2.32160529600728e-05	2.32160529600728e-05\\
0.7	5.3	1.72773008021923e-05	1.72773008021923e-05\\
0.7	5.4	1.27494831439105e-05	1.27494831439105e-05\\
0.7	5.5	9.69659186735609e-06	9.69659186735609e-06\\
0.7	5.6	7.85233051384849e-06	7.85233051384849e-06\\
0.7	5.7	6.93203673866641e-06	6.93203673866641e-06\\
0.7	5.8	6.75593070645954e-06	6.75593070645954e-06\\
0.7	5.9	7.27039033323399e-06	7.27039033323399e-06\\
0.7	6	8.53771162472357e-06	8.53771162472357e-06\\
0.8	0	9.55603290107429e-06	9.55603290107429e-06\\
0.8	0.1	8.40080700219542e-06	8.40080700219542e-06\\
0.8	0.2	8.07867013345647e-06	8.07867013345647e-06\\
0.8	0.3	8.35555717103981e-06	8.35555717103981e-06\\
0.8	0.4	9.15555543810042e-06	9.15555543810042e-06\\
0.8	0.5	1.0527179340078e-05	1.0527179340078e-05\\
0.8	0.6	1.2633613196887e-05	1.2633613196887e-05\\
0.8	0.7	1.56702140729318e-05	1.56702140729318e-05\\
0.8	0.8	1.963675168076e-05	1.963675168076e-05\\
0.8	0.9	2.40132946460495e-05	2.40132946460495e-05\\
0.8	1	2.76616555348463e-05	2.76616555348463e-05\\
0.8	1.1	2.93654412872391e-05	2.93654412872391e-05\\
0.8	1.2	2.87263634430411e-05	2.87263634430411e-05\\
0.8	1.3	2.63842447544321e-05	2.63842447544321e-05\\
0.8	1.4	2.3273170107021e-05	2.3273170107021e-05\\
0.8	1.5	1.98148468190987e-05	1.98148468190987e-05\\
0.8	1.6	1.60187352115024e-05	1.60187352115024e-05\\
0.8	1.7	1.55608950152595e-05	1.55608950152595e-05\\
0.8	1.8	0.000123060010075685	0.000123060010075685\\
0.8	1.9	0.000875922129587873	0.000875922129587873\\
0.8	2	0.000289621960758345	0.000289621960758345\\
0.8	2.1	1.81491239257674e-05	1.81491239257674e-05\\
0.8	2.2	1.68079482473265e-05	1.68079482473265e-05\\
0.8	2.3	3.1068197442476e-05	3.1068197442476e-05\\
0.8	2.4	0.000255642849540094	0.000255642849540094\\
0.8	2.5	0.000224155659069632	0.000224155659069632\\
0.8	2.6	3.64961348221687e-05	3.64961348221687e-05\\
0.8	2.7	2.18562848019333e-05	2.18562848019333e-05\\
0.8	2.8	2.30775966616337e-05	2.30775966616337e-05\\
0.8	2.9	2.08165169897728e-05	2.08165169897728e-05\\
0.8	3	1.4992849019954e-05	1.4992849019954e-05\\
0.8	3.1	1.2300894264708e-05	1.2300894264708e-05\\
0.8	3.2	9.72772362918152e-06	9.72772362918152e-06\\
0.8	3.3	7.66419957905257e-06	7.66419957905257e-06\\
0.8	3.4	6.7195217989308e-06	6.7195217989308e-06\\
0.8	3.5	1.04789031002527e-05	1.04789031002527e-05\\
0.8	3.6	0.00011140546002487	0.00011140546002487\\
0.8	3.7	0.00227924039803914	0.00227924039803914\\
0.8	3.8	9.17947350160499e-05	9.17947350160499e-05\\
0.8	3.9	4.12837094680377e-06	4.12837094680377e-06\\
0.8	4	2.24646655210734e-06	2.24646655210734e-06\\
0.8	4.1	3.50122051599717e-06	3.50122051599717e-06\\
0.8	4.2	6.0021408576169e-06	6.0021408576169e-06\\
0.8	4.3	9.79097402449375e-06	9.79097402449375e-06\\
0.8	4.4	1.51159772916909e-05	1.51159772916909e-05\\
0.8	4.5	2.20990472558252e-05	2.20990472558252e-05\\
0.8	4.6	3.03475343597849e-05	3.03475343597849e-05\\
0.8	4.7	3.85433776743249e-05	3.85433776743249e-05\\
0.8	4.8	4.44225380958583e-05	4.44225380958583e-05\\
0.8	4.9	4.56948316164242e-05	4.56948316164242e-05\\
0.8	5	4.16550116215626e-05	4.16550116215626e-05\\
0.8	5.1	3.39422260217668e-05	3.39422260217668e-05\\
0.8	5.2	2.53856254595296e-05	2.53856254595296e-05\\
0.8	5.3	1.81569764280464e-05	1.81569764280464e-05\\
0.8	5.4	1.30283530276058e-05	1.30283530276058e-05\\
0.8	5.5	9.81658274363375e-06	9.81658274363375e-06\\
0.8	5.6	8.05631788520029e-06	8.05631788520029e-06\\
0.8	5.7	7.36916078472938e-06	7.36916078472938e-06\\
0.8	5.8	7.56165260918997e-06	7.56165260918997e-06\\
0.8	5.9	8.61285480682122e-06	8.61285480682122e-06\\
0.8	6	1.06381790820503e-05	1.06381790820503e-05\\
0.9	0	9.7684787160362e-06	9.7684787160362e-06\\
0.9	0.1	8.37309287700122e-06	8.37309287700122e-06\\
0.9	0.2	8.05463385455255e-06	8.05463385455255e-06\\
0.9	0.3	8.49299410039921e-06	8.49299410039921e-06\\
0.9	0.4	9.57715521024902e-06	9.57715521024902e-06\\
0.9	0.5	1.13201682746637e-05	1.13201682746637e-05\\
0.9	0.6	1.38839224945302e-05	1.38839224945302e-05\\
0.9	0.7	1.75571679106499e-05	1.75571679106499e-05\\
0.9	0.8	2.24975659161398e-05	2.24975659161398e-05\\
0.9	0.9	2.82403827368569e-05	2.82403827368569e-05\\
0.9	1	3.34394017239481e-05	3.34394017239481e-05\\
0.9	1.1	3.65742575240756e-05	3.65742575240756e-05\\
0.9	1.2	3.73245174395418e-05	3.73245174395418e-05\\
0.9	1.3	3.69574913342933e-05	3.69574913342933e-05\\
0.9	1.4	3.72581582341651e-05	3.72581582341651e-05\\
0.9	1.5	3.92968086121745e-05	3.92968086121745e-05\\
0.9	1.6	4.16203235844633e-05	4.16203235844633e-05\\
0.9	1.7	3.36180936247585e-05	3.36180936247585e-05\\
0.9	1.8	2.44311149252214e-05	2.44311149252214e-05\\
0.9	1.9	0.00157131478074393	0.00157131478074393\\
0.9	2	2.9283233890296e-05	2.9283233890296e-05\\
0.9	2.1	1.44846207890975e-05	1.44846207890975e-05\\
0.9	2.2	1.85679170666315e-05	1.85679170666315e-05\\
0.9	2.3	2.23413912212764e-05	2.23413912212764e-05\\
0.9	2.4	6.09689124893665e-05	6.09689124893665e-05\\
0.9	2.5	0.000417818653984512	0.000417818653984512\\
0.9	2.6	3.17861219018682e-05	3.17861219018682e-05\\
0.9	2.7	4.65795690580717e-05	4.65795690580717e-05\\
0.9	2.8	4.96837356287932e-05	4.96837356287932e-05\\
0.9	2.9	4.07515116951069e-05	4.07515116951069e-05\\
0.9	3	3.03048227171307e-05	3.03048227171307e-05\\
0.9	3.1	2.39582905263156e-05	2.39582905263156e-05\\
0.9	3.2	1.8866769413549e-05	1.8866769413549e-05\\
0.9	3.3	1.48518207826704e-05	1.48518207826704e-05\\
0.9	3.4	1.15971796010401e-05	1.15971796010401e-05\\
0.9	3.5	9.04166218949244e-06	9.04166218949244e-06\\
0.9	3.6	1.96652086517567e-05	1.96652086517567e-05\\
0.9	3.7	0.00246597762734824	0.00246597762734824\\
0.9	3.8	1.15982519109876e-05	1.15982519109876e-05\\
0.9	3.9	2.65633126513887e-06	2.65633126513887e-06\\
0.9	4	4.72146211563804e-06	4.72146211563804e-06\\
0.9	4.1	7.18865817059655e-06	7.18865817059655e-06\\
0.9	4.2	1.02411276068375e-05	1.02411276068375e-05\\
0.9	4.3	1.43175822967234e-05	1.43175822967234e-05\\
0.9	4.4	1.9823922208803e-05	1.9823922208803e-05\\
0.9	4.5	2.70053240744525e-05	2.70053240744525e-05\\
0.9	4.6	3.5599023843148e-05	3.5599023843148e-05\\
0.9	4.7	4.43000065698619e-05	4.43000065698619e-05\\
0.9	4.8	5.05331497519844e-05	5.05331497519844e-05\\
0.9	4.9	5.1426904016827e-05	5.1426904016827e-05\\
0.9	5	4.59989316254327e-05	4.59989316254327e-05\\
0.9	5.1	3.63902708034066e-05	3.63902708034066e-05\\
0.9	5.2	2.6265893174589e-05	2.6265893174589e-05\\
0.9	5.3	1.82002138264296e-05	1.82002138264296e-05\\
0.9	5.4	1.28450142911315e-05	1.28450142911315e-05\\
0.9	5.5	9.74993337513564e-06	9.74993337513564e-06\\
0.9	5.6	8.28076216386412e-06	8.28076216386412e-06\\
0.9	5.7	8.01296626034023e-06	8.01296626034023e-06\\
0.9	5.8	8.78079949201931e-06	8.78079949201931e-06\\
0.9	5.9	1.06194587644341e-05	1.06194587644341e-05\\
0.9	6	1.36954162736736e-05	1.36954162736736e-05\\
1	0	1.01722564582844e-05	1.01722564582844e-05\\
1	0.1	8.10638895329623e-06	8.10638895329623e-06\\
1	0.2	7.53699929691496e-06	7.53699929691496e-06\\
1	0.3	7.9563977312146e-06	7.9563977312146e-06\\
1	0.4	9.23854591947382e-06	9.23854591947382e-06\\
1	0.5	1.13769839408346e-05	1.13769839408346e-05\\
1	0.6	1.44323716251555e-05	1.44323716251555e-05\\
1	0.7	1.86002735427637e-05	1.86002735427637e-05\\
1	0.8	2.40230176782424e-05	2.40230176782424e-05\\
1	0.9	3.01253502933551e-05	3.01253502933551e-05\\
1	1	3.52119051614282e-05	3.52119051614282e-05\\
1	1.1	3.75776181071264e-05	3.75776181071264e-05\\
1	1.2	3.74702923357954e-05	3.74702923357954e-05\\
1	1.3	3.72708109356272e-05	3.72708109356272e-05\\
1	1.4	4.00572348256906e-05	4.00572348256906e-05\\
1	1.5	4.95492311106259e-05	4.95492311106259e-05\\
1	1.6	7.29376558976158e-05	7.29376558976158e-05\\
1	1.7	0.000127918637603604	0.000127918637603604\\
1	1.8	0.000257723195958776	0.000257723195958776\\
1	1.9	0.00201679184698969	0.00201679184698969\\
1	2	3.20156580437435e-05	3.20156580437435e-05\\
1	2.1	2.49762580743749e-05	2.49762580743749e-05\\
1	2.2	2.08839398661735e-05	2.08839398661735e-05\\
1	2.3	1.94030398443832e-05	1.94030398443832e-05\\
1	2.4	2.11162751625335e-05	2.11162751625335e-05\\
1	2.5	0.000508907987023972	0.000508907987023972\\
1	2.6	0.000166012872947951	0.000166012872947951\\
1	2.7	0.0001078607407268	0.0001078607407268\\
1	2.8	7.251148516025e-05	7.251148516025e-05\\
1	2.9	5.06307773246442e-05	5.06307773246442e-05\\
1	3	3.90996090333294e-05	3.90996090333294e-05\\
1	3.1	3.05420692985e-05	3.05420692985e-05\\
1	3.2	2.44600933768336e-05	2.44600933768336e-05\\
1	3.3	2.01701687288271e-05	2.01701687288271e-05\\
1	3.4	1.72447057648411e-05	1.72447057648411e-05\\
1	3.5	1.53976328683446e-05	1.53976328683446e-05\\
1	3.6	1.43953921495638e-05	1.43953921495638e-05\\
1	3.7	0.00254780299468913	0.00254780299468913\\
1	3.8	7.38825282481114e-06	7.38825282481114e-06\\
1	3.9	7.49732682529613e-06	7.49732682529613e-06\\
1	4	8.29512592900214e-06	8.29512592900214e-06\\
1	4.1	9.79130264786952e-06	9.79130264786952e-06\\
1	4.2	1.21038904556181e-05	1.21038904556181e-05\\
1	4.3	1.54366991980443e-05	1.54366991980443e-05\\
1	4.4	2.00802754796672e-05	2.00802754796672e-05\\
1	4.5	2.63586829876836e-05	2.63586829876836e-05\\
1	4.6	3.43284416102835e-05	3.43284416102835e-05\\
1	4.7	4.30882225505573e-05	4.30882225505573e-05\\
1	4.8	5.01009355947483e-05	5.01009355947483e-05\\
1	4.9	5.18284278801451e-05	5.18284278801451e-05\\
1	5	4.64746409660121e-05	4.64746409660121e-05\\
1	5.1	3.6233448492659e-05	3.6233448492659e-05\\
1	5.2	2.5514525331702e-05	2.5514525331702e-05\\
1	5.3	1.73266355893492e-05	1.73266355893492e-05\\
1	5.4	1.22304443550799e-05	1.22304443550799e-05\\
1	5.5	9.57096772962408e-06	9.57096772962408e-06\\
1	5.6	8.63065677611156e-06	8.63065677611156e-06\\
1	5.7	9.0111686693789e-06	9.0111686693789e-06\\
1	5.8	1.06144176986446e-05	1.06144176986446e-05\\
1	5.9	1.35321745715248e-05	1.35321745715248e-05\\
1	6	1.79501232822491e-05	1.79501232822491e-05\\
1.1	0	1.17154139581541e-05	1.17154139581541e-05\\
1.1	0.1	8.10762479848573e-06	8.10762479848573e-06\\
1.1	0.2	6.86200694162925e-06	6.86200694162925e-06\\
1.1	0.3	6.90659670111985e-06	6.90659670111985e-06\\
1.1	0.4	8.02149129639986e-06	8.02149129639986e-06\\
1.1	0.5	1.02848609768425e-05	1.02848609768425e-05\\
1.1	0.6	1.37212264045834e-05	1.37212264045834e-05\\
1.1	0.7	1.8208391572374e-05	1.8208391572374e-05\\
1.1	0.8	2.35163727651521e-05	2.35163727651521e-05\\
1.1	0.9	2.87118366984398e-05	2.87118366984398e-05\\
1.1	1	3.16693013328195e-05	3.16693013328195e-05\\
1.1	1.1	3.07954148801889e-05	3.07954148801889e-05\\
1.1	1.2	2.73286693565995e-05	2.73286693565995e-05\\
1.1	1.3	2.42012816300499e-05	2.42012816300499e-05\\
1.1	1.4	2.34782208410142e-05	2.34782208410142e-05\\
1.1	1.5	2.58795174736184e-05	2.58795174736184e-05\\
1.1	1.6	2.98479635086261e-05	2.98479635086261e-05\\
1.1	1.7	2.40098859593569e-05	2.40098859593569e-05\\
1.1	1.8	1.53718906776324e-05	1.53718906776324e-05\\
1.1	1.9	0.00137481212448629	0.00137481212448629\\
1.1	2	1.39757177870303e-05	1.39757177870303e-05\\
1.1	2.1	3.72084113457639e-06	3.72084113457639e-06\\
1.1	2.2	4.33502520748042e-06	4.33502520748042e-06\\
1.1	2.3	5.51260947754111e-06	5.51260947754111e-06\\
1.1	2.4	3.46027041396616e-05	3.46027041396616e-05\\
1.1	2.5	0.000319359396083271	0.000319359396083271\\
1.1	2.6	2.32205943635332e-05	2.32205943635332e-05\\
1.1	2.7	3.58069959820803e-05	3.58069959820803e-05\\
1.1	2.8	3.68237331375674e-05	3.68237331375674e-05\\
1.1	2.9	2.92108972370209e-05	2.92108972370209e-05\\
1.1	3	2.86652422391117e-05	2.86652422391117e-05\\
1.1	3.1	2.39522960892056e-05	2.39522960892056e-05\\
1.1	3.2	1.99702447668743e-05	1.99702447668743e-05\\
1.1	3.3	1.68378422140769e-05	1.68378422140769e-05\\
1.1	3.4	1.45792583365608e-05	1.45792583365608e-05\\
1.1	3.5	1.38045367609214e-05	1.38045367609214e-05\\
1.1	3.6	3.15837312830155e-05	3.15837312830155e-05\\
1.1	3.7	0.0025324434423709	0.0025324434423709\\
1.1	3.8	1.83278277125415e-05	1.83278277125415e-05\\
1.1	3.9	3.13730494651276e-06	3.13730494651276e-06\\
1.1	4	4.87885075138324e-06	4.87885075138324e-06\\
1.1	4.1	6.9720549688866e-06	6.9720549688866e-06\\
1.1	4.2	9.366699839799e-06	9.366699839799e-06\\
1.1	4.3	1.22809423488006e-05	1.22809423488006e-05\\
1.1	4.4	1.59595542017342e-05	1.59595542017342e-05\\
1.1	4.5	2.07932388833155e-05	2.07932388833155e-05\\
1.1	4.6	2.72588796557779e-05	2.72588796557779e-05\\
1.1	4.7	3.52736551926371e-05	3.52736551926371e-05\\
1.1	4.8	4.30081626811733e-05	4.30081626811733e-05\\
1.1	4.9	4.65911759127636e-05	4.65911759127636e-05\\
1.1	5	4.29132220275955e-05	4.29132220275955e-05\\
1.1	5.1	3.34999706680673e-05	3.34999706680673e-05\\
1.1	5.2	2.32688452453725e-05	2.32688452453725e-05\\
1.1	5.3	1.56991820057826e-05	1.56991820057826e-05\\
1.1	5.4	1.13320330570422e-05	1.13320330570422e-05\\
1.1	5.5	9.4153810867311e-06	9.4153810867311e-06\\
1.1	5.6	9.25803819528285e-06	9.25803819528285e-06\\
1.1	5.7	1.05542074144699e-05	1.05542074144699e-05\\
1.1	5.8	1.32816101364834e-05	1.32816101364834e-05\\
1.1	5.9	1.75539559414892e-05	1.75539559414892e-05\\
1.1	6	2.35431204837702e-05	2.35431204837702e-05\\
1.2	0	1.64081414971228e-05	1.64081414971228e-05\\
1.2	0.1	9.1823441044826e-06	9.1823441044826e-06\\
1.2	0.2	6.55178258453573e-06	6.55178258453573e-06\\
1.2	0.3	5.8497819687903e-06	5.8497819687903e-06\\
1.2	0.4	6.36151348726351e-06	6.36151348726351e-06\\
1.2	0.5	8.16849256633047e-06	8.16849256633047e-06\\
1.2	0.6	1.15252612006916e-05	1.15252612006916e-05\\
1.2	0.7	1.61402242491764e-05	1.61402242491764e-05\\
1.2	0.8	2.10643400138283e-05	2.10643400138283e-05\\
1.2	0.9	2.48343685934844e-05	2.48343685934844e-05\\
1.2	1	2.52207456303026e-05	2.52207456303026e-05\\
1.2	1.1	2.13922739147132e-05	2.13922739147132e-05\\
1.2	1.2	1.59524627734875e-05	1.59524627734875e-05\\
1.2	1.3	1.1890175113157e-05	1.1890175113157e-05\\
1.2	1.4	9.9158039262105e-06	9.9158039262105e-06\\
1.2	1.5	9.10864333970391e-06	9.10864333970391e-06\\
1.2	1.6	7.84286092469491e-06	7.84286092469491e-06\\
1.2	1.7	8.18349892780529e-06	8.18349892780529e-06\\
1.2	1.8	3.9579164579911e-05	3.9579164579911e-05\\
1.2	1.9	0.000568559178688731	0.000568559178688731\\
1.2	2	0.000137133837468173	0.000137133837468173\\
1.2	2.1	4.43561469592985e-06	4.43561469592985e-06\\
1.2	2.2	2.396511430721e-06	2.396511430721e-06\\
1.2	2.3	7.82694439926924e-06	7.82694439926924e-06\\
1.2	2.4	0.00010734802199087	0.00010734802199087\\
1.2	2.5	0.000153846482338307	0.000153846482338307\\
1.2	2.6	2.22505452149202e-05	2.22505452149202e-05\\
1.2	2.7	1.97584422385832e-05	1.97584422385832e-05\\
1.2	2.8	1.53808882652922e-05	1.53808882652922e-05\\
1.2	2.9	1.31376426914946e-05	1.31376426914946e-05\\
1.2	3	1.59278710418362e-05	1.59278710418362e-05\\
1.2	3.1	1.48674494395537e-05	1.48674494395537e-05\\
1.2	3.2	1.38213362068761e-05	1.38213362068761e-05\\
1.2	3.3	1.35609797688806e-05	1.35609797688806e-05\\
1.2	3.4	1.60139982940031e-05	1.60139982940031e-05\\
1.2	3.5	2.93963942391532e-05	2.93963942391532e-05\\
1.2	3.6	0.000161137316181544	0.000161137316181544\\
1.2	3.7	0.00248795395911029	0.00248795395911029\\
1.2	3.8	0.000165127971873964	0.000165127971873964\\
1.2	3.9	9.54737821310869e-06	9.54737821310869e-06\\
1.2	4	3.64479919289237e-06	3.64479919289237e-06\\
1.2	4.1	4.32040722148763e-06	4.32040722148763e-06\\
1.2	4.2	6.14384408564781e-06	6.14384408564781e-06\\
1.2	4.3	8.53536923666758e-06	8.53536923666758e-06\\
1.2	4.4	1.13210672143263e-05	1.13210672143263e-05\\
1.2	4.5	1.46096459348338e-05	1.46096459348338e-05\\
1.2	4.6	1.89824262909267e-05	1.89824262909267e-05\\
1.2	4.7	2.5156014750887e-05	2.5156014750887e-05\\
1.2	4.8	3.2588628162186e-05	3.2588628162186e-05\\
1.2	4.9	3.7912105854276e-05	3.7912105854276e-05\\
1.2	5	3.66847190698734e-05	3.66847190698734e-05\\
1.2	5.1	2.9031807033277e-05	2.9031807033277e-05\\
1.2	5.2	2.00428644910711e-05	2.00428644910711e-05\\
1.2	5.3	1.36401632109295e-05	1.36401632109295e-05\\
1.2	5.4	1.03661161597284e-05	1.03661161597284e-05\\
1.2	5.5	9.46381726740024e-06	9.46381726740024e-06\\
1.2	5.6	1.03566709792951e-05	1.03566709792951e-05\\
1.2	5.7	1.28520070782261e-05	1.28520070782261e-05\\
1.2	5.8	1.69718910790643e-05	1.69718910790643e-05\\
1.2	5.9	2.28320070760241e-05	2.28320070760241e-05\\
1.2	6	3.06401516045404e-05	3.06401516045404e-05\\
1.3	0	2.91397185957493e-05	2.91397185957493e-05\\
1.3	0.1	1.2941711063547e-05	1.2941711063547e-05\\
1.3	0.2	7.25261728305364e-06	7.25261728305364e-06\\
1.3	0.3	5.28903396525469e-06	5.28903396525469e-06\\
1.3	0.4	4.95250995969071e-06	4.95250995969071e-06\\
1.3	0.5	5.8403875282639e-06	5.8403875282639e-06\\
1.3	0.6	8.36323818482754e-06	8.36323818482754e-06\\
1.3	0.7	1.26891767577158e-05	1.26891767577158e-05\\
1.3	0.8	1.73863776709623e-05	1.73863776709623e-05\\
1.3	0.9	2.02243159518394e-05	2.02243159518394e-05\\
1.3	1	1.91302021082641e-05	1.91302021082641e-05\\
1.3	1.1	1.42139374883733e-05	1.42139374883733e-05\\
1.3	1.2	9.13027126273706e-06	9.13027126273706e-06\\
1.3	1.3	6.3563116596533e-06	6.3563116596533e-06\\
1.3	1.4	5.61638187799455e-06	5.61638187799455e-06\\
1.3	1.5	5.86006763401916e-06	5.86006763401916e-06\\
1.3	1.6	7.32216959826893e-06	7.32216959826893e-06\\
1.3	1.7	1.48736109894061e-05	1.48736109894061e-05\\
1.3	1.8	3.74129133129009e-05	3.74129133129009e-05\\
1.3	1.9	0.000236935887817259	0.000236935887817259\\
1.3	2	0.000205501511906541	0.000205501511906541\\
1.3	2.1	3.7709154494298e-05	3.7709154494298e-05\\
1.3	2.2	1.727796695631e-05	1.727796695631e-05\\
1.3	2.3	4.74124775437666e-05	4.74124775437666e-05\\
1.3	2.4	0.000116710035406492	0.000116710035406492\\
1.3	2.5	0.00010327337854103	0.00010327337854103\\
1.3	2.6	2.00751729493468e-05	2.00751729493468e-05\\
1.3	2.7	5.15457079277749e-05	5.15457079277749e-05\\
1.3	2.8	2.33902507180958e-05	2.33902507180958e-05\\
1.3	2.9	1.00664806210868e-05	1.00664806210868e-05\\
1.3	3	1.04255846313395e-05	1.04255846313395e-05\\
1.3	3.1	1.04199487334482e-05	1.04199487334482e-05\\
1.3	3.2	1.17133200795612e-05	1.17133200795612e-05\\
1.3	3.3	1.59484270089174e-05	1.59484270089174e-05\\
1.3	3.4	2.90955235321024e-05	2.90955235321024e-05\\
1.3	3.5	6.89567946797628e-05	6.89567946797628e-05\\
1.3	3.6	0.000601950845078127	0.000601950845078127\\
1.3	3.7	0.00248838580447895	0.00248838580447895\\
1.3	3.8	0.00044328864007062	0.00044328864007062\\
1.3	3.9	5.21543274076556e-05	5.21543274076556e-05\\
1.3	4	9.87872065389724e-06	9.87872065389724e-06\\
1.3	4.1	5.23958086047783e-06	5.23958086047783e-06\\
1.3	4.2	5.4110882401823e-06	5.4110882401823e-06\\
1.3	4.3	6.84309005998023e-06	6.84309005998023e-06\\
1.3	4.4	8.73930165271292e-06	8.73930165271292e-06\\
1.3	4.5	1.07278670803512e-05	1.07278670803512e-05\\
1.3	4.6	1.30525207778548e-05	1.30525207778548e-05\\
1.3	4.7	1.67754333625065e-05	1.67754333625065e-05\\
1.3	4.8	2.26629876997869e-05	2.26629876997869e-05\\
1.3	4.9	2.87063777632371e-05	2.87063777632371e-05\\
1.3	5	2.97020122775194e-05	2.97020122775194e-05\\
1.3	5.1	2.39329424169028e-05	2.39329424169028e-05\\
1.3	5.2	1.64385042626811e-05	1.64385042626811e-05\\
1.3	5.3	1.15028589038801e-05	1.15028589038801e-05\\
1.3	5.4	9.5795643235677e-06	9.5795643235677e-06\\
1.3	5.5	9.94488291380465e-06	9.94488291380465e-06\\
1.3	5.6	1.21648678713634e-05	1.21648678713634e-05\\
1.3	5.7	1.61432233524235e-05	1.61432233524235e-05\\
1.3	5.8	2.19621552848443e-05	2.19621552848443e-05\\
1.3	5.9	2.98265219549379e-05	2.98265219549379e-05\\
1.3	6	4.0121899283286e-05	4.0121899283286e-05\\
1.4	0	5.79628785065118e-05	5.79628785065118e-05\\
1.4	0.1	2.27187070863134e-05	2.27187070863134e-05\\
1.4	0.2	1.01983824777809e-05	1.01983824777809e-05\\
1.4	0.3	5.70472593054406e-06	5.70472593054406e-06\\
1.4	0.4	4.22844057548455e-06	4.22844057548455e-06\\
1.4	0.5	4.16557540279418e-06	4.16557540279418e-06\\
1.4	0.6	5.42467122126067e-06	5.42467122126067e-06\\
1.4	0.7	8.72508933120827e-06	8.72508933120827e-06\\
1.4	0.8	1.3357468842979e-05	1.3357468842979e-05\\
1.4	0.9	1.62313027919606e-05	1.62313027919606e-05\\
1.4	1	1.50028730510155e-05	1.50028730510155e-05\\
1.4	1.1	1.02293481719981e-05	1.02293481719981e-05\\
1.4	1.2	6.33846906590053e-06	6.33846906590053e-06\\
1.4	1.3	5.30405603798447e-06	5.30405603798447e-06\\
1.4	1.4	6.47722945367207e-06	6.47722945367207e-06\\
1.4	1.5	8.94453143830783e-06	8.94453143830783e-06\\
1.4	1.6	1.34116262191842e-05	1.34116262191842e-05\\
1.4	1.7	1.94415555167694e-05	1.94415555167694e-05\\
1.4	1.8	3.45303401916977e-05	3.45303401916977e-05\\
1.4	1.9	0.000150528399692765	0.000150528399692765\\
1.4	2	0.000223385362870054	0.000223385362870054\\
1.4	2.1	0.000150225021296976	0.000150225021296976\\
1.4	2.2	8.24384283790194e-05	8.24384283790194e-05\\
1.4	2.3	8.52999633314082e-05	8.52999633314082e-05\\
1.4	2.4	7.94836390343066e-05	7.94836390343066e-05\\
1.4	2.5	0.000108561578406809	0.000108561578406809\\
1.4	2.6	3.94306876952446e-05	3.94306876952446e-05\\
1.4	2.7	6.27638769183594e-05	6.27638769183594e-05\\
1.4	2.8	7.21045455897795e-05	7.21045455897795e-05\\
1.4	2.9	1.86933693850573e-05	1.86933693850573e-05\\
1.4	3	9.82063055077321e-06	9.82063055077321e-06\\
1.4	3.1	9.5349865589036e-06	9.5349865589036e-06\\
1.4	3.2	1.2921882315333e-05	1.2921882315333e-05\\
1.4	3.3	2.32238185083824e-05	2.32238185083824e-05\\
1.4	3.4	5.07350331093008e-05	5.07350331093008e-05\\
1.4	3.5	0.000164935102831312	0.000164935102831312\\
1.4	3.6	0.00127812199788479	0.00127812199788479\\
1.4	3.7	0.00254959279203756	0.00254959279203756\\
1.4	3.8	0.000760442525603699	0.000760442525603699\\
1.4	3.9	0.000152166840285673	0.000152166840285673\\
1.4	4	3.42632712938344e-05	3.42632712938344e-05\\
1.4	4.1	1.23078787139428e-05	1.23078787139428e-05\\
1.4	4.2	8.11626745304616e-06	8.11626745304616e-06\\
1.4	4.3	7.85388180480668e-06	7.85388180480668e-06\\
1.4	4.4	8.68514468620128e-06	8.68514468620128e-06\\
1.4	4.5	9.60755577935239e-06	9.60755577935239e-06\\
1.4	4.6	1.02820047169388e-05	1.02820047169388e-05\\
1.4	4.7	1.15879660807022e-05	1.15879660807022e-05\\
1.4	4.8	1.52032854294502e-05	1.52032854294502e-05\\
1.4	4.9	2.08674986737716e-05	2.08674986737716e-05\\
1.4	5	2.33906574897172e-05	2.33906574897172e-05\\
1.4	5.1	1.90264106757268e-05	1.90264106757268e-05\\
1.4	5.2	1.29145048831987e-05	1.29145048831987e-05\\
1.4	5.3	9.61797563833304e-06	9.61797563833304e-06\\
1.4	5.4	9.28085880746865e-06	9.28085880746865e-06\\
1.4	5.5	1.11943167569446e-05	1.11943167569446e-05\\
1.4	5.6	1.50926251722512e-05	1.50926251722512e-05\\
1.4	5.7	2.10835308007756e-05	2.10835308007756e-05\\
1.4	5.8	2.94375948133452e-05	2.94375948133452e-05\\
1.4	5.9	4.04743370538935e-05	4.04743370538935e-05\\
1.4	6	5.46599775441689e-05	5.46599775441689e-05\\
1.5	0	0.000100001967946493	0.000100001967946493\\
1.5	0.1	4.04055392323576e-05	4.04055392323576e-05\\
1.5	0.2	1.71724111512963e-05	1.71724111512963e-05\\
1.5	0.3	7.99727294216005e-06	7.99727294216005e-06\\
1.5	0.4	4.4980579164902e-06	4.4980579164902e-06\\
1.5	0.5	3.41550448653868e-06	3.41550448653868e-06\\
1.5	0.6	3.63400713085195e-06	3.63400713085195e-06\\
1.5	0.7	5.43996820956766e-06	5.43996820956766e-06\\
1.5	0.8	9.62577017760334e-06	9.62577017760334e-06\\
1.5	0.9	1.33662590975062e-05	1.33662590975062e-05\\
1.5	1	1.29421393988697e-05	1.29421393988697e-05\\
1.5	1.1	8.56383137809256e-06	8.56383137809256e-06\\
1.5	1.2	6.09956555411293e-06	6.09956555411293e-06\\
1.5	1.3	7.70343471128981e-06	7.70343471128981e-06\\
1.5	1.4	1.1953910894579e-05	1.1953910894579e-05\\
1.5	1.5	1.58966780721963e-05	1.58966780721963e-05\\
1.5	1.6	1.8074498442914e-05	1.8074498442914e-05\\
1.5	1.7	2.18313011004249e-05	2.18313011004249e-05\\
1.5	1.8	4.80610375751907e-05	4.80610375751907e-05\\
1.5	1.9	0.000136622984789586	0.000136622984789586\\
1.5	2	0.000214049602310421	0.000214049602310421\\
1.5	2.1	0.000243125098976479	0.000243125098976479\\
1.5	2.2	0.000123985684970461	0.000123985684970461\\
1.5	2.3	6.03916351816207e-05	6.03916351816207e-05\\
1.5	2.4	5.0857146717302e-05	5.0857146717302e-05\\
1.5	2.5	0.000126611380305268	0.000126611380305268\\
1.5	2.6	9.68617934631439e-05	9.68617934631439e-05\\
1.5	2.7	5.16056790144804e-05	5.16056790144804e-05\\
1.5	2.8	6.00596281775495e-05	6.00596281775495e-05\\
1.5	2.9	2.47803424114334e-05	2.47803424114334e-05\\
1.5	3	1.13970776905771e-05	1.13970776905771e-05\\
1.5	3.1	1.03155097179071e-05	1.03155097179071e-05\\
1.5	3.2	1.53769362635044e-05	1.53769362635044e-05\\
1.5	3.3	3.18362105454248e-05	3.18362105454248e-05\\
1.5	3.4	8.36712550756576e-05	8.36712550756576e-05\\
1.5	3.5	0.000370978632816069	0.000370978632816069\\
1.5	3.6	0.00186963157804501	0.00186963157804501\\
1.5	3.7	0.0026065451315414	0.0026065451315414\\
1.5	3.8	0.00107320799744348	0.00107320799744348\\
1.5	3.9	0.000293297488945684	0.000293297488945684\\
1.5	4	8.28476251091514e-05	8.28476251091514e-05\\
1.5	4.1	3.08353985025065e-05	3.08353985025065e-05\\
1.5	4.2	1.66714051722252e-05	1.66714051722252e-05\\
1.5	4.3	1.26702738486296e-05	1.26702738486296e-05\\
1.5	4.4	1.16016576736965e-05	1.16016576736965e-05\\
1.5	4.5	1.1168069445589e-05	1.1168069445589e-05\\
1.5	4.6	1.04173744494324e-05	1.04173744494324e-05\\
1.5	4.7	9.48710369198839e-06	9.48710369198839e-06\\
1.5	4.8	1.04557775135989e-05	1.04557775135989e-05\\
1.5	4.9	1.49247407622535e-05	1.49247407622535e-05\\
1.5	5	1.83779869085513e-05	1.83779869085513e-05\\
1.5	5.1	1.46538058791642e-05	1.46538058791642e-05\\
1.5	5.2	9.77916319157685e-06	9.77916319157685e-06\\
1.5	5.3	8.40836149207213e-06	8.40836149207213e-06\\
1.5	5.4	1.0012107898028e-05	1.0012107898028e-05\\
1.5	5.5	1.40213928277674e-05	1.40213928277674e-05\\
1.5	5.6	2.0704661347687e-05	2.0704661347687e-05\\
1.5	5.7	3.05101825967817e-05	3.05101825967817e-05\\
1.5	5.8	4.34269904816318e-05	4.34269904816318e-05\\
1.5	5.9	5.9055856473378e-05	5.9055856473378e-05\\
1.5	6	7.75093278425964e-05	7.75093278425964e-05\\
1.6	0	0.00013970732474492	0.00013970732474492\\
1.6	0.1	6.01328096262346e-05	6.01328096262346e-05\\
1.6	0.2	2.75108942073341e-05	2.75108942073341e-05\\
1.6	0.3	1.30178461585554e-05	1.30178461585554e-05\\
1.6	0.4	6.48612836457775e-06	6.48612836457775e-06\\
1.6	0.5	3.75231009690923e-06	3.75231009690923e-06\\
1.6	0.6	3.00826509723308e-06	3.00826509723308e-06\\
1.6	0.7	3.65933287367344e-06	3.65933287367344e-06\\
1.6	0.8	6.56509496796182e-06	6.56509496796182e-06\\
1.6	0.9	1.14372388445994e-05	1.14372388445994e-05\\
1.6	1	1.25254195382461e-05	1.25254195382461e-05\\
1.6	1.1	8.33510980206728e-06	8.33510980206728e-06\\
1.6	1.2	8.61735417105125e-06	8.61735417105125e-06\\
1.6	1.3	1.46717542997747e-05	1.46717542997747e-05\\
1.6	1.4	1.83801007711625e-05	1.83801007711625e-05\\
1.6	1.5	1.80384629734688e-05	1.80384629734688e-05\\
1.6	1.6	1.865959348586e-05	1.865959348586e-05\\
1.6	1.7	3.1198148937704e-05	3.1198148937704e-05\\
1.6	1.8	7.45617956458891e-05	7.45617956458891e-05\\
1.6	1.9	0.000145861612278781	0.000145861612278781\\
1.6	2	0.000207282734085744	0.000207282734085744\\
1.6	2.1	0.000246316257374903	0.000246316257374903\\
1.6	2.2	0.000121623622590884	0.000121623622590884\\
1.6	2.3	4.58718407144472e-05	4.58718407144472e-05\\
1.6	2.4	4.1573315219065e-05	4.1573315219065e-05\\
1.6	2.5	0.000125677946481481	0.000125677946481481\\
1.6	2.6	0.000164765773001334	0.000164765773001334\\
1.6	2.7	5.55287304782368e-05	5.55287304782368e-05\\
1.6	2.8	2.81533675208189e-05	2.81533675208189e-05\\
1.6	2.9	1.62694804373487e-05	1.62694804373487e-05\\
1.6	3	1.39323260775129e-05	1.39323260775129e-05\\
1.6	3.1	1.25162661672455e-05	1.25162661672455e-05\\
1.6	3.2	1.9179239937366e-05	1.9179239937366e-05\\
1.6	3.3	4.3541255876375e-05	4.3541255876375e-05\\
1.6	3.4	0.00014078218286481	0.00014078218286481\\
1.6	3.5	0.000655246983204063	0.000655246983204063\\
1.6	3.6	0.00219142765667401	0.00219142765667401\\
1.6	3.7	0.00256471736695405	0.00256471736695405\\
1.6	3.8	0.00128424821322759	0.00128424821322759\\
1.6	3.9	0.000429286722644696	0.000429286722644696\\
1.6	4	0.000137600557757383	0.000137600557757383\\
1.6	4.1	5.60944366023153e-05	5.60944366023153e-05\\
1.6	4.2	3.12241597062133e-05	3.12241597062133e-05\\
1.6	4.3	2.24886257911396e-05	2.24886257911396e-05\\
1.6	4.4	1.88803071137405e-05	1.88803071137405e-05\\
1.6	4.5	1.63839866498991e-05	1.63839866498991e-05\\
1.6	4.6	1.36926446814983e-05	1.36926446814983e-05\\
1.6	4.7	1.0576544908542e-05	1.0576544908542e-05\\
1.6	4.8	8.28127866086711e-06	8.28127866086711e-06\\
1.6	4.9	1.05704278431441e-05	1.05704278431441e-05\\
1.6	5	1.47359877213448e-05	1.47359877213448e-05\\
1.6	5.1	1.07415449972088e-05	1.07415449972088e-05\\
1.6	5.2	7.5093730225344e-06	7.5093730225344e-06\\
1.6	5.3	8.82143510001798e-06	8.82143510001798e-06\\
1.6	5.4	1.35184879389516e-05	1.35184879389516e-05\\
1.6	5.5	2.23638004550348e-05	2.23638004550348e-05\\
1.6	5.6	3.5982289759743e-05	3.5982289759743e-05\\
1.6	5.7	5.27503826296117e-05	5.27503826296117e-05\\
1.6	5.8	6.98586539170568e-05	6.98586539170568e-05\\
1.6	5.9	8.65111465096237e-05	8.65111465096237e-05\\
1.6	6	0.00010500207412053	0.00010500207412053\\
1.7	0	0.000192298188136938	0.000192298188136938\\
1.7	0.1	8.63404589270773e-05	8.63404589270773e-05\\
1.7	0.2	4.36445820975225e-05	4.36445820975225e-05\\
1.7	0.3	2.31735166334492e-05	2.31735166334492e-05\\
1.7	0.4	1.23168198004816e-05	1.23168198004816e-05\\
1.7	0.5	6.57837281278576e-06	6.57837281278576e-06\\
1.7	0.6	3.88027604066245e-06	3.88027604066245e-06\\
1.7	0.7	3.26639252808823e-06	3.26639252808823e-06\\
1.7	0.8	4.75853403755213e-06	4.75853403755213e-06\\
1.7	0.9	9.63777456962658e-06	9.63777456962658e-06\\
1.7	1	1.33058447155575e-05	1.33058447155575e-05\\
1.7	1.1	9.54534769545563e-06	9.54534769545563e-06\\
1.7	1.2	1.57268024245872e-05	1.57268024245872e-05\\
1.7	1.3	1.68485917839417e-05	1.68485917839417e-05\\
1.7	1.4	1.4705344465542e-05	1.4705344465542e-05\\
1.7	1.5	1.57408558472975e-05	1.57408558472975e-05\\
1.7	1.6	2.47098327782303e-05	2.47098327782303e-05\\
1.7	1.7	5.85892844857599e-05	5.85892844857599e-05\\
1.7	1.8	0.000124061502260319	0.000124061502260319\\
1.7	1.9	0.000223874740675581	0.000223874740675581\\
1.7	2	0.000249349877759911	0.000249349877759911\\
1.7	2.1	0.000220473450530802	0.000220473450530802\\
1.7	2.2	0.00012139711383039	0.00012139711383039\\
1.7	2.3	6.03834560097269e-05	6.03834560097269e-05\\
1.7	2.4	5.30401544176359e-05	5.30401544176359e-05\\
1.7	2.5	0.000105783216706121	0.000105783216706121\\
1.7	2.6	0.000145034243499511	0.000145034243499511\\
1.7	2.7	5.4943846887039e-05	5.4943846887039e-05\\
1.7	2.8	1.78757389528961e-05	1.78757389528961e-05\\
1.7	2.9	1.06310455759533e-05	1.06310455759533e-05\\
1.7	3	1.77768778469234e-05	1.77768778469234e-05\\
1.7	3.1	1.75141123021083e-05	1.75141123021083e-05\\
1.7	3.2	2.77931670107622e-05	2.77931670107622e-05\\
1.7	3.3	6.64352283156429e-05	6.64352283156429e-05\\
1.7	3.4	0.000228869689735667	0.000228869689735667\\
1.7	3.5	0.000918163808749641	0.000918163808749641\\
1.7	3.6	0.00228145872986944	0.00228145872986944\\
1.7	3.7	0.00240248753356191	0.00240248753356191\\
1.7	3.8	0.00131597319170064	0.00131597319170064\\
1.7	3.9	0.000499435846684763	0.000499435846684763\\
1.7	4	0.000173614505396943	0.000173614505396943\\
1.7	4.1	7.28764537511273e-05	7.28764537511273e-05\\
1.7	4.2	4.19925129208113e-05	4.19925129208113e-05\\
1.7	4.3	3.19202005714604e-05	3.19202005714604e-05\\
1.7	4.4	2.9006081985176e-05	2.9006081985176e-05\\
1.7	4.5	2.76470632250616e-05	2.76470632250616e-05\\
1.7	4.6	2.37965571206926e-05	2.37965571206926e-05\\
1.7	4.7	1.69726257224012e-05	1.69726257224012e-05\\
1.7	4.8	1.01626427412789e-05	1.01626427412789e-05\\
1.7	4.9	7.40385229997186e-06	7.40385229997186e-06\\
1.7	5	1.23136278951019e-05	1.23136278951019e-05\\
1.7	5.1	7.14654713634961e-06	7.14654713634961e-06\\
1.7	5.2	8.03065013541673e-06	8.03065013541673e-06\\
1.7	5.3	1.56836773147659e-05	1.56836773147659e-05\\
1.7	5.4	3.23432753641476e-05	3.23432753641476e-05\\
1.7	5.5	5.58694761358043e-05	5.58694761358043e-05\\
1.7	5.6	7.70912776142858e-05	7.70912776142858e-05\\
1.7	5.7	9.06439309639395e-05	9.06439309639395e-05\\
1.7	5.8	9.95063597929289e-05	9.95063597929289e-05\\
1.7	5.9	0.000109997621867192	0.000109997621867192\\
1.7	6	0.000128085537730545	0.000128085537730545\\
1.8	0	0.000267417718386347	0.000267417718386347\\
1.8	0.1	0.000130195712271403	0.000130195712271403\\
1.8	0.2	7.73468061774382e-05	7.73468061774382e-05\\
1.8	0.3	5.23751103221075e-05	5.23751103221075e-05\\
1.8	0.4	3.7015514279348e-05	3.7015514279348e-05\\
1.8	0.5	2.54033024371134e-05	2.54033024371134e-05\\
1.8	0.6	1.62902012654579e-05	1.62902012654579e-05\\
1.8	0.7	9.47017707319555e-06	9.47017707319555e-06\\
1.8	0.8	5.54498074917294e-06	5.54498074917294e-06\\
1.8	0.9	7.32453206163511e-06	7.32453206163511e-06\\
1.8	1	1.48739076086086e-05	1.48739076086086e-05\\
1.8	1.1	1.45088129551339e-05	1.45088129551339e-05\\
1.8	1.2	1.22509356843141e-05	1.22509356843141e-05\\
1.8	1.3	1.6174385398245e-05	1.6174385398245e-05\\
1.8	1.4	1.99971122910298e-05	1.99971122910298e-05\\
1.8	1.5	2.8088885192987e-05	2.8088885192987e-05\\
1.8	1.6	5.85743112328123e-05	5.85743112328123e-05\\
1.8	1.7	0.000127324404725752	0.000127324404725752\\
1.8	1.8	0.000229007962476024	0.000229007962476024\\
1.8	1.9	0.000349056551920362	0.000349056551920362\\
1.8	2	0.000297306869223127	0.000297306869223127\\
1.8	2.1	0.000229937779626884	0.000229937779626884\\
1.8	2.2	0.000162831423828903	0.000162831423828903\\
1.8	2.3	0.000112084999864284	0.000112084999864284\\
1.8	2.4	8.42626732553122e-05	8.42626732553122e-05\\
1.8	2.5	8.65014734456444e-05	8.65014734456444e-05\\
1.8	2.6	8.34054001169393e-05	8.34054001169393e-05\\
1.8	2.7	3.98750274106755e-05	3.98750274106755e-05\\
1.8	2.8	1.38479823549306e-05	1.38479823549306e-05\\
1.8	2.9	8.01661741645251e-06	8.01661741645251e-06\\
1.8	3	2.32888965414171e-05	2.32888965414171e-05\\
1.8	3.1	2.65814730967796e-05	2.65814730967796e-05\\
1.8	3.2	4.51359437525773e-05	4.51359437525773e-05\\
1.8	3.3	0.000108483891604556	0.000108483891604556\\
1.8	3.4	0.000345675320142022	0.000345675320142022\\
1.8	3.5	0.00111946078546372	0.00111946078546372\\
1.8	3.6	0.00225280049028135	0.00225280049028135\\
1.8	3.7	0.00217659068917047	0.00217659068917047\\
1.8	3.8	0.00121577916753408	0.00121577916753408\\
1.8	3.9	0.000501964410188412	0.000501964410188412\\
1.8	4	0.000187952085768083	0.000187952085768083\\
1.8	4.1	8.06580516412115e-05	8.06580516412115e-05\\
1.8	4.2	4.56328620521293e-05	4.56328620521293e-05\\
1.8	4.3	3.41249017327406e-05	3.41249017327406e-05\\
1.8	4.4	3.21534321219371e-05	3.21534321219371e-05\\
1.8	4.5	3.65054811912006e-05	3.65054811912006e-05\\
1.8	4.6	4.59326075016134e-05	4.59326075016134e-05\\
1.8	4.7	5.16147660666487e-05	5.16147660666487e-05\\
1.8	4.8	3.487150912259e-05	3.487150912259e-05\\
1.8	4.9	1.15999534505581e-05	1.15999534505581e-05\\
1.8	5	1.09383128970267e-05	1.09383128970267e-05\\
1.8	5.1	9.91931227512713e-06	9.91931227512713e-06\\
1.8	5.2	4.00277108097715e-05	4.00277108097715e-05\\
1.8	5.3	9.11633182822459e-05	9.11633182822459e-05\\
1.8	5.4	0.000117071305642861	0.000117071305642861\\
1.8	5.5	0.000118203010652789	0.000118203010652789\\
1.8	5.6	0.00011190586639399	0.00011190586639399\\
1.8	5.7	0.000108294117694984	0.000108294117694984\\
1.8	5.8	0.000112582710229818	0.000112582710229818\\
1.8	5.9	0.000128682607049995	0.000128682607049995\\
1.8	6	0.000162577612827444	0.000162577612827444\\
1.9	0	0.000314783665109934	0.000314783665109934\\
1.9	0.1	0.000170908865500879	0.000170908865500879\\
1.9	0.2	0.000117020484187412	0.000117020484187412\\
1.9	0.3	9.80609442441248e-05	9.80609442441248e-05\\
1.9	0.4	9.44418794781911e-05	9.44418794781911e-05\\
1.9	0.5	9.79435437452538e-05	9.79435437452538e-05\\
1.9	0.6	0.000104137645891957	0.000104137645891957\\
1.9	0.7	0.000113666514276805	0.000113666514276805\\
1.9	0.8	0.000132782480723682	0.000132782480723682\\
1.9	0.9	0.000160974775068434	0.000160974775068434\\
1.9	1	0.000175711101656794	0.000175711101656794\\
1.9	1.1	0.000151455438617965	0.000151455438617965\\
1.9	1.2	0.000104921007214744	0.000104921007214744\\
1.9	1.3	7.26542051212647e-05	7.26542051212647e-05\\
1.9	1.4	6.66001722966736e-05	6.66001722966736e-05\\
1.9	1.5	8.66645311648882e-05	8.66645311648882e-05\\
1.9	1.6	0.000134305530050342	0.000134305530050342\\
1.9	1.7	0.000209247898877075	0.000209247898877075\\
1.9	1.8	0.000326741555443225	0.000326741555443225\\
1.9	1.9	0.000378757966339505	0.000378757966339505\\
1.9	2	0.000337843169759076	0.000337843169759076\\
1.9	2.1	0.000318608410790144	0.000318608410790144\\
1.9	2.2	0.000278781142557311	0.000278781142557311\\
1.9	2.3	0.000210940228900903	0.000210940228900903\\
1.9	2.4	0.000134451178504677	0.000134451178504677\\
1.9	2.5	8.47996205881851e-05	8.47996205881851e-05\\
1.9	2.6	5.72282331536791e-05	5.72282331536791e-05\\
1.9	2.7	3.16378560380342e-05	3.16378560380342e-05\\
1.9	2.8	1.40013306165883e-05	1.40013306165883e-05\\
1.9	2.9	8.32341467506449e-06	8.32341467506449e-06\\
1.9	3	3.28235613531715e-05	3.28235613531715e-05\\
1.9	3.1	4.15554681484065e-05	4.15554681484065e-05\\
1.9	3.2	7.37558510439877e-05	7.37558510439877e-05\\
1.9	3.3	0.000171289544842376	0.000171289544842376\\
1.9	3.4	0.000479913121813007	0.000479913121813007\\
1.9	3.5	0.00127000109706885	0.00127000109706885\\
1.9	3.6	0.0021713119355245	0.0021713119355245\\
1.9	3.7	0.00195084833846195	0.00195084833846195\\
1.9	3.8	0.00108695710773324	0.00108695710773324\\
1.9	3.9	0.000485640622599065	0.000485640622599065\\
1.9	4	0.000200883653061084	0.000200883653061084\\
1.9	4.1	9.01556242180312e-05	9.01556242180312e-05\\
1.9	4.2	4.99102934461968e-05	4.99102934461968e-05\\
1.9	4.3	3.4732023713968e-05	3.4732023713968e-05\\
1.9	4.4	2.89614405748856e-05	2.89614405748856e-05\\
1.9	4.5	2.81208964022905e-05	2.81208964022905e-05\\
1.9	4.6	3.24306350345126e-05	3.24306350345126e-05\\
1.9	4.7	4.51482258408625e-05	4.51482258408625e-05\\
1.9	4.8	7.09280319228013e-05	7.09280319228013e-05\\
1.9	4.9	0.000107418950835996	0.000107418950835996\\
1.9	5	0.000135198518609373	0.000135198518609373\\
1.9	5.1	0.000136074860445893	0.000136074860445893\\
1.9	5.2	0.000118880966250577	0.000118880966250577\\
1.9	5.3	0.000102826976900738	0.000102826976900738\\
1.9	5.4	9.53286165845763e-05	9.53286165845763e-05\\
1.9	5.5	9.43224121813653e-05	9.43224121813653e-05\\
1.9	5.6	9.78888178207301e-05	9.78888178207301e-05\\
1.9	5.7	0.000108003232383801	0.000108003232383801\\
1.9	5.8	0.000129809687213283	0.000129809687213283\\
1.9	5.9	0.000172824039280377	0.000172824039280377\\
1.9	6	0.000256532397801341	0.000256532397801341\\
2	0	0.000343052686298395	0.000343052686298395\\
2	0.1	0.000197878441892689	0.000197878441892689\\
2	0.2	0.000135487036212777	0.000135487036212777\\
2	0.3	0.000111671438771067	0.000111671438771067\\
2	0.4	0.000111788233563216	0.000111788233563216\\
2	0.5	0.000137571486376842	0.000137571486376842\\
2	0.6	0.000191602055787336	0.000191602055787336\\
2	0.7	0.000224275485258861	0.000224275485258861\\
2	0.8	0.000144336086030812	0.000144336086030812\\
2	0.9	4.14091088485764e-05	4.14091088485764e-05\\
2	1	1.79561298314651e-05	1.79561298314651e-05\\
2	1.1	5.49844455373983e-05	5.49844455373983e-05\\
2	1.2	0.00018348276987924	0.00018348276987924\\
2	1.3	0.00023524892686098	0.00023524892686098\\
2	1.4	0.000244223004976096	0.000244223004976096\\
2	1.5	0.000254454942932759	0.000254454942932759\\
2	1.6	0.000276963372550257	0.000276963372550257\\
2	1.7	0.000334861329367894	0.000334861329367894\\
2	1.8	0.000385508368491239	0.000385508368491239\\
2	1.9	0.000362506348256548	0.000362506348256548\\
2	2	0.000368153190581381	0.000368153190581381\\
2	2.1	0.000414335895859798	0.000414335895859798\\
2	2.2	0.000414273928075794	0.000414273928075794\\
2	2.3	0.000333757116631672	0.000333757116631672\\
2	2.4	0.000205463304852904	0.000205463304852904\\
2	2.5	0.000110262633746912	0.000110262633746912\\
2	2.6	6.47393207721138e-05	6.47393207721138e-05\\
2	2.7	3.90262053597829e-05	3.90262053597829e-05\\
2	2.8	2.14355234211155e-05	2.14355234211155e-05\\
2	2.9	1.37231046138267e-05	1.37231046138267e-05\\
2	3	5.2285106050336e-05	5.2285106050336e-05\\
2	3.1	6.70928943165126e-05	6.70928943165126e-05\\
2	3.2	0.000116451540693307	0.000116451540693307\\
2	3.3	0.000252585079975065	0.000252585079975065\\
2	3.4	0.000620506511191726	0.000620506511191726\\
2	3.5	0.00138030258428824	0.00138030258428824\\
2	3.6	0.00206455572336607	0.00206455572336607\\
2	3.7	0.00177691996833027	0.00177691996833027\\
2	3.8	0.00101836279275152	0.00101836279275152\\
2	3.9	0.000503004586639539	0.000503004586639539\\
2	4	0.000239251486441981	0.000239251486441981\\
2	4.1	0.00011787860975964	0.00011787860975964\\
2	4.2	6.67104640593825e-05	6.67104640593825e-05\\
2	4.3	4.62188220085317e-05	4.62188220085317e-05\\
2	4.4	3.79363856723613e-05	3.79363856723613e-05\\
2	4.5	3.38270012724615e-05	3.38270012724615e-05\\
2	4.6	3.11349555925262e-05	3.11349555925262e-05\\
2	4.7	2.78225794428239e-05	2.78225794428239e-05\\
2	4.8	1.76424771678379e-05	1.76424771678379e-05\\
2	4.9	5.70012939035784e-06	5.70012939035784e-06\\
2	5	2.52892795781546e-05	2.52892795781546e-05\\
2	5.1	4.067990840582e-06	4.067990840582e-06\\
2	5.2	1.18924758773212e-05	1.18924758773212e-05\\
2	5.3	2.89608320008754e-05	2.89608320008754e-05\\
2	5.4	5.28317007769621e-05	5.28317007769621e-05\\
2	5.5	7.23256640601387e-05	7.23256640601387e-05\\
2	5.6	8.58741232308542e-05	8.58741232308542e-05\\
2	5.7	0.000105506936338988	0.000105506936338988\\
2	5.8	0.000147833792225149	0.000147833792225149\\
2	5.9	0.000241621150940102	0.000241621150940102\\
2	6	0.000449942336854143	0.000449942336854143\\
2.1	0	0.000387105036603228	0.000387105036603228\\
2.1	0.1	0.000233619261719438	0.000233619261719438\\
2.1	0.2	0.000151314476487563	0.000151314476487563\\
2.1	0.3	0.000104921146109376	0.000104921146109376\\
2.1	0.4	7.91063168039077e-05	7.91063168039077e-05\\
2.1	0.5	6.57686105006675e-05	6.57686105006675e-05\\
2.1	0.6	5.5527951514065e-05	5.5527951514065e-05\\
2.1	0.7	3.41381761671613e-05	3.41381761671613e-05\\
2.1	0.8	1.64058186379591e-05	1.64058186379591e-05\\
2.1	0.9	1.39950303869044e-05	1.39950303869044e-05\\
2.1	1	1.73978772933361e-05	1.73978772933361e-05\\
2.1	1.1	2.47317403309912e-05	2.47317403309912e-05\\
2.1	1.2	5.05991342914328e-05	5.05991342914328e-05\\
2.1	1.3	0.000154252283953413	0.000154252283953413\\
2.1	1.4	0.000276442614279307	0.000276442614279307\\
2.1	1.5	0.000363203799201238	0.000363203799201238\\
2.1	1.6	0.000412024969734968	0.000412024969734968\\
2.1	1.7	0.000425428023496293	0.000425428023496293\\
2.1	1.8	0.000382357456655084	0.000382357456655084\\
2.1	1.9	0.000337610450770787	0.000337610450770787\\
2.1	2	0.000361541876114148	0.000361541876114148\\
2.1	2.1	0.000441691024943037	0.000441691024943037\\
2.1	2.2	0.000493957788080404	0.000493957788080404\\
2.1	2.3	0.000440601221363959	0.000440601221363959\\
2.1	2.4	0.000301923001011103	0.000301923001011103\\
2.1	2.5	0.000175722240187418	0.000175722240187418\\
2.1	2.6	0.000105386809057099	0.000105386809057099\\
2.1	2.7	6.77645252138728e-05	6.77645252138728e-05\\
2.1	2.8	4.33670721349332e-05	4.33670721349332e-05\\
2.1	2.9	3.03423886030751e-05	3.03423886030751e-05\\
2.1	3	9.10894346393888e-05	9.10894346393888e-05\\
2.1	3.1	0.000111775310877481	0.000111775310877481\\
2.1	3.2	0.000179807948118735	0.000179807948118735\\
2.1	3.3	0.000352177138252122	0.000352177138252122\\
2.1	3.4	0.000750642693709307	0.000750642693709307\\
2.1	3.5	0.00142030921597161	0.00142030921597161\\
2.1	3.6	0.00190071609313909	0.00190071609313909\\
2.1	3.7	0.00162908019474743	0.00162908019474743\\
2.1	3.8	0.00101045202286823	0.00101045202286823\\
2.1	3.9	0.000562103294877509	0.000562103294877509\\
2.1	4	0.000305492365692917	0.000305492365692917\\
2.1	4.1	0.000167741035705469	0.000167741035705469\\
2.1	4.2	0.000102324975344095	0.000102324975344095\\
2.1	4.3	7.71687002973166e-05	7.71687002973166e-05\\
2.1	4.4	7.32284938230198e-05	7.32284938230198e-05\\
2.1	4.5	7.63715415962785e-05	7.63715415962785e-05\\
2.1	4.6	6.72502863724976e-05	6.72502863724976e-05\\
2.1	4.7	3.53978280104487e-05	3.53978280104487e-05\\
2.1	4.8	1.18743079759084e-05	1.18743079759084e-05\\
2.1	4.9	1.21290174507527e-05	1.21290174507527e-05\\
2.1	5	2.42009188229504e-05	2.42009188229504e-05\\
2.1	5.1	9.12863697710726e-06	9.12863697710726e-06\\
2.1	5.2	5.49084635743498e-06	5.49084635743498e-06\\
2.1	5.3	1.56837645254219e-05	1.56837645254219e-05\\
2.1	5.4	3.48717988538552e-05	3.48717988538552e-05\\
2.1	5.5	4.52636974796906e-05	4.52636974796906e-05\\
2.1	5.6	5.09521250823604e-05	5.09521250823604e-05\\
2.1	5.7	6.84826526198513e-05	6.84826526198513e-05\\
2.1	5.8	0.000120759612706069	0.000120759612706069\\
2.1	5.9	0.000263521517459591	0.000263521517459591\\
2.1	6	0.000638807899046931	0.000638807899046931\\
2.2	0	0.000384517319308305	0.000384517319308305\\
2.2	0.1	0.000244150352068947	0.000244150352068947\\
2.2	0.2	0.000161375288826568	0.000161375288826568\\
2.2	0.3	0.000108438340674331	0.000108438340674331\\
2.2	0.4	7.44906322057641e-05	7.44906322057641e-05\\
2.2	0.5	5.37987611121075e-05	5.37987611121075e-05\\
2.2	0.6	3.91011136681822e-05	3.91011136681822e-05\\
2.2	0.7	2.15526557407741e-05	2.15526557407741e-05\\
2.2	0.8	1.19158143694923e-05	1.19158143694923e-05\\
2.2	0.9	1.24057659789075e-05	1.24057659789075e-05\\
2.2	1	1.72016355032043e-05	1.72016355032043e-05\\
2.2	1.1	2.3178549862413e-05	2.3178549862413e-05\\
2.2	1.2	4.43161395030768e-05	4.43161395030768e-05\\
2.2	1.3	9.21251938099349e-05	9.21251938099349e-05\\
2.2	1.4	0.000148713217791629	0.000148713217791629\\
2.2	1.5	0.000225575576819682	0.000225575576819682\\
2.2	1.6	0.000337743268007905	0.000337743268007905\\
2.2	1.7	0.000410451515023433	0.000410451515023433\\
2.2	1.8	0.000386674837484752	0.000386674837484752\\
2.2	1.9	0.000347541673432954	0.000347541673432954\\
2.2	2	0.000381228537277133	0.000381228537277133\\
2.2	2.1	0.000512791983942599	0.000512791983942599\\
2.2	2.2	0.000656701523687148	0.000656701523687148\\
2.2	2.3	0.000655727705907111	0.000655727705907111\\
2.2	2.4	0.000501222765336888	0.000501222765336888\\
2.2	2.5	0.000319669110834381	0.000319669110834381\\
2.2	2.6	0.00019674607625788	0.00019674607625788\\
2.2	2.7	0.000129424775357882	0.000129424775357882\\
2.2	2.8	9.09128276076721e-05	9.09128276076721e-05\\
2.2	2.9	6.94664185945859e-05	6.94664185945859e-05\\
2.2	3	0.000159186906994669	0.000159186906994669\\
2.2	3.1	0.000185070415577773	0.000185070415577773\\
2.2	3.2	0.00027028026667284	0.00027028026667284\\
2.2	3.3	0.000464719786076656	0.000464719786076656\\
2.2	3.4	0.000840922773816805	0.000840922773816805\\
2.2	3.5	0.00135234993892303	0.00135234993892303\\
2.2	3.6	0.00164635772265109	0.00164635772265109\\
2.2	3.7	0.00143646653135202	0.00143646653135202\\
2.2	3.8	0.000978193726476551	0.000978193726476551\\
2.2	3.9	0.000598549105648582	0.000598549105648582\\
2.2	4	0.000350528920137665	0.000350528920137665\\
2.2	4.1	0.000204733464826978	0.000204733464826978\\
2.2	4.2	0.000133405807876033	0.000133405807876033\\
2.2	4.3	0.000108509468725693	0.000108509468725693\\
2.2	4.4	0.00011328285371782	0.00011328285371782\\
2.2	4.5	0.000136942277621264	0.000136942277621264\\
2.2	4.6	0.000138250874087066	0.000138250874087066\\
2.2	4.7	7.33492665661993e-05	7.33492665661993e-05\\
2.2	4.8	2.97998112522942e-05	2.97998112522942e-05\\
2.2	4.9	2.21603778080972e-05	2.21603778080972e-05\\
2.2	5	2.47410193988738e-05	2.47410193988738e-05\\
2.2	5.1	1.68868864872999e-05	1.68868864872999e-05\\
2.2	5.2	1.26329565024774e-05	1.26329565024774e-05\\
2.2	5.3	1.69374546824274e-05	1.69374546824274e-05\\
2.2	5.4	2.32021133113739e-05	2.32021133113739e-05\\
2.2	5.5	2.37704497226329e-05	2.37704497226329e-05\\
2.2	5.6	2.59190719066877e-05	2.59190719066877e-05\\
2.2	5.7	3.84161603926626e-05	3.84161603926626e-05\\
2.2	5.8	7.85221823997046e-05	7.85221823997046e-05\\
2.2	5.9	0.000197792384432664	0.000197792384432664\\
2.2	6	0.0005362557479548	0.0005362557479548\\
2.3	0	0.000396710898662151	0.000396710898662151\\
2.3	0.1	0.00027242038441697	0.00027242038441697\\
2.3	0.2	0.000198901625558749	0.000198901625558749\\
2.3	0.3	0.000148582893501411	0.000148582893501411\\
2.3	0.4	0.000112004870473458	0.000112004870473458\\
2.3	0.5	8.75680928805738e-05	8.75680928805738e-05\\
2.3	0.6	6.85266336108649e-05	6.85266336108649e-05\\
2.3	0.7	3.961979541035e-05	3.961979541035e-05\\
2.3	0.8	1.66539011840728e-05	1.66539011840728e-05\\
2.3	0.9	1.30994718377594e-05	1.30994718377594e-05\\
2.3	1	1.73430496300576e-05	1.73430496300576e-05\\
2.3	1.1	2.57495619291428e-05	2.57495619291428e-05\\
2.3	1.2	4.704591152563e-05	4.704591152563e-05\\
2.3	1.3	6.70929897907155e-05	6.70929897907155e-05\\
2.3	1.4	9.78573182235551e-05	9.78573182235551e-05\\
2.3	1.5	0.000191071255640597	0.000191071255640597\\
2.3	1.6	0.000345001630939155	0.000345001630939155\\
2.3	1.7	0.000444876956362357	0.000444876956362357\\
2.3	1.8	0.000439214461945455	0.000439214461945455\\
2.3	1.9	0.000426141390675063	0.000426141390675063\\
2.3	2	0.000522861363812236	0.000522861363812236\\
2.3	2.1	0.000824328931259947	0.000824328931259947\\
2.3	2.2	0.00126883164259711	0.00126883164259711\\
2.3	2.3	0.00144762750620439	0.00144762750620439\\
2.3	2.4	0.00113397938640917	0.00113397938640917\\
2.3	2.5	0.00068398076942656	0.00068398076942656\\
2.3	2.6	0.000388133191148697	0.000388133191148697\\
2.3	2.7	0.000242643614908174	0.000242643614908174\\
2.3	2.8	0.000174182321268354	0.000174182321268354\\
2.3	2.9	0.000140502953304366	0.000140502953304366\\
2.3	3	0.000254831966517516	0.000254831966517516\\
2.3	3.1	0.000284901953306752	0.000284901953306752\\
2.3	3.2	0.000380655987911846	0.000380655987911846\\
2.3	3.3	0.000572554667738319	0.000572554667738319\\
2.3	3.4	0.000877462147152075	0.000877462147152075\\
2.3	3.5	0.00121236302338254	0.00121236302338254\\
2.3	3.6	0.00136901527171304	0.00136901527171304\\
2.3	3.7	0.00122734602847645	0.00122734602847645\\
2.3	3.8	0.000908616725291706	0.000908616725291706\\
2.3	3.9	0.000596737975023934	0.000596737975023934\\
2.3	4	0.000370544641193457	0.000370544641193457\\
2.3	4.1	0.000234177950263284	0.000234177950263284\\
2.3	4.2	0.000166719202915312	0.000166719202915312\\
2.3	4.3	0.000142818213443892	0.000142818213443892\\
2.3	4.4	0.000147238944667891	0.000147238944667891\\
2.3	4.5	0.000175792182141937	0.000175792182141937\\
2.3	4.6	0.00021608675428602	0.00021608675428602\\
2.3	4.7	0.000192382579771297	0.000192382579771297\\
2.3	4.8	9.66189473858526e-05	9.66189473858526e-05\\
2.3	4.9	4.0315227307631e-05	4.0315227307631e-05\\
2.3	5	2.72456349933241e-05	2.72456349933241e-05\\
2.3	5.1	2.81997710403493e-05	2.81997710403493e-05\\
2.3	5.2	3.18414379322215e-05	3.18414379322215e-05\\
2.3	5.3	2.94912333137049e-05	2.94912333137049e-05\\
2.3	5.4	2.65191793872085e-05	2.65191793872085e-05\\
2.3	5.5	2.31513435924432e-05	2.31513435924432e-05\\
2.3	5.6	2.29404435050593e-05	2.29404435050593e-05\\
2.3	5.7	2.94140949649402e-05	2.94140949649402e-05\\
2.3	5.8	5.02351427686287e-05	5.02351427686287e-05\\
2.3	5.9	0.000107393483074853	0.000107393483074853\\
2.3	6	0.000260286585258449	0.000260286585258449\\
2.4	0	0.000425618247191064	0.000425618247191064\\
2.4	0.1	0.000284717315080242	0.000284717315080242\\
2.4	0.2	0.000206181901829766	0.000206181901829766\\
2.4	0.3	0.00015620398703266	0.00015620398703266\\
2.4	0.4	0.000119806671936107	0.000119806671936107\\
2.4	0.5	9.34408851923571e-05	9.34408851923571e-05\\
2.4	0.6	7.69587642583801e-05	7.69587642583801e-05\\
2.4	0.7	6.77598529458415e-05	6.77598529458415e-05\\
2.4	0.8	5.37656401569176e-05	5.37656401569176e-05\\
2.4	0.9	2.45041371184043e-05	2.45041371184043e-05\\
2.4	1	1.79666416609843e-05	1.79666416609843e-05\\
2.4	1.1	4.53609280618604e-05	4.53609280618604e-05\\
2.4	1.2	0.000114892473512155	0.000114892473512155\\
2.4	1.3	0.000141504857397742	0.000141504857397742\\
2.4	1.4	0.000173782317118294	0.000173782317118294\\
2.4	1.5	0.000256278637645862	0.000256278637645862\\
2.4	1.6	0.000367030205625885	0.000367030205625885\\
2.4	1.7	0.000437440270072351	0.000437440270072351\\
2.4	1.8	0.000464670360102485	0.000464670360102485\\
2.4	1.9	0.000533127304950608	0.000533127304950608\\
2.4	2	0.000804926185453186	0.000804926185453186\\
2.4	2.1	0.0016321183650275	0.0016321183650275\\
2.4	2.2	0.00331263020041219	0.00331263020041219\\
2.4	2.3	0.00450412323976255	0.00450412323976255\\
2.4	2.4	0.00340442092709274	0.00340442092709274\\
2.4	2.5	0.00170153418693247	0.00170153418693247\\
2.4	2.6	0.000794075321327307	0.000794075321327307\\
2.4	2.7	0.000437172758629684	0.000437172758629684\\
2.4	2.8	0.000298458644634436	0.000298458644634436\\
2.4	2.9	0.000240716976293567	0.000240716976293567\\
2.4	3	0.000356145518007939	0.000356145518007939\\
2.4	3.1	0.000389920224860027	0.000389920224860027\\
2.4	3.2	0.000488389675906814	0.000488389675906814\\
2.4	3.3	0.000661049923622502	0.000661049923622502\\
2.4	3.4	0.00089049117912759	0.00089049117912759\\
2.4	3.5	0.00110061309947619	0.00110061309947619\\
2.4	3.6	0.00118290198431997	0.00118290198431997\\
2.4	3.7	0.00109070103027372	0.00109070103027372\\
2.4	3.8	0.000873410045054723	0.000873410045054723\\
2.4	3.9	0.000625450086017667	0.000625450086017667\\
2.4	4	0.000424226547803628	0.000424226547803628\\
2.4	4.1	0.000295394268408298	0.000295394268408298\\
2.4	4.2	0.000226179722260344	0.000226179722260344\\
2.4	4.3	0.00019548821789011	0.00019548821789011\\
2.4	4.4	0.000191668349707258	0.000191668349707258\\
2.4	4.5	0.000218074107735112	0.000218074107735112\\
2.4	4.6	0.000288911512434042	0.000288911512434042\\
2.4	4.7	0.00037871438521609	0.00037871438521609\\
2.4	4.8	0.000335341413111092	0.000335341413111092\\
2.4	4.9	0.000138640291941943	0.000138640291941943\\
2.4	5	3.24115531379772e-05	3.24115531379772e-05\\
2.4	5.1	6.95352053674868e-05	6.95352053674868e-05\\
2.4	5.2	8.42036417236091e-05	8.42036417236091e-05\\
2.4	5.3	7.15578188654836e-05	7.15578188654836e-05\\
2.4	5.4	5.22686368055055e-05	5.22686368055055e-05\\
2.4	5.5	3.69776143252649e-05	3.69776143252649e-05\\
2.4	5.6	2.8818476383036e-05	2.8818476383036e-05\\
2.4	5.7	2.80568540383011e-05	2.80568540383011e-05\\
2.4	5.8	3.59194821164399e-05	3.59194821164399e-05\\
2.4	5.9	5.9249195094289e-05	5.9249195094289e-05\\
2.4	6	0.000118889600348125	0.000118889600348125\\
2.5	0	0.000415333649279308	0.000415333649279308\\
2.5	0.1	0.000247200772398359	0.000247200772398359\\
2.5	0.2	0.000160411806006657	0.000160411806006657\\
2.5	0.3	0.000110660919067869	0.000110660919067869\\
2.5	0.4	7.93169048349084e-05	7.93169048349084e-05\\
2.5	0.5	6.09499277308549e-05	6.09499277308549e-05\\
2.5	0.6	5.54902516537747e-05	5.54902516537747e-05\\
2.5	0.7	6.75031032078964e-05	6.75031032078964e-05\\
2.5	0.8	0.000113337436068445	0.000113337436068445\\
2.5	0.9	0.000223827133589046	0.000223827133589046\\
2.5	1	0.000391541280868445	0.000391541280868445\\
2.5	1.1	0.000491566733931549	0.000491566733931549\\
2.5	1.2	0.000442880053019104	0.000442880053019104\\
2.5	1.3	0.000342326357799724	0.000342326357799724\\
2.5	1.4	0.000276809836899618	0.000276809836899618\\
2.5	1.5	0.000260397254976885	0.000260397254976885\\
2.5	1.6	0.000280493939323387	0.000280493939323387\\
2.5	1.7	0.000321614712184582	0.000321614712184582\\
2.5	1.8	0.000389402214449955	0.000389402214449955\\
2.5	1.9	0.000547454737294155	0.000547454737294155\\
2.5	2	0.00105112798331219	0.00105112798331219\\
2.5	2.1	0.00284234606954532	0.00284234606954532\\
2.5	2.2	0.00775208407343179	0.00775208407343179\\
2.5	2.3	0.012679627355448	0.012679627355448\\
2.5	2.4	0.00943031369546302	0.00943031369546302\\
2.5	2.5	0.00394874634812553	0.00394874634812553\\
2.5	2.6	0.00149975306781488	0.00149975306781488\\
2.5	2.7	0.000711693812606981	0.000711693812606981\\
2.5	2.8	0.000448144038651296	0.000448144038651296\\
2.5	2.9	0.000352411270537257	0.000352411270537257\\
2.5	3	0.00043916312316138	0.00043916312316138\\
2.5	3.1	0.000478415607996142	0.000478415607996142\\
2.5	3.2	0.000577515389120999	0.000577515389120999\\
2.5	3.3	0.000733064881032495	0.000733064881032495\\
2.5	3.4	0.000915980440080485	0.000915980440080485\\
2.5	3.5	0.00106236522465617	0.00106236522465617\\
2.5	3.6	0.00110593011010601	0.00110593011010601\\
2.5	3.7	0.00103296976643303	0.00103296976643303\\
2.5	3.8	0.000877850935924816	0.000877850935924816\\
2.5	3.9	0.000689735795333888	0.000689735795333888\\
2.5	4	0.000520093020074644	0.000520093020074644\\
2.5	4.1	0.00039609007121559	0.00039609007121559\\
2.5	4.2	0.000312387537929859	0.000312387537929859\\
2.5	4.3	0.000254195872533942	0.000254195872533942\\
2.5	4.4	0.000214278833123783	0.000214278833123783\\
2.5	4.5	0.000192924286069392	0.000192924286069392\\
2.5	4.6	0.000191955971341759	0.000191955971341759\\
2.5	4.7	0.000208962188837469	0.000208962188837469\\
2.5	4.8	0.000230881672049487	0.000230881672049487\\
2.5	4.9	0.000234146479798444	0.000234146479798444\\
2.5	5	0.000203960235539324	0.000203960235539324\\
2.5	5.1	0.000151241756817453	0.000151241756817453\\
2.5	5.2	9.95376363273958e-05	9.95376363273958e-05\\
2.5	5.3	6.31517716615679e-05	6.31517716615679e-05\\
2.5	5.4	4.21600205111441e-05	4.21600205111441e-05\\
2.5	5.5	3.10634492782717e-05	3.10634492782717e-05\\
2.5	5.6	2.59590341564599e-05	2.59590341564599e-05\\
2.5	5.7	2.561864511877e-05	2.561864511877e-05\\
2.5	5.8	3.10448598325587e-05	3.10448598325587e-05\\
2.5	5.9	4.68149090237125e-05	4.68149090237125e-05\\
2.5	6	8.66407955427293e-05	8.66407955427293e-05\\
2.6	0	0.000407381583526313	0.000407381583526313\\
2.6	0.1	0.000239075296726282	0.000239075296726282\\
2.6	0.2	0.000158793796934312	0.000158793796934312\\
2.6	0.3	0.000116609843602949	0.000116609843602949\\
2.6	0.4	9.32842834228085e-05	9.32842834228085e-05\\
2.6	0.5	8.21031081783596e-05	8.21031081783596e-05\\
2.6	0.6	7.91649652814742e-05	7.91649652814742e-05\\
2.6	0.7	7.75909088568233e-05	7.75909088568233e-05\\
2.6	0.8	6.28388122087177e-05	6.28388122087177e-05\\
2.6	0.9	2.87043619595593e-05	2.87043619595593e-05\\
2.6	1	4.82805572744697e-06	4.82805572744697e-06\\
2.6	1.1	2.80687681117848e-05	2.80687681117848e-05\\
2.6	1.2	9.92636956107559e-05	9.92636956107559e-05\\
2.6	1.3	0.000174542852257478	0.000174542852257478\\
2.6	1.4	0.000210089054802901	0.000210089054802901\\
2.6	1.5	0.000210844134069348	0.000210844134069348\\
2.6	1.6	0.000206033999540754	0.000206033999540754\\
2.6	1.7	0.000219964162219619	0.000219964162219619\\
2.6	1.8	0.000273896785968785	0.000273896785968785\\
2.6	1.9	0.000429794973718565	0.000429794973718565\\
2.6	2	0.00095920290342772	0.00095920290342772\\
2.6	2.1	0.00304451840378175	0.00304451840378175\\
2.6	2.2	0.00972411760560994	0.00972411760560994\\
2.6	2.3	0.0182027939498308	0.0182027939498308\\
2.6	2.4	0.0145757762355733	0.0145757762355733\\
2.6	2.5	0.0060486443347471	0.0060486443347471\\
2.6	2.6	0.00215280667098211	0.00215280667098211\\
2.6	2.7	0.000951287317510156	0.000951287317510156\\
2.6	2.8	0.000573125551195602	0.000573125551195602\\
2.6	2.9	0.00044951927418129	0.00044951927418129\\
2.6	3	0.000496139094065586	0.000496139094065586\\
2.6	3.1	0.00054335811519459	0.00054335811519459\\
2.6	3.2	0.000643515183450766	0.000643515183450766\\
2.6	3.3	0.000785564431189632	0.000785564431189632\\
2.6	3.4	0.00093593621876097	0.00093593621876097\\
2.6	3.5	0.00103793041260582	0.00103793041260582\\
2.6	3.6	0.00104557042006148	0.00104557042006148\\
2.6	3.7	0.000964327136394941	0.000964327136394941\\
2.6	3.8	0.000830630376771531	0.000830630376771531\\
2.6	3.9	0.000677094614305747	0.000677094614305747\\
2.6	4	0.0005328318923466	0.0005328318923466\\
2.6	4.1	0.000412959764830163	0.000412959764830163\\
2.6	4.2	0.000313067318353755	0.000313067318353755\\
2.6	4.3	0.000228294689183197	0.000228294689183197\\
2.6	4.4	0.000162805487533661	0.000162805487533661\\
2.6	4.5	0.000120539562601968	0.000120539562601968\\
2.6	4.6	9.91640932689786e-05	9.91640932689786e-05\\
2.6	4.7	8.94857733266402e-05	8.94857733266402e-05\\
2.6	4.8	6.64292699621658e-05	6.64292699621658e-05\\
2.6	4.9	2.93858306802002e-05	2.93858306802002e-05\\
2.6	5	2.66109231636903e-05	2.66109231636903e-05\\
2.6	5.1	1.50611373733383e-05	1.50611373733383e-05\\
2.6	5.2	2.56022937324085e-05	2.56022937324085e-05\\
2.6	5.3	2.32093320390429e-05	2.32093320390429e-05\\
2.6	5.4	1.70815227410575e-05	1.70815227410575e-05\\
2.6	5.5	1.47390818938093e-05	1.47390818938093e-05\\
2.6	5.6	1.55251389338854e-05	1.55251389338854e-05\\
2.6	5.7	1.92341369591218e-05	1.92341369591218e-05\\
2.6	5.8	2.74827528315679e-05	2.74827528315679e-05\\
2.6	5.9	4.56071526696406e-05	4.56071526696406e-05\\
2.6	6	8.9281942468505e-05	8.9281942468505e-05\\
2.7	0	0.000417171923181665	0.000417171923181665\\
2.7	0.1	0.000249422325457247	0.000249422325457247\\
2.7	0.2	0.000172791265464493	0.000172791265464493\\
2.7	0.3	0.000132847857336435	0.000132847857336435\\
2.7	0.4	0.000106199151030166	0.000106199151030166\\
2.7	0.5	8.34291797074333e-05	8.34291797074333e-05\\
2.7	0.6	6.46076687108994e-05	6.46076687108994e-05\\
2.7	0.7	4.9384325806416e-05	4.9384325806416e-05\\
2.7	0.8	2.54449591671506e-05	2.54449591671506e-05\\
2.7	0.9	6.55409833587874e-06	6.55409833587874e-06\\
2.7	1	4.61609998069493e-06	4.61609998069493e-06\\
2.7	1.1	5.38190135041487e-06	5.38190135041487e-06\\
2.7	1.2	1.73044082589709e-05	1.73044082589709e-05\\
2.7	1.3	5.29556066486442e-05	5.29556066486442e-05\\
2.7	1.4	0.000112942798003758	0.000112942798003758\\
2.7	1.5	0.000175222278812241	0.000175222278812241\\
2.7	1.6	0.000206068205717751	0.000206068205717751\\
2.7	1.7	0.00020687769385403	0.00020687769385403\\
2.7	1.8	0.000221292322264652	0.000221292322264652\\
2.7	1.9	0.000311348516875583	0.000311348516875583\\
2.7	2	0.000646607788250957	0.000646607788250957\\
2.7	2.1	0.00190013003100834	0.00190013003100834\\
2.7	2.2	0.00578762476940149	0.00578762476940149\\
2.7	2.3	0.0114030951220561	0.0114030951220561\\
2.7	2.4	0.0106923984782757	0.0106923984782757\\
2.7	2.5	0.00535108997784074	0.00535108997784074\\
2.7	2.6	0.00216993933387886	0.00216993933387886\\
2.7	2.7	0.00101510626796552	0.00101510626796552\\
2.7	2.8	0.000626964172341198	0.000626964172341198\\
2.7	2.9	0.000507980666824057	0.000507980666824057\\
2.7	3	0.000528367105905961	0.000528367105905961\\
2.7	3.1	0.000581945413828897	0.000581945413828897\\
2.7	3.2	0.000679385939932083	0.000679385939932083\\
2.7	3.3	0.000801884316267179	0.000801884316267179\\
2.7	3.4	0.000911798308366071	0.000911798308366071\\
2.7	3.5	0.000963100964743381	0.000963100964743381\\
2.7	3.6	0.000931517966376283	0.000931517966376283\\
2.7	3.7	0.000833903919110146	0.000833903919110146\\
2.7	3.8	0.000705690484521579	0.000705690484521579\\
2.7	3.9	0.00057347875216894	0.00057347875216894\\
2.7	4	0.000452132260748987	0.000452132260748987\\
2.7	4.1	0.000346234411477746	0.000346234411477746\\
2.7	4.2	0.000256055044205998	0.000256055044205998\\
2.7	4.3	0.000184868990403706	0.000184868990403706\\
2.7	4.4	0.000135217297974299	0.000135217297974299\\
2.7	4.5	0.000103075516307754	0.000103075516307754\\
2.7	4.6	7.92388392557647e-05	7.92388392557647e-05\\
2.7	4.7	5.65485994573562e-05	5.65485994573562e-05\\
2.7	4.8	3.92968146673095e-05	3.92968146673095e-05\\
2.7	4.9	2.87544517125703e-05	2.87544517125703e-05\\
2.7	5	2.47983361907469e-05	2.47983361907469e-05\\
2.7	5.1	1.44934485815215e-05	1.44934485815215e-05\\
2.7	5.2	9.15693733336637e-06	9.15693733336637e-06\\
2.7	5.3	9.01274272531044e-06	9.01274272531044e-06\\
2.7	5.4	9.8235204504728e-06	9.8235204504728e-06\\
2.7	5.5	1.0161310092895e-05	1.0161310092895e-05\\
2.7	5.6	1.18121240787767e-05	1.18121240787767e-05\\
2.7	5.7	1.64698912023991e-05	1.64698912023991e-05\\
2.7	5.8	2.66198163080457e-05	2.66198163080457e-05\\
2.7	5.9	4.8517301760401e-05	4.8517301760401e-05\\
2.7	6	0.000101037065085709	0.000101037065085709\\
2.8	0	0.000425467181476227	0.000425467181476227\\
2.8	0.1	0.00023751396315964	0.00023751396315964\\
2.8	0.2	0.000157091683735908	0.000157091683735908\\
2.8	0.3	0.000118886057435482	0.000118886057435482\\
2.8	0.4	9.87014993424773e-05	9.87014993424773e-05\\
2.8	0.5	8.63962034674535e-05	8.63962034674535e-05\\
2.8	0.6	7.15723538310005e-05	7.15723538310005e-05\\
2.8	0.7	4.0516988445218e-05	4.0516988445218e-05\\
2.8	0.8	1.32162721496573e-05	1.32162721496573e-05\\
2.8	0.9	5.3497029890465e-06	5.3497029890465e-06\\
2.8	1	4.68897530506812e-06	4.68897530506812e-06\\
2.8	1.1	4.81919384238206e-06	4.81919384238206e-06\\
2.8	1.2	7.03068038891732e-06	7.03068038891732e-06\\
2.8	1.3	1.6681739687863e-05	1.6681739687863e-05\\
2.8	1.4	4.28700192311684e-05	4.28700192311684e-05\\
2.8	1.5	9.71157013103842e-05	9.71157013103842e-05\\
2.8	1.6	0.000169917269788955	0.000169917269788955\\
2.8	1.7	0.000205743032477926	0.000205743032477926\\
2.8	1.8	0.000204213341176882	0.000204213341176882\\
2.8	1.9	0.000239470270413662	0.000239470270413662\\
2.8	2	0.000407532130039926	0.000407532130039926\\
2.8	2.1	0.000962701924717084	0.000962701924717084\\
2.8	2.2	0.00245860659382337	0.00245860659382337\\
2.8	2.3	0.00471721008812379	0.00471721008812379\\
2.8	2.4	0.00521323005891018	0.00521323005891018\\
2.8	2.5	0.00341085861623541	0.00341085861623541\\
2.8	2.6	0.00175006464103482	0.00175006464103482\\
2.8	2.7	0.000942233199281757	0.000942233199281757\\
2.8	2.8	0.000623061621384694	0.000623061621384694\\
2.8	2.9	0.000523326008399391	0.000523326008399391\\
2.8	3	0.000540788963686821	0.000540788963686821\\
2.8	3.1	0.000592273560190223	0.000592273560190223\\
2.8	3.2	0.00067899878875371	0.00067899878875371\\
2.8	3.3	0.000777131740897946	0.000777131740897946\\
2.8	3.4	0.0008482809909465	0.0008482809909465\\
2.8	3.5	0.000858345676636929	0.000858345676636929\\
2.8	3.6	0.000800061862537168	0.000800061862537168\\
2.8	3.7	0.000694560248182531	0.000694560248182531\\
2.8	3.8	0.000573302903575397	0.000573302903575397\\
2.8	3.9	0.000458731302073264	0.000458731302073264\\
2.8	4	0.000357897426947739	0.000357897426947739\\
2.8	4.1	0.000271695457936544	0.000271695457936544\\
2.8	4.2	0.000203672452576224	0.000203672452576224\\
2.8	4.3	0.000154969213402377	0.000154969213402377\\
2.8	4.4	0.000119752757338419	0.000119752757338419\\
2.8	4.5	9.14187504875339e-05	9.14187504875339e-05\\
2.8	4.6	6.83375147505916e-05	6.83375147505916e-05\\
2.8	4.7	5.03100148211093e-05	5.03100148211093e-05\\
2.8	4.8	3.59902058087685e-05	3.59902058087685e-05\\
2.8	4.9	2.86886595872178e-05	2.86886595872178e-05\\
2.8	5	2.43331901108662e-05	2.43331901108662e-05\\
2.8	5.1	1.54244426266867e-05	1.54244426266867e-05\\
2.8	5.2	9.04316592262396e-06	9.04316592262396e-06\\
2.8	5.3	6.42438892544966e-06	6.42438892544966e-06\\
2.8	5.4	6.30888577553925e-06	6.30888577553925e-06\\
2.8	5.5	8.02215071896721e-06	8.02215071896721e-06\\
2.8	5.6	1.1663978148924e-05	1.1663978148924e-05\\
2.8	5.7	1.86034579493723e-05	1.86034579493723e-05\\
2.8	5.8	3.22839533940989e-05	3.22839533940989e-05\\
2.8	5.9	6.14905559016636e-05	6.14905559016636e-05\\
2.8	6	0.00013231326019762	0.00013231326019762\\
2.9	0	0.000419425605038824	0.000419425605038824\\
2.9	0.1	0.000201612731358235	0.000201612731358235\\
2.9	0.2	0.000124921091943123	0.000124921091943123\\
2.9	0.3	9.98724350581362e-05	9.98724350581362e-05\\
2.9	0.4	9.76394606592804e-05	9.76394606592804e-05\\
2.9	0.5	9.79894761340847e-05	9.79894761340847e-05\\
2.9	0.6	7.27713658667717e-05	7.27713658667717e-05\\
2.9	0.7	3.17089825266909e-05	3.17089825266909e-05\\
2.9	0.8	1.0861384216415e-05	1.0861384216415e-05\\
2.9	0.9	5.50378531138304e-06	5.50378531138304e-06\\
2.9	1	5.00489215351102e-06	5.00489215351102e-06\\
2.9	1.1	5.33369564599163e-06	5.33369564599163e-06\\
2.9	1.2	6.11097740953552e-06	6.11097740953552e-06\\
2.9	1.3	9.14746005333047e-06	9.14746005333047e-06\\
2.9	1.4	1.77010358025348e-05	1.77010358025348e-05\\
2.9	1.5	3.96654395408372e-05	3.96654395408372e-05\\
2.9	1.6	8.69545084547773e-05	8.69545084547773e-05\\
2.9	1.7	0.000144663336295629	0.000144663336295629\\
2.9	1.8	0.000173282829396675	0.000173282829396675\\
2.9	1.9	0.000201594821963559	0.000201594821963559\\
2.9	2	0.00030341528715997	0.00030341528715997\\
2.9	2.1	0.000604516493984552	0.000604516493984552\\
2.9	2.2	0.00132280788184054	0.00132280788184054\\
2.9	2.3	0.00238029604814253	0.00238029604814253\\
2.9	2.4	0.00283847253126465	0.00283847253126465\\
2.9	2.5	0.00223177391653373	0.00223177391653373\\
2.9	2.6	0.00139270339833485	0.00139270339833485\\
2.9	2.7	0.00086252364205786	0.00086252364205786\\
2.9	2.8	0.000610598928834364	0.000610598928834364\\
2.9	2.9	0.000516569279644127	0.000516569279644127\\
2.9	3	0.000544286031076833	0.000544286031076833\\
2.9	3.1	0.000582138503600646	0.000582138503600646\\
2.9	3.2	0.000649814860872988	0.000649814860872988\\
2.9	3.3	0.000725520038133891	0.000725520038133891\\
2.9	3.4	0.000774275345772272	0.000774275345772272\\
2.9	3.5	0.000766713985837555	0.000766713985837555\\
2.9	3.6	0.000700833524423368	0.000700833524423368\\
2.9	3.7	0.000597492993623202	0.000597492993623202\\
2.9	3.8	0.000482861119033223	0.000482861119033223\\
2.9	3.9	0.000375899230564737	0.000375899230564737\\
2.9	4	0.000284427872508212	0.000284427872508212\\
2.9	4.1	0.000212260170648956	0.000212260170648956\\
2.9	4.2	0.000160941513292109	0.000160941513292109\\
2.9	4.3	0.000124998865632693	0.000124998865632693\\
2.9	4.4	9.65700047430898e-05	9.65700047430898e-05\\
2.9	4.5	7.33030205985035e-05	7.33030205985035e-05\\
2.9	4.6	5.51521826177067e-05	5.51521826177067e-05\\
2.9	4.7	4.09683103466056e-05	4.09683103466056e-05\\
2.9	4.8	3.23520220720666e-05	3.23520220720666e-05\\
2.9	4.9	2.92873427227856e-05	2.92873427227856e-05\\
2.9	5	2.48048046447895e-05	2.48048046447895e-05\\
2.9	5.1	1.63054351400439e-05	1.63054351400439e-05\\
2.9	5.2	9.42568464442605e-06	9.42568464442605e-06\\
2.9	5.3	6.15993245904646e-06	6.15993245904646e-06\\
2.9	5.4	5.50300161122921e-06	5.50300161122921e-06\\
2.9	5.5	7.11591424945611e-06	7.11591424945611e-06\\
2.9	5.6	1.200801951957e-05	1.200801951957e-05\\
2.9	5.7	2.27523532623595e-05	2.27523532623595e-05\\
2.9	5.8	4.47156922605857e-05	4.47156922605857e-05\\
2.9	5.9	9.16039911106527e-05	9.16039911106527e-05\\
2.9	6	0.000206313768372108	0.000206313768372108\\
3	0	0.000427122137818705	0.000427122137818705\\
3	0.1	0.000174507953520273	0.000174507953520273\\
3	0.2	9.97268745160351e-05	9.97268745160351e-05\\
3	0.3	8.21371112497624e-05	8.21371112497624e-05\\
3	0.4	8.88127039968443e-05	8.88127039968443e-05\\
3	0.5	9.45753069634926e-05	9.45753069634926e-05\\
3	0.6	6.83650006168825e-05	6.83650006168825e-05\\
3	0.7	2.9926210576513e-05	2.9926210576513e-05\\
3	0.8	1.11054855047874e-05	1.11054855047874e-05\\
3	0.9	6.08643699502037e-06	6.08643699502037e-06\\
3	1	5.61849033807895e-06	5.61849033807895e-06\\
3	1.1	6.21610143909971e-06	6.21610143909971e-06\\
3	1.2	6.94322590424932e-06	6.94322590424932e-06\\
3	1.3	8.40671860553471e-06	8.40671860553471e-06\\
3	1.4	1.1874650214761e-05	1.1874650214761e-05\\
3	1.5	2.11563432142312e-05	2.11563432142312e-05\\
3	1.6	4.52880798231389e-05	4.52880798231389e-05\\
3	1.7	8.98603295961037e-05	8.98603295961037e-05\\
3	1.8	0.00013776988188644	0.00013776988188644\\
3	1.9	0.000187159141945434	0.000187159141945434\\
3	2	0.000292075238693657	0.000292075238693657\\
3	2.1	0.000573387375592376	0.000573387375592376\\
3	2.2	0.00121875492923668	0.00121875492923668\\
3	2.3	0.00208590600942188	0.00208590600942188\\
3	2.4	0.0023751685871005	0.0023751685871005\\
3	2.5	0.00188239436114597	0.00188239436114597\\
3	2.6	0.0012514857185831	0.0012514857185831\\
3	2.7	0.000834765274034033	0.000834765274034033\\
3	2.8	0.000611835724889614	0.000611835724889614\\
3	2.9	0.000504530818675261	0.000504530818675261\\
3	3	0.000547917354717565	0.000547917354717565\\
3	3.1	0.000565994347306285	0.000565994347306285\\
3	3.2	0.000612046030416238	0.000612046030416238\\
3	3.3	0.000670701477270525	0.000670701477270525\\
3	3.4	0.00071129615488376	0.00071129615488376\\
3	3.5	0.000701914026315243	0.000701914026315243\\
3	3.6	0.000638275908948	0.000638275908948\\
3	3.7	0.000541126591338495	0.000541126591338495\\
3	3.8	0.000432441182870015	0.000432441182870015\\
3	3.9	0.000327749842471517	0.000327749842471517\\
3	4	0.00023893584727738	0.00023893584727738\\
3	4.1	0.000173830770130015	0.000173830770130015\\
3	4.2	0.000130209413566407	0.000130209413566407\\
3	4.3	9.83353375808525e-05	9.83353375808525e-05\\
3	4.4	7.22679063314822e-05	7.22679063314822e-05\\
3	4.5	5.27016039553395e-05	5.27016039553395e-05\\
3	4.6	3.97322867111731e-05	3.97322867111731e-05\\
3	4.7	3.23339369488988e-05	3.23339369488988e-05\\
3	4.8	3.02679499747963e-05	3.02679499747963e-05\\
3	4.9	3.01665035345284e-05	3.01665035345284e-05\\
3	5	2.58179709998088e-05	2.58179709998088e-05\\
3	5.1	1.72205096528221e-05	1.72205096528221e-05\\
3	5.2	9.91765503759004e-06	9.91765503759004e-06\\
3	5.3	6.12217635743536e-06	6.12217635743536e-06\\
3	5.4	5.15996679745118e-06	5.15996679745118e-06\\
3	5.5	6.64909102451514e-06	6.64909102451514e-06\\
3	5.6	1.20099240490179e-05	1.20099240490179e-05\\
3	5.7	2.56771587497336e-05	2.56771587497336e-05\\
3	5.8	5.79753597678858e-05	5.79753597678858e-05\\
3	5.9	0.000134657547827258	0.000134657547827258\\
3	6	0.000333367151539819	0.000333367151539819\\
3.1	0	0.000478425376326712	0.000478425376326712\\
3.1	0.1	0.00017807566130369	0.00017807566130369\\
3.1	0.2	9.13446024595854e-05	9.13446024595854e-05\\
3.1	0.3	6.94889799993275e-05	6.94889799993275e-05\\
3.1	0.4	7.33744478776183e-05	7.33744478776183e-05\\
3.1	0.5	8.10637432160467e-05	8.10637432160467e-05\\
3.1	0.6	6.48523611488467e-05	6.48523611488467e-05\\
3.1	0.7	3.15705513563901e-05	3.15705513563901e-05\\
3.1	0.8	1.24228979285796e-05	1.24228979285796e-05\\
3.1	0.9	7.07985816376322e-06	7.07985816376322e-06\\
3.1	1	6.65672139352671e-06	6.65672139352671e-06\\
3.1	1.1	7.55408833659043e-06	7.55408833659043e-06\\
3.1	1.2	8.58922314658469e-06	8.58922314658469e-06\\
3.1	1.3	9.89877812704429e-06	9.89877812704429e-06\\
3.1	1.4	1.20596224014774e-05	1.20596224014774e-05\\
3.1	1.5	1.78808603020735e-05	1.78808603020735e-05\\
3.1	1.6	3.50768554326602e-05	3.50768554326602e-05\\
3.1	1.7	7.4190354208431e-05	7.4190354208431e-05\\
3.1	1.8	0.000130855913326722	0.000130855913326722\\
3.1	1.9	0.00020068588486527	0.00020068588486527\\
3.1	2	0.000345551239490606	0.000345551239490606\\
3.1	2.1	0.000755657752496594	0.000755657752496594\\
3.1	2.2	0.00175358817505599	0.00175358817505599\\
3.1	2.3	0.00297744647909981	0.00297744647909981\\
3.1	2.4	0.00302911013820084	0.00302911013820084\\
3.1	2.5	0.00210565710533269	0.00210565710533269\\
3.1	2.6	0.00130006093480126	0.00130006093480126\\
3.1	2.7	0.000850670552042488	0.000850670552042488\\
3.1	2.8	0.000614392195619322	0.000614392195619322\\
3.1	2.9	0.000485193740841643	0.000485193740841643\\
3.1	3	0.000552305184030281	0.000552305184030281\\
3.1	3.1	0.000554104162087761	0.000554104162087761\\
3.1	3.2	0.000584572202087451	0.000584572202087451\\
3.1	3.3	0.000634353212266593	0.000634353212266593\\
3.1	3.4	0.000675096149620815	0.000675096149620815\\
3.1	3.5	0.00066801470076314	0.00066801470076314\\
3.1	3.6	0.0006037997498199	0.0006037997498199\\
3.1	3.7	0.000506649406013825	0.000506649406013825\\
3.1	3.8	0.000399281145788692	0.000399281145788692\\
3.1	3.9	0.000295276042266099	0.000295276042266099\\
3.1	4	0.000208557886381895	0.000208557886381895\\
3.1	4.1	0.000148110423142814	0.000148110423142814\\
3.1	4.2	0.000109025759200582	0.000109025759200582\\
3.1	4.3	7.97488799800529e-05	7.97488799800529e-05\\
3.1	4.4	5.53547143341748e-05	5.53547143341748e-05\\
3.1	4.5	3.8093882462488e-05	3.8093882462488e-05\\
3.1	4.6	2.86368260791786e-05	2.86368260791786e-05\\
3.1	4.7	2.57525564177259e-05	2.57525564177259e-05\\
3.1	4.8	2.80928339933545e-05	2.80928339933545e-05\\
3.1	4.9	3.07172497410215e-05	3.07172497410215e-05\\
3.1	5	2.68513140139944e-05	2.68513140139944e-05\\
3.1	5.1	1.78582348916754e-05	1.78582348916754e-05\\
3.1	5.2	1.01661940433724e-05	1.01661940433724e-05\\
3.1	5.3	6.10913427968953e-06	6.10913427968953e-06\\
3.1	5.4	4.93219880595024e-06	4.93219880595024e-06\\
3.1	5.5	6.15766268467246e-06	6.15766268467246e-06\\
3.1	5.6	1.1234239285962e-05	1.1234239285962e-05\\
3.1	5.7	2.57835499094044e-05	2.57835499094044e-05\\
3.1	5.8	6.61521341896512e-05	6.61521341896512e-05\\
3.1	5.9	0.000179543407175992	0.000179543407175992\\
3.1	6	0.000510978408784036	0.000510978408784036\\
3.2	0	0.000564781292095594	0.000564781292095594\\
3.2	0.1	0.000217856416244504	0.000217856416244504\\
3.2	0.2	0.000104172598538292	0.000104172598538292\\
3.2	0.3	6.73122281718109e-05	6.73122281718109e-05\\
3.2	0.4	5.96998391891932e-05	5.96998391891932e-05\\
3.2	0.5	6.23095938208024e-05	6.23095938208024e-05\\
3.2	0.6	5.67571686627984e-05	5.67571686627984e-05\\
3.2	0.7	3.35962595212329e-05	3.35962595212329e-05\\
3.2	0.8	1.45621794100936e-05	1.45621794100936e-05\\
3.2	0.9	8.58222900713167e-06	8.58222900713167e-06\\
3.2	1	8.28669072453135e-06	8.28669072453135e-06\\
3.2	1.1	9.52894744118173e-06	9.52894744118173e-06\\
3.2	1.2	1.10468784465128e-05	1.10468784465128e-05\\
3.2	1.3	1.29944041187364e-05	1.29944041187364e-05\\
3.2	1.4	1.5628758883822e-05	1.5628758883822e-05\\
3.2	1.5	2.24563690279111e-05	2.24563690279111e-05\\
3.2	1.6	4.30700019648898e-05	4.30700019648898e-05\\
3.2	1.7	9.00224034916116e-05	9.00224034916116e-05\\
3.2	1.8	0.000156684402744532	0.000156684402744532\\
3.2	1.9	0.000242458036426859	0.000242458036426859\\
3.2	2	0.000452167583945954	0.000452167583945954\\
3.2	2.1	0.00114451707478809	0.00114451707478809\\
3.2	2.2	0.00303081943856975	0.00303081943856975\\
3.2	2.3	0.00528015000981242	0.00528015000981242\\
3.2	2.4	0.00481350669775413	0.00481350669775413\\
3.2	2.5	0.00278272116798054	0.00278272116798054\\
3.2	2.6	0.00146099254644855	0.00146099254644855\\
3.2	2.7	0.000869829954251475	0.000869829954251475\\
3.2	2.8	0.000598642222968522	0.000598642222968522\\
3.2	2.9	0.000456652648927634	0.000456652648927634\\
3.2	3	0.000555063672388207	0.000555063672388207\\
3.2	3.1	0.00055106962413371	0.00055106962413371\\
3.2	3.2	0.000577571574880438	0.000577571574880438\\
3.2	3.3	0.000630085423449987	0.000630085423449987\\
3.2	3.4	0.000680378977945753	0.000680378977945753\\
3.2	3.5	0.000678446763083943	0.000678446763083943\\
3.2	3.6	0.000606717441459925	0.000606717441459925\\
3.2	3.7	0.000496756237463649	0.000496756237463649\\
3.2	3.8	0.000379964006485713	0.000379964006485713\\
3.2	3.9	0.000273257744225344	0.000273257744225344\\
3.2	4	0.000190155195691729	0.000190155195691729\\
3.2	4.1	0.000135473377815253	0.000135473377815253\\
3.2	4.2	0.000101164218694262	0.000101164218694262\\
3.2	4.3	7.45571922594714e-05	7.45571922594714e-05\\
3.2	4.4	5.01061478380017e-05	5.01061478380017e-05\\
3.2	4.5	3.17360538666917e-05	3.17360538666917e-05\\
3.2	4.6	2.20688743708558e-05	2.20688743708558e-05\\
3.2	4.7	2.00045797891986e-05	2.00045797891986e-05\\
3.2	4.8	2.43934725806793e-05	2.43934725806793e-05\\
3.2	4.9	2.99697439744999e-05	2.99697439744999e-05\\
3.2	5	2.72512963399567e-05	2.72512963399567e-05\\
3.2	5.1	1.77241210988101e-05	1.77241210988101e-05\\
3.2	5.2	9.77519550322492e-06	9.77519550322492e-06\\
3.2	5.3	5.91240179211607e-06	5.91240179211607e-06\\
3.2	5.4	4.86362124410332e-06	4.86362124410332e-06\\
3.2	5.5	6.06036858377871e-06	6.06036858377871e-06\\
3.2	5.6	1.10789319939463e-05	1.10789319939463e-05\\
3.2	5.7	2.64915157197058e-05	2.64915157197058e-05\\
3.2	5.8	7.47416699999303e-05	7.47416699999303e-05\\
3.2	5.9	0.000232193686268329	0.000232193686268329\\
3.2	6	0.000752957293973846	0.000752957293973846\\
3.3	0	0.000627049755407954	0.000627049755407954\\
3.3	0.1	0.000277945408530797	0.000277945408530797\\
3.3	0.2	0.000141051022346151	0.000141051022346151\\
3.3	0.3	8.32903750765999e-05	8.32903750765999e-05\\
3.3	0.4	5.81061912608469e-05	5.81061912608469e-05\\
3.3	0.5	4.82057020081348e-05	4.82057020081348e-05\\
3.3	0.6	4.41231624433208e-05	4.41231624433208e-05\\
3.3	0.7	3.37836104473358e-05	3.37836104473358e-05\\
3.3	0.8	1.76486183996727e-05	1.76486183996727e-05\\
3.3	0.9	1.06788733655501e-05	1.06788733655501e-05\\
3.3	1	1.06390763933299e-05	1.06390763933299e-05\\
3.3	1.1	1.22714002678574e-05	1.22714002678574e-05\\
3.3	1.2	1.47284169047432e-05	1.47284169047432e-05\\
3.3	1.3	1.85646635770774e-05	1.85646635770774e-05\\
3.3	1.4	2.43981708817306e-05	2.43981708817306e-05\\
3.3	1.5	3.99825893041968e-05	3.99825893041968e-05\\
3.3	1.6	7.97822739913415e-05	7.97822739913415e-05\\
3.3	1.7	0.000147021278964765	0.000147021278964765\\
3.3	1.8	0.000214323285548729	0.000214323285548729\\
3.3	1.9	0.000303146211245583	0.000303146211245583\\
3.3	2	0.000588125066114747	0.000588125066114747\\
3.3	2.1	0.00168861326774046	0.00168861326774046\\
3.3	2.2	0.00504534426205753	0.00504534426205753\\
3.3	2.3	0.00911855915693324	0.00911855915693324\\
3.3	2.4	0.00760862338577207	0.00760862338577207\\
3.3	2.5	0.0036445646617239	0.0036445646617239\\
3.3	2.6	0.00157881178540351	0.00157881178540351\\
3.3	2.7	0.000832622119037914	0.000832622119037914\\
3.3	2.8	0.0005480460839164	0.0005480460839164\\
3.3	2.9	0.000419166213184417	0.000419166213184417\\
3.3	3	0.00055091789521058	0.00055091789521058\\
3.3	3.1	0.000553405462803119	0.000553405462803119\\
3.3	3.2	0.000588031133301535	0.000588031133301535\\
3.3	3.3	0.000656439348604553	0.000656439348604553\\
3.3	3.4	0.000730761462941517	0.000730761462941517\\
3.3	3.5	0.000743528432883505	0.000743528432883505\\
3.3	3.6	0.000660150842044747	0.000660150842044747\\
3.3	3.7	0.000523876223266724	0.000523876223266724\\
3.3	3.8	0.00038679206922453	0.00038679206922453\\
3.3	3.9	0.000273450087852499	0.000273450087852499\\
3.3	4	0.000193210947738022	0.000193210947738022\\
3.3	4.1	0.00014370181152302	0.00014370181152302\\
3.3	4.2	0.000113435123624479	0.000113435123624479\\
3.3	4.3	8.81547760722505e-05	8.81547760722505e-05\\
3.3	4.4	5.96502846812796e-05	5.96502846812796e-05\\
3.3	4.5	3.40610970842601e-05	3.40610970842601e-05\\
3.3	4.6	1.95373956387181e-05	1.95373956387181e-05\\
3.3	4.7	1.5123641326836e-05	1.5123641326836e-05\\
3.3	4.8	1.86609767305843e-05	1.86609767305843e-05\\
3.3	4.9	2.68882240534518e-05	2.68882240534518e-05\\
3.3	5	2.64596442507105e-05	2.64596442507105e-05\\
3.3	5.1	1.63425218131477e-05	1.63425218131477e-05\\
3.3	5.2	8.55209562909386e-06	8.55209562909386e-06\\
3.3	5.3	5.55946483099635e-06	5.55946483099635e-06\\
3.3	5.4	5.24304752354764e-06	5.24304752354764e-06\\
3.3	5.5	7.39824103396885e-06	7.39824103396885e-06\\
3.3	5.6	1.43518644029229e-05	1.43518644029229e-05\\
3.3	5.7	3.47084691446545e-05	3.47084691446545e-05\\
3.3	5.8	9.85987987811869e-05	9.85987987811869e-05\\
3.3	5.9	0.000312200102933266	0.000312200102933266\\
3.3	6	0.00103320552386091	0.00103320552386091\\
3.4	0	0.000553230889149688	0.000553230889149688\\
3.4	0.1	0.00028473489622332	0.00028473489622332\\
3.4	0.2	0.000172664922564673	0.000172664922564673\\
3.4	0.3	0.000118023823804797	0.000118023823804797\\
3.4	0.4	8.35384001163278e-05	8.35384001163278e-05\\
3.4	0.5	5.81140933034981e-05	5.81140933034981e-05\\
3.4	0.6	4.16738968534229e-05	4.16738968534229e-05\\
3.4	0.7	3.28866927743349e-05	3.28866927743349e-05\\
3.4	0.8	2.24177173626662e-05	2.24177173626662e-05\\
3.4	0.9	1.34694077823125e-05	1.34694077823125e-05\\
3.4	1	1.36580631429187e-05	1.36580631429187e-05\\
3.4	1.1	1.5810014546977e-05	1.5810014546977e-05\\
3.4	1.2	2.13336714887541e-05	2.13336714887541e-05\\
3.4	1.3	3.03863089615617e-05	3.03863089615617e-05\\
3.4	1.4	5.03968629405348e-05	5.03968629405348e-05\\
3.4	1.5	0.000100658738732544	0.000100658738732544\\
3.4	1.6	0.00018472984541469	0.00018472984541469\\
3.4	1.7	0.00025262258332838	0.00025262258332838\\
3.4	1.8	0.000278342477344019	0.000278342477344019\\
3.4	1.9	0.000345958945497856	0.000345958945497856\\
3.4	2	0.000670478946509614	0.000670478946509614\\
3.4	2.1	0.00204390386167106	0.00204390386167106\\
3.4	2.2	0.00652747473464339	0.00652747473464339\\
3.4	2.3	0.012125072911657	0.012125072911657\\
3.4	2.4	0.00953276865343669	0.00953276865343669\\
3.4	2.5	0.00395371625957052	0.00395371625957052\\
3.4	2.6	0.00146445023486281	0.00146445023486281\\
3.4	2.7	0.000700514771955949	0.000700514771955949\\
3.4	2.8	0.000455044120253905	0.000455044120253905\\
3.4	2.9	0.000366231977798322	0.000366231977798322\\
3.4	3	0.000525580192573601	0.000525580192573601\\
3.4	3.1	0.00054343314757233	0.00054343314757233\\
3.4	3.2	0.000592342477612435	0.000592342477612435\\
3.4	3.3	0.000682115564242498	0.000682115564242498\\
3.4	3.4	0.000790846071025802	0.000790846071025802\\
3.4	3.5	0.000832513590412968	0.000832513590412968\\
3.4	3.6	0.000741954336170253	0.000741954336170253\\
3.4	3.7	0.000574647314117775	0.000574647314117775\\
3.4	3.8	0.000416048577706416	0.000416048577706416\\
3.4	3.9	0.000298595650200901	0.000298595650200901\\
3.4	4	0.000221963555512864	0.000221963555512864\\
3.4	4.1	0.000175793911106023	0.000175793911106023\\
3.4	4.2	0.000145691502802203	0.000145691502802203\\
3.4	4.3	0.000117950479866276	0.000117950479866276\\
3.4	4.4	8.46384175361958e-05	8.46384175361958e-05\\
3.4	4.5	4.91818994084103e-05	4.91818994084103e-05\\
3.4	4.6	2.36564562118085e-05	2.36564562118085e-05\\
3.4	4.7	1.27606537967875e-05	1.27606537967875e-05\\
3.4	4.8	1.23805858809774e-05	1.23805858809774e-05\\
3.4	4.9	2.08846025514412e-05	2.08846025514412e-05\\
3.4	5	2.43867779440323e-05	2.43867779440323e-05\\
3.4	5.1	1.34291091025118e-05	1.34291091025118e-05\\
3.4	5.2	6.88310754111196e-06	6.88310754111196e-06\\
3.4	5.3	5.7054772363409e-06	5.7054772363409e-06\\
3.4	5.4	7.74323597145224e-06	7.74323597145224e-06\\
3.4	5.5	1.38568623273218e-05	1.38568623273218e-05\\
3.4	5.6	2.74664494053218e-05	2.74664494053218e-05\\
3.4	5.7	5.98264366563342e-05	5.98264366563342e-05\\
3.4	5.8	0.000145891815640484	0.000145891815640484\\
3.4	5.9	0.000394355114842114	0.000394355114842114\\
3.4	6	0.00114001559566194	0.00114001559566194\\
3.5	0	0.000384236873521582	0.000384236873521582\\
3.5	0.1	0.000215932682420376	0.000215932682420376\\
3.5	0.2	0.000149220396738084	0.000149220396738084\\
3.5	0.3	0.000123948688250759	0.000123948688250759\\
3.5	0.4	0.000114276580473327	0.000114276580473327\\
3.5	0.5	0.000103560386923092	0.000103560386923092\\
3.5	0.6	8.0755031743669e-05	8.0755031743669e-05\\
3.5	0.7	5.03659979739691e-05	5.03659979739691e-05\\
3.5	0.8	3.14194002807189e-05	3.14194002807189e-05\\
3.5	0.9	1.86225507485473e-05	1.86225507485473e-05\\
3.5	1	1.69409385640233e-05	1.69409385640233e-05\\
3.5	1.1	2.16407325576761e-05	2.16407325576761e-05\\
3.5	1.2	3.73744903834527e-05	3.73744903834527e-05\\
3.5	1.3	7.09639391003626e-05	7.09639391003626e-05\\
3.5	1.4	0.000145319638535087	0.000145319638535087\\
3.5	1.5	0.000249853947049484	0.000249853947049484\\
3.5	1.6	0.000322774762784164	0.000322774762784164\\
3.5	1.7	0.000306558119748592	0.000306558119748592\\
3.5	1.8	0.000269919354682293	0.000269919354682293\\
3.5	1.9	0.000313572714728775	0.000313572714728775\\
3.5	2	0.000592548110652416	0.000592548110652416\\
3.5	2.1	0.00172782448776866	0.00172782448776866\\
3.5	2.2	0.00535295707165175	0.00535295707165175\\
3.5	2.3	0.00993378833996615	0.00993378833996615\\
3.5	2.4	0.00774478629527295	0.00774478629527295\\
3.5	2.5	0.00308878375756864	0.00308878375756864\\
3.5	2.6	0.001081473037038	0.001081473037038\\
3.5	2.7	0.000501812191029378	0.000501812191029378\\
3.5	2.8	0.000337975375112791	0.000337975375112791\\
3.5	2.9	0.000298034578955929	0.000298034578955929\\
3.5	3	0.000464633515415483	0.000464633515415483\\
3.5	3.1	0.000499768377466137	0.000499768377466137\\
3.5	3.2	0.000560824075801887	0.000560824075801887\\
3.5	3.3	0.000662723080739573	0.000662723080739573\\
3.5	3.4	0.000796272623961941	0.000796272623961941\\
3.5	3.5	0.000872904433278573	0.000872904433278573\\
3.5	3.6	0.000796005744322168	0.000796005744322168\\
3.5	3.7	0.000619159431409393	0.000619159431409393\\
3.5	3.8	0.000456435407596507	0.000456435407596507\\
3.5	3.9	0.00034575124539599	0.00034575124539599\\
3.5	4	0.000275940479001359	0.000275940479001359\\
3.5	4.1	0.000230052012832314	0.000230052012832314\\
3.5	4.2	0.000191923286567361	0.000191923286567361\\
3.5	4.3	0.000150978953631003	0.000150978953631003\\
3.5	4.4	0.000108600736121823	0.000108600736121823\\
3.5	4.5	7.11960899883042e-05	7.11960899883042e-05\\
3.5	4.6	4.0998836955344e-05	4.0998836955344e-05\\
3.5	4.7	1.88071490226802e-05	1.88071490226802e-05\\
3.5	4.8	9.16909490418523e-06	9.16909490418523e-06\\
3.5	4.9	1.23775565347159e-05	1.23775565347159e-05\\
3.5	5	2.15731663188451e-05	2.15731663188451e-05\\
3.5	5.1	9.04594338014088e-06	9.04594338014088e-06\\
3.5	5.2	6.25340585986462e-06	6.25340585986462e-06\\
3.5	5.3	1.07426623410224e-05	1.07426623410224e-05\\
3.5	5.4	2.17647000810827e-05	2.17647000810827e-05\\
3.5	5.5	3.6539507663945e-05	3.6539507663945e-05\\
3.5	5.6	5.74559209338444e-05	5.74559209338444e-05\\
3.5	5.7	9.56453234524026e-05	9.56453234524026e-05\\
3.5	5.8	0.00017775187271694	0.00017775187271694\\
3.5	5.9	0.000373160889250951	0.000373160889250951\\
3.5	6	0.000875096183779062	0.000875096183779062\\
3.6	0	0.000280633011688301	0.000280633011688301\\
3.6	0.1	0.000166893997500932	0.000166893997500932\\
3.6	0.2	0.00011876712096725	0.00011876712096725\\
3.6	0.3	0.000101076699060851	0.000101076699060851\\
3.6	0.4	0.000100887470281821	0.000100887470281821\\
3.6	0.5	0.000114376144993353	0.000114376144993353\\
3.6	0.6	0.000139806463776284	0.000139806463776284\\
3.6	0.7	0.000157392104064768	0.000157392104064768\\
3.6	0.8	0.000109830606002055	0.000109830606002055\\
3.6	0.9	4.30645639482142e-05	4.30645639482142e-05\\
3.6	1	1.97788336172726e-05	1.97788336172726e-05\\
3.6	1.1	4.95471273126051e-05	4.95471273126051e-05\\
3.6	1.2	0.000124752830714187	0.000124752830714187\\
3.6	1.3	0.000193629817575278	0.000193629817575278\\
3.6	1.4	0.000245446497133104	0.000245446497133104\\
3.6	1.5	0.000279973488878942	0.000279973488878942\\
3.6	1.6	0.000265380035092594	0.000265380035092594\\
3.6	1.7	0.000214933933504257	0.000214933933504257\\
3.6	1.8	0.000189819571654763	0.000189819571654763\\
3.6	1.9	0.000232899110248398	0.000232899110248398\\
3.6	2	0.000418673779126612	0.000418673779126612\\
3.6	2.1	0.00103290704086732	0.00103290704086732\\
3.6	2.2	0.00274937091857389	0.00274937091857389\\
3.6	2.3	0.00486111175662856	0.00486111175662856\\
3.6	2.4	0.00394946842493271	0.00394946842493271\\
3.6	2.5	0.00169802952722323	0.00169802952722323\\
3.6	2.6	0.000629464379429793	0.000629464379429793\\
3.6	2.7	0.00030846371728001	0.00030846371728001\\
3.6	2.8	0.000227573928462031	0.000227573928462031\\
3.6	2.9	0.000223865013300386	0.000223865013300386\\
3.6	3	0.00036948437393253	0.00036948437393253\\
3.6	3.1	0.000415830357645002	0.000415830357645002\\
3.6	3.2	0.000484471684798905	0.000484471684798905\\
3.6	3.3	0.000588485080545172	0.000588485080545172\\
3.6	3.4	0.000729466263572982	0.000729466263572982\\
3.6	3.5	0.000835374793244098	0.000835374793244098\\
3.6	3.6	0.000797063624378558	0.000797063624378558\\
3.6	3.7	0.000649068306962991	0.000649068306962991\\
3.6	3.8	0.000512568741547182	0.000512568741547182\\
3.6	3.9	0.000426244770142666	0.000426244770142666\\
3.6	4	0.000367209304640437	0.000367209304640437\\
3.6	4.1	0.000314093932982388	0.000314093932982388\\
3.6	4.2	0.000257861321690552	0.000257861321690552\\
3.6	4.3	0.000196780847120062	0.000196780847120062\\
3.6	4.4	0.000137670041734934	0.000137670041734934\\
3.6	4.5	9.02654522913079e-05	9.02654522913079e-05\\
3.6	4.6	5.88362223744731e-05	5.88362223744731e-05\\
3.6	4.7	4.02363138480124e-05	4.02363138480124e-05\\
3.6	4.8	2.4476217466497e-05	2.4476217466497e-05\\
3.6	4.9	8.15555888779258e-06	8.15555888779258e-06\\
3.6	5	1.89084979324583e-05	1.89084979324583e-05\\
3.6	5.1	7.33461625946563e-06	7.33461625946563e-06\\
3.6	5.2	2.45350544638433e-05	2.45350544638433e-05\\
3.6	5.3	4.18088934653448e-05	4.18088934653448e-05\\
3.6	5.4	4.98646170457405e-05	4.98646170457405e-05\\
3.6	5.5	5.75926218649558e-05	5.75926218649558e-05\\
3.6	5.6	7.08833254151855e-05	7.08833254151855e-05\\
3.6	5.7	9.62134834176193e-05	9.62134834176193e-05\\
3.6	5.8	0.000146751157433997	0.000146751157433997\\
3.6	5.9	0.000254280038077514	0.000254280038077514\\
3.6	6	0.00050103210471206	0.00050103210471206\\
3.7	0	0.000244697204166603	0.000244697204166603\\
3.7	0.1	0.000152513545496344	0.000152513545496344\\
3.7	0.2	0.00010863189804581	0.00010863189804581\\
3.7	0.3	8.89278675777505e-05	8.89278675777505e-05\\
3.7	0.4	8.33423821460949e-05	8.33423821460949e-05\\
3.7	0.5	8.82441776055567e-05	8.82441776055567e-05\\
3.7	0.6	0.000103421083647478	0.000103421083647478\\
3.7	0.7	0.000130205090046315	0.000130205090046315\\
3.7	0.8	0.000168322180694386	0.000168322180694386\\
3.7	0.9	0.000211110283716528	0.000211110283716528\\
3.7	1	0.000245024050075176	0.000245024050075176\\
3.7	1.1	0.000259246260788365	0.000259246260788365\\
3.7	1.2	0.000253877888869488	0.000253877888869488\\
3.7	1.3	0.000235053537290271	0.000235053537290271\\
3.7	1.4	0.000208250915207127	0.000208250915207127\\
3.7	1.5	0.000177994697106785	0.000177994697106785\\
3.7	1.6	0.000148918464558788	0.000148918464558788\\
3.7	1.7	0.000129415912584093	0.000129415912584093\\
3.7	1.8	0.000134119403739544	0.000134119403739544\\
3.7	1.9	0.000180000409273533	0.000180000409273533\\
3.7	2	0.000292591754266398	0.000292591754266398\\
3.7	2.1	0.000552959440040846	0.000552959440040846\\
3.7	2.2	0.00113984977029979	0.00113984977029979\\
3.7	2.3	0.00180384190405369	0.00180384190405369\\
3.7	2.4	0.00152947892036225	0.00152947892036225\\
3.7	2.5	0.000742089401104455	0.000742089401104455\\
3.7	2.6	0.000308397479267324	0.000308397479267324\\
3.7	2.7	0.000168723615766677	0.000168723615766677\\
3.7	2.8	0.000140017136798107	0.000140017136798107\\
3.7	2.9	0.000151170782198328	0.000151170782198328\\
3.7	3	0.000261517424390584	0.000261517424390584\\
3.7	3.1	0.000308776853414199	0.000308776853414199\\
3.7	3.2	0.000382070579567869	0.000382070579567869\\
3.7	3.3	0.000490572120163476	0.000490572120163476\\
3.7	3.4	0.000638151252764539	0.000638151252764539\\
3.7	3.5	0.000768856121635521	0.000768856121635521\\
3.7	3.6	0.000774240501399145	0.000774240501399145\\
3.7	3.7	0.000671399733902578	0.000671399733902578\\
3.7	3.8	0.000578684727103672	0.000578684727103672\\
3.7	3.9	0.000529162658568998	0.000529162658568998\\
3.7	4	0.000480502614133339	0.000480502614133339\\
3.7	4.1	0.000409606159028033	0.000409606159028033\\
3.7	4.2	0.000331131396296087	0.000331131396296087\\
3.7	4.3	0.000259510290246421	0.000259510290246421\\
3.7	4.4	0.000198939641578283	0.000198939641578283\\
3.7	4.5	0.000150624077468392	0.000150624077468392\\
3.7	4.6	0.000115036822262409	0.000115036822262409\\
3.7	4.7	9.08541586320828e-05	9.08541586320828e-05\\
3.7	4.8	7.53304340375233e-05	7.53304340375233e-05\\
3.7	4.9	6.55968082503043e-05	6.55968082503043e-05\\
3.7	5	5.94260894267541e-05	5.94260894267541e-05\\
3.7	5.1	5.54213530334968e-05	5.54213530334968e-05\\
3.7	5.2	5.29611039016802e-05	5.29611039016802e-05\\
3.7	5.3	5.20419336110312e-05	5.20419336110312e-05\\
3.7	5.4	5.31161215032819e-05	5.31161215032819e-05\\
3.7	5.5	5.70769781381265e-05	5.70769781381265e-05\\
3.7	5.6	6.55538718497222e-05	6.55538718497222e-05\\
3.7	5.7	8.17443237579734e-05	8.17443237579734e-05\\
3.7	5.8	0.000112377906661392	0.000112377906661392\\
3.7	5.9	0.000172493369400993	0.000172493369400993\\
3.7	6	0.000297665711527835	0.000297665711527835\\
3.8	0	0.000215047453662172	0.000215047453662172\\
3.8	0.1	0.00013827108743057	0.00013827108743057\\
3.8	0.2	9.86845295249629e-05	9.86845295249629e-05\\
3.8	0.3	7.86750868834297e-05	7.86750868834297e-05\\
3.8	0.4	6.95352213747935e-05	6.95352213747935e-05\\
3.8	0.5	6.62369036332601e-05	6.62369036332601e-05\\
3.8	0.6	6.42548051889746e-05	6.42548051889746e-05\\
3.8	0.7	5.79718926210514e-05	5.79718926210514e-05\\
3.8	0.8	4.5696225742389e-05	4.5696225742389e-05\\
3.8	0.9	2.98565995058098e-05	2.98565995058098e-05\\
3.8	1	1.20286527995867e-05	1.20286527995867e-05\\
3.8	1.1	3.04081484528075e-05	3.04081484528075e-05\\
3.8	1.2	7.72601581545316e-05	7.72601581545316e-05\\
3.8	1.3	0.000125337488215706	0.000125337488215706\\
3.8	1.4	0.000142995745641659	0.000142995745641659\\
3.8	1.5	0.000134404210781629	0.000134404210781629\\
3.8	1.6	0.000117257909853834	0.000117257909853834\\
3.8	1.7	0.000108764595350833	0.000108764595350833\\
3.8	1.8	0.000122078005985552	0.000122078005985552\\
3.8	1.9	0.00016527877639824	0.00016527877639824\\
3.8	2	0.000233592994544027	0.000233592994544027\\
3.8	2.1	0.000337481794502277	0.000337481794502277\\
3.8	2.2	0.000527188053378146	0.000527188053378146\\
3.8	2.3	0.000702377965450176	0.000702377965450176\\
3.8	2.4	0.000582440580500248	0.000582440580500248\\
3.8	2.5	0.000306188808074529	0.000306188808074529\\
3.8	2.6	0.000141709582356401	0.000141709582356401\\
3.8	2.7	8.79562139271773e-05	8.79562139271773e-05\\
3.8	2.8	8.07401915693324e-05	8.07401915693324e-05\\
3.8	2.9	9.07494949186023e-05	9.07494949186023e-05\\
3.8	3	0.000169562796931598	0.000169562796931598\\
3.8	3.1	0.000211536931675126	0.000211536931675126\\
3.8	3.2	0.000288149615339564	0.000288149615339564\\
3.8	3.3	0.000408465687704527	0.000408465687704527\\
3.8	3.4	0.000572692091899441	0.000572692091899441\\
3.8	3.5	0.000729351305653991	0.000729351305653991\\
3.8	3.6	0.00076739140557805	0.00076739140557805\\
3.8	3.7	0.000692583571526078	0.000692583571526078\\
3.8	3.8	0.000621254815901665	0.000621254815901665\\
3.8	3.9	0.00057441215772395	0.00057441215772395\\
3.8	4	0.000494766378917338	0.000494766378917338\\
3.8	4.1	0.00038003023875405	0.00038003023875405\\
3.8	4.2	0.000275997608574512	0.000275997608574512\\
3.8	4.3	0.000201316247505963	0.000201316247505963\\
3.8	4.4	0.00015048264085669	0.00015048264085669\\
3.8	4.5	0.000115124798825256	0.000115124798825256\\
3.8	4.6	8.87738087344665e-05	8.87738087344665e-05\\
3.8	4.7	6.50504877732347e-05	6.50504877732347e-05\\
3.8	4.8	3.53985265048396e-05	3.53985265048396e-05\\
3.8	4.9	7.64479141753582e-06	7.64479141753582e-06\\
3.8	5	8.94575749996238e-06	8.94575749996238e-06\\
3.8	5.1	6.51572088174479e-06	6.51572088174479e-06\\
3.8	5.2	2.86015678636485e-05	2.86015678636485e-05\\
3.8	5.3	4.71554269382352e-05	4.71554269382352e-05\\
3.8	5.4	5.39037753943929e-05	5.39037753943929e-05\\
3.8	5.5	5.87601493801195e-05	5.87601493801195e-05\\
3.8	5.6	6.6460504494126e-05	6.6460504494126e-05\\
3.8	5.7	8.02518567944113e-05	8.02518567944113e-05\\
3.8	5.8	0.000104890294171187	0.000104890294171187\\
3.8	5.9	0.000149713682559461	0.000149713682559461\\
3.8	6	0.000234701971501259	0.000234701971501259\\
3.9	0	0.000176416560283244	0.000176416560283244\\
3.9	0.1	0.000117991123480683	0.000117991123480683\\
3.9	0.2	8.61982385503344e-05	8.61982385503344e-05\\
3.9	0.3	6.85202967112629e-05	6.85202967112629e-05\\
3.9	0.4	5.79224068448069e-05	5.79224068448069e-05\\
3.9	0.5	5.0174073756259e-05	5.0174073756259e-05\\
3.9	0.6	4.34726178531329e-05	4.34726178531329e-05\\
3.9	0.7	3.69862203732333e-05	3.69862203732333e-05\\
3.9	0.8	2.44296358345642e-05	2.44296358345642e-05\\
3.9	0.9	1.20765611750017e-05	1.20765611750017e-05\\
3.9	1	9.80937913700783e-06	9.80937913700783e-06\\
3.9	1.1	1.13272699452441e-05	1.13272699452441e-05\\
3.9	1.2	2.33656426197379e-05	2.33656426197379e-05\\
3.9	1.3	4.15846846566238e-05	4.15846846566238e-05\\
3.9	1.4	6.00973773137211e-05	6.00973773137211e-05\\
3.9	1.5	7.23456414518216e-05	7.23456414518216e-05\\
3.9	1.6	8.10571301273226e-05	8.10571301273226e-05\\
3.9	1.7	8.90033578429206e-05	8.90033578429206e-05\\
3.9	1.8	0.000105593145991328	0.000105593145991328\\
3.9	1.9	0.000144922758919814	0.000144922758919814\\
3.9	2	0.000197065754542224	0.000197065754542224\\
3.9	2.1	0.000254643478884964	0.000254643478884964\\
3.9	2.2	0.000332036934175432	0.000332036934175432\\
3.9	2.3	0.000357286892567969	0.000357286892567969\\
3.9	2.4	0.000259950489931513	0.000259950489931513\\
3.9	2.5	0.000137069575684174	0.000137069575684174\\
3.9	2.6	6.96836530107326e-05	6.96836530107326e-05\\
3.9	2.7	5.00632744425296e-05	5.00632744425296e-05\\
3.9	2.8	4.88596322744951e-05	4.88596322744951e-05\\
3.9	2.9	5.17467963262245e-05	5.17467963262245e-05\\
3.9	3	0.000106751586303705	0.000106751586303705\\
3.9	3.1	0.000142749329389501	0.000142749329389501\\
3.9	3.2	0.000222539424717374	0.000222539424717374\\
3.9	3.3	0.000360213204625041	0.000360213204625041\\
3.9	3.4	0.000549904899313362	0.000549904899313362\\
3.9	3.5	0.000736615886389461	0.000736615886389461\\
3.9	3.6	0.000799102140366637	0.000799102140366637\\
3.9	3.7	0.000724387397951567	0.000724387397951567\\
3.9	3.8	0.000627911484975103	0.000627911484975103\\
3.9	3.9	0.000528314057871923	0.000528314057871923\\
3.9	4	0.00038894361226233	0.00038894361226233\\
3.9	4.1	0.00025025023180672	0.00025025023180672\\
3.9	4.2	0.000157625025542344	0.000157625025542344\\
3.9	4.3	0.000105079618139643	0.000105079618139643\\
3.9	4.4	7.40148485038498e-05	7.40148485038498e-05\\
3.9	4.5	5.19183297578398e-05	5.19183297578398e-05\\
3.9	4.6	3.1763559429373e-05	3.1763559429373e-05\\
3.9	4.7	1.38249717191975e-05	1.38249717191975e-05\\
3.9	4.8	4.89152527517735e-06	4.89152527517735e-06\\
3.9	4.9	3.95508036266279e-06	3.95508036266279e-06\\
3.9	5	7.60044681516138e-06	7.60044681516138e-06\\
3.9	5.1	4.56545299237972e-06	4.56545299237972e-06\\
3.9	5.2	4.56899472016991e-06	4.56899472016991e-06\\
3.9	5.3	1.04723953887618e-05	1.04723953887618e-05\\
3.9	5.4	2.30028143535761e-05	2.30028143535761e-05\\
3.9	5.5	3.70939935065732e-05	3.70939935065732e-05\\
3.9	5.6	5.12408021802706e-05	5.12408021802706e-05\\
3.9	5.7	6.8810417821098e-05	6.8810417821098e-05\\
3.9	5.8	9.49130163364095e-05	9.49130163364095e-05\\
3.9	5.9	0.000137454011472657	0.000137454011472657\\
3.9	6	0.000211147177386102	0.000211147177386102\\
4	0	0.000145658953552357	0.000145658953552357\\
4	0.1	0.000102370782333996	0.000102370782333996\\
4	0.2	7.73504282281339e-05	7.73504282281339e-05\\
4	0.3	6.21132600743432e-05	6.21132600743432e-05\\
4	0.4	5.18956346136436e-05	5.18956346136436e-05\\
4	0.5	4.3961874879882e-05	4.3961874879882e-05\\
4	0.6	3.55808062407512e-05	3.55808062407512e-05\\
4	0.7	2.37887857943077e-05	2.37887857943077e-05\\
4	0.8	1.29710713267687e-05	1.29710713267687e-05\\
4	0.9	8.41036263133222e-06	8.41036263133222e-06\\
4	1	8.18365223078316e-06	8.18365223078316e-06\\
4	1.1	8.4197314592277e-06	8.4197314592277e-06\\
4	1.2	1.2233970150258e-05	1.2233970150258e-05\\
4	1.3	2.18981545095193e-05	2.18981545095193e-05\\
4	1.4	3.12546391126482e-05	3.12546391126482e-05\\
4	1.5	3.76496253027718e-05	3.76496253027718e-05\\
4	1.6	4.46389449685555e-05	4.46389449685555e-05\\
4	1.7	5.665013435775e-05	5.665013435775e-05\\
4	1.8	7.3642274258397e-05	7.3642274258397e-05\\
4	1.9	0.00010526984371632	0.00010526984371632\\
4	2	0.000159640737572824	0.000159640737572824\\
4	2.1	0.000229315849928522	0.000229315849928522\\
4	2.2	0.000288113548858963	0.000288113548858963\\
4	2.3	0.000247785594603824	0.000247785594603824\\
4	2.4	0.000144865995084805	0.000144865995084805\\
4	2.5	7.28709559647804e-05	7.28709559647804e-05\\
4	2.6	4.04962914180476e-05	4.04962914180476e-05\\
4	2.7	3.3954043574031e-05	3.3954043574031e-05\\
4	2.8	3.32905673112262e-05	3.32905673112262e-05\\
4	2.9	2.96750018409629e-05	2.96750018409629e-05\\
4	3	6.75660194448464e-05	6.75660194448464e-05\\
4	3.1	9.81864976232081e-05	9.81864976232081e-05\\
4	3.2	0.000179334917769125	0.000179334917769125\\
4	3.3	0.000335757947757867	0.000335757947757867\\
4	3.4	0.000555295865224059	0.000555295865224059\\
4	3.5	0.000776579537811869	0.000776579537811869\\
4	3.6	0.000865300218455546	0.000865300218455546\\
4	3.7	0.000768846170345035	0.000768846170345035\\
4	3.8	0.000611103546254218	0.000611103546254218\\
4	3.9	0.0004415235309924	0.0004415235309924\\
4	4	0.000267935689509342	0.000267935689509342\\
4	4.1	0.000145474881081182	0.000145474881081182\\
4	4.2	8.19297169361945e-05	8.19297169361945e-05\\
4	4.3	4.99605999799869e-05	4.99605999799869e-05\\
4	4.4	3.08937159385944e-05	3.08937159385944e-05\\
4	4.5	1.72881697485935e-05	1.72881697485935e-05\\
4	4.6	8.18034521221369e-06	8.18034521221369e-06\\
4	4.7	3.87414891759183e-06	3.87414891759183e-06\\
4	4.8	2.90594583664059e-06	2.90594583664059e-06\\
4	4.9	4.45455503377574e-06	4.45455503377574e-06\\
4	5	6.95115165921532e-06	6.95115165921532e-06\\
4	5.1	5.80417769130192e-06	5.80417769130192e-06\\
4	5.2	4.23855070781356e-06	4.23855070781356e-06\\
4	5.3	4.69294372293949e-06	4.69294372293949e-06\\
4	5.4	7.69055255439849e-06	7.69055255439849e-06\\
4	5.5	1.43909539386768e-05	1.43909539386768e-05\\
4	5.6	2.55296422429448e-05	2.55296422429448e-05\\
4	5.7	4.18279844338207e-05	4.18279844338207e-05\\
4	5.8	6.59116751660675e-05	6.59116751660675e-05\\
4	5.9	0.000103637980790682	0.000103637980790682\\
4	6	0.000166172105178079	0.000166172105178079\\
4.1	0	0.000135025701888082	0.000135025701888082\\
4.1	0.1	9.99328223220391e-05	9.99328223220391e-05\\
4.1	0.2	7.75181334099271e-05	7.75181334099271e-05\\
4.1	0.3	6.15628304462107e-05	6.15628304462107e-05\\
4.1	0.4	4.85725784192459e-05	4.85725784192459e-05\\
4.1	0.5	3.65145029332096e-05	3.65145029332096e-05\\
4.1	0.6	2.45566869595693e-05	2.45566869595693e-05\\
4.1	0.7	1.46229963704441e-05	1.46229963704441e-05\\
4.1	0.8	9.07673826843093e-06	9.07673826843093e-06\\
4.1	0.9	7.2378013128974e-06	7.2378013128974e-06\\
4.1	1	7.24402355613926e-06	7.24402355613926e-06\\
4.1	1.1	7.32109716918586e-06	7.32109716918586e-06\\
4.1	1.2	8.50897584674236e-06	8.50897584674236e-06\\
4.1	1.3	1.32664468132625e-05	1.32664468132625e-05\\
4.1	1.4	2.17244843886717e-05	2.17244843886717e-05\\
4.1	1.5	2.90041950970255e-05	2.90041950970255e-05\\
4.1	1.6	3.25229344631271e-05	3.25229344631271e-05\\
4.1	1.7	3.76567378320665e-05	3.76567378320665e-05\\
4.1	1.8	4.95470482387218e-05	4.95470482387218e-05\\
4.1	1.9	7.61984439822974e-05	7.61984439822974e-05\\
4.1	2	0.000136183170049357	0.000136183170049357\\
4.1	2.1	0.00023505507645023	0.00023505507645023\\
4.1	2.2	0.000307188669639878	0.000307188669639878\\
4.1	2.3	0.00021720610638986	0.00021720610638986\\
4.1	2.4	0.000101191977908175	0.000101191977908175\\
4.1	2.5	4.86448356058042e-05	4.86448356058042e-05\\
4.1	2.6	2.80995829339684e-05	2.80995829339684e-05\\
4.1	2.7	2.52557218731766e-05	2.52557218731766e-05\\
4.1	2.8	2.25894994964286e-05	2.25894994964286e-05\\
4.1	2.9	1.62217981558774e-05	1.62217981558774e-05\\
4.1	3	4.28809974782934e-05	4.28809974782934e-05\\
4.1	3.1	6.80610824408859e-05	6.80610824408859e-05\\
4.1	3.2	0.000143761230166249	0.000143761230166249\\
4.1	3.3	0.000308666178417084	0.000308666178417084\\
4.1	3.4	0.000553188508640658	0.000553188508640658\\
4.1	3.5	0.000812995778253798	0.000812995778253798\\
4.1	3.6	0.00094448247697286	0.00094448247697286\\
4.1	3.7	0.000820248755641996	0.000820248755641996\\
4.1	3.8	0.000582428564880454	0.000582428564880454\\
4.1	3.9	0.000348689182180062	0.000348689182180062\\
4.1	4	0.000170470390113719	0.000170470390113719\\
4.1	4.1	7.79871511534377e-05	7.79871511534377e-05\\
4.1	4.2	3.85531800906788e-05	3.85531800906788e-05\\
4.1	4.3	2.05927429851353e-05	2.05927429851353e-05\\
4.1	4.4	1.10277557360705e-05	1.10277557360705e-05\\
4.1	4.5	5.69252058549307e-06	5.69252058549307e-06\\
4.1	4.6	3.12730750642678e-06	3.12730750642678e-06\\
4.1	4.7	2.33738404571157e-06	2.33738404571157e-06\\
4.1	4.8	2.88212755569946e-06	2.88212755569946e-06\\
4.1	4.9	4.83259377585555e-06	4.83259377585555e-06\\
4.1	5	6.86148760706386e-06	6.86148760706386e-06\\
4.1	5.1	6.67577125979022e-06	6.67577125979022e-06\\
4.1	5.2	5.28501838413271e-06	5.28501838413271e-06\\
4.1	5.3	4.65751512707724e-06	4.65751512707724e-06\\
4.1	5.4	5.28312117400144e-06	5.28312117400144e-06\\
4.1	5.5	7.52124068427549e-06	7.52124068427549e-06\\
4.1	5.6	1.22890824031917e-05	1.22890824031917e-05\\
4.1	5.7	2.1183174518769e-05	2.1183174518769e-05\\
4.1	5.8	3.67396497879099e-05	3.67396497879099e-05\\
4.1	5.9	6.32716675075985e-05	6.32716675075985e-05\\
4.1	6	0.000108507864968626	0.000108507864968626\\
4.2	0	0.000139697850442228	0.000139697850442228\\
4.2	0.1	0.000104283514777077	0.000104283514777077\\
4.2	0.2	7.80071743389326e-05	7.80071743389326e-05\\
4.2	0.3	5.70437691473226e-05	5.70437691473226e-05\\
4.2	0.4	4.00042599229872e-05	4.00042599229872e-05\\
4.2	0.5	2.65181063609355e-05	2.65181063609355e-05\\
4.2	0.6	1.67470175870079e-05	1.67470175870079e-05\\
4.2	0.7	1.08031145878551e-05	1.08031145878551e-05\\
4.2	0.8	7.89869335177367e-06	7.89869335177367e-06\\
4.2	0.9	6.95083071228788e-06	6.95083071228788e-06\\
4.2	1	6.89347091327554e-06	6.89347091327554e-06\\
4.2	1.1	6.82827039492218e-06	6.82827039492218e-06\\
4.2	1.2	7.08418815099605e-06	7.08418815099605e-06\\
4.2	1.3	8.94319033405565e-06	8.94319033405565e-06\\
4.2	1.4	1.41196278590271e-05	1.41196278590271e-05\\
4.2	1.5	2.41824910852669e-05	2.41824910852669e-05\\
4.2	1.6	3.35613888753356e-05	3.35613888753356e-05\\
4.2	1.7	3.63092855206303e-05	3.63092855206303e-05\\
4.2	1.8	4.12091284174407e-05	4.12091284174407e-05\\
4.2	1.9	6.6958574078127e-05	6.6958574078127e-05\\
4.2	2	0.000140831839053507	0.000140831839053507\\
4.2	2.1	0.000268380700265232	0.000268380700265232\\
4.2	2.2	0.000368077387829479	0.000368077387829479\\
4.2	2.3	0.000228001768660428	0.000228001768660428\\
4.2	2.4	8.81250206046851e-05	8.81250206046851e-05\\
4.2	2.5	4.17894490141898e-05	4.17894490141898e-05\\
4.2	2.6	2.29244792754704e-05	2.29244792754704e-05\\
4.2	2.7	1.91753084099735e-05	1.91753084099735e-05\\
4.2	2.8	1.37676863902921e-05	1.37676863902921e-05\\
4.2	2.9	8.36847320300356e-06	8.36847320300356e-06\\
4.2	3	2.75631739295441e-05	2.75631739295441e-05\\
4.2	3.1	4.6694224522228e-05	4.6694224522228e-05\\
4.2	3.2	0.000107082542359887	0.000107082542359887\\
4.2	3.3	0.000255551194085685	0.000255551194085685\\
4.2	3.4	0.000506196456156619	0.000506196456156619\\
4.2	3.5	0.000810483753871215	0.000810483753871215\\
4.2	3.6	0.0010257496510224	0.0010257496510224\\
4.2	3.7	0.000887127916405748	0.000887127916405748\\
4.2	3.8	0.000548563763895778	0.000548563763895778\\
4.2	3.9	0.000259618507513034	0.000259618507513034\\
4.2	4	0.000100130950332056	0.000100130950332056\\
4.2	4.1	3.86355826748827e-05	3.86355826748827e-05\\
4.2	4.2	1.69750271526209e-05	1.69750271526209e-05\\
4.2	4.3	8.40075600504551e-06	8.40075600504551e-06\\
4.2	4.4	4.47061926941152e-06	4.47061926941152e-06\\
4.2	4.5	2.64071730603416e-06	2.64071730603416e-06\\
4.2	4.6	1.98157573669898e-06	1.98157573669898e-06\\
4.2	4.7	2.16166920189035e-06	2.16166920189035e-06\\
4.2	4.8	3.28206624439798e-06	3.28206624439798e-06\\
4.2	4.9	5.34367695376727e-06	5.34367695376727e-06\\
4.2	5	7.25876281114505e-06	7.25876281114505e-06\\
4.2	5.1	7.60808170867836e-06	7.60808170867836e-06\\
4.2	5.2	6.62483704621873e-06	6.62483704621873e-06\\
4.2	5.3	5.63342683251308e-06	5.63342683251308e-06\\
4.2	5.4	5.39066111986291e-06	5.39066111986291e-06\\
4.2	5.5	6.13410597894193e-06	6.13410597894193e-06\\
4.2	5.6	8.21054301994952e-06	8.21054301994952e-06\\
4.2	5.7	1.24546944948032e-05	1.24546944948032e-05\\
4.2	5.8	2.05550794546398e-05	2.05550794546398e-05\\
4.2	5.9	3.56096081455984e-05	3.56096081455984e-05\\
4.2	6	6.30884504996733e-05	6.30884504996733e-05\\
4.3	0	0.000143778011847171	0.000143778011847171\\
4.3	0.1	0.000102115920560835	0.000102115920560835\\
4.3	0.2	7.03391059857531e-05	7.03391059857531e-05\\
4.3	0.3	4.6690666352286e-05	4.6690666352286e-05\\
4.3	0.4	3.01240485820743e-05	3.01240485820743e-05\\
4.3	0.5	1.93703762660102e-05	1.93703762660102e-05\\
4.3	0.6	1.29736057887338e-05	1.29736057887338e-05\\
4.3	0.7	9.47637632588755e-06	9.47637632588755e-06\\
4.3	0.8	7.76896034347087e-06	7.76896034347087e-06\\
4.3	0.9	7.16645636512055e-06	7.16645636512055e-06\\
4.3	1	7.01566334369516e-06	7.01566334369516e-06\\
4.3	1.1	6.77235799146693e-06	6.77235799146693e-06\\
4.3	1.2	6.54438366179692e-06	6.54438366179692e-06\\
4.3	1.3	6.97573971906024e-06	6.97573971906024e-06\\
4.3	1.4	9.11915817474028e-06	9.11915817474028e-06\\
4.3	1.5	1.62548738478082e-05	1.62548738478082e-05\\
4.3	1.6	3.29484318373592e-05	3.29484318373592e-05\\
4.3	1.7	4.49344538760507e-05	4.49344538760507e-05\\
4.3	1.8	4.39102098162537e-05	4.39102098162537e-05\\
4.3	1.9	6.78139803984327e-05	6.78139803984327e-05\\
4.3	2	0.000164681563228941	0.000164681563228941\\
4.3	2.1	0.000311907757072338	0.000311907757072338\\
4.3	2.2	0.000443688381409639	0.000443688381409639\\
4.3	2.3	0.000264188934033757	0.000264188934033757\\
4.3	2.4	8.82227388110117e-05	8.82227388110117e-05\\
4.3	2.5	4.25406910384795e-05	4.25406910384795e-05\\
4.3	2.6	2.11812964069009e-05	2.11812964069009e-05\\
4.3	2.7	1.53010095389868e-05	1.53010095389868e-05\\
4.3	2.8	7.96503417106417e-06	7.96503417106417e-06\\
4.3	2.9	4.74946972753125e-06	4.74946972753125e-06\\
4.3	3	1.88216736912269e-05	1.88216736912269e-05\\
4.3	3.1	3.17782371059167e-05	3.17782371059167e-05\\
4.3	3.2	7.14433271348054e-05	7.14433271348054e-05\\
4.3	3.3	0.000179965214379568	0.000179965214379568\\
4.3	3.4	0.00040692011790521	0.00040692011790521\\
4.3	3.5	0.000755301781899439	0.000755301781899439\\
4.3	3.6	0.00111574987542411	0.00111574987542411\\
4.3	3.7	0.000988042647743136	0.000988042647743136\\
4.3	3.8	0.000506636193706968	0.000506636193706968\\
4.3	3.9	0.000180752564784004	0.000180752564784004\\
4.3	4	5.51085395987697e-05	5.51085395987697e-05\\
4.3	4.1	1.84536199612755e-05	1.84536199612755e-05\\
4.3	4.2	7.73224874850487e-06	7.73224874850487e-06\\
4.3	4.3	3.96301545422012e-06	3.96301545422012e-06\\
4.3	4.4	2.39827867333617e-06	2.39827867333617e-06\\
4.3	4.5	1.79971537356304e-06	1.79971537356304e-06\\
4.3	4.6	1.81846205336458e-06	1.81846205336458e-06\\
4.3	4.7	2.47858857723877e-06	2.47858857723877e-06\\
4.3	4.8	3.97273094234927e-06	3.97273094234927e-06\\
4.3	4.9	6.1586477727813e-06	6.1586477727813e-06\\
4.3	5	8.11543753668023e-06	8.11543753668023e-06\\
4.3	5.1	8.80127987345088e-06	8.80127987345088e-06\\
4.3	5.2	8.16075265554396e-06	8.16075265554396e-06\\
4.3	5.3	7.04756932294488e-06	7.04756932294488e-06\\
4.3	5.4	6.27576122444777e-06	6.27576122444777e-06\\
4.3	5.5	6.2228656984818e-06	6.2228656984818e-06\\
4.3	5.6	7.09847314909607e-06	7.09847314909607e-06\\
4.3	5.7	9.27099117075357e-06	9.27099117075357e-06\\
4.3	5.8	1.35623005898378e-05	1.35623005898378e-05\\
4.3	5.9	2.16597911848704e-05	2.16597911848704e-05\\
4.3	6	3.68042661043802e-05	3.68042661043802e-05\\
4.4	0	0.000135791133402001	0.000135791133402001\\
4.4	0.1	8.9843919408808e-05	8.9843919408808e-05\\
4.4	0.2	5.74339688781126e-05	5.74339688781126e-05\\
4.4	0.3	3.60422981896427e-05	3.60422981896427e-05\\
4.4	0.4	2.2953459588189e-05	2.2953459588189e-05\\
4.4	0.5	1.54808410862629e-05	1.54808410862629e-05\\
4.4	0.6	1.14057919954651e-05	1.14057919954651e-05\\
4.4	0.7	9.22797047549831e-06	9.22797047549831e-06\\
4.4	0.8	8.16195956098312e-06	8.16195956098312e-06\\
4.4	0.9	7.7695551825845e-06	7.7695551825845e-06\\
4.4	1	7.57566156055352e-06	7.57566156055352e-06\\
4.4	1.1	7.17406418764302e-06	7.17406418764302e-06\\
4.4	1.2	6.59259596620294e-06	6.59259596620294e-06\\
4.4	1.3	6.23875021008451e-06	6.23875021008451e-06\\
4.4	1.4	6.6866054851329e-06	6.6866054851329e-06\\
4.4	1.5	9.76474357048699e-06	9.76474357048699e-06\\
4.4	1.6	2.34836517814904e-05	2.34836517814904e-05\\
4.4	1.7	5.26307100733921e-05	5.26307100733921e-05\\
4.4	1.8	5.3677493217679e-05	5.3677493217679e-05\\
4.4	1.9	7.00451833722708e-05	7.00451833722708e-05\\
4.4	2	0.00017842378120518	0.00017842378120518\\
4.4	2.1	0.000310489523353579	0.000310489523353579\\
4.4	2.2	0.000442792336129113	0.000442792336129113\\
4.4	2.3	0.000291399905217996	0.000291399905217996\\
4.4	2.4	8.83599973635812e-05	8.83599973635812e-05\\
4.4	2.5	4.32131097397771e-05	4.32131097397771e-05\\
4.4	2.6	2.10473901051981e-05	2.10473901051981e-05\\
4.4	2.7	1.29817602520529e-05	1.29817602520529e-05\\
4.4	2.8	4.70789191900114e-06	4.70789191900114e-06\\
4.4	2.9	3.39031541019819e-06	3.39031541019819e-06\\
4.4	3	1.42523031768278e-05	1.42523031768278e-05\\
4.4	3.1	2.18723800483154e-05	2.18723800483154e-05\\
4.4	3.2	4.32144489808822e-05	4.32144489808822e-05\\
4.4	3.3	0.000106755814395898	0.000106755814395898\\
4.4	3.4	0.00028007745690746	0.00028007745690746\\
4.4	3.5	0.000645161945555074	0.000645161945555074\\
4.4	3.6	0.00121105393814081	0.00121105393814081\\
4.4	3.7	0.00114164654853593	0.00114164654853593\\
4.4	3.8	0.000452783836933867	0.000452783836933867\\
4.4	3.9	0.000117563824209465	0.000117563824209465\\
4.4	4	2.8536722722763e-05	2.8536722722763e-05\\
4.4	4.1	8.8413155327382e-06	8.8413155327382e-06\\
4.4	4.2	3.96927081077067e-06	3.96927081077067e-06\\
4.4	4.3	2.36882401872592e-06	2.36882401872592e-06\\
4.4	4.4	1.77960089504171e-06	1.77960089504171e-06\\
4.4	4.5	1.71775861815723e-06	1.71775861815723e-06\\
4.4	4.6	2.13932033771064e-06	2.13932033771064e-06\\
4.4	4.7	3.19925520144745e-06	3.19925520144745e-06\\
4.4	4.8	5.0385040480176e-06	5.0385040480176e-06\\
4.4	4.9	7.38785937579262e-06	7.38785937579262e-06\\
4.4	5	9.41602535230568e-06	9.41602535230568e-06\\
4.4	5.1	1.02838897872488e-05	1.02838897872488e-05\\
4.4	5.2	9.84982716254183e-06	9.84982716254183e-06\\
4.4	5.3	8.6882933886412e-06	8.6882933886412e-06\\
4.4	5.4	7.53872230581263e-06	7.53872230581263e-06\\
4.4	5.5	6.88894759427882e-06	6.88894759427882e-06\\
4.4	5.6	6.98459511594975e-06	6.98459511594975e-06\\
4.4	5.7	8.03711072375861e-06	8.03711072375861e-06\\
4.4	5.8	1.04423703598844e-05	1.04423703598844e-05\\
4.4	5.9	1.50366239420106e-05	1.50366239420106e-05\\
4.4	6	2.35043259930895e-05	2.35043259930895e-05\\
4.5	0	0.000117241294332841	0.000117241294332841\\
4.5	0.1	7.32098670109322e-05	7.32098670109322e-05\\
4.5	0.2	4.49233695300062e-05	4.49233695300062e-05\\
4.5	0.3	2.80094008428843e-05	2.80094008428843e-05\\
4.5	0.4	1.85435935368337e-05	1.85435935368337e-05\\
4.5	0.5	1.34906958286335e-05	1.34906958286335e-05\\
4.5	0.6	1.08426686159863e-05	1.08426686159863e-05\\
4.5	0.7	9.48429555063073e-06	9.48429555063073e-06\\
4.5	0.8	8.89524863587451e-06	8.89524863587451e-06\\
4.5	0.9	8.74351018592581e-06	8.74351018592581e-06\\
4.5	1	8.62388313615998e-06	8.62388313615998e-06\\
4.5	1.1	8.16823494987711e-06	8.16823494987711e-06\\
4.5	1.2	7.33390979152728e-06	7.33390979152728e-06\\
4.5	1.3	6.43251057056494e-06	6.43251057056494e-06\\
4.5	1.4	5.9195082040011e-06	5.9195082040011e-06\\
4.5	1.5	6.6272501263336e-06	6.6272501263336e-06\\
4.5	1.6	1.29490713825801e-05	1.29490713825801e-05\\
4.5	1.7	4.54333508149476e-05	4.54333508149476e-05\\
4.5	1.8	6.72875151074337e-05	6.72875151074337e-05\\
4.5	1.9	7.48174220923745e-05	7.48174220923745e-05\\
4.5	2	0.000171839155076584	0.000171839155076584\\
4.5	2.1	0.000241872038753033	0.000241872038753033\\
4.5	2.2	0.00031478283612396	0.00031478283612396\\
4.5	2.3	0.000286051289018037	0.000286051289018037\\
4.5	2.4	9.22150286960315e-05	9.22150286960315e-05\\
4.5	2.5	4.3510726241282e-05	4.3510726241282e-05\\
4.5	2.6	2.59620378955207e-05	2.59620378955207e-05\\
4.5	2.7	1.24907152675104e-05	1.24907152675104e-05\\
4.5	2.8	3.21244232796215e-06	3.21244232796215e-06\\
4.5	2.9	3.15194447493617e-06	3.15194447493617e-06\\
4.5	3	1.23110626132379e-05	1.23110626132379e-05\\
4.5	3.1	1.5821970265419e-05	1.5821970265419e-05\\
4.5	3.2	2.5119027573402e-05	2.5119027573402e-05\\
4.5	3.3	5.45122636395066e-05	5.45122636395066e-05\\
4.5	3.4	0.000158314891900593	0.000158314891900593\\
4.5	3.5	0.000479753010153314	0.000479753010153314\\
4.5	3.6	0.00126508319057487	0.00126508319057487\\
4.5	3.7	0.00136825628996374	0.00136825628996374\\
4.5	3.8	0.000384966415115745	0.000384966415115745\\
4.5	3.9	6.94228901688579e-05	6.94228901688579e-05\\
4.5	4	1.36841545758166e-05	1.36841545758166e-05\\
4.5	4.1	4.56551572480583e-06	4.56551572480583e-06\\
4.5	4.2	2.5483406527693e-06	2.5483406527693e-06\\
4.5	4.3	1.93105771948391e-06	1.93105771948391e-06\\
4.5	4.4	1.83807823759958e-06	1.83807823759958e-06\\
4.5	4.5	2.15290887434729e-06	2.15290887434729e-06\\
4.5	4.6	2.96927890031216e-06	2.96927890031216e-06\\
4.5	4.7	4.45322452171885e-06	4.45322452171885e-06\\
4.5	4.8	6.62652725259009e-06	6.62652725259009e-06\\
4.5	4.9	9.09828577032982e-06	9.09828577032982e-06\\
4.5	5	1.10981859534521e-05	1.10981859534521e-05\\
4.5	5.1	1.19553161405952e-05	1.19553161405952e-05\\
4.5	5.2	1.15566656125348e-05	1.15566656125348e-05\\
4.5	5.3	1.03354308125597e-05	1.03354308125597e-05\\
4.5	5.4	8.91656664916144e-06	8.91656664916144e-06\\
4.5	5.5	7.80345380193382e-06	7.80345380193382e-06\\
4.5	5.6	7.29319050349237e-06	7.29319050349237e-06\\
4.5	5.7	7.56676616786226e-06	7.56676616786226e-06\\
4.5	5.8	8.84249068459406e-06	8.84249068459406e-06\\
4.5	5.9	1.1547096095276e-05	1.1547096095276e-05\\
4.5	6	1.6543826865406e-05	1.6543826865406e-05\\
4.6	0	9.51198111701123e-05	9.51198111701123e-05\\
4.6	0.1	5.74060940031745e-05	5.74060940031745e-05\\
4.6	0.2	3.4970402852237e-05	3.4970402852237e-05\\
4.6	0.3	2.24348827743003e-05	2.24348827743003e-05\\
4.6	0.4	1.57716180826713e-05	1.57716180826713e-05\\
4.6	0.5	1.23520829825687e-05	1.23520829825687e-05\\
4.6	0.6	1.06649885407546e-05	1.06649885407546e-05\\
4.6	0.7	9.94835697096688e-06	9.94835697096688e-06\\
4.6	0.8	9.83771685630189e-06	9.83771685630189e-06\\
4.6	0.9	1.00475729694054e-05	1.00475729694054e-05\\
4.6	1	1.02031495753761e-05	1.02031495753761e-05\\
4.6	1.1	9.92847976235164e-06	9.92847976235164e-06\\
4.6	1.2	9.07209719092102e-06	9.07209719092102e-06\\
4.6	1.3	7.80458604251398e-06	7.80458604251398e-06\\
4.6	1.4	6.55286398973513e-06	6.55286398973513e-06\\
4.6	1.5	5.93593882614769e-06	5.93593882614769e-06\\
4.6	1.6	7.70322190713119e-06	7.70322190713119e-06\\
4.6	1.7	2.65215470275578e-05	2.65215470275578e-05\\
4.6	1.8	7.98174688995981e-05	7.98174688995981e-05\\
4.6	1.9	9.11102104657132e-05	9.11102104657132e-05\\
4.6	2	0.000158247766008775	0.000158247766008775\\
4.6	2.1	0.000138301409462464	0.000138301409462464\\
4.6	2.2	0.000140727297858574	0.000140727297858574\\
4.6	2.3	0.000203623122237313	0.000203623122237313\\
4.6	2.4	0.000122594695497588	0.000122594695497588\\
4.6	2.5	5.7105206973176e-05	5.7105206973176e-05\\
4.6	2.6	5.34831582172588e-05	5.34831582172588e-05\\
4.6	2.7	1.29871167906404e-05	1.29871167906404e-05\\
4.6	2.8	3.21978964423734e-06	3.21978964423734e-06\\
4.6	2.9	4.41478434762586e-06	4.41478434762586e-06\\
4.6	3	1.32105742942437e-05	1.32105742942437e-05\\
4.6	3.1	1.3539353123887e-05	1.3539353123887e-05\\
4.6	3.2	1.63351111408494e-05	1.63351111408494e-05\\
4.6	3.3	2.69546023783621e-05	2.69546023783621e-05\\
4.6	3.4	7.12996821525274e-05	7.12996821525274e-05\\
4.6	3.5	0.00028198540495739	0.00028198540495739\\
4.6	3.6	0.00117027249444578	0.00117027249444578\\
4.6	3.7	0.00168583156834959	0.00168583156834959\\
4.6	3.8	0.000296417005862686	0.000296417005862686\\
4.6	3.9	3.42176201260749e-05	3.42176201260749e-05\\
4.6	4	6.40506999062227e-06	6.40506999062227e-06\\
4.6	4.1	2.98865767153483e-06	2.98865767153483e-06\\
4.6	4.2	2.30818273639551e-06	2.30818273639551e-06\\
4.6	4.3	2.24678528957099e-06	2.24678528957099e-06\\
4.6	4.4	2.56002836546655e-06	2.56002836546655e-06\\
4.6	4.5	3.28851856793707e-06	3.28851856793707e-06\\
4.6	4.6	4.55557869762895e-06	4.55557869762895e-06\\
4.6	4.7	6.44807742875105e-06	6.44807742875105e-06\\
4.6	4.8	8.83218149780159e-06	8.83218149780159e-06\\
4.6	4.9	1.12288864866599e-05	1.12288864866599e-05\\
4.6	5	1.29686061699827e-05	1.29686061699827e-05\\
4.6	5.1	1.35774755017417e-05	1.35774755017417e-05\\
4.6	5.2	1.30377025946567e-05	1.30377025946567e-05\\
4.6	5.3	1.17182758142434e-05	1.17182758142434e-05\\
4.6	5.4	1.01260484142276e-05	1.01260484142276e-05\\
4.6	5.5	8.7026878067673e-06	8.7026878067673e-06\\
4.6	5.6	7.7477117056556e-06	7.7477117056556e-06\\
4.6	5.7	7.44205279970434e-06	7.44205279970434e-06\\
4.6	5.8	7.93441867486995e-06	7.93441867486995e-06\\
4.6	5.9	9.45931682696602e-06	9.45931682696602e-06\\
4.6	6	1.24785774570456e-05	1.24785774570456e-05\\
4.7	0	7.42357989199338e-05	7.42357989199338e-05\\
4.7	0.1	4.42758287280735e-05	4.42758287280735e-05\\
4.7	0.2	2.74353213908192e-05	2.74353213908192e-05\\
4.7	0.3	1.84147023271036e-05	1.84147023271036e-05\\
4.7	0.4	1.37647861124629e-05	1.37647861124629e-05\\
4.7	0.5	1.14867501767118e-05	1.14867501767118e-05\\
4.7	0.6	1.05278479742016e-05	1.05278479742016e-05\\
4.7	0.7	1.03779033274448e-05	1.03779033274448e-05\\
4.7	0.8	1.07726724864393e-05	1.07726724864393e-05\\
4.7	0.9	1.14707542241367e-05	1.14707542241367e-05\\
4.7	1	1.21430635636273e-05	1.21430635636273e-05\\
4.7	1.1	1.24262618099546e-05	1.24262618099546e-05\\
4.7	1.2	1.20663059681458e-05	1.20663059681458e-05\\
4.7	1.3	1.09861317266831e-05	1.09861317266831e-05\\
4.7	1.4	9.31401438015023e-06	9.31401438015023e-06\\
4.7	1.5	7.55435311464504e-06	7.55435311464504e-06\\
4.7	1.6	6.84092453660904e-06	6.84092453660904e-06\\
4.7	1.7	1.31410491211354e-05	1.31410491211354e-05\\
4.7	1.8	8.17849957035872e-05	8.17849957035872e-05\\
4.7	1.9	0.000140083714993332	0.000140083714993332\\
4.7	2	0.000126765109927657	0.000126765109927657\\
4.7	2.1	5.46235612223748e-05	5.46235612223748e-05\\
4.7	2.2	6.12211941126134e-05	6.12211941126134e-05\\
4.7	2.3	9.80071949242105e-05	9.80071949242105e-05\\
4.7	2.4	0.000201028666042923	0.000201028666042923\\
4.7	2.5	0.000119162726106429	0.000119162726106429\\
4.7	2.6	0.000160357697158634	0.000160357697158634\\
4.7	2.7	9.92103384318593e-06	9.92103384318593e-06\\
4.7	2.8	7.08751440225698e-06	7.08751440225698e-06\\
4.7	2.9	9.63732957980018e-06	9.63732957980018e-06\\
4.7	3	1.97224695091501e-05	1.97224695091501e-05\\
4.7	3.1	1.67751878080103e-05	1.67751878080103e-05\\
4.7	3.2	1.53259087424457e-05	1.53259087424457e-05\\
4.7	3.3	1.70691659780919e-05	1.70691659780919e-05\\
4.7	3.4	2.94496501565348e-05	2.94496501565348e-05\\
4.7	3.5	0.000113854558510612	0.000113854558510612\\
4.7	3.6	0.000823564159278366	0.000823564159278366\\
4.7	3.7	0.00209479468342828	0.00209479468342828\\
4.7	3.8	0.000182732507922388	0.000182732507922388\\
4.7	3.9	1.30479070500816e-05	1.30479070500816e-05\\
4.7	4	3.82676990492321e-06	3.82676990492321e-06\\
4.7	4.1	3.05302060541849e-06	3.05302060541849e-06\\
4.7	4.2	3.17272474013165e-06	3.17272474013165e-06\\
4.7	4.3	3.61592662943391e-06	3.61592662943391e-06\\
4.7	4.4	4.37585133737812e-06	4.37585133737812e-06\\
4.7	4.5	5.53450193350751e-06	5.53450193350751e-06\\
4.7	4.6	7.16634566792073e-06	7.16634566792073e-06\\
4.7	4.7	9.23594955291423e-06	9.23594955291423e-06\\
4.7	4.8	1.14891006821653e-05	1.14891006821653e-05\\
4.7	4.9	1.34555140700613e-05	1.34555140700613e-05\\
4.7	5	1.46422597541997e-05	1.46422597541997e-05\\
4.7	5.1	1.47931035142481e-05	1.47931035142481e-05\\
4.7	5.2	1.39927457076002e-05	1.39927457076002e-05\\
4.7	5.3	1.25651078297929e-05	1.25651078297929e-05\\
4.7	5.4	1.09002339503903e-05	1.09002339503903e-05\\
4.7	5.5	9.33752676673106e-06	9.33752676673106e-06\\
4.7	5.6	8.12679835916277e-06	8.12679835916277e-06\\
4.7	5.7	7.43460857281302e-06	7.43460857281302e-06\\
4.7	5.8	7.38168942776012e-06	7.38168942776012e-06\\
4.7	5.9	8.11192484475188e-06	8.11192484475188e-06\\
4.7	6	9.88550496540659e-06	9.88550496540659e-06\\
4.8	0	5.65138219662202e-05	5.65138219662202e-05\\
4.8	0.1	3.39099842442359e-05	3.39099842442359e-05\\
4.8	0.2	2.16823548074991e-05	2.16823548074991e-05\\
4.8	0.3	1.52918788208418e-05	1.52918788208418e-05\\
4.8	0.4	1.20824098724505e-05	1.20824098724505e-05\\
4.8	0.5	1.06332214400178e-05	1.06332214400178e-05\\
4.8	0.6	1.02344796112433e-05	1.02344796112433e-05\\
4.8	0.7	1.05471479713483e-05	1.05471479713483e-05\\
4.8	0.8	1.13858673577924e-05	1.13858673577924e-05\\
4.8	0.9	1.25734318057467e-05	1.25734318057467e-05\\
4.8	1	1.38738262534498e-05	1.38738262534498e-05\\
4.8	1.1	1.50271183699833e-05	1.50271183699833e-05\\
4.8	1.2	1.58329354048576e-05	1.58329354048576e-05\\
4.8	1.3	1.61380709900007e-05	1.61380709900007e-05\\
4.8	1.4	1.56677384859728e-05	1.56677384859728e-05\\
4.8	1.5	1.39501297465203e-05	1.39501297465203e-05\\
4.8	1.6	1.09837297671703e-05	1.09837297671703e-05\\
4.8	1.7	9.73452374655313e-06	9.73452374655313e-06\\
4.8	1.8	5.44258337397483e-05	5.44258337397483e-05\\
4.8	1.9	0.000271662606754684	0.000271662606754684\\
4.8	2	6.47665762194255e-05	6.47665762194255e-05\\
4.8	2.1	2.28641960135888e-05	2.28641960135888e-05\\
4.8	2.2	3.41332363391055e-05	3.41332363391055e-05\\
4.8	2.3	6.35145913560665e-05	6.35145913560665e-05\\
4.8	2.4	0.000267793768426311	0.000267793768426311\\
4.8	2.5	0.00034430093388442	0.00034430093388442\\
4.8	2.6	0.00017614295183769	0.00017614295183769\\
4.8	2.7	1.38400083179708e-05	1.38400083179708e-05\\
4.8	2.8	2.48005276072435e-05	2.48005276072435e-05\\
4.8	2.9	2.14393438847198e-05	2.14393438847198e-05\\
4.8	3	4.00134457022958e-05	4.00134457022958e-05\\
4.8	3.1	3.14656645167323e-05	3.14656645167323e-05\\
4.8	3.2	2.41212180978878e-05	2.41212180978878e-05\\
4.8	3.3	1.88445764610385e-05	1.88445764610385e-05\\
4.8	3.4	1.74718172648959e-05	1.74718172648959e-05\\
4.8	3.5	3.26048768107718e-05	3.26048768107718e-05\\
4.8	3.6	0.00033722091072496	0.00033722091072496\\
4.8	3.7	0.00255732919370024	0.00255732919370024\\
4.8	3.8	6.81700958172503e-05	6.81700958172503e-05\\
4.8	3.9	5.40256469709649e-06	5.40256469709649e-06\\
4.8	4	4.55450687193384e-06	4.55450687193384e-06\\
4.8	4.1	5.3126167930491e-06	5.3126167930491e-06\\
4.8	4.2	6.05004098572411e-06	6.05004098572411e-06\\
4.8	4.3	6.82334905532217e-06	6.82334905532217e-06\\
4.8	4.4	7.78745370807941e-06	7.78745370807941e-06\\
4.8	4.5	9.04019278282015e-06	9.04019278282015e-06\\
4.8	4.6	1.05937997102162e-05	1.05937997102162e-05\\
4.8	4.7	1.23285106136763e-05	1.23285106136763e-05\\
4.8	4.8	1.39696067090372e-05	1.39696067090372e-05\\
4.8	4.9	1.51572140865091e-05	1.51572140865091e-05\\
4.8	5	1.56034566066793e-05	1.56034566066793e-05\\
4.8	5.1	1.5227800561383e-05	1.5227800561383e-05\\
4.8	5.2	1.41689982432287e-05	1.41689982432287e-05\\
4.8	5.3	1.26896619266371e-05	1.26896619266371e-05\\
4.8	5.4	1.10662758976481e-05	1.10662758976481e-05\\
4.8	5.5	9.52960540042065e-06	9.52960540042065e-06\\
4.8	5.6	8.25534083114896e-06	8.25534083114896e-06\\
4.8	5.7	7.37405413187187e-06	7.37405413187187e-06\\
4.8	5.8	6.98650347781384e-06	6.98650347781384e-06\\
4.8	5.9	7.1944950845456e-06	7.1944950845456e-06\\
4.8	6	8.15518862896428e-06	8.15518862896428e-06\\
4.9	0	4.23941791118937e-05	4.23941791118937e-05\\
4.9	0.1	2.59398679857757e-05	2.59398679857757e-05\\
4.9	0.2	1.72408473929965e-05	1.72408473929965e-05\\
4.9	0.3	1.27608804917848e-05	1.27608804917848e-05\\
4.9	0.4	1.05799941926879e-05	1.05799941926879e-05\\
4.9	0.5	9.72003577660607e-06	9.72003577660607e-06\\
4.9	0.6	9.70479065813289e-06	9.70479065813289e-06\\
4.9	0.7	1.03074037078745e-05	1.03074037078745e-05\\
4.9	0.8	1.14042628191173e-05	1.14042628191173e-05\\
4.9	0.9	1.28885442207574e-05	1.28885442207574e-05\\
4.9	1	1.46406500152989e-05	1.46406500152989e-05\\
4.9	1.1	1.65676836227356e-05	1.65676836227356e-05\\
4.9	1.2	1.8686356755739e-05	1.8686356755739e-05\\
4.9	1.3	2.11605153107354e-05	2.11605153107354e-05\\
4.9	1.4	2.41796338377909e-05	2.41796338377909e-05\\
4.9	1.5	2.75385726434826e-05	2.75385726434826e-05\\
4.9	1.6	2.94462550061384e-05	2.94462550061384e-05\\
4.9	1.7	2.3658503527373e-05	2.3658503527373e-05\\
4.9	1.8	1.98844186672329e-05	1.98844186672329e-05\\
4.9	1.9	0.000534299558459082	0.000534299558459082\\
4.9	2	1.71486736879545e-05	1.71486736879545e-05\\
4.9	2.1	1.13100743600602e-05	1.13100743600602e-05\\
4.9	2.2	1.91122838156934e-05	1.91122838156934e-05\\
4.9	2.3	3.92015709977315e-05	3.92015709977315e-05\\
4.9	2.4	0.000117580218517791	0.000117580218517791\\
4.9	2.5	0.000927460533707857	0.000927460533707857\\
4.9	2.6	2.40321080716146e-05	2.40321080716146e-05\\
4.9	2.7	5.66623442663722e-05	5.66623442663722e-05\\
4.9	2.8	5.27237755570061e-05	5.27237755570061e-05\\
4.9	2.9	3.84842148525038e-05	3.84842148525038e-05\\
4.9	3	8.18152036405593e-05	8.18152036405593e-05\\
4.9	3.1	6.31214720216675e-05	6.31214720216675e-05\\
4.9	3.2	4.76436005842742e-05	4.76436005842742e-05\\
4.9	3.3	3.48679181543517e-05	3.48679181543517e-05\\
4.9	3.4	2.43041198305443e-05	2.43041198305443e-05\\
4.9	3.5	1.69559966735373e-05	1.69559966735373e-05\\
4.9	3.6	5.11761736317623e-05	5.11761736317623e-05\\
4.9	3.7	0.00298206103737139	0.00298206103737139\\
4.9	3.8	1.09500128371076e-05	1.09500128371076e-05\\
4.9	3.9	8.52449520081074e-06	8.52449520081074e-06\\
4.9	4	1.11780678019142e-05	1.11780678019142e-05\\
4.9	4.1	1.16029914207254e-05	1.16029914207254e-05\\
4.9	4.2	1.15541954246708e-05	1.15541954246708e-05\\
4.9	4.3	1.16571313302556e-05	1.16571313302556e-05\\
4.9	4.4	1.20554948613831e-05	1.20554948613831e-05\\
4.9	4.5	1.27454531756259e-05	1.27454531756259e-05\\
4.9	4.6	1.36434652547771e-05	1.36434652547771e-05\\
4.9	4.7	1.45833556161157e-05	1.45833556161157e-05\\
4.9	4.8	1.53328533268402e-05	1.53328533268402e-05\\
4.9	4.9	1.56620714872541e-05	1.56620714872541e-05\\
4.9	5	1.54352642071377e-05	1.54352642071377e-05\\
4.9	5.1	1.46621500538048e-05	1.46621500538048e-05\\
4.9	5.2	1.34752885102764e-05	1.34752885102764e-05\\
4.9	5.3	1.20610798233407e-05	1.20610798233407e-05\\
4.9	5.4	1.05965548638785e-05	1.05965548638785e-05\\
4.9	5.5	9.22162659369463e-06	9.22162659369463e-06\\
4.9	5.6	8.04270882857089e-06	8.04270882857089e-06\\
4.9	5.7	7.14658027812538e-06	7.14658027812538e-06\\
4.9	5.8	6.61043536453786e-06	6.61043536453786e-06\\
4.9	5.9	6.51303253743646e-06	6.51303253743646e-06\\
4.9	6	6.95979298140966e-06	6.95979298140966e-06\\
5	0	3.16854391433244e-05	3.16854391433244e-05\\
5	0.1	1.99685101913556e-05	1.99685101913556e-05\\
5	0.2	1.38413732831732e-05	1.38413732831732e-05\\
5	0.3	1.07156266007747e-05	1.07156266007747e-05\\
5	0.4	9.24965267191012e-06	9.24965267191012e-06\\
5	0.5	8.7745504222773e-06	8.7745504222773e-06\\
5	0.6	8.96472954709583e-06	8.96472954709583e-06\\
5	0.7	9.66326608390772e-06	9.66326608390772e-06\\
5	0.8	1.07894722925981e-05	1.07894722925981e-05\\
5	0.9	1.22923661489478e-05	1.22923661489478e-05\\
5	1	1.41443665830359e-05	1.41443665830359e-05\\
5	1.1	1.63830605763014e-05	1.63830605763014e-05\\
5	1.2	1.9197462423852e-05	1.9197462423852e-05\\
5	1.3	2.30307065109116e-05	2.30307065109116e-05\\
5	1.4	2.86694012464022e-05	2.86694012464022e-05\\
5	1.5	3.73332956487479e-05	3.73332956487479e-05\\
5	1.6	5.08852709660893e-05	5.08852709660893e-05\\
5	1.7	7.25726731935055e-05	7.25726731935055e-05\\
5	1.8	0.000109448504588977	0.000109448504588977\\
5	1.9	0.000771733567041433	0.000771733567041433\\
5	2	1.71978720646392e-05	1.71978720646392e-05\\
5	2.1	2.19877049020589e-05	2.19877049020589e-05\\
5	2.2	3.08610172750202e-05	3.08610172750202e-05\\
5	2.3	4.61351094296292e-05	4.61351094296292e-05\\
5	2.4	6.99272878900565e-05	6.99272878900565e-05\\
5	2.5	0.00142087423318453	0.00142087423318453\\
5	2.6	0.000114834732960234	0.000114834732960234\\
5	2.7	8.92227289059148e-05	8.92227289059148e-05\\
5	2.8	6.43104399287967e-05	6.43104399287967e-05\\
5	2.9	4.44645597959173e-05	4.44645597959173e-05\\
5	3	0.000106390505365362	0.000106390505365362\\
5	3.1	8.15923608696904e-05	8.15923608696904e-05\\
5	3.2	6.32970235168639e-05	6.32970235168639e-05\\
5	3.3	4.95242441885062e-05	4.95242441885062e-05\\
5	3.4	3.90313158489111e-05	3.90313158489111e-05\\
5	3.5	3.10934656679644e-05	3.10934656679644e-05\\
5	3.6	2.52313631671937e-05	2.52313631671937e-05\\
5	3.7	0.00324035043163794	0.00324035043163794\\
5	3.8	3.36034733217324e-05	3.36034733217324e-05\\
5	3.9	2.49882413090827e-05	2.49882413090827e-05\\
5	4	2.00373702566973e-05	2.00373702566973e-05\\
5	4.1	1.71720930710339e-05	1.71720930710339e-05\\
5	4.2	1.55644350565282e-05	1.55644350565282e-05\\
5	4.3	1.4751861159408e-05	1.4751861159408e-05\\
5	4.4	1.44581748285822e-05	1.44581748285822e-05\\
5	4.5	1.44953263247514e-05	1.44953263247514e-05\\
5	4.6	1.47007841233173e-05	1.47007841233173e-05\\
5	4.7	1.49062830905242e-05	1.49062830905242e-05\\
5	4.8	1.49446002500296e-05	1.49446002500296e-05\\
5	4.9	1.46868656674655e-05	1.46868656674655e-05\\
5	5	1.40818151802989e-05	1.40818151802989e-05\\
5	5.1	1.31660439996203e-05	1.31660439996203e-05\\
5	5.2	1.20380899882105e-05	1.20380899882105e-05\\
5	5.3	1.08159715376159e-05	1.08159715376159e-05\\
5	5.4	9.60356118732876e-06	9.60356118732876e-06\\
5	5.5	8.47828030970448e-06	8.47828030970448e-06\\
5	5.6	7.49720906089599e-06	7.49720906089599e-06\\
5	5.7	6.71021506388299e-06	6.71021506388299e-06\\
5	5.8	6.16970188770761e-06	6.16970188770761e-06\\
5	5.9	5.93594524700966e-06	5.93594524700966e-06\\
5	6	6.08639059887599e-06	6.08639059887599e-06\\
5.1	0	2.39084455980628e-05	2.39084455980628e-05\\
5.1	0.1	1.5628521389669e-05	1.5628521389669e-05\\
5.1	0.2	1.13056451677389e-05	1.13056451677389e-05\\
5.1	0.3	9.11166049243027e-06	9.11166049243027e-06\\
5.1	0.4	8.11956872376808e-06	8.11956872376808e-06\\
5.1	0.5	7.86695240759377e-06	7.86695240759377e-06\\
5.1	0.6	8.12494353919228e-06	8.12494353919228e-06\\
5.1	0.7	8.78293697087732e-06	8.78293697087732e-06\\
5.1	0.8	9.79224506957062e-06	9.79224506957062e-06\\
5.1	0.9	1.11409917667519e-05	1.11409917667519e-05\\
5.1	1	1.28533870179585e-05	1.28533870179585e-05\\
5.1	1.1	1.50171682103853e-05	1.50171682103853e-05\\
5.1	1.2	1.78368649587577e-05	1.78368649587577e-05\\
5.1	1.3	2.1681574424603e-05	2.1681574424603e-05\\
5.1	1.4	2.70305846232928e-05	2.70305846232928e-05\\
5.1	1.5	3.40190697120348e-05	3.40190697120348e-05\\
5.1	1.6	4.04821593096754e-05	4.04821593096754e-05\\
5.1	1.7	3.58823426139051e-05	3.58823426139051e-05\\
5.1	1.8	3.07725705855471e-05	3.07725705855471e-05\\
5.1	1.9	0.000710042407901095	0.000710042407901095\\
5.1	2	1.88752431150551e-05	1.88752431150551e-05\\
5.1	2.1	1.1015885700158e-05	1.1015885700158e-05\\
5.1	2.2	1.82704352052958e-05	1.82704352052958e-05\\
5.1	2.3	3.5832132971475e-05	3.5832132971475e-05\\
5.1	2.4	0.000116383597578761	0.000116383597578761\\
5.1	2.5	0.000984326747875897	0.000984326747875897\\
5.1	2.6	2.22249940006485e-05	2.22249940006485e-05\\
5.1	2.7	3.24836520197145e-05	3.24836520197145e-05\\
5.1	2.8	3.16286609372148e-05	3.16286609372148e-05\\
5.1	2.9	2.32921312089696e-05	2.32921312089696e-05\\
5.1	3	6.91149287071674e-05	6.91149287071674e-05\\
5.1	3.1	5.33484641656948e-05	5.33484641656948e-05\\
5.1	3.2	4.04555481810027e-05	4.04555481810027e-05\\
5.1	3.3	2.98405499232668e-05	2.98405499232668e-05\\
5.1	3.4	2.09306026963211e-05	2.09306026963211e-05\\
5.1	3.5	1.4513306182747e-05	1.4513306182747e-05\\
5.1	3.6	4.22078692158279e-05	4.22078692158279e-05\\
5.1	3.7	0.00323135092046627	0.00323135092046627\\
5.1	3.8	1.28648585434517e-05	1.28648585434517e-05\\
5.1	3.9	9.33088733777165e-06	9.33088733777165e-06\\
5.1	4	1.24747270112561e-05	1.24747270112561e-05\\
5.1	4.1	1.32353913266866e-05	1.32353913266866e-05\\
5.1	4.2	1.3267858580555e-05	1.3267858580555e-05\\
5.1	4.3	1.32109015497708e-05	1.32109015497708e-05\\
5.1	4.4	1.32107961566647e-05	1.32107961566647e-05\\
5.1	4.5	1.32587405035932e-05	1.32587405035932e-05\\
5.1	4.6	1.32931234695987e-05	1.32931234695987e-05\\
5.1	4.7	1.32335885914119e-05	1.32335885914119e-05\\
5.1	4.8	1.30050926170418e-05	1.30050926170418e-05\\
5.1	4.9	1.2562463347819e-05	1.2562463347819e-05\\
5.1	5	1.19063291674255e-05	1.19063291674255e-05\\
5.1	5.1	1.1081060630439e-05	1.1081060630439e-05\\
5.1	5.2	1.01566363532954e-05	1.01566363532954e-05\\
5.1	5.3	9.20582404373245e-06	9.20582404373245e-06\\
5.1	5.4	8.28798159891203e-06	8.28798159891203e-06\\
5.1	5.5	7.44431471111128e-06	7.44431471111128e-06\\
5.1	5.6	6.70309635261821e-06	6.70309635261821e-06\\
5.1	5.7	6.08973072883913e-06	6.08973072883913e-06\\
5.1	5.8	5.63579670280128e-06	5.63579670280128e-06\\
5.1	5.9	5.38415245129867e-06	5.38415245129867e-06\\
5.1	6	5.39290766483648e-06	5.39290766483648e-06\\
5.2	0	1.84737758294769e-05	1.84737758294769e-05\\
5.2	0.1	1.25778139966697e-05	1.25778139966697e-05\\
5.2	0.2	9.48123451956623e-06	9.48123451956623e-06\\
5.2	0.3	7.90931854474824e-06	7.90931854474824e-06\\
5.2	0.4	7.21791442381244e-06	7.21791442381244e-06\\
5.2	0.5	7.08220744549813e-06	7.08220744549813e-06\\
5.2	0.6	7.3408306823956e-06	7.3408306823956e-06\\
5.2	0.7	7.92153040540202e-06	7.92153040540202e-06\\
5.2	0.8	8.80579267624252e-06	8.80579267624252e-06\\
5.2	0.9	1.00111327183351e-05	1.00111327183351e-05\\
5.2	1	1.15837045416318e-05	1.15837045416318e-05\\
5.2	1.1	1.35988007145871e-05	1.35988007145871e-05\\
5.2	1.2	1.61522645468736e-05	1.61522645468736e-05\\
5.2	1.3	1.92734628567857e-05	1.92734628567857e-05\\
5.2	1.4	2.25927751498685e-05	2.25927751498685e-05\\
5.2	1.5	2.46425062578865e-05	2.46425062578865e-05\\
5.2	1.6	2.3219405540262e-05	2.3219405540262e-05\\
5.2	1.7	2.28617899028915e-05	2.28617899028915e-05\\
5.2	1.8	8.83246458144454e-05	8.83246458144454e-05\\
5.2	1.9	0.000486092352476769	0.000486092352476769\\
5.2	2	8.91002060968113e-05	8.91002060968113e-05\\
5.2	2.1	2.0301558590652e-05	2.0301558590652e-05\\
5.2	2.2	2.76340374930705e-05	2.76340374930705e-05\\
5.2	2.3	5.32379761904586e-05	5.32379761904586e-05\\
5.2	2.4	0.000178524844241392	0.000178524844241392\\
5.2	2.5	0.000398508742016548	0.000398508742016548\\
5.2	2.6	0.000156693757007058	0.000156693757007058\\
5.2	2.7	8.76488380513903e-06	8.76488380513903e-06\\
5.2	2.8	8.16389624924449e-06	8.16389624924449e-06\\
5.2	2.9	6.8874017855461e-06	6.8874017855461e-06\\
5.2	3	2.73370146817581e-05	2.73370146817581e-05\\
5.2	3.1	2.16409507356565e-05	2.16409507356565e-05\\
5.2	3.2	1.66069570554488e-05	1.66069570554488e-05\\
5.2	3.3	1.29170854854256e-05	1.29170854854256e-05\\
5.2	3.4	1.18939122782483e-05	1.18939122782483e-05\\
5.2	3.5	2.1919197597509e-05	2.1919197597509e-05\\
5.2	3.6	0.000245848866714237	0.000245848866714237\\
5.2	3.7	0.00295513675688749	0.00295513675688749\\
5.2	3.8	8.74682135749908e-05	8.74682135749908e-05\\
5.2	3.9	7.54460490500178e-06	7.54460490500178e-06\\
5.2	4	6.11071044431857e-06	6.11071044431857e-06\\
5.2	4.1	7.35126362755525e-06	7.35126362755525e-06\\
5.2	4.2	8.49147082642191e-06	8.49147082642191e-06\\
5.2	4.3	9.32610007064166e-06	9.32610007064166e-06\\
5.2	4.4	9.92012043928284e-06	9.92012043928284e-06\\
5.2	4.5	1.03190838787906e-05	1.03190838787906e-05\\
5.2	4.6	1.05316558815644e-05	1.05316558815644e-05\\
5.2	4.7	1.05498978736574e-05	1.05498978736574e-05\\
5.2	4.8	1.0369133891677e-05	1.0369133891677e-05\\
5.2	4.9	1.00016078243455e-05	1.00016078243455e-05\\
5.2	5	9.4808871795917e-06	9.4808871795917e-06\\
5.2	5.1	8.85656480497979e-06	8.85656480497979e-06\\
5.2	5.2	8.1831434905533e-06	8.1831434905533e-06\\
5.2	5.3	7.50904955072463e-06	7.50904955072463e-06\\
5.2	5.4	6.87013910698576e-06	6.87013910698576e-06\\
5.2	5.5	6.28899966577072e-06	6.28899966577072e-06\\
5.2	5.6	5.77913061413478e-06	5.77913061413478e-06\\
5.2	5.7	5.35199899378515e-06	5.35199899378515e-06\\
5.2	5.8	5.02439674546872e-06	5.02439674546872e-06\\
5.2	5.9	4.82386240754515e-06	4.82386240754515e-06\\
5.2	6	4.79209392498397e-06	4.79209392498397e-06\\
5.3	0	1.48106438425532e-05	1.48106438425532e-05\\
5.3	0.1	1.05166799655153e-05	1.05166799655153e-05\\
5.3	0.2	8.23484433043173e-06	8.23484433043173e-06\\
5.3	0.3	7.07124494735317e-06	7.07124494735317e-06\\
5.3	0.4	6.57168610001928e-06	6.57168610001928e-06\\
5.3	0.5	6.50703006669146e-06	6.50703006669146e-06\\
5.3	0.6	6.76953780103699e-06	6.76953780103699e-06\\
5.3	0.7	7.32410252678135e-06	7.32410252678135e-06\\
5.3	0.8	8.18287645725135e-06	8.18287645725135e-06\\
5.3	0.9	9.38649250417637e-06	9.38649250417637e-06\\
5.3	1	1.09827260816128e-05	1.09827260816128e-05\\
5.3	1.1	1.29926562654391e-05	1.29926562654391e-05\\
5.3	1.2	1.53409526903993e-05	1.53409526903993e-05\\
5.3	1.3	1.77167446053911e-05	1.77167446053911e-05\\
5.3	1.4	1.94541067410436e-05	1.94541067410436e-05\\
5.3	1.5	2.00518679131501e-05	2.00518679131501e-05\\
5.3	1.6	2.1366755050827e-05	2.1366755050827e-05\\
5.3	1.7	3.58383251752458e-05	3.58383251752458e-05\\
5.3	1.8	0.000144388223986977	0.000144388223986977\\
5.3	1.9	0.000323602689342764	0.000323602689342764\\
5.3	2	0.000187978637589739	0.000187978637589739\\
5.3	2.1	6.36507945526841e-05	6.36507945526841e-05\\
5.3	2.2	4.47144554250047e-05	4.47144554250047e-05\\
5.3	2.3	5.17101167661555e-05	5.17101167661555e-05\\
5.3	2.4	0.000103820692564136	0.000103820692564136\\
5.3	2.5	0.000153933113230594	0.000153933113230594\\
5.3	2.6	0.000135491540338652	0.000135491540338652\\
5.3	2.7	1.71723266496861e-05	1.71723266496861e-05\\
5.3	2.8	3.54784189846024e-06	3.54784189846024e-06\\
5.3	2.9	2.29938968099449e-06	2.29938968099449e-06\\
5.3	3	1.09968504987816e-05	1.09968504987816e-05\\
5.3	3.1	9.18619836424371e-06	9.18619836424371e-06\\
5.3	3.2	8.24200105882593e-06	8.24200105882593e-06\\
5.3	3.3	9.03671828274805e-06	9.03671828274805e-06\\
5.3	3.4	1.53972898148492e-05	1.53972898148492e-05\\
5.3	3.5	5.83158642343766e-05	5.83158642343766e-05\\
5.3	3.6	0.000650626300392333	0.000650626300392333\\
5.3	3.7	0.00251986963502488	0.00251986963502488\\
5.3	3.8	0.000241162132232482	0.000241162132232482\\
5.3	3.9	2.21211907872027e-05	2.21211907872027e-05\\
5.3	4	6.62341589915927e-06	6.62341589915927e-06\\
5.3	4.1	5.30156605821115e-06	5.30156605821115e-06\\
5.3	4.2	5.70555971059801e-06	5.70555971059801e-06\\
5.3	4.3	6.39528617468909e-06	6.39528617468909e-06\\
5.3	4.4	7.04594849141693e-06	7.04594849141693e-06\\
5.3	4.5	7.55196132025354e-06	7.55196132025354e-06\\
5.3	4.6	7.86957419952043e-06	7.86957419952043e-06\\
5.3	4.7	7.984085813311e-06	7.984085813311e-06\\
5.3	4.8	7.90458874114232e-06	7.90458874114232e-06\\
5.3	4.9	7.66102096621037e-06	7.66102096621037e-06\\
5.3	5	7.29751148344293e-06	7.29751148344293e-06\\
5.3	5.1	6.86304246800833e-06	6.86304246800833e-06\\
5.3	5.2	6.40262606077985e-06	6.40262606077985e-06\\
5.3	5.3	5.9515321018991e-06	5.9515321018991e-06\\
5.3	5.4	5.53332503878224e-06	5.53332503878224e-06\\
5.3	5.5	5.16110828318027e-06	5.16110828318027e-06\\
5.3	5.6	4.84099392185091e-06	4.84099392185091e-06\\
5.3	5.7	4.57696631285367e-06	4.57696631285367e-06\\
5.3	5.8	4.37630030427872e-06	4.37630030427872e-06\\
5.3	5.9	4.25448926177556e-06	4.25448926177556e-06\\
5.3	6	4.23898844940393e-06	4.23898844940393e-06\\
5.4	0	1.2449239881868e-05	1.2449239881868e-05\\
5.4	0.1	9.21160638821928e-06	9.21160638821928e-06\\
5.4	0.2	7.46682533750183e-06	7.46682533750183e-06\\
5.4	0.3	6.57667405128163e-06	6.57667405128163e-06\\
5.4	0.4	6.21659343498743e-06	6.21659343498743e-06\\
5.4	0.5	6.22886451655757e-06	6.22886451655757e-06\\
5.4	0.6	6.55288151650782e-06	6.55288151650782e-06\\
5.4	0.7	7.19118067568306e-06	7.19118067568306e-06\\
5.4	0.8	8.18741441627812e-06	8.18741441627812e-06\\
5.4	0.9	9.60079105510992e-06	9.60079105510992e-06\\
5.4	1	1.14647012795694e-05	1.14647012795694e-05\\
5.4	1.1	1.37224335319754e-05	1.37224335319754e-05\\
5.4	1.2	1.61553111780439e-05	1.61553111780439e-05\\
5.4	1.3	1.83884422693288e-05	1.83884422693288e-05\\
5.4	1.4	2.01834525747779e-05	2.01834525747779e-05\\
5.4	1.5	2.22856838055539e-05	2.22856838055539e-05\\
5.4	1.6	2.91403729679906e-05	2.91403729679906e-05\\
5.4	1.7	6.46621212636675e-05	6.46621212636675e-05\\
5.4	1.8	0.000170470007841189	0.000170470007841189\\
5.4	1.9	0.000247233168659629	0.000247233168659629\\
5.4	2	0.000248024586146788	0.000248024586146788\\
5.4	2.1	0.000170474703005971	0.000170474703005971\\
5.4	2.2	9.37959514387161e-05	9.37959514387161e-05\\
5.4	2.3	8.66012807398517e-05	8.66012807398517e-05\\
5.4	2.4	7.93928035018532e-05	7.93928035018532e-05\\
5.4	2.5	7.48000019010336e-05	7.48000019010336e-05\\
5.4	2.6	6.48646410739213e-05	6.48646410739213e-05\\
5.4	2.7	2.73364393819667e-05	2.73364393819667e-05\\
5.4	2.8	4.7779025556563e-06	4.7779025556563e-06\\
5.4	2.9	1.63251888887085e-06	1.63251888887085e-06\\
5.4	3	7.01126953417241e-06	7.01126953417241e-06\\
5.4	3.1	6.29951275564833e-06	6.29951275564833e-06\\
5.4	3.2	7.07617187785337e-06	7.07617187785337e-06\\
5.4	3.3	1.11107230237372e-05	1.11107230237372e-05\\
5.4	3.4	2.80746360210899e-05	2.80746360210899e-05\\
5.4	3.5	0.0001295596521045	0.0001295596521045\\
5.4	3.6	0.000991058537790501	0.000991058537790501\\
5.4	3.7	0.00207241421543489	0.00207241421543489\\
5.4	3.8	0.000401221598521496	0.000401221598521496\\
5.4	3.9	5.91862502643597e-05	5.91862502643597e-05\\
5.4	4	1.35440260847075e-05	1.35440260847075e-05\\
5.4	4.1	6.48179123670138e-06	6.48179123670138e-06\\
5.4	4.2	5.21590824250009e-06	5.21590824250009e-06\\
5.4	4.3	5.19287747119611e-06	5.19287747119611e-06\\
5.4	4.4	5.48850246338382e-06	5.48850246338382e-06\\
5.4	4.5	5.81234497236831e-06	5.81234497236831e-06\\
5.4	4.6	6.04346052344655e-06	6.04346052344655e-06\\
5.4	4.7	6.13199479513824e-06	6.13199479513824e-06\\
5.4	4.8	6.07181380151028e-06	6.07181380151028e-06\\
5.4	4.9	5.88617368893162e-06	5.88617368893162e-06\\
5.4	5	5.61404760050701e-06	5.61404760050701e-06\\
5.4	5.1	5.29751894963653e-06	5.29751894963653e-06\\
5.4	5.2	4.97289509517383e-06	4.97289509517383e-06\\
5.4	5.3	4.66653188439705e-06	4.66653188439705e-06\\
5.4	5.4	4.39447137014971e-06	4.39447137014971e-06\\
5.4	5.5	4.16431678747669e-06	4.16431678747669e-06\\
5.4	5.6	3.97813280274544e-06	3.97813280274544e-06\\
5.4	5.7	3.83583665295147e-06	3.83583665295147e-06\\
5.4	5.8	3.73893506501835e-06	3.73893506501835e-06\\
5.4	5.9	3.69439069385804e-06	3.69439069385804e-06\\
5.4	6	3.71821714779739e-06	3.71821714779739e-06\\
5.5	0	1.10527539176883e-05	1.10527539176883e-05\\
5.5	0.1	8.50802598427954e-06	8.50802598427954e-06\\
5.5	0.2	7.12453924091874e-06	7.12453924091874e-06\\
5.5	0.3	6.43639618506597e-06	6.43639618506597e-06\\
5.5	0.4	6.21178139906948e-06	6.21178139906948e-06\\
5.5	0.5	6.35147528675134e-06	6.35147528675134e-06\\
5.5	0.6	6.84074717584775e-06	6.84074717584775e-06\\
5.5	0.7	7.7237454285625e-06	7.7237454285625e-06\\
5.5	0.8	9.07979340354767e-06	9.07979340354767e-06\\
5.5	0.9	1.09840133795836e-05	1.09840133795836e-05\\
5.5	1	1.34418737089606e-05	1.34418737089606e-05\\
5.5	1.1	1.63174725192463e-05	1.63174725192463e-05\\
5.5	1.2	1.93292998146001e-05	1.93292998146001e-05\\
5.5	1.3	2.22134845745373e-05	2.22134845745373e-05\\
5.5	1.4	2.5177780601242e-05	2.5177780601242e-05\\
5.5	1.5	3.03645499647173e-05	3.03645499647173e-05\\
5.5	1.6	4.79975453346132e-05	4.79975453346132e-05\\
5.5	1.7	0.000111119871972444	0.000111119871972444\\
5.5	1.8	0.000193094115174174	0.000193094115174174\\
5.5	1.9	0.000224509615317019	0.000224509615317019\\
5.5	2	0.000247577954381568	0.000247577954381568\\
5.5	2.1	0.000236920672329336	0.000236920672329336\\
5.5	2.2	0.000183839875529839	0.000183839875529839\\
5.5	2.3	0.00015023310622695	0.00015023310622695\\
5.5	2.4	7.42231583030815e-05	7.42231583030815e-05\\
5.5	2.5	4.62932649290922e-05	4.62932649290922e-05\\
5.5	2.6	3.78613277298094e-05	3.78613277298094e-05\\
5.5	2.7	2.4370370043279e-05	2.4370370043279e-05\\
5.5	2.8	7.50973902238038e-06	7.50973902238038e-06\\
5.5	2.9	2.57413652941168e-06	2.57413652941168e-06\\
5.5	3	8.10413730513715e-06	8.10413730513715e-06\\
5.5	3.1	7.51635543218047e-06	7.51635543218047e-06\\
5.5	3.2	9.74335933133548e-06	9.74335933133548e-06\\
5.5	3.3	1.84490913994288e-05	1.84490913994288e-05\\
5.5	3.4	5.29526442274709e-05	5.29526442274709e-05\\
5.5	3.5	0.000230993661347718	0.000230993661347718\\
5.5	3.6	0.00114873951100868	0.00114873951100868\\
5.5	3.7	0.00172316903998405	0.00172316903998405\\
5.5	3.8	0.000527661605247404	0.000527661605247404\\
5.5	3.9	0.000117109761403622	0.000117109761403622\\
5.5	4	3.02343873536577e-05	3.02343873536577e-05\\
5.5	4.1	1.14358253001773e-05	1.14358253001773e-05\\
5.5	4.2	6.78916351951328e-06	6.78916351951328e-06\\
5.5	4.3	5.48797821827279e-06	5.48797821827279e-06\\
5.5	4.4	5.15579341697715e-06	5.15579341697715e-06\\
5.5	4.5	5.11836099108969e-06	5.11836099108969e-06\\
5.5	4.6	5.12700584634661e-06	5.12700584634661e-06\\
5.5	4.7	5.08010162084287e-06	5.08010162084287e-06\\
5.5	4.8	4.9467568958313e-06	4.9467568958313e-06\\
5.5	4.9	4.7358253024306e-06	4.7358253024306e-06\\
5.5	5	4.47563881229588e-06	4.47563881229588e-06\\
5.5	5.1	4.19890095514434e-06	4.19890095514434e-06\\
5.5	5.2	3.93333951392675e-06	3.93333951392675e-06\\
5.5	5.3	3.69794491403751e-06	3.69794491403751e-06\\
5.5	5.4	3.50307911736567e-06	3.50307911736567e-06\\
5.5	5.5	3.3524281683279e-06	3.3524281683279e-06\\
5.5	5.6	3.24543563294318e-06	3.24543563294318e-06\\
5.5	5.7	3.17971612884628e-06	3.17971612884628e-06\\
5.5	5.8	3.15351188389943e-06	3.15351188389943e-06\\
5.5	5.9	3.16837911213059e-06	3.16837911213059e-06\\
5.5	6	3.23210719545686e-06	3.23210719545686e-06\\
5.6	0	1.0416264218837e-05	1.0416264218837e-05\\
5.6	0.1	8.33340371878085e-06	8.33340371878085e-06\\
5.6	0.2	7.21296110721919e-06	7.21296110721919e-06\\
5.6	0.3	6.70939566154581e-06	6.70939566154581e-06\\
5.6	0.4	6.66354185541907e-06	6.66354185541907e-06\\
5.6	0.5	7.03148861740216e-06	7.03148861740216e-06\\
5.6	0.6	7.85080837278087e-06	7.85080837278087e-06\\
5.6	0.7	9.21835793092518e-06	9.21835793092518e-06\\
5.6	0.8	1.12570562633227e-05	1.12570562633227e-05\\
5.6	0.9	1.40518662664563e-05	1.40518662664563e-05\\
5.6	1	1.7560032699066e-05	1.7560032699066e-05\\
5.6	1.1	2.15562291347711e-05	2.15562291347711e-05\\
5.6	1.2	2.57083237369764e-05	2.57083237369764e-05\\
5.6	1.3	2.99043477784988e-05	2.99043477784988e-05\\
5.6	1.4	3.53089992254012e-05	3.53089992254012e-05\\
5.6	1.5	4.77625292784829e-05	4.77625292784829e-05\\
5.6	1.6	8.70663731758819e-05	8.70663731758819e-05\\
5.6	1.7	0.000175567315593535	0.000175567315593535\\
5.6	1.8	0.000229116067080604	0.000229116067080604\\
5.6	1.9	0.000238474948359843	0.000238474948359843\\
5.6	2	0.000245245134736086	0.000245245134736086\\
5.6	2.1	0.000244676334489711	0.000244676334489711\\
5.6	2.2	0.000231082299671329	0.000231082299671329\\
5.6	2.3	0.00018135958829711	0.00018135958829711\\
5.6	2.4	7.28118359860943e-05	7.28118359860943e-05\\
5.6	2.5	3.7727191101084e-05	3.7727191101084e-05\\
5.6	2.6	2.95758436693233e-05	2.95758436693233e-05\\
5.6	2.7	2.22328311000043e-05	2.22328311000043e-05\\
5.6	2.8	1.04011713390927e-05	1.04011713390927e-05\\
5.6	2.9	5.25632934785911e-06	5.25632934785911e-06\\
5.6	3	1.4690295991195e-05	1.4690295991195e-05\\
5.6	3.1	1.36620576826373e-05	1.36620576826373e-05\\
5.6	3.2	1.82856533569035e-05	1.82856533569035e-05\\
5.6	3.3	3.58962331870764e-05	3.58962331870764e-05\\
5.6	3.4	9.95604374384997e-05	9.95604374384997e-05\\
5.6	3.5	0.000367677855850193	0.000367677855850193\\
5.6	3.6	0.00124172443640953	0.00124172443640953\\
5.6	3.7	0.00152548942904938	0.00152548942904938\\
5.6	3.8	0.00063072372602374	0.00063072372602374\\
5.6	3.9	0.000191846753661584	0.000191846753661584\\
5.6	4	6.024299140594e-05	6.024299140594e-05\\
5.6	4.1	2.24722949680615e-05	2.24722949680615e-05\\
5.6	4.2	1.1196920202439e-05	1.1196920202439e-05\\
5.6	4.3	7.41954153804952e-06	7.41954153804952e-06\\
5.6	4.4	5.98006761015307e-06	5.98006761015307e-06\\
5.6	4.5	5.35212322815982e-06	5.35212322815982e-06\\
5.6	4.6	5.00388741924905e-06	5.00388741924905e-06\\
5.6	4.7	4.7259829186972e-06	4.7259829186972e-06\\
5.6	4.8	4.44083576101807e-06	4.44083576101807e-06\\
5.6	4.9	4.13388838558688e-06	4.13388838558688e-06\\
5.6	5	3.81916383269624e-06	3.81916383269624e-06\\
5.6	5.1	3.51875891304087e-06	3.51875891304087e-06\\
5.6	5.2	3.25175632294924e-06	3.25175632294924e-06\\
5.6	5.3	3.03016297650425e-06	3.03016297650425e-06\\
5.6	5.4	2.85908098835984e-06	2.85908098835984e-06\\
5.6	5.5	2.73860656525155e-06	2.73860656525155e-06\\
5.6	5.6	2.66593305971865e-06	2.66593305971865e-06\\
5.6	5.7	2.63712497786823e-06	2.63712497786823e-06\\
5.6	5.8	2.6486610240196e-06	2.6486610240196e-06\\
5.6	5.9	2.69907735499818e-06	2.69907735499818e-06\\
5.6	6	2.79097365901438e-06	2.79097365901438e-06\\
5.7	0	1.04560907297035e-05	1.04560907297035e-05\\
5.7	0.1	8.70109721772399e-06	8.70109721772399e-06\\
5.7	0.2	7.81028513764344e-06	7.81028513764344e-06\\
5.7	0.3	7.52960976508413e-06	7.52960976508413e-06\\
5.7	0.4	7.76681710451337e-06	7.76681710451337e-06\\
5.7	0.5	8.54177165697194e-06	8.54177165697194e-06\\
5.7	0.6	9.96171462556307e-06	9.96171462556307e-06\\
5.7	0.7	1.21942446438622e-05	1.21942446438622e-05\\
5.7	0.8	1.54094184272439e-05	1.54094184272439e-05\\
5.7	0.9	1.96761608821611e-05	1.96761608821611e-05\\
5.7	1	2.4851986901191e-05	2.4851986901191e-05\\
5.7	1.1	3.05757919551374e-05	3.05757919551374e-05\\
5.7	1.2	3.65313932044005e-05	3.65313932044005e-05\\
5.7	1.3	4.33484974151343e-05	4.33484974151343e-05\\
5.7	1.4	5.49919576110317e-05	5.49919576110317e-05\\
5.7	1.5	8.45522900038478e-05	8.45522900038478e-05\\
5.7	1.6	0.000159686367347883	0.000159686367347883\\
5.7	1.7	0.000261535219553342	0.000261535219553342\\
5.7	1.8	0.000289759763410927	0.000289759763410927\\
5.7	1.9	0.000296422002250298	0.000296422002250298\\
5.7	2	0.000295680650257363	0.000295680650257363\\
5.7	2.1	0.000284765547045741	0.000284765547045741\\
5.7	2.2	0.000278669764083854	0.000278669764083854\\
5.7	2.3	0.00021048313420637	0.00021048313420637\\
5.7	2.4	8.92351510187671e-05	8.92351510187671e-05\\
5.7	2.5	4.52216538927799e-05	4.52216538927799e-05\\
5.7	2.6	3.31826579766356e-05	3.31826579766356e-05\\
5.7	2.7	2.65176120357731e-05	2.65176120357731e-05\\
5.7	2.8	1.61803761023787e-05	1.61803761023787e-05\\
5.7	2.9	1.11107654594036e-05	1.11107654594036e-05\\
5.7	3	3.35544491173684e-05	3.35544491173684e-05\\
5.7	3.1	3.29151370579182e-05	3.29151370579182e-05\\
5.7	3.2	4.35223982379358e-05	4.35223982379358e-05\\
5.7	3.3	8.01696123252865e-05	8.01696123252865e-05\\
5.7	3.4	0.000197004980364947	0.000197004980364947\\
5.7	3.5	0.000589839006531838	0.000589839006531838\\
5.7	3.6	0.00142049107915863	0.00142049107915863\\
5.7	3.7	0.00150406945756934	0.00150406945756934\\
5.7	3.8	0.000747875985816622	0.000747875985816622\\
5.7	3.9	0.000286649815698113	0.000286649815698113\\
5.7	4	0.000107523011551123	0.000107523011551123\\
5.7	4.1	4.31526980536789e-05	4.31526980536789e-05\\
5.7	4.2	2.03413770494287e-05	2.03413770494287e-05\\
5.7	4.3	1.18236877304611e-05	1.18236877304611e-05\\
5.7	4.4	8.28875268681604e-06	8.28875268681604e-06\\
5.7	4.5	6.60868042511669e-06	6.60868042511669e-06\\
5.7	4.6	5.66236908813662e-06	5.66236908813662e-06\\
5.7	4.7	5.01031859847498e-06	5.01031859847498e-06\\
5.7	4.8	4.47621676680447e-06	4.47621676680447e-06\\
5.7	4.9	3.99891777992471e-06	3.99891777992471e-06\\
5.7	5	3.56785561870141e-06	3.56785561870141e-06\\
5.7	5.1	3.19004895548312e-06	3.19004895548312e-06\\
5.7	5.2	2.87389200047587e-06	2.87389200047587e-06\\
5.7	5.3	2.62324956808524e-06	2.62324956808524e-06\\
5.7	5.4	2.43708387206528e-06	2.43708387206528e-06\\
5.7	5.5	2.31114727928628e-06	2.31114727928628e-06\\
5.7	5.6	2.23982694281615e-06	2.23982694281615e-06\\
5.7	5.7	2.21748476860161e-06	2.21748476860161e-06\\
5.7	5.8	2.23935909487533e-06	2.23935909487533e-06\\
5.7	5.9	2.30238097756242e-06	2.30238097756242e-06\\
5.7	6	2.40627444665441e-06	2.40627444665441e-06\\
5.8	0	1.12078850277527e-05	1.12078850277527e-05\\
5.8	0.1	9.72654028594492e-06	9.72654028594492e-06\\
5.8	0.2	9.09979457910661e-06	9.09979457910661e-06\\
5.8	0.3	9.15554758301614e-06	9.15554758301614e-06\\
5.8	0.4	9.87833679981264e-06	9.87833679981264e-06\\
5.8	0.5	1.13750840080739e-05	1.13750840080739e-05\\
5.8	0.6	1.38511416893249e-05	1.38511416893249e-05\\
5.8	0.7	1.7561242489896e-05	1.7561242489896e-05\\
5.8	0.8	2.27029620911513e-05	2.27029620911513e-05\\
5.8	0.9	2.92644225070037e-05	2.92644225070037e-05\\
5.8	1	3.69360188499409e-05	3.69360188499409e-05\\
5.8	1.1	4.53068138071054e-05	4.53068138071054e-05\\
5.8	1.2	5.4690047239461e-05	5.4690047239461e-05\\
5.8	1.3	6.80989086405053e-05	6.80989086405053e-05\\
5.8	1.4	9.54014228345615e-05	9.54014228345615e-05\\
5.8	1.5	0.000160388841043722	0.000160388841043722\\
5.8	1.6	0.000283211372783298	0.000283211372783298\\
5.8	1.7	0.00038350542262423	0.00038350542262423\\
5.8	1.8	0.000402003043696768	0.000402003043696768\\
5.8	1.9	0.000435152058653923	0.000435152058653923\\
5.8	2	0.00045210084997467	0.00045210084997467\\
5.8	2.1	0.000437099657455663	0.000437099657455663\\
5.8	2.2	0.000430547991946744	0.000430547991946744\\
5.8	2.3	0.00032516403639256	0.00032516403639256\\
5.8	2.4	0.000158455307748764	0.000158455307748764\\
5.8	2.5	8.37384678682099e-05	8.37384678682099e-05\\
5.8	2.6	5.64574780797689e-05	5.64574780797689e-05\\
5.8	2.7	4.46770830262713e-05	4.46770830262713e-05\\
5.8	2.8	3.18922479520504e-05	3.18922479520504e-05\\
5.8	2.9	2.61798684698937e-05	2.61798684698937e-05\\
5.8	3	8.61906639258501e-05	8.61906639258501e-05\\
5.8	3.1	9.35329971136434e-05	9.35329971136434e-05\\
5.8	3.2	0.000122483000106131	0.000122483000106131\\
5.8	3.3	0.000203359687773884	0.000203359687773884\\
5.8	3.4	0.000430566144926103	0.000430566144926103\\
5.8	3.5	0.00102760616853027	0.00102760616853027\\
5.8	3.6	0.00183552428424944	0.00183552428424944\\
5.8	3.7	0.00169975046616753	0.00169975046616753\\
5.8	3.8	0.000930278419053888	0.000930278419053888\\
5.8	3.9	0.000416608240301902	0.000416608240301902\\
5.8	4	0.000179449272385212	0.000179449272385212\\
5.8	4.1	7.87849306235119e-05	7.87849306235119e-05\\
5.8	4.2	3.74976883671761e-05	3.74976883671761e-05\\
5.8	4.3	2.04370813094746e-05	2.04370813094746e-05\\
5.8	4.4	1.29547472506008e-05	1.29547472506008e-05\\
5.8	4.5	9.32032486853456e-06	9.32032486853456e-06\\
5.8	4.6	7.30718870088657e-06	7.30718870088657e-06\\
5.8	4.7	6.01579679300864e-06	6.01579679300864e-06\\
5.8	4.8	5.06906764508372e-06	5.06906764508372e-06\\
5.8	4.9	4.31166255261684e-06	4.31166255261684e-06\\
5.8	5	3.68523021121859e-06	3.68523021121859e-06\\
5.8	5.1	3.17007479372541e-06	3.17007479372541e-06\\
5.8	5.2	2.75771218702352e-06	2.75771218702352e-06\\
5.8	5.3	2.44013416128854e-06	2.44013416128854e-06\\
5.8	5.4	2.20754733149375e-06	2.20754733149375e-06\\
5.8	5.5	2.04934342904074e-06	2.04934342904074e-06\\
5.8	5.6	1.9555431814416e-06	1.9555431814416e-06\\
5.8	5.7	1.91773396634445e-06	1.91773396634445e-06\\
5.8	5.8	1.92946453602076e-06	1.92946453602076e-06\\
5.8	5.9	1.98641377603491e-06	1.98641377603491e-06\\
5.8	6	2.08670943335738e-06	2.08670943335738e-06\\
5.9	0	1.28468919690085e-05	1.28468919690085e-05\\
5.9	0.1	1.16687223431094e-05	1.16687223431094e-05\\
5.9	0.2	1.14336651162864e-05	1.14336651162864e-05\\
5.9	0.3	1.20620527121848e-05	1.20620527121848e-05\\
5.9	0.4	1.36443095542592e-05	1.36443095542592e-05\\
5.9	0.5	1.64141275821149e-05	1.64141275821149e-05\\
5.9	0.6	2.07079479808162e-05	2.07079479808162e-05\\
5.9	0.7	2.68654517926217e-05	2.68654517926217e-05\\
5.9	0.8	3.50560703082637e-05	3.50560703082637e-05\\
5.9	0.9	4.51230323370951e-05	4.51230323370951e-05\\
5.9	1	5.66977774255508e-05	5.66977774255508e-05\\
5.9	1.1	6.99619308448729e-05	6.99619308448729e-05\\
5.9	1.2	8.7462433220099e-05	8.7462433220099e-05\\
5.9	1.3	0.000117589233225682	0.000117589233225682\\
5.9	1.4	0.000180993554207502	0.000180993554207502\\
5.9	1.5	0.000311077190589752	0.000311077190589752\\
5.9	1.6	0.00048802517824773	0.00048802517824773\\
5.9	1.7	0.000583796216883883	0.000583796216883883\\
5.9	1.8	0.000630498938385406	0.000630498938385406\\
5.9	1.9	0.000747890152017934	0.000747890152017934\\
5.9	2	0.000838821063541136	0.000838821063541136\\
5.9	2.1	0.0008669057744867	0.0008669057744867\\
5.9	2.2	0.000909323381015635	0.000909323381015635\\
5.9	2.3	0.000733879081391334	0.000733879081391334\\
5.9	2.4	0.000412874033164972	0.000412874033164972\\
5.9	2.5	0.000231893352969344	0.000231893352969344\\
5.9	2.6	0.000145911894912108	0.000145911894912108\\
5.9	2.7	0.000109359458700245	0.000109359458700245\\
5.9	2.8	8.3960333765632e-05	8.3960333765632e-05\\
5.9	2.9	7.55412364869867e-05	7.55412364869867e-05\\
5.9	3	0.000242103195576454	0.000242103195576454\\
5.9	3.1	0.000286335600600235	0.000286335600600235\\
5.9	3.2	0.0003741073871442	0.0003741073871442\\
5.9	3.3	0.000568298383833874	0.000568298383833874\\
5.9	3.4	0.00104338018930949	0.00104338018930949\\
5.9	3.5	0.00198541732227346	0.00198541732227346\\
5.9	3.6	0.00273060761272316	0.00273060761272316\\
5.9	3.7	0.00222160023710099	0.00222160023710099\\
5.9	3.8	0.00125314327429441	0.00125314327429441\\
5.9	3.9	0.000614240036959708	0.000614240036959708\\
5.9	4	0.000290995585907646	0.000290995585907646\\
5.9	4.1	0.000138166554229065	0.000138166554229065\\
5.9	4.2	6.80900500551619e-05	6.80900500551619e-05\\
5.9	4.3	3.6354057962026e-05	3.6354057962026e-05\\
5.9	4.4	2.16553269302036e-05	2.16553269302036e-05\\
5.9	4.5	1.43956219180779e-05	1.43956219180779e-05\\
5.9	4.6	1.04399730622828e-05	1.04399730622828e-05\\
5.9	4.7	8.02109955814089e-06	8.02109955814089e-06\\
5.9	4.8	6.37006555846683e-06	6.37006555846683e-06\\
5.9	4.9	5.14705925633973e-06	5.14705925633973e-06\\
5.9	5	4.20113500089836e-06	4.20113500089836e-06\\
5.9	5.1	3.46237755721715e-06	3.46237755721715e-06\\
5.9	5.2	2.89234359364212e-06	2.89234359364212e-06\\
5.9	5.3	2.46326315243254e-06	2.46326315243254e-06\\
5.9	5.4	2.15135361997054e-06	2.15135361997054e-06\\
5.9	5.5	1.93589470166297e-06	1.93589470166297e-06\\
5.9	5.6	1.79975091999334e-06	1.79975091999334e-06\\
5.9	5.7	1.72968440260714e-06	1.72968440260714e-06\\
5.9	5.8	1.71618144999129e-06	1.71618144999129e-06\\
5.9	5.9	1.75302499268988e-06	1.75302499268988e-06\\
5.9	6	1.83694138141645e-06	1.83694138141645e-06\\
6	0	1.5745423895692e-05	1.5745423895692e-05\\
6	0.1	1.50163150995393e-05	1.50163150995393e-05\\
6	0.2	1.54541185932481e-05	1.54541185932481e-05\\
6	0.3	1.71067380381771e-05	1.71067380381771e-05\\
6	0.4	2.02230226149875e-05	2.02230226149875e-05\\
6	0.5	2.52173788198985e-05	2.52173788198985e-05\\
6	0.6	3.25836756656364e-05	3.25836756656364e-05\\
6	0.7	4.27215081868807e-05	4.27215081868807e-05\\
6	0.8	5.57372264224589e-05	5.57372264224589e-05\\
6	0.9	7.14758350269073e-05	7.14758350269073e-05\\
6	1	9.02109069509908e-05	9.02109069509908e-05\\
6	1.1	0.000114415939711886	0.000114415939711886\\
6	1.2	0.000152075316212828	0.000152075316212828\\
6	1.3	0.0002227162557427	0.0002227162557427\\
6	1.4	0.000364277319554759	0.000364277319554759\\
6	1.5	0.000604474113245023	0.000604474113245023\\
6	1.6	0.000845751988094684	0.000845751988094684\\
6	1.7	0.00096758905686971	0.00096758905686971\\
6	1.8	0.00112846262696168	0.00112846262696168\\
6	1.9	0.00146401305271639	0.00146401305271639\\
6	2	0.00179092072123909	0.00179092072123909\\
6	2.1	0.00209878089915939	0.00209878089915939\\
6	2.2	0.00252855710920211	0.00252855710920211\\
6	2.3	0.00231591874648568	0.00231591874648568\\
6	2.4	0.00149096850323925	0.00149096850323925\\
6	2.5	0.000906715920758072	0.000906715920758072\\
6	2.6	0.000563058008691641	0.000563058008691641\\
6	2.7	0.000397683871723753	0.000397683871723753\\
6	2.8	0.000310678446800344	0.000310678446800344\\
6	2.9	0.000287040627780813	0.000287040627780813\\
6	3	0.000750704840495812	0.000750704840495812\\
6	3.1	0.000915216864476333	0.000915216864476333\\
6	3.2	0.0011872744278955	0.0011872744278955\\
6	3.3	0.00169470345043422	0.00169470345043422\\
6	3.4	0.00273287033350775	0.00273287033350775\\
6	3.5	0.00420555746212853	0.00420555746212853\\
6	3.6	0.00463029226465507	0.00463029226465507\\
6	3.7	0.00333222444613614	0.00333222444613614\\
6	3.8	0.00185228597035677	0.00185228597035677\\
6	3.9	0.000942748632045888	0.000942748632045888\\
6	4	0.000471646305710235	0.000471646305710235\\
6	4.1	0.000236936188386943	0.000236936188386943\\
6	4.2	0.000121320837042346	0.000121320837042346\\
6	4.3	6.49480585412338e-05	6.49480585412338e-05\\
6	4.4	3.74039764557897e-05	3.74039764557897e-05\\
6	4.5	2.35062963409883e-05	2.35062963409883e-05\\
6	4.6	1.60086029196684e-05	1.60086029196684e-05\\
6	4.7	1.15812288244081e-05	1.15812288244081e-05\\
6	4.8	8.71125382764848e-06	8.71125382764848e-06\\
6	4.9	6.7040164824822e-06	6.7040164824822e-06\\
6	5	5.2320874766181e-06	5.2320874766181e-06\\
6	5.1	4.13166149271941e-06	4.13166149271941e-06\\
6	5.2	3.30996934754908e-06	3.30996934754908e-06\\
6	5.3	2.70485800926476e-06	2.70485800926476e-06\\
6	5.4	2.2691292937712e-06	2.2691292937712e-06\\
6	5.5	1.96550292776703e-06	1.96550292776703e-06\\
6	5.6	1.76511059065104e-06	1.76511059065104e-06\\
6	5.7	1.64659356326982e-06	1.64659356326982e-06\\
6	5.8	1.59503174879498e-06	1.59503174879498e-06\\
6	5.9	1.60076351011499e-06	1.60076351011499e-06\\
6	6	1.65833978452825e-06	1.65833978452825e-06\\
};
\end{axis}
\end{tikzpicture}%
		}
	\caption[Replication Results with Rotated Initialisation]{Replication Results with Rotated Initialisation: \itshape The initialisation of $\psips$ is rotated by $90^{\circ}$ compared to \cite{Schwartz2014}, which leads to both sources being located in the non-zero area of $\psi_{1\bm p}^{\text(0)bf}$.}
	\label{fig:resultsReplicationAlternativeRotated}
\end{figure}

\begin{figure}[!htbp]
\centering
    \iftoggle{quick}{%
		\includegraphics[width=\textwidth]{plots/schwartz2014-variation/s=2-sloc=schwartz2014-T60=0.7-prior=equal-results}
		}{%
			\setlength{\figurewidth}{\textwidth}
    % This file was created by matlab2tikz.
%
\begin{tikzpicture}

\begin{axis}[%
width=4.462in,
height=4.075in,
at={(1.733in,0.55in)},
scale only axis,
axis on top,
xmin=0,
xmax=6,
ymin=0,
ymax=6,
axis background/.style={fill=white},
axis x line*=bottom,
axis y line*=left
]
\addplot [forget plot] graphics [xmin=-0.05, xmax=6.05, ymin=-0.05, ymax=6.05] {s=2-sloc=schwartz2014-T60=0.7-prior=equal-results-1.png};
\addplot [color=green, line width=2.0pt, draw=none, mark size=6.0pt, mark=x, mark options={solid, green}, forget plot]
  table[row sep=crcr]{%
1.8	1\\
2	1\\
2.4	1\\
2.6	1\\
3.6	1\\
3.8	1\\
5	1.8\\
5	2\\
5	2.4\\
5	2.6\\
5	3.6\\
5	3.8\\
1.8	5\\
2	5\\
2.4	5\\
2.6	5\\
3.6	5\\
3.8	5\\
1	1.8\\
1	2\\
1	2.4\\
1	2.6\\
1	3.6\\
1	3.8\\
};
\addplot [color=white, line width=2.0pt, draw=none, mark size=8.0pt, mark=x, mark options={solid, white}, forget plot]
  table[row sep=crcr]{%
2.6	2.3\\
3.4	2.3\\
};
\addplot [color=red, line width=2.0pt, draw=none, mark size=8.0pt, mark=x, mark options={solid, red}, forget plot]
  table[row sep=crcr]{%
2.6	2.3\\
3.4	2.3\\
};
\end{axis}

\begin{axis}[%
width=5.867in,
height=4.6in,
at={(6.933in,0.2in)},
scale only axis,
xmin=0,
xmax=6,
tick align=outside,
ymin=0,
ymax=6,
zmin=0,
zmax=0.02,
view={-65}{25},
axis background/.style={fill=white},
axis x line*=bottom,
axis y line*=left,
axis z line*=left,
xmajorgrids,
ymajorgrids,
zmajorgrids
]

\addplot3[%
surf,
shader=interp, colormap={mymap}{[1pt] rgb(0pt)=(0.2422,0.1504,0.6603); rgb(1pt)=(0.25039,0.164995,0.707614); rgb(2pt)=(0.257771,0.181781,0.751138); rgb(3pt)=(0.264729,0.197757,0.795214); rgb(4pt)=(0.270648,0.214676,0.836371); rgb(5pt)=(0.275114,0.234238,0.870986); rgb(6pt)=(0.2783,0.255871,0.899071); rgb(7pt)=(0.280333,0.278233,0.9221); rgb(8pt)=(0.281338,0.300595,0.941376); rgb(9pt)=(0.281014,0.322757,0.957886); rgb(10pt)=(0.279467,0.344671,0.971676); rgb(11pt)=(0.275971,0.366681,0.982905); rgb(12pt)=(0.269914,0.3892,0.9906); rgb(13pt)=(0.260243,0.412329,0.995157); rgb(14pt)=(0.244033,0.435833,0.998833); rgb(15pt)=(0.220643,0.460257,0.997286); rgb(16pt)=(0.196333,0.484719,0.989152); rgb(17pt)=(0.183405,0.507371,0.979795); rgb(18pt)=(0.178643,0.528857,0.968157); rgb(19pt)=(0.176438,0.549905,0.952019); rgb(20pt)=(0.168743,0.570262,0.935871); rgb(21pt)=(0.154,0.5902,0.9218); rgb(22pt)=(0.146029,0.609119,0.907857); rgb(23pt)=(0.138024,0.627629,0.89729); rgb(24pt)=(0.124814,0.645929,0.888343); rgb(25pt)=(0.111252,0.6635,0.876314); rgb(26pt)=(0.0952095,0.679829,0.859781); rgb(27pt)=(0.0688714,0.694771,0.839357); rgb(28pt)=(0.0296667,0.708167,0.816333); rgb(29pt)=(0.00357143,0.720267,0.7917); rgb(30pt)=(0.00665714,0.731214,0.766014); rgb(31pt)=(0.0433286,0.741095,0.73941); rgb(32pt)=(0.0963952,0.75,0.712038); rgb(33pt)=(0.140771,0.7584,0.684157); rgb(34pt)=(0.1717,0.766962,0.655443); rgb(35pt)=(0.193767,0.775767,0.6251); rgb(36pt)=(0.216086,0.7843,0.5923); rgb(37pt)=(0.246957,0.791795,0.556743); rgb(38pt)=(0.290614,0.79729,0.518829); rgb(39pt)=(0.340643,0.8008,0.478857); rgb(40pt)=(0.3909,0.802871,0.435448); rgb(41pt)=(0.445629,0.802419,0.390919); rgb(42pt)=(0.5044,0.7993,0.348); rgb(43pt)=(0.561562,0.794233,0.304481); rgb(44pt)=(0.617395,0.787619,0.261238); rgb(45pt)=(0.671986,0.779271,0.2227); rgb(46pt)=(0.7242,0.769843,0.191029); rgb(47pt)=(0.773833,0.759805,0.16461); rgb(48pt)=(0.820314,0.749814,0.153529); rgb(49pt)=(0.863433,0.7406,0.159633); rgb(50pt)=(0.903543,0.733029,0.177414); rgb(51pt)=(0.939257,0.728786,0.209957); rgb(52pt)=(0.972757,0.729771,0.239443); rgb(53pt)=(0.995648,0.743371,0.237148); rgb(54pt)=(0.996986,0.765857,0.219943); rgb(55pt)=(0.995205,0.789252,0.202762); rgb(56pt)=(0.9892,0.813567,0.188533); rgb(57pt)=(0.978629,0.838629,0.176557); rgb(58pt)=(0.967648,0.8639,0.16429); rgb(59pt)=(0.96101,0.889019,0.153676); rgb(60pt)=(0.959671,0.913457,0.142257); rgb(61pt)=(0.962795,0.937338,0.12651); rgb(62pt)=(0.969114,0.960629,0.106362); rgb(63pt)=(0.9769,0.9839,0.0805)}, mesh/rows=61]
table[row sep=crcr, point meta=\thisrow{c}] {%
%
x	y	z	c\\
0	0	6.44574199242952e-06	6.44574199242952e-06\\
0	0.1	6.47654026504991e-06	6.47654026504991e-06\\
0	0.2	6.98391232825092e-06	6.98391232825092e-06\\
0	0.3	8.02071017730981e-06	8.02071017730981e-06\\
0	0.4	9.71498153306488e-06	9.71498153306488e-06\\
0	0.5	1.22792756976916e-05	1.22792756976916e-05\\
0	0.6	1.60278748166934e-05	1.60278748166934e-05\\
0	0.7	2.13967321686285e-05	2.13967321686285e-05\\
0	0.8	2.89556180246614e-05	2.89556180246614e-05\\
0	0.9	3.94107661869683e-05	3.94107661869683e-05\\
0	1	5.36795199087587e-05	5.36795199087587e-05\\
0	1.1	7.33346183634529e-05	7.33346183634529e-05\\
0	1.2	0.000101803218613612	0.000101803218613612\\
0	1.3	0.000145843963055906	0.000145843963055906\\
0	1.4	0.000215266439554825	0.000215266439554825\\
0	1.5	0.000321075750395782	0.000321075750395782\\
0	1.6	0.000478998559556697	0.000478998559556697\\
0	1.7	0.000708314732202147	0.000708314732202147\\
0	1.8	0.0010074700451199	0.0010074700451199\\
0	1.9	0.00139948312589765	0.00139948312589765\\
0	2	0.00216424513310822	0.00216424513310822\\
0	2.1	0.00385510385843252	0.00385510385843252\\
0	2.2	0.00546197172865419	0.00546197172865419\\
0	2.3	0.00458888046408777	0.00458888046408777\\
0	2.4	0.00258274408025563	0.00258274408025563\\
0	2.5	0.00126763761886765	0.00126763761886765\\
0	2.6	0.000715308067608353	0.000715308067608353\\
0	2.7	0.000492486992590288	0.000492486992590288\\
0	2.8	0.000382814315733296	0.000382814315733296\\
0	2.9	0.000340954158179165	0.000340954158179165\\
0	3	0.000390020502629002	0.000390020502629002\\
0	3.1	0.000575679566118221	0.000575679566118221\\
0	3.2	0.000929757005771144	0.000929757005771144\\
0	3.3	0.00141836763948231	0.00141836763948231\\
0	3.4	0.00198193834844763	0.00198193834844763\\
0	3.5	0.00258787523321531	0.00258787523321531\\
0	3.6	0.00305180886235656	0.00305180886235656\\
0	3.7	0.00293143349375578	0.00293143349375578\\
0	3.8	0.00217112711186431	0.00217112711186431\\
0	3.9	0.00127371905592845	0.00127371905592845\\
0	4	0.000626837715122521	0.000626837715122521\\
0	4.1	0.000282869437162029	0.000282869437162029\\
0	4.2	0.000132813433668158	0.000132813433668158\\
0	4.3	7.22998238504401e-05	7.22998238504401e-05\\
0	4.4	4.6822098851535e-05	4.6822098851535e-05\\
0	4.5	3.5380789371579e-05	3.5380789371579e-05\\
0	4.6	3.03019725961725e-05	3.03019725961725e-05\\
0	4.7	2.81092956983843e-05	2.81092956983843e-05\\
0	4.8	2.6692091629517e-05	2.6692091629517e-05\\
0	4.9	2.47457809303866e-05	2.47457809303866e-05\\
0	5	2.18318552923012e-05	2.18318552923012e-05\\
0	5.1	1.82527487561644e-05	1.82527487561644e-05\\
0	5.2	1.46102519093791e-05	1.46102519093791e-05\\
0	5.3	1.13916448801434e-05	1.13916448801434e-05\\
0	5.4	8.81931116198544e-06	8.81931116198544e-06\\
0	5.5	6.90385927105757e-06	6.90385927105757e-06\\
0	5.6	5.55187481454831e-06	5.55187481454831e-06\\
0	5.7	4.64617920209325e-06	4.64617920209325e-06\\
0	5.8	4.08578395554014e-06	4.08578395554014e-06\\
0	5.9	3.79917037608312e-06	3.79917037608312e-06\\
0	6	3.74541324603009e-06	3.74541324603009e-06\\
0.1	0	5.88266794180992e-06	5.88266794180992e-06\\
0.1	0.1	5.65532400657263e-06	5.65532400657263e-06\\
0.1	0.2	5.86515981470986e-06	5.86515981470986e-06\\
0.1	0.3	6.51904099769749e-06	6.51904099769749e-06\\
0.1	0.4	7.68484000070577e-06	7.68484000070577e-06\\
0.1	0.5	9.4878899731335e-06	9.4878899731335e-06\\
0.1	0.6	1.21131222345095e-05	1.21131222345095e-05\\
0.1	0.7	1.58116402300964e-05	1.58116402300964e-05\\
0.1	0.8	2.09059243512776e-05	2.09059243512776e-05\\
0.1	0.9	2.77757072304639e-05	2.77757072304639e-05\\
0.1	1	3.68180449568025e-05	3.68180449568025e-05\\
0.1	1.1	4.85206385423914e-05	4.85206385423914e-05\\
0.1	1.2	6.41239396819236e-05	6.41239396819236e-05\\
0.1	1.3	8.72443173609858e-05	8.72443173609858e-05\\
0.1	1.4	0.000125121784743397	0.000125121784743397\\
0.1	1.5	0.000186755190787162	0.000186755190787162\\
0.1	1.6	0.000280032934465374	0.000280032934465374\\
0.1	1.7	0.00040980427243879	0.00040980427243879\\
0.1	1.8	0.000565835921288049	0.000565835921288049\\
0.1	1.9	0.000744190171256785	0.000744190171256785\\
0.1	2	0.00105271892440779	0.00105271892440779\\
0.1	2.1	0.00172812411130171	0.00172812411130171\\
0.1	2.2	0.00218576622127735	0.00218576622127735\\
0.1	2.3	0.00160501779983341	0.00160501779983341\\
0.1	2.4	0.000881027114462502	0.000881027114462502\\
0.1	2.5	0.000436068520387884	0.000436068520387884\\
0.1	2.6	0.000228701640892161	0.000228701640892161\\
0.1	2.7	0.000136025275997176	0.000136025275997176\\
0.1	2.8	8.89903134861861e-05	8.89903134861861e-05\\
0.1	2.9	6.88367437644042e-05	6.88367437644042e-05\\
0.1	3	7.47320958627255e-05	7.47320958627255e-05\\
0.1	3.1	0.000114265519234834	0.000114265519234834\\
0.1	3.2	0.00020707434906465	0.00020707434906465\\
0.1	3.3	0.000383231709800235	0.000383231709800235\\
0.1	3.4	0.000690744606478349	0.000690744606478349\\
0.1	3.5	0.0012045491145809	0.0012045491145809\\
0.1	3.6	0.00183826522266662	0.00183826522266662\\
0.1	3.7	0.00198088120294231	0.00198088120294231\\
0.1	3.8	0.00140609878980721	0.00140609878980721\\
0.1	3.9	0.00073506609953446	0.00073506609953446\\
0.1	4	0.000317479873352581	0.000317479873352581\\
0.1	4.1	0.00012698319280293	0.00012698319280293\\
0.1	4.2	5.5204824600693e-05	5.5204824600693e-05\\
0.1	4.3	2.95422111823069e-05	2.95422111823069e-05\\
0.1	4.4	2.0030917721344e-05	2.0030917721344e-05\\
0.1	4.5	1.67569135292839e-05	1.67569135292839e-05\\
0.1	4.6	1.62715165737368e-05	1.62715165737368e-05\\
0.1	4.7	1.69430605632108e-05	1.69430605632108e-05\\
0.1	4.8	1.76372675537418e-05	1.76372675537418e-05\\
0.1	4.9	1.75627548468823e-05	1.75627548468823e-05\\
0.1	5	1.64378233208567e-05	1.64378233208567e-05\\
0.1	5.1	1.44799188627199e-05	1.44799188627199e-05\\
0.1	5.2	1.21497548494907e-05	1.21497548494907e-05\\
0.1	5.3	9.87521569491016e-06	9.87521569491016e-06\\
0.1	5.4	7.91873009728731e-06	7.91873009728731e-06\\
0.1	5.5	6.37928052537941e-06	6.37928052537941e-06\\
0.1	5.6	5.25117959373467e-06	5.25117959373467e-06\\
0.1	5.7	4.48290360418762e-06	4.48290360418762e-06\\
0.1	5.8	4.01607573765325e-06	4.01607573765325e-06\\
0.1	5.9	3.80559474144532e-06	3.80559474144532e-06\\
0.1	6	3.82820153974458e-06	3.82820153974458e-06\\
0.2	0	5.7786815632989e-06	5.7786815632989e-06\\
0.2	0.1	5.31400536998905e-06	5.31400536998905e-06\\
0.2	0.2	5.2897332407797e-06	5.2897332407797e-06\\
0.2	0.3	5.68043815952929e-06	5.68043815952929e-06\\
0.2	0.4	6.51838789939976e-06	6.51838789939976e-06\\
0.2	0.5	7.88066073455428e-06	7.88066073455428e-06\\
0.2	0.6	9.87781158592516e-06	9.87781158592516e-06\\
0.2	0.7	1.26457149976464e-05	1.26457149976464e-05\\
0.2	0.8	1.63463549604863e-05	1.63463549604863e-05\\
0.2	0.9	2.11728795715116e-05	2.11728795715116e-05\\
0.2	1	2.7324875182765e-05	2.7324875182765e-05\\
0.2	1.1	3.49379700287839e-05	3.49379700287839e-05\\
0.2	1.2	4.42279358493252e-05	4.42279358493252e-05\\
0.2	1.3	5.66350028059429e-05	5.66350028059429e-05\\
0.2	1.4	7.70723575934769e-05	7.70723575934769e-05\\
0.2	1.5	0.000114125431207042	0.000114125431207042\\
0.2	1.6	0.00017483537307238	0.00017483537307238\\
0.2	1.7	0.000259548898136744	0.000259548898136744\\
0.2	1.8	0.0003586521780231	0.0003586521780231\\
0.2	1.9	0.000468119661031413	0.000468119661031413\\
0.2	2	0.00062117268457658	0.00062117268457658\\
0.2	2.1	0.000929485014043941	0.000929485014043941\\
0.2	2.2	0.00103368763295064	0.00103368763295064\\
0.2	2.3	0.000673730177925622	0.000673730177925622\\
0.2	2.4	0.000393764275796136	0.000393764275796136\\
0.2	2.5	0.000212974980114778	0.000212974980114778\\
0.2	2.6	0.000103374609893311	0.000103374609893311\\
0.2	2.7	5.03247520088923e-05	5.03247520088923e-05\\
0.2	2.8	2.74551832145793e-05	2.74551832145793e-05\\
0.2	2.9	1.94094612474875e-05	1.94094612474875e-05\\
0.2	3	2.02280575912836e-05	2.02280575912836e-05\\
0.2	3.1	3.00859123554156e-05	3.00859123554156e-05\\
0.2	3.2	5.60004940924443e-05	5.60004940924443e-05\\
0.2	3.3	0.000119123593463465	0.000119123593463465\\
0.2	3.4	0.000274661301567146	0.000274661301567146\\
0.2	3.5	0.000640863302024211	0.000640863302024211\\
0.2	3.6	0.00127589294247143	0.00127589294247143\\
0.2	3.7	0.00151945208720447	0.00151945208720447\\
0.2	3.8	0.000984796245492651	0.000984796245492651\\
0.2	3.9	0.000435565277154899	0.000435565277154899\\
0.2	4	0.000157781933543923	0.000157781933543923\\
0.2	4.1	5.46884759823542e-05	5.46884759823542e-05\\
0.2	4.2	2.24874959064941e-05	2.24874959064941e-05\\
0.2	4.3	1.25458252735441e-05	1.25458252735441e-05\\
0.2	4.4	9.56993543567715e-06	9.56993543567715e-06\\
0.2	4.5	9.3158344061868e-06	9.3158344061868e-06\\
0.2	4.6	1.04292553957758e-05	1.04292553957758e-05\\
0.2	4.7	1.21708450003684e-05	1.21708450003684e-05\\
0.2	4.8	1.38245249689543e-05	1.38245249689543e-05\\
0.2	4.9	1.473843930749e-05	1.473843930749e-05\\
0.2	5	1.45724496301218e-05	1.45724496301218e-05\\
0.2	5.1	1.34098598975752e-05	1.34098598975752e-05\\
0.2	5.2	1.16280901024604e-05	1.16280901024604e-05\\
0.2	5.3	9.6666088856242e-06	9.6666088856242e-06\\
0.2	5.4	7.85836126374514e-06	7.85836126374514e-06\\
0.2	5.5	6.37884220641552e-06	6.37884220641552e-06\\
0.2	5.6	5.2765504516261e-06	5.2765504516261e-06\\
0.2	5.7	4.52947629752627e-06	4.52947629752627e-06\\
0.2	5.8	4.09304301294703e-06	4.09304301294703e-06\\
0.2	5.9	3.92959352075561e-06	3.92959352075561e-06\\
0.2	6	4.02339053582186e-06	4.02339053582186e-06\\
0.3	0	6.09372653981831e-06	6.09372653981831e-06\\
0.3	0.1	5.38063136553664e-06	5.38063136553664e-06\\
0.3	0.2	5.13873964208975e-06	5.13873964208975e-06\\
0.3	0.3	5.31862062061163e-06	5.31862062061163e-06\\
0.3	0.4	5.93190907451713e-06	5.93190907451713e-06\\
0.3	0.5	7.03328240894e-06	7.03328240894e-06\\
0.3	0.6	8.69787166331031e-06	8.69787166331031e-06\\
0.3	0.7	1.09930211760459e-05	1.09930211760459e-05\\
0.3	0.8	1.39595474225332e-05	1.39595474225332e-05\\
0.3	0.9	1.76256796514965e-05	1.76256796514965e-05\\
0.3	1	2.20448592115875e-05	2.20448592115875e-05\\
0.3	1.1	2.72809975814282e-05	2.72809975814282e-05\\
0.3	1.2	3.33147720516343e-05	3.33147720516343e-05\\
0.3	1.3	4.03919973565613e-05	4.03919973565613e-05\\
0.3	1.4	5.1091155023326e-05	5.1091155023326e-05\\
0.3	1.5	7.32109938376885e-05	7.32109938376885e-05\\
0.3	1.6	0.000116302233961422	0.000116302233961422\\
0.3	1.7	0.000179680128896763	0.000179680128896763\\
0.3	1.8	0.000253357517460421	0.000253357517460421\\
0.3	1.9	0.000344933154082465	0.000344933154082465\\
0.3	2	0.000456803511765311	0.000456803511765311\\
0.3	2.1	0.000631878789560933	0.000631878789560933\\
0.3	2.2	0.000615089685559876	0.000615089685559876\\
0.3	2.3	0.000357620327260658	0.000357620327260658\\
0.3	2.4	0.000232678810626512	0.000232678810626512\\
0.3	2.5	0.000143957241551429	0.000143957241551429\\
0.3	2.6	6.29287398840014e-05	6.29287398840014e-05\\
0.3	2.7	2.46073610621099e-05	2.46073610621099e-05\\
0.3	2.8	1.23183428217707e-05	1.23183428217707e-05\\
0.3	2.9	8.7562065665051e-06	8.7562065665051e-06\\
0.3	3	8.52526846633311e-06	8.52526846633311e-06\\
0.3	3.1	1.12343724713276e-05	1.12343724713276e-05\\
0.3	3.2	1.90050478267067e-05	1.90050478267067e-05\\
0.3	3.3	4.16581480802993e-05	4.16581480802993e-05\\
0.3	3.4	0.00011825268696811	0.00011825268696811\\
0.3	3.5	0.000370431899433628	0.000370431899433628\\
0.3	3.6	0.000984498266378514	0.000984498266378514\\
0.3	3.7	0.00129745515290645	0.00129745515290645\\
0.3	3.8	0.000731054658712153	0.000731054658712153\\
0.3	3.9	0.000257847217982512	0.000257847217982512\\
0.3	4	7.47348552658298e-05	7.47348552658298e-05\\
0.3	4.1	2.24586421351127e-05	2.24586421351127e-05\\
0.3	4.2	9.29265617695336e-06	9.29265617695336e-06\\
0.3	4.3	5.90265754363968e-06	5.90265754363968e-06\\
0.3	4.4	5.4250221181951e-06	5.4250221181951e-06\\
0.3	4.5	6.32130289466295e-06	6.32130289466295e-06\\
0.3	4.6	8.17000142712972e-06	8.17000142712972e-06\\
0.3	4.7	1.06080707988706e-05	1.06080707988706e-05\\
0.3	4.8	1.3033831048094e-05	1.3033831048094e-05\\
0.3	4.9	1.47179656411017e-05	1.47179656411017e-05\\
0.3	5	1.51468107466328e-05	1.51468107466328e-05\\
0.3	5.1	1.42797857875515e-05	1.42797857875515e-05\\
0.3	5.2	1.25061608487501e-05	1.25061608487501e-05\\
0.3	5.3	1.03800946909292e-05	1.03800946909292e-05\\
0.3	5.4	8.36235469906138e-06	8.36235469906138e-06\\
0.3	5.5	6.7093582891485e-06	6.7093582891485e-06\\
0.3	5.6	5.49752924493327e-06	5.49752924493327e-06\\
0.3	5.7	4.70199912538429e-06	4.70199912538429e-06\\
0.3	5.8	4.26697903074272e-06	4.26697903074272e-06\\
0.3	5.9	4.14783896438703e-06	4.14783896438703e-06\\
0.3	6	4.33120091920477e-06	4.33120091920477e-06\\
0.4	0	6.8174195960195e-06	6.8174195960195e-06\\
0.4	0.1	5.83951586611631e-06	5.83951586611631e-06\\
0.4	0.2	5.37440032381014e-06	5.37440032381014e-06\\
0.4	0.3	5.35717745601452e-06	5.35717745601452e-06\\
0.4	0.4	5.79052334188099e-06	5.79052334188099e-06\\
0.4	0.5	6.72590923231304e-06	6.72590923231304e-06\\
0.4	0.6	8.23613530778031e-06	8.23613530778031e-06\\
0.4	0.7	1.03662974494285e-05	1.03662974494285e-05\\
0.4	0.8	1.30757026253588e-05	1.30757026253588e-05\\
0.4	0.9	1.62231340047827e-05	1.62231340047827e-05\\
0.4	1	1.96479347159528e-05	1.96479347159528e-05\\
0.4	1.1	2.32982875883789e-05	2.32982875883789e-05\\
0.4	1.2	2.72332338332833e-05	2.72332338332833e-05\\
0.4	1.3	3.15038038142265e-05	3.15038038142265e-05\\
0.4	1.4	3.69813631943375e-05	3.69813631943375e-05\\
0.4	1.5	4.91513833751296e-05	4.91513833751296e-05\\
0.4	1.6	8.10244906633884e-05	8.10244906633884e-05\\
0.4	1.7	0.000136940471240025	0.000136940471240025\\
0.4	1.8	0.000199969973189079	0.000199969973189079\\
0.4	1.9	0.000292248891457399	0.000292248891457399\\
0.4	2	0.000415778942776332	0.000415778942776332\\
0.4	2.1	0.000551614349123864	0.000551614349123864\\
0.4	2.2	0.000480098166219294	0.000480098166219294\\
0.4	2.3	0.000256056095444398	0.000256056095444398\\
0.4	2.4	0.000179084855341413	0.000179084855341413\\
0.4	2.5	0.000125819166557039	0.000125819166557039\\
0.4	2.6	4.7341212461091e-05	4.7341212461091e-05\\
0.4	2.7	1.62325108116417e-05	1.62325108116417e-05\\
0.4	2.8	9.21742648306741e-06	9.21742648306741e-06\\
0.4	2.9	6.91592476941836e-06	6.91592476941836e-06\\
0.4	3	5.8753487363141e-06	5.8753487363141e-06\\
0.4	3.1	6.2500540691857e-06	6.2500540691857e-06\\
0.4	3.2	8.58228726634188e-06	8.58228726634188e-06\\
0.4	3.3	1.67119251600422e-05	1.67119251600422e-05\\
0.4	3.4	5.2033019096257e-05	5.2033019096257e-05\\
0.4	3.5	0.000217182124934419	0.000217182124934419\\
0.4	3.6	0.000810615940050075	0.000810615940050075\\
0.4	3.7	0.00121385612628858	0.00121385612628858\\
0.4	3.8	0.00056078026831174	0.00056078026831174\\
0.4	3.9	0.000146477923987332	0.000146477923987332\\
0.4	4	3.25202944299744e-05	3.25202944299744e-05\\
0.4	4.1	8.88720923746023e-06	8.88720923746023e-06\\
0.4	4.2	4.14781815901891e-06	4.14781815901891e-06\\
0.4	4.3	3.32168399782304e-06	3.32168399782304e-06\\
0.4	4.4	3.86142582998051e-06	3.86142582998051e-06\\
0.4	4.5	5.41512772036421e-06	5.41512772036421e-06\\
0.4	4.6	7.97400590227737e-06	7.97400590227737e-06\\
0.4	4.7	1.13081520926777e-05	1.13081520926777e-05\\
0.4	4.8	1.47236678970561e-05	1.47236678970561e-05\\
0.4	4.9	1.72163854734813e-05	1.72163854734813e-05\\
0.4	5	1.7990900550368e-05	1.7990900550368e-05\\
0.4	5.1	1.69294019788577e-05	1.69294019788577e-05\\
0.4	5.2	1.45936974261116e-05	1.45936974261116e-05\\
0.4	5.3	1.18117975324833e-05	1.18117975324833e-05\\
0.4	5.4	9.2474457052958e-06	9.2474457052958e-06\\
0.4	5.5	7.23002848006114e-06	7.23002848006114e-06\\
0.4	5.6	5.8191493869592e-06	5.8191493869592e-06\\
0.4	5.7	4.94439920654911e-06	4.94439920654911e-06\\
0.4	5.8	4.5125324941817e-06	4.5125324941817e-06\\
0.4	5.9	4.46126447703427e-06	4.46126447703427e-06\\
0.4	6	4.77996419101902e-06	4.77996419101902e-06\\
0.5	0	7.90925964323552e-06	7.90925964323552e-06\\
0.5	0.1	6.68264518024304e-06	6.68264518024304e-06\\
0.5	0.2	5.99863145067468e-06	5.99863145067468e-06\\
0.5	0.3	5.78568978657769e-06	5.78568978657769e-06\\
0.5	0.4	6.05138807308546e-06	6.05138807308546e-06\\
0.5	0.5	6.86150949801444e-06	6.86150949801444e-06\\
0.5	0.6	8.31648548215693e-06	8.31648548215693e-06\\
0.5	0.7	1.04919536205811e-05	1.04919536205811e-05\\
0.5	0.8	1.33310079929684e-05	1.33310079929684e-05\\
0.5	0.9	1.65481162139365e-05	1.65481162139365e-05\\
0.5	1	1.96839660981279e-05	1.96839660981279e-05\\
0.5	1.1	2.23740515494854e-05	2.23740515494854e-05\\
0.5	1.2	2.46227029500499e-05	2.46227029500499e-05\\
0.5	1.3	2.67570312497902e-05	2.67570312497902e-05\\
0.5	1.4	2.92955551245826e-05	2.92955551245826e-05\\
0.5	1.5	3.49246725350177e-05	3.49246725350177e-05\\
0.5	1.6	5.62610706972828e-05	5.62610706972828e-05\\
0.5	1.7	0.000112997376702575	0.000112997376702575\\
0.5	1.8	0.000180703140923013	0.000180703140923013\\
0.5	1.9	0.000285861680562909	0.000285861680562909\\
0.5	2	0.000440035593839112	0.000440035593839112\\
0.5	2.1	0.000532257550575507	0.000532257550575507\\
0.5	2.2	0.000420902601871628	0.000420902601871628\\
0.5	2.3	0.000243916133452761	0.000243916133452761\\
0.5	2.4	0.000170935259828205	0.000170935259828205\\
0.5	2.5	0.000130420260887203	0.000130420260887203\\
0.5	2.6	4.11450825257956e-05	4.11450825257956e-05\\
0.5	2.7	1.57935489992526e-05	1.57935489992526e-05\\
0.5	2.8	1.17043733318955e-05	1.17043733318955e-05\\
0.5	2.9	8.5082200174334e-06	8.5082200174334e-06\\
0.5	3	6.05696127102638e-06	6.05696127102638e-06\\
0.5	3.1	4.97624095846403e-06	4.97624095846403e-06\\
0.5	3.2	5.28243954501446e-06	5.28243954501446e-06\\
0.5	3.3	8.21432126136073e-06	8.21432126136073e-06\\
0.5	3.4	2.27116991140656e-05	2.27116991140656e-05\\
0.5	3.5	0.000117389740885264	0.000117389740885264\\
0.5	3.6	0.000674666016918275	0.000674666016918275\\
0.5	3.7	0.00122991801008148	0.00122991801008148\\
0.5	3.8	0.000426978274663442	0.000426978274663442\\
0.5	3.9	7.49086480871064e-05	7.49086480871064e-05\\
0.5	4	1.24988166175056e-05	1.24988166175056e-05\\
0.5	4.1	3.58021559614318e-06	3.58021559614318e-06\\
0.5	4.2	2.22571115250656e-06	2.22571115250656e-06\\
0.5	4.3	2.45262518782973e-06	2.45262518782973e-06\\
0.5	4.4	3.64276119678378e-06	3.64276119678378e-06\\
0.5	4.5	5.98015933113711e-06	5.98015933113711e-06\\
0.5	4.6	9.66362465377391e-06	9.66362465377391e-06\\
0.5	4.7	1.44350562598267e-05	1.44350562598267e-05\\
0.5	4.8	1.92770764792434e-05	1.92770764792434e-05\\
0.5	4.9	2.26576773367533e-05	2.26576773367533e-05\\
0.5	5	2.33887937232975e-05	2.33887937232975e-05\\
0.5	5.1	2.14156093548626e-05	2.14156093548626e-05\\
0.5	5.2	1.77627413521245e-05	1.77627413521245e-05\\
0.5	5.3	1.37584112042705e-05	1.37584112042705e-05\\
0.5	5.4	1.03261727390495e-05	1.03261727390495e-05\\
0.5	5.5	7.80790984908602e-06	7.80790984908602e-06\\
0.5	5.6	6.16362071254975e-06	6.16362071254975e-06\\
0.5	5.7	5.22190131225893e-06	5.22190131225893e-06\\
0.5	5.8	4.82814233408453e-06	4.82814233408453e-06\\
0.5	5.9	4.89901961737785e-06	4.89901961737785e-06\\
0.5	6	5.43491634104622e-06	5.43491634104622e-06\\
0.6	0	9.22701533034913e-06	9.22701533034913e-06\\
0.6	0.1	7.83597993678267e-06	7.83597993678267e-06\\
0.6	0.2	6.99460346748534e-06	6.99460346748534e-06\\
0.6	0.3	6.615026020838e-06	6.615026020838e-06\\
0.6	0.4	6.7224145286725e-06	6.7224145286725e-06\\
0.6	0.5	7.41725673117339e-06	7.41725673117339e-06\\
0.6	0.6	8.85584765728053e-06	8.85584765728053e-06\\
0.6	0.7	1.11991179628628e-05	1.11991179628628e-05\\
0.6	0.8	1.44710423504082e-05	1.44710423504082e-05\\
0.6	0.9	1.83409273777031e-05	1.83409273777031e-05\\
0.6	1	2.20311739048184e-05	2.20311739048184e-05\\
0.6	1.1	2.46337883871551e-05	2.46337883871551e-05\\
0.6	1.2	2.57422811759819e-05	2.57422811759819e-05\\
0.6	1.3	2.57896958158561e-05	2.57896958158561e-05\\
0.6	1.4	2.57417491339652e-05	2.57417491339652e-05\\
0.6	1.5	2.71420496850671e-05	2.71420496850671e-05\\
0.6	1.6	3.75063048217679e-05	3.75063048217679e-05\\
0.6	1.7	9.06389975568604e-05	9.06389975568604e-05\\
0.6	1.8	0.000189617036949088	0.000189617036949088\\
0.6	1.9	0.000336749376878546	0.000336749376878546\\
0.6	2	0.000488108062576275	0.000488108062576275\\
0.6	2.1	0.000362838190634291	0.000362838190634291\\
0.6	2.2	0.000239034852446149	0.000239034852446149\\
0.6	2.3	0.000235141906352622	0.000235141906352622\\
0.6	2.4	0.000189541337919035	0.000189541337919035\\
0.6	2.5	0.000155608255138024	0.000155608255138024\\
0.6	2.6	4.1974160658554e-05	4.1974160658554e-05\\
0.6	2.7	2.26996994030287e-05	2.26996994030287e-05\\
0.6	2.8	1.77922146701668e-05	1.77922146701668e-05\\
0.6	2.9	1.1471858673002e-05	1.1471858673002e-05\\
0.6	3	7.19871671635948e-06	7.19871671635948e-06\\
0.6	3.1	4.9624906762118e-06	4.9624906762118e-06\\
0.6	3.2	4.30130958538324e-06	4.30130958538324e-06\\
0.6	3.3	5.27050016685088e-06	5.27050016685088e-06\\
0.6	3.4	1.06202092832311e-05	1.06202092832311e-05\\
0.6	3.5	5.22023546304162e-05	5.22023546304162e-05\\
0.6	3.6	0.000513076265680561	0.000513076265680561\\
0.6	3.7	0.00133326176812346	0.00133326176812346\\
0.6	3.8	0.000299039280197849	0.000299039280197849\\
0.6	3.9	3.08673505927159e-05	3.08673505927159e-05\\
0.6	4	4.29282010069036e-06	4.29282010069036e-06\\
0.6	4.1	1.72318382226541e-06	1.72318382226541e-06\\
0.6	4.2	1.69049825761454e-06	1.69049825761454e-06\\
0.6	4.3	2.59601414616829e-06	2.59601414616829e-06\\
0.6	4.4	4.62930226752667e-06	4.62930226752667e-06\\
0.6	4.5	8.26667808168473e-06	8.26667808168473e-06\\
0.6	4.6	1.37981214770025e-05	1.37981214770025e-05\\
0.6	4.7	2.07579147537045e-05	2.07579147537045e-05\\
0.6	4.8	2.75046831219669e-05	2.75046831219669e-05\\
0.6	4.9	3.16639370704259e-05	3.16639370704259e-05\\
0.6	5	3.15936305021084e-05	3.15936305021084e-05\\
0.6	5.1	2.76228652288818e-05	2.76228652288818e-05\\
0.6	5.2	2.16976955176505e-05	2.16976955176505e-05\\
0.6	5.3	1.58947065243473e-05	1.58947065243473e-05\\
0.6	5.4	1.13637832913052e-05	1.13637832913052e-05\\
0.6	5.5	8.30926830862399e-06	8.30926830862399e-06\\
0.6	5.6	6.47185544337197e-06	6.47185544337197e-06\\
0.6	5.7	5.52490511848336e-06	5.52490511848336e-06\\
0.6	5.8	5.24182533741238e-06	5.24182533741238e-06\\
0.6	5.9	5.52768625856796e-06	5.52768625856796e-06\\
0.6	6	6.41201014177213e-06	6.41201014177213e-06\\
0.7	0	1.05080075841151e-05	1.05080075841151e-05\\
0.7	0.1	9.07795812366521e-06	9.07795812366521e-06\\
0.7	0.2	8.23251188752017e-06	8.23251188752017e-06\\
0.7	0.3	7.80122611961447e-06	7.80122611961447e-06\\
0.7	0.4	7.80683525792917e-06	7.80683525792917e-06\\
0.7	0.5	8.39843462151778e-06	8.39843462151778e-06\\
0.7	0.6	9.81540765691687e-06	9.81540765691687e-06\\
0.7	0.7	1.23495704179216e-05	1.23495704179216e-05\\
0.7	0.8	1.62090771781928e-05	1.62090771781928e-05\\
0.7	0.9	2.11981643551415e-05	2.11981643551415e-05\\
0.7	1	2.6367377582297e-05	2.6367377582297e-05\\
0.7	1.1	3.01548229348111e-05	3.01548229348111e-05\\
0.7	1.2	3.13200239157632e-05	3.13200239157632e-05\\
0.7	1.3	2.99196063272198e-05	2.99196063272198e-05\\
0.7	1.4	2.72421876506536e-05	2.72421876506536e-05\\
0.7	1.5	2.50076193119007e-05	2.50076193119007e-05\\
0.7	1.6	2.65390244983936e-05	2.65390244983936e-05\\
0.7	1.7	5.68022411066128e-05	5.68022411066128e-05\\
0.7	1.8	0.0002103447759976	0.0002103447759976\\
0.7	1.9	0.000496415329292983	0.000496415329292983\\
0.7	2	0.000489969726597537	0.000489969726597537\\
0.7	2.1	9.57551353979557e-05	9.57551353979557e-05\\
0.7	2.2	4.85373320422762e-05	4.85373320422762e-05\\
0.7	2.3	0.000123117183182075	0.000123117183182075\\
0.7	2.4	0.000232204947737064	0.000232204947737064\\
0.7	2.5	0.000226653926606036	0.000226653926606036\\
0.7	2.6	5.25908913745651e-05	5.25908913745651e-05\\
0.7	2.7	3.27575015735764e-05	3.27575015735764e-05\\
0.7	2.8	2.06858497120167e-05	2.06858497120167e-05\\
0.7	2.9	1.30348005667847e-05	1.30348005667847e-05\\
0.7	3	8.57213545553542e-06	8.57213545553542e-06\\
0.7	3.1	6.07836140502606e-06	6.07836140502606e-06\\
0.7	3.2	4.82785841516132e-06	4.82785841516132e-06\\
0.7	3.3	4.66642253378056e-06	4.66642253378056e-06\\
0.7	3.4	6.44843566215553e-06	6.44843566215553e-06\\
0.7	3.5	1.93408546133874e-05	1.93408546133874e-05\\
0.7	3.6	0.000288844462829318	0.000288844462829318\\
0.7	3.7	0.00151449864757065	0.00151449864757065\\
0.7	3.8	0.000165201880671389	0.000165201880671389\\
0.7	3.9	8.92071472063281e-06	8.92071472063281e-06\\
0.7	4	1.66202361105617e-06	1.66202361105617e-06\\
0.7	4.1	1.34627945796402e-06	1.34627945796402e-06\\
0.7	4.2	2.09443586583358e-06	2.09443586583358e-06\\
0.7	4.3	3.88849223979134e-06	3.88849223979134e-06\\
0.7	4.4	7.2599962007663e-06	7.2599962007663e-06\\
0.7	4.5	1.28445471647087e-05	1.28445471647087e-05\\
0.7	4.6	2.09347884263152e-05	2.09347884263152e-05\\
0.7	4.7	3.0704284958154e-05	3.0704284958154e-05\\
0.7	4.8	3.95901751957051e-05	3.95901751957051e-05\\
0.7	4.9	4.40613038150338e-05	4.40613038150338e-05\\
0.7	5	4.20545947064007e-05	4.20545947064007e-05\\
0.7	5.1	3.47899837847168e-05	3.47899837847168e-05\\
0.7	5.2	2.56884483529649e-05	2.56884483529649e-05\\
0.7	5.3	1.77274394967179e-05	1.77274394967179e-05\\
0.7	5.4	1.20921926306743e-05	1.20921926306743e-05\\
0.7	5.5	8.62034282501733e-06	8.62034282501733e-06\\
0.7	5.6	6.71727781693881e-06	6.71727781693881e-06\\
0.7	5.7	5.87830860402581e-06	5.87830860402581e-06\\
0.7	5.8	5.82087789991983e-06	5.82087789991983e-06\\
0.7	5.9	6.4641140040512e-06	6.4641140040512e-06\\
0.7	6	7.89258956069872e-06	7.89258956069872e-06\\
0.8	0	1.15103741000763e-05	1.15103741000763e-05\\
0.8	0.1	1.00532099738462e-05	1.00532099738462e-05\\
0.8	0.2	9.38192203814063e-06	9.38192203814063e-06\\
0.8	0.3	9.1256241691229e-06	9.1256241691229e-06\\
0.8	0.4	9.20241691275128e-06	9.20241691275128e-06\\
0.8	0.5	9.76590417311979e-06	9.76590417311979e-06\\
0.8	0.6	1.1145659519505e-05	1.1145659519505e-05\\
0.8	0.7	1.37893202720443e-05	1.37893202720443e-05\\
0.8	0.8	1.81458693133219e-05	1.81458693133219e-05\\
0.8	0.9	2.43173227449801e-05	2.43173227449801e-05\\
0.8	1	3.14753020886595e-05	3.14753020886595e-05\\
0.8	1.1	3.76413838718931e-05	3.76413838718931e-05\\
0.8	1.2	4.06469880603143e-05	4.06469880603143e-05\\
0.8	1.3	3.9759505776403e-05	3.9759505776403e-05\\
0.8	1.4	3.61340446325857e-05	3.61340446325857e-05\\
0.8	1.5	3.15777811585294e-05	3.15777811585294e-05\\
0.8	1.6	2.74215203561925e-05	2.74215203561925e-05\\
0.8	1.7	3.02574519552159e-05	3.02574519552159e-05\\
0.8	1.8	0.000164699747946537	0.000164699747946537\\
0.8	1.9	0.000882175133051652	0.000882175133051652\\
0.8	2	0.000258611629957175	0.000258611629957175\\
0.8	2.1	1.33043389966555e-05	1.33043389966555e-05\\
0.8	2.2	1.01715132865559e-05	1.01715132865559e-05\\
0.8	2.3	2.81646461651619e-05	2.81646461651619e-05\\
0.8	2.4	0.000231546953806939	0.000231546953806939\\
0.8	2.5	0.000412687120411017	0.000412687120411017\\
0.8	2.6	6.56973900334802e-05	6.56973900334802e-05\\
0.8	2.7	2.88998104603783e-05	2.88998104603783e-05\\
0.8	2.8	1.99383564570732e-05	1.99383564570732e-05\\
0.8	2.9	1.58351124501623e-05	1.58351124501623e-05\\
0.8	3	1.26236344943242e-05	1.26236344943242e-05\\
0.8	3.1	9.94937415348767e-06	9.94937415348767e-06\\
0.8	3.2	7.82370851465224e-06	7.82370851465224e-06\\
0.8	3.3	6.30812752259073e-06	6.30812752259073e-06\\
0.8	3.4	5.8025890997567e-06	5.8025890997567e-06\\
0.8	3.5	8.78393874694376e-06	8.78393874694376e-06\\
0.8	3.6	8.62676867519814e-05	8.62676867519814e-05\\
0.8	3.7	0.0017446615495272	0.0017446615495272\\
0.8	3.8	5.2397767828173e-05	5.2397767828173e-05\\
0.8	3.9	2.20814731135073e-06	2.20814731135073e-06\\
0.8	4	1.34573024228727e-06	1.34573024228727e-06\\
0.8	4.1	2.12012702398234e-06	2.12012702398234e-06\\
0.8	4.2	3.78451459933781e-06	3.78451459933781e-06\\
0.8	4.3	6.71469504307049e-06	6.71469504307049e-06\\
0.8	4.4	1.15853523676643e-05	1.15853523676643e-05\\
0.8	4.5	1.91646707917119e-05	1.91646707917119e-05\\
0.8	4.6	2.97946227824907e-05	2.97946227824907e-05\\
0.8	4.7	4.23275837191149e-05	4.23275837191149e-05\\
0.8	4.8	5.31731270979916e-05	5.31731270979916e-05\\
0.8	4.9	5.74091374874429e-05	5.74091374874429e-05\\
0.8	5	5.25605042789395e-05	5.25605042789395e-05\\
0.8	5.1	4.11928278213608e-05	4.11928278213608e-05\\
0.8	5.2	2.86265033342233e-05	2.86265033342233e-05\\
0.8	5.3	1.86885843890453e-05	1.86885843890453e-05\\
0.8	5.4	1.22875095054559e-05	1.22875095054559e-05\\
0.8	5.5	8.6877408363608e-06	8.6877408363608e-06\\
0.8	5.6	6.92360047782083e-06	6.92360047782083e-06\\
0.8	5.7	6.35217923481642e-06	6.35217923481642e-06\\
0.8	5.8	6.68304930792032e-06	6.68304930792032e-06\\
0.8	5.9	7.88618273771475e-06	7.88618273771475e-06\\
0.8	6	1.01226200459109e-05	1.01226200459109e-05\\
0.9	0	1.23043122981406e-05	1.23043122981406e-05\\
0.9	0.1	1.05030654485767e-05	1.05030654485767e-05\\
0.9	0.2	9.9873758263816e-06	9.9873758263816e-06\\
0.9	0.3	1.01142859786373e-05	1.01142859786373e-05\\
0.9	0.4	1.05590001935458e-05	1.05590001935458e-05\\
0.9	0.5	1.13130089399637e-05	1.13130089399637e-05\\
0.9	0.6	1.27043634542146e-05	1.27043634542146e-05\\
0.9	0.7	1.53132877405004e-05	1.53132877405004e-05\\
0.9	0.8	1.98062351330482e-05	1.98062351330482e-05\\
0.9	0.9	2.65746082582044e-05	2.65746082582044e-05\\
0.9	1	3.50470666074801e-05	3.50470666074801e-05\\
0.9	1.1	4.3219488578797e-05	4.3219488578797e-05\\
0.9	1.2	4.85543856519375e-05	4.85543856519375e-05\\
0.9	1.3	5.00315752753352e-05	5.00315752753352e-05\\
0.9	1.4	4.90694075410294e-05	4.90694075410294e-05\\
0.9	1.5	4.82234900133015e-05	4.82234900133015e-05\\
0.9	1.6	4.86708088550178e-05	4.86708088550178e-05\\
0.9	1.7	4.44530059703115e-05	4.44530059703115e-05\\
0.9	1.8	4.78354413772626e-05	4.78354413772626e-05\\
0.9	1.9	0.00157832576242431	0.00157832576242431\\
0.9	2	2.20827264956947e-05	2.20827264956947e-05\\
0.9	2.1	9.80603850430073e-06	9.80603850430073e-06\\
0.9	2.2	1.20870163185151e-05	1.20870163185151e-05\\
0.9	2.3	1.43097027555361e-05	1.43097027555361e-05\\
0.9	2.4	7.38246230847106e-05	7.38246230847106e-05\\
0.9	2.5	0.000766875861708913	0.000766875861708913\\
0.9	2.6	3.90279191386626e-05	3.90279191386626e-05\\
0.9	2.7	3.05373827248101e-05	3.05373827248101e-05\\
0.9	2.8	3.19711412726179e-05	3.19711412726179e-05\\
0.9	2.9	2.75149223519802e-05	2.75149223519802e-05\\
0.9	3	2.23198626211864e-05	2.23198626211864e-05\\
0.9	3.1	1.78675507416601e-05	1.78675507416601e-05\\
0.9	3.2	1.4276346643597e-05	1.4276346643597e-05\\
0.9	3.3	1.13479353059892e-05	1.13479353059892e-05\\
0.9	3.4	8.88965267727825e-06	8.88965267727825e-06\\
0.9	3.5	7.08668137894896e-06	7.08668137894896e-06\\
0.9	3.6	1.59524250563711e-05	1.59524250563711e-05\\
0.9	3.7	0.00196280862048625	0.00196280862048625\\
0.9	3.8	6.22748975096307e-06	6.22748975096307e-06\\
0.9	3.9	1.86511543397602e-06	1.86511543397602e-06\\
0.9	4	3.04046975682389e-06	3.04046975682389e-06\\
0.9	4.1	4.49280760632094e-06	4.49280760632094e-06\\
0.9	4.2	6.59102845443591e-06	6.59102845443591e-06\\
0.9	4.3	9.94926675410848e-06	9.94926675410848e-06\\
0.9	4.4	1.53648914267122e-05	1.53648914267122e-05\\
0.9	4.5	2.37495354913588e-05	2.37495354913588e-05\\
0.9	4.6	3.56434650372904e-05	3.56434650372904e-05\\
0.9	4.7	4.99350553573409e-05	4.99350553573409e-05\\
0.9	4.8	6.23610262996922e-05	6.23610262996922e-05\\
0.9	4.9	6.65652664702172e-05	6.65652664702172e-05\\
0.9	5	5.93443816992724e-05	5.93443816992724e-05\\
0.9	5.1	4.4537248619336e-05	4.4537248619336e-05\\
0.9	5.2	2.93971510876609e-05	2.93971510876609e-05\\
0.9	5.3	1.83824800946639e-05	1.83824800946639e-05\\
0.9	5.4	1.18800000322045e-05	1.18800000322045e-05\\
0.9	5.5	8.55479496177128e-06	8.55479496177128e-06\\
0.9	5.6	7.17633104102921e-06	7.17633104102921e-06\\
0.9	5.7	7.06957279640815e-06	7.06957279640815e-06\\
0.9	5.8	8.00192408034952e-06	8.00192408034952e-06\\
0.9	5.9	1.00210960907455e-05	1.00210960907455e-05\\
0.9	6	1.33554165472568e-05	1.33554165472568e-05\\
1	0	1.35190651198002e-05	1.35190651198002e-05\\
1	0.1	1.06053873425852e-05	1.06053873425852e-05\\
1	0.2	9.82568315753211e-06	9.82568315753211e-06\\
1	0.3	1.02214792662534e-05	1.02214792662534e-05\\
1	0.4	1.12359898549567e-05	1.12359898549567e-05\\
1	0.5	1.25259339570923e-05	1.25259339570923e-05\\
1	0.6	1.41425727393645e-05	1.41425727393645e-05\\
1	0.7	1.66284417148172e-05	1.66284417148172e-05\\
1	0.8	2.07714947232912e-05	2.07714947232912e-05\\
1	0.9	2.70829873520783e-05	2.70829873520783e-05\\
1	1	3.50310221326936e-05	3.50310221326936e-05\\
1	1.1	4.26020207334711e-05	4.26020207334711e-05\\
1	1.2	4.7421893449997e-05	4.7421893449997e-05\\
1	1.3	4.90130395224622e-05	4.90130395224622e-05\\
1	1.4	4.96577456278537e-05	4.96577456278537e-05\\
1	1.5	5.34608327256047e-05	5.34608327256047e-05\\
1	1.6	6.63221581154445e-05	6.63221581154445e-05\\
1	1.7	9.94846547799882e-05	9.94846547799882e-05\\
1	1.8	0.000179545445709168	0.000179545445709168\\
1	1.9	0.00209294417558654	0.00209294417558654\\
1	2	3.49403901805616e-05	3.49403901805616e-05\\
1	2.1	2.64600359084019e-05	2.64600359084019e-05\\
1	2.2	2.23648141850574e-05	2.23648141850574e-05\\
1	2.3	2.20721823574769e-05	2.20721823574769e-05\\
1	2.4	2.60604436138017e-05	2.60604436138017e-05\\
1	2.5	0.000979148151270534	0.000979148151270534\\
1	2.6	8.88411094941131e-05	8.88411094941131e-05\\
1	2.7	6.24531844673148e-05	6.24531844673148e-05\\
1	2.8	4.62246425273449e-05	4.62246425273449e-05\\
1	2.9	3.54915959978884e-05	3.54915959978884e-05\\
1	3	2.80580681116727e-05	2.80580681116727e-05\\
1	3.1	2.27739750961889e-05	2.27739750961889e-05\\
1	3.2	1.89164599576687e-05	1.89164599576687e-05\\
1	3.3	1.60189265513398e-05	1.60189265513398e-05\\
1	3.4	1.3831865405278e-05	1.3831865405278e-05\\
1	3.5	1.22420067774399e-05	1.22420067774399e-05\\
1	3.6	1.11623327130134e-05	1.11623327130134e-05\\
1	3.7	0.0020955511431271	0.0020955511431271\\
1	3.8	5.74088068457581e-06	5.74088068457581e-06\\
1	3.9	5.30476760519006e-06	5.30476760519006e-06\\
1	4	5.46680638374257e-06	5.46680638374257e-06\\
1	4.1	6.22062711034405e-06	6.22062711034405e-06\\
1	4.2	7.76672834529793e-06	7.76672834529793e-06\\
1	4.3	1.05217395984373e-05	1.05217395984373e-05\\
1	4.4	1.51564768519995e-05	1.51564768519995e-05\\
1	4.5	2.25829190346247e-05	2.25829190346247e-05\\
1	4.6	3.36292800481614e-05	3.36292800481614e-05\\
1	4.7	4.78123502964967e-05	4.78123502964967e-05\\
1	4.8	6.12094005144708e-05	6.12094005144708e-05\\
1	4.9	6.65231702796976e-05	6.65231702796976e-05\\
1	5	5.91549104366679e-05	5.91549104366679e-05\\
1	5.1	4.32714666224324e-05	4.32714666224324e-05\\
1	5.2	2.75524429267845e-05	2.75524429267845e-05\\
1	5.3	1.68419133818583e-05	1.68419133818583e-05\\
1	5.4	1.10158163597907e-05	1.10158163597907e-05\\
1	5.5	8.36537401399203e-06	8.36537401399203e-06\\
1	5.6	7.62071794477045e-06	7.62071794477045e-06\\
1	5.7	8.20362658256022e-06	8.20362658256022e-06\\
1	5.8	9.98288077053173e-06	9.98288077053173e-06\\
1	5.9	1.30658338561074e-05	1.30658338561074e-05\\
1	6	1.76860902709133e-05	1.76860902709133e-05\\
1.1	0	1.65177731321043e-05	1.65177731321043e-05\\
1.1	0.1	1.11109843631812e-05	1.11109843631812e-05\\
1.1	0.2	9.27048815518657e-06	9.27048815518657e-06\\
1.1	0.3	9.34999823874358e-06	9.34999823874358e-06\\
1.1	0.4	1.06511964684153e-05	1.06511964684153e-05\\
1.1	0.5	1.26316717152092e-05	1.26316717152092e-05\\
1.1	0.6	1.4829136489245e-05	1.4829136489245e-05\\
1.1	0.7	1.73185784825581e-05	1.73185784825581e-05\\
1.1	0.8	2.0779871373064e-05	2.0779871373064e-05\\
1.1	0.9	2.57146768399173e-05	2.57146768399173e-05\\
1.1	1	3.14205704814569e-05	3.14205704814569e-05\\
1.1	1.1	3.58032407617329e-05	3.58032407617329e-05\\
1.1	1.2	3.69998613222462e-05	3.69998613222462e-05\\
1.1	1.3	3.53503519075153e-05	3.53503519075153e-05\\
1.1	1.4	3.31347579749243e-05	3.31347579749243e-05\\
1.1	1.5	3.27927713735362e-05	3.27927713735362e-05\\
1.1	1.6	3.52132635903542e-05	3.52132635903542e-05\\
1.1	1.7	3.50232806544353e-05	3.50232806544353e-05\\
1.1	1.8	3.61186432206531e-05	3.61186432206531e-05\\
1.1	1.9	0.00165261150096254	0.00165261150096254\\
1.1	2	1.5004464591256e-05	1.5004464591256e-05\\
1.1	2.1	4.23897868012192e-06	4.23897868012192e-06\\
1.1	2.2	5.14180271890422e-06	5.14180271890422e-06\\
1.1	2.3	6.44429130312272e-06	6.44429130312272e-06\\
1.1	2.4	6.25271722666153e-05	6.25271722666153e-05\\
1.1	2.5	0.000698115812736009	0.000698115812736009\\
1.1	2.6	3.13537756785749e-05	3.13537756785749e-05\\
1.1	2.7	2.30090364741544e-05	2.30090364741544e-05\\
1.1	2.8	2.41774597919709e-05	2.41774597919709e-05\\
1.1	2.9	2.18769708627292e-05	2.18769708627292e-05\\
1.1	3	1.89089317158318e-05	1.89089317158318e-05\\
1.1	3.1	1.62231633443511e-05	1.62231633443511e-05\\
1.1	3.2	1.39815145282048e-05	1.39815145282048e-05\\
1.1	3.3	1.20990790555601e-05	1.20990790555601e-05\\
1.1	3.4	1.0475578921885e-05	1.0475578921885e-05\\
1.1	3.5	9.6239991218683e-06	9.6239991218683e-06\\
1.1	3.6	2.54678592951203e-05	2.54678592951203e-05\\
1.1	3.7	0.00210426140964015	0.00210426140964015\\
1.1	3.8	1.04378795615552e-05	1.04378795615552e-05\\
1.1	3.9	2.67154644657929e-06	2.67154644657929e-06\\
1.1	4	3.62283010551088e-06	3.62283010551088e-06\\
1.1	4.1	4.69489365813712e-06	4.69489365813712e-06\\
1.1	4.2	6.06253835750086e-06	6.06253835750086e-06\\
1.1	4.3	8.13453341200673e-06	8.13453341200673e-06\\
1.1	4.4	1.14642089616873e-05	1.14642089616873e-05\\
1.1	4.5	1.68348915338839e-05	1.68348915338839e-05\\
1.1	4.6	2.52189621155654e-05	2.52189621155654e-05\\
1.1	4.7	3.70386055693964e-05	3.70386055693964e-05\\
1.1	4.8	4.99223588289505e-05	4.99223588289505e-05\\
1.1	4.9	5.69934890768127e-05	5.69934890768127e-05\\
1.1	5	5.19641488156019e-05	5.19641488156019e-05\\
1.1	5.1	3.78240438448871e-05	3.78240438448871e-05\\
1.1	5.2	2.36893822976121e-05	2.36893822976121e-05\\
1.1	5.3	1.45514255511493e-05	1.45514255511493e-05\\
1.1	5.4	1.00050978387406e-05	1.00050978387406e-05\\
1.1	5.5	8.32861561785619e-06	8.32861561785619e-06\\
1.1	5.6	8.4428181456574e-06	8.4428181456574e-06\\
1.1	5.7	9.94014487658002e-06	9.94014487658002e-06\\
1.1	5.8	1.27639393509594e-05	1.27639393509594e-05\\
1.1	5.9	1.70053076240241e-05	1.70053076240241e-05\\
1.1	6	2.28146213655971e-05	2.28146213655971e-05\\
1.2	0	2.38466036106058e-05	2.38466036106058e-05\\
1.2	0.1	1.33333812710893e-05	1.33333812710893e-05\\
1.2	0.2	9.23725465443081e-06	9.23725465443081e-06\\
1.2	0.3	8.1721448212245e-06	8.1721448212245e-06\\
1.2	0.4	8.9685600918522e-06	8.9685600918522e-06\\
1.2	0.5	1.11422995075917e-05	1.11422995075917e-05\\
1.2	0.6	1.40331404367432e-05	1.40331404367432e-05\\
1.2	0.7	1.68879701922899e-05	1.68879701922899e-05\\
1.2	0.8	1.97416081308165e-05	1.97416081308165e-05\\
1.2	0.9	2.30266747842053e-05	2.30266747842053e-05\\
1.2	1	2.60363129926944e-05	2.60363129926944e-05\\
1.2	1.1	2.69253053219391e-05	2.69253053219391e-05\\
1.2	1.2	2.48566265397053e-05	2.48566265397053e-05\\
1.2	1.3	2.11409382627869e-05	2.11409382627869e-05\\
1.2	1.4	1.78244419192621e-05	1.78244419192621e-05\\
1.2	1.5	1.61556031258909e-05	1.61556031258909e-05\\
1.2	1.6	1.65760359515804e-05	1.65760359515804e-05\\
1.2	1.7	2.2287562719252e-05	2.2287562719252e-05\\
1.2	1.8	5.8428036996592e-05	5.8428036996592e-05\\
1.2	1.9	0.000850458643439421	0.000850458643439421\\
1.2	2	0.000167927441901157	0.000167927441901157\\
1.2	2.1	6.10667202717485e-06	6.10667202717485e-06\\
1.2	2.2	4.15409844595476e-06	4.15409844595476e-06\\
1.2	2.3	1.93635697896382e-05	1.93635697896382e-05\\
1.2	2.4	0.000180958102945477	0.000180958102945477\\
1.2	2.5	0.000353920419778238	0.000353920419778238\\
1.2	2.6	5.67723970878561e-05	5.67723970878561e-05\\
1.2	2.7	2.45068513387984e-05	2.45068513387984e-05\\
1.2	2.8	1.50750957703458e-05	1.50750957703458e-05\\
1.2	2.9	1.13508030284877e-05	1.13508030284877e-05\\
1.2	3	9.32433543630325e-06	9.32433543630325e-06\\
1.2	3.1	8.0454805496978e-06	8.0454805496978e-06\\
1.2	3.2	7.30497997717513e-06	7.30497997717513e-06\\
1.2	3.3	7.24422283753024e-06	7.24422283753024e-06\\
1.2	3.4	9.08137915259767e-06	9.08137915259767e-06\\
1.2	3.5	2.02031084305145e-05	2.02031084305145e-05\\
1.2	3.6	0.000124462376498814	0.000124462376498814\\
1.2	3.7	0.00201614511655748	0.00201614511655748\\
1.2	3.8	8.15894460747276e-05	8.15894460747276e-05\\
1.2	3.9	6.46746538614507e-06	6.46746538614507e-06\\
1.2	4	3.04012368069811e-06	3.04012368069811e-06\\
1.2	4.1	3.33341623088564e-06	3.33341623088564e-06\\
1.2	4.2	4.28695632932917e-06	4.28695632932917e-06\\
1.2	4.3	5.70949024272972e-06	5.70949024272972e-06\\
1.2	4.4	7.83797243092481e-06	7.83797243092481e-06\\
1.2	4.5	1.115260425255e-05	1.115260425255e-05\\
1.2	4.6	1.64217931048414e-05	1.64217931048414e-05\\
1.2	4.7	2.45387778759589e-05	2.45387778759589e-05\\
1.2	4.8	3.49993732782033e-05	3.49993732782033e-05\\
1.2	4.9	4.29170963274216e-05	4.29170963274216e-05\\
1.2	5	4.11224049981942e-05	4.11224049981942e-05\\
1.2	5.1	3.03465123791706e-05	3.03465123791706e-05\\
1.2	5.2	1.90770572961559e-05	1.90770572961559e-05\\
1.2	5.3	1.21777176842712e-05	1.21777176842712e-05\\
1.2	5.4	9.1907290434655e-06	9.1907290434655e-06\\
1.2	5.5	8.66635037185401e-06	8.66635037185401e-06\\
1.2	5.6	9.81801812093394e-06	9.81801812093394e-06\\
1.2	5.7	1.23658880006741e-05	1.23658880006741e-05\\
1.2	5.8	1.62413680154441e-05	1.62413680154441e-05\\
1.2	5.9	2.14471377944101e-05	2.14471377944101e-05\\
1.2	6	2.80329274290306e-05	2.80329274290306e-05\\
1.3	0	3.94971773811182e-05	3.94971773811182e-05\\
1.3	0.1	1.93683937594531e-05	1.93683937594531e-05\\
1.3	0.2	1.0980398543989e-05	1.0980398543989e-05\\
1.3	0.3	7.77540768091091e-06	7.77540768091091e-06\\
1.3	0.4	7.24241550849599e-06	7.24241550849599e-06\\
1.3	0.5	8.62051688881202e-06	8.62051688881202e-06\\
1.3	0.6	1.15596083266135e-05	1.15596083266135e-05\\
1.3	0.7	1.50024522973647e-05	1.50024522973647e-05\\
1.3	0.8	1.77556472124879e-05	1.77556472124879e-05\\
1.3	0.9	1.98227567698678e-05	1.98227567698678e-05\\
1.3	1	2.08305658643206e-05	2.08305658643206e-05\\
1.3	1.1	1.9574059445708e-05	1.9574059445708e-05\\
1.3	1.2	1.64097267160183e-05	1.64097267160183e-05\\
1.3	1.3	1.32499636159809e-05	1.32499636159809e-05\\
1.3	1.4	1.16937238428705e-05	1.16937238428705e-05\\
1.3	1.5	1.27270501069749e-05	1.27270501069749e-05\\
1.3	1.6	1.81322518142034e-05	1.81322518142034e-05\\
1.3	1.7	2.76770969411397e-05	2.76770969411397e-05\\
1.3	1.8	6.56188057575693e-05	6.56188057575693e-05\\
1.3	1.9	0.00041505551612894	0.00041505551612894\\
1.3	2	0.000274715710282366	0.000274715710282366\\
1.3	2.1	5.76681775098003e-05	5.76681775098003e-05\\
1.3	2.2	3.65061205285342e-05	3.65061205285342e-05\\
1.3	2.3	9.55976033647666e-05	9.55976033647666e-05\\
1.3	2.4	0.000178378266852874	0.000178378266852874\\
1.3	2.5	0.000204347662688785	0.000204347662688785\\
1.3	2.6	7.00815887802762e-05	7.00815887802762e-05\\
1.3	2.7	4.89419378729568e-05	4.89419378729568e-05\\
1.3	2.8	2.13909980970561e-05	2.13909980970561e-05\\
1.3	2.9	1.03600744265862e-05	1.03600744265862e-05\\
1.3	3	6.19884333374106e-06	6.19884333374106e-06\\
1.3	3.1	4.81087042489807e-06	4.81087042489807e-06\\
1.3	3.2	4.88038575771228e-06	4.88038575771228e-06\\
1.3	3.3	6.93002502179844e-06	6.93002502179844e-06\\
1.3	3.4	1.5326229137845e-05	1.5326229137845e-05\\
1.3	3.5	4.60513723393079e-05	4.60513723393079e-05\\
1.3	3.6	0.00043796274808889	0.00043796274808889\\
1.3	3.7	0.0019091233505083	0.0019091233505083\\
1.3	3.8	0.000265363249344733	0.000265363249344733\\
1.3	3.9	2.93055356878851e-05	2.93055356878851e-05\\
1.3	4	7.14327671172434e-06	7.14327671172434e-06\\
1.3	4.1	4.29523843154948e-06	4.29523843154948e-06\\
1.3	4.2	4.23176620532693e-06	4.23176620532693e-06\\
1.3	4.3	4.92405816527295e-06	4.92405816527295e-06\\
1.3	4.4	6.10067873217359e-06	6.10067873217359e-06\\
1.3	4.5	7.89740442106464e-06	7.89740442106464e-06\\
1.3	4.6	1.07372502428159e-05	1.07372502428159e-05\\
1.3	4.7	1.54263466501095e-05	1.54263466501095e-05\\
1.3	4.8	2.2600152910077e-05	2.2600152910077e-05\\
1.3	4.9	2.98887818074015e-05	2.98887818074015e-05\\
1.3	5	3.05971564928919e-05	3.05971564928919e-05\\
1.3	5.1	2.31529285012727e-05	2.31529285012727e-05\\
1.3	5.2	1.48693244916924e-05	1.48693244916924e-05\\
1.3	5.3	1.02446358708259e-05	1.02446358708259e-05\\
1.3	5.4	8.84179833078002e-06	8.84179833078002e-06\\
1.3	5.5	9.55264681415427e-06	9.55264681415427e-06\\
1.3	5.6	1.18008321306234e-05	1.18008321306234e-05\\
1.3	5.7	1.5340990358099e-05	1.5340990358099e-05\\
1.3	5.8	2.00637820641952e-05	2.00637820641952e-05\\
1.3	5.9	2.59161113597476e-05	2.59161113597476e-05\\
1.3	6	3.30290716994634e-05	3.30290716994634e-05\\
1.4	0	6.54746954991499e-05	6.54746954991499e-05\\
1.4	0.1	3.14265937211519e-05	3.14265937211519e-05\\
1.4	0.2	1.6082522456344e-05	1.6082522456344e-05\\
1.4	0.3	9.29854362859517e-06	9.29854362859517e-06\\
1.4	0.4	6.67517452752437e-06	6.67517452752437e-06\\
1.4	0.5	6.48845192103058e-06	6.48845192103058e-06\\
1.4	0.6	8.35629918359487e-06	8.35629918359487e-06\\
1.4	0.7	1.18741905011176e-05	1.18741905011176e-05\\
1.4	0.8	1.51320081431343e-05	1.51320081431343e-05\\
1.4	0.9	1.68184315734602e-05	1.68184315734602e-05\\
1.4	1	1.68840372832966e-05	1.68840372832966e-05\\
1.4	1.1	1.4869490539897e-05	1.4869490539897e-05\\
1.4	1.2	1.21593105793421e-05	1.21593105793421e-05\\
1.4	1.3	1.08399747436144e-05	1.08399747436144e-05\\
1.4	1.4	1.22080678254252e-05	1.22080678254252e-05\\
1.4	1.5	1.74407508149154e-05	1.74407508149154e-05\\
1.4	1.6	2.48328763554945e-05	2.48328763554945e-05\\
1.4	1.7	2.99227060728691e-05	2.99227060728691e-05\\
1.4	1.8	8.23963102902532e-05	8.23963102902532e-05\\
1.4	1.9	0.00026712674496963	0.00026712674496963\\
1.4	2	0.00031015340967992	0.00031015340967992\\
1.4	2.1	0.000254802808679191	0.000254802808679191\\
1.4	2.2	0.000151601885981349	0.000151601885981349\\
1.4	2.3	0.000127980332565727	0.000127980332565727\\
1.4	2.4	0.000140946073012068	0.000140946073012068\\
1.4	2.5	0.000170326649214941	0.000170326649214941\\
1.4	2.6	9.23359144528478e-05	9.23359144528478e-05\\
1.4	2.7	6.61643623806778e-05	6.61643623806778e-05\\
1.4	2.8	3.57477763493375e-05	3.57477763493375e-05\\
1.4	2.9	1.32919409481932e-05	1.32919409481932e-05\\
1.4	3	6.48428366759832e-06	6.48428366759832e-06\\
1.4	3.1	4.75698434131326e-06	4.75698434131326e-06\\
1.4	3.2	5.58495193536351e-06	5.58495193536351e-06\\
1.4	3.3	1.05457327637074e-05	1.05457327637074e-05\\
1.4	3.4	2.75884573659194e-05	2.75884573659194e-05\\
1.4	3.5	0.000108930964572336	0.000108930964572336\\
1.4	3.6	0.000941971300615335	0.000941971300615335\\
1.4	3.7	0.00186246880860195	0.00186246880860195\\
1.4	3.8	0.000513552857353441	0.000513552857353441\\
1.4	3.9	9.67704215838609e-05	9.67704215838609e-05\\
1.4	4	2.30514635719829e-05	2.30514635719829e-05\\
1.4	4.1	9.72863786169082e-06	9.72863786169082e-06\\
1.4	4.2	6.93064070389319e-06	6.93064070389319e-06\\
1.4	4.3	6.36178454620758e-06	6.36178454620758e-06\\
1.4	4.4	6.50892168001113e-06	6.50892168001113e-06\\
1.4	4.5	7.09001250931175e-06	7.09001250931175e-06\\
1.4	4.6	8.19924832185389e-06	8.19924832185389e-06\\
1.4	4.7	1.03826450775361e-05	1.03826450775361e-05\\
1.4	4.8	1.46212387475368e-05	1.46212387475368e-05\\
1.4	4.9	2.04907924963014e-05	2.04907924963014e-05\\
1.4	5	2.2573827560952e-05	2.2573827560952e-05\\
1.4	5.1	1.74972639140733e-05	1.74972639140733e-05\\
1.4	5.2	1.16279973198467e-05	1.16279973198467e-05\\
1.4	5.3	9.00916474495729e-06	9.00916474495729e-06\\
1.4	5.4	9.11915499204507e-06	9.11915499204507e-06\\
1.4	5.5	1.10367181917731e-05	1.10367181917731e-05\\
1.4	5.6	1.43134213137349e-05	1.43134213137349e-05\\
1.4	5.7	1.88136442312165e-05	1.88136442312165e-05\\
1.4	5.8	2.44610344018093e-05	2.44610344018093e-05\\
1.4	5.9	3.12540344194475e-05	3.12540344194475e-05\\
1.4	6	3.95716830855524e-05	3.95716830855524e-05\\
1.5	0	9.701963967612e-05	9.701963967612e-05\\
1.5	0.1	4.81937586645989e-05	4.81937586645989e-05\\
1.5	0.2	2.49873648497952e-05	2.49873648497952e-05\\
1.5	0.3	1.36426876318765e-05	1.36426876318765e-05\\
1.5	0.4	8.1770029151836e-06	8.1770029151836e-06\\
1.5	0.5	5.94209272979417e-06	5.94209272979417e-06\\
1.5	0.6	6.00842511127987e-06	6.00842511127987e-06\\
1.5	0.7	8.43253431281507e-06	8.43253431281507e-06\\
1.5	0.8	1.22630943382041e-05	1.22630943382041e-05\\
1.5	0.9	1.44617005317419e-05	1.44617005317419e-05\\
1.5	1	1.44355466485631e-05	1.44355466485631e-05\\
1.5	1.1	1.23904973007201e-05	1.23904973007201e-05\\
1.5	1.2	1.09148179388236e-05	1.09148179388236e-05\\
1.5	1.3	1.2421037348539e-05	1.2421037348539e-05\\
1.5	1.4	1.80467287439893e-05	1.80467287439893e-05\\
1.5	1.5	2.58891795854023e-05	2.58891795854023e-05\\
1.5	1.6	2.94011209451221e-05	2.94011209451221e-05\\
1.5	1.7	4.29352973860687e-05	4.29352973860687e-05\\
1.5	1.8	0.00012725520051198	0.00012725520051198\\
1.5	1.9	0.000221739266772959	0.000221739266772959\\
1.5	2	0.000285665147440335	0.000285665147440335\\
1.5	2.1	0.000372152202551118	0.000372152202551118\\
1.5	2.2	0.000219164073012714	0.000219164073012714\\
1.5	2.3	0.000114831220942847	0.000114831220942847\\
1.5	2.4	0.00011314925210751	0.00011314925210751\\
1.5	2.5	0.000172273017117682	0.000172273017117682\\
1.5	2.6	0.00013013797225793	0.00013013797225793\\
1.5	2.7	6.86251198751217e-05	6.86251198751217e-05\\
1.5	2.8	4.20986294202286e-05	4.20986294202286e-05\\
1.5	2.9	1.73088339725811e-05	1.73088339725811e-05\\
1.5	3	8.44162271032471e-06	8.44162271032471e-06\\
1.5	3.1	6.42438111721547e-06	6.42438111721547e-06\\
1.5	3.2	8.21143479472365e-06	8.21143479472365e-06\\
1.5	3.3	1.6892348442806e-05	1.6892348442806e-05\\
1.5	3.4	4.94609932003041e-05	4.94609932003041e-05\\
1.5	3.5	0.000260303441694518	0.000260303441694518\\
1.5	3.6	0.00144798934196461	0.00144798934196461\\
1.5	3.7	0.00189342145927507	0.00189342145927507\\
1.5	3.8	0.000776962316408716	0.000776962316408716\\
1.5	3.9	0.000236379957271914	0.000236379957271914\\
1.5	4	6.45266670750335e-05	6.45266670750335e-05\\
1.5	4.1	2.42101367155981e-05	2.42101367155981e-05\\
1.5	4.2	1.49036109093204e-05	1.49036109093204e-05\\
1.5	4.3	1.18574109768314e-05	1.18574109768314e-05\\
1.5	4.4	1.00359262162688e-05	1.00359262162688e-05\\
1.5	4.5	8.85317981359307e-06	8.85317981359307e-06\\
1.5	4.6	8.22962943838331e-06	8.22962943838331e-06\\
1.5	4.7	8.3429648296797e-06	8.3429648296797e-06\\
1.5	4.8	1.01981252656473e-05	1.01981252656473e-05\\
1.5	4.9	1.45644457295237e-05	1.45644457295237e-05\\
1.5	5	1.73393031025352e-05	1.73393031025352e-05\\
1.5	5.1	1.350236790932e-05	1.350236790932e-05\\
1.5	5.2	9.3870050565002e-06	9.3870050565002e-06\\
1.5	5.3	8.56035386190506e-06	8.56035386190506e-06\\
1.5	5.4	1.01077358016043e-05	1.01077358016043e-05\\
1.5	5.5	1.32970200263767e-05	1.32970200263767e-05\\
1.5	5.6	1.81699257990479e-05	1.81699257990479e-05\\
1.5	5.7	2.47577837267385e-05	2.47577837267385e-05\\
1.5	5.8	3.26282807260432e-05	3.26282807260432e-05\\
1.5	5.9	4.14473835451875e-05	4.14473835451875e-05\\
1.5	6	5.19035861894102e-05	5.19035861894102e-05\\
1.6	0	0.000135887599041975	0.000135887599041975\\
1.6	0.1	6.8455310300604e-05	6.8455310300604e-05\\
1.6	0.2	3.6475522999102e-05	3.6475522999102e-05\\
1.6	0.3	2.01823217620939e-05	2.01823217620939e-05\\
1.6	0.4	1.1758155236086e-05	1.1758155236086e-05\\
1.6	0.5	7.47800359716694e-06	7.47800359716694e-06\\
1.6	0.6	5.60964002632607e-06	5.60964002632607e-06\\
1.6	0.7	5.99950666830656e-06	5.99950666830656e-06\\
1.6	0.8	9.35944116574149e-06	9.35944116574149e-06\\
1.6	0.9	1.28132800721833e-05	1.28132800721833e-05\\
1.6	1	1.32369604406894e-05	1.32369604406894e-05\\
1.6	1.1	1.13666041725708e-05	1.13666041725708e-05\\
1.6	1.2	1.2290301992436e-05	1.2290301992436e-05\\
1.6	1.3	1.83479090953151e-05	1.83479090953151e-05\\
1.6	1.4	2.67215587689705e-05	2.67215587689705e-05\\
1.6	1.5	3.11210035610889e-05	3.11210035610889e-05\\
1.6	1.6	3.85510470922865e-05	3.85510470922865e-05\\
1.6	1.7	8.09894162731172e-05	8.09894162731172e-05\\
1.6	1.8	0.000182833576161254	0.000182833576161254\\
1.6	1.9	0.000205114556295803	0.000205114556295803\\
1.6	2	0.000238640096839393	0.000238640096839393\\
1.6	2.1	0.00031613852275534	0.00031613852275534\\
1.6	2.2	0.000217178981332381	0.000217178981332381\\
1.6	2.3	0.000112936706780463	0.000112936706780463\\
1.6	2.4	9.94212897809603e-05	9.94212897809603e-05\\
1.6	2.5	0.000161227914765477	0.000161227914765477\\
1.6	2.6	0.000159476567889084	0.000159476567889084\\
1.6	2.7	7.55664829710676e-05	7.55664829710676e-05\\
1.6	2.8	3.91268681908687e-05	3.91268681908687e-05\\
1.6	2.9	2.02086250289303e-05	2.02086250289303e-05\\
1.6	3	1.16191273687171e-05	1.16191273687171e-05\\
1.6	3.1	9.41217164227133e-06	9.41217164227133e-06\\
1.6	3.2	1.22314457898724e-05	1.22314457898724e-05\\
1.6	3.3	2.64623129412714e-05	2.64623129412714e-05\\
1.6	3.4	9.40712694706548e-05	9.40712694706548e-05\\
1.6	3.5	0.00051072625569351	0.00051072625569351\\
1.6	3.6	0.00178369997906508	0.00178369997906508\\
1.6	3.7	0.00192883095832879	0.00192883095832879\\
1.6	3.8	0.00100484476013008	0.00100484476013008\\
1.6	3.9	0.000414115821373076	0.000414115821373076\\
1.6	4	0.000140642077261468	0.000140642077261468\\
1.6	4.1	4.92595518600895e-05	4.92595518600895e-05\\
1.6	4.2	2.6937828184044e-05	2.6937828184044e-05\\
1.6	4.3	2.17958823089803e-05	2.17958823089803e-05\\
1.6	4.4	1.89102813604296e-05	1.89102813604296e-05\\
1.6	4.5	1.50285068844142e-05	1.50285068844142e-05\\
1.6	4.6	1.15099575340936e-05	1.15099575340936e-05\\
1.6	4.7	8.99826063607762e-06	8.99826063607762e-06\\
1.6	4.8	8.17133897178769e-06	8.17133897178769e-06\\
1.6	4.9	1.09245603616741e-05	1.09245603616741e-05\\
1.6	5	1.43271551280169e-05	1.43271551280169e-05\\
1.6	5.1	1.06434054797355e-05	1.06434054797355e-05\\
1.6	5.2	8.03283566999239e-06	8.03283566999239e-06\\
1.6	5.3	9.13744037996408e-06	9.13744037996408e-06\\
1.6	5.4	1.27611683813844e-05	1.27611683813844e-05\\
1.6	5.5	1.96824147539244e-05	1.96824147539244e-05\\
1.6	5.6	3.02160595722589e-05	3.02160595722589e-05\\
1.6	5.7	4.19325451420075e-05	4.19325451420075e-05\\
1.6	5.8	5.213673572797e-05	5.213673572797e-05\\
1.6	5.9	6.14901760840484e-05	6.14901760840484e-05\\
1.6	6	7.31234652612616e-05	7.31234652612616e-05\\
1.7	0	0.000204615462706095	0.000204615462706095\\
1.7	0.1	0.000105561743891556	0.000105561743891556\\
1.7	0.2	5.93340079739853e-05	5.93340079739853e-05\\
1.7	0.3	3.40040509969237e-05	3.40040509969237e-05\\
1.7	0.4	1.9102003538953e-05	1.9102003538953e-05\\
1.7	0.5	1.09213280619711e-05	1.09213280619711e-05\\
1.7	0.6	7.10405376863193e-06	7.10405376863193e-06\\
1.7	0.7	5.72806083824905e-06	5.72806083824905e-06\\
1.7	0.8	6.7323532911549e-06	6.7323532911549e-06\\
1.7	0.9	1.13912461615481e-05	1.13912461615481e-05\\
1.7	1	1.28628720602068e-05	1.28628720602068e-05\\
1.7	1.1	1.17672274725412e-05	1.17672274725412e-05\\
1.7	1.2	1.75489974520593e-05	1.75489974520593e-05\\
1.7	1.3	2.50436228095082e-05	2.50436228095082e-05\\
1.7	1.4	2.9205419316237e-05	2.9205419316237e-05\\
1.7	1.5	3.5571265067203e-05	3.5571265067203e-05\\
1.7	1.6	6.11463120194532e-05	6.11463120194532e-05\\
1.7	1.7	0.000135993240282856	0.000135993240282856\\
1.7	1.8	0.000221943702742945	0.000221943702742945\\
1.7	1.9	0.000243002302325129	0.000243002302325129\\
1.7	2	0.000263613742044106	0.000263613742044106\\
1.7	2.1	0.000283192295149218	0.000283192295149218\\
1.7	2.2	0.000202724024723148	0.000202724024723148\\
1.7	2.3	0.000130812815666269	0.000130812815666269\\
1.7	2.4	0.000107635203604118	0.000107635203604118\\
1.7	2.5	0.000131927164474044	0.000131927164474044\\
1.7	2.6	0.000139078034416926	0.000139078034416926\\
1.7	2.7	7.80808264702272e-05	7.80808264702272e-05\\
1.7	2.8	3.91373266383794e-05	3.91373266383794e-05\\
1.7	2.9	2.36494857658725e-05	2.36494857658725e-05\\
1.7	3	1.68330945034305e-05	1.68330945034305e-05\\
1.7	3.1	1.52266174993051e-05	1.52266174993051e-05\\
1.7	3.2	2.04871894719251e-05	2.04871894719251e-05\\
1.7	3.3	4.67118157421577e-05	4.67118157421577e-05\\
1.7	3.4	0.000176828069921681	0.000176828069921681\\
1.7	3.5	0.000784133762902781	0.000784133762902781\\
1.7	3.6	0.00193006105373733	0.00193006105373733\\
1.7	3.7	0.001894905859929	0.001894905859929\\
1.7	3.8	0.00109546415312761	0.00109546415312761\\
1.7	3.9	0.000510681754037464	0.000510681754037464\\
1.7	4	0.000206751859837774	0.000206751859837774\\
1.7	4.1	7.69485298243802e-05	7.69485298243802e-05\\
1.7	4.2	3.60551566536316e-05	3.60551566536316e-05\\
1.7	4.3	2.65130747697498e-05	2.65130747697498e-05\\
1.7	4.4	2.61854057928751e-05	2.61854057928751e-05\\
1.7	4.5	2.59613731337644e-05	2.59613731337644e-05\\
1.7	4.6	2.1484531586324e-05	2.1484531586324e-05\\
1.7	4.7	1.47653495825498e-05	1.47653495825498e-05\\
1.7	4.8	9.25185688817684e-06	9.25185688817684e-06\\
1.7	4.9	8.29879723183211e-06	8.29879723183211e-06\\
1.7	5	1.29285264737801e-05	1.29285264737801e-05\\
1.7	5.1	8.11315621938572e-06	8.11315621938572e-06\\
1.7	5.2	8.40640689834948e-06	8.40640689834948e-06\\
1.7	5.3	1.46923467659214e-05	1.46923467659214e-05\\
1.7	5.4	3.08547695920276e-05	3.08547695920276e-05\\
1.7	5.5	5.50561088719545e-05	5.50561088719545e-05\\
1.7	5.6	7.28339869304504e-05	7.28339869304504e-05\\
1.7	5.7	7.88543339157455e-05	7.88543339157455e-05\\
1.7	5.8	8.11874681360007e-05	8.11874681360007e-05\\
1.7	5.9	8.75443024702658e-05	8.75443024702658e-05\\
1.7	6	0.000102159483009917	0.000102159483009917\\
1.8	0	0.000307816458589934	0.000307816458589934\\
1.8	0.1	0.000172748640723257	0.000172748640723257\\
1.8	0.2	0.00011285753543927	0.00011285753543927\\
1.8	0.3	8.01762506538676e-05	8.01762506538676e-05\\
1.8	0.4	5.60029636417583e-05	5.60029636417583e-05\\
1.8	0.5	3.50169441431788e-05	3.50169441431788e-05\\
1.8	0.6	1.9079743723197e-05	1.9079743723197e-05\\
1.8	0.7	1.02365921119551e-05	1.02365921119551e-05\\
1.8	0.8	7.2585790175288e-06	7.2585790175288e-06\\
1.8	0.9	8.29348343668465e-06	8.29348343668465e-06\\
1.8	1	1.29490282105104e-05	1.29490282105104e-05\\
1.8	1.1	1.57646930361631e-05	1.57646930361631e-05\\
1.8	1.2	2.33772862609269e-05	2.33772862609269e-05\\
1.8	1.3	2.97551318355899e-05	2.97551318355899e-05\\
1.8	1.4	3.43251997873144e-05	3.43251997873144e-05\\
1.8	1.5	5.0908534932654e-05	5.0908534932654e-05\\
1.8	1.6	0.000101746069279798	0.000101746069279798\\
1.8	1.7	0.000192934876855806	0.000192934876855806\\
1.8	1.8	0.000288516799143428	0.000288516799143428\\
1.8	1.9	0.000375122585150536	0.000375122585150536\\
1.8	2	0.000378254282821056	0.000378254282821056\\
1.8	2.1	0.000302331107958012	0.000302331107958012\\
1.8	2.2	0.000212393950559981	0.000212393950559981\\
1.8	2.3	0.000170665620905671	0.000170665620905671\\
1.8	2.4	0.000132667962767827	0.000132667962767827\\
1.8	2.5	0.000107084026644935	0.000107084026644935\\
1.8	2.6	9.41988915454227e-05	9.41988915454227e-05\\
1.8	2.7	6.41198629392682e-05	6.41198629392682e-05\\
1.8	2.8	3.77061578944853e-05	3.77061578944853e-05\\
1.8	2.9	2.69050394862875e-05	2.69050394862875e-05\\
1.8	3	2.35742069731941e-05	2.35742069731941e-05\\
1.8	3.1	2.52527196308333e-05	2.52527196308333e-05\\
1.8	3.2	3.74450134481851e-05	3.74450134481851e-05\\
1.8	3.3	8.72547209445013e-05	8.72547209445013e-05\\
1.8	3.4	0.00029616471978778	0.00029616471978778\\
1.8	3.5	0.000996270127959216	0.000996270127959216\\
1.8	3.6	0.00194349821641577	0.00194349821641577\\
1.8	3.7	0.00177270833343262	0.00177270833343262\\
1.8	3.8	0.00101502829514782	0.00101502829514782\\
1.8	3.9	0.000474034688601614	0.000474034688601614\\
1.8	4	0.000201640068681531	0.000201640068681531\\
1.8	4.1	8.5305300613663e-05	8.5305300613663e-05\\
1.8	4.2	4.05780951959152e-05	4.05780951959152e-05\\
1.8	4.3	2.63401324236363e-05	2.63401324236363e-05\\
1.8	4.4	2.44320954542145e-05	2.44320954542145e-05\\
1.8	4.5	2.83546118200681e-05	2.83546118200681e-05\\
1.8	4.6	3.55803262074947e-05	3.55803262074947e-05\\
1.8	4.7	4.11234805272077e-05	4.11234805272077e-05\\
1.8	4.8	2.99213676372378e-05	2.99213676372378e-05\\
1.8	4.9	1.07875678383545e-05	1.07875678383545e-05\\
1.8	5	1.28054983373343e-05	1.28054983373343e-05\\
1.8	5.1	9.77842101555599e-06	9.77842101555599e-06\\
1.8	5.2	4.2380250405784e-05	4.2380250405784e-05\\
1.8	5.3	0.000110743526949536	0.000110743526949536\\
1.8	5.4	0.000135379760577181	0.000135379760577181\\
1.8	5.5	0.000119692893962355	0.000119692893962355\\
1.8	5.6	0.00010131328846413	0.00010131328846413\\
1.8	5.7	9.3596575821469e-05	9.3596575821469e-05\\
1.8	5.8	9.79984569331282e-05	9.79984569331282e-05\\
1.8	5.9	0.000114970981033726	0.000114970981033726\\
1.8	6	0.000148449438286874	0.000148449438286874\\
1.9	0	0.000381163189115407	0.000381163189115407\\
1.9	0.1	0.000230041118517011	0.000230041118517011\\
1.9	0.2	0.000168349594509923	0.000168349594509923\\
1.9	0.3	0.000146607979591334	0.000146607979591334\\
1.9	0.4	0.000142188105483345	0.000142188105483345\\
1.9	0.5	0.000140391801973009	0.000140391801973009\\
1.9	0.6	0.000130675268979315	0.000130675268979315\\
1.9	0.7	0.000113279149479989	0.000113279149479989\\
1.9	0.8	9.81300932204048e-05	9.81300932204048e-05\\
1.9	0.9	9.05872372128387e-05	9.05872372128387e-05\\
1.9	1	8.68479418531867e-05	8.68479418531867e-05\\
1.9	1.1	8.10497621380437e-05	8.10497621380437e-05\\
1.9	1.2	7.22794629608534e-05	7.22794629608534e-05\\
1.9	1.3	6.65760652262369e-05	6.65760652262369e-05\\
1.9	1.4	7.70004367207528e-05	7.70004367207528e-05\\
1.9	1.5	0.000120889662137569	0.000120889662137569\\
1.9	1.6	0.00019469168588764	0.00019469168588764\\
1.9	1.7	0.000258540758605285	0.000258540758605285\\
1.9	1.8	0.00034937703248264	0.00034937703248264\\
1.9	1.9	0.000462584363869054	0.000462584363869054\\
1.9	2	0.000461009997272682	0.000461009997272682\\
1.9	2.1	0.000365167609004507	0.000365167609004507\\
1.9	2.2	0.000286845548476555	0.000286845548476555\\
1.9	2.3	0.000246670871222853	0.000246670871222853\\
1.9	2.4	0.000171394325468633	0.000171394325468633\\
1.9	2.5	0.000101880853138054	0.000101880853138054\\
1.9	2.6	6.97333166360788e-05	6.97333166360788e-05\\
1.9	2.7	4.9434748593585e-05	4.9434748593585e-05\\
1.9	2.8	3.36687144203125e-05	3.36687144203125e-05\\
1.9	2.9	2.75866687031643e-05	2.75866687031643e-05\\
1.9	3	2.86199039909167e-05	2.86199039909167e-05\\
1.9	3.1	3.67989384288322e-05	3.67989384288322e-05\\
1.9	3.2	6.22483946591654e-05	6.22483946591654e-05\\
1.9	3.3	0.000146235308169257	0.000146235308169257\\
1.9	3.4	0.000425588908494506	0.000425588908494506\\
1.9	3.5	0.0011257576389229	0.0011257576389229\\
1.9	3.6	0.0018507529891049	0.0018507529891049\\
1.9	3.7	0.00157130750468328	0.00157130750468328\\
1.9	3.8	0.0008528079004336	0.0008528079004336\\
1.9	3.9	0.000390236769219097	0.000390236769219097\\
1.9	4	0.000169777045855149	0.000169777045855149\\
1.9	4.1	7.82318019389369e-05	7.82318019389369e-05\\
1.9	4.2	4.14718867064472e-05	4.14718867064472e-05\\
1.9	4.3	2.74812625519252e-05	2.74812625519252e-05\\
1.9	4.4	2.38940414959813e-05	2.38940414959813e-05\\
1.9	4.5	2.57086014731101e-05	2.57086014731101e-05\\
1.9	4.6	3.14276823588807e-05	3.14276823588807e-05\\
1.9	4.7	4.24704664613569e-05	4.24704664613569e-05\\
1.9	4.8	6.25317696832026e-05	6.25317696832026e-05\\
1.9	4.9	9.1556913708648e-05	9.1556913708648e-05\\
1.9	5	0.000115546233148161	0.000115546233148161\\
1.9	5.1	0.000117278028630094	0.000117278028630094\\
1.9	5.2	0.000101573970043898	0.000101573970043898\\
1.9	5.3	8.46752746805898e-05	8.46752746805898e-05\\
1.9	5.4	7.39719354677333e-05	7.39719354677333e-05\\
1.9	5.5	7.00257994160174e-05	7.00257994160174e-05\\
1.9	5.6	7.33318732440231e-05	7.33318732440231e-05\\
1.9	5.7	8.54598959227191e-05	8.54598959227191e-05\\
1.9	5.8	0.000109443053646163	0.000109443053646163\\
1.9	5.9	0.000153370310776094	0.000153370310776094\\
1.9	6	0.00023771791588614	0.00023771791588614\\
2	0	0.000443872182541227	0.000443872182541227\\
2	0.1	0.000269868312101031	0.000269868312101031\\
2	0.2	0.000189643894755275	0.000189643894755275\\
2	0.3	0.000157231201748079	0.000157231201748079\\
2	0.4	0.000153017690972266	0.000153017690972266\\
2	0.5	0.000164897053190375	0.000164897053190375\\
2	0.6	0.000165819750747009	0.000165819750747009\\
2	0.7	0.000115963380781063	0.000115963380781063\\
2	0.8	4.58426007726941e-05	4.58426007726941e-05\\
2	0.9	1.94043208506908e-05	1.94043208506908e-05\\
2	1	2.03878668680969e-05	2.03878668680969e-05\\
2	1.1	2.81523298786754e-05	2.81523298786754e-05\\
2	1.2	5.90314640248396e-05	5.90314640248396e-05\\
2	1.3	9.93693105562067e-05	9.93693105562067e-05\\
2	1.4	0.000151558408278	0.000151558408278\\
2	1.5	0.000226719830862091	0.000226719830862091\\
2	1.6	0.000282909874956039	0.000282909874956039\\
2	1.7	0.000308564765867273	0.000308564765867273\\
2	1.8	0.000362786074353143	0.000362786074353143\\
2	1.9	0.000432056446253642	0.000432056446253642\\
2	2	0.000462470939061127	0.000462470939061127\\
2	2.1	0.000452039134231594	0.000452039134231594\\
2	2.2	0.000421900639412689	0.000421900639412689\\
2	2.3	0.000360693677513062	0.000360693677513062\\
2	2.4	0.000234614009100435	0.000234614009100435\\
2	2.5	0.00012629773198899	0.00012629773198899\\
2	2.6	7.43790980184938e-05	7.43790980184938e-05\\
2	2.7	4.9867599742238e-05	4.9867599742238e-05\\
2	2.8	3.54475658318479e-05	3.54475658318479e-05\\
2	2.9	3.03647038499082e-05	3.03647038499082e-05\\
2	3	3.39076131404775e-05	3.39076131404775e-05\\
2	3.1	4.87709310379456e-05	4.87709310379456e-05\\
2	3.2	9.05086109225572e-05	9.05086109225572e-05\\
2	3.3	0.000212060701760688	0.000212060701760688\\
2	3.4	0.000547247341229435	0.000547247341229435\\
2	3.5	0.00121459657848443	0.00121459657848443\\
2	3.6	0.00174687493288425	0.00174687493288425\\
2	3.7	0.00141647798659832	0.00141647798659832\\
2	3.8	0.000762255266138726	0.000762255266138726\\
2	3.9	0.000361510247624548	0.000361510247624548\\
2	4	0.00017570442017274	0.00017570442017274\\
2	4.1	9.30092019844424e-05	9.30092019844424e-05\\
2	4.2	5.67131788435179e-05	5.67131788435179e-05\\
2	4.3	4.20890653093821e-05	4.20890653093821e-05\\
2	4.4	3.90390367854819e-05	3.90390367854819e-05\\
2	4.5	4.32357051648567e-05	4.32357051648567e-05\\
2	4.6	5.20713359901357e-05	5.20713359901357e-05\\
2	4.7	5.95102793663858e-05	5.95102793663858e-05\\
2	4.8	4.13010656939455e-05	4.13010656939455e-05\\
2	4.9	8.12109326242044e-06	8.12109326242044e-06\\
2	5	3.63453141640993e-05	3.63453141640993e-05\\
2	5.1	5.93105255173474e-06	5.93105255173474e-06\\
2	5.2	2.13325228948453e-05	2.13325228948453e-05\\
2	5.3	3.71459358722816e-05	3.71459358722816e-05\\
2	5.4	4.91932133960597e-05	4.91932133960597e-05\\
2	5.5	5.38872336673362e-05	5.38872336673362e-05\\
2	5.6	5.75933934504845e-05	5.75933934504845e-05\\
2	5.7	6.96872404314257e-05	6.96872404314257e-05\\
2	5.8	0.000101030226498315	0.000101030226498315\\
2	5.9	0.000176695665864773	0.000176695665864773\\
2	6	0.000361067708741745	0.000361067708741745\\
2.1	0	0.000532721818966856	0.000532721818966856\\
2.1	0.1	0.000319925126869466	0.000319925126869466\\
2.1	0.2	0.000205144010869289	0.000205144010869289\\
2.1	0.3	0.000140106422286895	0.000140106422286895\\
2.1	0.4	0.000100132514413679	0.000100132514413679\\
2.1	0.5	7.06257822911653e-05	7.06257822911653e-05\\
2.1	0.6	4.47311373324329e-05	4.47311373324329e-05\\
2.1	0.7	2.44812651350481e-05	2.44812651350481e-05\\
2.1	0.8	1.52455199540645e-05	1.52455199540645e-05\\
2.1	0.9	1.33450196203621e-05	1.33450196203621e-05\\
2.1	1	1.91150303302894e-05	1.91150303302894e-05\\
2.1	1.1	2.79615140060011e-05	2.79615140060011e-05\\
2.1	1.2	4.29966662892601e-05	4.29966662892601e-05\\
2.1	1.3	7.74557237685969e-05	7.74557237685969e-05\\
2.1	1.4	0.000137321722636924	0.000137321722636924\\
2.1	1.5	0.00023115024377624	0.00023115024377624\\
2.1	1.6	0.000306929038410542	0.000306929038410542\\
2.1	1.7	0.000336278324580015	0.000336278324580015\\
2.1	1.8	0.000359475861966844	0.000359475861966844\\
2.1	1.9	0.000397721036643844	0.000397721036643844\\
2.1	2	0.000458252555569231	0.000458252555569231\\
2.1	2.1	0.000533352329983611	0.000533352329983611\\
2.1	2.2	0.000568591581888036	0.000568591581888036\\
2.1	2.3	0.000505247511196969	0.000505247511196969\\
2.1	2.4	0.000348708928063779	0.000348708928063779\\
2.1	2.5	0.000200285303553564	0.000200285303553564\\
2.1	2.6	0.000115124976916424	0.000115124976916424\\
2.1	2.7	7.37883049521926e-05	7.37883049521926e-05\\
2.1	2.8	5.23423364512875e-05	5.23423364512875e-05\\
2.1	2.9	4.41152476178053e-05	4.41152476178053e-05\\
2.1	3	4.80750542495585e-05	4.80750542495585e-05\\
2.1	3.1	6.86753618206205e-05	6.86753618206205e-05\\
2.1	3.2	0.000126162863664358	0.000126162863664358\\
2.1	3.3	0.000280231720620125	0.000280231720620125\\
2.1	3.4	0.000650411276559804	0.000650411276559804\\
2.1	3.5	0.0012695957822734	0.0012695957822734\\
2.1	3.6	0.00167760919104566	0.00167760919104566\\
2.1	3.7	0.00138968242523112	0.00138968242523112\\
2.1	3.8	0.000821851653220743	0.000821851653220743\\
2.1	3.9	0.00043149932375891	0.00043149932375891\\
2.1	4	0.000236517567108467	0.000236517567108467\\
2.1	4.1	0.000146130574394282	0.000146130574394282\\
2.1	4.2	0.000107754853223003	0.000107754853223003\\
2.1	4.3	9.91930174521477e-05	9.91930174521477e-05\\
2.1	4.4	0.00011018183556094	0.00011018183556094\\
2.1	4.5	0.000128296442724068	0.000128296442724068\\
2.1	4.6	0.000121356146845314	0.000121356146845314\\
2.1	4.7	6.03379621575278e-05	6.03379621575278e-05\\
2.1	4.8	1.51151806349364e-05	1.51151806349364e-05\\
2.1	4.9	1.53288367922321e-05	1.53288367922321e-05\\
2.1	5	3.38131269111526e-05	3.38131269111526e-05\\
2.1	5.1	1.35251677357709e-05	1.35251677357709e-05\\
2.1	5.2	7.6265284964818e-06	7.6265284964818e-06\\
2.1	5.3	2.82852087661477e-05	2.82852087661477e-05\\
2.1	5.4	5.1870909750636e-05	5.1870909750636e-05\\
2.1	5.5	4.51132293836824e-05	4.51132293836824e-05\\
2.1	5.6	3.59830968399569e-05	3.59830968399569e-05\\
2.1	5.7	3.9500205070788e-05	3.9500205070788e-05\\
2.1	5.8	6.62114072559237e-05	6.62114072559237e-05\\
2.1	5.9	0.000154398350889828	0.000154398350889828\\
2.1	6	0.000424749951758138	0.000424749951758138\\
2.2	0	0.000596080257098469	0.000596080257098469\\
2.2	0.1	0.000352653039243044	0.000352653039243044\\
2.2	0.2	0.000217471659911652	0.000217471659911652\\
2.2	0.3	0.000140560345197644	0.000140560345197644\\
2.2	0.4	9.68589134089626e-05	9.68589134089626e-05\\
2.2	0.5	7.02509457475636e-05	7.02509457475636e-05\\
2.2	0.6	4.76826116257201e-05	4.76826116257201e-05\\
2.2	0.7	2.54752558255907e-05	2.54752558255907e-05\\
2.2	0.8	1.33911935014755e-05	1.33911935014755e-05\\
2.2	0.9	1.13921582212643e-05	1.13921582212643e-05\\
2.2	1	1.78596412365761e-05	1.78596412365761e-05\\
2.2	1.1	2.46625072925762e-05	2.46625072925762e-05\\
2.2	1.2	4.72938611979572e-05	4.72938611979572e-05\\
2.2	1.3	7.32179902548982e-05	7.32179902548982e-05\\
2.2	1.4	0.00010136354848576	0.00010136354848576\\
2.2	1.5	0.000154306957398977	0.000154306957398977\\
2.2	1.6	0.000233780553041339	0.000233780553041339\\
2.2	1.7	0.000310801653523755	0.000310801653523755\\
2.2	1.8	0.00036270570945431	0.00036270570945431\\
2.2	1.9	0.000407910222383334	0.000407910222383334\\
2.2	2	0.000494139772625298	0.000494139772625298\\
2.2	2.1	0.000649867188878595	0.000649867188878595\\
2.2	2.2	0.00080392935729745	0.00080392935729745\\
2.2	2.3	0.000802185775830672	0.000802185775830672\\
2.2	2.4	0.000607233275010979	0.000607233275010979\\
2.2	2.5	0.000371013304864277	0.000371013304864277\\
2.2	2.6	0.000214588323002377	0.000214588323002377\\
2.2	2.7	0.000135533757869246	0.000135533757869246\\
2.2	2.8	9.67538155607979e-05	9.67538155607979e-05\\
2.2	2.9	8.04118208638317e-05	8.04118208638317e-05\\
2.2	3	8.27220684481312e-05	8.27220684481312e-05\\
2.2	3.1	0.000108370815145664	0.000108370815145664\\
2.2	3.2	0.000178401194356636	0.000178401194356636\\
2.2	3.3	0.000346818091871343	0.000346818091871343\\
2.2	3.4	0.000697899014560006	0.000697899014560006\\
2.2	3.5	0.00120649622893474	0.00120649622893474\\
2.2	3.6	0.00151872359573041	0.00151872359573041\\
2.2	3.7	0.00134818930379933	0.00134818930379933\\
2.2	3.8	0.000927863627390619	0.000927863627390619\\
2.2	3.9	0.000548482459835126	0.000548482459835126\\
2.2	4	0.00030766711500586	0.00030766711500586\\
2.2	4.1	0.000187435209428112	0.000187435209428112\\
2.2	4.2	0.000142471550163207	0.000142471550163207\\
2.2	4.3	0.00014359046471087	0.00014359046471087\\
2.2	4.4	0.000173095082154428	0.000173095082154428\\
2.2	4.5	0.000192565619448254	0.000192565619448254\\
2.2	4.6	0.000151102080921958	0.000151102080921958\\
2.2	4.7	8.07827468249862e-05	8.07827468249862e-05\\
2.2	4.8	3.86544171126227e-05	3.86544171126227e-05\\
2.2	4.9	2.75493533795858e-05	2.75493533795858e-05\\
2.2	5	3.19626964064774e-05	3.19626964064774e-05\\
2.2	5.1	2.30751159196741e-05	2.30751159196741e-05\\
2.2	5.2	1.71243434986478e-05	1.71243434986478e-05\\
2.2	5.3	2.32401999015928e-05	2.32401999015928e-05\\
2.2	5.4	3.08540541964515e-05	3.08540541964515e-05\\
2.2	5.5	2.72892807982706e-05	2.72892807982706e-05\\
2.2	5.6	2.29728710217262e-05	2.29728710217262e-05\\
2.2	5.7	2.69984687519336e-05	2.69984687519336e-05\\
2.2	5.8	4.9563140420566e-05	4.9563140420566e-05\\
2.2	5.9	0.000126414988563017	0.000126414988563017\\
2.2	6	0.000370980851730362	0.000370980851730362\\
2.3	0	0.000673253161275998	0.000673253161275998\\
2.3	0.1	0.000411470820912456	0.000411470820912456\\
2.3	0.2	0.000267488773166996	0.000267488773166996\\
2.3	0.3	0.000185928897511222	0.000185928897511222\\
2.3	0.4	0.000138480759324613	0.000138480759324613\\
2.3	0.5	0.000107647714893391	0.000107647714893391\\
2.3	0.6	8.01469786669558e-05	8.01469786669558e-05\\
2.3	0.7	4.48106613880585e-05	4.48106613880585e-05\\
2.3	0.8	1.889539681087e-05	1.889539681087e-05\\
2.3	0.9	1.21122102150012e-05	1.21122102150012e-05\\
2.3	1	1.7040009967976e-05	1.7040009967976e-05\\
2.3	1.1	2.49224977239537e-05	2.49224977239537e-05\\
2.3	1.2	4.35237156656821e-05	4.35237156656821e-05\\
2.3	1.3	6.37875087456692e-05	6.37875087456692e-05\\
2.3	1.4	9.31544001280504e-05	9.31544001280504e-05\\
2.3	1.5	0.000155056501856531	0.000155056501856531\\
2.3	1.6	0.00025385902024046	0.00025385902024046\\
2.3	1.7	0.000345552013956213	0.000345552013956213\\
2.3	1.8	0.000403709933316748	0.000403709933316748\\
2.3	1.9	0.000469399154837639	0.000469399154837639\\
2.3	2	0.000631809248414817	0.000631809248414817\\
2.3	2.1	0.000999488525539657	0.000999488525539657\\
2.3	2.2	0.00153437294519384	0.00153437294519384\\
2.3	2.3	0.00177907781578863	0.00177907781578863\\
2.3	2.4	0.0013750408820448	0.0013750408820448\\
2.3	2.5	0.000783467329230797	0.000783467329230797\\
2.3	2.6	0.000420028291201985	0.000420028291201985\\
2.3	2.7	0.000254942914222245	0.000254942914222245\\
2.3	2.8	0.000181859371754366	0.000181859371754366\\
2.3	2.9	0.000150489715496418	0.000150489715496418\\
2.3	3	0.000147522181351409	0.000147522181351409\\
2.3	3.1	0.000174864445078349	0.000174864445078349\\
2.3	3.2	0.000248602623271281	0.000248602623271281\\
2.3	3.3	0.00040432403166905	0.00040432403166905\\
2.3	3.4	0.000680267601944386	0.000680267601944386\\
2.3	3.5	0.00102910941833635	0.00102910941833635\\
2.3	3.6	0.00124421924426794	0.00124421924426794\\
2.3	3.7	0.00118030533044156	0.00118030533044156\\
2.3	3.8	0.000923059454270853	0.000923059454270853\\
2.3	3.9	0.000609732662406412	0.000609732662406412\\
2.3	4	0.000357312682760133	0.000357312682760133\\
2.3	4.1	0.00021494309282974	0.00021494309282974\\
2.3	4.2	0.00015734171787893	0.00015734171787893\\
2.3	4.3	0.000152445125092773	0.000152445125092773\\
2.3	4.4	0.000186206324119756	0.000186206324119756\\
2.3	4.5	0.000241826685438993	0.000241826685438993\\
2.3	4.6	0.000270124200323178	0.000270124200323178\\
2.3	4.7	0.000222532134033286	0.000222532134033286\\
2.3	4.8	0.000141218782968393	0.000141218782968393\\
2.3	4.9	5.91126321068419e-05	5.91126321068419e-05\\
2.3	5	3.16857415540811e-05	3.16857415540811e-05\\
2.3	5.1	4.09791852504906e-05	4.09791852504906e-05\\
2.3	5.2	4.96703973875706e-05	4.96703973875706e-05\\
2.3	5.3	4.08259157907242e-05	4.08259157907242e-05\\
2.3	5.4	4.52528252732268e-05	4.52528252732268e-05\\
2.3	5.5	4.05795147175945e-05	4.05795147175945e-05\\
2.3	5.6	3.2880221071666e-05	3.2880221071666e-05\\
2.3	5.7	3.40001856780805e-05	3.40001856780805e-05\\
2.3	5.8	5.08364440653553e-05	5.08364440653553e-05\\
2.3	5.9	0.000103030992233631	0.000103030992233631\\
2.3	6	0.000249604334604155	0.000249604334604155\\
2.4	0	0.000627454960035916	0.000627454960035916\\
2.4	0.1	0.000376240811559626	0.000376240811559626\\
2.4	0.2	0.000247082009367345	0.000247082009367345\\
2.4	0.3	0.000177546346851138	0.000177546346851138\\
2.4	0.4	0.000136699861858365	0.000136699861858365\\
2.4	0.5	0.000109250619626943	0.000109250619626943\\
2.4	0.6	8.9610077808474e-05	8.9610077808474e-05\\
2.4	0.7	7.71243591235413e-05	7.71243591235413e-05\\
2.4	0.8	6.18159011869871e-05	6.18159011869871e-05\\
2.4	0.9	3.07779767068116e-05	3.07779767068116e-05\\
2.4	1	1.72015418066545e-05	1.72015418066545e-05\\
2.4	1.1	4.86574144964723e-05	4.86574144964723e-05\\
2.4	1.2	0.000148125469860655	0.000148125469860655\\
2.4	1.3	0.000220054100407959	0.000220054100407959\\
2.4	1.4	0.000238699077415113	0.000238699077415113\\
2.4	1.5	0.000278119712376064	0.000278119712376064\\
2.4	1.6	0.000342090930178663	0.000342090930178663\\
2.4	1.7	0.00039993644054461	0.00039993644054461\\
2.4	1.8	0.00045339399895324	0.00045339399895324\\
2.4	1.9	0.00057138620741673	0.00057138620741673\\
2.4	2	0.00092082195385137	0.00092082195385137\\
2.4	2.1	0.00188105679341941	0.00188105679341941\\
2.4	2.2	0.00379154777858179	0.00379154777858179\\
2.4	2.3	0.0052101738266836	0.0052101738266836\\
2.4	2.4	0.00391881152011732	0.00391881152011732\\
2.4	2.5	0.00187378781431168	0.00187378781431168\\
2.4	2.6	0.000836545841655884	0.000836545841655884\\
2.4	2.7	0.000454153732531464	0.000454153732531464\\
2.4	2.8	0.000310304446309699	0.000310304446309699\\
2.4	2.9	0.000252180151303195	0.000252180151303195\\
2.4	3	0.000238161719400641	0.000238161719400641\\
2.4	3.1	0.000261866307837349	0.000261866307837349\\
2.4	3.2	0.000331455879894149	0.000331455879894149\\
2.4	3.3	0.000464440693056341	0.000464440693056341\\
2.4	3.4	0.000668832757474909	0.000668832757474909\\
2.4	3.5	0.00089768595443098	0.00089768595443098\\
2.4	3.6	0.00103680680157239	0.00103680680157239\\
2.4	3.7	0.00101193014947137	0.00101193014947137\\
2.4	3.8	0.000846163391779002	0.000846163391779002\\
2.4	3.9	0.000605111859653938	0.000605111859653938\\
2.4	4	0.000386830289638222	0.000386830289638222\\
2.4	4.1	0.000252143187870971	0.000252143187870971\\
2.4	4.2	0.000191630777118858	0.000191630777118858\\
2.4	4.3	0.000179261215506071	0.000179261215506071\\
2.4	4.4	0.000203011587796885	0.000203011587796885\\
2.4	4.5	0.000272561805287829	0.000272561805287829\\
2.4	4.6	0.000433475703446707	0.000433475703446707\\
2.4	4.7	0.000689937661776671	0.000689937661776671\\
2.4	4.8	0.000642024612601657	0.000642024612601657\\
2.4	4.9	0.000275456473344245	0.000275456473344245\\
2.4	5	3.38218261589711e-05	3.38218261589711e-05\\
2.4	5.1	0.00013875306353376	0.00013875306353376\\
2.4	5.2	0.000197113518571516	0.000197113518571516\\
2.4	5.3	0.000307365003293043	0.000307365003293043\\
2.4	5.4	0.000258056774202005	0.000258056774202005\\
2.4	5.5	0.000123488589516846	0.000123488589516846\\
2.4	5.6	5.89808238591466e-05	5.89808238591466e-05\\
2.4	5.7	4.09492464101461e-05	4.09492464101461e-05\\
2.4	5.8	4.37967934948429e-05	4.37967934948429e-05\\
2.4	5.9	6.68354382298571e-05	6.68354382298571e-05\\
2.4	6	0.000132788604331806	0.000132788604331806\\
2.5	0	0.000468712045614344	0.000468712045614344\\
2.5	0.1	0.000275181572812947	0.000275181572812947\\
2.5	0.2	0.000184880854009823	0.000184880854009823\\
2.5	0.3	0.000139357511578418	0.000139357511578418\\
2.5	0.4	0.000113195065217977	0.000113195065217977\\
2.5	0.5	9.7795231525115e-05	9.7795231525115e-05\\
2.5	0.6	9.46071873019306e-05	9.46071873019306e-05\\
2.5	0.7	0.000114455136561676	0.000114455136561676\\
2.5	0.8	0.000180870735641471	0.000180870735641471\\
2.5	0.9	0.000319891647242618	0.000319891647242618\\
2.5	1	0.000492915706897511	0.000492915706897511\\
2.5	1.1	0.000565803480674455	0.000565803480674455\\
2.5	1.2	0.000493164179539095	0.000493164179539095\\
2.5	1.3	0.000383252956080476	0.000383252956080476\\
2.5	1.4	0.000314934383673551	0.000314934383673551\\
2.5	1.5	0.000297005084209061	0.000297005084209061\\
2.5	1.6	0.000318195911756891	0.000318195911756891\\
2.5	1.7	0.000362147798976746	0.000362147798976746\\
2.5	1.8	0.000430702125036276	0.000430702125036276\\
2.5	1.9	0.000603361434838638	0.000603361434838638\\
2.5	2	0.00116502162284623	0.00116502162284623\\
2.5	2.1	0.0030648312810651	0.0030648312810651\\
2.5	2.2	0.00808747160986983	0.00808747160986983\\
2.5	2.3	0.0133259846736692	0.0133259846736692\\
2.5	2.4	0.0101587965317985	0.0101587965317985\\
2.5	2.5	0.00422604365076925	0.00422604365076925\\
2.5	2.6	0.00156109585876558	0.00156109585876558\\
2.5	2.7	0.000733067008995123	0.000733067008995123\\
2.5	2.8	0.000464125678148388	0.000464125678148388\\
2.5	2.9	0.000365207582479301	0.000365207582479301\\
2.5	3	0.000336645036670336	0.000336645036670336\\
2.5	3.1	0.000356972086966243	0.000356972086966243\\
2.5	3.2	0.000426746045354042	0.000426746045354042\\
2.5	3.3	0.000551269001293851	0.000551269001293851\\
2.5	3.4	0.000721525995470875	0.000721525995470875\\
2.5	3.5	0.000887249145955341	0.000887249145955341\\
2.5	3.6	0.000969188931464563	0.000969188931464563\\
2.5	3.7	0.000931837892604482	0.000931837892604482\\
2.5	3.8	0.00079314368266174	0.00079314368266174\\
2.5	3.9	0.000595278685803899	0.000595278685803899\\
2.5	4	0.00041323557961596	0.00041323557961596\\
2.5	4.1	0.000296264006263993	0.000296264006263993\\
2.5	4.2	0.00023455363517221	0.00023455363517221\\
2.5	4.3	0.000198388845816783	0.000198388845816783\\
2.5	4.4	0.000169810723853306	0.000169810723853306\\
2.5	4.5	0.000150817902308582	0.000150817902308582\\
2.5	4.6	0.000153695658814063	0.000153695658814063\\
2.5	4.7	0.000190603656090273	0.000190603656090273\\
2.5	4.8	0.000263693652370365	0.000263693652370365\\
2.5	4.9	0.000336518475955728	0.000336518475955728\\
2.5	5	0.000342249614765298	0.000342249614765298\\
2.5	5.1	0.000273391933424835	0.000273391933424835\\
2.5	5.2	0.000185325622260169	0.000185325622260169\\
2.5	5.3	0.00011542634538387	0.00011542634538387\\
2.5	5.4	6.95454787813659e-05	6.95454787813659e-05\\
2.5	5.5	4.3227334260166e-05	4.3227334260166e-05\\
2.5	5.6	3.02732375907354e-05	3.02732375907354e-05\\
2.5	5.7	2.59079889555618e-05	2.59079889555618e-05\\
2.5	5.8	2.85073089830659e-05	2.85073089830659e-05\\
2.5	5.9	4.11309399394621e-05	4.11309399394621e-05\\
2.5	6	7.77367698733032e-05	7.77367698733032e-05\\
2.6	0	0.000402830158195972	0.000402830158195972\\
2.6	0.1	0.000247182758233496	0.000247182758233496\\
2.6	0.2	0.000174153680120786	0.000174153680120786\\
2.6	0.3	0.000134475631786495	0.000134475631786495\\
2.6	0.4	0.000107943113056958	0.000107943113056958\\
2.6	0.5	8.80433013958403e-05	8.80433013958403e-05\\
2.6	0.6	7.29736736402258e-05	7.29736736402258e-05\\
2.6	0.7	5.89014360178279e-05	5.89014360178279e-05\\
2.6	0.8	3.9192489118778e-05	3.9192489118778e-05\\
2.6	0.9	1.62285784849151e-05	1.62285784849151e-05\\
2.6	1	4.83185823174002e-06	4.83185823174002e-06\\
2.6	1.1	1.93003999786878e-05	1.93003999786878e-05\\
2.6	1.2	6.71569049787338e-05	6.71569049787338e-05\\
2.6	1.3	0.000127806949457206	0.000127806949457206\\
2.6	1.4	0.000177124335951358	0.000177124335951358\\
2.6	1.5	0.000212728303019231	0.000212728303019231\\
2.6	1.6	0.000242511454991779	0.000242511454991779\\
2.6	1.7	0.000275095661617784	0.000275095661617784\\
2.6	1.8	0.000330443573329715	0.000330443573329715\\
2.6	1.9	0.000488343824142787	0.000488343824142787\\
2.6	2	0.00104516432216433	0.00104516432216433\\
2.6	2.1	0.00316683186015044	0.00316683186015044\\
2.6	2.2	0.00974952777394843	0.00974952777394843\\
2.6	2.3	0.018410687943751	0.018410687943751\\
2.6	2.4	0.0153895031221958	0.0153895031221958\\
2.6	2.5	0.0065367529364846	0.0065367529364846\\
2.6	2.6	0.00229281503061276	0.00229281503061276\\
2.6	2.7	0.000996700563820708	0.000996700563820708\\
2.6	2.8	0.00059845675312122	0.00059845675312122\\
2.6	2.9	0.000462505831563466	0.000462505831563466\\
2.6	3	0.000424979351565246	0.000424979351565246\\
2.6	3.1	0.000447198174328776	0.000447198174328776\\
2.6	3.2	0.000523968428366789	0.000523968428366789\\
2.6	3.3	0.00065087512924081	0.00065087512924081\\
2.6	3.4	0.000801173103271739	0.000801173103271739\\
2.6	3.5	0.000914871297989019	0.000914871297989019\\
2.6	3.6	0.000936866572527492	0.000936866572527492\\
2.6	3.7	0.000866919498786283	0.000866919498786283\\
2.6	3.8	0.000730658736520801	0.000730658736520801\\
2.6	3.9	0.000558669198816064	0.000558669198816064\\
2.6	4	0.00040408163156729	0.00040408163156729\\
2.6	4.1	0.000301817348671013	0.000301817348671013\\
2.6	4.2	0.000240428842898132	0.000240428842898132\\
2.6	4.3	0.000191880233888449	0.000191880233888449\\
2.6	4.4	0.000143541119803698	0.000143541119803698\\
2.6	4.5	0.00010183816735414	0.00010183816735414\\
2.6	4.6	7.49499216901137e-05	7.49499216901137e-05\\
2.6	4.7	6.31454513347426e-05	6.31454513347426e-05\\
2.6	4.8	5.73801677022396e-05	5.73801677022396e-05\\
2.6	4.9	3.74914857827048e-05	3.74914857827048e-05\\
2.6	5	3.0235068061305e-05	3.0235068061305e-05\\
2.6	5.1	1.85576045244873e-05	1.85576045244873e-05\\
2.6	5.2	2.08889036923202e-05	2.08889036923202e-05\\
2.6	5.3	1.4670040171712e-05	1.4670040171712e-05\\
2.6	5.4	1.02532728353833e-05	1.02532728353833e-05\\
2.6	5.5	9.3733374361773e-06	9.3733374361773e-06\\
2.6	5.6	1.08950124208425e-05	1.08950124208425e-05\\
2.6	5.7	1.42422941661884e-05	1.42422941661884e-05\\
2.6	5.8	2.01018542365677e-05	2.01018542365677e-05\\
2.6	5.9	3.24213719012633e-05	3.24213719012633e-05\\
2.6	6	6.48101735203633e-05	6.48101735203633e-05\\
2.7	0	0.000420238363406386	0.000420238363406386\\
2.7	0.1	0.000263299319624895	0.000263299319624895\\
2.7	0.2	0.000185016175073666	0.000185016175073666\\
2.7	0.3	0.00013836540763768	0.00013836540763768\\
2.7	0.4	0.000103107325394895	0.000103107325394895\\
2.7	0.5	7.35587017298709e-05	7.35587017298709e-05\\
2.7	0.6	5.08261340023005e-05	5.08261340023005e-05\\
2.7	0.7	3.32258561479916e-05	3.32258561479916e-05\\
2.7	0.8	1.59138959361274e-05	1.59138959361274e-05\\
2.7	0.9	5.43887352297996e-06	5.43887352297996e-06\\
2.7	1	6.33913009896238e-06	6.33913009896238e-06\\
2.7	1.1	6.36705355926202e-06	6.36705355926202e-06\\
2.7	1.2	1.61331609871967e-05	1.61331609871967e-05\\
2.7	1.3	4.23131392291226e-05	4.23131392291226e-05\\
2.7	1.4	8.01777732602021e-05	8.01777732602021e-05\\
2.7	1.5	0.000129122063136845	0.000129122063136845\\
2.7	1.6	0.000178310710663462	0.000178310710663462\\
2.7	1.7	0.000211376680771147	0.000211376680771147\\
2.7	1.8	0.000242735019297175	0.000242735019297175\\
2.7	1.9	0.000336915657148688	0.000336915657148688\\
2.7	2	0.00068297984200358	0.00068297984200358\\
2.7	2.1	0.00198321366573021	0.00198321366573021\\
2.7	2.2	0.00602873063887221	0.00602873063887221\\
2.7	2.3	0.0120589689323927	0.0120589689323927\\
2.7	2.4	0.0116942041475831	0.0116942041475831\\
2.7	2.5	0.00598709977868918	0.00598709977868918\\
2.7	2.6	0.00240007946565096	0.00240007946565096\\
2.7	2.7	0.00109943259978154	0.00109943259978154\\
2.7	2.8	0.000670524346508134	0.000670524346508134\\
2.7	2.9	0.00052746349110027	0.00052746349110027\\
2.7	3	0.000492566887761675	0.000492566887761675\\
2.7	3.1	0.000518449745714081	0.000518449745714081\\
2.7	3.2	0.000598133199880744	0.000598133199880744\\
2.7	3.3	0.000719642872620897	0.000719642872620897\\
2.7	3.4	0.000838264126187578	0.000838264126187578\\
2.7	3.5	0.000892319242155034	0.000892319242155034\\
2.7	3.6	0.000859029339475259	0.000859029339475259\\
2.7	3.7	0.000762427792293684	0.000762427792293684\\
2.7	3.8	0.000629649538140716	0.000629649538140716\\
2.7	3.9	0.000483441768685996	0.000483441768685996\\
2.7	4	0.000355102734849808	0.000355102734849808\\
2.7	4.1	0.000267174261957514	0.000267174261957514\\
2.7	4.2	0.000215951413049263	0.000215951413049263\\
2.7	4.3	0.000182854428484489	0.000182854428484489\\
2.7	4.4	0.000150937008296245	0.000150937008296245\\
2.7	4.5	0.000116070038240793	0.000116070038240793\\
2.7	4.6	8.76000802178551e-05	8.76000802178551e-05\\
2.7	4.7	6.83470065538667e-05	6.83470065538667e-05\\
2.7	4.8	4.37898855909899e-05	4.37898855909899e-05\\
2.7	4.9	2.25245126061928e-05	2.25245126061928e-05\\
2.7	5	2.34087798465275e-05	2.34087798465275e-05\\
2.7	5.1	1.07380227697627e-05	1.07380227697627e-05\\
2.7	5.2	8.42521177022401e-06	8.42521177022401e-06\\
2.7	5.3	7.74557200013918e-06	7.74557200013918e-06\\
2.7	5.4	6.85350680776842e-06	6.85350680776842e-06\\
2.7	5.5	6.31659310523955e-06	6.31659310523955e-06\\
2.7	5.6	7.31373465214823e-06	7.31373465214823e-06\\
2.7	5.7	1.07144306471926e-05	1.07144306471926e-05\\
2.7	5.8	1.77537719531206e-05	1.77537719531206e-05\\
2.7	5.9	3.22939432510813e-05	3.22939432510813e-05\\
2.7	6	6.93567560875493e-05	6.93567560875493e-05\\
2.8	0	0.00046555134579105	0.00046555134579105\\
2.8	0.1	0.000283305399854008	0.000283305399854008\\
2.8	0.2	0.000192832776144954	0.000192832776144954\\
2.8	0.3	0.000142304246728928	0.000142304246728928\\
2.8	0.4	0.000109435688506199	0.000109435688506199\\
2.8	0.5	8.30467813938927e-05	8.30467813938927e-05\\
2.8	0.6	5.5654658737758e-05	5.5654658737758e-05\\
2.8	0.7	2.76158787294257e-05	2.76158787294257e-05\\
2.8	0.8	1.03655541278869e-05	1.03655541278869e-05\\
2.8	0.9	6.5173058299793e-06	6.5173058299793e-06\\
2.8	1	8.47087196334736e-06	8.47087196334736e-06\\
2.8	1.1	8.4809691241228e-06	8.4809691241228e-06\\
2.8	1.2	9.50686911533936e-06	9.50686911533936e-06\\
2.8	1.3	1.81111729034798e-05	1.81111729034798e-05\\
2.8	1.4	3.7969701377277e-05	3.7969701377277e-05\\
2.8	1.5	7.19004167262437e-05	7.19004167262437e-05\\
2.8	1.6	0.000118832688903198	0.000118832688903198\\
2.8	1.7	0.000160537299651569	0.000160537299651569\\
2.8	1.8	0.000187667858824583	0.000187667858824583\\
2.8	1.9	0.000238507875922642	0.000238507875922642\\
2.8	2	0.000414178663171227	0.000414178663171227\\
2.8	2.1	0.00100912338898273	0.00100912338898273\\
2.8	2.2	0.00268135594887848	0.00268135594887848\\
2.8	2.3	0.00527759137581685	0.00527759137581685\\
2.8	2.4	0.00591240766865094	0.00591240766865094\\
2.8	2.5	0.0038940558514526	0.0038940558514526\\
2.8	2.6	0.00198372506897664	0.00198372506897664\\
2.8	2.7	0.00105666825618104	0.00105666825618104\\
2.8	2.8	0.00069390924734296	0.00069390924734296\\
2.8	2.9	0.000566709131833729	0.000566709131833729\\
2.8	3	0.000537655593107924	0.000537655593107924\\
2.8	3.1	0.000563053262928567	0.000563053262928567\\
2.8	3.2	0.000637300365231727	0.000637300365231727\\
2.8	3.3	0.000742894592402716	0.000742894592402716\\
2.8	3.4	0.000824471012311182	0.000824471012311182\\
2.8	3.5	0.000830789264220219	0.000830789264220219\\
2.8	3.6	0.000764857616721006	0.000764857616721006\\
2.8	3.7	0.000657293088551074	0.000657293088551074\\
2.8	3.8	0.000534503415863654	0.000534503415863654\\
2.8	3.9	0.000414783253746477	0.000414783253746477\\
2.8	4	0.000311745499341027	0.000311745499341027\\
2.8	4.1	0.000239550481606904	0.000239550481606904\\
2.8	4.2	0.00020356530866102	0.00020356530866102\\
2.8	4.3	0.000191662203222808	0.000191662203222808\\
2.8	4.4	0.00017600752669324	0.00017600752669324\\
2.8	4.5	0.000139196450492401	0.000139196450492401\\
2.8	4.6	9.62029696207268e-05	9.62029696207268e-05\\
2.8	4.7	5.69600162859239e-05	5.69600162859239e-05\\
2.8	4.8	2.87680383909573e-05	2.87680383909573e-05\\
2.8	4.9	2.2320888961539e-05	2.2320888961539e-05\\
2.8	5	2.03097449923744e-05	2.03097449923744e-05\\
2.8	5.1	1.05787339195822e-05	1.05787339195822e-05\\
2.8	5.2	5.43310344262961e-06	5.43310344262961e-06\\
2.8	5.3	4.58172456336619e-06	4.58172456336619e-06\\
2.8	5.4	4.61438214139567e-06	4.61438214139567e-06\\
2.8	5.5	5.55675019863553e-06	5.55675019863553e-06\\
2.8	5.6	7.55846126977571e-06	7.55846126977571e-06\\
2.8	5.7	1.13103908957391e-05	1.13103908957391e-05\\
2.8	5.8	1.90400613074899e-05	1.90400613074899e-05\\
2.8	5.9	3.66228720957897e-05	3.66228720957897e-05\\
2.8	6	8.34853991620208e-05	8.34853991620208e-05\\
2.9	0	0.000509925582616459	0.000509925582616459\\
2.9	0.1	0.000291916737190883	0.000291916737190883\\
2.9	0.2	0.000192529355324637	0.000192529355324637\\
2.9	0.3	0.000145112192420436	0.000145112192420436\\
2.9	0.4	0.000118586611303159	0.000118586611303159\\
2.9	0.5	9.08158884246689e-05	9.08158884246689e-05\\
2.9	0.6	5.31711061921927e-05	5.31711061921927e-05\\
2.9	0.7	2.29287207727553e-05	2.29287207727553e-05\\
2.9	0.8	1.0817912911146e-05	1.0817912911146e-05\\
2.9	0.9	9.04962740435664e-06	9.04962740435664e-06\\
2.9	1	1.1295402349261e-05	1.1295402349261e-05\\
2.9	1.1	1.20476525113106e-05	1.20476525113106e-05\\
2.9	1.2	1.10494029401695e-05	1.10494029401695e-05\\
2.9	1.3	1.28457004383646e-05	1.28457004383646e-05\\
2.9	1.4	2.04228949647728e-05	2.04228949647728e-05\\
2.9	1.5	3.855493668089e-05	3.855493668089e-05\\
2.9	1.6	7.25740918648309e-05	7.25740918648309e-05\\
2.9	1.7	0.000115457515156019	0.000115457515156019\\
2.9	1.8	0.00015042731261847	0.00015042731261847\\
2.9	1.9	0.000189937679524592	0.000189937679524592\\
2.9	2	0.000296417394914131	0.000296417394914131\\
2.9	2.1	0.000620899788199697	0.000620899788199697\\
2.9	2.2	0.00144173611472997	0.00144173611472997\\
2.9	2.3	0.00268327585341389	0.00268327585341389\\
2.9	2.4	0.00322287539965382	0.00322287539965382\\
2.9	2.5	0.00254574798072083	0.00254574798072083\\
2.9	2.6	0.00160300121020069	0.00160300121020069\\
2.9	2.7	0.00100205157694161	0.00100205157694161\\
2.9	2.8	0.00071371789585882	0.00071371789585882\\
2.9	2.9	0.000595369548050781	0.000595369548050781\\
2.9	3	0.000559054050631234	0.000559054050631234\\
2.9	3.1	0.00057224719170313	0.00057224719170313\\
2.9	3.2	0.000630884274033527	0.000630884274033527\\
2.9	3.3	0.000715051220039237	0.000715051220039237\\
2.9	3.4	0.000771478221495801	0.000771478221495801\\
2.9	3.5	0.000761123333102553	0.000761123333102553\\
2.9	3.6	0.000691383399724938	0.000691383399724938\\
2.9	3.7	0.000585711127650902	0.000585711127650902\\
2.9	3.8	0.000468659117058204	0.000468659117058204\\
2.9	3.9	0.000359054988764096	0.000359054988764096\\
2.9	4	0.000268943949240514	0.000268943949240514\\
2.9	4.1	0.000211644345605912	0.000211644345605912\\
2.9	4.2	0.000191948976154053	0.000191948976154053\\
2.9	4.3	0.000191745285302819	0.000191745285302819\\
2.9	4.4	0.000170728208280129	0.000170728208280129\\
2.9	4.5	0.000118618103208045	0.000118618103208045\\
2.9	4.6	6.86717337140403e-05	6.86717337140403e-05\\
2.9	4.7	3.69605404558241e-05	3.69605404558241e-05\\
2.9	4.8	2.42592884281415e-05	2.42592884281415e-05\\
2.9	4.9	2.30135766122535e-05	2.30135766122535e-05\\
2.9	5	1.9106663171037e-05	1.9106663171037e-05\\
2.9	5.1	1.0431322102553e-05	1.0431322102553e-05\\
2.9	5.2	4.8343161948673e-06	4.8343161948673e-06\\
2.9	5.3	3.18545272299543e-06	3.18545272299543e-06\\
2.9	5.4	3.22759140589757e-06	3.22759140589757e-06\\
2.9	5.5	4.48613383998528e-06	4.48613383998528e-06\\
2.9	5.6	7.83813728438056e-06	7.83813728438056e-06\\
2.9	5.7	1.44664221445652e-05	1.44664221445652e-05\\
2.9	5.8	2.68253479568362e-05	2.68253479568362e-05\\
2.9	5.9	5.35313302833043e-05	5.35313302833043e-05\\
2.9	6	0.000124599232765588	0.000124599232765588\\
3	0	0.000550213413214944	0.000550213413214944\\
3	0.1	0.000292448130777273	0.000292448130777273\\
3	0.2	0.000181489109976459	0.000181489109976459\\
3	0.3	0.00013248162534392	0.00013248162534392\\
3	0.4	0.00010718513414195	0.00010718513414195\\
3	0.5	8.00558864977512e-05	8.00558864977512e-05\\
3	0.6	4.62537247057441e-05	4.62537247057441e-05\\
3	0.7	2.29166937970952e-05	2.29166937970952e-05\\
3	0.8	1.38078355348922e-05	1.38078355348922e-05\\
3	0.9	1.25355417832901e-05	1.25355417832901e-05\\
3	1	1.4835019236925e-05	1.4835019236925e-05\\
3	1.1	1.64050947010707e-05	1.64050947010707e-05\\
3	1.2	1.51522398360092e-05	1.51522398360092e-05\\
3	1.3	1.39667034664581e-05	1.39667034664581e-05\\
3	1.4	1.59296967411183e-05	1.59296967411183e-05\\
3	1.5	2.46402886896186e-05	2.46402886896186e-05\\
3	1.6	4.57581661363422e-05	4.57581661363422e-05\\
3	1.7	8.09976450752862e-05	8.09976450752862e-05\\
3	1.8	0.000121235508723116	0.000121235508723116\\
3	1.9	0.000167666201483106	0.000167666201483106\\
3	2	0.000269022568740319	0.000269022568740319\\
3	2.1	0.000560311771664738	0.000560311771664738\\
3	2.2	0.00127311109268604	0.00127311109268604\\
3	2.3	0.00227405982252817	0.00227405982252817\\
3	2.4	0.00264479638660288	0.00264479638660288\\
3	2.5	0.00214416795161515	0.00214416795161515\\
3	2.6	0.00146436048904818	0.00146436048904818\\
3	2.7	0.00099510646751127	0.00099510646751127\\
3	2.8	0.0007369477180006	0.0007369477180006\\
3	2.9	0.000606902684680828	0.000606902684680828\\
3	3	0.000547545957060954	0.000547545957060954\\
3	3.1	0.000537142807164369	0.000537142807164369\\
3	3.2	0.00057221067137226	0.00057221067137226\\
3	3.3	0.000636423288099967	0.000636423288099967\\
3	3.4	0.000688599882450971	0.000688599882450971\\
3	3.5	0.000692971885554812	0.000692971885554812\\
3	3.6	0.000642035547316994	0.000642035547316994\\
3	3.7	0.000546991033175367	0.000546991033175367\\
3	3.8	0.000431090977031507	0.000431090977031507\\
3	3.9	0.000318106487765234	0.000318106487765234\\
3	4	0.000229059467554668	0.000229059467554668\\
3	4.1	0.00017851696407198	0.00017851696407198\\
3	4.2	0.000163102172671954	0.000163102172671954\\
3	4.3	0.000155614269055478	0.000155614269055478\\
3	4.4	0.00012056701024721	0.00012056701024721\\
3	4.5	7.1791342655903e-05	7.1791342655903e-05\\
3	4.6	4.05000825547501e-05	4.05000825547501e-05\\
3	4.7	2.66472722817678e-05	2.66472722817678e-05\\
3	4.8	2.35507867383989e-05	2.35507867383989e-05\\
3	4.9	2.37877706310468e-05	2.37877706310468e-05\\
3	5	1.88319407825138e-05	1.88319407825138e-05\\
3	5.1	1.04586494854689e-05	1.04586494854689e-05\\
3	5.2	4.85118598806919e-06	4.85118598806919e-06\\
3	5.3	2.7109507230756e-06	2.7109507230756e-06\\
3	5.4	2.45669808824058e-06	2.45669808824058e-06\\
3	5.5	3.50979106700977e-06	3.50979106700977e-06\\
3	5.6	6.89685612356394e-06	6.89685612356394e-06\\
3	5.7	1.56532652507311e-05	1.56532652507311e-05\\
3	5.8	3.61537896102097e-05	3.61537896102097e-05\\
3	5.9	8.49550352206411e-05	8.49550352206411e-05\\
3	6	0.000217777153922105	0.000217777153922105\\
3.1	0	0.000606470814870553	0.000606470814870553\\
3.1	0.1	0.000305442110879485	0.000305442110879485\\
3.1	0.2	0.000175313748166502	0.000175313748166502\\
3.1	0.3	0.00011702075726537	0.00011702075726537\\
3.1	0.4	8.74122974368028e-05	8.74122974368028e-05\\
3.1	0.5	6.40971077864629e-05	6.40971077864629e-05\\
3.1	0.6	4.18163374002294e-05	4.18163374002294e-05\\
3.1	0.7	2.61627421634258e-05	2.61627421634258e-05\\
3.1	0.8	1.8551991770047e-05	1.8551991770047e-05\\
3.1	0.9	1.68813285860367e-05	1.68813285860367e-05\\
3.1	1	1.8955325307254e-05	1.8955325307254e-05\\
3.1	1.1	2.13468570844837e-05	2.13468570844837e-05\\
3.1	1.2	2.06072091025136e-05	2.06072091025136e-05\\
3.1	1.3	1.77556010362148e-05	1.77556010362148e-05\\
3.1	1.4	1.68218160939207e-05	1.68218160939207e-05\\
3.1	1.5	2.1610280324004e-05	2.1610280324004e-05\\
3.1	1.6	3.66316643729946e-05	3.66316643729946e-05\\
3.1	1.7	6.61133615494662e-05	6.61133615494662e-05\\
3.1	1.8	0.000107992819900862	0.000107992819900862\\
3.1	1.9	0.000167104421248725	0.000167104421248725\\
3.1	2	0.000303881807702875	0.000303881807702875\\
3.1	2.1	0.000717649549264727	0.000717649549264727\\
3.1	2.2	0.00178315397369207	0.00178315397369207\\
3.1	2.3	0.0031687436725706	0.0031687436725706\\
3.1	2.4	0.00332712067019068	0.00332712067019068\\
3.1	2.5	0.00239189314889688	0.00239189314889688\\
3.1	2.6	0.00152109666300658	0.00152109666300658\\
3.1	2.7	0.00100719066166925	0.00100719066166925\\
3.1	2.8	0.000731491921707616	0.000731491921707616\\
3.1	2.9	0.000580517293682669	0.000580517293682669\\
3.1	3	0.000499377504985564	0.000499377504985564\\
3.1	3.1	0.000469642880856494	0.000469642880856494\\
3.1	3.2	0.00048758330885496	0.00048758330885496\\
3.1	3.3	0.000543532256721167	0.000543532256721167\\
3.1	3.4	0.000607879581875602	0.000607879581875602\\
3.1	3.5	0.000639504776324879	0.000639504776324879\\
3.1	3.6	0.000610938639007938	0.000610938639007938\\
3.1	3.7	0.000525632384814219	0.000525632384814219\\
3.1	3.8	0.000409983499036212	0.000409983499036212\\
3.1	3.9	0.000292557875142909	0.000292557875142909\\
3.1	4	0.000200890870283928	0.000200890870283928\\
3.1	4.1	0.000149098668921795	0.000149098668921795\\
3.1	4.2	0.000126934305162988	0.000126934305162988\\
3.1	4.3	0.000107637694184982	0.000107637694184982\\
3.1	4.4	7.28256514850625e-05	7.28256514850625e-05\\
3.1	4.5	4.06399497222718e-05	4.06399497222718e-05\\
3.1	4.6	2.53108789514189e-05	2.53108789514189e-05\\
3.1	4.7	2.1306755296984e-05	2.1306755296984e-05\\
3.1	4.8	2.30325464578026e-05	2.30325464578026e-05\\
3.1	4.9	2.41841895328302e-05	2.41841895328302e-05\\
3.1	5	1.87837726182626e-05	1.87837726182626e-05\\
3.1	5.1	1.04921153097865e-05	1.04921153097865e-05\\
3.1	5.2	4.99519658508971e-06	4.99519658508971e-06\\
3.1	5.3	2.66665913380584e-06	2.66665913380584e-06\\
3.1	5.4	2.17676169017801e-06	2.17676169017801e-06\\
3.1	5.5	2.87722611752684e-06	2.87722611752684e-06\\
3.1	5.6	5.47725720504663e-06	5.47725720504663e-06\\
3.1	5.7	1.35074949777345e-05	1.35074949777345e-05\\
3.1	5.8	3.829942628015e-05	3.829942628015e-05\\
3.1	5.9	0.000115470819141948	0.000115470819141948\\
3.1	6	0.000368020965120565	0.000368020965120565\\
3.2	0	0.000697854476640958	0.000697854476640958\\
3.2	0.1	0.000353733729461281	0.000353733729461281\\
3.2	0.2	0.000195823343394557	0.000195823343394557\\
3.2	0.3	0.000118450065180425	0.000118450065180425\\
3.2	0.4	7.67812871232191e-05	7.67812871232191e-05\\
3.2	0.5	5.24593636300602e-05	5.24593636300602e-05\\
3.2	0.6	3.87216264812715e-05	3.87216264812715e-05\\
3.2	0.7	3.07580611732353e-05	3.07580611732353e-05\\
3.2	0.8	2.47586872272409e-05	2.47586872272409e-05\\
3.2	0.9	2.18175611116527e-05	2.18175611116527e-05\\
3.2	1	2.32209982185456e-05	2.32209982185456e-05\\
3.2	1.1	2.64777111909855e-05	2.64777111909855e-05\\
3.2	1.2	2.66467901567121e-05	2.66467901567121e-05\\
3.2	1.3	2.28543290272979e-05	2.28543290272979e-05\\
3.2	1.4	2.07880692382186e-05	2.07880692382186e-05\\
3.2	1.5	2.57485068879515e-05	2.57485068879515e-05\\
3.2	1.6	4.22599386902102e-05	4.22599386902102e-05\\
3.2	1.7	7.37980219843956e-05	7.37980219843956e-05\\
3.2	1.8	0.000119486347694665	0.000119486347694665\\
3.2	1.9	0.000196132113186375	0.000196132113186375\\
3.2	2	0.00040645853825945	0.00040645853825945\\
3.2	2.1	0.0011243994055257	0.0011243994055257\\
3.2	2.2	0.00314322972733063	0.00314322972733063\\
3.2	2.3	0.00564230468975494	0.00564230468975494\\
3.2	2.4	0.00524710459475091	0.00524710459475091\\
3.2	2.5	0.00310499349970204	0.00310499349970204\\
3.2	2.6	0.00166333794009395	0.00166333794009395\\
3.2	2.7	0.000992084113747204	0.000992084113747204\\
3.2	2.8	0.00068084591689909	0.00068084591689909\\
3.2	2.9	0.000519554938239234	0.000519554938239234\\
3.2	3	0.000433462613871221	0.000433462613871221\\
3.2	3.1	0.000400709503771943	0.000400709503771943\\
3.2	3.2	0.00041741825190613	0.00041741825190613\\
3.2	3.3	0.000480120612785022	0.000480120612785022\\
3.2	3.4	0.000565787821655306	0.000565787821655306\\
3.2	3.5	0.000622277171627156	0.000622277171627156\\
3.2	3.6	0.000603023087608543	0.000603023087608543\\
3.2	3.7	0.000514105486939237	0.000514105486939237\\
3.2	3.8	0.000395598682058928	0.000395598682058928\\
3.2	3.9	0.000279672784396566	0.000279672784396566\\
3.2	4	0.000189987009542456	0.000189987009542456\\
3.2	4.1	0.000136339306973385	0.000136339306973385\\
3.2	4.2	0.000106445993935573	0.000106445993935573\\
3.2	4.3	7.97898060633741e-05	7.97898060633741e-05\\
3.2	4.4	4.94046951006957e-05	4.94046951006957e-05\\
3.2	4.5	2.6947887105554e-05	2.6947887105554e-05\\
3.2	4.6	1.76463203460771e-05	1.76463203460771e-05\\
3.2	4.7	1.70135237647436e-05	1.70135237647436e-05\\
3.2	4.8	2.10316218741754e-05	2.10316218741754e-05\\
3.2	4.9	2.34520104396397e-05	2.34520104396397e-05\\
3.2	5	1.83313157358383e-05	1.83313157358383e-05\\
3.2	5.1	1.02175921027379e-05	1.02175921027379e-05\\
3.2	5.2	5.04473600706792e-06	5.04473600706792e-06\\
3.2	5.3	2.85950894658503e-06	2.85950894658503e-06\\
3.2	5.4	2.3130111190309e-06	2.3130111190309e-06\\
3.2	5.5	2.72523358769561e-06	2.72523358769561e-06\\
3.2	5.6	4.68898550430326e-06	4.68898550430326e-06\\
3.2	5.7	1.17187947560381e-05	1.17187947560381e-05\\
3.2	5.8	3.83507961741155e-05	3.83507961741155e-05\\
3.2	5.9	0.000145144402626701	0.000145144402626701\\
3.2	6	0.000585598585398872	0.000585598585398872\\
3.3	0	0.000768714738489136	0.000768714738489136\\
3.3	0.1	0.00041075105127525	0.00041075105127525\\
3.3	0.2	0.000237691838942887	0.000237691838942887\\
3.3	0.3	0.000143933642081726	0.000143933642081726\\
3.3	0.4	8.67922340135918e-05	8.67922340135918e-05\\
3.3	0.5	5.25694041127302e-05	5.25694041127302e-05\\
3.3	0.6	3.69720876866415e-05	3.69720876866415e-05\\
3.3	0.7	3.38761449979792e-05	3.38761449979792e-05\\
3.3	0.8	3.18131177772586e-05	3.18131177772586e-05\\
3.3	0.9	2.68970187943677e-05	2.68970187943677e-05\\
3.3	1	2.68548296094011e-05	2.68548296094011e-05\\
3.3	1.1	3.12652979475426e-05	3.12652979475426e-05\\
3.3	1.2	3.25973417223998e-05	3.25973417223998e-05\\
3.3	1.3	2.87521864498392e-05	2.87521864498392e-05\\
3.3	1.4	2.89851299704627e-05	2.89851299704627e-05\\
3.3	1.5	4.09851951870833e-05	4.09851951870833e-05\\
3.3	1.6	7.02505796022306e-05	7.02505796022306e-05\\
3.3	1.7	0.000113535232155687	0.000113535232155687\\
3.3	1.8	0.000165309258546682	0.000165309258546682\\
3.3	1.9	0.000263279827810468	0.000263279827810468\\
3.3	2	0.000578189412067627	0.000578189412067627\\
3.3	2.1	0.00177317886965406	0.00177317886965406\\
3.3	2.2	0.00538837930124295	0.00538837930124295\\
3.3	2.3	0.00972259027732512	0.00972259027732512\\
3.3	2.4	0.00811714100317007	0.00811714100317007\\
3.3	2.5	0.00395143948225607	0.00395143948225607\\
3.3	2.6	0.00174888485149558	0.00174888485149558\\
3.3	2.7	0.00092518612645208	0.00092518612645208\\
3.3	2.8	0.00060150012916975	0.00060150012916975\\
3.3	2.9	0.00045010020459404	0.00045010020459404\\
3.3	3	0.000375127927697698	0.000375127927697698\\
3.3	3.1	0.000352501039897631	0.000352501039897631\\
3.3	3.2	0.000380073376919779	0.000380073376919779\\
3.3	3.3	0.000460801468103209	0.000460801468103209\\
3.3	3.4	0.000576102598766045	0.000576102598766045\\
3.3	3.5	0.000655159147141614	0.000655159147141614\\
3.3	3.6	0.000625744964638712	0.000625744964638712\\
3.3	3.7	0.000510572813975154	0.000510572813975154\\
3.3	3.8	0.000381116544887286	0.000381116544887286\\
3.3	3.9	0.000271762848299384	0.000271762848299384\\
3.3	4	0.000190399735076551	0.000190399735076551\\
3.3	4.1	0.000138423269296302	0.000138423269296302\\
3.3	4.2	0.000105758273994907	0.000105758273994907\\
3.3	4.3	7.71041489964633e-05	7.71041489964633e-05\\
3.3	4.4	4.74806025215977e-05	4.74806025215977e-05\\
3.3	4.5	2.50097570820891e-05	2.50097570820891e-05\\
3.3	4.6	1.45259291665006e-05	1.45259291665006e-05\\
3.3	4.7	1.2881707970772e-05	1.2881707970772e-05\\
3.3	4.8	1.68332549612981e-05	1.68332549612981e-05\\
3.3	4.9	2.086519036136e-05	2.086519036136e-05\\
3.3	5	1.7027050803626e-05	1.7027050803626e-05\\
3.3	5.1	9.37783763851626e-06	9.37783763851626e-06\\
3.3	5.2	4.93611226099835e-06	4.93611226099835e-06\\
3.3	5.3	3.3974864411702e-06	3.3974864411702e-06\\
3.3	5.4	2.97455934499806e-06	2.97455934499806e-06\\
3.3	5.5	3.36463740205842e-06	3.36463740205842e-06\\
3.3	5.6	5.73551388155713e-06	5.73551388155713e-06\\
3.3	5.7	1.45732854335015e-05	1.45732854335015e-05\\
3.3	5.8	4.96568270168871e-05	4.96568270168871e-05\\
3.3	5.9	0.000201050018412094	0.000201050018412094\\
3.3	6	0.000872439670539877	0.000872439670539877\\
3.4	0	0.000697701681960155	0.000697701681960155\\
3.4	0.1	0.000386028930583133	0.000386028930583133\\
3.4	0.2	0.000241497935831626	0.000241497935831626\\
3.4	0.3	0.000167310400153485	0.000167310400153485\\
3.4	0.4	0.000120106623711007	0.000120106623711007\\
3.4	0.5	8.06044258551916e-05	8.06044258551916e-05\\
3.4	0.6	4.92132436949368e-05	4.92132436949368e-05\\
3.4	0.7	3.5713954995642e-05	3.5713954995642e-05\\
3.4	0.8	3.73450030674996e-05	3.73450030674996e-05\\
3.4	0.9	3.20688265437608e-05	3.20688265437608e-05\\
3.4	1	2.89719430094816e-05	2.89719430094816e-05\\
3.4	1.1	3.55836175355486e-05	3.55836175355486e-05\\
3.4	1.2	3.83520163333761e-05	3.83520163333761e-05\\
3.4	1.3	3.83477812405121e-05	3.83477812405121e-05\\
3.4	1.4	5.10033519807518e-05	5.10033519807518e-05\\
3.4	1.5	8.45819870536818e-05	8.45819870536818e-05\\
3.4	1.6	0.000137446113038043	0.000137446113038043\\
3.4	1.7	0.000188277066355582	0.000188277066355582\\
3.4	1.8	0.000235824152049121	0.000235824152049121\\
3.4	1.9	0.000343964338111836	0.000343964338111836\\
3.4	2	0.000731862207360845	0.000731862207360845\\
3.4	2.1	0.00226640986015287	0.00226640986015287\\
3.4	2.2	0.00704072229787853	0.00704072229787853\\
3.4	2.3	0.0126238140074861	0.0126238140074861\\
3.4	2.4	0.00981550904804031	0.00981550904804031\\
3.4	2.5	0.00418864906552302	0.00418864906552302\\
3.4	2.6	0.00162426340025133	0.00162426340025133\\
3.4	2.7	0.00079541247594122	0.00079541247594122\\
3.4	2.8	0.000507452062915207	0.000507452062915207\\
3.4	2.9	0.000385049688901515	0.000385049688901515\\
3.4	3	0.000330548211539069	0.000330548211539069\\
3.4	3.1	0.000323665135014706	0.000323665135014706\\
3.4	3.2	0.000366130202320494	0.000366130202320494\\
3.4	3.3	0.000467590735778661	0.000467590735778661\\
3.4	3.4	0.000615043720187194	0.000615043720187194\\
3.4	3.5	0.000715098681770774	0.000715098681770774\\
3.4	3.6	0.000661721085712552	0.000661721085712552\\
3.4	3.7	0.000507431138229287	0.000507431138229287\\
3.4	3.8	0.000366311074086923	0.000366311074086923\\
3.4	3.9	0.000268849377828339	0.000268849377828339\\
3.4	4	0.000201164968191966	0.000201164968191966\\
3.4	4.1	0.000153780290271968	0.000153780290271968\\
3.4	4.2	0.000120113639227111	0.000120113639227111\\
3.4	4.3	9.20856870453e-05	9.20856870453e-05\\
3.4	4.4	6.43822130708415e-05	6.43822130708415e-05\\
3.4	4.5	3.76687510670464e-05	3.76687510670464e-05\\
3.4	4.6	1.86707673990537e-05	1.86707673990537e-05\\
3.4	4.7	1.09811226374621e-05	1.09811226374621e-05\\
3.4	4.8	1.15531083933216e-05	1.15531083933216e-05\\
3.4	4.9	1.6256253113294e-05	1.6256253113294e-05\\
3.4	5	1.48758368494802e-05	1.48758368494802e-05\\
3.4	5.1	7.95232167861543e-06	7.95232167861543e-06\\
3.4	5.2	5.09998978411009e-06	5.09998978411009e-06\\
3.4	5.3	4.73382768641039e-06	4.73382768641039e-06\\
3.4	5.4	4.93437978717063e-06	4.93437978717063e-06\\
3.4	5.5	6.97898382629563e-06	6.97898382629563e-06\\
3.4	5.6	1.28911739445569e-05	1.28911739445569e-05\\
3.4	5.7	2.98913662150537e-05	2.98913662150537e-05\\
3.4	5.8	8.51734071834408e-05	8.51734071834408e-05\\
3.4	5.9	0.00028256432579086	0.00028256432579086\\
3.4	6	0.00102448041806564	0.00102448041806564\\
3.5	0	0.000498686492854194	0.000498686492854194\\
3.5	0.1	0.000278443403744891	0.000278443403744891\\
3.5	0.2	0.000182122877501811	0.000182122877501811\\
3.5	0.3	0.000142175772843502	0.000142175772843502\\
3.5	0.4	0.000130506213856163	0.000130506213856163\\
3.5	0.5	0.000128809271313016	0.000128809271313016\\
3.5	0.6	0.000112156459551921	0.000112156459551921\\
3.5	0.7	6.84798436476757e-05	6.84798436476757e-05\\
3.5	0.8	4.04806768764512e-05	4.04806768764512e-05\\
3.5	0.9	3.89784406089852e-05	3.89784406089852e-05\\
3.5	1	2.90614114746176e-05	2.90614114746176e-05\\
3.5	1.1	4.11811629881355e-05	4.11811629881355e-05\\
3.5	1.2	4.8194389202487e-05	4.8194389202487e-05\\
3.5	1.3	7.36368287190363e-05	7.36368287190363e-05\\
3.5	1.4	0.000115488573918477	0.000115488573918477\\
3.5	1.5	0.000164253142733585	0.000164253142733585\\
3.5	1.6	0.000212407958219212	0.000212407958219212\\
3.5	1.7	0.00024166157934635	0.00024166157934635\\
3.5	1.8	0.000266717703608972	0.000266717703608972\\
3.5	1.9	0.000353402218275393	0.000353402218275393\\
3.5	2	0.000686055521414408	0.000686055521414408\\
3.5	2.1	0.00195410154521457	0.00195410154521457\\
3.5	2.2	0.00576554125998526	0.00576554125998526\\
3.5	2.3	0.0101412885710486	0.0101412885710486\\
3.5	2.4	0.00779792076985707	0.00779792076985707\\
3.5	2.5	0.00326522548645504	0.00326522548645504\\
3.5	2.6	0.00124066121794075	0.00124066121794075\\
3.5	2.7	0.000608190535525119	0.000608190535525119\\
3.5	2.8	0.000400651436180959	0.000400651436180959\\
3.5	2.9	0.000319374960589138	0.000319374960589138\\
3.5	3	0.000289409753511853	0.000289409753511853\\
3.5	3.1	0.000299098046970272	0.000299098046970272\\
3.5	3.2	0.000353622889795018	0.000353622889795018\\
3.5	3.3	0.000465174661711898	0.000465174661711898\\
3.5	3.4	0.000626654246449374	0.000626654246449374\\
3.5	3.5	0.000739258003146869	0.000739258003146869\\
3.5	3.6	0.000676186201010589	0.000676186201010589\\
3.5	3.7	0.000507064764894272	0.000507064764894272\\
3.5	3.8	0.000370501420281137	0.000370501420281137\\
3.5	3.9	0.000290662574925469	0.000290662574925469\\
3.5	4	0.000239933954434839	0.000239933954434839\\
3.5	4.1	0.000200128666015263	0.000200128666015263\\
3.5	4.2	0.000163520889235835	0.000163520889235835\\
3.5	4.3	0.000127227400918932	0.000127227400918932\\
3.5	4.4	9.39717067851147e-05	9.39717067851147e-05\\
3.5	4.5	6.64897675961598e-05	6.64897675961598e-05\\
3.5	4.6	4.2091598902106e-05	4.2091598902106e-05\\
3.5	4.7	2.01406100735109e-05	2.01406100735109e-05\\
3.5	4.8	9.70941915971235e-06	9.70941915971235e-06\\
3.5	4.9	1.04612228118375e-05	1.04612228118375e-05\\
3.5	5	1.23993909431844e-05	1.23993909431844e-05\\
3.5	5.1	6.46491890283747e-06	6.46491890283747e-06\\
3.5	5.2	6.93405600328546e-06	6.93405600328546e-06\\
3.5	5.3	9.77260280979205e-06	9.77260280979205e-06\\
3.5	5.4	1.65514895905532e-05	1.65514895905532e-05\\
3.5	5.5	2.50940468618915e-05	2.50940468618915e-05\\
3.5	5.6	3.78206740786232e-05	3.78206740786232e-05\\
3.5	5.7	6.44931119755712e-05	6.44931119755712e-05\\
3.5	5.8	0.000129703301288156	0.000129703301288156\\
3.5	5.9	0.000305374891699653	0.000305374891699653\\
3.5	6	0.000819261083487076	0.000819261083487076\\
3.6	0	0.00031325624544478	0.00031325624544478\\
3.6	0.1	0.000182972097177273	0.000182972097177273\\
3.6	0.2	0.000124738641592906	0.000124738641592906\\
3.6	0.3	0.000101521606485961	0.000101521606485961\\
3.6	0.4	9.97868047065478e-05	9.97868047065478e-05\\
3.6	0.5	0.000117016189245314	0.000117016189245314\\
3.6	0.6	0.000154736101791353	0.000154736101791353\\
3.6	0.7	0.000195212922675059	0.000195212922675059\\
3.6	0.8	0.000155408163325427	0.000155408163325427\\
3.6	0.9	5.4272352199901e-05	5.4272352199901e-05\\
3.6	1	2.73522357434787e-05	2.73522357434787e-05\\
3.6	1.1	6.01641494105235e-05	6.01641494105235e-05\\
3.6	1.2	0.000142422632636801	0.000142422632636801\\
3.6	1.3	0.000170133384683152	0.000170133384683152\\
3.6	1.4	0.000174042690814327	0.000174042690814327\\
3.6	1.5	0.000193096273332974	0.000193096273332974\\
3.6	1.6	0.000212572719097507	0.000212572719097507\\
3.6	1.7	0.000213836573317841	0.000213836573317841\\
3.6	1.8	0.000216770859137078	0.000216770859137078\\
3.6	1.9	0.000268635215457514	0.000268635215457514\\
3.6	2	0.000467638668038121	0.000467638668038121\\
3.6	2.1	0.00113975889202718	0.00113975889202718\\
3.6	2.2	0.00296116103814721	0.00296116103814721\\
3.6	2.3	0.00499437152238073	0.00499437152238073\\
3.6	2.4	0.00401098613554424	0.00401098613554424\\
3.6	2.5	0.00183806902273412	0.00183806902273412\\
3.6	2.6	0.00076392749431468	0.00076392749431468\\
3.6	2.7	0.000407036767355568	0.000407036767355568\\
3.6	2.8	0.000291533469166971	0.000291533469166971\\
3.6	2.9	0.000250208749243712	0.000250208749243712\\
3.6	3	0.000241808901555136	0.000241808901555136\\
3.6	3.1	0.000264843547380265	0.000264843547380265\\
3.6	3.2	0.000325303077356068	0.000325303077356068\\
3.6	3.3	0.000431018708598946	0.000431018708598946\\
3.6	3.4	0.000579286581711595	0.000579286581711595\\
3.6	3.5	0.00069502022031529	0.00069502022031529\\
3.6	3.6	0.000657441122308483	0.000657441122308483\\
3.6	3.7	0.000516235591868396	0.000516235591868396\\
3.6	3.8	0.000404144934229901	0.000404144934229901\\
3.6	3.9	0.000346179016989849	0.000346179016989849\\
3.6	4	0.000313709316825993	0.000313709316825993\\
3.6	4.1	0.000286358855031589	0.000286358855031589\\
3.6	4.2	0.000253535951208247	0.000253535951208247\\
3.6	4.3	0.000209002894580352	0.000209002894580352\\
3.6	4.4	0.000155836773728567	0.000155836773728567\\
3.6	4.5	0.000106354616896764	0.000106354616896764\\
3.6	4.6	7.14202757945746e-05	7.14202757945746e-05\\
3.6	4.7	5.20564956734087e-05	5.20564956734087e-05\\
3.6	4.8	3.51571065245074e-05	3.51571065245074e-05\\
3.6	4.9	1.09665282651876e-05	1.09665282651876e-05\\
3.6	5	1.03077334825238e-05	1.03077334825238e-05\\
3.6	5.1	9.79510448416715e-06	9.79510448416715e-06\\
3.6	5.2	2.96326999607889e-05	2.96326999607889e-05\\
3.6	5.3	4.54657859047862e-05	4.54657859047862e-05\\
3.6	5.4	5.07397504890071e-05	5.07397504890071e-05\\
3.6	5.5	5.60303593469356e-05	5.60303593469356e-05\\
3.6	5.6	6.65644176839728e-05	6.65644176839728e-05\\
3.6	5.7	8.86519156000808e-05	8.86519156000808e-05\\
3.6	5.8	0.000136058062817242	0.000136058062817242\\
3.6	5.9	0.000243479124558206	0.000243479124558206\\
3.6	6	0.00050499902792392	0.00050499902792392\\
3.7	0	0.00022115773583609	0.00022115773583609\\
3.7	0.1	0.000143922064545498	0.000143922064545498\\
3.7	0.2	0.000105887820569915	0.000105887820569915\\
3.7	0.3	8.82355821825962e-05	8.82355821825962e-05\\
3.7	0.4	8.26861314051849e-05	8.26861314051849e-05\\
3.7	0.5	8.56209355997058e-05	8.56209355997058e-05\\
3.7	0.6	9.5365795396531e-05	9.5365795396531e-05\\
3.7	0.7	0.000110393762791835	0.000110393762791835\\
3.7	0.8	0.000127840303147838	0.000127840303147838\\
3.7	0.9	0.000143346921303998	0.000143346921303998\\
3.7	1	0.000153398633592006	0.000153398633592006\\
3.7	1.1	0.000157929456048639	0.000157929456048639\\
3.7	1.2	0.000159751114010445	0.000159751114010445\\
3.7	1.3	0.000162329651541962	0.000162329651541962\\
3.7	1.4	0.000168310274143429	0.000168310274143429\\
3.7	1.5	0.000175884061324385	0.000175884061324385\\
3.7	1.6	0.000175047903767277	0.000175047903767277\\
3.7	1.7	0.000162138975660726	0.000162138975660726\\
3.7	1.8	0.000157344546237418	0.000157344546237418\\
3.7	1.9	0.000188199000643853	0.000188199000643853\\
3.7	2	0.000291811518248832	0.000291811518248832\\
3.7	2.1	0.000576588316895982	0.000576588316895982\\
3.7	2.2	0.00123295277077994	0.00123295277077994\\
3.7	2.3	0.00191372664844694	0.00191372664844694\\
3.7	2.4	0.00161633103875658	0.00161633103875658\\
3.7	2.5	0.000844706125030659	0.000844706125030659\\
3.7	2.6	0.000404027719063341	0.000404027719063341\\
3.7	2.7	0.000245703553558909	0.000245703553558909\\
3.7	2.8	0.000196698339170756	0.000196698339170756\\
3.7	2.9	0.00018185252994368	0.00018185252994368\\
3.7	3	0.000186384119889532	0.000186384119889532\\
3.7	3.1	0.000216764483775541	0.000216764483775541\\
3.7	3.2	0.000278671538641982	0.000278671538641982\\
3.7	3.3	0.00037505034801153	0.00037505034801153\\
3.7	3.4	0.000507441155012382	0.000507441155012382\\
3.7	3.5	0.000629853962520276	0.000629853962520276\\
3.7	3.6	0.000634456296295541	0.000634456296295541\\
3.7	3.7	0.00053915173069708	0.00053915173069708\\
3.7	3.8	0.000461302811211057	0.000461302811211057\\
3.7	3.9	0.000427077403028381	0.000427077403028381\\
3.7	4	0.000408283007189933	0.000408283007189933\\
3.7	4.1	0.000389462033109241	0.000389462033109241\\
3.7	4.2	0.000368652637694185	0.000368652637694185\\
3.7	4.3	0.000342120207155057	0.000342120207155057\\
3.7	4.4	0.000301887154247539	0.000301887154247539\\
3.7	4.5	0.000248414468307923	0.000248414468307923\\
3.7	4.6	0.00019473166167452	0.00019473166167452\\
3.7	4.7	0.000152242700111244	0.000152242700111244\\
3.7	4.8	0.000122904468961633	0.000122904468961633\\
3.7	4.9	0.000103347327696623	0.000103347327696623\\
3.7	5	8.97588710828361e-05	8.97588710828361e-05\\
3.7	5.1	7.97176547497875e-05	7.97176547497875e-05\\
3.7	5.2	7.22023001074568e-05	7.22023001074568e-05\\
3.7	5.3	6.70125511834836e-05	6.70125511834836e-05\\
3.7	5.4	6.43307258101566e-05	6.43307258101566e-05\\
3.7	5.5	6.47432283582957e-05	6.47432283582957e-05\\
3.7	5.6	6.96956624771427e-05	6.96956624771427e-05\\
3.7	5.7	8.24107990274415e-05	8.24107990274415e-05\\
3.7	5.8	0.000109937117167165	0.000109937117167165\\
3.7	5.9	0.000168411696728377	0.000168411696728377\\
3.7	6	0.000297138591046875	0.000297138591046875\\
3.8	0	0.00019961220723433	0.00019961220723433\\
3.8	0.1	0.000147156699529833	0.000147156699529833\\
3.8	0.2	0.00011839421140875	0.00011839421140875\\
3.8	0.3	0.000103288057165254	0.000103288057165254\\
3.8	0.4	9.66073731602214e-05	9.66073731602214e-05\\
3.8	0.5	9.52164878982e-05	9.52164878982e-05\\
3.8	0.6	9.65434552796235e-05	9.65434552796235e-05\\
3.8	0.7	9.69333819479032e-05	9.69333819479032e-05\\
3.8	0.8	8.39229045794105e-05	8.39229045794105e-05\\
3.8	0.9	3.90308697770163e-05	3.90308697770163e-05\\
3.8	1	1.28797289061777e-05	1.28797289061777e-05\\
3.8	1.1	4.22696122334057e-05	4.22696122334057e-05\\
3.8	1.2	0.000103733173166336	0.000103733173166336\\
3.8	1.3	0.000143238169692381	0.000143238169692381\\
3.8	1.4	0.000166764665310552	0.000166764665310552\\
3.8	1.5	0.000174247425734999	0.000174247425734999\\
3.8	1.6	0.00016616348558141	0.00016616348558141\\
3.8	1.7	0.00014642525209145	0.00014642525209145\\
3.8	1.8	0.000135209646722169	0.000135209646722169\\
3.8	1.9	0.000153244805106967	0.000153244805106967\\
3.8	2	0.000209961554557702	0.000209961554557702\\
3.8	2.1	0.000333305676692226	0.000333305676692226\\
3.8	2.2	0.000571321990612025	0.000571321990612025\\
3.8	2.3	0.000780165899004593	0.000780165899004593\\
3.8	2.4	0.000660153663218477	0.000660153663218477\\
3.8	2.5	0.000379174887892611	0.000379174887892611\\
3.8	2.6	0.000205385855724049	0.000205385855724049\\
3.8	2.7	0.000141370177076544	0.000141370177076544\\
3.8	2.8	0.000122947364691569	0.000122947364691569\\
3.8	2.9	0.000118914115263075	0.000118914115263075\\
3.8	3	0.00012902036030029	0.00012902036030029\\
3.8	3.1	0.000162010636677832	0.000162010636677832\\
3.8	3.2	0.00022421199871621	0.00022421199871621\\
3.8	3.3	0.000320628705524013	0.000320628705524013\\
3.8	3.4	0.000458809859690183	0.000458809859690183\\
3.8	3.5	0.000605908571366767	0.000605908571366767\\
3.8	3.6	0.000649230857925932	0.000649230857925932\\
3.8	3.7	0.00058614300030155	0.00058614300030155\\
3.8	3.8	0.000530032418322437	0.000530032418322437\\
3.8	3.9	0.00049506730834219	0.00049506730834219\\
3.8	4	0.000443354554541556	0.000443354554541556\\
3.8	4.1	0.000379551969728274	0.000379551969728274\\
3.8	4.2	0.00032888518616578	0.00032888518616578\\
3.8	4.3	0.000292185115845395	0.000292185115845395\\
3.8	4.4	0.000249695668325306	0.000249695668325306\\
3.8	4.5	0.000190110036083608	0.000190110036083608\\
3.8	4.6	0.000120768279475116	0.000120768279475116\\
3.8	4.7	5.66050166166485e-05	5.66050166166485e-05\\
3.8	4.8	1.43778807055477e-05	1.43778807055477e-05\\
3.8	4.9	1.87695650963422e-06	1.87695650963422e-06\\
3.8	5	8.07509758274674e-06	8.07509758274674e-06\\
3.8	5.1	1.86073856901778e-06	1.86073856901778e-06\\
3.8	5.2	1.06855046122265e-05	1.06855046122265e-05\\
3.8	5.3	2.60067354819686e-05	2.60067354819686e-05\\
3.8	5.4	3.75179927511882e-05	3.75179927511882e-05\\
3.8	5.5	4.57181770166939e-05	4.57181770166939e-05\\
3.8	5.6	5.37564345058646e-05	5.37564345058646e-05\\
3.8	5.7	6.5212969501398e-05	6.5212969501398e-05\\
3.8	5.8	8.50852804395037e-05	8.50852804395037e-05\\
3.8	5.9	0.000122605083539929	0.000122605083539929\\
3.8	6	0.000197658017484277	0.000197658017484277\\
3.9	0	0.000196759726671688	0.000196759726671688\\
3.9	0.1	0.000154595178552968	0.000154595178552968\\
3.9	0.2	0.000128532136392648	0.000128532136392648\\
3.9	0.3	0.000112223886396265	0.000112223886396265\\
3.9	0.4	0.000101163796952325	0.000101163796952325\\
3.9	0.5	9.04149717764954e-05	9.04149717764954e-05\\
3.9	0.6	7.35821799834641e-05	7.35821799834641e-05\\
3.9	0.7	4.87809186883445e-05	4.87809186883445e-05\\
3.9	0.8	2.60900806461959e-05	2.60900806461959e-05\\
3.9	0.9	1.32149034747595e-05	1.32149034747595e-05\\
3.9	1	1.07555022132789e-05	1.07555022132789e-05\\
3.9	1.1	1.31004277475271e-05	1.31004277475271e-05\\
3.9	1.2	2.75194901877127e-05	2.75194901877127e-05\\
3.9	1.3	5.62726696078233e-05	5.62726696078233e-05\\
3.9	1.4	9.09126427708073e-05	9.09126427708073e-05\\
3.9	1.5	0.000119623748145721	0.000119623748145721\\
3.9	1.6	0.000133239767642666	0.000133239767642666\\
3.9	1.7	0.000130783654290607	0.000130783654290607\\
3.9	1.8	0.000123884464087216	0.000123884464087216\\
3.9	1.9	0.000136090428141621	0.000136090428141621\\
3.9	2	0.000174321660574234	0.000174321660574234\\
3.9	2.1	0.000243765788992407	0.000243765788992407\\
3.9	2.2	0.000351690553902109	0.000351690553902109\\
3.9	2.3	0.000408678852280926	0.000408678852280926\\
3.9	2.4	0.0003201552833506	0.0003201552833506\\
3.9	2.5	0.000188680083404364	0.000188680083404364\\
3.9	2.6	0.000111363309425745	0.000111363309425745\\
3.9	2.7	8.27895754598322e-05	8.27895754598322e-05\\
3.9	2.8	7.21780867032303e-05	7.21780867032303e-05\\
3.9	2.9	6.9305007749766e-05	6.9305007749766e-05\\
3.9	3	7.99906163524553e-05	7.99906163524553e-05\\
3.9	3.1	0.000111662695129247	0.000111662695129247\\
3.9	3.2	0.000173411935243581	0.000173411935243581\\
3.9	3.3	0.000278966465130779	0.000278966465130779\\
3.9	3.4	0.000446065163337763	0.000446065163337763\\
3.9	3.5	0.000644487678257025	0.000644487678257025\\
3.9	3.6	0.000727623908476834	0.000727623908476834\\
3.9	3.7	0.000666528310929054	0.000666528310929054\\
3.9	3.8	0.000585287465015695	0.000585287465015695\\
3.9	3.9	0.000488991628457458	0.000488991628457458\\
3.9	4	0.00035797557027998	0.00035797557027998\\
3.9	4.1	0.000246196439448791	0.000246196439448791\\
3.9	4.2	0.000180880427178684	0.000180880427178684\\
3.9	4.3	0.000136115541405318	0.000136115541405318\\
3.9	4.4	8.78181620141985e-05	8.78181620141985e-05\\
3.9	4.5	4.18911675454264e-05	4.18911675454264e-05\\
3.9	4.6	1.3608675582493e-05	1.3608675582493e-05\\
3.9	4.7	3.20986299471272e-06	3.20986299471272e-06\\
3.9	4.8	1.40775202226781e-06	1.40775202226781e-06\\
3.9	4.9	2.8513085617949e-06	2.8513085617949e-06\\
3.9	5	7.12850930412035e-06	7.12850930412035e-06\\
3.9	5.1	3.23594398578328e-06	3.23594398578328e-06\\
3.9	5.2	1.82144162141339e-06	1.82144162141339e-06\\
3.9	5.3	3.32212501362107e-06	3.32212501362107e-06\\
3.9	5.4	9.12660550880915e-06	9.12660550880915e-06\\
3.9	5.5	1.89469088733931e-05	1.89469088733931e-05\\
3.9	5.6	3.14397787356388e-05	3.14397787356388e-05\\
3.9	5.7	4.69709654853368e-05	4.69709654853368e-05\\
3.9	5.8	6.82552455711355e-05	6.82552455711355e-05\\
3.9	5.9	0.000101322584918871	0.000101322584918871\\
3.9	6	0.00015853603639115	0.00015853603639115\\
4	0	0.000185706773472605	0.000185706773472605\\
4	0.1	0.000147448065141858	0.000147448065141858\\
4	0.2	0.000120737437307086	0.000120737437307086\\
4	0.3	9.99055238581537e-05	9.99055238581537e-05\\
4	0.4	8.04386269559546e-05	8.04386269559546e-05\\
4	0.5	5.99311923341227e-05	5.99311923341227e-05\\
4	0.6	3.98586897238574e-05	3.98586897238574e-05\\
4	0.7	2.37490804374233e-05	2.37490804374233e-05\\
4	0.8	1.36029700782665e-05	1.36029700782665e-05\\
4	0.9	9.42314535593953e-06	9.42314535593953e-06\\
4	1	9.18896101806283e-06	9.18896101806283e-06\\
4	1.1	9.79088629126146e-06	9.79088629126146e-06\\
4	1.2	1.39795922883208e-05	1.39795922883208e-05\\
4	1.3	2.39009477817991e-05	2.39009477817991e-05\\
4	1.4	3.84259610871579e-05	3.84259610871579e-05\\
4	1.5	5.58399588927843e-05	5.58399588927843e-05\\
4	1.6	7.41702615057582e-05	7.41702615057582e-05\\
4	1.7	9.00880535391837e-05	9.00880535391837e-05\\
4	1.8	9.94895594595316e-05	9.94895594595316e-05\\
4	1.9	0.000114293500771984	0.000114293500771984\\
4	2	0.000150347012204056	0.000150347012204056\\
4	2.1	0.000215264642331393	0.000215264642331393\\
4	2.2	0.000289887269500066	0.000289887269500066\\
4	2.3	0.00028319550934221	0.00028319550934221\\
4	2.4	0.000190764501863775	0.000190764501863775\\
4	2.5	0.000109242135552414	0.000109242135552414\\
4	2.6	6.91650429833013e-05	6.91650429833013e-05\\
4	2.7	5.40416013885728e-05	5.40416013885728e-05\\
4	2.8	4.39144696668465e-05	4.39144696668465e-05\\
4	2.9	3.92571919256252e-05	3.92571919256252e-05\\
4	3	4.70903226725844e-05	4.70903226725844e-05\\
4	3.1	7.38465952746933e-05	7.38465952746933e-05\\
4	3.2	0.000133291947537779	0.000133291947537779\\
4	3.3	0.000250817395285362	0.000250817395285362\\
4	3.4	0.000458799749589159	0.000458799749589159\\
4	3.5	0.000734568677772362	0.000734568677772362\\
4	3.6	0.000871809014747361	0.000871809014747361\\
4	3.7	0.000771953698535575	0.000771953698535575\\
4	3.8	0.000598250188151912	0.000598250188151912\\
4	3.9	0.000405763131911733	0.000405763131911733\\
4	4	0.000229948455586992	0.000229948455586992\\
4	4.1	0.000126137815700785	0.000126137815700785\\
4	4.2	7.4560998812279e-05	7.4560998812279e-05\\
4	4.3	4.08873258385e-05	4.08873258385e-05\\
4	4.4	1.74499859838192e-05	1.74499859838192e-05\\
4	4.5	5.74120566405977e-06	5.74120566405977e-06\\
4	4.6	1.88953197768598e-06	1.88953197768598e-06\\
4	4.7	1.15701292615384e-06	1.15701292615384e-06\\
4	4.8	1.74438232501373e-06	1.74438232501373e-06\\
4	4.9	4.02665135422763e-06	4.02665135422763e-06\\
4	5	6.8518372626744e-06	6.8518372626744e-06\\
4	5.1	4.82697029354491e-06	4.82697029354491e-06\\
4	5.2	2.8550131340747e-06	2.8550131340747e-06\\
4	5.3	2.23953833939628e-06	2.23953833939628e-06\\
4	5.4	2.94070682853018e-06	2.94070682853018e-06\\
4	5.5	5.87103635044624e-06	5.87103635044624e-06\\
4	5.6	1.25000413937816e-05	1.25000413937816e-05\\
4	5.7	2.47053047590879e-05	2.47053047590879e-05\\
4	5.8	4.49889969973585e-05	4.49889969973585e-05\\
4	5.9	7.75183714996979e-05	7.75183714996979e-05\\
4	6	0.000130351554987174	0.000130351554987174\\
4.1	0	0.000172059152337458	0.000172059152337458\\
4.1	0.1	0.000135117424219139	0.000135117424219139\\
4.1	0.2	0.000106399651020765	0.000106399651020765\\
4.1	0.3	8.16664924745893e-05	8.16664924745893e-05\\
4.1	0.4	5.91071379265917e-05	5.91071379265917e-05\\
4.1	0.5	3.93516335996159e-05	3.93516335996159e-05\\
4.1	0.6	2.41897932010826e-05	2.41897932010826e-05\\
4.1	0.7	1.45453661891331e-05	1.45453661891331e-05\\
4.1	0.8	9.65887493827153e-06	9.65887493827153e-06\\
4.1	0.9	8.05751309778369e-06	8.05751309778369e-06\\
4.1	1	8.10706047598882e-06	8.10706047598882e-06\\
4.1	1.1	8.3821617113319e-06	8.3821617113319e-06\\
4.1	1.2	1.00111564813527e-05	1.00111564813527e-05\\
4.1	1.3	1.47036420805281e-05	1.47036420805281e-05\\
4.1	1.4	2.22966586021015e-05	2.22966586021015e-05\\
4.1	1.5	3.23057269186994e-05	3.23057269186994e-05\\
4.1	1.6	4.51014401928216e-05	4.51014401928216e-05\\
4.1	1.7	6.03087227774827e-05	6.03087227774827e-05\\
4.1	1.8	7.5007345334165e-05	7.5007345334165e-05\\
4.1	1.9	9.50559664474969e-05	9.50559664474969e-05\\
4.1	2	0.000136970895264499	0.000136970895264499\\
4.1	2.1	0.000215711599500688	0.000215711599500688\\
4.1	2.2	0.000292349659374025	0.000292349659374025\\
4.1	2.3	0.000245284538528664	0.000245284538528664\\
4.1	2.4	0.000136378791101753	0.000136378791101753\\
4.1	2.5	7.34582422097835e-05	7.34582422097835e-05\\
4.1	2.6	5.04304187329604e-05	5.04304187329604e-05\\
4.1	2.7	4.20044149031457e-05	4.20044149031457e-05\\
4.1	2.8	3.06967374120695e-05	3.06967374120695e-05\\
4.1	2.9	2.43111227337456e-05	2.43111227337456e-05\\
4.1	3	2.9357243170214e-05	2.9357243170214e-05\\
4.1	3.1	5.06901343779072e-05	5.06901343779072e-05\\
4.1	3.2	0.000105020986342724	0.000105020986342724\\
4.1	3.3	0.000228161261221393	0.000228161261221393\\
4.1	3.4	0.000467740802684166	0.000467740802684166\\
4.1	3.5	0.000825139426051031	0.000825139426051031\\
4.1	3.6	0.00104458663302418	0.00104458663302418\\
4.1	3.7	0.000876245798636837	0.000876245798636837\\
4.1	3.8	0.0005652217855364	0.0005652217855364\\
4.1	3.9	0.000294424600477931	0.000294424600477931\\
4.1	4	0.000126192200097044	0.000126192200097044\\
4.1	4.1	5.39496829348816e-05	5.39496829348816e-05\\
4.1	4.2	2.4150110008114e-05	2.4150110008114e-05\\
4.1	4.3	9.92138979598439e-06	9.92138979598439e-06\\
4.1	4.4	3.6880839270434e-06	3.6880839270434e-06\\
4.1	4.5	1.53204744176494e-06	1.53204744176494e-06\\
4.1	4.6	1.00604471513049e-06	1.00604471513049e-06\\
4.1	4.7	1.23518545041587e-06	1.23518545041587e-06\\
4.1	4.8	2.3488217413318e-06	2.3488217413318e-06\\
4.1	4.9	4.8699775688953e-06	4.8699775688953e-06\\
4.1	5	7.1161849125228e-06	7.1161849125228e-06\\
4.1	5.1	6.06315296164931e-06	6.06315296164931e-06\\
4.1	5.2	4.07785876906774e-06	4.07785876906774e-06\\
4.1	5.3	3.14447806991377e-06	3.14447806991377e-06\\
4.1	5.4	2.89346774060729e-06	2.89346774060729e-06\\
4.1	5.5	3.53769288903218e-06	3.53769288903218e-06\\
4.1	5.6	5.78275822825937e-06	5.78275822825937e-06\\
4.1	5.7	1.11453282970047e-05	1.11453282970047e-05\\
4.1	5.8	2.27580785429343e-05	2.27580785429343e-05\\
4.1	5.9	4.61097204605038e-05	4.61097204605038e-05\\
4.1	6	8.97672686615963e-05	8.97672686615963e-05\\
4.2	0	0.000161925910044192	0.000161925910044192\\
4.2	0.1	0.000123573858611687	0.000123573858611687\\
4.2	0.2	9.18679190814905e-05	9.18679190814905e-05\\
4.2	0.3	6.47881758925884e-05	6.47881758925884e-05\\
4.2	0.4	4.26172371113949e-05	4.26172371113949e-05\\
4.2	0.5	2.64500514318537e-05	2.64500514318537e-05\\
4.2	0.6	1.63149570791833e-05	1.63149570791833e-05\\
4.2	0.7	1.08057431006868e-05	1.08057431006868e-05\\
4.2	0.8	8.2414972461341e-06	8.2414972461341e-06\\
4.2	0.9	7.44941267886312e-06	7.44941267886312e-06\\
4.2	1	7.43252355314677e-06	7.43252355314677e-06\\
4.2	1.1	7.48097233348054e-06	7.48097233348054e-06\\
4.2	1.2	8.13626019777467e-06	8.13626019777467e-06\\
4.2	1.3	1.05621076484175e-05	1.05621076484175e-05\\
4.2	1.4	1.5343134048779e-05	1.5343134048779e-05\\
4.2	1.5	2.2909683450399e-05	2.2909683450399e-05\\
4.2	1.6	3.46136825868819e-05	3.46136825868819e-05\\
4.2	1.7	5.0379794720116e-05	5.0379794720116e-05\\
4.2	1.8	6.4757088916466e-05	6.4757088916466e-05\\
4.2	1.9	8.87889922881779e-05	8.87889922881779e-05\\
4.2	2	0.000145488704662155	0.000145488704662155\\
4.2	2.1	0.000241223800508047	0.000241223800508047\\
4.2	2.2	0.000327033841036262	0.000327033841036262\\
4.2	2.3	0.000246631041527006	0.000246631041527006\\
4.2	2.4	0.000112748863384654	0.000112748863384654\\
4.2	2.5	5.61278503798389e-05	5.61278503798389e-05\\
4.2	2.6	4.19595166912184e-05	4.19595166912184e-05\\
4.2	2.7	3.69911283365946e-05	3.69911283365946e-05\\
4.2	2.8	2.35843934734357e-05	2.35843934734357e-05\\
4.2	2.9	1.69472225323521e-05	1.69472225323521e-05\\
4.2	3	2.10923101758447e-05	2.10923101758447e-05\\
4.2	3.1	3.79569167543701e-05	3.79569167543701e-05\\
4.2	3.2	8.21482485517508e-05	8.21482485517508e-05\\
4.2	3.3	0.000191846102957135	0.000191846102957135\\
4.2	3.4	0.000426590535304895	0.000426590535304895\\
4.2	3.5	0.000833822429067326	0.000833822429067326\\
4.2	3.6	0.00118322354590967	0.00118322354590967\\
4.2	3.7	0.000956377668863122	0.000956377668863122\\
4.2	3.8	0.000492130513571774	0.000492130513571774\\
4.2	3.9	0.000187592020180954	0.000187592020180954\\
4.2	4	6.032866281967e-05	6.032866281967e-05\\
4.2	4.1	2.0347095761804e-05	2.0347095761804e-05\\
4.2	4.2	7.54397820570605e-06	7.54397820570605e-06\\
4.2	4.3	3.00908509427965e-06	3.00908509427965e-06\\
4.2	4.4	1.42604674991886e-06	1.42604674991886e-06\\
4.2	4.5	9.53720057137014e-07	9.53720057137014e-07\\
4.2	4.6	1.00326718673275e-06	1.00326718673275e-06\\
4.2	4.7	1.58780249997158e-06	1.58780249997158e-06\\
4.2	4.8	3.12665788358247e-06	3.12665788358247e-06\\
4.2	4.9	5.78625718088212e-06	5.78625718088212e-06\\
4.2	5	7.86784822737729e-06	7.86784822737729e-06\\
4.2	5.1	7.32771536051047e-06	7.32771536051047e-06\\
4.2	5.2	5.44977129194585e-06	5.44977129194585e-06\\
4.2	5.3	4.19479435710276e-06	4.19479435710276e-06\\
4.2	5.4	3.6845911465034e-06	3.6845911465034e-06\\
4.2	5.5	3.68878033478737e-06	3.68878033478737e-06\\
4.2	5.6	4.46770990049399e-06	4.46770990049399e-06\\
4.2	5.7	6.66729684015118e-06	6.66729684015118e-06\\
4.2	5.8	1.18168649983321e-05	1.18168649983321e-05\\
4.2	5.9	2.35341464181513e-05	2.35341464181513e-05\\
4.2	6	4.93473632055196e-05	4.93473632055196e-05\\
4.3	0	0.000150450100272077	0.000150450100272077\\
4.3	0.1	0.000108456399027245	0.000108456399027245\\
4.3	0.2	7.4458569492805e-05	7.4458569492805e-05\\
4.3	0.3	4.82974998182239e-05	4.82974998182239e-05\\
4.3	0.4	3.01153422397802e-05	3.01153422397802e-05\\
4.3	0.5	1.88940303525438e-05	1.88940303525438e-05\\
4.3	0.6	1.26152834332745e-05	1.26152834332745e-05\\
4.3	0.7	9.32987241979258e-06	9.32987241979258e-06\\
4.3	0.8	7.78506205525071e-06	7.78506205525071e-06\\
4.3	0.9	7.25929574069862e-06	7.25929574069862e-06\\
4.3	1	7.12173505813189e-06	7.12173505813189e-06\\
4.3	1.1	6.96637168138038e-06	6.96637168138038e-06\\
4.3	1.2	7.08506614940406e-06	7.08506614940406e-06\\
4.3	1.3	8.24637065727928e-06	8.24637065727928e-06\\
4.3	1.4	1.1217552158558e-05	1.1217552158558e-05\\
4.3	1.5	1.71139177760215e-05	1.71139177760215e-05\\
4.3	1.6	2.8803465329191e-05	2.8803465329191e-05\\
4.3	1.7	4.92196661469636e-05	4.92196661469636e-05\\
4.3	1.8	6.71721310860524e-05	6.71721310860524e-05\\
4.3	1.9	9.58739986454408e-05	9.58739986454408e-05\\
4.3	2	0.000180236885994863	0.000180236885994863\\
4.3	2.1	0.000290543465466497	0.000290543465466497\\
4.3	2.2	0.00037697195479158	0.00037697195479158\\
4.3	2.3	0.000266172505999192	0.000266172505999192\\
4.3	2.4	0.000103029410254169	0.000103029410254169\\
4.3	2.5	4.77896020828589e-05	4.77896020828589e-05\\
4.3	2.6	3.77467097053016e-05	3.77467097053016e-05\\
4.3	2.7	3.2585834296554e-05	3.2585834296554e-05\\
4.3	2.8	1.7722968053073e-05	1.7722968053073e-05\\
4.3	2.9	1.30026007489384e-05	1.30026007489384e-05\\
4.3	3	1.75214597212619e-05	1.75214597212619e-05\\
4.3	3.1	2.99474092259428e-05	2.99474092259428e-05\\
4.3	3.2	5.90727952520732e-05	5.90727952520732e-05\\
4.3	3.3	0.000136473416541705	0.000136473416541705\\
4.3	3.4	0.000327718553485685	0.000327718553485685\\
4.3	3.5	0.000735970586710684	0.000735970586710684\\
4.3	3.6	0.00126377369125261	0.00126377369125261\\
4.3	3.7	0.00101134123320374	0.00101134123320374\\
4.3	3.8	0.00038932223883634	0.00038932223883634\\
4.3	3.9	0.000105414844115323	0.000105414844115323\\
4.3	4	2.62690128667261e-05	2.62690128667261e-05\\
4.3	4.1	7.6115991513746e-06	7.6115991513746e-06\\
4.3	4.2	2.85745494049124e-06	2.85745494049124e-06\\
4.3	4.3	1.42609543196198e-06	1.42609543196198e-06\\
4.3	4.4	9.72007970648513e-07	9.72007970648513e-07\\
4.3	4.5	9.2931102455079e-07	9.2931102455079e-07\\
4.3	4.6	1.24925473709083e-06	1.24925473709083e-06\\
4.3	4.7	2.18451680662216e-06	2.18451680662216e-06\\
4.3	4.8	4.16297280405729e-06	4.16297280405729e-06\\
4.3	4.9	7.02362037660204e-06	7.02362037660204e-06\\
4.3	5	9.10463301640521e-06	9.10463301640521e-06\\
4.3	5.1	8.81379709454927e-06	8.81379709454927e-06\\
4.3	5.2	6.98108044035491e-06	6.98108044035491e-06\\
4.3	5.3	5.35965382848763e-06	5.35965382848763e-06\\
4.3	5.4	4.50888788377474e-06	4.50888788377474e-06\\
4.3	5.5	4.21517274810028e-06	4.21517274810028e-06\\
4.3	5.6	4.41780289693685e-06	4.41780289693685e-06\\
4.3	5.7	5.38616866505713e-06	5.38616866505713e-06\\
4.3	5.8	7.76095457107621e-06	7.76095457107621e-06\\
4.3	5.9	1.31170036704332e-05	1.31170036704332e-05\\
4.3	6	2.53160713665401e-05	2.53160713665401e-05\\
4.4	0	0.000130568581032572	0.000130568581032572\\
4.4	0.1	8.74343784667176e-05	8.74343784667176e-05\\
4.4	0.2	5.59965617006999e-05	5.59965617006999e-05\\
4.4	0.3	3.49915880169745e-05	3.49915880169745e-05\\
4.4	0.4	2.22011889837638e-05	2.22011889837638e-05\\
4.4	0.5	1.49523734941695e-05	1.49523734941695e-05\\
4.4	0.6	1.09910345428409e-05	1.09910345428409e-05\\
4.4	0.7	8.87099535778148e-06	8.87099535778148e-06\\
4.4	0.8	7.83460741822707e-06	7.83460741822707e-06\\
4.4	0.9	7.41979294813271e-06	7.41979294813271e-06\\
4.4	1	7.19108164216882e-06	7.19108164216882e-06\\
4.4	1.1	6.88671875175261e-06	6.88671875175261e-06\\
4.4	1.2	6.67115643089712e-06	6.67115643089712e-06\\
4.4	1.3	7.03799394294525e-06	7.03799394294525e-06\\
4.4	1.4	8.64271432151216e-06	8.64271432151216e-06\\
4.4	1.5	1.27491892858269e-05	1.27491892858269e-05\\
4.4	1.6	2.31191535503597e-05	2.31191535503597e-05\\
4.4	1.7	4.8358127537963e-05	4.8358127537963e-05\\
4.4	1.8	7.6653810051551e-05	7.6653810051551e-05\\
4.4	1.9	0.000112960968941164	0.000112960968941164\\
4.4	2	0.00023563148654092	0.00023563148654092\\
4.4	2.1	0.000343009014174551	0.000343009014174551\\
4.4	2.2	0.000407369263025601	0.000407369263025601\\
4.4	2.3	0.000281639231976224	0.000281639231976224\\
4.4	2.4	0.000100329457491258	0.000100329457491258\\
4.4	2.5	4.54066565832465e-05	4.54066565832465e-05\\
4.4	2.6	3.7006183935448e-05	3.7006183935448e-05\\
4.4	2.7	2.83429653170412e-05	2.83429653170412e-05\\
4.4	2.8	1.30740272558784e-05	1.30740272558784e-05\\
4.4	2.9	1.10145770873931e-05	1.10145770873931e-05\\
4.4	3	1.57721079961826e-05	1.57721079961826e-05\\
4.4	3.1	2.34014761167739e-05	2.34014761167739e-05\\
4.4	3.2	3.81765584356355e-05	3.81765584356355e-05\\
4.4	3.3	8.14964048289952e-05	8.14964048289952e-05\\
4.4	3.4	0.000214864849574769	0.000214864849574769\\
4.4	3.5	0.000581823038311757	0.000581823038311757\\
4.4	3.6	0.00130517966543031	0.00130517966543031\\
4.4	3.7	0.00106717152936127	0.00106717152936127\\
4.4	3.8	0.00028518904237432	0.00028518904237432\\
4.4	3.9	5.48152950382229e-05	5.48152950382229e-05\\
4.4	4	1.10460465603361e-05	1.10460465603361e-05\\
4.4	4.1	3.19154137155829e-06	3.19154137155829e-06\\
4.4	4.2	1.50524126305524e-06	1.50524126305524e-06\\
4.4	4.3	1.04362010930316e-06	1.04362010930316e-06\\
4.4	4.4	9.59921004347243e-07	9.59921004347243e-07\\
4.4	4.5	1.14912623887312e-06	1.14912623887312e-06\\
4.4	4.6	1.76023740801899e-06	1.76023740801899e-06\\
4.4	4.7	3.1472550983828e-06	3.1472550983828e-06\\
4.4	4.8	5.63827104337609e-06	5.63827104337609e-06\\
4.4	4.9	8.75815463635233e-06	8.75815463635233e-06\\
4.4	5	1.0835448032977e-05	1.0835448032977e-05\\
4.4	5.1	1.05637547512569e-05	1.05637547512569e-05\\
4.4	5.2	8.6292772916651e-06	8.6292772916651e-06\\
4.4	5.3	6.61641473163872e-06	6.61641473163872e-06\\
4.4	5.4	5.30829510701644e-06	5.30829510701644e-06\\
4.4	5.5	4.66334268297992e-06	4.66334268297992e-06\\
4.4	5.6	4.5040476036356e-06	4.5040476036356e-06\\
4.4	5.7	4.86337007780497e-06	4.86337007780497e-06\\
4.4	5.8	6.01270870992612e-06	6.01270870992612e-06\\
4.4	5.9	8.60352855789138e-06	8.60352855789138e-06\\
4.4	6	1.42429256632156e-05	1.42429256632156e-05\\
4.5	0	0.000103859887794589	0.000103859887794589\\
4.5	0.1	6.57561140507163e-05	6.57561140507163e-05\\
4.5	0.2	4.10545212933737e-05	4.10545212933737e-05\\
4.5	0.3	2.61731206605832e-05	2.61731206605832e-05\\
4.5	0.4	1.76897794532966e-05	1.76897794532966e-05\\
4.5	0.5	1.29816699757368e-05	1.29816699757368e-05\\
4.5	0.6	1.03738794870192e-05	1.03738794870192e-05\\
4.5	0.7	8.95607821083165e-06	8.95607821083165e-06\\
4.5	0.8	8.26505199863516e-06	8.26505199863516e-06\\
4.5	0.9	7.96329427325632e-06	7.96329427325632e-06\\
4.5	1	7.71938490874774e-06	7.71938490874774e-06\\
4.5	1.1	7.34685037182258e-06	7.34685037182258e-06\\
4.5	1.2	6.94457624391874e-06	6.94457624391874e-06\\
4.5	1.3	6.82309750404261e-06	6.82309750404261e-06\\
4.5	1.4	7.42877012790195e-06	7.42877012790195e-06\\
4.5	1.5	9.74110204198071e-06	9.74110204198071e-06\\
4.5	1.6	1.7294819048928e-05	1.7294819048928e-05\\
4.5	1.7	4.26974282977188e-05	4.26974282977188e-05\\
4.5	1.8	8.71173205401952e-05	8.71173205401952e-05\\
4.5	1.9	0.000137572630038795	0.000137572630038795\\
4.5	2	0.000295418803511438	0.000295418803511438\\
4.5	2.1	0.000343200546009774	0.000343200546009774\\
4.5	2.2	0.000333698459260059	0.000333698459260059\\
4.5	2.3	0.000257822621777264	0.000257822621777264\\
4.5	2.4	0.000105114059183595	0.000105114059183595\\
4.5	2.5	5.12939054031091e-05	5.12939054031091e-05\\
4.5	2.6	4.57565863185999e-05	4.57565863185999e-05\\
4.5	2.7	2.74331518783934e-05	2.74331518783934e-05\\
4.5	2.8	1.06049713059903e-05	1.06049713059903e-05\\
4.5	2.9	1.0534082094396e-05	1.0534082094396e-05\\
4.5	3	1.42326327718855e-05	1.42326327718855e-05\\
4.5	3.1	1.75560352110052e-05	1.75560352110052e-05\\
4.5	3.2	2.33423026795878e-05	2.33423026795878e-05\\
4.5	3.3	4.27132396042178e-05	4.27132396042178e-05\\
4.5	3.4	0.000120440306337873	0.000120440306337873\\
4.5	3.5	0.000415370874218976	0.000415370874218976\\
4.5	3.6	0.00130484098919419	0.00130484098919419\\
4.5	3.7	0.00116552364811988	0.00116552364811988\\
4.5	3.8	0.00019966073779053	0.00019966073779053\\
4.5	3.9	2.67888755653346e-05	2.67888755653346e-05\\
4.5	4	4.71033309063083e-06	4.71033309063083e-06\\
4.5	4.1	1.72285157626287e-06	1.72285157626287e-06\\
4.5	4.2	1.17622738631082e-06	1.17622738631082e-06\\
4.5	4.3	1.08960627514688e-06	1.08960627514688e-06\\
4.5	4.4	1.23071557747364e-06	1.23071557747364e-06\\
4.5	4.5	1.69202990983363e-06	1.69202990983363e-06\\
4.5	4.6	2.73400634023982e-06	2.73400634023982e-06\\
4.5	4.7	4.72770560540455e-06	4.72770560540455e-06\\
4.5	4.8	7.78585422567548e-06	7.78585422567548e-06\\
4.5	4.9	1.10961735476582e-05	1.10961735476582e-05\\
4.5	5	1.29982559315679e-05	1.29982559315679e-05\\
4.5	5.1	1.24871199574166e-05	1.24871199574166e-05\\
4.5	5.2	1.02815114872845e-05	1.02815114872845e-05\\
4.5	5.3	7.85218971287086e-06	7.85218971287086e-06\\
4.5	5.4	6.0613205681357e-06	6.0613205681357e-06\\
4.5	5.5	5.00306006799223e-06	5.00306006799223e-06\\
4.5	5.6	4.49979648394984e-06	4.49979648394984e-06\\
4.5	5.7	4.4541111578171e-06	4.4541111578171e-06\\
4.5	5.8	4.93921948446396e-06	4.93921948446396e-06\\
4.5	5.9	6.22767428779108e-06	6.22767428779108e-06\\
4.5	6	8.97875211469408e-06	8.97875211469408e-06\\
4.6	0	7.78806348238252e-05	7.78806348238252e-05\\
4.6	0.1	4.84342242301647e-05	4.84342242301647e-05\\
4.6	0.2	3.08263135841213e-05	3.08263135841213e-05\\
4.6	0.3	2.07473133108059e-05	2.07473133108059e-05\\
4.6	0.4	1.51093309374879e-05	1.51093309374879e-05\\
4.6	0.5	1.19717213490764e-05	1.19717213490764e-05\\
4.6	0.6	1.02424058822039e-05	1.02424058822039e-05\\
4.6	0.7	9.36217173708989e-06	9.36217173708989e-06\\
4.6	0.8	9.01532566517219e-06	9.01532566517219e-06\\
4.6	0.9	8.9210935423431e-06	8.9210935423431e-06\\
4.6	1	8.80078966509235e-06	8.80078966509235e-06\\
4.6	1.1	8.5008422414002e-06	8.5008422414002e-06\\
4.6	1.2	8.07826091807737e-06	8.07826091807737e-06\\
4.6	1.3	7.72213284797526e-06	7.72213284797526e-06\\
4.6	1.4	7.68099095290619e-06	7.68099095290619e-06\\
4.6	1.5	8.52198815966958e-06	8.52198815966958e-06\\
4.6	1.6	1.2673635983978e-05	1.2673635983978e-05\\
4.6	1.7	3.25491673077799e-05	3.25491673077799e-05\\
4.6	1.8	9.2672909659827e-05	9.2672909659827e-05\\
4.6	1.9	0.000170877123719963	0.000170877123719963\\
4.6	2	0.000311442912649241	0.000311442912649241\\
4.6	2.1	0.000213190260067375	0.000213190260067375\\
4.6	2.2	0.000148246151842716	0.000148246151842716\\
4.6	2.3	0.000159379404720317	0.000159379404720317\\
4.6	2.4	0.000123588401870403	0.000123588401870403\\
4.6	2.5	7.81736247714094e-05	7.81736247714094e-05\\
4.6	2.6	8.19078765938887e-05	8.19078765938887e-05\\
4.6	2.7	2.76461099873893e-05	2.76461099873893e-05\\
4.6	2.8	1.04508044573724e-05	1.04508044573724e-05\\
4.6	2.9	1.18590530304689e-05	1.18590530304689e-05\\
4.6	3	1.33676177602457e-05	1.33676177602457e-05\\
4.6	3.1	1.33095197896563e-05	1.33095197896563e-05\\
4.6	3.2	1.45263692181262e-05	1.45263692181262e-05\\
4.6	3.3	2.15230219389856e-05	2.15230219389856e-05\\
4.6	3.4	5.51122276858896e-05	5.51122276858896e-05\\
4.6	3.5	0.000247743857912147	0.000247743857912147\\
4.6	3.6	0.00119191173737016	0.00119191173737016\\
4.6	3.7	0.00134339182329259	0.00134339182329259\\
4.6	3.8	0.000130223558916101	0.000130223558916101\\
4.6	3.9	1.16409694259594e-05	1.16409694259594e-05\\
4.6	4	2.31742056929118e-06	2.31742056929118e-06\\
4.6	4.1	1.41016415662363e-06	1.41016415662363e-06\\
4.6	4.2	1.36514745540483e-06	1.36514745540483e-06\\
4.6	4.3	1.53949631951876e-06	1.53949631951876e-06\\
4.6	4.4	1.97757929990055e-06	1.97757929990055e-06\\
4.6	4.5	2.87989453158606e-06	2.87989453158606e-06\\
4.6	4.6	4.54913626202079e-06	4.54913626202079e-06\\
4.6	4.7	7.24104337178953e-06	7.24104337178953e-06\\
4.6	4.8	1.07333506645495e-05	1.07333506645495e-05\\
4.6	4.9	1.39326742827245e-05	1.39326742827245e-05\\
4.6	5	1.53386636004782e-05	1.53386636004782e-05\\
4.6	5.1	1.43324658166368e-05	1.43324658166368e-05\\
4.6	5.2	1.17358567012518e-05	1.17358567012518e-05\\
4.6	5.3	8.90810236009903e-06	8.90810236009903e-06\\
4.6	5.4	6.69118885674202e-06	6.69118885674202e-06\\
4.6	5.5	5.24639041069661e-06	5.24639041069661e-06\\
4.6	5.6	4.42222753983134e-06	4.42222753983134e-06\\
4.6	5.7	4.06473516119332e-06	4.06473516119332e-06\\
4.6	5.8	4.13211411151631e-06	4.13211411151631e-06\\
4.6	5.9	4.71476944269339e-06	4.71476944269339e-06\\
4.6	6	6.08736466422263e-06	6.08736466422263e-06\\
4.7	0	5.74230227624814e-05	5.74230227624814e-05\\
4.7	0.1	3.63044677632015e-05	3.63044677632015e-05\\
4.7	0.2	2.41607720261168e-05	2.41607720261168e-05\\
4.7	0.3	1.73105143609549e-05	1.73105143609549e-05\\
4.7	0.4	1.34791266012223e-05	1.34791266012223e-05\\
4.7	0.5	1.13712961713833e-05	1.13712961713833e-05\\
4.7	0.6	1.0301257467647e-05	1.0301257467647e-05\\
4.7	0.7	9.91245055571742e-06	9.91245055571742e-06\\
4.7	0.8	9.96514165448733e-06	9.96514165448733e-06\\
4.7	0.9	1.02145161215622e-05	1.02145161215622e-05\\
4.7	1	1.04186106641851e-05	1.04186106641851e-05\\
4.7	1.1	1.04412484891242e-05	1.04412484891242e-05\\
4.7	1.2	1.03089148277504e-05	1.03089148277504e-05\\
4.7	1.3	1.0120645737877e-05	1.0120645737877e-05\\
4.7	1.4	9.91186706542457e-06	9.91186706542457e-06\\
4.7	1.5	9.77386105286991e-06	9.77386105286991e-06\\
4.7	1.6	1.08750430657294e-05	1.08750430657294e-05\\
4.7	1.7	2.19286783649051e-05	2.19286783649051e-05\\
4.7	1.8	8.88658384305956e-05	8.88658384305956e-05\\
4.7	1.9	0.00022946131025273	0.00022946131025273\\
4.7	2	0.000211917777873734	0.000211917777873734\\
4.7	2.1	6.35012461854003e-05	6.35012461854003e-05\\
4.7	2.2	5.06380388838477e-05	5.06380388838477e-05\\
4.7	2.3	6.86811667704405e-05	6.86811667704405e-05\\
4.7	2.4	0.000159690077768387	0.000159690077768387\\
4.7	2.5	0.000169812058841113	0.000169812058841113\\
4.7	2.6	0.000170668457753282	0.000170668457753282\\
4.7	2.7	2.20561246706464e-05	2.20561246706464e-05\\
4.7	2.8	1.41966039228875e-05	1.41966039228875e-05\\
4.7	2.9	1.7417422653807e-05	1.7417422653807e-05\\
4.7	3	1.60063905848811e-05	1.60063905848811e-05\\
4.7	3.1	1.28002387343257e-05	1.28002387343257e-05\\
4.7	3.2	1.09873929629486e-05	1.09873929629486e-05\\
4.7	3.3	1.22314562759158e-05	1.22314562759158e-05\\
4.7	3.4	2.23548910122782e-05	2.23548910122782e-05\\
4.7	3.5	9.96857536000212e-05	9.96857536000212e-05\\
4.7	3.6	0.000862332528694269	0.000862332528694269\\
4.7	3.7	0.00161595478750413	0.00161595478750413\\
4.7	3.8	6.94863175978405e-05	6.94863175978405e-05\\
4.7	3.9	4.46201761402027e-06	4.46201761402027e-06\\
4.7	4	1.82309338522875e-06	1.82309338522875e-06\\
4.7	4.1	1.92515249377491e-06	1.92515249377491e-06\\
4.7	4.2	2.28550052956773e-06	2.28550052956773e-06\\
4.7	4.3	2.83886203636671e-06	2.83886203636671e-06\\
4.7	4.4	3.75143241273546e-06	3.75143241273546e-06\\
4.7	4.5	5.25578952239528e-06	5.25578952239528e-06\\
4.7	4.6	7.5825711841492e-06	7.5825711841492e-06\\
4.7	4.7	1.07399513723263e-05	1.07399513723263e-05\\
4.7	4.8	1.41833136426058e-05	1.41833136426058e-05\\
4.7	4.9	1.6760503116074e-05	1.6760503116074e-05\\
4.7	5	1.73358268847836e-05	1.73358268847836e-05\\
4.7	5.1	1.56989689428948e-05	1.56989689428948e-05\\
4.7	5.2	1.2726888098969e-05	1.2726888098969e-05\\
4.7	5.3	9.61309197872566e-06	9.61309197872566e-06\\
4.7	5.4	7.10443761563599e-06	7.10443761563599e-06\\
4.7	5.5	5.37603535592445e-06	5.37603535592445e-06\\
4.7	5.6	4.3024125743047e-06	4.3024125743047e-06\\
4.7	5.7	3.71351061014169e-06	3.71351061014169e-06\\
4.7	5.8	3.50830055501474e-06	3.50830055501474e-06\\
4.7	5.9	3.68006634512816e-06	3.68006634512816e-06\\
4.7	6	4.33081093750571e-06	4.33081093750571e-06\\
4.8	0	4.29047630893928e-05	4.29047630893928e-05\\
4.8	0.1	2.81107506684742e-05	2.81107506684742e-05\\
4.8	0.2	1.96859086204466e-05	1.96859086204466e-05\\
4.8	0.3	1.49221521387606e-05	1.49221521387606e-05\\
4.8	0.4	1.22709183821919e-05	1.22709183821919e-05\\
4.8	0.5	1.08946441671523e-05	1.08946441671523e-05\\
4.8	0.6	1.0358817840987e-05	1.0358817840987e-05\\
4.8	0.7	1.04221503600786e-05	1.04221503600786e-05\\
4.8	0.8	1.08956850251839e-05	1.08956850251839e-05\\
4.8	0.9	1.15779224258822e-05	1.15779224258822e-05\\
4.8	1	1.22812788810803e-05	1.22812788810803e-05\\
4.8	1.1	1.2917072714308e-05	1.2917072714308e-05\\
4.8	1.2	1.35407224734573e-05	1.35407224734573e-05\\
4.8	1.3	1.42714237993396e-05	1.42714237993396e-05\\
4.8	1.4	1.50723721422199e-05	1.50723721422199e-05\\
4.8	1.5	1.54405553712741e-05	1.54405553712741e-05\\
4.8	1.6	1.46008729539735e-05	1.46008729539735e-05\\
4.8	1.7	1.5828321959472e-05	1.5828321959472e-05\\
4.8	1.8	6.92105987034396e-05	6.92105987034396e-05\\
4.8	1.9	0.00035542502836359	0.00035542502836359\\
4.8	2	7.16683852615328e-05	7.16683852615328e-05\\
4.8	2.1	2.07052636771856e-05	2.07052636771856e-05\\
4.8	2.2	3.00925747188017e-05	3.00925747188017e-05\\
4.8	2.3	5.32331359718383e-05	5.32331359718383e-05\\
4.8	2.4	0.000162973986698609	0.000162973986698609\\
4.8	2.5	0.000463787482145814	0.000463787482145814\\
4.8	2.6	0.00017046007591499	0.00017046007591499\\
4.8	2.7	1.92719552553831e-05	1.92719552553831e-05\\
4.8	2.8	3.0533938595846e-05	3.0533938595846e-05\\
4.8	2.9	3.36392967163715e-05	3.36392967163715e-05\\
4.8	3	2.70303374400109e-05	2.70303374400109e-05\\
4.8	3.1	1.94642323823285e-05	1.94642323823285e-05\\
4.8	3.2	1.38602420238747e-05	1.38602420238747e-05\\
4.8	3.3	1.07408646958694e-05	1.07408646958694e-05\\
4.8	3.4	1.11860623793039e-05	1.11860623793039e-05\\
4.8	3.5	2.59305732675093e-05	2.59305732675093e-05\\
4.8	3.6	0.000363722549367143	0.000363722549367143\\
4.8	3.7	0.00195753308586781	0.00195753308586781\\
4.8	3.8	2.32554277983912e-05	2.32554277983912e-05\\
4.8	3.9	2.53755004711854e-06	2.53755004711854e-06\\
4.8	4	3.16951064327856e-06	3.16951064327856e-06\\
4.8	4.1	4.14460412139756e-06	4.14460412139756e-06\\
4.8	4.2	4.9711524441945e-06	4.9711524441945e-06\\
4.8	4.3	5.89294694879284e-06	5.89294694879284e-06\\
4.8	4.4	7.17428785244536e-06	7.17428785244536e-06\\
4.8	4.5	9.03083596383565e-06	9.03083596383565e-06\\
4.8	4.6	1.15422724508423e-05	1.15422724508423e-05\\
4.8	4.7	1.44806776529119e-05	1.44806776529119e-05\\
4.8	4.8	1.7178870887729e-05	1.7178870887729e-05\\
4.8	4.9	1.86906975215564e-05	1.86906975215564e-05\\
4.8	5	1.83254385219349e-05	1.83254385219349e-05\\
4.8	5.1	1.61636061803928e-05	1.61636061803928e-05\\
4.8	5.2	1.30110939822618e-05	1.30110939822618e-05\\
4.8	5.3	9.82959290662908e-06	9.82959290662908e-06\\
4.8	5.4	7.22660054597464e-06	7.22660054597464e-06\\
4.8	5.5	5.3642159155323e-06	5.3642159155323e-06\\
4.8	5.6	4.1450725980528e-06	4.1450725980528e-06\\
4.8	5.7	3.40898060178321e-06	3.40898060178321e-06\\
4.8	5.8	3.0340622446083e-06	3.0340622446083e-06\\
4.8	5.9	2.96550560616054e-06	2.96550560616054e-06\\
4.8	6	3.22150401387797e-06	3.22150401387797e-06\\
4.9	0	3.29186450762242e-05	3.29186450762242e-05\\
4.9	0.1	2.24723254655585e-05	2.24723254655585e-05\\
4.9	0.2	1.64907018042444e-05	1.64907018042444e-05\\
4.9	0.3	1.30957492207317e-05	1.30957492207317e-05\\
4.9	0.4	1.12543254693196e-05	1.12543254693196e-05\\
4.9	0.5	1.04191001129608e-05	1.04191001129608e-05\\
4.9	0.6	1.03017410477009e-05	1.03017410477009e-05\\
4.9	0.7	1.07276644076223e-05	1.07276644076223e-05\\
4.9	0.8	1.15486571654829e-05	1.15486571654829e-05\\
4.9	0.9	1.26159213023143e-05	1.26159213023143e-05\\
4.9	1	1.38184244898503e-05	1.38184244898503e-05\\
4.9	1.1	1.51585263277928e-05	1.51585263277928e-05\\
4.9	1.2	1.6804532683354e-05	1.6804532683354e-05\\
4.9	1.3	1.90760403243165e-05	1.90760403243165e-05\\
4.9	1.4	2.23440480336401e-05	2.23440480336401e-05\\
4.9	1.5	2.67277109460648e-05	2.67277109460648e-05\\
4.9	1.6	3.09333213612892e-05	3.09333213612892e-05\\
4.9	1.7	2.84478748445469e-05	2.84478748445469e-05\\
4.9	1.8	3.14664967297369e-05	3.14664967297369e-05\\
4.9	1.9	0.000577113887068722	0.000577113887068722\\
4.9	2	1.93719142813654e-05	1.93719142813654e-05\\
4.9	2.1	1.44045425864112e-05	1.44045425864112e-05\\
4.9	2.2	2.71037263571504e-05	2.71037263571504e-05\\
4.9	2.3	5.48835363761837e-05	5.48835363761837e-05\\
4.9	2.4	9.59675372482923e-05	9.59675372482923e-05\\
4.9	2.5	0.00115374571128957	0.00115374571128957\\
4.9	2.6	3.19367610644147e-05	3.19367610644147e-05\\
4.9	2.7	5.76588015651555e-05	5.76588015651555e-05\\
4.9	2.8	7.47430736724444e-05	7.47430736724444e-05\\
4.9	2.9	6.42074894981867e-05	6.42074894981867e-05\\
4.9	3	4.87270425793348e-05	4.87270425793348e-05\\
4.9	3.1	3.53026260657424e-05	3.53026260657424e-05\\
4.9	3.2	2.50562630381915e-05	2.50562630381915e-05\\
4.9	3.3	1.74795218157511e-05	1.74795218157511e-05\\
4.9	3.4	1.2069301894105e-05	1.2069301894105e-05\\
4.9	3.5	9.68466187850594e-06	9.68466187850594e-06\\
4.9	3.6	4.52478136042659e-05	4.52478136042659e-05\\
4.9	3.7	0.00228473045592927	0.00228473045592927\\
4.9	3.8	4.38997911271982e-06	4.38997911271982e-06\\
4.9	3.9	6.85594737005404e-06	6.85594737005404e-06\\
4.9	4	9.56163020951024e-06	9.56163020951024e-06\\
4.9	4.1	9.99901715070574e-06	9.99901715070574e-06\\
4.9	4.2	1.00943740201644e-05	1.00943740201644e-05\\
4.9	4.3	1.04971269528915e-05	1.04971269528915e-05\\
4.9	4.4	1.14174851079488e-05	1.14174851079488e-05\\
4.9	4.5	1.29075337956617e-05	1.29075337956617e-05\\
4.9	4.6	1.48541180834724e-05	1.48541180834724e-05\\
4.9	4.7	1.69038276715603e-05	1.69038276715603e-05\\
4.9	4.8	1.84642657362688e-05	1.84642657362688e-05\\
4.9	4.9	1.88939123903727e-05	1.88939123903727e-05\\
4.9	5	1.78425976930297e-05	1.78425976930297e-05\\
4.9	5.1	1.54959001460999e-05	1.54959001460999e-05\\
4.9	5.2	1.24816529532426e-05	1.24816529532426e-05\\
4.9	5.3	9.50496946534507e-06	9.50496946534507e-06\\
4.9	5.4	7.02763032371756e-06	7.02763032371756e-06\\
4.9	5.5	5.19412478801272e-06	5.19412478801272e-06\\
4.9	5.6	3.94217630475076e-06	3.94217630475076e-06\\
4.9	5.7	3.14092593845389e-06	3.14092593845389e-06\\
4.9	5.8	2.6742333734758e-06	2.6742333734758e-06\\
4.9	5.9	2.47067111444648e-06	2.47067111444648e-06\\
4.9	6	2.50950598902368e-06	2.50950598902368e-06\\
5	0	2.60100896069549e-05	2.60100896069549e-05\\
5	0.1	1.84622173760287e-05	1.84622173760287e-05\\
5	0.2	1.41004396643822e-05	1.41004396643822e-05\\
5	0.3	1.16363733357277e-05	1.16363733357277e-05\\
5	0.4	1.03698197341524e-05	1.03698197341524e-05\\
5	0.5	9.92683870915415e-06	9.92683870915415e-06\\
5	0.6	1.01004107561588e-05	1.01004107561588e-05\\
5	0.7	1.0753051337765e-05	1.0753051337765e-05\\
5	0.8	1.1766079610374e-05	1.1766079610374e-05\\
5	0.9	1.30374269539191e-05	1.30374269539191e-05\\
5	1	1.45257071474896e-05	1.45257071474896e-05\\
5	1.1	1.63161334795853e-05	1.63161334795853e-05\\
5	1.2	1.86747655311172e-05	1.86747655311172e-05\\
5	1.3	2.20823809547351e-05	2.20823809547351e-05\\
5	1.4	2.72756800411253e-05	2.72756800411253e-05\\
5	1.5	3.53469483978573e-05	3.53469483978573e-05\\
5	1.6	4.79953492312308e-05	4.79953492312308e-05\\
5	1.7	6.8170246368222e-05	6.8170246368222e-05\\
5	1.8	0.000101733364803184	0.000101733364803184\\
5	1.9	0.000764094265708204	0.000764094265708204\\
5	2	1.93762468799025e-05	1.93762468799025e-05\\
5	2.1	2.68838992948424e-05	2.68838992948424e-05\\
5	2.2	3.87490313887639e-05	3.87490313887639e-05\\
5	2.3	5.69011105708286e-05	5.69011105708286e-05\\
5	2.4	8.271741105978e-05	8.271741105978e-05\\
5	2.5	0.00172648662102185	0.00172648662102185\\
5	2.6	0.000145192768201793	0.000145192768201793\\
5	2.7	0.000126489724101654	0.000126489724101654\\
5	2.8	0.00010167431893844	0.00010167431893844\\
5	2.9	7.72178430023407e-05	7.72178430023407e-05\\
5	3	5.70996456341793e-05	5.70996456341793e-05\\
5	3.1	4.21147889928817e-05	4.21147889928817e-05\\
5	3.2	3.13376279196578e-05	3.13376279196578e-05\\
5	3.3	2.35458662964895e-05	2.35458662964895e-05\\
5	3.4	1.78350330893121e-05	1.78350330893121e-05\\
5	3.5	1.3653409884159e-05	1.3653409884159e-05\\
5	3.6	1.06521768326398e-05	1.06521768326398e-05\\
5	3.7	0.00247457726974944	0.00247457726974944\\
5	3.8	3.14122795844874e-05	3.14122795844874e-05\\
5	3.9	2.21497185240197e-05	2.21497185240197e-05\\
5	4	1.72112992636862e-05	1.72112992636862e-05\\
5	4.1	1.45966376942664e-05	1.45966376942664e-05\\
5	4.2	1.33642559621146e-05	1.33642559621146e-05\\
5	4.3	1.30649661100767e-05	1.30649661100767e-05\\
5	4.4	1.34655891348727e-05	1.34655891348727e-05\\
5	4.5	1.43869453612689e-05	1.43869453612689e-05\\
5	4.6	1.55946997033605e-05	1.55946997033605e-05\\
5	4.7	1.67429325526578e-05	1.67429325526578e-05\\
5	4.8	1.74016075398703e-05	1.74016075398703e-05\\
5	4.9	1.71850984169221e-05	1.71850984169221e-05\\
5	5	1.59310013972197e-05	1.59310013972197e-05\\
5	5.1	1.38058601166838e-05	1.38058601166838e-05\\
5	5.2	1.12359880519268e-05	1.12359880519268e-05\\
5	5.3	8.699872239145e-06	8.699872239145e-06\\
5	5.4	6.53466836229031e-06	6.53466836229031e-06\\
5	5.5	4.8715709373902e-06	4.8715709373902e-06\\
5	5.6	3.68769259125146e-06	3.68769259125146e-06\\
5	5.7	2.89310749193871e-06	2.89310749193871e-06\\
5	5.8	2.39426836793438e-06	2.39426836793438e-06\\
5	5.9	2.12298112580589e-06	2.12298112580589e-06\\
5	6	2.04473022680424e-06	2.04473022680424e-06\\
5.1	0	2.11725440129577e-05	2.11725440129577e-05\\
5.1	0.1	1.556525518613e-05	1.556525518613e-05\\
5.1	0.2	1.23064387720791e-05	1.23064387720791e-05\\
5.1	0.3	1.04959272041255e-05	1.04959272041255e-05\\
5.1	0.4	9.64005553109905e-06	9.64005553109905e-06\\
5.1	0.5	9.46618515280483e-06	9.46618515280483e-06\\
5.1	0.6	9.8115679340403e-06	9.8115679340403e-06\\
5.1	0.7	1.0558619356436e-05	1.0558619356436e-05\\
5.1	0.8	1.1608323826263e-05	1.1608323826263e-05\\
5.1	0.9	1.28914099851697e-05	1.28914099851697e-05\\
5.1	1	1.44083329321276e-05	1.44083329321276e-05\\
5.1	1.1	1.62749978337955e-05	1.62749978337955e-05\\
5.1	1.2	1.87502333057487e-05	1.87502333057487e-05\\
5.1	1.3	2.22301210905495e-05	2.22301210905495e-05\\
5.1	1.4	2.71761951571736e-05	2.71761951571736e-05\\
5.1	1.5	3.38148042486592e-05	3.38148042486592e-05\\
5.1	1.6	4.08417727877561e-05	4.08417727877561e-05\\
5.1	1.7	4.02623407102982e-05	4.02623407102982e-05\\
5.1	1.8	4.63048179557614e-05	4.63048179557614e-05\\
5.1	1.9	0.000717557095657427	0.000717557095657427\\
5.1	2	2.20647513544053e-05	2.20647513544053e-05\\
5.1	2.1	1.32104863326792e-05	1.32104863326792e-05\\
5.1	2.2	2.23073075603901e-05	2.23073075603901e-05\\
5.1	2.3	4.25715988060294e-05	4.25715988060294e-05\\
5.1	2.4	0.00010825633498001	0.00010825633498001\\
5.1	2.5	0.00125875197381201	0.00125875197381201\\
5.1	2.6	2.92737232401071e-05	2.92737232401071e-05\\
5.1	2.7	3.68359676612293e-05	3.68359676612293e-05\\
5.1	2.8	4.75192148375944e-05	4.75192148375944e-05\\
5.1	2.9	4.16760424217876e-05	4.16760424217876e-05\\
5.1	3	3.24845059172118e-05	3.24845059172118e-05\\
5.1	3.1	2.42401575362467e-05	2.42401575362467e-05\\
5.1	3.2	1.77285979096402e-05	1.77285979096402e-05\\
5.1	3.3	1.27278821978814e-05	1.27278821978814e-05\\
5.1	3.4	8.98783109849181e-06	8.98783109849181e-06\\
5.1	3.5	7.24432525706487e-06	7.24432525706487e-06\\
5.1	3.6	3.229798022996e-05	3.229798022996e-05\\
5.1	3.7	0.00243481689131783	0.00243481689131783\\
5.1	3.8	5.08824020026986e-06	5.08824020026986e-06\\
5.1	3.9	6.71370492929441e-06	6.71370492929441e-06\\
5.1	4	9.48686232920432e-06	9.48686232920432e-06\\
5.1	4.1	1.02271903869326e-05	1.02271903869326e-05\\
5.1	4.2	1.05446995327298e-05	1.05446995327298e-05\\
5.1	4.3	1.09878904068733e-05	1.09878904068733e-05\\
5.1	4.4	1.16699862445558e-05	1.16699862445558e-05\\
5.1	4.5	1.25425755490402e-05	1.25425755490402e-05\\
5.1	4.6	1.3458352368166e-05	1.3458352368166e-05\\
5.1	4.7	1.41947588336641e-05	1.41947588336641e-05\\
5.1	4.8	1.44987847344354e-05	1.44987847344354e-05\\
5.1	4.9	1.41653697265299e-05	1.41653697265299e-05\\
5.1	5	1.31258386861342e-05	1.31258386861342e-05\\
5.1	5.1	1.14959563150402e-05	1.14959563150402e-05\\
5.1	5.2	9.54128404960867e-06	9.54128404960867e-06\\
5.1	5.3	7.57106486713215e-06	7.57106486713215e-06\\
5.1	5.4	5.82653689298024e-06	5.82653689298024e-06\\
5.1	5.5	4.42720455567335e-06	4.42720455567335e-06\\
5.1	5.6	3.38514299823938e-06	3.38514299823938e-06\\
5.1	5.7	2.65261489273674e-06	2.65261489273674e-06\\
5.1	5.8	2.16599503244592e-06	2.16599503244592e-06\\
5.1	5.9	1.87121803453593e-06	1.87121803453593e-06\\
5.1	6	1.73403609259863e-06	1.73403609259863e-06\\
5.2	0	1.78002968707991e-05	1.78002968707991e-05\\
5.2	0.1	1.35149611278611e-05	1.35149611278611e-05\\
5.2	0.2	1.10264145979946e-05	1.10264145979946e-05\\
5.2	0.3	9.68338115942351e-06	9.68338115942351e-06\\
5.2	0.4	9.12025421138006e-06	9.12025421138006e-06\\
5.2	0.5	9.12524842889668e-06	9.12524842889668e-06\\
5.2	0.6	9.56202788104482e-06	9.56202788104482e-06\\
5.2	0.7	1.03274006663837e-05	1.03274006663837e-05\\
5.2	0.8	1.13402203660384e-05	1.13402203660384e-05\\
5.2	0.9	1.25574078990872e-05	1.25574078990872e-05\\
5.2	1	1.40038832916566e-05	1.40038832916566e-05\\
5.2	1.1	1.57951283210651e-05	1.57951283210651e-05\\
5.2	1.2	1.81296412964508e-05	1.81296412964508e-05\\
5.2	1.3	2.12172210477991e-05	2.12172210477991e-05\\
5.2	1.4	2.50523659369695e-05	2.50523659369695e-05\\
5.2	1.5	2.88444298865079e-05	2.88444298865079e-05\\
5.2	1.6	3.05391392670385e-05	3.05391392670385e-05\\
5.2	1.7	3.49315409947012e-05	3.49315409947012e-05\\
5.2	1.8	0.000118835888291312	0.000118835888291312\\
5.2	1.9	0.00054353824067882	0.00054353824067882\\
5.2	2	7.79895578823222e-05	7.79895578823222e-05\\
5.2	2.1	2.57993333885887e-05	2.57993333885887e-05\\
5.2	2.2	2.90687776215114e-05	2.90687776215114e-05\\
5.2	2.3	5.66825842659893e-05	5.66825842659893e-05\\
5.2	2.4	0.000195498928854585	0.000195498928854585\\
5.2	2.5	0.000534534290260602	0.000534534290260602\\
5.2	2.6	0.000112125555197396	0.000112125555197396\\
5.2	2.7	1.28704746803944e-05	1.28704746803944e-05\\
5.2	2.8	1.20930483220817e-05	1.20930483220817e-05\\
5.2	2.9	1.26591909371769e-05	1.26591909371769e-05\\
5.2	3	1.08752141280466e-05	1.08752141280466e-05\\
5.2	3.1	8.49958583118011e-06	8.49958583118011e-06\\
5.2	3.2	6.49214412974849e-06	6.49214412974849e-06\\
5.2	3.3	5.28782613638658e-06	5.28782613638658e-06\\
5.2	3.4	5.68040708713084e-06	5.68040708713084e-06\\
5.2	3.5	1.32582090077571e-05	1.32582090077571e-05\\
5.2	3.6	0.000220924735647516	0.000220924735647516\\
5.2	3.7	0.00217406828705344	0.00217406828705344\\
5.2	3.8	3.0848216378381e-05	3.0848216378381e-05\\
5.2	3.9	3.51622479780419e-06	3.51622479780419e-06\\
5.2	4	3.8303637423916e-06	3.8303637423916e-06\\
5.2	4.1	4.99722349164033e-06	4.99722349164033e-06\\
5.2	4.2	6.08104719041741e-06	6.08104719041741e-06\\
5.2	4.3	7.0994537384305e-06	7.0994537384305e-06\\
5.2	4.4	8.11218355366739e-06	8.11218355366739e-06\\
5.2	4.5	9.10662463565211e-06	9.10662463565211e-06\\
5.2	4.6	9.99988404514006e-06	9.99988404514006e-06\\
5.2	4.7	1.06643427882471e-05	1.06643427882471e-05\\
5.2	4.8	1.09639186861561e-05	1.09639186861561e-05\\
5.2	4.9	1.07972713407415e-05	1.07972713407415e-05\\
5.2	5	1.01389303808044e-05	1.01389303808044e-05\\
5.2	5.1	9.06163431699513e-06	9.06163431699513e-06\\
5.2	5.2	7.72303583666341e-06	7.72303583666341e-06\\
5.2	5.3	6.31661923415174e-06	6.31661923415174e-06\\
5.2	5.4	5.01106325517365e-06	5.01106325517365e-06\\
5.2	5.5	3.91031259161423e-06	3.91031259161423e-06\\
5.2	5.6	3.04907632345406e-06	3.04907632345406e-06\\
5.2	5.7	2.4138524344953e-06	2.4138524344953e-06\\
5.2	5.8	1.96984587878882e-06	1.96984587878882e-06\\
5.2	5.9	1.68084285597868e-06	1.68084285597868e-06\\
5.2	6	1.5197496944619e-06	1.5197496944619e-06\\
5.3	0	1.55406917577036e-05	1.55406917577036e-05\\
5.3	0.1	1.21706607121184e-05	1.21706607121184e-05\\
5.3	0.2	1.02297232153919e-05	1.02297232153919e-05\\
5.3	0.3	9.22700888771652e-06	9.22700888771652e-06\\
5.3	0.4	8.87910694816388e-06	8.87910694816388e-06\\
5.3	0.5	9.0144161001392e-06	9.0144161001392e-06\\
5.3	0.6	9.51740569142258e-06	9.51740569142258e-06\\
5.3	0.7	1.03021702054132e-05	1.03021702054132e-05\\
5.3	0.8	1.13110596099155e-05	1.13110596099155e-05\\
5.3	0.9	1.25315235819256e-05	1.25315235819256e-05\\
5.3	1	1.40174727867903e-05	1.40174727867903e-05\\
5.3	1.1	1.58973557714106e-05	1.58973557714106e-05\\
5.3	1.2	1.83467895194589e-05	1.83467895194589e-05\\
5.3	1.3	2.14840771877982e-05	2.14840771877982e-05\\
5.3	1.4	2.51343434873851e-05	2.51343434873851e-05\\
5.3	1.5	2.87704428602907e-05	2.87704428602907e-05\\
5.3	1.6	3.44085047475584e-05	3.44085047475584e-05\\
5.3	1.7	6.3000958636095e-05	6.3000958636095e-05\\
5.3	1.8	0.000168576699036036	0.000168576699036036\\
5.3	1.9	0.000421677544640112	0.000421677544640112\\
5.3	2	0.000215483327478711	0.000215483327478711\\
5.3	2.1	8.22068156881858e-05	8.22068156881858e-05\\
5.3	2.2	7.48113528532801e-05	7.48113528532801e-05\\
5.3	2.3	9.28502397942645e-05	9.28502397942645e-05\\
5.3	2.4	0.000181331225407179	0.000181331225407179\\
5.3	2.5	0.000206691420633321	0.000206691420633321\\
5.3	2.6	0.000103221428793908	0.000103221428793908\\
5.3	2.7	2.20726722552832e-05	2.20726722552832e-05\\
5.3	2.8	6.65412380357669e-06	6.65412380357669e-06\\
5.3	2.9	4.70346865354177e-06	4.70346865354177e-06\\
5.3	3	3.94847433349857e-06	3.94847433349857e-06\\
5.3	3.1	3.35584604351395e-06	3.35584604351395e-06\\
5.3	3.2	3.14460342947424e-06	3.14460342947424e-06\\
5.3	3.3	3.82193219780145e-06	3.82193219780145e-06\\
5.3	3.4	7.60631814088454e-06	7.60631814088454e-06\\
5.3	3.5	3.75145901006809e-05	3.75145901006809e-05\\
5.3	3.6	0.00057032433350779	0.00057032433350779\\
5.3	3.7	0.00179762384507076	0.00179762384507076\\
5.3	3.8	9.61232979244627e-05	9.61232979244627e-05\\
5.3	3.9	8.44256794284121e-06	8.44256794284121e-06\\
5.3	4	3.39037566121233e-06	3.39037566121233e-06\\
5.3	4.1	3.28559643393697e-06	3.28559643393697e-06\\
5.3	4.2	3.81916201692897e-06	3.81916201692897e-06\\
5.3	4.3	4.54175119155715e-06	4.54175119155715e-06\\
5.3	4.4	5.35528267748428e-06	5.35528267748428e-06\\
5.3	4.5	6.1869528000717e-06	6.1869528000717e-06\\
5.3	4.6	6.94689068993093e-06	6.94689068993093e-06\\
5.3	4.7	7.53495486557255e-06	7.53495486557255e-06\\
5.3	4.8	7.86060611105493e-06	7.86060611105493e-06\\
5.3	4.9	7.86331858376021e-06	7.86331858376021e-06\\
5.3	5	7.52852234558432e-06	7.52852234558432e-06\\
5.3	5.1	6.89557923744771e-06	6.89557923744771e-06\\
5.3	5.2	6.05255995258126e-06	6.05255995258126e-06\\
5.3	5.3	5.1150965622996e-06	5.1150965622996e-06\\
5.3	5.4	4.19564294592614e-06	4.19564294592614e-06\\
5.3	5.5	3.37710013630478e-06	3.37710013630478e-06\\
5.3	5.6	2.70215154137684e-06	2.70215154137684e-06\\
5.3	5.7	2.17889610120597e-06	2.17889610120597e-06\\
5.3	5.8	1.79472971774576e-06	1.79472971774576e-06\\
5.3	5.9	1.52985082796445e-06	1.52985082796445e-06\\
5.3	6	1.3663772606692e-06	1.3663772606692e-06\\
5.4	0	1.41868342739725e-05	1.41868342739725e-05\\
5.4	0.1	1.14576092261119e-05	1.14576092261119e-05\\
5.4	0.2	9.91345498800671e-06	9.91345498800671e-06\\
5.4	0.3	9.16950779692712e-06	9.16950779692712e-06\\
5.4	0.4	8.99841772424854e-06	8.99841772424854e-06\\
5.4	0.5	9.25932469123814e-06	9.25932469123814e-06\\
5.4	0.6	9.85837237707917e-06	9.85837237707917e-06\\
5.4	0.7	1.07338735165264e-05	1.07338735165264e-05\\
5.4	0.8	1.18617328908767e-05	1.18617328908767e-05\\
5.4	0.9	1.327345248834e-05	1.327345248834e-05\\
5.4	1	1.50745447468056e-05	1.50745447468056e-05\\
5.4	1.1	1.74472259634738e-05	1.74472259634738e-05\\
5.4	1.2	2.06103297723164e-05	2.06103297723164e-05\\
5.4	1.3	2.46962535006332e-05	2.46962535006332e-05\\
5.4	1.4	2.96577294551084e-05	2.96577294551084e-05\\
5.4	1.5	3.6373752958061e-05	3.6373752958061e-05\\
5.4	1.6	5.28290828676923e-05	5.28290828676923e-05\\
5.4	1.7	0.000106177303173577	0.000106177303173577\\
5.4	1.8	0.000216176543535607	0.000216176543535607\\
5.4	1.9	0.0003755445851796	0.0003755445851796\\
5.4	2	0.000333339616868302	0.000333339616868302\\
5.4	2.1	0.000234099973666521	0.000234099973666521\\
5.4	2.2	0.000201466383906628	0.000201466383906628\\
5.4	2.3	0.000207003911625545	0.000207003911625545\\
5.4	2.4	0.00016669019637057	0.00016669019637057\\
5.4	2.5	0.000105936382589916	0.000105936382589916\\
5.4	2.6	6.72228363833237e-05	6.72228363833237e-05\\
5.4	2.7	2.55164848689093e-05	2.55164848689093e-05\\
5.4	2.8	8.16737451398525e-06	8.16737451398525e-06\\
5.4	2.9	3.54782527179481e-06	3.54782527179481e-06\\
5.4	3	2.47444168009838e-06	2.47444168009838e-06\\
5.4	3.1	2.26097220702861e-06	2.26097220702861e-06\\
5.4	3.2	2.67782384616837e-06	2.67782384616837e-06\\
5.4	3.3	4.62389507747256e-06	4.62389507747256e-06\\
5.4	3.4	1.40295354474975e-05	1.40295354474975e-05\\
5.4	3.5	8.97129321900347e-05	8.97129321900347e-05\\
5.4	3.6	0.000814841819899416	0.000814841819899416\\
5.4	3.7	0.00143371310547431	0.00143371310547431\\
5.4	3.8	0.00018143787965935	0.00018143787965935\\
5.4	3.9	2.31530248273727e-05	2.31530248273727e-05\\
5.4	4	6.10737300545001e-06	6.10737300545001e-06\\
5.4	4.1	3.70737476800292e-06	3.70737476800292e-06\\
5.4	4.2	3.4188135332844e-06	3.4188135332844e-06\\
5.4	4.3	3.62891860829951e-06	3.62891860829951e-06\\
5.4	4.4	4.0339475462423e-06	4.0339475462423e-06\\
5.4	4.5	4.5233380259994e-06	4.5233380259994e-06\\
5.4	4.6	5.01069599690428e-06	5.01069599690428e-06\\
5.4	4.7	5.41416850181916e-06	5.41416850181916e-06\\
5.4	4.8	5.66512902394794e-06	5.66512902394794e-06\\
5.4	4.9	5.71852607682586e-06	5.71852607682586e-06\\
5.4	5	5.55871540911567e-06	5.55871540911567e-06\\
5.4	5.1	5.20108434818016e-06	5.20108434818016e-06\\
5.4	5.2	4.68941565115514e-06	4.68941565115514e-06\\
5.4	5.3	4.08731619960144e-06	4.08731619960144e-06\\
5.4	5.4	3.46391298173751e-06	3.46391298173751e-06\\
5.4	5.5	2.87839057558605e-06	2.87839057558605e-06\\
5.4	5.6	2.36981659916934e-06	2.36981659916934e-06\\
5.4	5.7	1.95558613945078e-06	1.95558613945078e-06\\
5.4	5.8	1.63671010254191e-06	1.63671010254191e-06\\
5.4	5.9	1.40554595919396e-06	1.40554595919396e-06\\
5.4	6	1.25249132987914e-06	1.25249132987914e-06\\
5.5	0	1.36255376897105e-05	1.36255376897105e-05\\
5.5	0.1	1.13503483146207e-05	1.13503483146207e-05\\
5.5	0.2	1.01055481335542e-05	1.01055481335542e-05\\
5.5	0.3	9.57886916779985e-06	9.57886916779985e-06\\
5.5	0.4	9.58436454267649e-06	9.58436454267649e-06\\
5.5	0.5	1.00087857776642e-05	1.00087857776642e-05\\
5.5	0.6	1.07845971790546e-05	1.07845971790546e-05\\
5.5	0.7	1.18848511032341e-05	1.18848511032341e-05\\
5.5	0.8	1.33355487443566e-05	1.33355487443566e-05\\
5.5	0.9	1.5237773973748e-05	1.5237773973748e-05\\
5.5	1	1.77868279382893e-05	1.77868279382893e-05\\
5.5	1.1	2.12649859465336e-05	2.12649859465336e-05\\
5.5	1.2	2.59657101377568e-05	2.59657101377568e-05\\
5.5	1.3	3.20667654764039e-05	3.20667654764039e-05\\
5.5	1.4	4.00357802077297e-05	4.00357802077297e-05\\
5.5	1.5	5.36721670860097e-05	5.36721670860097e-05\\
5.5	1.6	8.78885563531385e-05	8.78885563531385e-05\\
5.5	1.7	0.000165572676339676	0.000165572676339676\\
5.5	1.8	0.000274478935396697	0.000274478935396697\\
5.5	1.9	0.00037310527772026	0.00037310527772026\\
5.5	2	0.000380835855164529	0.000380835855164529\\
5.5	2.1	0.000379801702175736	0.000379801702175736\\
5.5	2.2	0.000378948638655585	0.000378948638655585\\
5.5	2.3	0.000322806331301268	0.000322806331301268\\
5.5	2.4	0.000161750656409543	0.000161750656409543\\
5.5	2.5	7.75832805240235e-05	7.75832805240235e-05\\
5.5	2.6	4.90712580836195e-05	4.90712580836195e-05\\
5.5	2.7	2.28972299549128e-05	2.28972299549128e-05\\
5.5	2.8	9.76183287799673e-06	9.76183287799673e-06\\
5.5	2.9	4.34698120513747e-06	4.34698120513747e-06\\
5.5	3	2.68803549077931e-06	2.68803549077931e-06\\
5.5	3.1	2.54095538712441e-06	2.54095538712441e-06\\
5.5	3.2	3.55757351216928e-06	3.55757351216928e-06\\
5.5	3.3	7.73210592236262e-06	7.73210592236262e-06\\
5.5	3.4	2.81837469443348e-05	2.81837469443348e-05\\
5.5	3.5	0.000160202623063142	0.000160202623063142\\
5.5	3.6	0.000897150386308355	0.000897150386308355\\
5.5	3.7	0.00116763651849828	0.00116763651849828\\
5.5	3.8	0.000263183720155181	0.000263183720155181\\
5.5	3.9	5.06723034087601e-05	5.06723034087601e-05\\
5.5	4	1.34923312359708e-05	1.34923312359708e-05\\
5.5	4.1	6.14501980574762e-06	6.14501980574762e-06\\
5.5	4.2	4.37295489112069e-06	4.37295489112069e-06\\
5.5	4.3	3.89712422223111e-06	3.89712422223111e-06\\
5.5	4.4	3.83698344970595e-06	3.83698344970595e-06\\
5.5	4.5	3.94704819408427e-06	3.94704819408427e-06\\
5.5	4.6	4.11826270213343e-06	4.11826270213343e-06\\
5.5	4.7	4.27790747305326e-06	4.27790747305326e-06\\
5.5	4.8	4.37270687631481e-06	4.37270687631481e-06\\
5.5	4.9	4.36808860726205e-06	4.36808860726205e-06\\
5.5	5	4.24815547055294e-06	4.24815547055294e-06\\
5.5	5.1	4.0145638977163e-06	4.0145638977163e-06\\
5.5	5.2	3.68486211838716e-06	3.68486211838716e-06\\
5.5	5.3	3.28946596219347e-06	3.28946596219347e-06\\
5.5	5.4	2.86598298419963e-06	2.86598298419963e-06\\
5.5	5.5	2.45141953116322e-06	2.45141953116322e-06\\
5.5	5.6	2.07501325755018e-06	2.07501325755018e-06\\
5.5	5.7	1.75456379189289e-06	1.75456379189289e-06\\
5.5	5.8	1.49704604727147e-06	1.49704604727147e-06\\
5.5	5.9	1.3020477003903e-06	1.3020477003903e-06\\
5.5	6	1.16589897461261e-06	1.16589897461261e-06\\
5.6	0	1.38211879880012e-05	1.38211879880012e-05\\
5.6	0.1	1.18772256763038e-05	1.18772256763038e-05\\
5.6	0.2	1.08791649945868e-05	1.08791649945868e-05\\
5.6	0.3	1.05668617618083e-05	1.05668617618083e-05\\
5.6	0.4	1.07885088000004e-05	1.07885088000004e-05\\
5.6	0.5	1.14604614657823e-05	1.14604614657823e-05\\
5.6	0.6	1.2551567067616e-05	1.2551567067616e-05\\
5.6	0.7	1.40885138142281e-05	1.40885138142281e-05\\
5.6	0.8	1.61773553953856e-05	1.61773553953856e-05\\
5.6	0.9	1.90308795125431e-05	1.90308795125431e-05\\
5.6	1	2.29796364773092e-05	2.29796364773092e-05\\
5.6	1.1	2.84231558744034e-05	2.84231558744034e-05\\
5.6	1.2	3.57001850307803e-05	3.57001850307803e-05\\
5.6	1.3	4.52061807545342e-05	4.52061807545342e-05\\
5.6	1.4	5.91173937041344e-05	5.91173937041344e-05\\
5.6	1.5	8.65644470929525e-05	8.65644470929525e-05\\
5.6	1.6	0.000148577620630337	0.000148577620630337\\
5.6	1.7	0.000248015941043299	0.000248015941043299\\
5.6	1.8	0.000342407915165649	0.000342407915165649\\
5.6	1.9	0.000399317495681736	0.000399317495681736\\
5.6	2	0.000415379696882499	0.000415379696882499\\
5.6	2.1	0.000464083201890854	0.000464083201890854\\
5.6	2.2	0.000481043795574885	0.000481043795574885\\
5.6	2.3	0.000341662802688601	0.000341662802688601\\
5.6	2.4	0.00015467588716599	0.00015467588716599\\
5.6	2.5	7.42879065620496e-05	7.42879065620496e-05\\
5.6	2.6	4.66390233946945e-05	4.66390233946945e-05\\
5.6	2.7	2.38942945211739e-05	2.38942945211739e-05\\
5.6	2.8	1.15622864237528e-05	1.15622864237528e-05\\
5.6	2.9	6.46605018156144e-06	6.46605018156144e-06\\
5.6	3	4.48457993659102e-06	4.48457993659102e-06\\
5.6	3.1	4.49109736835699e-06	4.49109736835699e-06\\
5.6	3.2	6.88741968254657e-06	6.88741968254657e-06\\
5.6	3.3	1.61900171626e-05	1.61900171626e-05\\
5.6	3.4	5.59468238091628e-05	5.59468238091628e-05\\
5.6	3.5	0.000253350536929679	0.000253350536929679\\
5.6	3.6	0.000943879411973313	0.000943879411973313\\
5.6	3.7	0.00103419731510001	0.00103419731510001\\
5.6	3.8	0.000343139332652347	0.000343139332652347\\
5.6	3.9	9.37043407297335e-05	9.37043407297335e-05\\
5.6	4	2.88498096343358e-05	2.88498096343358e-05\\
5.6	4.1	1.19266230147151e-05	1.19266230147151e-05\\
5.6	4.2	7.0749779502712e-06	7.0749779502712e-06\\
5.6	4.3	5.35497403901839e-06	5.35497403901839e-06\\
5.6	4.4	4.61174286603796e-06	4.61174286603796e-06\\
5.6	4.5	4.24818774414481e-06	4.24818774414481e-06\\
5.6	4.6	4.05226630733684e-06	4.05226630733684e-06\\
5.6	4.7	3.92449458323678e-06	3.92449458323678e-06\\
5.6	4.8	3.80886704931087e-06	3.80886704931087e-06\\
5.6	4.9	3.67277780835285e-06	3.67277780835285e-06\\
5.6	5	3.4989023874455e-06	3.4989023874455e-06\\
5.6	5.1	3.28083253461219e-06	3.28083253461219e-06\\
5.6	5.2	3.02093613224503e-06	3.02093613224503e-06\\
5.6	5.3	2.72927286424577e-06	2.72927286424577e-06\\
5.6	5.4	2.42194016406603e-06	2.42194016406603e-06\\
5.6	5.5	2.11791664378402e-06	2.11791664378402e-06\\
5.6	5.6	1.83501137506173e-06	1.83501137506173e-06\\
5.6	5.7	1.58655166811965e-06	1.58655166811965e-06\\
5.6	5.8	1.38009739878836e-06	1.38009739878836e-06\\
5.6	5.9	1.21825046239965e-06	1.21825046239965e-06\\
5.6	6	1.10067746185233e-06	1.10067746185233e-06\\
5.7	0	1.48181615274994e-05	1.48181615274994e-05\\
5.7	0.1	1.31346885763643e-05	1.31346885763643e-05\\
5.7	0.2	1.23734977636843e-05	1.23734977636843e-05\\
5.7	0.3	1.23151862535473e-05	1.23151862535473e-05\\
5.7	0.4	1.28417371700174e-05	1.28417371700174e-05\\
5.7	0.5	1.39088609596603e-05	1.39088609596603e-05\\
5.7	0.6	1.55431070222068e-05	1.55431070222068e-05\\
5.7	0.7	1.78601303325663e-05	1.78601303325663e-05\\
5.7	0.8	2.10956118125069e-05	2.10956118125069e-05\\
5.7	0.9	2.56279613059113e-05	2.56279613059113e-05\\
5.7	1	3.19507352950536e-05	3.19507352950536e-05\\
5.7	1.1	4.05607263277694e-05	4.05607263277694e-05\\
5.7	1.2	5.19624626315551e-05	5.19624626315551e-05\\
5.7	1.3	6.77308176617996e-05	6.77308176617996e-05\\
5.7	1.4	9.41549945130518e-05	9.41549945130518e-05\\
5.7	1.5	0.000147968702933038	0.000147968702933038\\
5.7	1.6	0.000248151128860843	0.000248151128860843\\
5.7	1.7	0.00035575907408964	0.00035575907408964\\
5.7	1.8	0.000424267760234524	0.000424267760234524\\
5.7	1.9	0.000468297113032421	0.000468297113032421\\
5.7	2	0.000506266490374803	0.000506266490374803\\
5.7	2.1	0.000583519077535071	0.000583519077535071\\
5.7	2.2	0.000573427464171084	0.000573427464171084\\
5.7	2.3	0.000358940560105347	0.000358940560105347\\
5.7	2.4	0.000167820088958367	0.000167820088958367\\
5.7	2.5	8.8836419642161e-05	8.8836419642161e-05\\
5.7	2.6	5.881496280037e-05	5.881496280037e-05\\
5.7	2.7	3.39196188429104e-05	3.39196188429104e-05\\
5.7	2.8	1.77208404971538e-05	1.77208404971538e-05\\
5.7	2.9	1.171349017821e-05	1.171349017821e-05\\
5.7	3	1.02445554186201e-05	1.02445554186201e-05\\
5.7	3.1	1.17887176444639e-05	1.17887176444639e-05\\
5.7	3.2	1.85267769171178e-05	1.85267769171178e-05\\
5.7	3.3	4.02397851640336e-05	4.02397851640336e-05\\
5.7	3.4	0.000118256286215572	0.000118256286215572\\
5.7	3.5	0.000417994830285853	0.000417994830285853\\
5.7	3.6	0.00108644749952442	0.00108644749952442\\
5.7	3.7	0.00105306210878491	0.00105306210878491\\
5.7	3.8	0.000447907330729412	0.000447907330729412\\
5.7	3.9	0.000158953370718126	0.000158953370718126\\
5.7	4	5.76962666248393e-05	5.76962666248393e-05\\
5.7	4.1	2.3970239427521e-05	2.3970239427521e-05\\
5.7	4.2	1.28467451436135e-05	1.28467451436135e-05\\
5.7	4.3	8.62911123039118e-06	8.62911123039118e-06\\
5.7	4.4	6.64351359074153e-06	6.64351359074153e-06\\
5.7	4.5	5.51599302059847e-06	5.51599302059847e-06\\
5.7	4.6	4.78500277830949e-06	4.78500277830949e-06\\
5.7	4.7	4.26161335490888e-06	4.26161335490888e-06\\
5.7	4.8	3.85410722849223e-06	3.85410722849223e-06\\
5.7	4.9	3.51306053744504e-06	3.51306053744504e-06\\
5.7	5	3.21003399194836e-06	3.21003399194836e-06\\
5.7	5.1	2.92785864667972e-06	2.92785864667972e-06\\
5.7	5.2	2.65643549560394e-06	2.65643549560394e-06\\
5.7	5.3	2.39138242447142e-06	2.39138242447142e-06\\
5.7	5.4	2.13351987821666e-06	2.13351987821666e-06\\
5.7	5.5	1.88780104159151e-06	1.88780104159151e-06\\
5.7	5.6	1.6613117146927e-06	1.6613117146927e-06\\
5.7	5.7	1.46094548326791e-06	1.46094548326791e-06\\
5.7	5.8	1.29170799567449e-06	1.29170799567449e-06\\
5.7	5.9	1.15620950165465e-06	1.15620950165465e-06\\
5.7	6	1.05524144042917e-06	1.05524144042917e-06\\
5.8	0	1.67549545221339e-05	1.67549545221339e-05\\
5.8	0.1	1.53104811456281e-05	1.53104811456281e-05\\
5.8	0.2	1.4824894610979e-05	1.4824894610979e-05\\
5.8	0.3	1.51170749563767e-05	1.51170749563767e-05\\
5.8	0.4	1.61126117173015e-05	1.61126117173015e-05\\
5.8	0.5	1.78306023930353e-05	1.78306023930353e-05\\
5.8	0.6	2.03965738788037e-05	2.03965738788037e-05\\
5.8	0.7	2.40742089172339e-05	2.40742089172339e-05\\
5.8	0.8	2.92969742334418e-05	2.92969742334418e-05\\
5.8	0.9	3.66613272167816e-05	3.66613272167816e-05\\
5.8	1	4.68458012051252e-05	4.68458012051252e-05\\
5.8	1.1	6.0586310227483e-05	6.0586310227483e-05\\
5.8	1.2	7.93841841982146e-05	7.93841841982146e-05\\
5.8	1.3	0.000108199069416562	0.000108199069416562\\
5.8	1.4	0.000160559011685875	0.000160559011685875\\
5.8	1.5	0.000259762358186487	0.000259762358186487\\
5.8	1.6	0.000400430876370853	0.000400430876370853\\
5.8	1.7	0.000496522946132988	0.000496522946132988\\
5.8	1.8	0.00054780095359103	0.00054780095359103\\
5.8	1.9	0.000619787349338553	0.000619787349338553\\
5.8	2	0.000719702748583318	0.000719702748583318\\
5.8	2.1	0.000851108831960191	0.000851108831960191\\
5.8	2.2	0.000795359942051866	0.000795359942051866\\
5.8	2.3	0.00047406098547875	0.00047406098547875\\
5.8	2.4	0.000233551841099119	0.000233551841099119\\
5.8	2.5	0.000133639474939785	0.000133639474939785\\
5.8	2.6	9.55695399759883e-05	9.55695399759883e-05\\
5.8	2.7	6.50643334475332e-05	6.50643334475332e-05\\
5.8	2.8	3.83421695452086e-05	3.83421695452086e-05\\
5.8	2.9	2.77616854045642e-05	2.77616854045642e-05\\
5.8	3	2.89843324284826e-05	2.89843324284826e-05\\
5.8	3.1	3.87819274800715e-05	3.87819274800715e-05\\
5.8	3.2	6.08684361895665e-05	6.08684361895665e-05\\
5.8	3.3	0.000116623672241388	0.000116623672241388\\
5.8	3.4	0.000287449898865418	0.000287449898865418\\
5.8	3.5	0.000781362391220642	0.000781362391220642\\
5.8	3.6	0.00147117023315866	0.00147117023315866\\
5.8	3.7	0.00127893908775837	0.00127893908775837\\
5.8	3.8	0.000628179822754188	0.000628179822754188\\
5.8	3.9	0.000265156268799617	0.000265156268799617\\
5.8	4	0.000110235679015549	0.000110235679015549\\
5.8	4.1	4.79812059114286e-05	4.79812059114286e-05\\
5.8	4.2	2.44956700394873e-05	2.44956700394873e-05\\
5.8	4.3	1.51662821790507e-05	1.51662821790507e-05\\
5.8	4.4	1.0761869809395e-05	1.0761869809395e-05\\
5.8	4.5	8.24340285609578e-06	8.24340285609578e-06\\
5.8	4.6	6.59291218953837e-06	6.59291218953837e-06\\
5.8	4.7	5.42070615194169e-06	5.42070615194169e-06\\
5.8	4.8	4.54736709961211e-06	4.54736709961211e-06\\
5.8	4.9	3.87575716272317e-06	3.87575716272317e-06\\
5.8	5	3.34628391456017e-06	3.34628391456017e-06\\
5.8	5.1	2.91865192192804e-06	2.91865192192804e-06\\
5.8	5.2	2.56401566987124e-06	2.56401566987124e-06\\
5.8	5.3	2.26181100347535e-06	2.26181100347535e-06\\
5.8	5.4	1.99845273625331e-06	1.99845273625331e-06\\
5.8	5.5	1.76632268229266e-06	1.76632268229266e-06\\
5.8	5.6	1.56234850260746e-06	1.56234850260746e-06\\
5.8	5.7	1.38620200655788e-06	1.38620200655788e-06\\
5.8	5.8	1.23859243226268e-06	1.23859243226268e-06\\
5.8	5.9	1.12015260907725e-06	1.12015260907725e-06\\
5.8	6	1.03111321027244e-06	1.03111321027244e-06\\
5.9	0	1.98927080710332e-05	1.98927080710332e-05\\
5.9	0.1	1.87233969086705e-05	1.87233969086705e-05\\
5.9	0.2	1.86207488170784e-05	1.86207488170784e-05\\
5.9	0.3	1.94522158063413e-05	1.94522158063413e-05\\
5.9	0.4	2.12142843544511e-05	2.12142843544511e-05\\
5.9	0.5	2.40374670418944e-05	2.40374670418944e-05\\
5.9	0.6	2.82168142681697e-05	2.82168142681697e-05\\
5.9	0.7	3.42515094305902e-05	3.42515094305902e-05\\
5.9	0.8	4.28606414452545e-05	4.28606414452545e-05\\
5.9	0.9	5.49431793751801e-05	5.49431793751801e-05\\
5.9	1	7.15719018164662e-05	7.15719018164662e-05\\
5.9	1.1	9.45222028774067e-05	9.45222028774067e-05\\
5.9	1.2	0.000128370043407425	0.000128370043407425\\
5.9	1.3	0.00018492785779584	0.00018492785779584\\
5.9	1.4	0.000287575239615163	0.000287575239615163\\
5.9	1.5	0.000455035313343644	0.000455035313343644\\
5.9	1.6	0.00062693207245014	0.00062693207245014\\
5.9	1.7	0.000704072581951115	0.000704072581951115\\
5.9	1.8	0.000772761344900406	0.000772761344900406\\
5.9	1.9	0.000931347334506591	0.000931347334506591\\
5.9	2	0.00116945914918697	0.00116945914918697\\
5.9	2.1	0.00144186904766571	0.00144186904766571\\
5.9	2.2	0.00134922499068301	0.00134922499068301\\
5.9	2.3	0.000819780082158073	0.000819780082158073\\
5.9	2.4	0.000426001459021043	0.000426001459021043\\
5.9	2.5	0.000255070311192291	0.000255070311192291\\
5.9	2.6	0.000194963308840087	0.000194963308840087\\
5.9	2.7	0.000157665598818898	0.000157665598818898\\
5.9	2.8	0.00011012021567227	0.00011012021567227\\
5.9	2.9	8.55531361918843e-05	8.55531361918843e-05\\
5.9	3	9.5936720664809e-05	9.5936720664809e-05\\
5.9	3.1	0.00013891745336641	0.00013891745336641\\
5.9	3.2	0.000218112253742881	0.000218112253742881\\
5.9	3.3	0.000383202855603013	0.000383202855603013\\
5.9	3.4	0.00080386856336338	0.00080386856336338\\
5.9	3.5	0.00168280527203342	0.00168280527203342\\
5.9	3.6	0.00239886805643195	0.00239886805643195\\
5.9	3.7	0.00187778066152187	0.00187778066152187\\
5.9	3.8	0.000985451097065787	0.000985451097065787\\
5.9	3.9	0.000458000703206138	0.000458000703206138\\
5.9	4	0.000207842322723072	0.000207842322723072\\
5.9	4.1	9.5166506238193e-05	9.5166506238193e-05\\
5.9	4.2	4.77377739135105e-05	4.77377739135105e-05\\
5.9	4.3	2.79157522830671e-05	2.79157522830671e-05\\
5.9	4.4	1.8654754696686e-05	1.8654754696686e-05\\
5.9	4.5	1.34996844884463e-05	1.34996844884463e-05\\
5.9	4.6	1.01760700633559e-05	1.01760700633559e-05\\
5.9	4.7	7.84875553298057e-06	7.84875553298057e-06\\
5.9	4.8	6.15804408254623e-06	6.15804408254623e-06\\
5.9	4.9	4.91084720173646e-06	4.91084720173646e-06\\
5.9	5	3.98322984051211e-06	3.98322984051211e-06\\
5.9	5.1	3.2873196760271e-06	3.2873196760271e-06\\
5.9	5.2	2.7584101518823e-06	2.7584101518823e-06\\
5.9	5.3	2.34892460321883e-06	2.34892460321883e-06\\
5.9	5.4	2.02483309260489e-06	2.02483309260489e-06\\
5.9	5.5	1.7629175089439e-06	1.7629175089439e-06\\
5.9	5.6	1.54828883970128e-06	1.54828883970128e-06\\
5.9	5.7	1.37201793804533e-06	1.37201793804533e-06\\
5.9	5.8	1.22901061099935e-06	1.22901061099935e-06\\
5.9	5.9	1.11636610387122e-06	1.11636610387122e-06\\
5.9	6	1.03238364617628e-06	1.03238364617628e-06\\
6	0	2.46696271177532e-05	2.46696271177532e-05\\
6	0.1	2.38967507633873e-05	2.38967507633873e-05\\
6	0.2	2.43999952121891e-05	2.43999952121891e-05\\
6	0.3	2.61267600503455e-05	2.61267600503455e-05\\
6	0.4	2.92005415356122e-05	2.92005415356122e-05\\
6	0.5	3.39478998323059e-05	3.39478998323059e-05\\
6	0.6	4.09456251908215e-05	4.09456251908215e-05\\
6	0.7	5.10617278703301e-05	5.10617278703301e-05\\
6	0.8	6.5465326101821e-05	6.5465326101821e-05\\
6	0.9	8.56732877132996e-05	8.56732877132996e-05\\
6	1	0.000114018928834348	0.000114018928834348\\
6	1.1	0.000155410302070879	0.000155410302070879\\
6	1.2	0.000221290118853729	0.000221290118853729\\
6	1.3	0.000335039141095851	0.000335039141095851\\
6	1.4	0.000528174777085321	0.000528174777085321\\
6	1.5	0.000787146388972944	0.000787146388972944\\
6	1.6	0.000981072260046972	0.000981072260046972\\
6	1.7	0.00106123465652138	0.00106123465652138\\
6	1.8	0.00121418671240749	0.00121418671240749\\
6	1.9	0.00156303815124349	0.00156303815124349\\
6	2	0.00210382973294183	0.00210382973294183\\
6	2.1	0.00275205956637007	0.00275205956637007\\
6	2.2	0.00271597455812838	0.00271597455812838\\
6	2.3	0.00177480865757589	0.00177480865757589\\
6	2.4	0.000987090394110506	0.000987090394110506\\
6	2.5	0.000613674384707944	0.000613674384707944\\
6	2.6	0.000493437648374816	0.000493437648374816\\
6	2.7	0.000459106409394325	0.000459106409394325\\
6	2.8	0.000383015257920058	0.000383015257920058\\
6	2.9	0.000322718942297092	0.000322718942297092\\
6	3	0.000362945288783449	0.000362945288783449\\
6	3.1	0.000524650298895291	0.000524650298895291\\
6	3.2	0.000818465999646245	0.000818465999646245\\
6	3.3	0.00136420736179514	0.00136420736179514\\
6	3.4	0.00247379323398408	0.00247379323398408\\
6	3.5	0.00409162118643632	0.00409162118643632\\
6	3.6	0.00465317907593288	0.00465317907593288\\
6	3.7	0.00330504091179243	0.00330504091179243\\
6	3.8	0.00175692859115895	0.00175692859115895\\
6	3.9	0.000846186944312679	0.000846186944312679\\
6	4	0.000398746506245843	0.000398746506245843\\
6	4.1	0.000188557991359227	0.000188557991359227\\
6	4.2	9.41589534836923e-05	9.41589534836923e-05\\
6	4.3	5.28434556069971e-05	5.28434556069971e-05\\
6	4.4	3.36337702453226e-05	3.36337702453226e-05\\
6	4.5	2.33377871062159e-05	2.33377871062159e-05\\
6	4.6	1.6910510337744e-05	1.6910510337744e-05\\
6	4.7	1.2484136348365e-05	1.2484136348365e-05\\
6	4.8	9.31152972727983e-06	9.31152972727983e-06\\
6	4.9	7.01955011246244e-06	7.01955011246244e-06\\
6	5	5.36845183386623e-06	5.36845183386623e-06\\
6	5.1	4.18202885993035e-06	4.18202885993035e-06\\
6	5.2	3.3272869895448e-06	3.3272869895448e-06\\
6	5.3	2.70579093433951e-06	2.70579093433951e-06\\
6	5.4	2.246977884656e-06	2.246977884656e-06\\
6	5.5	1.90192790479366e-06	1.90192790479366e-06\\
6	5.6	1.63782022684024e-06	1.63782022684024e-06\\
6	5.7	1.43331465812561e-06	1.43331465812561e-06\\
6	5.8	1.27490059600554e-06	1.27490059600554e-06\\
6	5.9	1.15417588069116e-06	1.15417588069116e-06\\
6	6	1.0659957296965e-06	1.0659957296965e-06\\
};
\end{axis}
\end{tikzpicture}%
		}
	\caption[Replication Results with Equal Initialisation]{Replication Results with Equal Initialisation: \itshape For this trial, the initialisation of $\psips$ does not divide the room in any way, but rather initialises $\psip$ equally for every grid point.}
	\label{fig:resultsReplicationAlternativeEqual}
\end{figure}

First, the results show that the localisation performance was not altered by using these alternative initialisations. In both trials, the sources were identified correctly and the \gls{mae} is zero. Even the pattern of the spurious Gaussian component weights is identical to the original replication trial. Therefore it can be concluded that the different initialisations of $\psips$ used above does not impact the results of the source localisation algorithm. Further, the identical performance exhibited when equally initialising all $\psips$, effectively removing the $S$-dimension like proposed at the end of \autoref{sec:algLocEst}, allows for the optimisation of the algorithms complexity and memory requirements. This, in turn, permits trials with larger $S$, as the memory requirement does no longer increase with additional sources. Therefore, the upcoming trials can be performed for $2-7$~sources, which allows for the evaluation of the localisation performance in a less sparse environment. \FloatBarrier