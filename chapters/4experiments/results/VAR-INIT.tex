\subsubsection*{Initial Variance}

% BOXPLOT
\begin{figure}[H]
\iftoggle{quick}{
    \includegraphics[width=\textwidth]{plots/boxplots/boxplot-joined-var-val-est}
}{%
    	\begin{tikzpicture}
	    % This file was created by matplotlib2tikz v0.6.14.
\definecolor{color0}{rgb}{0.8,0.207843137254902,0.219607843137255}
\definecolor{color1}{rgb}{1,0.647058823529412,0}
\definecolor{color2}{rgb}{0.0235294117647059,0.603921568627451,0.952941176470588}
\definecolor{color3}{rgb}{0.219607843137255,0.00784313725490196,0.509803921568627}
\definecolor{color4}{rgb}{0.76078431372549,0,0.470588235294118}

\begin{axis}[
xlabel={$S$},
ylabel={MAE},
xmin=0.5, xmax=6.5,
ymin=0, ymax=2.5,
width=\figurewidth,
height=\figureheight,
xtick={1,2,3,4,5,6},
xticklabels={2,3,4,5,6,7},
ytick={0,0.5,1,1.5,2,2.5},
minor xtick={},
minor ytick={},
tick align=outside,
tick pos=left,
x grid style={white!69.019607843137251!black},
ymajorgrids,
y grid style={white!69.019607843137251!black}
]
\addplot [line width=1.0pt, black, opacity=1, forget plot]
table {%
0.61 0
0.69 0
0.69 0.0707106781186547
0.61 0.0707106781186547
0.61 0
};
\addplot [line width=1.0pt, black, opacity=1, forget plot]
table {%
0.65 0
0.65 0
};
\addplot [line width=1.0pt, black, opacity=1, forget plot]
table {%
0.65 0.0707106781186547
0.65 0.161803398874989
};
\addplot [line width=1.0pt, black, forget plot]
table {%
0.63 0
0.67 0
};
\addplot [line width=1.0pt, black, forget plot]
table {%
0.63 0.161803398874989
0.67 0.161803398874989
};
\addplot [line width=0.5pt, black, opacity=0.2, mark=*, mark size=1, mark options={solid}, only marks, forget plot]
table {%
0.65 1.35
0.65 0.1802775637732
0.65 0.431959610746632
0.65 0.208113883008419
0.65 0.522015325445528
0.65 1.55
0.65 0.19142135623731
0.65 1.37134601395307
0.65 0.584161925296378
0.65 0.565685424949238
0.65 0.1802775637732
0.65 0.20811388300842
0.65 0.228824561127073
0.65 0.212132034355965
0.65 0.273606797749979
0.65 0.223606797749979
0.65 0.2302775637732
0.65 0.273606797749979
0.65 0.1802775637732
0.65 0.212132034355964
0.65 1.17750785917759
0.65 1.30862523283024
0.65 0.206155281280883
0.65 0.25
};
\addplot [line width=1.0pt, black, opacity=1, forget plot]
table {%
1.61 0
1.69 0
1.69 0.141421356237309
1.61 0.141421356237309
1.61 0
};
\addplot [line width=1.0pt, black, opacity=1, forget plot]
table {%
1.65 0
1.65 0
};
\addplot [line width=1.0pt, black, opacity=1, forget plot]
table {%
1.65 0.141421356237309
1.65 0.3
};
\addplot [line width=1.0pt, black, forget plot]
table {%
1.63 0
1.67 0
};
\addplot [line width=1.0pt, black, forget plot]
table {%
1.63 0.3
1.67 0.3
};
\addplot [line width=0.5pt, black, opacity=0.2, mark=*, mark size=1, mark options={solid}, only marks, forget plot]
table {%
1.65 0.986154146165801
1.65 0.480116263352131
1.65 0.380788655293195
1.65 0.570790047568533
1.65 0.996242942258564
1.65 0.430116263352131
1.65 0.361803398874989
1.65 0.643397840021018
1.65 1.47054411698527
1.65 2.00431079251519
1.65 1.29576099429883
1.65 0.471825158515465
1.65 0.602079728939615
1.65 0.739294558148147
1.65 1.14127122105133
1.65 0.51478150704935
1.65 0.955538513813742
1.65 1.20108644332213
1.65 1.30450070628092
1.65 0.403884361523179
};
\addplot [line width=1.0pt, black, opacity=1, forget plot]
table {%
2.61 0
2.69 0
2.69 0.215000725204468
2.61 0.215000725204468
2.61 0
};
\addplot [line width=1.0pt, black, opacity=1, forget plot]
table {%
2.65 0
2.65 0
};
\addplot [line width=1.0pt, black, opacity=1, forget plot]
table {%
2.65 0.215000725204468
2.65 0.52169905660283
};
\addplot [line width=1.0pt, black, forget plot]
table {%
2.63 0
2.67 0
};
\addplot [line width=1.0pt, black, forget plot]
table {%
2.63 0.52169905660283
2.67 0.52169905660283
};
\addplot [line width=0.5pt, black, opacity=0.2, mark=*, mark size=1, mark options={solid}, only marks, forget plot]
table {%
2.65 2.00826362750745
2.65 0.69462219947249
2.65 1.17161074304224
2.65 0.697213595499958
2.65 1.2354136515872
2.65 2.20558160450335
2.65 1.04403065089106
2.65 0.856225774829856
2.65 0.786601731282472
2.65 0.790569415042095
2.65 1.34164078649987
2.65 1.47872346215153
2.65 1.03523499553598
2.65 0.910232526704263
2.65 0.85
2.65 1.7585505395069
2.65 0.781024967590666
2.65 2.22036033111745
2.65 0.890224690738243
2.65 1.48178210632764
2.65 0.793303437365925
2.65 1.07591422643416
2.65 0.919238815542512
2.65 1.31596285675422
2.65 0.931864372217826
2.65 1.2747548783982
2.65 0.707647321898295
2.65 2.05985009635531
};
\addplot [line width=1.0pt, black, opacity=1, forget plot]
table {%
3.61 0
3.69 0
3.69 0.273606797749979
3.61 0.273606797749979
3.61 0
};
\addplot [line width=1.0pt, black, opacity=1, forget plot]
table {%
3.65 0
3.65 0
};
\addplot [line width=1.0pt, black, opacity=1, forget plot]
table {%
3.65 0.273606797749979
3.65 0.672681202353686
};
\addplot [line width=1.0pt, black, forget plot]
table {%
3.63 0
3.67 0
};
\addplot [line width=1.0pt, black, forget plot]
table {%
3.63 0.672681202353686
3.67 0.672681202353686
};
\addplot [line width=0.5pt, black, opacity=0.2, mark=*, mark size=1, mark options={solid}, only marks, forget plot]
table {%
3.65 0.850105002057062
3.65 1.31244047484067
3.65 1.85
3.65 1.26491106406735
3.65 1.7939932825281
3.65 0.96422879709595
3.65 0.882061006154109
3.65 0.70178344238091
3.65 2.29042333582833
3.65 1.14017542509914
3.65 2.08626460450251
3.65 0.807774721070176
3.65 1.07227203448992
3.65 0.701920240520265
3.65 1.39950786911779
3.65 3.15559217664563
3.65 2.02418645523178
3.65 0.707106781186548
3.65 1.45773797371132
3.65 1.92375140077057
3.65 0.69489858835952
3.65 1.19485503248612
3.65 1.20933866224478
3.65 1.57499590220956
3.65 0.743303437365925
3.65 1.60080624192708
};
\addplot [line width=1.0pt, black, opacity=1, forget plot]
table {%
4.61 0.0499999999999999
4.69 0.0499999999999999
4.69 0.287132034355964
4.61 0.287132034355964
4.61 0.0499999999999999
};
\addplot [line width=1.0pt, black, opacity=1, forget plot]
table {%
4.65 0.0499999999999999
4.65 0
};
\addplot [line width=1.0pt, black, opacity=1, forget plot]
table {%
4.65 0.287132034355964
4.65 0.640312423743285
};
\addplot [line width=1.0pt, black, forget plot]
table {%
4.63 0
4.67 0
};
\addplot [line width=1.0pt, black, forget plot]
table {%
4.63 0.640312423743285
4.67 0.640312423743285
};
\addplot [line width=0.5pt, black, opacity=0.2, mark=*, mark size=1, mark options={solid}, only marks, forget plot]
table {%
4.65 1.54446692784152
4.65 0.667083203206317
4.65 1.09658560997307
4.65 1.51269597148718
4.65 0.650556773098224
4.65 1.20974135047432
4.65 0.742442890089805
4.65 1.20433963806152
4.65 0.768441663795104
4.65 0.95524865872714
4.65 1.12171718896571
4.65 0.710633520177595
4.65 2.50370804117493
4.65 0.902372813917084
4.65 0.85156097709407
4.65 1.04624294225856
4.65 0.912598297577935
4.65 0.71879404448665
4.65 1.17686827602846
4.65 1.71952980385224
4.65 0.8
4.65 1.07147511812422
4.65 0.756637297521078
4.65 1.67600849701358
4.65 0.919238815542512
4.65 0.95524865872714
4.65 0.886002257333468
4.65 0.806225774829855
};
\addplot [line width=1.0pt, black, opacity=1, forget plot]
table {%
5.61 0.05
5.69 0.05
5.69 0.320156211871643
5.61 0.320156211871643
5.61 0.05
};
\addplot [line width=1.0pt, black, opacity=1, forget plot]
table {%
5.65 0.05
5.65 0
};
\addplot [line width=1.0pt, black, opacity=1, forget plot]
table {%
5.65 0.320156211871643
5.65 0.715891053163818
};
\addplot [line width=1.0pt, black, forget plot]
table {%
5.63 0
5.67 0
};
\addplot [line width=1.0pt, black, forget plot]
table {%
5.63 0.715891053163818
5.67 0.715891053163818
};
\addplot [line width=0.5pt, black, opacity=0.2, mark=*, mark size=1, mark options={solid}, only marks, forget plot]
table {%
5.65 1.05118980208143
5.65 1.03077640640441
5.65 1.29065885535211
5.65 1.39608369646099
5.65 0.903680323186751
5.65 0.940224690738242
5.65 1.4484805114987
5.65 1.60854025206363
5.65 0.842072641592628
5.65 1.86018016555873
5.65 1.05312490532866
5.65 1.35399362039845
5.65 0.857233019691503
5.65 1.5305563481645
5.65 1.10804634113355
5.65 1.25
5.65 0.823606797749979
5.65 1.01242283656583
5.65 0.751664818918645
};
\addplot [line width=1.0pt, color0, opacity=1, forget plot]
table {%
0.75 0
0.83 0
0.83 0.111803398874989
0.75 0.111803398874989
0.75 0
};
\addplot [line width=1.0pt, color0, opacity=1, forget plot]
table {%
0.79 0
0.79 0
};
\addplot [line width=1.0pt, color0, opacity=1, forget plot]
table {%
0.79 0.111803398874989
0.79 0.276865959399538
};
\addplot [line width=1.0pt, color0, forget plot]
table {%
0.77 0
0.81 0
};
\addplot [line width=1.0pt, color0, forget plot]
table {%
0.77 0.276865959399538
0.81 0.276865959399538
};
\addplot [line width=0.5pt, color0, opacity=0.2, mark=*, mark size=1, mark options={solid}, only marks, forget plot]
table {%
0.79 0.320156211871642
0.79 0.321698920010509
0.79 0.320156211871642
0.79 0.763246651345095
0.79 1.34001891424752
0.79 0.320710678118655
0.79 0.33996891847538
0.79 0.380277563773199
0.79 0.36180339887499
0.79 0.570087712549569
0.79 0.381720680758398
0.79 0.320156211871642
0.79 0.673355196269952
0.79 1.36529464379659
0.79 0.291547594742265
0.79 1.20933866224478
0.79 1.38443702165243
0.79 0.294317475868634
0.79 0.325661653798294
0.79 1.37382022948737
};
\addplot [line width=1.0pt, color0, opacity=1, forget plot]
table {%
1.75 0
1.83 0
1.83 0.125888347648319
1.75 0.125888347648319
1.75 0
};
\addplot [line width=1.0pt, color0, opacity=1, forget plot]
table {%
1.79 0
1.79 0
};
\addplot [line width=1.0pt, color0, opacity=1, forget plot]
table {%
1.79 0.125888347648319
1.79 0.292080962648189
};
\addplot [line width=1.0pt, color0, forget plot]
table {%
1.77 0
1.81 0
};
\addplot [line width=1.0pt, color0, forget plot]
table {%
1.77 0.292080962648189
1.81 0.292080962648189
};
\addplot [line width=0.5pt, color0, opacity=0.2, mark=*, mark size=1, mark options={solid}, only marks, forget plot]
table {%
1.79 0.360555127546399
1.79 0.465356789468263
1.79 2.14009345590327
1.79 0.677488823824227
1.79 0.320156211871642
1.79 0.403553390593274
1.79 0.522015325445528
1.79 0.320156211871642
1.79 0.431959610746632
1.79 0.584161925296378
1.79 0.837726079295291
1.79 2.04309920234053
1.79 0.431959610746632
1.79 1.33136400709533
1.79 0.360555127546399
1.79 0.440512483795333
};
\addplot [line width=1.0pt, color0, opacity=1, forget plot]
table {%
2.75 0
2.83 0
2.83 0.161803398874989
2.75 0.161803398874989
2.75 0
};
\addplot [line width=1.0pt, color0, opacity=1, forget plot]
table {%
2.79 0
2.79 0
};
\addplot [line width=1.0pt, color0, opacity=1, forget plot]
table {%
2.79 0.161803398874989
2.79 0.370156211871643
};
\addplot [line width=1.0pt, color0, forget plot]
table {%
2.77 0
2.81 0
};
\addplot [line width=1.0pt, color0, forget plot]
table {%
2.77 0.370156211871643
2.81 0.370156211871643
};
\addplot [line width=0.5pt, color0, opacity=0.2, mark=*, mark size=1, mark options={solid}, only marks, forget plot]
table {%
2.79 0.570087712549569
2.79 0.743303437365926
2.79 0.851097957023136
2.79 0.410679596594035
2.79 1.45206908733244
2.79 1.29250779955944
2.79 0.997805656208457
2.79 0.51478150704935
2.79 1.46473440639562
2.79 0.860232526704263
2.79 2.24722585387465
2.79 0.715891053163818
2.79 0.960935368856897
2.79 2.18169978352615
};
\addplot [line width=1.0pt, color0, opacity=1, forget plot]
table {%
3.75 0.0499999999999998
3.83 0.0499999999999998
3.83 0.2352081728299
3.75 0.2352081728299
3.75 0.0499999999999998
};
\addplot [line width=1.0pt, color0, opacity=1, forget plot]
table {%
3.79 0.0499999999999998
3.79 0
};
\addplot [line width=1.0pt, color0, opacity=1, forget plot]
table {%
3.79 0.2352081728299
3.79 0.510977222864645
};
\addplot [line width=1.0pt, color0, forget plot]
table {%
3.77 0
3.81 0
};
\addplot [line width=1.0pt, color0, forget plot]
table {%
3.77 0.510977222864645
3.81 0.510977222864645
};
\addplot [line width=0.5pt, color0, opacity=0.2, mark=*, mark size=1, mark options={solid}, only marks, forget plot]
table {%
3.79 1.20933866224478
3.79 1.07935165724615
3.79 0.776208734813001
3.79 0.68309518948453
3.79 1.96720093040294
3.79 0.680073525436772
3.79 1.69193464704018
3.79 1.00431791820731
3.79 1.87416648139913
3.79 1.41155942134931
3.79 1.52571847149921
3.79 0.602079728939615
3.79 0.922179996414975
3.79 0.763723639395254
3.79 1.30399362039845
3.79 0.51478150704935
3.79 1.57069063257455
3.79 0.515154392492244
3.79 0.970846978815294
3.79 1.78465682975747
3.79 0.936002257333467
3.79 0.56478150704935
3.79 1.60415410236136
3.79 1.35076001848209
};
\addplot [line width=1.0pt, color0, opacity=1, forget plot]
table {%
4.75 0
4.83 0
4.83 0.225274489255784
4.75 0.225274489255784
4.75 0
};
\addplot [line width=1.0pt, color0, opacity=1, forget plot]
table {%
4.79 0
4.79 0
};
\addplot [line width=1.0pt, color0, opacity=1, forget plot]
table {%
4.79 0.225274489255784
4.79 0.559016994374948
};
\addplot [line width=1.0pt, color0, forget plot]
table {%
4.77 0
4.81 0
};
\addplot [line width=1.0pt, color0, forget plot]
table {%
4.77 0.559016994374948
4.81 0.559016994374948
};
\addplot [line width=0.5pt, color0, opacity=0.2, mark=*, mark size=1, mark options={solid}, only marks, forget plot]
table {%
4.79 2.0335397882833
4.79 1.48475642064959
4.79 0.585234995535981
4.79 0.960468635614927
4.79 0.676498204307083
4.79 1.22505031618017
4.79 0.937489925564583
4.79 0.646848206938645
4.79 0.636396103067893
4.79 1.32990778863778
4.79 0.982735008131951
4.79 0.81401411548458
4.79 0.930277563773199
4.79 1.7464769689744
4.79 0.710327780786685
4.79 0.896168616937017
4.79 0.715891053163818
4.79 1.09988138961956
4.79 1.20104121494643
4.79 0.756637297521078
};
\addplot [line width=1.0pt, color0, opacity=1, forget plot]
table {%
5.75 0.05
5.83 0.05
5.83 0.294898439976609
5.75 0.294898439976609
5.75 0.05
};
\addplot [line width=1.0pt, color0, opacity=1, forget plot]
table {%
5.79 0.05
5.79 0
};
\addplot [line width=1.0pt, color0, opacity=1, forget plot]
table {%
5.79 0.294898439976609
5.79 0.654152298679729
};
\addplot [line width=1.0pt, color0, forget plot]
table {%
5.77 0
5.81 0
};
\addplot [line width=1.0pt, color0, forget plot]
table {%
5.77 0.654152298679729
5.81 0.654152298679729
};
\addplot [line width=0.5pt, color0, opacity=0.2, mark=*, mark size=1, mark options={solid}, only marks, forget plot]
table {%
5.79 1.61180339887499
5.79 0.941547594742265
5.79 0.667083203206317
5.79 1.41473440639562
5.79 1.66887598427011
5.79 1.1111432804529
5.79 0.962414379544733
5.79 1.34629120178363
5.79 1.33416640641263
5.79 0.68309518948453
5.79 1.11336254173556
5.79 0.908863408830665
5.79 1.15
5.79 0.790569415042095
5.79 1.01417621279207
5.79 1.13771762530915
};
\addplot [line width=1.0pt, color1, opacity=1, forget plot]
table {%
0.89 0
0.97 0
0.97 0.111803398874989
0.89 0.111803398874989
0.89 0
};
\addplot [line width=1.0pt, color1, opacity=1, forget plot]
table {%
0.93 0
0.93 0
};
\addplot [line width=1.0pt, color1, opacity=1, forget plot]
table {%
0.93 0.111803398874989
0.93 0.262132034355965
};
\addplot [line width=1.0pt, color1, forget plot]
table {%
0.91 0
0.95 0
};
\addplot [line width=1.0pt, color1, forget plot]
table {%
0.91 0.262132034355965
0.95 0.262132034355965
};
\addplot [line width=0.5pt, color1, opacity=0.2, mark=*, mark size=1, mark options={solid}, only marks, forget plot]
table {%
0.93 0.467083010035604
0.93 0.431959610746632
0.93 1.36473440639562
0.93 0.424264068711929
0.93 0.320156211871642
0.93 0.316227766016838
0.93 2.26238471563565
0.93 0.743303437365925
0.93 0.813216876123687
0.93 1.83371208208922
0.93 0.282842712474619
0.93 0.435228539452839
0.93 0.431265805665054
0.93 1.38655827727319
0.93 1.36655250605964
};
\addplot [line width=1.0pt, color1, opacity=1, forget plot]
table {%
1.89 0
1.97 0
1.97 0.141421356237309
1.89 0.141421356237309
1.89 0
};
\addplot [line width=1.0pt, color1, opacity=1, forget plot]
table {%
1.93 0
1.93 0
};
\addplot [line width=1.0pt, color1, opacity=1, forget plot]
table {%
1.93 0.141421356237309
1.93 0.338391446781619
};
\addplot [line width=1.0pt, color1, forget plot]
table {%
1.91 0
1.95 0
};
\addplot [line width=1.0pt, color1, forget plot]
table {%
1.91 0.338391446781619
1.95 0.338391446781619
};
\addplot [line width=0.5pt, color1, opacity=0.2, mark=*, mark size=1, mark options={solid}, only marks, forget plot]
table {%
1.93 0.912414379544733
1.93 1.16726175299288
1.93 1.39629120178363
1.93 1.52601696124808
1.93 0.602079728939615
1.93 0.390512483795333
1.93 0.998683298050514
1.93 1.35830777072061
1.93 0.906637297521078
1.93 0.45
1.93 1.32171027843833
};
\addplot [line width=1.0pt, color1, opacity=1, forget plot]
table {%
2.89 0
2.97 0
2.97 0.158113883008419
2.89 0.158113883008419
2.89 0
};
\addplot [line width=1.0pt, color1, opacity=1, forget plot]
table {%
2.93 0
2.93 0
};
\addplot [line width=1.0pt, color1, opacity=1, forget plot]
table {%
2.93 0.158113883008419
2.93 0.390512483795333
};
\addplot [line width=1.0pt, color1, forget plot]
table {%
2.91 0
2.95 0
};
\addplot [line width=1.0pt, color1, forget plot]
table {%
2.91 0.390512483795333
2.95 0.390512483795333
};
\addplot [line width=0.5pt, color1, opacity=0.2, mark=*, mark size=1, mark options={solid}, only marks, forget plot]
table {%
2.93 1.31284461382142
2.93 0.712958975826751
2.93 1.57069063257455
2.93 0.615244921770892
2.93 1.63773062836584
2.93 1.47716364330549
2.93 0.403112887414927
2.93 0.607477751039076
2.93 1.45933866224478
2.93 0.424264068711928
2.93 0.894427190999916
2.93 1.32071067811865
2.93 1.02983257257983
2.93 0.510555127546399
2.93 0.751664818918645
};
\addplot [line width=1.0pt, color1, opacity=1, forget plot]
table {%
3.89 0
3.97 0
3.97 0.182514076993644
3.89 0.182514076993644
3.89 0
};
\addplot [line width=1.0pt, color1, opacity=1, forget plot]
table {%
3.93 0
3.93 0
};
\addplot [line width=1.0pt, color1, opacity=1, forget plot]
table {%
3.93 0.182514076993644
3.93 0.432968950979575
};
\addplot [line width=1.0pt, color1, forget plot]
table {%
3.91 0
3.95 0
};
\addplot [line width=1.0pt, color1, forget plot]
table {%
3.91 0.432968950979575
3.95 0.432968950979575
};
\addplot [line width=0.5pt, color1, opacity=0.2, mark=*, mark size=1, mark options={solid}, only marks, forget plot]
table {%
3.93 1.99678886411344
3.93 1.91077940659284
3.93 0.544534243652237
3.93 1.11351985322196
3.93 0.502769256906871
3.93 0.473606797749979
3.93 0.474264068711929
3.93 1.14620911896943
3.93 1.04043260233424
3.93 0.531507290636732
3.93 0.542409734721485
3.93 0.728010988928052
3.93 1.37931142241337
3.93 1.67378058461996
3.93 0.604152298679729
};
\addplot [line width=1.0pt, color1, opacity=1, forget plot]
table {%
4.89 0.05
4.97 0.05
4.97 0.250247060472964
4.89 0.250247060472964
4.89 0.05
};
\addplot [line width=1.0pt, color1, opacity=1, forget plot]
table {%
4.93 0.05
4.93 0
};
\addplot [line width=1.0pt, color1, opacity=1, forget plot]
table {%
4.93 0.250247060472964
4.93 0.494974746830583
};
\addplot [line width=1.0pt, color1, forget plot]
table {%
4.91 0
4.95 0
};
\addplot [line width=1.0pt, color1, forget plot]
table {%
4.91 0.494974746830583
4.95 0.494974746830583
};
\addplot [line width=0.5pt, color1, opacity=0.2, mark=*, mark size=1, mark options={solid}, only marks, forget plot]
table {%
4.93 0.696419413859206
4.93 0.643310689511722
4.93 0.965660395791398
4.93 1.61968538207526
4.93 1.28490890352285
4.93 0.6
4.93 1.35
4.93 1.27071067811865
4.93 0.56315356820846
4.93 0.80642559834748
4.93 1.90809876610471
4.93 0.901387818865998
4.93 1.14127122105133
4.93 0.618465843842649
4.93 1.01980390271856
4.93 0.813586841255899
4.93 1.69297765539702
};
\addplot [line width=1.0pt, color1, opacity=1, forget plot]
table {%
5.89 0.05
5.97 0.05
5.97 0.256155281280883
5.89 0.256155281280883
5.89 0.05
};
\addplot [line width=1.0pt, color1, opacity=1, forget plot]
table {%
5.93 0.05
5.93 0
};
\addplot [line width=1.0pt, color1, opacity=1, forget plot]
table {%
5.93 0.256155281280883
5.93 0.55
};
\addplot [line width=1.0pt, color1, forget plot]
table {%
5.91 0
5.95 0
};
\addplot [line width=1.0pt, color1, forget plot]
table {%
5.91 0.55
5.95 0.55
};
\addplot [line width=0.5pt, color1, opacity=0.2, mark=*, mark size=1, mark options={solid}, only marks, forget plot]
table {%
5.93 0.813216876123687
5.93 1.2665274696415
5.93 1.34344084212139
5.93 0.633095189484531
5.93 0.94185487435959
5.93 1.52069063257456
5.93 1.07004672795163
5.93 1.00498756211209
5.93 0.635234995535981
5.93 1.20433963806152
5.93 0.60926816869581
5.93 0.602079728939615
5.93 0.615244921770892
5.93 1.85539753152795
5.93 1.48123874689897
5.93 0.905538513813742
5.93 1.18137084989848
5.93 0.81421176329558
};
\addplot [line width=1.0pt, color2, opacity=1, forget plot]
table {%
1.03 0
1.11 0
1.11 0.070710678118655
1.03 0.070710678118655
1.03 0
};
\addplot [line width=1.0pt, color2, opacity=1, forget plot]
table {%
1.07 0
1.07 0
};
\addplot [line width=1.0pt, color2, opacity=1, forget plot]
table {%
1.07 0.070710678118655
1.07 0.16180339887499
};
\addplot [line width=1.0pt, color2, forget plot]
table {%
1.05 0
1.09 0
};
\addplot [line width=1.0pt, color2, forget plot]
table {%
1.05 0.16180339887499
1.09 0.16180339887499
};
\addplot [line width=0.5pt, color2, opacity=0.2, mark=*, mark size=1, mark options={solid}, only marks, forget plot]
table {%
1.07 0.2302775637732
1.07 0.206155281280883
1.07 0.320156211871642
1.07 0.212132034355964
1.07 0.25
1.07 0.341421356237309
1.07 0.212132034355965
1.07 0.223606797749979
1.07 0.276865959399538
1.07 0.522210011530505
1.07 0.276865959399538
1.07 0.1802775637732
1.07 0.282842712474619
1.07 0.1802775637732
1.07 0.299535239245728
1.07 0.212132034355965
1.07 0.51478150704935
1.07 0.262132034355964
1.07 0.25
1.07 0.320156211871643
1.07 1.27229404716528
1.07 0.180277563773199
1.07 0.212132034355964
1.07 0.212132034355964
};
\addplot [line width=1.0pt, color2, opacity=1, forget plot]
table {%
2.03 0
2.11 0
2.11 0.1
2.03 0.1
2.03 0
};
\addplot [line width=1.0pt, color2, opacity=1, forget plot]
table {%
2.07 0
2.07 0
};
\addplot [line width=1.0pt, color2, opacity=1, forget plot]
table {%
2.07 0.1
2.07 0.25
};
\addplot [line width=1.0pt, color2, forget plot]
table {%
2.05 0
2.09 0
};
\addplot [line width=1.0pt, color2, forget plot]
table {%
2.05 0.25
2.09 0.25
};
\addplot [line width=0.5pt, color2, opacity=0.2, mark=*, mark size=1, mark options={solid}, only marks, forget plot]
table {%
2.07 0.299535239245728
2.07 0.261803398874989
2.07 0.320710678118655
2.07 1.18848643240047
2.07 0.581507290636732
2.07 1.26757956618859
2.07 0.35
2.07 0.269917281883408
2.07 1.32382022948737
2.07 0.262132034355964
2.07 1.28829024430725
2.07 0.320156211871642
2.07 0.471825158515464
2.07 0.292080962648189
};
\addplot [line width=1.0pt, color2, opacity=1, forget plot]
table {%
3.03 0
3.11 0
3.11 0.16180339887499
3.03 0.16180339887499
3.03 0
};
\addplot [line width=1.0pt, color2, opacity=1, forget plot]
table {%
3.07 0
3.07 0
};
\addplot [line width=1.0pt, color2, opacity=1, forget plot]
table {%
3.07 0.16180339887499
3.07 0.4
};
\addplot [line width=1.0pt, color2, forget plot]
table {%
3.05 0
3.09 0
};
\addplot [line width=1.0pt, color2, forget plot]
table {%
3.05 0.4
3.09 0.4
};
\addplot [line width=0.5pt, color2, opacity=0.2, mark=*, mark size=1, mark options={solid}, only marks, forget plot]
table {%
3.07 1.20415945787923
3.07 1.22576506721313
3.07 1.11018016555873
3.07 1.10113577727726
3.07 0.962132034355964
3.07 0.65
3.07 1.05475115548645
3.07 1.23490890352285
3.07 1.37167186124442
3.07 0.906155281280883
3.07 1.38293166859393
3.07 0.585492185168005
3.07 1.03106493541421
3.07 1.26655250605964
3.07 1.88243289924736
3.07 0.48139027471269
3.07 0.90146931829632
3.07 1.30096118312577
3.07 0.472358526421389
3.07 0.412132034355964
3.07 0.806225774829855
3.07 2
};
\addplot [line width=1.0pt, color2, opacity=1, forget plot]
table {%
4.03 0.0499999999999998
4.11 0.0499999999999998
4.11 0.182514076993644
4.03 0.182514076993644
4.03 0.0499999999999998
};
\addplot [line width=1.0pt, color2, opacity=1, forget plot]
table {%
4.07 0.0499999999999998
4.07 0
};
\addplot [line width=1.0pt, color2, opacity=1, forget plot]
table {%
4.07 0.182514076993644
4.07 0.36225827286092
};
\addplot [line width=1.0pt, color2, forget plot]
table {%
4.05 0
4.09 0
};
\addplot [line width=1.0pt, color2, forget plot]
table {%
4.05 0.36225827286092
4.09 0.36225827286092
};
\addplot [line width=0.5pt, color2, opacity=0.2, mark=*, mark size=1, mark options={solid}, only marks, forget plot]
table {%
4.07 0.441547594742265
4.07 0.96937215740457
4.07 1.20104121494643
4.07 1.32003787824441
4.07 0.420710678118655
4.07 2.31387138364065
4.07 0.75
4.07 0.863133825081604
4.07 1.31529464379659
4.07 1.99060292373944
4.07 1.27337830595216
4.07 1.23977926564762
4.07 1.1629703349613
4.07 0.474341649025257
4.07 0.85
4.07 0.626719685164907
4.07 0.461106256960522
4.07 0.962414379544733
4.07 0.969238815542512
4.07 1.01176920308357
4.07 0.592588126334965
4.07 0.756637297521078
4.07 1.10113577727726
};
\addplot [line width=1.0pt, color2, opacity=1, forget plot]
table {%
5.03 0.0499999999999998
5.11 0.0499999999999998
5.11 0.25
5.03 0.25
5.03 0.0499999999999998
};
\addplot [line width=1.0pt, color2, opacity=1, forget plot]
table {%
5.07 0.0499999999999998
5.07 0
};
\addplot [line width=1.0pt, color2, opacity=1, forget plot]
table {%
5.07 0.25
5.07 0.539003586140866
};
\addplot [line width=1.0pt, color2, forget plot]
table {%
5.05 0
5.09 0
};
\addplot [line width=1.0pt, color2, forget plot]
table {%
5.05 0.539003586140866
5.09 0.539003586140866
};
\addplot [line width=0.5pt, color2, opacity=0.2, mark=*, mark size=1, mark options={solid}, only marks, forget plot]
table {%
5.07 1.16122525895587
5.07 1.45618938158314
5.07 0.760977222864644
5.07 1.39522778527418
5.07 2.02989898732233
5.07 1.05686482428446
5.07 1.10113577727726
5.07 0.680073525436772
5.07 1.3729530217746
5.07 0.667083203206317
5.07 1.61470878610121
5.07 1.08173352245803
5.07 1.12361025271221
5.07 1.06066017177982
5.07 1.28695334678856
5.07 1.44714353830598
5.07 0.718794044486651
5.07 1.78218697243355
5.07 0.633818724320517
};
\addplot [line width=1.0pt, color2, opacity=1, forget plot]
table {%
6.03 0.05
6.11 0.05
6.11 0.223606797749979
6.03 0.223606797749979
6.03 0.05
};
\addplot [line width=1.0pt, color2, opacity=1, forget plot]
table {%
6.07 0.05
6.07 0
};
\addplot [line width=1.0pt, color2, opacity=1, forget plot]
table {%
6.07 0.223606797749979
6.07 0.461577568108952
};
\addplot [line width=1.0pt, color2, forget plot]
table {%
6.05 0
6.09 0
};
\addplot [line width=1.0pt, color2, forget plot]
table {%
6.05 0.461577568108952
6.09 0.461577568108952
};
\addplot [line width=0.5pt, color2, opacity=0.2, mark=*, mark size=1, mark options={solid}, only marks, forget plot]
table {%
6.07 1.32382022948737
6.07 0.524341649025257
6.07 1.2369316876853
6.07 0.626498204307083
6.07 0.996242942258564
6.07 1.03077640640442
6.07 0.98488578017961
6.07 0.524341649025257
6.07 0.813941029804985
6.07 0.912414379544733
6.07 1.15
6.07 1.11191284758509
6.07 0.730073525436772
6.07 1.09658560997307
6.07 0.664071449734353
6.07 1.06345964923757
6.07 0.6
6.07 1.82482875908947
6.07 1.29660978857887
6.07 0.75
6.07 0.90146931829632
6.07 1.35
6.07 0.540832691319598
};
\addplot [line width=1.0pt, color3, opacity=1, forget plot]
table {%
1.17 0
1.25 0
1.25 0.0780330085889911
1.17 0.0780330085889911
1.17 0
};
\addplot [line width=1.0pt, color3, opacity=1, forget plot]
table {%
1.21 0
1.21 0
};
\addplot [line width=1.0pt, color3, opacity=1, forget plot]
table {%
1.21 0.0780330085889911
1.21 0.1802775637732
};
\addplot [line width=1.0pt, color3, forget plot]
table {%
1.19 0
1.23 0
};
\addplot [line width=1.0pt, color3, forget plot]
table {%
1.19 0.1802775637732
1.23 0.1802775637732
};
\addplot [line width=0.5pt, color3, opacity=0.2, mark=*, mark size=1, mark options={solid}, only marks, forget plot]
table {%
1.21 0.223606797749979
1.21 0.863133825081603
1.21 0.230277563773199
1.21 0.602398579101954
1.21 0.282842712474619
1.21 0.394646111349609
1.21 0.208113883008419
1.21 1.70308788046558
1.21 0.223606797749979
1.21 0.320710678118655
1.21 0.366754374554629
1.21 0.206155281280883
1.21 0.269917281883408
1.21 0.930642840095968
1.21 0.206155281280883
1.21 0.497702876023148
1.21 0.223606797749979
1.21 0.280277563773199
1.21 0.282842712474619
};
\addplot [line width=1.0pt, color3, opacity=1, forget plot]
table {%
2.17 0
2.25 0
2.25 0.111803398874989
2.17 0.111803398874989
2.17 0
};
\addplot [line width=1.0pt, color3, opacity=1, forget plot]
table {%
2.21 0
2.21 0
};
\addplot [line width=1.0pt, color3, opacity=1, forget plot]
table {%
2.21 0.111803398874989
2.21 0.253224755112299
};
\addplot [line width=1.0pt, color3, forget plot]
table {%
2.19 0
2.23 0
};
\addplot [line width=1.0pt, color3, forget plot]
table {%
2.19 0.253224755112299
2.23 0.253224755112299
};
\addplot [line width=0.5pt, color3, opacity=0.2, mark=*, mark size=1, mark options={solid}, only marks, forget plot]
table {%
2.21 1.42709804267158
2.21 1.29197423483742
2.21 1.50646046820591
2.21 0.5
2.21 0.282842712474619
2.21 0.5
2.21 0.353553390593274
2.21 0.816673666841092
2.21 0.96327271717131
2.21 1.2369316876853
2.21 0.583390451188127
2.21 1.34629120178363
2.21 0.865891053163818
};
\addplot [line width=1.0pt, color3, opacity=1, forget plot]
table {%
3.17 0
3.25 0
3.25 0.158113883008419
3.17 0.158113883008419
3.17 0
};
\addplot [line width=1.0pt, color3, opacity=1, forget plot]
table {%
3.21 0
3.21 0
};
\addplot [line width=1.0pt, color3, opacity=1, forget plot]
table {%
3.21 0.158113883008419
3.21 0.374848804633566
};
\addplot [line width=1.0pt, color3, forget plot]
table {%
3.19 0
3.23 0
};
\addplot [line width=1.0pt, color3, forget plot]
table {%
3.19 0.374848804633566
3.23 0.374848804633566
};
\addplot [line width=0.5pt, color3, opacity=0.2, mark=*, mark size=1, mark options={solid}, only marks, forget plot]
table {%
3.21 0.908863408830666
3.21 1.90727378306406
3.21 0.677200187265876
3.21 0.969238815542512
3.21 0.653112887414928
3.21 0.403884361523178
3.21 1.5
3.21 1.35369863706809
3.21 1.26589889011722
3.21 1.25415945787923
3.21 0.930053761886914
3.21 0.429762079030862
3.21 0.906754233828454
3.21 0.781024967590666
3.21 1.16729628809172
3.21 1.02082439194738
3.21 1.16726175299288
3.21 1.30862523283024
};
\addplot [line width=1.0pt, color3, opacity=1, forget plot]
table {%
4.17 0.0499999999999998
4.25 0.0499999999999998
4.25 0.16180339887499
4.17 0.16180339887499
4.17 0.0499999999999998
};
\addplot [line width=1.0pt, color3, opacity=1, forget plot]
table {%
4.21 0.0499999999999998
4.21 0
};
\addplot [line width=1.0pt, color3, opacity=1, forget plot]
table {%
4.21 0.16180339887499
4.21 0.320156211871642
};
\addplot [line width=1.0pt, color3, forget plot]
table {%
4.19 0
4.23 0
};
\addplot [line width=1.0pt, color3, forget plot]
table {%
4.19 0.320156211871642
4.23 0.320156211871642
};
\addplot [line width=0.5pt, color3, opacity=0.2, mark=*, mark size=1, mark options={solid}, only marks, forget plot]
table {%
4.21 1.08077640640442
4.21 0.570710678118655
4.21 0.908113883008419
4.21 0.813941029804985
4.21 0.403112887414927
4.21 0.675252310603783
4.21 0.483021240680421
4.21 0.35
4.21 1.52268120235369
4.21 0.492442890089805
4.21 0.670820393249937
4.21 0.808504559443626
4.21 1.74710960480222
4.21 1.04403065089106
4.21 1.31284461382142
4.21 0.574390307216884
4.21 0.380788655293195
4.21 0.354950975679639
4.21 1.10235672509421
4.21 0.353553390593274
4.21 1.22198356443372
};
\addplot [line width=1.0pt, color3, opacity=1, forget plot]
table {%
5.17 0.0499999999999998
5.25 0.0499999999999998
5.25 0.215000725204468
5.17 0.215000725204468
5.17 0.0499999999999998
};
\addplot [line width=1.0pt, color3, opacity=1, forget plot]
table {%
5.21 0.0499999999999998
5.21 0
};
\addplot [line width=1.0pt, color3, opacity=1, forget plot]
table {%
5.21 0.215000725204468
5.21 0.441547594742265
};
\addplot [line width=1.0pt, color3, forget plot]
table {%
5.19 0
5.23 0
};
\addplot [line width=1.0pt, color3, forget plot]
table {%
5.19 0.441547594742265
5.23 0.441547594742265
};
\addplot [line width=0.5pt, color3, opacity=0.2, mark=*, mark size=1, mark options={solid}, only marks, forget plot]
table {%
5.21 0.652079728939614
5.21 0.97082439194738
5.21 1.28866362846897
5.21 2.33539416862632
5.21 1.3729530217746
5.21 1.7367127793433
5.21 0.94339811320566
5.21 0.650556773098224
5.21 0.517924273618613
5.21 1.83234413521942
5.21 1.05492034799999
5.21 1.25933866224478
5.21 1.55777320416261
5.21 1.01242283656583
5.21 0.591421356237309
5.21 1.0361541461658
5.21 0.47169905660283
5.21 1.1683574163638
5.21 0.776579726851068
5.21 0.85173564570932
5.21 1.05948100502085
5.21 0.93407708461347
};
\addplot [line width=1.0pt, color3, opacity=1, forget plot]
table {%
6.17 0.05
6.25 0.05
6.25 0.2352081728299
6.17 0.2352081728299
6.17 0.05
};
\addplot [line width=1.0pt, color3, opacity=1, forget plot]
table {%
6.21 0.05
6.21 0
};
\addplot [line width=1.0pt, color3, opacity=1, forget plot]
table {%
6.21 0.2352081728299
6.21 0.5
};
\addplot [line width=1.0pt, color3, forget plot]
table {%
6.19 0
6.23 0
};
\addplot [line width=1.0pt, color3, forget plot]
table {%
6.19 0.5
6.23 0.5
};
\addplot [line width=0.5pt, color3, opacity=0.2, mark=*, mark size=1, mark options={solid}, only marks, forget plot]
table {%
6.21 0.832165848854662
6.21 0.56478150704935
6.21 0.542442890089805
6.21 1.72016381818975
6.21 0.75178344238091
6.21 0.640832691319599
6.21 1.09201648339208
6.21 0.790312423743285
6.21 1.35096118312577
6.21 1.1683574163638
6.21 0.813216876123687
6.21 0.764852927038918
6.21 0.762266181688147
6.21 0.707106781186547
6.21 0.682060078537598
6.21 0.515154392492244
6.21 0.936341443563041
6.21 0.806225774829855
6.21 1.3247548783982
6.21 1.75071414000116
};
\addplot [line width=1.0pt, color4, opacity=1, forget plot]
table {%
1.31 0
1.39 0
1.39 0.070710678118655
1.31 0.070710678118655
1.31 0
};
\addplot [line width=1.0pt, color4, opacity=1, forget plot]
table {%
1.35 0
1.35 0
};
\addplot [line width=1.0pt, color4, opacity=1, forget plot]
table {%
1.35 0.070710678118655
1.35 0.16180339887499
};
\addplot [line width=1.0pt, color4, forget plot]
table {%
1.33 0
1.37 0
};
\addplot [line width=1.0pt, color4, forget plot]
table {%
1.33 0.16180339887499
1.37 0.16180339887499
};
\addplot [line width=0.5pt, color4, opacity=0.2, mark=*, mark size=1, mark options={solid}, only marks, forget plot]
table {%
1.35 0.394646111349609
1.35 0.282842712474619
1.35 0.269917281883409
1.35 1.46673911583866
1.35 0.432842712474619
1.35 0.230277563773199
1.35 0.515154392492244
1.35 0.223606797749979
1.35 0.223606797749979
1.35 0.361803398874989
1.35 0.262132034355964
1.35 0.460977222864644
1.35 0.1802775637732
1.35 2.10965570193067
1.35 0.308113883008419
1.35 0.191421356237309
1.35 0.1802775637732
1.35 0.403553390593274
1.35 0.273606797749979
1.35 0.282842712474619
1.35 0.1802775637732
1.35 1.85071414000116
1.35 0.180277563773199
1.35 0.206155281280883
1.35 0.461223161913987
1.35 1.6650258555759
1.35 0.180277563773199
};
\addplot [line width=1.0pt, color4, opacity=1, forget plot]
table {%
2.31 0
2.39 0
2.39 0.14142135623731
2.31 0.14142135623731
2.31 0
};
\addplot [line width=1.0pt, color4, opacity=1, forget plot]
table {%
2.35 0
2.35 0
};
\addplot [line width=1.0pt, color4, opacity=1, forget plot]
table {%
2.35 0.14142135623731
2.35 0.323935433230954
};
\addplot [line width=1.0pt, color4, forget plot]
table {%
2.33 0
2.37 0
};
\addplot [line width=1.0pt, color4, forget plot]
table {%
2.33 0.323935433230954
2.37 0.323935433230954
};
\addplot [line width=0.5pt, color4, opacity=0.2, mark=*, mark size=1, mark options={solid}, only marks, forget plot]
table {%
2.35 1.40830777072061
2.35 2.10060966544099
2.35 0.75
2.35 0.403884361523178
2.35 1.52096408039233
2.35 0.353553390593274
2.35 1.19443118969736
2.35 1.42042863170523
2.35 1.57997311700891
2.35 0.905538513813742
2.35 0.572780621739634
2.35 0.636396103067893
2.35 1.30862523283024
};
\addplot [line width=1.0pt, color4, opacity=1, forget plot]
table {%
3.31 0
3.39 0
3.39 0.161803398874989
3.31 0.161803398874989
3.31 0
};
\addplot [line width=1.0pt, color4, opacity=1, forget plot]
table {%
3.35 0
3.35 0
};
\addplot [line width=1.0pt, color4, opacity=1, forget plot]
table {%
3.35 0.161803398874989
3.35 0.394646111349608
};
\addplot [line width=1.0pt, color4, forget plot]
table {%
3.33 0
3.37 0
};
\addplot [line width=1.0pt, color4, forget plot]
table {%
3.33 0.394646111349608
3.37 0.394646111349608
};
\addplot [line width=0.5pt, color4, opacity=0.2, mark=*, mark size=1, mark options={solid}, only marks, forget plot]
table {%
3.35 2.12602916254693
3.35 1.00055754037412
3.35 3.76981650343561
3.35 1.49204215827886
3.35 1.87125077542427
3.35 0.561226770423347
3.35 0.561066219066395
3.35 1.20108644332213
3.35 2.10965821423159
3.35 0.430277563773199
3.35 1.5233628202243
3.35 0.813941029804985
3.35 0.873212459828649
3.35 1.01319121774099
3.35 0.552268050859363
3.35 1.36473440639562
3.35 1.52078619745851
3.35 1.79513230710162
3.35 1.25399362039845
};
\addplot [line width=1.0pt, color4, opacity=1, forget plot]
table {%
4.31 0
4.39 0
4.39 0.180836692078311
4.31 0.180836692078311
4.31 0
};
\addplot [line width=1.0pt, color4, opacity=1, forget plot]
table {%
4.35 0
4.35 0
};
\addplot [line width=1.0pt, color4, opacity=1, forget plot]
table {%
4.35 0.180836692078311
4.35 0.430788655293196
};
\addplot [line width=1.0pt, color4, forget plot]
table {%
4.33 0
4.37 0
};
\addplot [line width=1.0pt, color4, forget plot]
table {%
4.33 0.430788655293196
4.37 0.430788655293196
};
\addplot [line width=0.5pt, color4, opacity=0.2, mark=*, mark size=1, mark options={solid}, only marks, forget plot]
table {%
4.35 0.95
4.35 1.35830777072061
4.35 1.01176920308357
4.35 0.559016994374947
4.35 0.806225774829855
4.35 1.1647559034407
4.35 1.89225278035148
4.35 1.98305320150519
4.35 1.29184577099558
4.35 1.08488578017961
4.35 1.3865424623862
4.35 0.602079728939615
4.35 1.05349002360185
4.35 0.726498204307083
4.35 0.696419413859206
4.35 2.3215252336715
};
\addplot [line width=1.0pt, color4, opacity=1, forget plot]
table {%
5.31 0
5.39 0
5.39 0.243566017177982
5.31 0.243566017177982
5.31 0
};
\addplot [line width=1.0pt, color4, opacity=1, forget plot]
table {%
5.35 0
5.35 0
};
\addplot [line width=1.0pt, color4, opacity=1, forget plot]
table {%
5.35 0.243566017177982
5.35 0.531507290636733
};
\addplot [line width=1.0pt, color4, forget plot]
table {%
5.33 0
5.37 0
};
\addplot [line width=1.0pt, color4, forget plot]
table {%
5.33 0.531507290636733
5.37 0.531507290636733
};
\addplot [line width=0.5pt, color4, opacity=0.2, mark=*, mark size=1, mark options={solid}, only marks, forget plot]
table {%
5.35 2.25204439394692
5.35 0.832165848854661
5.35 0.97082439194738
5.35 0.697208882425738
5.35 2.27160576834922
5.35 1.92093727122985
5.35 1.32003787824441
5.35 0.75
5.35 0.793303437365925
5.35 1.68410988375058
5.35 1
5.35 1.47478702530923
5.35 0.626498204307083
5.35 1.53770465938516
};
\addplot [line width=1.0pt, color4, opacity=1, forget plot]
table {%
6.31 0.05
6.39 0.05
6.39 0.229187811788605
6.31 0.229187811788605
6.31 0.05
};
\addplot [line width=1.0pt, color4, opacity=1, forget plot]
table {%
6.35 0.05
6.35 0
};
\addplot [line width=1.0pt, color4, opacity=1, forget plot]
table {%
6.35 0.229187811788605
6.35 0.474341649025257
};
\addplot [line width=1.0pt, color4, forget plot]
table {%
6.33 0
6.37 0
};
\addplot [line width=1.0pt, color4, forget plot]
table {%
6.33 0.474341649025257
6.37 0.474341649025257
};
\addplot [line width=0.5pt, color4, opacity=0.2, mark=*, mark size=1, mark options={solid}, only marks, forget plot]
table {%
6.35 1.84513230710162
6.35 0.762132034355964
6.35 0.8
6.35 0.526311493152526
6.35 0.901920240520265
6.35 1.36645241104247
6.35 0.777817459305202
6.35 1.00124921972504
6.35 0.767919560544393
6.35 1.04407241234917
6.35 1.25099960031968
6.35 1.30399362039844
6.35 1.32004335184355
6.35 1.27769323391806
6.35 1.00623058987491
6.35 1.18234963238727
6.35 1.67563125566842
6.35 0.532842712474619
6.35 1.25399362039845
6.35 1.00124921972504
6.35 1.30386490545749
6.35 0.761803398874989
6.35 1.27026034939501
6.35 1.54164338901763
6.35 0.895331803242187
6.35 0.669781156610112
};
\addplot [line width=1.0pt, black, opacity=1, forget plot]
table {%
0.61 0
0.69 0
};
\addplot [line width=1.0pt, black, dashed, mark=x, mark size=3, mark options={solid}, forget plot]
table {%
0.65 0.0872899830220492
};
\addplot [line width=1.0pt, black, opacity=1, forget plot]
table {%
1.61 0.0499999999999999
1.69 0.0499999999999999
};
\addplot [line width=1.0pt, black, dashed, mark=x, mark size=3, mark options={solid}, forget plot]
table {%
1.65 0.138357173785249
};
\addplot [line width=1.0pt, black, opacity=1, forget plot]
table {%
2.61 0.0500000000000003
2.69 0.0500000000000003
};
\addplot [line width=1.0pt, black, dashed, mark=x, mark size=3, mark options={solid}, forget plot]
table {%
2.65 0.245182135441024
};
\addplot [line width=1.0pt, black, opacity=1, forget plot]
table {%
3.61 0.070710678118655
3.69 0.070710678118655
};
\addplot [line width=1.0pt, black, dashed, mark=x, mark size=3, mark options={solid}, forget plot]
table {%
3.65 0.283040686883383
};
\addplot [line width=1.0pt, black, opacity=1, forget plot]
table {%
4.61 0.1
4.69 0.1
};
\addplot [line width=1.0pt, black, dashed, mark=x, mark size=3, mark options={solid}, forget plot]
table {%
4.65 0.260476092348864
};
\addplot [line width=1.0pt, black, opacity=1, forget plot]
table {%
5.61 0.120710678118655
5.69 0.120710678118655
};
\addplot [line width=1.0pt, black, dashed, mark=x, mark size=3, mark options={solid}, forget plot]
table {%
5.65 0.265127646762684
};
\addplot [line width=1.0pt, color0, opacity=1, forget plot]
table {%
0.75 0
0.83 0
};
\addplot [line width=1.0pt, color0, dashed, mark=x, mark size=3, mark options={solid}, forget plot]
table {%
0.79 0.100816885214417
};
\addplot [line width=1.0pt, color0, opacity=1, forget plot]
table {%
1.75 0.05
1.83 0.05
};
\addplot [line width=1.0pt, color0, dashed, mark=x, mark size=3, mark options={solid}, forget plot]
table {%
1.79 0.118801789249429
};
\addplot [line width=1.0pt, color0, opacity=1, forget plot]
table {%
2.75 0.0500000000000003
2.83 0.0500000000000003
};
\addplot [line width=1.0pt, color0, dashed, mark=x, mark size=3, mark options={solid}, forget plot]
table {%
2.79 0.16010468385041
};
\addplot [line width=1.0pt, color0, opacity=1, forget plot]
table {%
3.75 0.1
3.83 0.1
};
\addplot [line width=1.0pt, color0, dashed, mark=x, mark size=3, mark options={solid}, forget plot]
table {%
3.79 0.237073753575777
};
\addplot [line width=1.0pt, color0, opacity=1, forget plot]
table {%
4.75 0.1
4.83 0.1
};
\addplot [line width=1.0pt, color0, dashed, mark=x, mark size=3, mark options={solid}, forget plot]
table {%
4.79 0.207780427649447
};
\addplot [line width=1.0pt, color0, opacity=1, forget plot]
table {%
5.75 0.131066017177982
5.83 0.131066017177982
};
\addplot [line width=1.0pt, color0, dashed, mark=x, mark size=3, mark options={solid}, forget plot]
table {%
5.79 0.244392934342671
};
\addplot [line width=1.0pt, color1, opacity=1, forget plot]
table {%
0.89 0.0499999999999998
0.97 0.0499999999999998
};
\addplot [line width=1.0pt, color1, dashed, mark=x, mark size=3, mark options={solid}, forget plot]
table {%
0.93 0.110055195557424
};
\addplot [line width=1.0pt, color1, opacity=1, forget plot]
table {%
1.89 0.0500000000000003
1.97 0.0500000000000003
};
\addplot [line width=1.0pt, color1, dashed, mark=x, mark size=3, mark options={solid}, forget plot]
table {%
1.93 0.126224698504023
};
\addplot [line width=1.0pt, color1, opacity=1, forget plot]
table {%
2.89 0.05
2.97 0.05
};
\addplot [line width=1.0pt, color1, dashed, mark=x, mark size=3, mark options={solid}, forget plot]
table {%
2.93 0.147426660043925
};
\addplot [line width=1.0pt, color1, opacity=1, forget plot]
table {%
3.89 0.0707106781186547
3.97 0.0707106781186547
};
\addplot [line width=1.0pt, color1, dashed, mark=x, mark size=3, mark options={solid}, forget plot]
table {%
3.93 0.162966137130819
};
\addplot [line width=1.0pt, color1, opacity=1, forget plot]
table {%
4.89 0.145710678118655
4.97 0.145710678118655
};
\addplot [line width=1.0pt, color1, dashed, mark=x, mark size=3, mark options={solid}, forget plot]
table {%
4.93 0.224748734918003
};
\addplot [line width=1.0pt, color1, opacity=1, forget plot]
table {%
5.89 0.116257038496822
5.97 0.116257038496822
};
\addplot [line width=1.0pt, color1, dashed, mark=x, mark size=3, mark options={solid}, forget plot]
table {%
5.93 0.220157735850325
};
\addplot [line width=1.0pt, color2, opacity=1, forget plot]
table {%
1.03 0
1.11 0
};
\addplot [line width=1.0pt, color2, dashed, mark=x, mark size=3, mark options={solid}, forget plot]
table {%
1.07 0.0655164361249598
};
\addplot [line width=1.0pt, color2, opacity=1, forget plot]
table {%
2.03 0.0499999999999998
2.11 0.0499999999999998
};
\addplot [line width=1.0pt, color2, dashed, mark=x, mark size=3, mark options={solid}, forget plot]
table {%
2.07 0.0876833156535148
};
\addplot [line width=1.0pt, color2, opacity=1, forget plot]
table {%
3.03 0.0707106781186547
3.11 0.0707106781186547
};
\addplot [line width=1.0pt, color2, dashed, mark=x, mark size=3, mark options={solid}, forget plot]
table {%
3.07 0.188716299073459
};
\addplot [line width=1.0pt, color2, opacity=1, forget plot]
table {%
4.03 0.0707106781186549
4.11 0.0707106781186549
};
\addplot [line width=1.0pt, color2, dashed, mark=x, mark size=3, mark options={solid}, forget plot]
table {%
4.07 0.195789952834811
};
\addplot [line width=1.0pt, color2, opacity=1, forget plot]
table {%
5.03 0.1
5.11 0.1
};
\addplot [line width=1.0pt, color2, dashed, mark=x, mark size=3, mark options={solid}, forget plot]
table {%
5.07 0.222697726278375
};
\addplot [line width=1.0pt, color2, opacity=1, forget plot]
table {%
6.03 0.111803398874989
6.11 0.111803398874989
};
\addplot [line width=1.0pt, color2, dashed, mark=x, mark size=3, mark options={solid}, forget plot]
table {%
6.07 0.214635894637932
};
\addplot [line width=1.0pt, color3, opacity=1, forget plot]
table {%
1.17 0
1.25 0
};
\addplot [line width=1.0pt, color3, dashed, mark=x, mark size=3, mark options={solid}, forget plot]
table {%
1.21 0.0749524488868239
};
\addplot [line width=1.0pt, color3, opacity=1, forget plot]
table {%
2.17 0.0499999999999998
2.25 0.0499999999999998
};
\addplot [line width=1.0pt, color3, dashed, mark=x, mark size=3, mark options={solid}, forget plot]
table {%
2.21 0.107515367473041
};
\addplot [line width=1.0pt, color3, opacity=1, forget plot]
table {%
3.17 0.05
3.25 0.05
};
\addplot [line width=1.0pt, color3, dashed, mark=x, mark size=3, mark options={solid}, forget plot]
table {%
3.21 0.162931699314661
};
\addplot [line width=1.0pt, color3, opacity=1, forget plot]
table {%
4.17 0.0707106781186548
4.25 0.0707106781186548
};
\addplot [line width=1.0pt, color3, dashed, mark=x, mark size=3, mark options={solid}, forget plot]
table {%
4.21 0.158530214441077
};
\addplot [line width=1.0pt, color3, opacity=1, forget plot]
table {%
5.17 0.0999999999999999
5.25 0.0999999999999999
};
\addplot [line width=1.0pt, color3, dashed, mark=x, mark size=3, mark options={solid}, forget plot]
table {%
5.21 0.217886726175233
};
\addplot [line width=1.0pt, color3, opacity=1, forget plot]
table {%
6.17 0.111803398874989
6.25 0.111803398874989
};
\addplot [line width=1.0pt, color3, dashed, mark=x, mark size=3, mark options={solid}, forget plot]
table {%
6.21 0.210422645209879
};
\addplot [line width=1.0pt, color4, opacity=1, forget plot]
table {%
1.31 0
1.39 0
};
\addplot [line width=1.0pt, color4, dashed, mark=x, mark size=3, mark options={solid}, forget plot]
table {%
1.35 0.093678144189488
};
\addplot [line width=1.0pt, color4, opacity=1, forget plot]
table {%
2.31 0.05
2.39 0.05
};
\addplot [line width=1.0pt, color4, dashed, mark=x, mark size=3, mark options={solid}, forget plot]
table {%
2.35 0.136372575855112
};
\addplot [line width=1.0pt, color4, opacity=1, forget plot]
table {%
3.31 0.0500000000000002
3.39 0.0500000000000002
};
\addplot [line width=1.0pt, color4, dashed, mark=x, mark size=3, mark options={solid}, forget plot]
table {%
3.35 0.208342452755889
};
\addplot [line width=1.0pt, color4, opacity=1, forget plot]
table {%
4.31 0.0707106781186548
4.39 0.0707106781186548
};
\addplot [line width=1.0pt, color4, dashed, mark=x, mark size=3, mark options={solid}, forget plot]
table {%
4.35 0.18268515040223
};
\addplot [line width=1.0pt, color4, opacity=1, forget plot]
table {%
5.31 0.0707106781186551
5.39 0.0707106781186551
};
\addplot [line width=1.0pt, color4, dashed, mark=x, mark size=3, mark options={solid}, forget plot]
table {%
5.35 0.199635397139792
};
\addplot [line width=1.0pt, color4, opacity=1, forget plot]
table {%
6.31 0.11180339887499
6.39 0.11180339887499
};
\addplot [line width=1.0pt, color4, dashed, mark=x, mark size=3, mark options={solid}, forget plot]
table {%
6.35 0.24530237149585
};
\end{axis}

\node at ({$(current bounding box.south west)!0.5!(current bounding box.south east)$}|-{$(current bounding box.south west)!0.98!(current bounding box.north west)$})[
  anchor=north,
  text=black,
  rotate=0.0
]{ };

	    \begin{customlegend}[
legend entries={$\sigma^{2, \text{(0)}}=0.1$,$\sigma^{2, \text{(0)}}=0.5$,$\sigma^{2, \text{(0)}}=1.0$,$\sigma^{2, \text{(0)}}=2.0$,$\sigma^{2, \text{(0)}}=3.0$,$\sigma^{2, \text{(0)}}=5.0$},
legend cell align=left,
legend style={at={(0.05,5.37)}, anchor=north west, draw=white!80.0!black, font=\footnotesize,fill opacity=0.5, draw opacity=1,text opacity=1}]
% the following are the "images" and numbers in the legend
    \addlegendimage{area legend,black,fill=black, fill opacity=1}
    \addlegendimage{area legend,color0,fill=color0, fill opacity=1}
    \addlegendimage{area legend,color1,fill=color1, fill opacity=1}
    \addlegendimage{area legend,color2,fill=color2, fill opacity=1}
    \addlegendimage{area legend,color3,fill=color3, fill opacity=1}
    \addlegendimage{area legend,color4,fill=color4, fill opacity=1}
\end{customlegend}
	\end{tikzpicture}
}
	\caption[Evaluation Results for Varying Initial Variances]{Evaluation Results for Varying Initial Variances ($L=10$, $n=200$)}
	\label{fig:trialVarianceNotFixed}
\end{figure}

For this trial the number of \gls{em} iterations is increased to $L=10$, as the convergence properties of the variance is examined. As \autoref{fig:trialVarianceNotFixed} shows, the initial variance $\sigma^{2, \text{(0)}}$ does not have a consistent effect on the localisation performance. For $S=7$ there is a slight improvement evident for higher $\sigma^{2, \text{(0)}}$, but for $S=3$ lower initial variances seem to do slightly better. Overall, the difference among different $\sigma^{2, \text{(0)}}$ is very small and likely not attributable to any advantage, a certain initial value might have over another. This could be explained by a fast convergence speed of the variance, so that initial values within a reasonable range as those that have been tested in this trial have no bearing on the localisation performance. As \cite{Schwartz2014} used a fixed variance, the next evaluation will show, whether setting a fixed variance will lead to an advantage in localisation performance.