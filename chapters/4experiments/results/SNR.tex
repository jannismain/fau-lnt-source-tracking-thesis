\subsubsection*{\glsentrydesc{snr}}
%% all-in-one boxplot
\begin{figure}[!ht]
\iftoggle{quick}{%
    \includegraphics[width=\textwidth]{plots/boxplots/boxplot-joined-SNR}
}{%
    \begin{tikzpicture}
	    % This file was created by matplotlib2tikz v0.6.14.
\definecolor{color0}{rgb}{0.8,0.207843137254902,0.219607843137255}
\definecolor{color1}{rgb}{1,0.647058823529412,0}
\definecolor{color2}{rgb}{0.0235294117647059,0.603921568627451,0.952941176470588}
\definecolor{color3}{rgb}{0.219607843137255,0.00784313725490196,0.509803921568627}

\begin{axis}[
xlabel={number of sources ($S$)},
ylabel={mean localisation error},
xmin=0.5, xmax=6.5,
ymin=0, ymax=2.5,
width=\figurewidth,
height=\figureheight,
xtick={1,2,3,4,5,6},
xticklabels={2,3,4,5,6,7},
ytick={0,0.5,1,1.5,2,2.5},
minor xtick={},
minor ytick={},
tick align=outside,
tick pos=left,
x grid style={white!69.019607843137251!black},
ymajorgrids,
y grid style={white!69.019607843137251!black}
]
\addplot [line width=1.0pt, black, opacity=1, forget plot]
table {%
0.71 0
0.79 0
0.79 0.14142135623731
0.71 0.14142135623731
0.71 0
};
\addplot [line width=1.0pt, black, opacity=1, forget plot]
table {%
0.75 0
0.75 0
};
\addplot [line width=1.0pt, black, opacity=1, forget plot]
table {%
0.75 0.14142135623731
0.75 0.341547594742265
};
\addplot [line width=1.0pt, black, forget plot]
table {%
0.73 0
0.77 0
};
\addplot [line width=1.0pt, black, forget plot]
table {%
0.73 0.341547594742265
0.77 0.341547594742265
};
\addplot [line width=1.0pt, black, opacity=0.2, mark=*, mark size=1, mark options={solid}, only marks, forget plot]
table {%
0.75 1.30384048104053
0.75 0.694660482157192
0.75 0.518669010554817
0.75 0.381720680758398
0.75 0.403884361523178
0.75 1.14236596587959
0.75 0.629727672493603
0.75 0.403112887414928
0.75 0.700944867164838
0.75 1.41014705087354
0.75 0.497702876023148
};
\addplot [line width=1.0pt, black, opacity=1, forget plot]
table {%
1.71 0.0333333333333332
1.79 0.0333333333333332
1.79 0.179615333825597
1.71 0.179615333825597
1.71 0.0333333333333332
};
\addplot [line width=1.0pt, black, opacity=1, forget plot]
table {%
1.75 0.0333333333333332
1.75 0
};
\addplot [line width=1.0pt, black, opacity=1, forget plot]
table {%
1.75 0.179615333825597
1.75 0.389001636985213
};
\addplot [line width=1.0pt, black, forget plot]
table {%
1.73 0
1.77 0
};
\addplot [line width=1.0pt, black, forget plot]
table {%
1.73 0.389001636985213
1.77 0.389001636985213
};
\addplot [line width=1.0pt, black, opacity=0.2, mark=*, mark size=1, mark options={solid}, only marks, forget plot]
table {%
1.75 1.67116980736225
1.75 0.936670798525062
1.75 0.847834595376724
1.75 0.869226987360353
1.75 1.71617263394592
1.75 0.883583757334196
1.75 1.52691489891608
1.75 0.985815641434352
1.75 0.622826878618383
1.75 0.763450444278588
1.75 0.97016577248476
1.75 1.01623534988749
1.75 0.739532895847734
1.75 1.30649864429524
1.75 0.738054501352472
1.75 0.633625567074838
1.75 1.89467765559472
1.75 0.424264068711929
1.75 0.724493748554388
1.75 0.659609807601865
1.75 1.0295630140987
1.75 1.36829171678492
1.75 1.33333333333333
};
\addplot [line width=1.0pt, black, opacity=1, forget plot]
table {%
2.71 0.0707106781186548
2.79 0.0707106781186548
2.79 0.485121287580844
2.71 0.485121287580844
2.71 0.0707106781186548
};
\addplot [line width=1.0pt, black, opacity=1, forget plot]
table {%
2.75 0.0707106781186548
2.75 0
};
\addplot [line width=1.0pt, black, opacity=1, forget plot]
table {%
2.75 0.485121287580844
2.75 1.07648945844305
};
\addplot [line width=1.0pt, black, forget plot]
table {%
2.73 0
2.77 0
};
\addplot [line width=1.0pt, black, forget plot]
table {%
2.73 1.07648945844305
2.77 1.07648945844305
};
\addplot [line width=1.0pt, black, opacity=0.2, mark=*, mark size=1, mark options={solid}, only marks, forget plot]
table {%
2.75 1.12541655099018
2.75 1.21604587685333
2.75 1.35873330119747
2.75 1.16018016555873
2.75 1.3348020272152
2.75 1.35814081650878
2.75 1.19171893311731
};
\addplot [line width=1.0pt, black, opacity=1, forget plot]
table {%
3.71 0.0809016994374947
3.79 0.0809016994374947
3.79 0.487994923610762
3.71 0.487994923610762
3.71 0.0809016994374947
};
\addplot [line width=1.0pt, black, opacity=1, forget plot]
table {%
3.75 0.0809016994374947
3.75 0
};
\addplot [line width=1.0pt, black, opacity=1, forget plot]
table {%
3.75 0.487994923610762
3.75 1.05643623705788
};
\addplot [line width=1.0pt, black, forget plot]
table {%
3.73 0
3.77 0
};
\addplot [line width=1.0pt, black, forget plot]
table {%
3.73 1.05643623705788
3.77 1.05643623705788
};
\addplot [line width=1.0pt, black, opacity=0.2, mark=*, mark size=1, mark options={solid}, only marks, forget plot]
table {%
3.75 1.23478896316371
3.75 1.42608938005497
3.75 1.15619235781782
3.75 1.28898363856257
3.75 1.14717412576052
3.75 1.73148433964139
3.75 1.50173443984357
3.75 1.17663204947805
3.75 1.13035227821223
};
\addplot [line width=1.0pt, black, opacity=1, forget plot]
table {%
4.71 0.0912570384968223
4.79 0.0912570384968223
4.79 0.557056691605868
4.71 0.557056691605868
4.71 0.0912570384968223
};
\addplot [line width=1.0pt, black, opacity=1, forget plot]
table {%
4.75 0.0912570384968223
4.75 0
};
\addplot [line width=1.0pt, black, opacity=1, forget plot]
table {%
4.75 0.557056691605868
4.75 1.24765875723692
};
\addplot [line width=1.0pt, black, forget plot]
table {%
4.73 0
4.77 0
};
\addplot [line width=1.0pt, black, forget plot]
table {%
4.73 1.24765875723692
4.77 1.24765875723692
};
\addplot [line width=1.0pt, black, opacity=0.2, mark=*, mark size=1, mark options={solid}, only marks, forget plot]
table {%
4.75 2.24801531865128
4.75 1.29508771254957
4.75 1.40582988604975
4.75 1.3524962799298
};
\addplot [line width=1.0pt, black, opacity=1, forget plot]
table {%
5.71 0.109241929153869
5.79 0.109241929153869
5.79 0.42630019628071
5.71 0.42630019628071
5.71 0.109241929153869
};
\addplot [line width=1.0pt, black, opacity=1, forget plot]
table {%
5.75 0.109241929153869
5.75 0
};
\addplot [line width=1.0pt, black, opacity=1, forget plot]
table {%
5.75 0.42630019628071
5.75 0.885828603018025
};
\addplot [line width=1.0pt, black, forget plot]
table {%
5.73 0
5.77 0
};
\addplot [line width=1.0pt, black, forget plot]
table {%
5.73 0.885828603018025
5.77 0.885828603018025
};
\addplot [line width=1.0pt, black, opacity=0.2, mark=*, mark size=1, mark options={solid}, only marks, forget plot]
table {%
5.75 1.01493333816588
5.75 0.979358822164515
5.75 1.34168132653353
5.75 1.73133773184488
5.75 1.14451604405345
5.75 0.967452923129838
5.75 1.03305840397518
5.75 1.92043358301823
5.75 1.39156207689245
5.75 1.03160494507209
};
\addplot [line width=1.0pt, color0, opacity=1, forget plot]
table {%
0.835 0
0.915 0
0.915 0.152028470752105
0.835 0.152028470752105
0.835 0
};
\addplot [line width=1.0pt, color0, opacity=1, forget plot]
table {%
0.875 0
0.875 0
};
\addplot [line width=1.0pt, color0, opacity=1, forget plot]
table {%
0.875 0.152028470752105
0.875 0.35
};
\addplot [line width=1.0pt, color0, forget plot]
table {%
0.855 0
0.895 0
};
\addplot [line width=1.0pt, color0, forget plot]
table {%
0.855 0.35
0.895 0.35
};
\addplot [line width=1.0pt, color0, opacity=0.2, mark=*, mark size=1, mark options={solid}, only marks, forget plot]
table {%
0.875 1.55489036784117
0.875 0.626719685164906
0.875 0.85
0.875 0.893627323681879
0.875 0.390512483795333
0.875 0.833229150767059
0.875 0.403112887414928
0.875 0.430116263352131
0.875 0.611803398874989
0.875 0.480116263352131
0.875 1.3349741126024
0.875 0.403350993617255
0.875 0.386432845054083
0.875 0.860002900817871
0.875 0.641254786637844
0.875 0.463120276247819
0.875 0.98183854086727
0.875 1.25104121494643
};
\addplot [line width=1.0pt, color0, opacity=1, forget plot]
table {%
1.835 0.0333333333333332
1.915 0.0333333333333332
1.915 0.173933314896997
1.835 0.173933314896997
1.835 0.0333333333333332
};
\addplot [line width=1.0pt, color0, opacity=1, forget plot]
table {%
1.875 0.0333333333333332
1.875 0
};
\addplot [line width=1.0pt, color0, opacity=1, forget plot]
table {%
1.875 0.173933314896997
1.875 0.360555127546399
};
\addplot [line width=1.0pt, color0, forget plot]
table {%
1.855 0
1.895 0
};
\addplot [line width=1.0pt, color0, forget plot]
table {%
1.855 0.360555127546399
1.895 0.360555127546399
};
\addplot [line width=1.0pt, color0, opacity=0.2, mark=*, mark size=1, mark options={solid}, only marks, forget plot]
table {%
1.875 0.733333333333334
1.875 0.484609027603543
1.875 0.504424865014052
1.875 0.944314448278322
1.875 0.9
1.875 0.840590735478528
1.875 1.47346907368669
1.875 0.765305030027194
1.875 0.394961926726537
1.875 1.50151436970302
1.875 0.843642639007922
1.875 0.77819752539448
1.875 0.758478416872766
1.875 0.682915855895474
1.875 0.944717052842777
1.875 0.422334970318546
1.875 0.861031729828177
1.875 0.86932908026563
1.875 0.60929401986999
1.875 0.946957754476948
};
\addplot [line width=1.0pt, color0, opacity=1, forget plot]
table {%
2.835 0.05
2.915 0.05
2.915 0.238703965836182
2.835 0.238703965836182
2.835 0.05
};
\addplot [line width=1.0pt, color0, opacity=1, forget plot]
table {%
2.875 0.05
2.875 0
};
\addplot [line width=1.0pt, color0, opacity=1, forget plot]
table {%
2.875 0.238703965836182
2.875 0.521404587551653
};
\addplot [line width=1.0pt, color0, forget plot]
table {%
2.855 0
2.895 0
};
\addplot [line width=1.0pt, color0, forget plot]
table {%
2.855 0.521404587551653
2.895 0.521404587551653
};
\addplot [line width=1.0pt, color0, opacity=0.2, mark=*, mark size=1, mark options={solid}, only marks, forget plot]
table {%
2.875 0.839566957129123
2.875 0.835266525331879
2.875 0.91251766865654
2.875 1.06451547803822
2.875 0.65
2.875 0.632696656665011
2.875 0.585113528717183
2.875 0.928026882645764
2.875 0.817542528065281
2.875 0.581453393965167
2.875 1.00581017278513
2.875 0.879217571850369
2.875 0.632949445058608
2.875 0.70603845890534
2.875 0.619243216200235
2.875 0.5680770730829
2.875 1.44361723063619
2.875 1.47849754451585
2.875 0.907535669162405
2.875 1.00694184892053
2.875 0.962499548866116
2.875 0.53033008588991
2.875 1.5364339949832
2.875 1.1898944078848
2.875 0.631485167480651
2.875 0.63978150704935
2.875 0.552268050859363
2.875 0.985272058492637
2.875 0.621182982939793
2.875 0.780971473206198
2.875 0.605463931666567
2.875 0.895200891291831
2.875 1.43628722492005
2.875 0.840017482303755
2.875 0.733383200720534
};
\addplot [line width=1.0pt, color0, opacity=1, forget plot]
table {%
3.835 0.0500000000000002
3.915 0.0500000000000002
3.915 0.506133256749339
3.835 0.506133256749339
3.835 0.0500000000000002
};
\addplot [line width=1.0pt, color0, opacity=1, forget plot]
table {%
3.875 0.0500000000000002
3.875 0
};
\addplot [line width=1.0pt, color0, opacity=1, forget plot]
table {%
3.875 0.506133256749339
3.875 1.18819127855732
};
\addplot [line width=1.0pt, color0, forget plot]
table {%
3.855 0
3.895 0
};
\addplot [line width=1.0pt, color0, forget plot]
table {%
3.855 1.18819127855732
3.895 1.18819127855732
};
\addplot [line width=1.0pt, color0, opacity=0.2, mark=*, mark size=1, mark options={solid}, only marks, forget plot]
table {%
3.875 1.58904502333085
3.875 1.19877968697179
3.875 1.23019306268829
3.875 1.20029757688
3.875 2.20458458731162
3.875 1.28719686396591
3.875 1.21542692499621
};
\addplot [line width=1.0pt, color0, opacity=1, forget plot]
table {%
4.835 0.0901387818865997
4.915 0.0901387818865997
4.915 0.552739482038997
4.835 0.552739482038997
4.835 0.0901387818865997
};
\addplot [line width=1.0pt, color0, opacity=1, forget plot]
table {%
4.875 0.0901387818865997
4.875 0
};
\addplot [line width=1.0pt, color0, opacity=1, forget plot]
table {%
4.875 0.552739482038997
4.875 1.22736113858953
};
\addplot [line width=1.0pt, color0, forget plot]
table {%
4.855 0
4.895 0
};
\addplot [line width=1.0pt, color0, forget plot]
table {%
4.855 1.22736113858953
4.895 1.22736113858953
};
\addplot [line width=1.0pt, color0, opacity=0.2, mark=*, mark size=1, mark options={solid}, only marks, forget plot]
table {%
4.875 1.38508122019336
4.875 1.46211699256003
4.875 1.43363956135355
4.875 1.45328026361052
4.875 1.31794499278532
4.875 1.41875343506978
4.875 1.28506832866748
4.875 1.30377155007344
4.875 1.53380112858296
};
\addplot [line width=1.0pt, color0, opacity=1, forget plot]
table {%
5.835 0.102308230480331
5.915 0.102308230480331
5.915 0.415552387633531
5.835 0.415552387633531
5.835 0.102308230480331
};
\addplot [line width=1.0pt, color0, opacity=1, forget plot]
table {%
5.875 0.102308230480331
5.875 0
};
\addplot [line width=1.0pt, color0, opacity=1, forget plot]
table {%
5.875 0.415552387633531
5.875 0.882082567546926
};
\addplot [line width=1.0pt, color0, forget plot]
table {%
5.855 0
5.895 0
};
\addplot [line width=1.0pt, color0, forget plot]
table {%
5.855 0.882082567546926
5.895 0.882082567546926
};
\addplot [line width=1.0pt, color0, opacity=0.2, mark=*, mark size=1, mark options={solid}, only marks, forget plot]
table {%
5.875 1.35067043468208
5.875 1.20131234281707
5.875 1.11319506271304
5.875 1.62008447907126
5.875 1.30904194149184
5.875 1.04508076270126
5.875 0.900585284672828
5.875 0.955512549521321
5.875 1.48507425101145
5.875 0.964114847653933
5.875 0.982026156685062
5.875 0.896906223443158
5.875 0.961164507581587
};
\addplot [line width=1.0pt, color1, opacity=1, forget plot]
table {%
0.96 0
1.04 0
1.04 0.1802775637732
0.96 0.1802775637732
0.96 0
};
\addplot [line width=1.0pt, color1, opacity=1, forget plot]
table {%
1 0
1 0
};
\addplot [line width=1.0pt, color1, opacity=1, forget plot]
table {%
1 0.1802775637732
1 0.45
};
\addplot [line width=1.0pt, color1, forget plot]
table {%
0.98 0
1.02 0
};
\addplot [line width=1.0pt, color1, forget plot]
table {%
0.98 0.45
1.02 0.45
};
\addplot [line width=1.0pt, color1, opacity=0.2, mark=*, mark size=1, mark options={solid}, only marks, forget plot]
table {%
1 0.782623792124927
1 0.700944867164838
1 1.39074855636306
1 0.47169905660283
1 0.673709602464916
1 1.16843559812366
1 2.65092747592052
1 0.56315356820846
1 1.43427118417164
1 0.611703806613907
1 0.478270094880062
1 0.629812939611249
1 1.37077123007641
1 0.475413521637608
1 0.515154392492244
1 0.540832691319598
1 0.48441569028811
1 0.703166210152331
1 1.24595745795604
};
\addplot [line width=1.0pt, color1, opacity=1, forget plot]
table {%
1.96 0.0333333333333334
2.04 0.0333333333333334
2.04 0.299125612319133
1.96 0.299125612319133
1.96 0.0333333333333334
};
\addplot [line width=1.0pt, color1, opacity=1, forget plot]
table {%
2 0.0333333333333334
2 0
};
\addplot [line width=1.0pt, color1, opacity=1, forget plot]
table {%
2 0.299125612319133
2 0.697474812267084
};
\addplot [line width=1.0pt, color1, forget plot]
table {%
1.98 0
2.02 0
};
\addplot [line width=1.0pt, color1, forget plot]
table {%
1.98 0.697474812267084
2.02 0.697474812267084
};
\addplot [line width=1.0pt, color1, opacity=0.2, mark=*, mark size=1, mark options={solid}, only marks, forget plot]
table {%
2 1.45116773595546
2 0.874964360773976
2 1.25626242185628
2 1.36409614651092
2 0.760700498867611
2 0.953060205168264
2 0.748839931403053
2 1.0704232790736
2 0.76955975870768
2 1.11001876932486
2 1.04695987005105
2 0.86932908026563
2 0.801387685344754
2 0.736357401145817
2 0.755724679717565
2 1.42700552240669
2 1.23297963869285
2 1.2833398076254
2 0.845053044125728
2 1.33259258191769
2 1.06785172193445
2 1.43397127342004
2 1.28093298825663
2 1.89343674791716
2 0.984358536847072
};
\addplot [line width=1.0pt, color1, opacity=1, forget plot]
table {%
2.96 0.103812116288267
3.04 0.103812116288267
3.04 0.540975730697202
2.96 0.540975730697202
2.96 0.103812116288267
};
\addplot [line width=1.0pt, color1, opacity=1, forget plot]
table {%
3 0.103812116288267
3 0
};
\addplot [line width=1.0pt, color1, opacity=1, forget plot]
table {%
3 0.540975730697202
3 1.18296402795114
};
\addplot [line width=1.0pt, color1, forget plot]
table {%
2.98 0
3.02 0
};
\addplot [line width=1.0pt, color1, forget plot]
table {%
2.98 1.18296402795114
3.02 1.18296402795114
};
\addplot [line width=1.0pt, color1, opacity=0.2, mark=*, mark size=1, mark options={solid}, only marks, forget plot]
table {%
3 1.24208011817632
3 1.36696457789838
3 1.22033041573856
3 1.1994784519324
3 1.4386069465465
3 1.83873258290054
3 1.51892796736401
3 1.55137884797332
};
\addplot [line width=1.0pt, color1, opacity=1, forget plot]
table {%
3.96 0.133738344428268
4.04 0.133738344428268
4.04 0.625510655535147
3.96 0.625510655535147
3.96 0.133738344428268
};
\addplot [line width=1.0pt, color1, opacity=1, forget plot]
table {%
4 0.133738344428268
4 0
};
\addplot [line width=1.0pt, color1, opacity=1, forget plot]
table {%
4 0.625510655535147
4 1.28045346311218
};
\addplot [line width=1.0pt, color1, forget plot]
table {%
3.98 0
4.02 0
};
\addplot [line width=1.0pt, color1, forget plot]
table {%
3.98 1.28045346311218
4.02 1.28045346311218
};
\addplot [line width=1.0pt, color1, opacity=0.2, mark=*, mark size=1, mark options={solid}, only marks, forget plot]
table {%
4 1.401387818866
4 1.3648800980282
4 1.41195776723543
4 1.51686285300102
};
\addplot [line width=1.0pt, color1, opacity=1, forget plot]
table {%
4.96 0.141257038496822
5.04 0.141257038496822
5.04 0.678890446697227
4.96 0.678890446697227
4.96 0.141257038496822
};
\addplot [line width=1.0pt, color1, opacity=1, forget plot]
table {%
5 0.141257038496822
5 0
};
\addplot [line width=1.0pt, color1, opacity=1, forget plot]
table {%
5 0.678890446697227
5 1.46513692947726
};
\addplot [line width=1.0pt, color1, forget plot]
table {%
4.98 0
5.02 0
};
\addplot [line width=1.0pt, color1, forget plot]
table {%
4.98 1.46513692947726
5.02 1.46513692947726
};
\addplot [line width=1.0pt, color1, opacity=0.2, mark=*, mark size=1, mark options={solid}, only marks, forget plot]
table {%
5 1.83979309728567
5 1.60892371030648
5 1.65870747229115
};
\addplot [line width=1.0pt, color1, opacity=1, forget plot]
table {%
5.96 0.175685110880123
6.04 0.175685110880123
6.04 0.58341542912783
5.96 0.58341542912783
5.96 0.175685110880123
};
\addplot [line width=1.0pt, color1, opacity=1, forget plot]
table {%
6 0.175685110880123
6 0
};
\addplot [line width=1.0pt, color1, opacity=1, forget plot]
table {%
6 0.58341542912783
6 1.18942960039246
};
\addplot [line width=1.0pt, color1, forget plot]
table {%
5.98 0
6.02 0
};
\addplot [line width=1.0pt, color1, forget plot]
table {%
5.98 1.18942960039246
6.02 1.18942960039246
};
\addplot [line width=1.0pt, color1, opacity=0.2, mark=*, mark size=1, mark options={solid}, only marks, forget plot]
table {%
6 1.34781499080887
6 1.6478486288308
6 1.37336789804748
6 1.37662151277181
6 1.37868933526156
};
\addplot [line width=1.0pt, color2, opacity=1, forget plot]
table {%
1.085 0.0499999999999998
1.165 0.0499999999999998
1.165 0.208113883008418
1.085 0.208113883008418
1.085 0.0499999999999998
};
\addplot [line width=1.0pt, color2, opacity=1, forget plot]
table {%
1.125 0.0499999999999998
1.125 0
};
\addplot [line width=1.0pt, color2, opacity=1, forget plot]
table {%
1.125 0.208113883008418
1.125 0.4159415253899
};
\addplot [line width=1.0pt, color2, forget plot]
table {%
1.105 0
1.145 0
};
\addplot [line width=1.0pt, color2, forget plot]
table {%
1.105 0.4159415253899
1.145 0.4159415253899
};
\addplot [line width=1.0pt, color2, opacity=0.2, mark=*, mark size=1, mark options={solid}, only marks, forget plot]
table {%
1.125 0.500826941470786
1.125 0.636396103067893
1.125 0.514916286289917
1.125 0.531933840032642
1.125 0.860988903191241
1.125 0.581507290636732
1.125 0.540832691319599
1.125 0.519258240356725
1.125 2.2277624388087
1.125 0.951387818865997
1.125 1.29713035798796
1.125 0.474264068711928
1.125 0.502315882670322
1.125 0.596667765076216
1.125 1.5635676215701
1.125 1.05533683297418
1.125 0.532288246227607
1.125 0.955523914828305
1.125 1.05446891902851
1.125 0.62658490592434
};
\addplot [line width=1.0pt, color2, opacity=1, forget plot]
table {%
2.085 0.0666666666666664
2.165 0.0666666666666664
2.165 0.503334722269288
2.085 0.503334722269288
2.085 0.0666666666666664
};
\addplot [line width=1.0pt, color2, opacity=1, forget plot]
table {%
2.125 0.0666666666666664
2.125 0
};
\addplot [line width=1.0pt, color2, opacity=1, forget plot]
table {%
2.125 0.503334722269288
2.125 1.1488949080401
};
\addplot [line width=1.0pt, color2, forget plot]
table {%
2.105 0
2.145 0
};
\addplot [line width=1.0pt, color2, forget plot]
table {%
2.105 1.1488949080401
2.145 1.1488949080401
};
\addplot [line width=1.0pt, color2, opacity=0.2, mark=*, mark size=1, mark options={solid}, only marks, forget plot]
table {%
2.125 1.61587405711317
2.125 1.38124792278356
2.125 1.23881887902153
2.125 1.2206767340727
2.125 1.64119973392618
2.125 1.30990232734526
2.125 1.21278199839203
2.125 1.21219540083848
2.125 1.30049507812115
2.125 1.20053749461806
2.125 1.63580608896439
2.125 1.51592902043004
2.125 1.69807478473819
};
\addplot [line width=1.0pt, color2, opacity=1, forget plot]
table {%
3.085 0.0999999999999999
3.165 0.0999999999999999
3.165 0.725234818582267
3.085 0.725234818582267
3.085 0.0999999999999999
};
\addplot [line width=1.0pt, color2, opacity=1, forget plot]
table {%
3.125 0.0999999999999999
3.125 0
};
\addplot [line width=1.0pt, color2, opacity=1, forget plot]
table {%
3.125 0.725234818582267
3.125 1.65754649990307
};
\addplot [line width=1.0pt, color2, forget plot]
table {%
3.105 0
3.145 0
};
\addplot [line width=1.0pt, color2, forget plot]
table {%
3.105 1.65754649990307
3.145 1.65754649990307
};
\addplot [line width=1.0pt, color2, opacity=1, forget plot]
table {%
4.085 0.123927669529664
4.165 0.123927669529664
4.165 0.720963745926263
4.085 0.720963745926263
4.085 0.123927669529664
};
\addplot [line width=1.0pt, color2, opacity=1, forget plot]
table {%
4.125 0.123927669529664
4.125 0
};
\addplot [line width=1.0pt, color2, opacity=1, forget plot]
table {%
4.125 0.720963745926263
4.125 1.6011083310896
};
\addplot [line width=1.0pt, color2, forget plot]
table {%
4.105 0
4.145 0
};
\addplot [line width=1.0pt, color2, forget plot]
table {%
4.105 1.6011083310896
4.145 1.6011083310896
};
\addplot [line width=1.0pt, color2, opacity=0.2, mark=*, mark size=1, mark options={solid}, only marks, forget plot]
table {%
4.125 1.71234840706734
};
\addplot [line width=1.0pt, color2, opacity=1, forget plot]
table {%
5.085 0.163398045187485
5.165 0.163398045187485
5.165 0.667925470398605
5.085 0.667925470398605
5.085 0.163398045187485
};
\addplot [line width=1.0pt, color2, opacity=1, forget plot]
table {%
5.125 0.163398045187485
5.125 0.025
};
\addplot [line width=1.0pt, color2, opacity=1, forget plot]
table {%
5.125 0.667925470398605
5.125 1.36193166608039
};
\addplot [line width=1.0pt, color2, forget plot]
table {%
5.105 0.025
5.145 0.025
};
\addplot [line width=1.0pt, color2, forget plot]
table {%
5.105 1.36193166608039
5.145 1.36193166608039
};
\addplot [line width=1.0pt, color2, opacity=0.2, mark=*, mark size=1, mark options={solid}, only marks, forget plot]
table {%
5.125 1.44409599165785
5.125 1.61404853671259
5.125 1.744901183567
5.125 1.60137667125031
5.125 1.44361678647066
5.125 1.47724798757639
5.125 1.81409651876217
};
\addplot [line width=1.0pt, color2, opacity=1, forget plot]
table {%
6.085 0.195124014594064
6.165 0.195124014594064
6.165 0.641042224133932
6.085 0.641042224133932
6.085 0.195124014594064
};
\addplot [line width=1.0pt, color2, opacity=1, forget plot]
table {%
6.125 0.195124014594064
6.125 0.0249999999999999
};
\addplot [line width=1.0pt, color2, opacity=1, forget plot]
table {%
6.125 0.641042224133932
6.125 1.30581750186973
};
\addplot [line width=1.0pt, color2, forget plot]
table {%
6.105 0.0249999999999999
6.145 0.0249999999999999
};
\addplot [line width=1.0pt, color2, forget plot]
table {%
6.105 1.30581750186973
6.145 1.30581750186973
};
\addplot [line width=1.0pt, color2, opacity=0.2, mark=*, mark size=1, mark options={solid}, only marks, forget plot]
table {%
6.125 2.00168145841845
6.125 1.67364495093241
6.125 1.75372998498909
};
\addplot [line width=1.0pt, color3, opacity=1, forget plot]
table {%
1.21 0.0499999999999998
1.29 0.0499999999999998
1.29 0.210518160398157
1.21 0.210518160398157
1.21 0.0499999999999998
};
\addplot [line width=1.0pt, color3, opacity=1, forget plot]
table {%
1.25 0.0499999999999998
1.25 0
};
\addplot [line width=1.0pt, color3, opacity=1, forget plot]
table {%
1.25 0.210518160398157
1.25 0.420156211871642
};
\addplot [line width=1.0pt, color3, forget plot]
table {%
1.23 0
1.27 0
};
\addplot [line width=1.0pt, color3, forget plot]
table {%
1.23 0.420156211871642
1.27 0.420156211871642
};
\addplot [line width=1.0pt, color3, opacity=0.2, mark=*, mark size=1, mark options={solid}, only marks, forget plot]
table {%
1.25 0.716049687839784
1.25 0.640512483795333
1.25 1.3247042985171
1.25 1.15108644332213
1.25 0.58851648071345
1.25 0.681891111424558
1.25 0.52169905660283
1.25 0.667132504145527
1.25 0.480116263352131
1.25 1.35231123991628
1.25 0.721663858094396
1.25 1.30108644332213
1.25 0.611803398874989
1.25 0.815891053163818
1.25 0.635234995535981
1.25 0.657691517108185
1.25 0.60926816869581
};
\addplot [line width=1.0pt, color3, opacity=1, forget plot]
table {%
2.21 0.0789892388718256
2.29 0.0789892388718256
2.29 0.685012267895355
2.21 0.685012267895355
2.21 0.0789892388718256
};
\addplot [line width=1.0pt, color3, opacity=1, forget plot]
table {%
2.25 0.0789892388718256
2.25 0
};
\addplot [line width=1.0pt, color3, opacity=1, forget plot]
table {%
2.25 0.685012267895355
2.25 1.45144805324697
};
\addplot [line width=1.0pt, color3, forget plot]
table {%
2.23 0
2.27 0
};
\addplot [line width=1.0pt, color3, forget plot]
table {%
2.23 1.45144805324697
2.27 1.45144805324697
};
\addplot [line width=1.0pt, color3, opacity=0.2, mark=*, mark size=1, mark options={solid}, only marks, forget plot]
table {%
2.25 1.9524825658591
2.25 1.68115594604447
2.25 1.60398271348484
2.25 1.63381845336224
2.25 1.73819918350196
2.25 2.03229391031323
2.25 2.01801462313269
};
\addplot [line width=1.0pt, color3, opacity=1, forget plot]
table {%
3.21 0.160327974015615
3.29 0.160327974015615
3.29 0.849752444704236
3.21 0.849752444704236
3.21 0.160327974015615
};
\addplot [line width=1.0pt, color3, opacity=1, forget plot]
table {%
3.25 0.160327974015615
3.25 0.0353553390593273
};
\addplot [line width=1.0pt, color3, opacity=1, forget plot]
table {%
3.25 0.849752444704236
3.25 1.72872914930157
};
\addplot [line width=1.0pt, color3, forget plot]
table {%
3.23 0.0353553390593273
3.27 0.0353553390593273
};
\addplot [line width=1.0pt, color3, forget plot]
table {%
3.23 1.72872914930157
3.27 1.72872914930157
};
\addplot [line width=1.0pt, color3, opacity=0.2, mark=*, mark size=1, mark options={solid}, only marks, forget plot]
table {%
3.25 1.91018955208292
3.25 1.93821623079755
3.25 2.04966460018711
};
\addplot [line width=1.0pt, color3, opacity=1, forget plot]
table {%
4.21 0.264866253099674
4.29 0.264866253099674
4.29 0.881246361164786
4.21 0.881246361164786
4.21 0.264866253099674
};
\addplot [line width=1.0pt, color3, opacity=1, forget plot]
table {%
4.25 0.264866253099674
4.25 0.0249999999999999
};
\addplot [line width=1.0pt, color3, opacity=1, forget plot]
table {%
4.25 0.881246361164786
4.25 1.53072476590464
};
\addplot [line width=1.0pt, color3, forget plot]
table {%
4.23 0.0249999999999999
4.27 0.0249999999999999
};
\addplot [line width=1.0pt, color3, forget plot]
table {%
4.23 1.53072476590464
4.27 1.53072476590464
};
\addplot [line width=1.0pt, color3, opacity=0.2, mark=*, mark size=1, mark options={solid}, only marks, forget plot]
table {%
4.25 1.95252611005012
};
\addplot [line width=1.0pt, color3, opacity=1, forget plot]
table {%
5.21 0.237296737670565
5.29 0.237296737670565
5.29 0.826151369172303
5.21 0.826151369172303
5.21 0.237296737670565
};
\addplot [line width=1.0pt, color3, opacity=1, forget plot]
table {%
5.25 0.237296737670565
5.25 0.025
};
\addplot [line width=1.0pt, color3, opacity=1, forget plot]
table {%
5.25 0.826151369172303
5.25 1.60709495971588
};
\addplot [line width=1.0pt, color3, forget plot]
table {%
5.23 0.025
5.27 0.025
};
\addplot [line width=1.0pt, color3, forget plot]
table {%
5.23 1.60709495971588
5.27 1.60709495971588
};
\addplot [line width=1.0pt, color3, opacity=0.2, mark=*, mark size=1, mark options={solid}, only marks, forget plot]
table {%
5.25 1.8344234231239
};
\addplot [line width=1.0pt, color3, opacity=1, forget plot]
table {%
6.21 0.261628800597555
6.29 0.261628800597555
6.29 0.711026448033977
6.21 0.711026448033977
6.21 0.261628800597555
};
\addplot [line width=1.0pt, color3, opacity=1, forget plot]
table {%
6.25 0.261628800597555
6.25 0.025
};
\addplot [line width=1.0pt, color3, opacity=1, forget plot]
table {%
6.25 0.711026448033977
6.25 1.344602727156
};
\addplot [line width=1.0pt, color3, forget plot]
table {%
6.23 0.025
6.27 0.025
};
\addplot [line width=1.0pt, color3, forget plot]
table {%
6.23 1.344602727156
6.27 1.344602727156
};
\addplot [line width=1.0pt, color3, opacity=0.2, mark=*, mark size=1, mark options={solid}, only marks, forget plot]
table {%
6.25 2.00720685572947
6.25 1.46655012461818
6.25 1.75965930305283
6.25 1.7033774087694
};
\addplot [line width=1.0pt, black, opacity=1, forget plot]
table {%
0.71 0.0499999999999998
0.79 0.0499999999999998
};
\addplot [line width=1.0pt, black, dashed, mark=x, mark size=3, mark options={solid}, forget plot]
table {%
0.75 0.103201918911421
};
\addplot [line width=1.0pt, black, opacity=1, forget plot]
table {%
1.71 0.0666666666666667
1.79 0.0666666666666667
};
\addplot [line width=1.0pt, black, dashed, mark=x, mark size=3, mark options={solid}, forget plot]
table {%
1.75 0.192914525605671
};
\addplot [line width=1.0pt, black, opacity=1, forget plot]
table {%
2.71 0.154304001977173
2.79 0.154304001977173
};
\addplot [line width=1.0pt, black, dashed, mark=x, mark size=3, mark options={solid}, forget plot]
table {%
2.75 0.303630724385165
};
\addplot [line width=1.0pt, black, opacity=1, forget plot]
table {%
3.71 0.193427400616229
3.79 0.193427400616229
};
\addplot [line width=1.0pt, black, dashed, mark=x, mark size=3, mark options={solid}, forget plot]
table {%
3.75 0.327065521730923
};
\addplot [line width=1.0pt, black, opacity=1, forget plot]
table {%
4.71 0.199613900595886
4.79 0.199613900595886
};
\addplot [line width=1.0pt, black, dashed, mark=x, mark size=3, mark options={solid}, forget plot]
table {%
4.75 0.355194351174054
};
\addplot [line width=1.0pt, black, opacity=1, forget plot]
table {%
5.71 0.193087160190115
5.79 0.193087160190115
};
\addplot [line width=1.0pt, black, dashed, mark=x, mark size=3, mark options={solid}, forget plot]
table {%
5.75 0.308500525658535
};
\addplot [line width=1.0pt, color0, opacity=1, forget plot]
table {%
0.835 0.0499999999999998
0.915 0.0499999999999998
};
\addplot [line width=1.0pt, color0, dashed, mark=x, mark size=3, mark options={solid}, forget plot]
table {%
0.875 0.125017110619512
};
\addplot [line width=1.0pt, color0, opacity=1, forget plot]
table {%
1.835 0.0666666666666667
1.915 0.0666666666666667
};
\addplot [line width=1.0pt, color0, dashed, mark=x, mark size=3, mark options={solid}, forget plot]
table {%
1.875 0.155881722628191
};
\addplot [line width=1.0pt, color0, opacity=1, forget plot]
table {%
2.835 0.103077640640442
2.915 0.103077640640442
};
\addplot [line width=1.0pt, color0, dashed, mark=x, mark size=3, mark options={solid}, forget plot]
table {%
2.875 0.24038527794314
};
\addplot [line width=1.0pt, color0, opacity=1, forget plot]
table {%
3.835 0.144953200754732
3.915 0.144953200754732
};
\addplot [line width=1.0pt, color0, dashed, mark=x, mark size=3, mark options={solid}, forget plot]
table {%
3.875 0.320915049808084
};
\addplot [line width=1.0pt, color0, opacity=1, forget plot]
table {%
4.835 0.189626091568228
4.915 0.189626091568228
};
\addplot [line width=1.0pt, color0, dashed, mark=x, mark size=3, mark options={solid}, forget plot]
table {%
4.875 0.358746074362444
};
\addplot [line width=1.0pt, color0, opacity=1, forget plot]
table {%
5.835 0.212501999098811
5.915 0.212501999098811
};
\addplot [line width=1.0pt, color0, dashed, mark=x, mark size=3, mark options={solid}, forget plot]
table {%
5.875 0.317157174802269
};
\addplot [line width=1.0pt, color1, opacity=1, forget plot]
table {%
0.96 0.0500000000000002
1.04 0.0500000000000002
};
\addplot [line width=1.0pt, color1, dashed, mark=x, mark size=3, mark options={solid}, forget plot]
table {%
1 0.162540128816541
};
\addplot [line width=1.0pt, color1, opacity=1, forget plot]
table {%
1.96 0.106639093961136
2.04 0.106639093961136
};
\addplot [line width=1.0pt, color1, dashed, mark=x, mark size=3, mark options={solid}, forget plot]
table {%
2 0.254182143160617
};
\addplot [line width=1.0pt, color1, opacity=1, forget plot]
table {%
2.96 0.225266902835433
3.04 0.225266902835433
};
\addplot [line width=1.0pt, color1, dashed, mark=x, mark size=3, mark options={solid}, forget plot]
table {%
3 0.362879051910614
};
\addplot [line width=1.0pt, color1, opacity=1, forget plot]
table {%
3.96 0.344423225217581
4.04 0.344423225217581
};
\addplot [line width=1.0pt, color1, dashed, mark=x, mark size=3, mark options={solid}, forget plot]
table {%
4 0.436274313295025
};
\addplot [line width=1.0pt, color1, opacity=1, forget plot]
table {%
4.96 0.332337044714244
5.04 0.332337044714244
};
\addplot [line width=1.0pt, color1, dashed, mark=x, mark size=3, mark options={solid}, forget plot]
table {%
5 0.458997366046748
};
\addplot [line width=1.0pt, color1, opacity=1, forget plot]
table {%
5.96 0.348925998193767
6.04 0.348925998193767
};
\addplot [line width=1.0pt, color1, dashed, mark=x, mark size=3, mark options={solid}, forget plot]
table {%
6 0.419435475198472
};
\addplot [line width=1.0pt, color2, opacity=1, forget plot]
table {%
1.085 0.0707106781186546
1.165 0.0707106781186546
};
\addplot [line width=1.0pt, color2, dashed, mark=x, mark size=3, mark options={solid}, forget plot]
table {%
1.125 0.16797653541589
};
\addplot [line width=1.0pt, color2, opacity=1, forget plot]
table {%
2.085 0.144280904158206
2.165 0.144280904158206
};
\addplot [line width=1.0pt, color2, dashed, mark=x, mark size=3, mark options={solid}, forget plot]
table {%
2.125 0.342733292505696
};
\addplot [line width=1.0pt, color2, opacity=1, forget plot]
table {%
3.085 0.275984957221428
3.165 0.275984957221428
};
\addplot [line width=1.0pt, color2, dashed, mark=x, mark size=3, mark options={solid}, forget plot]
table {%
3.125 0.450364275040593
};
\addplot [line width=1.0pt, color2, opacity=1, forget plot]
table {%
4.085 0.323808161121566
4.165 0.323808161121566
};
\addplot [line width=1.0pt, color2, dashed, mark=x, mark size=3, mark options={solid}, forget plot]
table {%
4.125 0.451371108769039
};
\addplot [line width=1.0pt, color2, opacity=1, forget plot]
table {%
5.085 0.364405052038868
5.165 0.364405052038868
};
\addplot [line width=1.0pt, color2, dashed, mark=x, mark size=3, mark options={solid}, forget plot]
table {%
5.125 0.467331612164226
};
\addplot [line width=1.0pt, color2, opacity=1, forget plot]
table {%
6.085 0.388973714660886
6.165 0.388973714660886
};
\addplot [line width=1.0pt, color2, dashed, mark=x, mark size=3, mark options={solid}, forget plot]
table {%
6.125 0.457655304538824
};
\addplot [line width=1.0pt, color3, opacity=1, forget plot]
table {%
1.21 0.070710678118655
1.29 0.070710678118655
};
\addplot [line width=1.0pt, color3, dashed, mark=x, mark size=3, mark options={solid}, forget plot]
table {%
1.25 0.16686116069469
};
\addplot [line width=1.0pt, color3, opacity=1, forget plot]
table {%
2.21 0.225652597122411
2.29 0.225652597122411
};
\addplot [line width=1.0pt, color3, dashed, mark=x, mark size=3, mark options={solid}, forget plot]
table {%
2.25 0.422213097133527
};
\addplot [line width=1.0pt, color3, opacity=1, forget plot]
table {%
3.21 0.480199650173389
3.29 0.480199650173389
};
\addplot [line width=1.0pt, color3, dashed, mark=x, mark size=3, mark options={solid}, forget plot]
table {%
3.25 0.563753100309146
};
\addplot [line width=1.0pt, color3, opacity=1, forget plot]
table {%
4.21 0.505708131133393
4.29 0.505708131133393
};
\addplot [line width=1.0pt, color3, dashed, mark=x, mark size=3, mark options={solid}, forget plot]
table {%
4.25 0.584617428377441
};
\addplot [line width=1.0pt, color3, opacity=1, forget plot]
table {%
5.21 0.459287943192255
5.29 0.459287943192255
};
\addplot [line width=1.0pt, color3, dashed, mark=x, mark size=3, mark options={solid}, forget plot]
table {%
5.25 0.557208047586385
};
\addplot [line width=1.0pt, color3, opacity=1, forget plot]
table {%
6.21 0.44726555164435
6.29 0.44726555164435
};
\addplot [line width=1.0pt, color3, dashed, mark=x, mark size=3, mark options={solid}, forget plot]
table {%
6.25 0.518532712134289
};
\end{axis}

\node at ({$(current bounding box.south west)!0.5!(current bounding box.south east)$}|-{$(current bounding box.south west)!0.98!(current bounding box.north west)$})[
  anchor=north,
  text=black,
  rotate=0.0
]{ };

	    \begin{customlegend}[
legend entries={no noise,SNR\ $=30$~dB,SNR\ $=15$~dB, SNR\ $=10$~dB, SNR\ $=5$~dB},
legend cell align=left,
legend style={at={(0.05,5.37)}, anchor=north west, draw=white!80.0!black, font=\footnotesize,fill opacity=0.5, draw opacity=1,text opacity=1}]
    \addlegendimage{area legend,black,fill=black, fill opacity=1}
    \addlegendimage{area legend,color0,fill=color0, fill opacity=1}
    \addlegendimage{area legend,color1,fill=color1, fill opacity=1}
    \addlegendimage{area legend,color2,fill=color2, fill opacity=1}
    \addlegendimage{area legend,color3,fill=color3, fill opacity=1}
\end{customlegend}
	\end{tikzpicture}

}
	\caption[Evaluation Results for Different Noise Levels]{Evaluation Results for Different Noise Levels ($n=200$).}
	\label{fig:trialSNR}
\end{figure}

In \autoref{fig:trialSNR} the results of the \gls{snr} evaluation are presented. These show that, from ${\text{SNR}=30~\text{dB}}$ onwards, adding more noise to the received signal increases the localisation error and therefore decreases localisation performance. However, comparing the trials without noise to those with the least amount of noise ${\text{SNR}=30~\text{dB}}$ suggests, that a minimal amount of noise does not always decrease localisation performance. In fact, for $S=3$ and $S=4$ the algorithm performs best with ${\text{SNR}=30~\text{dB}}$, marginally better than in a noiseless environment. For all other $S$, the results for minimal noise and no noise are about the same in regards to the variance, median and mean of the \gls{mae}.