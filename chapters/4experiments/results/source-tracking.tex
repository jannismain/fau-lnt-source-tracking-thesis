
\subsection{Source Tracking}

\subsubsection{Comparison of Variance Estimation}

% Variance Estimate over Time. Graph looks almost the same for all three scenarios, therefore showing only one scenario should be enough
\begin{figure}[H]
	\setlength\figureheight{6cm}
	\setlength\figurewidth{0.95\textwidth}
	\iftoggle{quick}{%
		\includegraphics[width=\textwidth]{plots/tracking/variance/crossing-variance-comparison.png}
	}{%
		% This file was created by matlab2tikz.
%
\definecolor{lms_red}{rgb}{0.80000,0.20784,0.21961}%
\definecolor{darkgray}{rgb}{0.8,0.8,0.8}%
%
\begin{tikzpicture}

\begin{axis}[%
width=0.951\figurewidth,
height=\figureheight,
at={(0\figurewidth,0\figureheight)},
scale only axis,
xmin=0,
xmax=497,
xtick={0,99.2,198.4,297.6,396.8,497},
xticklabels={{0},{1},{2},{3},{4},{5}},
xlabel style={font=\color{white!15!black}},
xlabel={$t$~[s]},
ymin=0,
ymax=5,
ylabel style={font=\color{white!15!black}},
ylabel={$\sigma^{2,\text{(t)}}$},
axis background/.style={fill=white},
xmajorgrids,
ymajorgrids,
legend style={legend cell align=left, align=left, draw=white!15!black}
]
\addplot [color=lms_red, line width=1.0pt]
  table[row sep=crcr]{%
1	0.1\\
2	0.201997591636175\\
3	0.312023156744886\\
4	0.38527660827563\\
5	0.426412211130185\\
6	0.455553447076937\\
7	0.481764224372865\\
8	0.487307162116875\\
9	0.503408989535439\\
10	0.533925251592931\\
11	0.583097183714533\\
12	0.655450103023131\\
13	0.719318553899198\\
14	0.765007414385179\\
15	0.790070952503562\\
16	0.81966455631174\\
17	0.851946635251246\\
18	0.882659757003189\\
19	0.900825128609396\\
20	0.897280737149013\\
21	0.918306513785199\\
22	0.942925346882032\\
23	0.96364028113376\\
24	0.974548986758836\\
25	1.00165010518299\\
26	1.03944553038585\\
27	1.06657096196127\\
28	1.08370993186571\\
29	1.10332149643417\\
30	1.12855694022905\\
31	1.13783979133569\\
32	1.15025315831826\\
33	1.15873345349527\\
34	1.1745742915909\\
35	1.17111492161293\\
36	1.16381855994909\\
37	1.16141138073843\\
38	1.15989096499095\\
39	1.15727250416835\\
40	1.17115744745779\\
41	1.19796689442519\\
42	1.20603040258331\\
43	1.221502278392\\
44	1.24311808327252\\
45	1.25330204977228\\
46	1.26108841474479\\
47	1.26166115717842\\
48	1.27197135648412\\
49	1.26963417598048\\
50	1.24339025964786\\
51	1.21142589536759\\
52	1.19550336561611\\
53	1.18243193377787\\
54	1.18773770708476\\
55	1.1864125459501\\
56	1.17970200063497\\
57	1.16932215401163\\
58	1.15760306564734\\
59	1.14562300735978\\
60	1.15108244006617\\
61	1.16080919299993\\
62	1.16381470345146\\
63	1.16801981070209\\
64	1.18737710998061\\
65	1.20227324002343\\
66	1.21710676533593\\
67	1.22564914148483\\
68	1.23012448988065\\
69	1.23654798771052\\
70	1.24979499831366\\
71	1.2670731455639\\
72	1.2758102691829\\
73	1.28292922932418\\
74	1.29773948608864\\
75	1.30991943810536\\
76	1.31420808786663\\
77	1.30931191231248\\
78	1.31021591811689\\
79	1.30826979510844\\
80	1.30195910010481\\
81	1.30833488714876\\
82	1.30458266685593\\
83	1.28336966414756\\
84	1.25529140760147\\
85	1.22081035951213\\
86	1.19039807229924\\
87	1.17080688884714\\
88	1.15485901472802\\
89	1.14950574921557\\
90	1.14542563810742\\
91	1.14032596108248\\
92	1.1351422297561\\
93	1.13668548203051\\
94	1.14451107379459\\
95	1.1538749405846\\
96	1.14944528609262\\
97	1.15714389132284\\
98	1.16590164905428\\
99	1.16823396403008\\
100	1.18194949546012\\
101	1.18836536898987\\
102	1.19964928027012\\
103	1.21453565784117\\
104	1.23472176297616\\
105	1.24966779619871\\
106	1.26120269520915\\
107	1.27080235610508\\
108	1.27849139681272\\
109	1.27404939177397\\
110	1.2680891202271\\
111	1.26916533191236\\
112	1.26530094702314\\
113	1.27217906636974\\
114	1.28881345926658\\
115	1.30182773519857\\
116	1.30455175976149\\
117	1.31496981496217\\
118	1.31557641785732\\
119	1.30968388473591\\
120	1.31759482703798\\
121	1.31975802294296\\
122	1.32247912393355\\
123	1.32130529855948\\
124	1.3214688289446\\
125	1.31588486733238\\
126	1.32577227405278\\
127	1.32320872764244\\
128	1.31714803500501\\
129	1.31846827521221\\
130	1.32077291333593\\
131	1.32605214010482\\
132	1.33258180512212\\
133	1.34250859549286\\
134	1.35403691808753\\
135	1.35552574052047\\
136	1.35531059358615\\
137	1.35931018643005\\
138	1.35570270497085\\
139	1.35804484976729\\
140	1.36018809837083\\
141	1.36376344969819\\
142	1.36413119415634\\
143	1.36797826007516\\
144	1.37460975849348\\
145	1.38143804403676\\
146	1.37388898071517\\
147	1.36418911764865\\
148	1.36160224284683\\
149	1.35378286612697\\
150	1.32382354890625\\
151	1.27448504616424\\
152	1.24782601829553\\
153	1.23597693775497\\
154	1.23134211328385\\
155	1.23083659268847\\
156	1.21659156156863\\
157	1.20531969233792\\
158	1.19972274021527\\
159	1.20264233037996\\
160	1.20824563940622\\
161	1.16698477520772\\
162	1.12062776620527\\
163	1.0928614066191\\
164	1.09401633418126\\
165	1.10800406715042\\
166	1.12779106586611\\
167	1.14070549370943\\
168	1.15604873415485\\
169	1.1784909566836\\
170	1.18933088184094\\
171	1.18492645279111\\
172	1.18431346290253\\
173	1.17982787169968\\
174	1.16487502508551\\
175	1.15911573466677\\
176	1.16126301216199\\
177	1.16567704669181\\
178	1.17102014692386\\
179	1.17500468011669\\
180	1.18146779302336\\
181	1.19137366406902\\
182	1.17221304938948\\
183	1.16064780639839\\
184	1.14884490731623\\
185	1.14952241785498\\
186	1.1617853815502\\
187	1.17449834144626\\
188	1.18772322080948\\
189	1.19717666727598\\
190	1.22596043171514\\
191	1.23633508720559\\
192	1.23878998233843\\
193	1.21875190881479\\
194	1.21127721473008\\
195	1.2034250374199\\
196	1.19865071025449\\
197	1.20348526280089\\
198	1.21269863323275\\
199	1.21801871904291\\
200	1.23759229245255\\
201	1.25120063800806\\
202	1.25429814196576\\
203	1.25769790550904\\
204	1.26600472925108\\
205	1.27971719159827\\
206	1.28722426198325\\
207	1.272153655283\\
208	1.22086075335561\\
209	1.16854343444188\\
210	1.12843722362907\\
211	1.10217880102451\\
212	1.08729174666319\\
213	1.06650501732904\\
214	1.05926807998898\\
215	1.05262101017235\\
216	1.04647433523952\\
217	1.04049495881221\\
218	1.02215493573782\\
219	1.00248284214327\\
220	0.998422517851571\\
221	0.995457374029064\\
222	1.00201380504784\\
223	1.00708960205278\\
224	1.01264234033464\\
225	1.01267857501818\\
226	1.02111732271508\\
227	1.03234345493356\\
228	1.05652510746851\\
229	1.08661419909593\\
230	1.10738313226982\\
231	1.12534567839465\\
232	1.14116828427118\\
233	1.14901465984767\\
234	1.15823322258404\\
235	1.15673846187114\\
236	1.14980521533816\\
237	1.13123197159884\\
238	1.10881442473135\\
239	1.08913993104171\\
240	1.07002339806613\\
241	1.05212070546926\\
242	1.01658328967868\\
243	0.980381151483154\\
244	0.96573278931886\\
245	0.963293280373296\\
246	0.968793771612271\\
247	0.979911553693651\\
248	0.999626922913841\\
249	1.01997593647324\\
250	1.0311084487094\\
251	1.02483329845902\\
252	1.02849800852288\\
253	1.02690938067323\\
254	1.03310441466394\\
255	1.04098267984514\\
256	1.0381871108867\\
257	1.01149315824461\\
258	0.989048966175545\\
259	0.975035687327411\\
260	0.975881971412192\\
261	0.988805940364851\\
262	1.02078513315094\\
263	1.03924200349815\\
264	1.04755195415104\\
265	1.06668311424301\\
266	1.09344721170002\\
267	1.11916306971405\\
268	1.14276944147673\\
269	1.16289229587981\\
270	1.17542897393461\\
271	1.13742174123811\\
272	1.0724640634287\\
273	1.01539635622635\\
274	0.994210197922359\\
275	0.989211971500136\\
276	0.996086006548032\\
277	1.00658946621782\\
278	1.02215226928496\\
279	1.04661713980636\\
280	1.05511823053721\\
281	1.0706623378482\\
282	1.07895364739098\\
283	1.08195975713753\\
284	1.09827027462153\\
285	1.1168243124657\\
286	1.14114795170246\\
287	1.16856318198841\\
288	1.187237837366\\
289	1.20028927462371\\
290	1.21028407681529\\
291	1.22557105398669\\
292	1.2373457600943\\
293	1.25227890154497\\
294	1.26673766587417\\
295	1.28527582518073\\
296	1.2990372059585\\
297	1.29879813851167\\
298	1.30253225837102\\
299	1.31151072915542\\
300	1.31622522833254\\
301	1.32636321939906\\
302	1.33534948294587\\
303	1.33649442587356\\
304	1.33965074190078\\
305	1.34110613992256\\
306	1.35359038722256\\
307	1.31440998836529\\
308	1.23277146494307\\
309	1.15678742951724\\
310	1.10469346256491\\
311	1.08167155598184\\
312	1.07211152727257\\
313	1.06817750186207\\
314	1.0719422292298\\
315	1.09470799486999\\
316	1.10594266362041\\
317	1.10808944082072\\
318	1.11309415679087\\
319	1.12137692736076\\
320	1.1064615179932\\
321	1.09224381392742\\
322	1.11184175197659\\
323	1.13273274655282\\
324	1.15876143408845\\
325	1.1775377866822\\
326	1.19879117982598\\
327	1.2125193721214\\
328	1.21148786262649\\
329	1.20687209137446\\
330	1.17049175861159\\
331	1.12618819359412\\
332	1.09010201251624\\
333	1.06350953512151\\
334	1.0550024280119\\
335	1.04757867057135\\
336	1.04285220450871\\
337	1.04448356725857\\
338	1.05196919984748\\
339	1.07494319630779\\
340	1.09441199463644\\
341	1.11443442181995\\
342	1.12656154985521\\
343	1.15119663070216\\
344	1.16903998856058\\
345	1.18423208036958\\
346	1.19966272277092\\
347	1.21592142831961\\
348	1.22977619964988\\
349	1.24083102489965\\
350	1.23617306995418\\
351	1.20965366456754\\
352	1.18608447475523\\
353	1.17007885877613\\
354	1.15639973575528\\
355	1.14727325290876\\
356	1.14571080835515\\
357	1.14657452135494\\
358	1.16365076745033\\
359	1.17583551985915\\
360	1.19411092997978\\
361	1.20946454661197\\
362	1.21581945771748\\
363	1.22545171638708\\
364	1.22936271355818\\
365	1.23201879027948\\
366	1.22348900007952\\
367	1.21287860523131\\
368	1.19832762257891\\
369	1.16821388634133\\
370	1.13463154197837\\
371	1.11575278682868\\
372	1.10779117430435\\
373	1.10362512369034\\
374	1.10035727871824\\
375	1.1029433910097\\
376	1.10306952708099\\
377	1.10923167631061\\
378	1.10768308160211\\
379	1.11616981893124\\
380	1.12827619755352\\
381	1.12081738104444\\
382	1.11446891817642\\
383	1.1177637534394\\
384	1.12445211506235\\
385	1.14233874520856\\
386	1.16329909197322\\
387	1.18595737520359\\
388	1.19596629191229\\
389	1.1989986760602\\
390	1.21071987911378\\
391	1.22352119537686\\
392	1.24404451457876\\
393	1.2629918662559\\
394	1.28141990355415\\
395	1.28950593147449\\
396	1.28882625780593\\
397	1.29746010829918\\
398	1.30852410563438\\
399	1.31553205497472\\
400	1.32442472866383\\
401	1.3298352785172\\
402	1.33453432333166\\
403	1.33910117425427\\
404	1.34357024822361\\
405	1.34946433366682\\
406	1.35662662394743\\
407	1.35603001796096\\
408	1.35771871512843\\
409	1.3548363220116\\
410	1.33717886428465\\
411	1.33265571405207\\
412	1.3353243477587\\
413	1.3094504985369\\
414	1.24817910118261\\
415	1.20082281671827\\
416	1.17321726065575\\
417	1.15412008552832\\
418	1.13961437832998\\
419	1.11776887716538\\
420	1.10312220092471\\
421	1.09998855433601\\
422	1.09348405716319\\
423	1.07859014722335\\
424	1.06536704707716\\
425	1.05751370739816\\
426	1.06045421489114\\
427	1.07032515942034\\
428	1.08280041002343\\
429	1.08787702455221\\
430	1.08239839798338\\
431	1.08204329346985\\
432	1.06663142022833\\
433	1.05372405604424\\
434	1.05721793205591\\
435	1.08106127754684\\
436	1.1044912954399\\
437	1.12974662415304\\
438	1.15053427350103\\
439	1.1667077611751\\
440	1.17363788020955\\
441	1.18564087030637\\
442	1.20296518827087\\
443	1.21549360270753\\
444	1.22498803141349\\
445	1.21371305854568\\
446	1.19285478352453\\
447	1.17180636665055\\
448	1.16184812971388\\
449	1.16931831984452\\
450	1.18252190024889\\
451	1.19119671652784\\
452	1.20272107036506\\
453	1.20323038456204\\
454	1.20278178060452\\
455	1.20517891675973\\
456	1.21889871310922\\
457	1.22849907554402\\
458	1.23598264362293\\
459	1.24267182604049\\
460	1.24002997061057\\
461	1.24496754407458\\
462	1.23018477553621\\
463	1.21793468921592\\
464	1.21684586783637\\
465	1.23134350535047\\
466	1.23722146314689\\
467	1.24785761821645\\
468	1.25790788863046\\
469	1.26777882640788\\
470	1.28133019672054\\
471	1.29089615695238\\
472	1.29001473129235\\
473	1.29261946258232\\
474	1.28712146067914\\
475	1.29057728832845\\
476	1.2953819050648\\
477	1.29122924174668\\
478	1.28866638633778\\
479	1.28051016398074\\
480	1.28515526665958\\
481	1.29200516891173\\
482	1.29278666066271\\
483	1.28722720769432\\
484	1.2935387146857\\
485	1.30963483167743\\
486	1.32126165947432\\
487	1.31727132093415\\
488	1.3291865699266\\
489	1.32909927220238\\
490	1.33449498379023\\
491	1.32787403002145\\
492	1.316024324398\\
493	1.308601307574\\
494	1.31219335588344\\
495	1.30098394136995\\
496	1.30601384563677\\
497	1.3078398269966\\
};
\addlegendentry{CREM}

\addplot [color=darkgray, line width=1.0pt]
  table[row sep=crcr]{%
1	0.1\\
2	0.10006067673506\\
3	0.100135917382134\\
4	0.100201247204955\\
5	0.10025522146676\\
6	0.10030710150206\\
7	0.100367228027336\\
8	0.100413180949903\\
9	0.100478382834756\\
10	0.100569787259653\\
11	0.100702546051727\\
12	0.100890453623351\\
13	0.101092988043872\\
14	0.10130094998492\\
15	0.101498699286915\\
16	0.101733980219698\\
17	0.102011213258919\\
18	0.102321284709644\\
19	0.102627980406611\\
20	0.102898128957869\\
21	0.10329742408424\\
22	0.103749552757333\\
23	0.104268702629182\\
24	0.104784532599463\\
25	0.105450831484648\\
26	0.106308340183331\\
27	0.107208074076907\\
28	0.108146401339337\\
29	0.10921440495737\\
30	0.110487085125432\\
31	0.111721135999366\\
32	0.113155214411305\\
33	0.1146803440324\\
34	0.116441930044868\\
35	0.118128084291461\\
36	0.119941626038107\\
37	0.121994068775721\\
38	0.124337276993739\\
39	0.126837897901731\\
40	0.130086390432822\\
41	0.134243154342189\\
42	0.138214720055873\\
43	0.142797128887202\\
44	0.148022614312589\\
45	0.153358770187075\\
46	0.159317807355658\\
47	0.165510574691744\\
48	0.173043802616469\\
49	0.180263687815322\\
50	0.186273454701858\\
51	0.192280652402041\\
52	0.20005806660672\\
53	0.208626695006037\\
54	0.219555078799408\\
55	0.230660821711875\\
56	0.242405952308938\\
57	0.254645995954109\\
58	0.266976968901822\\
59	0.279959049237271\\
60	0.296784966451889\\
61	0.31586871638103\\
62	0.334850727333742\\
63	0.355469460165474\\
64	0.381138622546688\\
65	0.407439829381837\\
66	0.434256185302236\\
67	0.461954031359509\\
68	0.488799399201884\\
69	0.517845496719705\\
70	0.548636994337541\\
71	0.583231679476484\\
72	0.614419552327433\\
73	0.645841076918351\\
74	0.682636829846367\\
75	0.718759094430168\\
76	0.751268936640922\\
77	0.778639129797937\\
78	0.807390602405259\\
79	0.832117529090789\\
80	0.854688423463972\\
81	0.88569526418828\\
82	0.908845131879331\\
83	0.917071281380026\\
84	0.918899182342492\\
85	0.91291319551218\\
86	0.907593771886882\\
87	0.907654697391428\\
88	0.90988742227231\\
89	0.921141234069418\\
90	0.932091913352893\\
91	0.941423277984998\\
92	0.949589732612036\\
93	0.963822961882586\\
94	0.984283386832659\\
95	1.00159967102186\\
96	1.01040002460317\\
97	1.02821798831578\\
98	1.04399592256022\\
99	1.0538160098207\\
100	1.0745125016302\\
101	1.08908822404631\\
102	1.10422069113185\\
103	1.1234669187773\\
104	1.15022061460855\\
105	1.17056800042332\\
106	1.18543557521438\\
107	1.20031036797426\\
108	1.21277480058818\\
109	1.21333550615282\\
110	1.21303150942864\\
111	1.21648143052103\\
112	1.21800562856287\\
113	1.22757550468986\\
114	1.24595108913725\\
115	1.26226981472409\\
116	1.2679254848176\\
117	1.2814309935625\\
118	1.28467345753135\\
119	1.28120493257664\\
120	1.2908492975134\\
121	1.29155807680825\\
122	1.29500063898962\\
123	1.29620367490675\\
124	1.29755074435159\\
125	1.29258370485544\\
126	1.30335024205167\\
127	1.30189551052562\\
128	1.29636805951657\\
129	1.29909209544276\\
130	1.3036036106787\\
131	1.30983639188746\\
132	1.31750248888587\\
133	1.32833405752121\\
134	1.33994421989489\\
135	1.34129077513382\\
136	1.34183998478149\\
137	1.34600234888597\\
138	1.34233986684943\\
139	1.34584478426485\\
140	1.34882773345902\\
141	1.35216292354629\\
142	1.35415934037333\\
143	1.35838606331614\\
144	1.36377063529719\\
145	1.37082206217794\\
146	1.36336618959953\\
147	1.35177404745278\\
148	1.34992569070807\\
149	1.34338340767759\\
150	1.3104458225115\\
151	1.25539990352152\\
152	1.22597909644247\\
153	1.21547437154665\\
154	1.21091659421362\\
155	1.20903116849444\\
156	1.19495772565104\\
157	1.18299783075187\\
158	1.17854513862284\\
159	1.18228963652118\\
160	1.1907471801949\\
161	1.14903802334814\\
162	1.10327384230746\\
163	1.07553696698571\\
164	1.07615933600871\\
165	1.08805268509378\\
166	1.10804898127833\\
167	1.12237544889435\\
168	1.13853055678934\\
169	1.16236153933723\\
170	1.17393912889664\\
171	1.17039675422693\\
172	1.1694348321563\\
173	1.16501109197907\\
174	1.14879937223754\\
175	1.1427017202771\\
176	1.14472949004108\\
177	1.1497808622632\\
178	1.1558198252409\\
179	1.15842699183489\\
180	1.16654350444212\\
181	1.17778846038121\\
182	1.15858530824645\\
183	1.14669913354141\\
184	1.13438114145271\\
185	1.13522703959286\\
186	1.14742350006045\\
187	1.15975187333701\\
188	1.17358973049848\\
189	1.18302177667217\\
190	1.21076041131739\\
191	1.22087866106477\\
192	1.22352562791319\\
193	1.2042248972019\\
194	1.19683442674703\\
195	1.18967596137173\\
196	1.18716707126227\\
197	1.19286494060316\\
198	1.20184710802173\\
199	1.20726590342635\\
200	1.22740984477565\\
201	1.24154952172713\\
202	1.24571049639855\\
203	1.25047188185216\\
204	1.2586859542201\\
205	1.27339790857055\\
206	1.28033984673015\\
207	1.26482005084955\\
208	1.215784277673\\
209	1.16475436498206\\
210	1.12466708238066\\
211	1.09811136925324\\
212	1.08442152253209\\
213	1.06476216076\\
214	1.05729183581172\\
215	1.05110231796207\\
216	1.044342670978\\
217	1.03777377765975\\
218	1.01914392712367\\
219	0.999942246054292\\
220	0.995275728593187\\
221	0.992623237012673\\
222	0.99876845222437\\
223	1.00222486831079\\
224	1.00790109433054\\
225	1.00766474381788\\
226	1.0148173878946\\
227	1.02413580730884\\
228	1.04776627800217\\
229	1.07752492534921\\
230	1.09830460738844\\
231	1.11502000407122\\
232	1.13100958124926\\
233	1.1397806910384\\
234	1.15019119019473\\
235	1.14930146866677\\
236	1.14162080672469\\
237	1.12272419738133\\
238	1.10147544190879\\
239	1.08149989699735\\
240	1.06224236887757\\
241	1.04448154884065\\
242	1.00895190050046\\
243	0.973267703585301\\
244	0.959412003031365\\
245	0.95678480461369\\
246	0.963093498885644\\
247	0.974353666816232\\
248	0.993822430714288\\
249	1.01465088510559\\
250	1.02513895956843\\
251	1.01935493935253\\
252	1.02364736700233\\
253	1.02167730755925\\
254	1.02869494745209\\
255	1.03634924391399\\
256	1.03344143785646\\
257	1.00693141941201\\
258	0.984106331056145\\
259	0.970595333119616\\
260	0.971380996003144\\
261	0.983272225028683\\
262	1.01406976719185\\
263	1.0323832311275\\
264	1.04042779386018\\
265	1.0592818001235\\
266	1.08474050472056\\
267	1.10937880511768\\
268	1.1328179647799\\
269	1.15357149241564\\
270	1.16698521121607\\
271	1.12952710582144\\
272	1.06585784043143\\
273	1.00966183880242\\
274	0.988533416548123\\
275	0.984018342213713\\
276	0.990513364341736\\
277	1.00023544322701\\
278	1.01674321173417\\
279	1.04181372735792\\
280	1.05044744265229\\
281	1.06537421206647\\
282	1.0738156422404\\
283	1.07720387350965\\
284	1.09422972789662\\
285	1.112315379549\\
286	1.13573439510078\\
287	1.16329320531437\\
288	1.18094886403674\\
289	1.1933989492131\\
290	1.20359398907506\\
291	1.21949446517724\\
292	1.23061932195053\\
293	1.24643325741081\\
294	1.26096985275938\\
295	1.2794254636405\\
296	1.29320568476799\\
297	1.29334186706582\\
298	1.29803555066615\\
299	1.30751336024248\\
300	1.31239093322716\\
301	1.32303518358932\\
302	1.33176561691693\\
303	1.33263097057203\\
304	1.33546448510468\\
305	1.33720217035421\\
306	1.34990804020628\\
307	1.31017476357052\\
308	1.22743982433117\\
309	1.15132322124312\\
310	1.09908679894743\\
311	1.07613103633959\\
312	1.06618464670934\\
313	1.0622314675544\\
314	1.06614535496937\\
315	1.08978269951106\\
316	1.10091685626949\\
317	1.10253177306634\\
318	1.10680567972482\\
319	1.11528916696258\\
320	1.10068880718941\\
321	1.08642603447839\\
322	1.10665483344498\\
323	1.12827010353902\\
324	1.15294822742329\\
325	1.17287731301804\\
326	1.19449389647032\\
327	1.20807089567224\\
328	1.20793036506436\\
329	1.20340794850483\\
330	1.1669343755737\\
331	1.12275353143857\\
332	1.08575378054101\\
333	1.05819697271883\\
334	1.04913846031564\\
335	1.04157362100177\\
336	1.03543740652215\\
337	1.03710884665534\\
338	1.04523681077657\\
339	1.06804822818364\\
340	1.08721989215629\\
341	1.10691408029131\\
342	1.11946265669002\\
343	1.14434393123264\\
344	1.1622791771614\\
345	1.17575041162697\\
346	1.19055208404233\\
347	1.2072887641407\\
348	1.22157024898487\\
349	1.23249552320213\\
350	1.22831355329391\\
351	1.20074389220316\\
352	1.1759456809566\\
353	1.15959658947609\\
354	1.14655637620409\\
355	1.13695643176534\\
356	1.13513938280441\\
357	1.13617331696356\\
358	1.15456816156127\\
359	1.16689600472115\\
360	1.18482803852794\\
361	1.20014991907736\\
362	1.20674391900995\\
363	1.21702043447952\\
364	1.22131082545349\\
365	1.22464870825352\\
366	1.21557386249838\\
367	1.204869173155\\
368	1.19118650055368\\
369	1.16103591016411\\
370	1.12757223327806\\
371	1.10801226366256\\
372	1.09984041938633\\
373	1.09519132875252\\
374	1.09283496419385\\
375	1.09583177260435\\
376	1.09585346835176\\
377	1.1015757139836\\
378	1.10015856786325\\
379	1.10842658265201\\
380	1.12135364335885\\
381	1.11335867846776\\
382	1.1070633717175\\
383	1.11120298725754\\
384	1.11712981588238\\
385	1.13443995863885\\
386	1.15605678308874\\
387	1.17877317510917\\
388	1.1886562833135\\
389	1.19217736410414\\
390	1.20418601629613\\
391	1.21784753920209\\
392	1.23999506520735\\
393	1.2586555252484\\
394	1.27753177234356\\
395	1.28582611973228\\
396	1.2842134550054\\
397	1.29240168385737\\
398	1.30206804503702\\
399	1.31005270508377\\
400	1.32007728676178\\
401	1.32611137212215\\
402	1.33082379693894\\
403	1.33552631997114\\
404	1.33956507766043\\
405	1.34527169209415\\
406	1.35314504837159\\
407	1.35211320352995\\
408	1.35380642433048\\
409	1.34986982303936\\
410	1.33183307553448\\
411	1.3277349102632\\
412	1.33062273035876\\
413	1.30444290155669\\
414	1.24053897882535\\
415	1.19113944243906\\
416	1.16373209918903\\
417	1.14413453175945\\
418	1.12837554744743\\
419	1.10564064829316\\
420	1.09035632951774\\
421	1.08704396712539\\
422	1.0798740449825\\
423	1.06623625343524\\
424	1.05242961585134\\
425	1.04398844168563\\
426	1.04748594282114\\
427	1.05783217353467\\
428	1.07097147012902\\
429	1.07749537157127\\
430	1.07225889344441\\
431	1.07256823085183\\
432	1.05646776299969\\
433	1.04294321177115\\
434	1.04706597583031\\
435	1.07146477595429\\
436	1.09607755399874\\
437	1.12178822108871\\
438	1.14316149352228\\
439	1.1596708105796\\
440	1.16806773669589\\
441	1.18066774893499\\
442	1.19867303953863\\
443	1.21155851388221\\
444	1.22124790026277\\
445	1.21011509269364\\
446	1.18919979622317\\
447	1.16708413912633\\
448	1.15787686267966\\
449	1.16574708901982\\
450	1.17895761185895\\
451	1.18786517525583\\
452	1.1993318410806\\
453	1.19989655575121\\
454	1.20047249330929\\
455	1.20297391052442\\
456	1.21672629169664\\
457	1.22652698871672\\
458	1.23455228345059\\
459	1.2414730381114\\
460	1.23830385046799\\
461	1.24270998336387\\
462	1.22924379279018\\
463	1.2167466180844\\
464	1.21440850590483\\
465	1.22871380002331\\
466	1.23477766675888\\
467	1.24627625075411\\
468	1.25611939668003\\
469	1.2658856173408\\
470	1.27942578127228\\
471	1.2888576741351\\
472	1.28703968688214\\
473	1.28975914804928\\
474	1.28484944273214\\
475	1.28842671556669\\
476	1.29345097155871\\
477	1.28933808434855\\
478	1.28698172703257\\
479	1.27777688824513\\
480	1.28208160618125\\
481	1.28848911256199\\
482	1.28917390629907\\
483	1.28316660995678\\
484	1.28912767780309\\
485	1.30557888445773\\
486	1.31703184106431\\
487	1.31300861537557\\
488	1.3252966650843\\
489	1.32592298307151\\
490	1.33191532549068\\
491	1.32505203206989\\
492	1.31285734639063\\
493	1.30492848144566\\
494	1.30812804703451\\
495	1.29745634386691\\
496	1.30256327746098\\
497	1.30434855815206\\
};
\addlegendentry{TREM}

\addplot [color=lms_red, line width=1.0pt, forget plot]
  table[row sep=crcr]{%
1	0.5\\
2	0.566223006099474\\
3	0.640521373782678\\
4	0.682590123558483\\
5	0.695590132426825\\
6	0.697426030699392\\
7	0.700689176399277\\
8	0.685159031067904\\
9	0.682169267864605\\
10	0.695135549543847\\
11	0.728821079236728\\
12	0.785563528484107\\
13	0.836048913218817\\
14	0.870592343867571\\
15	0.885819877167408\\
16	0.906337779575855\\
17	0.930806098867182\\
18	0.952235571801845\\
19	0.963339280432518\\
20	0.953245946242823\\
21	0.968355376130552\\
22	0.986225439033047\\
23	1.00200107368016\\
24	1.00786889658588\\
25	1.03019451214278\\
26	1.06503181163898\\
27	1.08893085477783\\
28	1.10422272749559\\
29	1.12176435999236\\
30	1.14439962128721\\
31	1.15037811967234\\
32	1.16130911253564\\
33	1.16824863255895\\
34	1.17943399525139\\
35	1.17484384423344\\
36	1.16704377867498\\
37	1.16304917807275\\
38	1.16087045245457\\
39	1.15600642128248\\
40	1.16948541338114\\
41	1.19668724450232\\
42	1.20470748395964\\
43	1.2206499926185\\
44	1.24017109941244\\
45	1.24882909907741\\
46	1.25644523457331\\
47	1.25655875522184\\
48	1.26790690697704\\
49	1.26541789031224\\
50	1.23826681971792\\
51	1.20692037396794\\
52	1.19206874108852\\
53	1.1788777860265\\
54	1.1833734987545\\
55	1.18102958879264\\
56	1.17472939851738\\
57	1.16506141846839\\
58	1.15294712615701\\
59	1.1409718936372\\
60	1.14595367829542\\
61	1.15533433106349\\
62	1.15783458842594\\
63	1.16159547956416\\
64	1.18127156573669\\
65	1.19622219476315\\
66	1.20923556866659\\
67	1.21817190467635\\
68	1.22235776787972\\
69	1.22940385389582\\
70	1.24075520371514\\
71	1.25701184450648\\
72	1.26341586390891\\
73	1.26946397120644\\
74	1.28457022787605\\
75	1.29683218196943\\
76	1.30181735657691\\
77	1.29740369947047\\
78	1.29975311169552\\
79	1.29627919716736\\
80	1.28935432453996\\
81	1.29565025420203\\
82	1.29180202929871\\
83	1.27031215941112\\
84	1.24114094121829\\
85	1.20516375960168\\
86	1.17296939467593\\
87	1.15024974576514\\
88	1.13337942593252\\
89	1.12783134720134\\
90	1.12268648473375\\
91	1.11802371084849\\
92	1.11290429361882\\
93	1.11494165538855\\
94	1.12373357957315\\
95	1.13078666648925\\
96	1.12840303785796\\
97	1.13803668034394\\
98	1.14657966629756\\
99	1.14935063663531\\
100	1.16369031452747\\
101	1.17131297773453\\
102	1.18105706015336\\
103	1.19611780041142\\
104	1.2174480809573\\
105	1.23389541288681\\
106	1.24428636006515\\
107	1.25429497553779\\
108	1.26298727806955\\
109	1.26000244912191\\
110	1.25496216749\\
111	1.25544846008268\\
112	1.25336382123131\\
113	1.2599398878224\\
114	1.27600192629006\\
115	1.28974299253437\\
116	1.29341242490112\\
117	1.30499490466057\\
118	1.30554316271243\\
119	1.29945474704401\\
120	1.30812282449138\\
121	1.30924453605659\\
122	1.31094435994444\\
123	1.31048974666789\\
124	1.3113467036628\\
125	1.3059270789175\\
126	1.31556870690345\\
127	1.31343321669984\\
128	1.30768807911707\\
129	1.30928389055357\\
130	1.3125249214258\\
131	1.3173907709835\\
132	1.32454608057989\\
133	1.33500785283357\\
134	1.34613568566915\\
135	1.347289658026\\
136	1.34747519109887\\
137	1.35124115928266\\
138	1.34675968566329\\
139	1.35000937750353\\
140	1.35313474305719\\
141	1.35663320371053\\
142	1.35831897077093\\
143	1.36258724268648\\
144	1.36798999806111\\
145	1.37406994574445\\
146	1.36629832488949\\
147	1.35519514320404\\
148	1.35322607883787\\
149	1.3466229679778\\
150	1.31412856365187\\
151	1.26023086837004\\
152	1.23093974299372\\
153	1.21983574548516\\
154	1.21573096617687\\
155	1.21397312912242\\
156	1.19978519416576\\
157	1.18775848768812\\
158	1.18279492287719\\
159	1.1858954392332\\
160	1.19361960933145\\
161	1.15159072216013\\
162	1.10532691320014\\
163	1.07774837140158\\
164	1.0788569158599\\
165	1.09186829342706\\
166	1.11177888224287\\
167	1.12599559020443\\
168	1.14200788902611\\
169	1.16585880342992\\
170	1.17715484676983\\
171	1.17356468870814\\
172	1.17245975701448\\
173	1.16786339315194\\
174	1.152072337049\\
175	1.14610327389799\\
176	1.14796106438507\\
177	1.15282993409017\\
178	1.15866518332292\\
179	1.16201916830028\\
180	1.17003137236312\\
181	1.18083324585366\\
182	1.16180107950426\\
183	1.15009624245246\\
184	1.13835584415321\\
185	1.13924164693341\\
186	1.1510966529813\\
187	1.16365015714029\\
188	1.17766079484912\\
189	1.18708205641327\\
190	1.21508462578242\\
191	1.22530117297949\\
192	1.22789632450724\\
193	1.20812135811239\\
194	1.20069911141391\\
195	1.19327194694487\\
196	1.1899865212417\\
197	1.19552836637074\\
198	1.20473168893205\\
199	1.21015619123059\\
200	1.23022370155779\\
201	1.24447818839762\\
202	1.24791723762333\\
203	1.25219382593677\\
204	1.26022848085581\\
205	1.27485808530106\\
206	1.28186753793341\\
207	1.26646530576766\\
208	1.2166495338184\\
209	1.16543573793116\\
210	1.12569672469007\\
211	1.09898769408369\\
212	1.08482003133601\\
213	1.06461957918341\\
214	1.05724002217469\\
215	1.05098500603674\\
216	1.04426019244374\\
217	1.03798967020155\\
218	1.01939485855177\\
219	1.00039302897121\\
220	0.995764764975832\\
221	0.993133664989393\\
222	0.99937152325981\\
223	1.00324155162876\\
224	1.00916015020487\\
225	1.00928579799372\\
226	1.01733908385377\\
227	1.02726117577653\\
228	1.05070053246618\\
229	1.08079238164602\\
230	1.10162465857695\\
231	1.11844005575926\\
232	1.13420799861913\\
233	1.14290820601339\\
234	1.15253048067543\\
235	1.15114732526423\\
236	1.14360816288292\\
237	1.12490481527859\\
238	1.10355768522863\\
239	1.08385389847694\\
240	1.06489380891744\\
241	1.04697074264313\\
242	1.01153055640142\\
243	0.975747409517563\\
244	0.961750241950151\\
245	0.959437147536671\\
246	0.965827849782099\\
247	0.977131407122555\\
248	0.996683346754849\\
249	1.01722074477059\\
250	1.0276731839651\\
251	1.02216393002503\\
252	1.02614056320964\\
253	1.02429167289622\\
254	1.0310624971174\\
255	1.03862230575201\\
256	1.03569891064654\\
257	1.00907747747973\\
258	0.986406469377411\\
259	0.972539104461649\\
260	0.973330427196088\\
261	0.98535182866408\\
262	1.01672306525003\\
263	1.03486978514578\\
264	1.04274239928433\\
265	1.0616702499157\\
266	1.08699585752808\\
267	1.11190986507287\\
268	1.13564468297147\\
269	1.15597336501985\\
270	1.16895521130417\\
271	1.13134167453607\\
272	1.06750120315827\\
273	1.01105322727435\\
274	0.989886373049449\\
275	0.985030101669881\\
276	0.991713046810649\\
277	1.00130450270476\\
278	1.01733123410138\\
279	1.04231810013887\\
280	1.05087710532387\\
281	1.06580687273586\\
282	1.07427216252598\\
283	1.07752637218368\\
284	1.0940262646323\\
285	1.11225049938566\\
286	1.13578587796363\\
287	1.16328339476748\\
288	1.1809731704676\\
289	1.19379032803687\\
290	1.2037517248447\\
291	1.21940499282622\\
292	1.23095289927348\\
293	1.24656559326748\\
294	1.26113619967384\\
295	1.27996666845495\\
296	1.29368548313616\\
297	1.29378172730934\\
298	1.2983469179337\\
299	1.30771688702081\\
300	1.31264975053412\\
301	1.32326221903684\\
302	1.33214152357755\\
303	1.33322857097372\\
304	1.33592193731953\\
305	1.33753099665855\\
306	1.3502232777239\\
307	1.31057969884192\\
308	1.22829245717138\\
309	1.1523380222142\\
310	1.10013908207929\\
311	1.07726975190305\\
312	1.06725434462321\\
313	1.06330602566892\\
314	1.06726421395575\\
315	1.09059400117187\\
316	1.10165649672491\\
317	1.1035547668234\\
318	1.10844511794512\\
319	1.11681480696649\\
320	1.10201449168168\\
321	1.08759350896597\\
322	1.10760569108735\\
323	1.12881101942533\\
324	1.15374239132129\\
325	1.17353902827314\\
326	1.19495194103666\\
327	1.20853915434981\\
328	1.20812607970102\\
329	1.20346282587972\\
330	1.16688141848183\\
331	1.12253751347382\\
332	1.08558759389439\\
333	1.05815066686964\\
334	1.04883755881533\\
335	1.04122964064146\\
336	1.03550181983838\\
337	1.03709869363276\\
338	1.04517886823112\\
339	1.0680211427348\\
340	1.08731846511274\\
341	1.107101140375\\
342	1.11985833750404\\
343	1.14462740039363\\
344	1.16250187059441\\
345	1.1763242471603\\
346	1.19111041271714\\
347	1.20764634663531\\
348	1.22212690373487\\
349	1.23318058354\\
350	1.22916463521077\\
351	1.20162808300943\\
352	1.17695618307826\\
353	1.16056654131493\\
354	1.14729991392634\\
355	1.13797943946487\\
356	1.13638747244565\\
357	1.13756005245159\\
358	1.15578018198469\\
359	1.16809281496487\\
360	1.18603696990286\\
361	1.20141772311889\\
362	1.20779316977062\\
363	1.21798930692714\\
364	1.22235338467597\\
365	1.22574935463651\\
366	1.21676657098746\\
367	1.20614265539757\\
368	1.19222388909345\\
369	1.1619260109364\\
370	1.12842750360893\\
371	1.10911104479891\\
372	1.10108894767774\\
373	1.09659606305696\\
374	1.09394375750049\\
375	1.09683360231735\\
376	1.09676803352747\\
377	1.10244335135412\\
378	1.10085992926836\\
379	1.10923341282417\\
380	1.12207441106683\\
381	1.11396012289177\\
382	1.1075424018446\\
383	1.11142734798168\\
384	1.11750302274117\\
385	1.13486232174972\\
386	1.15653871596765\\
387	1.17957590415826\\
388	1.18948866443472\\
389	1.19289092773282\\
390	1.20500596963575\\
391	1.2185977213186\\
392	1.24050722542313\\
393	1.25910972014221\\
394	1.27803065135165\\
395	1.28629992896208\\
396	1.28466167109648\\
397	1.29292965821113\\
398	1.3027983981797\\
399	1.31053409049368\\
400	1.32042935516235\\
401	1.32651380805875\\
402	1.33132206422057\\
403	1.33605498723699\\
404	1.34036083154857\\
405	1.34593942605193\\
406	1.3537516196161\\
407	1.35277321523568\\
408	1.35447881775461\\
409	1.35076114794252\\
410	1.33266291429377\\
411	1.32862426228165\\
412	1.33141406414174\\
413	1.30527558212566\\
414	1.24191380637631\\
415	1.19296552493874\\
416	1.16562090384945\\
417	1.14614133758958\\
418	1.13068550510107\\
419	1.10819229961422\\
420	1.09294219927164\\
421	1.08970844593853\\
422	1.08255461462101\\
423	1.06866181765082\\
424	1.05505532866722\\
425	1.046628048898\\
426	1.04995528478272\\
427	1.06025055430314\\
428	1.07319355796213\\
429	1.07940372504889\\
430	1.07414313321469\\
431	1.07436255130763\\
432	1.05838536816485\\
433	1.04515508225228\\
434	1.04910368896105\\
435	1.07336511144315\\
436	1.09764882204898\\
437	1.1233542509334\\
438	1.14454222140509\\
439	1.16107412327669\\
440	1.16910059630029\\
441	1.18172119543319\\
442	1.19957444791728\\
443	1.21246911910115\\
444	1.22201697143751\\
445	1.21087005233029\\
446	1.18994677826369\\
447	1.16795008121175\\
448	1.15864702606515\\
449	1.16640016970779\\
450	1.17978384166646\\
451	1.18850189612582\\
452	1.19992248400226\\
453	1.20049001222738\\
454	1.20093173736058\\
455	1.20335559212261\\
456	1.21715268924077\\
457	1.22710754790846\\
458	1.23497784758873\\
459	1.24188448899551\\
460	1.23879199655472\\
461	1.24327395561521\\
462	1.22953498195078\\
463	1.21678871133104\\
464	1.2146400954338\\
465	1.22892652697539\\
466	1.23505616760378\\
467	1.2464460232774\\
468	1.2562541685376\\
469	1.26584806262684\\
470	1.27941605108742\\
471	1.28896082246156\\
472	1.28728949777803\\
473	1.29004961558334\\
474	1.28509683655774\\
475	1.28872193099435\\
476	1.29399126114049\\
477	1.29002024437029\\
478	1.28765382908253\\
479	1.27849281029202\\
480	1.28282333764399\\
481	1.28919859964244\\
482	1.28990089902053\\
483	1.28390231354306\\
484	1.29002472944527\\
485	1.3063976384287\\
486	1.31786641342943\\
487	1.31409959630641\\
488	1.32625666597416\\
489	1.32674087649962\\
490	1.33267765539295\\
491	1.32596121802067\\
492	1.31395500286926\\
493	1.30628656727546\\
494	1.30941877894653\\
495	1.29866361852257\\
496	1.3037020910839\\
497	1.3055438208497\\
};
\addplot [color=darkgray, line width=1.0pt, forget plot]
  table[row sep=crcr]{%
1	0.5\\
2	0.500039395006603\\
3	0.500091098584121\\
4	0.500130503002895\\
5	0.500154852735777\\
6	0.500173670847485\\
7	0.500197019870743\\
8	0.500202347524959\\
9	0.500222301094533\\
10	0.500263766232896\\
11	0.500341109062787\\
12	0.500468642179317\\
13	0.500604098262834\\
14	0.500737424321123\\
15	0.500850830843385\\
16	0.500992478963123\\
17	0.501167164082129\\
18	0.501361919662915\\
19	0.501542537978049\\
20	0.501669794713973\\
21	0.501911987459997\\
22	0.502191869600149\\
23	0.502517272429587\\
24	0.502821841243306\\
25	0.50325207348043\\
26	0.503851035443802\\
27	0.50446124873944\\
28	0.505078477967117\\
29	0.505788423283522\\
30	0.506659961672455\\
31	0.507449688457357\\
32	0.508392893780957\\
33	0.509383697521843\\
34	0.510548645918864\\
35	0.511564271436623\\
36	0.51265011116689\\
37	0.51388792214893\\
38	0.515335872562604\\
39	0.516839996388012\\
40	0.518995313387765\\
41	0.521965254994179\\
42	0.524618371464138\\
43	0.527775646605529\\
44	0.531403842778833\\
45	0.53497503385627\\
46	0.538993444328247\\
47	0.543034695363298\\
48	0.548237033900415\\
49	0.552885525509986\\
50	0.556055170714076\\
51	0.558912698066196\\
52	0.563351934359626\\
53	0.568190586048264\\
54	0.575130158747748\\
55	0.581934074141331\\
56	0.588938619248415\\
57	0.596073524991998\\
58	0.602925008826651\\
59	0.610015488266847\\
60	0.620661347074961\\
61	0.633423759600947\\
62	0.645385737064977\\
63	0.658670540836282\\
64	0.676602955905213\\
65	0.694969058453293\\
66	0.713417256045101\\
67	0.732624007730589\\
68	0.750562004719932\\
69	0.770152792310927\\
70	0.791484434775338\\
71	0.816521095223894\\
72	0.837919591105534\\
73	0.859701773718405\\
74	0.88671438475803\\
75	0.912886189661038\\
76	0.935761790380309\\
77	0.953381823288759\\
78	0.973031785103481\\
79	0.98848259469423\\
80	1.00188589418829\\
81	1.02387044969513\\
82	1.03832966574862\\
83	1.03817167274644\\
84	1.03175830019434\\
85	1.01767286747376\\
86	1.00474920456168\\
87	0.997745488226658\\
88	0.993339149003947\\
89	0.998568165746336\\
90	1.00369303141502\\
91	1.00739966469369\\
92	1.0105411512841\\
93	1.02027367115626\\
94	1.03656537689242\\
95	1.05014682999222\\
96	1.05523713223999\\
97	1.06963518374361\\
98	1.0822239664805\\
99	1.08892067790427\\
100	1.10676221645363\\
101	1.11897099814774\\
102	1.13174402342833\\
103	1.14914351687392\\
104	1.17441104025743\\
105	1.19285726168188\\
106	1.20617652975549\\
107	1.21967033804359\\
108	1.23106706851188\\
109	1.23018782102125\\
110	1.22869984039104\\
111	1.23112618754661\\
112	1.23157755798345\\
113	1.24018384920233\\
114	1.25786740478578\\
115	1.27345622463674\\
116	1.27841183010524\\
117	1.2913308243071\\
118	1.29411272277943\\
119	1.29019444434006\\
120	1.29927787186\\
121	1.29938133619766\\
122	1.30250312854551\\
123	1.3033417182437\\
124	1.30419024333099\\
125	1.29876150686147\\
126	1.30921028198106\\
127	1.30732425044109\\
128	1.30138158706728\\
129	1.30411744811128\\
130	1.30851346845394\\
131	1.31472016193414\\
132	1.32228000606307\\
133	1.33299271543777\\
134	1.34428109801104\\
135	1.34527850962209\\
136	1.34554388868198\\
137	1.34965421764572\\
138	1.34572059846613\\
139	1.34900254421513\\
140	1.3515243895968\\
141	1.35469367195213\\
142	1.35658358799313\\
143	1.36082468193917\\
144	1.36624017709098\\
145	1.37326602094328\\
146	1.3658229167833\\
147	1.35402805112117\\
148	1.35219594210854\\
149	1.34548419474861\\
150	1.31244590066117\\
151	1.25720091107612\\
152	1.22770013226872\\
153	1.21725125794817\\
154	1.21266220948069\\
155	1.21064966007258\\
156	1.1964002280362\\
157	1.18435192589476\\
158	1.17988413660634\\
159	1.18366784717881\\
160	1.19205680501072\\
161	1.15024375142232\\
162	1.10442383555553\\
163	1.07658188631798\\
164	1.07705357488936\\
165	1.0888414226625\\
166	1.10897145645059\\
167	1.12324462270728\\
168	1.13939852178884\\
169	1.16319685673284\\
170	1.17481137393464\\
171	1.17121980876972\\
172	1.17036674286633\\
173	1.1658963619681\\
174	1.14971255922262\\
175	1.14357928533126\\
176	1.14547802774793\\
177	1.1504761689702\\
178	1.15662123044331\\
179	1.15918265423493\\
180	1.16729632999716\\
181	1.17848582175341\\
182	1.15918395476204\\
183	1.14721933194558\\
184	1.13487332520311\\
185	1.13572659433162\\
186	1.14792126340718\\
187	1.16019945319068\\
188	1.17401329136015\\
189	1.18358064593447\\
190	1.21132422947291\\
191	1.2215134049223\\
192	1.22419075886709\\
193	1.20499833546098\\
194	1.19764687274992\\
195	1.19043199659853\\
196	1.18784324855136\\
197	1.19347884227025\\
198	1.20249417921077\\
199	1.20789112290801\\
200	1.22802992659547\\
201	1.24209730525078\\
202	1.24644100742221\\
203	1.25122168861082\\
204	1.25941798490156\\
205	1.2743497966096\\
206	1.28124030077597\\
207	1.26555789734879\\
208	1.21641565803808\\
209	1.1652871390958\\
210	1.12520673400779\\
211	1.09858205010864\\
212	1.08491653225038\\
213	1.06516561188062\\
214	1.05764401398895\\
215	1.05142570014558\\
216	1.04465708165593\\
217	1.03804730821919\\
218	1.01944715927679\\
219	1.000265073795\\
220	0.99551945740557\\
221	0.99279095787192\\
222	0.998984478350336\\
223	1.00252945060999\\
224	1.00823585705757\\
225	1.00794164477507\\
226	1.01506551869457\\
227	1.02448302699513\\
228	1.04820251020256\\
229	1.07811357153665\\
230	1.09890678309287\\
231	1.1155784550655\\
232	1.13161196892315\\
233	1.14036820092209\\
234	1.15089394874772\\
235	1.1500241656209\\
236	1.1423223153508\\
237	1.12334249499837\\
238	1.10206499542076\\
239	1.08209659643447\\
240	1.06277443991718\\
241	1.04502879228685\\
242	1.00943597894405\\
243	0.973692514178093\\
244	0.959825900945732\\
245	0.957211938174211\\
246	0.9635012187925\\
247	0.974775807232988\\
248	0.994291166472173\\
249	1.01513831129549\\
250	1.02559335707546\\
251	1.01978633722268\\
252	1.02412099446776\\
253	1.02210779860633\\
254	1.02906913208275\\
255	1.03667294210805\\
256	1.03376771370669\\
257	1.00729144348402\\
258	0.984511424381079\\
259	0.970956340410789\\
260	0.971798789828724\\
261	0.983694737403155\\
262	1.01448713635671\\
263	1.03272729574152\\
264	1.04072009794283\\
265	1.05958570664479\\
266	1.08503667791988\\
267	1.1097111770299\\
268	1.13313659774224\\
269	1.1539314098233\\
270	1.16737838156688\\
271	1.12995567955926\\
272	1.06627486852119\\
273	1.01009696522571\\
274	0.988982875717367\\
275	0.984439724255352\\
276	0.99083435508466\\
277	1.00055615403051\\
278	1.01706100869627\\
279	1.04209730247708\\
280	1.05080170300616\\
281	1.06578908410693\\
282	1.07419695686713\\
283	1.07751617460081\\
284	1.09457463065293\\
285	1.11269926483475\\
286	1.13612946039868\\
287	1.16380022715973\\
288	1.18143924242784\\
289	1.19385717439706\\
290	1.20407036708007\\
291	1.21999206431972\\
292	1.23120303527756\\
293	1.24697454809289\\
294	1.26149276683313\\
295	1.27991379501977\\
296	1.29364403187535\\
297	1.29376563690141\\
298	1.29843585650192\\
299	1.30791004448923\\
300	1.31281288097661\\
301	1.32352848034679\\
302	1.33231281075014\\
303	1.3331838815256\\
304	1.33601199295702\\
305	1.33773974293555\\
306	1.35049442810283\\
307	1.31078417580862\\
308	1.22802415337257\\
309	1.15188594946133\\
310	1.09963828142759\\
311	1.07667149753565\\
312	1.06665231402758\\
313	1.0626529832571\\
314	1.06655947592839\\
315	1.09018335086334\\
316	1.1012523754298\\
317	1.10289090459004\\
318	1.10715788028462\\
319	1.11568505315098\\
320	1.10105614589335\\
321	1.08678765686291\\
322	1.10698652769207\\
323	1.12863995284023\\
324	1.15332985547489\\
325	1.17322595114953\\
326	1.19483639854813\\
327	1.20847973688195\\
328	1.20832714580574\\
329	1.20376528207871\\
330	1.16728977645889\\
331	1.12305400598651\\
332	1.0859446559563\\
333	1.05835986452573\\
334	1.04923492922265\\
335	1.04163089115994\\
336	1.03545156250474\\
337	1.03710209434825\\
338	1.04530515108687\\
339	1.06805760011581\\
340	1.08718452317533\\
341	1.10696228853432\\
342	1.11955578682561\\
343	1.14445224624698\\
344	1.16241878691705\\
345	1.17592310464621\\
346	1.19077079133316\\
347	1.20750476764761\\
348	1.22185870927716\\
349	1.23277057887048\\
350	1.22861347847369\\
351	1.20099222819948\\
352	1.17610590800078\\
353	1.15970524919337\\
354	1.14662016368728\\
355	1.13706998394952\\
356	1.13533629874165\\
357	1.13645178401507\\
358	1.15486770041586\\
359	1.1671711283329\\
360	1.18508296964215\\
361	1.20039331398613\\
362	1.20700206541418\\
363	1.21733203476052\\
364	1.22162859546735\\
365	1.22497458693895\\
366	1.21587649459212\\
367	1.20514525089095\\
368	1.19141744730968\\
369	1.16126397379674\\
370	1.12784316171121\\
371	1.10833119259384\\
372	1.10014121713618\\
373	1.09546895251666\\
374	1.09308215569489\\
375	1.09604914925342\\
376	1.09608365270136\\
377	1.1017689326199\\
378	1.10033453421689\\
379	1.10860371638017\\
380	1.12151095944554\\
381	1.11352441020049\\
382	1.10720491067539\\
383	1.11135677025153\\
384	1.11725158043229\\
385	1.1345259515047\\
386	1.15620619748948\\
387	1.17890251139357\\
388	1.18878404643112\\
389	1.19234981623679\\
390	1.20436740672922\\
391	1.21803993067978\\
392	1.24021366652827\\
393	1.25889258151428\\
394	1.27780616227172\\
395	1.28609126965281\\
396	1.28449747366318\\
397	1.29269868463421\\
398	1.30239434101124\\
399	1.31040132057128\\
400	1.32043849485505\\
401	1.32643973224256\\
402	1.33114136520283\\
403	1.33582139821937\\
404	1.33981877408995\\
405	1.34552977695901\\
406	1.35340040243915\\
407	1.35239135772955\\
408	1.354064382149\\
409	1.35008322565543\\
410	1.33202593680396\\
411	1.32794102995286\\
412	1.33083136140671\\
413	1.30470731939103\\
414	1.24090350262259\\
415	1.19154930866435\\
416	1.16412721911954\\
417	1.14452871046995\\
418	1.12879982177882\\
419	1.10608665573099\\
420	1.0907921084306\\
421	1.08750836968287\\
422	1.08034081969262\\
423	1.06669440022854\\
424	1.05290627097487\\
425	1.04446267295706\\
426	1.04791329359663\\
427	1.05821717032965\\
428	1.07134921601762\\
429	1.07782470955521\\
430	1.0725536463315\\
431	1.07284717502236\\
432	1.05674185094918\\
433	1.04320502528066\\
434	1.04727318964277\\
435	1.07166311533115\\
436	1.0962639350283\\
437	1.12196748429843\\
438	1.14336482613068\\
439	1.15983419339808\\
440	1.16821195880814\\
441	1.18077651595709\\
442	1.19879735987372\\
443	1.2116903903466\\
444	1.22136384971726\\
445	1.21026754913921\\
446	1.18936423032085\\
447	1.16724696442877\\
448	1.15804642466321\\
449	1.16589080123666\\
450	1.17913951798135\\
451	1.18799972433894\\
452	1.19942286100866\\
453	1.19999637026111\\
454	1.20054262295141\\
455	1.20301913487979\\
456	1.21680750900798\\
457	1.22669576065517\\
458	1.23471127114261\\
459	1.24160588841821\\
460	1.23847400956604\\
461	1.24287228000554\\
462	1.22937888510679\\
463	1.21687768636347\\
464	1.21451694641179\\
465	1.22882110576303\\
466	1.23495703675312\\
467	1.2464598297447\\
468	1.25634225679356\\
469	1.26613790240191\\
470	1.27964495853866\\
471	1.28909375069875\\
472	1.2873042762199\\
473	1.29003852357867\\
474	1.28510990595623\\
475	1.28867774599994\\
476	1.29367945157507\\
477	1.28956509073575\\
478	1.28722120841467\\
479	1.27800588504666\\
480	1.28230564303911\\
481	1.2887105254475\\
482	1.28942137880176\\
483	1.2833940336299\\
484	1.2893472341925\\
485	1.30581946273162\\
486	1.31728759510414\\
487	1.31322731065876\\
488	1.32551435873414\\
489	1.32607840685999\\
490	1.3320849459263\\
491	1.32525440713897\\
492	1.31308500072633\\
493	1.30507280632399\\
494	1.30824766378355\\
495	1.29756616200739\\
496	1.30269612759986\\
497	1.30447690947409\\
};
\addplot [color=lms_red, line width=1.0pt, forget plot]
  table[row sep=crcr]{%
1	1\\
2	1.02219266475049\\
3	1.05609128715974\\
4	1.06060240001296\\
5	1.0393086009462\\
6	1.01008127824871\\
7	0.983821465595724\\
8	0.941403518590872\\
9	0.914140116762325\\
10	0.90557488131456\\
11	0.920428703506839\\
12	0.960222712644514\\
13	0.995676145347635\\
14	1.01662390243728\\
15	1.01895350676663\\
16	1.02746382669583\\
17	1.04152651956165\\
18	1.05346373153546\\
19	1.05542530587869\\
20	1.03679819116007\\
21	1.04399349638419\\
22	1.05466639811163\\
23	1.06536354040473\\
24	1.06573013946777\\
25	1.08243793093862\\
26	1.11273376186839\\
27	1.13262962233501\\
28	1.14441835673492\\
29	1.15848789524656\\
30	1.17774616652431\\
31	1.18095698249489\\
32	1.18923005730569\\
33	1.19381012486132\\
34	1.2025694201267\\
35	1.19545577720609\\
36	1.18578215948149\\
37	1.17967066935449\\
38	1.17621895133254\\
39	1.16984503717284\\
40	1.1821427902054\\
41	1.20849709473289\\
42	1.21554775513308\\
43	1.23092854267823\\
44	1.24991709531872\\
45	1.25785714700077\\
46	1.26482167505066\\
47	1.26405547497615\\
48	1.27501058594559\\
49	1.27184239102342\\
50	1.24439179806015\\
51	1.21254742277471\\
52	1.19754290283443\\
53	1.18369025966592\\
54	1.18728251882049\\
55	1.18450795321814\\
56	1.17787150289843\\
57	1.16800723449035\\
58	1.15534821555978\\
59	1.14305776875449\\
60	1.14802128653183\\
61	1.15746535662078\\
62	1.15962002833296\\
63	1.16334768962917\\
64	1.18287131429276\\
65	1.19763302780633\\
66	1.21056449921528\\
67	1.2201746629988\\
68	1.22451562890315\\
69	1.23131448597254\\
70	1.24253956224441\\
71	1.25875420977619\\
72	1.26473674351857\\
73	1.27049960819576\\
74	1.28548622683987\\
75	1.29757001291279\\
76	1.30274742681337\\
77	1.29870338033094\\
78	1.30116091935411\\
79	1.29733549336043\\
80	1.29006749078111\\
81	1.29618935829023\\
82	1.29238419582199\\
83	1.27069460433379\\
84	1.24118985576077\\
85	1.20481763211528\\
86	1.17215081823313\\
87	1.1488190317694\\
88	1.13160543075435\\
89	1.1263406874577\\
90	1.12106289418884\\
91	1.11620716768447\\
92	1.11095657906626\\
93	1.11308203726238\\
94	1.12207925017441\\
95	1.12922758571332\\
96	1.12705695775982\\
97	1.13689618150254\\
98	1.14515585877043\\
99	1.14783437912136\\
100	1.16207625895289\\
101	1.16977887999898\\
102	1.17927331024052\\
103	1.19430503529934\\
104	1.21618109671893\\
105	1.23271991424142\\
106	1.24315217255582\\
107	1.25334645436347\\
108	1.26195455766192\\
109	1.25899671846034\\
110	1.25428400371344\\
111	1.25451477068499\\
112	1.25283931690275\\
113	1.25960658941812\\
114	1.27567481604202\\
115	1.28935779666969\\
116	1.29297039728306\\
117	1.3045583160917\\
118	1.30525293212261\\
119	1.29938526341138\\
120	1.30802326217765\\
121	1.30836641326892\\
122	1.30991197642279\\
123	1.30982281871627\\
124	1.31047238054212\\
125	1.30492991956292\\
126	1.31461963717892\\
127	1.3124714510403\\
128	1.30636006430175\\
129	1.30800974822896\\
130	1.31133450674193\\
131	1.31643235548683\\
132	1.32386503707899\\
133	1.33437467356425\\
134	1.34557253646655\\
135	1.34670824066813\\
136	1.34696468111232\\
137	1.35067818043156\\
138	1.3462740966799\\
139	1.34949128467769\\
140	1.35247630902325\\
141	1.3557976588739\\
142	1.35758734395272\\
143	1.36187317583742\\
144	1.36726141869252\\
145	1.37354831784448\\
146	1.36581616039115\\
147	1.35439019097281\\
148	1.35248343160002\\
149	1.34593724377744\\
150	1.31314466237327\\
151	1.2584949588732\\
152	1.22899455026012\\
153	1.2181369347681\\
154	1.21398037272257\\
155	1.21207651063777\\
156	1.1978257209825\\
157	1.18575001687879\\
158	1.18088635700206\\
159	1.18415781780711\\
160	1.19230089965303\\
161	1.15031672837164\\
162	1.10424772612017\\
163	1.07659833556357\\
164	1.07756506696285\\
165	1.08997582507521\\
166	1.10983884470627\\
167	1.12412043432591\\
168	1.14012343730685\\
169	1.16398001022797\\
170	1.175344531753\\
171	1.17181973916501\\
172	1.17077346800904\\
173	1.16614264327877\\
174	1.1501603863146\\
175	1.14417698232728\\
176	1.14608107629545\\
177	1.15101381506814\\
178	1.1571865846301\\
179	1.16045205659369\\
180	1.1687095911004\\
181	1.17964224692807\\
182	1.16056862866999\\
183	1.14878980134907\\
184	1.13690171287801\\
185	1.1377765810249\\
186	1.14963311073819\\
187	1.1622051319398\\
188	1.17608804124205\\
189	1.1857293155049\\
190	1.21376817119303\\
191	1.22378361623696\\
192	1.22635780556657\\
193	1.20673827152477\\
194	1.19931249654371\\
195	1.19183413337539\\
196	1.18882381990436\\
197	1.19435103470011\\
198	1.20340815039145\\
199	1.20897872728029\\
200	1.22922814831841\\
201	1.24346519714641\\
202	1.24716847050266\\
203	1.25164047355084\\
204	1.25957879335373\\
205	1.27425822689195\\
206	1.28117328404299\\
207	1.26551904202519\\
208	1.21604140757072\\
209	1.16495029589972\\
210	1.12515741670583\\
211	1.09849345641696\\
212	1.08463656840408\\
213	1.06460194794456\\
214	1.05719132867104\\
215	1.05095586974716\\
216	1.04416564743979\\
217	1.03778982267572\\
218	1.01914626508178\\
219	1.00016489819744\\
220	0.995536185921721\\
221	0.992834381488073\\
222	0.999026345421686\\
223	1.00277189826524\\
224	1.0086448890443\\
225	1.00864196051497\\
226	1.01634250270849\\
227	1.02595205829115\\
228	1.04959999957274\\
229	1.07957550739906\\
230	1.10038827239139\\
231	1.1169348841465\\
232	1.13275918171704\\
233	1.1415172010689\\
234	1.15157772941934\\
235	1.15036028474286\\
236	1.14270351681365\\
237	1.12398262145428\\
238	1.102718720278\\
239	1.08293659815111\\
240	1.06386305336107\\
241	1.0459918030937\\
242	1.0104678387997\\
243	0.974708464445447\\
244	0.960825830619417\\
245	0.958487716453688\\
246	0.964862889000535\\
247	0.976100451929854\\
248	0.995731070499822\\
249	1.0163378644318\\
250	1.02671205029947\\
251	1.0212098817903\\
252	1.02533216047469\\
253	1.02341217910453\\
254	1.0302620219375\\
255	1.03782361050639\\
256	1.03490485043087\\
257	1.00833007154187\\
258	0.985617251306647\\
259	0.971879743604223\\
260	0.972667562691755\\
261	0.984619862037464\\
262	1.01567088107498\\
263	1.03393715801284\\
264	1.04176450999803\\
265	1.06068271067315\\
266	1.08607220751241\\
267	1.11077550912392\\
268	1.13438931482882\\
269	1.15487156629445\\
270	1.16802928029772\\
271	1.13052785969235\\
272	1.06679289094558\\
273	1.01046214811289\\
274	0.989356442496866\\
275	0.984679269390049\\
276	0.991269874620994\\
277	1.00083893061545\\
278	1.01698539532169\\
279	1.04206378144473\\
280	1.05063340272324\\
281	1.06561082296518\\
282	1.07402519796453\\
283	1.0773422174355\\
284	1.09402336691949\\
285	1.11221856443564\\
286	1.13566682102655\\
287	1.16321250397587\\
288	1.18083876274513\\
289	1.19350597871441\\
290	1.20360261483741\\
291	1.21939939868318\\
292	1.23085782001497\\
293	1.24653179056252\\
294	1.26106690217271\\
295	1.2797446093659\\
296	1.29350618583184\\
297	1.29361396113636\\
298	1.29829307114919\\
299	1.30768878274637\\
300	1.31265078730491\\
301	1.32333667984773\\
302	1.33222084430924\\
303	1.33316420860976\\
304	1.33582730703766\\
305	1.33747673857833\\
306	1.35015781297606\\
307	1.31049845978295\\
308	1.22793639048845\\
309	1.15189724208305\\
310	1.09966132970537\\
311	1.07674129322532\\
312	1.06670015905022\\
313	1.06273286506318\\
314	1.06669476540642\\
315	1.09021339918018\\
316	1.10135318462642\\
317	1.10313264219508\\
318	1.10772077205752\\
319	1.11615325644715\\
320	1.10143436028795\\
321	1.08706424597856\\
322	1.10716480230527\\
323	1.12854770656763\\
324	1.15330290841304\\
325	1.17323333467466\\
326	1.19467599174165\\
327	1.2083020462637\\
328	1.20803029916736\\
329	1.203370278274\\
330	1.16684076956462\\
331	1.12249191115229\\
332	1.0853817223783\\
333	1.05785019154765\\
334	1.04864850522305\\
335	1.04103117834899\\
336	1.03499002801175\\
337	1.03662111762193\\
338	1.04477384578979\\
339	1.0676017968851\\
340	1.08686196328745\\
341	1.10666491593632\\
342	1.11938716680004\\
343	1.14424946598716\\
344	1.16218829848016\\
345	1.17591443247054\\
346	1.19073053359791\\
347	1.20730589517499\\
348	1.22175412213021\\
349	1.23273375590277\\
350	1.22870099262421\\
351	1.20108840577443\\
352	1.17629369469392\\
353	1.15987886030292\\
354	1.14671989685603\\
355	1.13731269028481\\
356	1.13563983428407\\
357	1.1367706056611\\
358	1.15511587518005\\
359	1.16743415755375\\
360	1.18530887840528\\
361	1.20067740370346\\
362	1.20715185469135\\
363	1.21745268559646\\
364	1.22182898767919\\
365	1.22528648570133\\
366	1.21620249510513\\
367	1.20553255460352\\
368	1.19172120264035\\
369	1.16149510840498\\
370	1.12806294001528\\
371	1.10863067262543\\
372	1.10056309408081\\
373	1.09598330088036\\
374	1.09347981284306\\
375	1.09639788338009\\
376	1.09640861604601\\
377	1.10199622465303\\
378	1.10045820204994\\
379	1.10877985005226\\
380	1.12165994342546\\
381	1.11365246132393\\
382	1.10733842441208\\
383	1.11135854746746\\
384	1.11729915739913\\
385	1.13458487267091\\
386	1.15628871506821\\
387	1.17920509207922\\
388	1.18911993969694\\
389	1.19259116537749\\
390	1.20465211007374\\
391	1.21831769588579\\
392	1.24038080942477\\
393	1.25900389072403\\
394	1.27790090664591\\
395	1.28613043739822\\
396	1.28449934634364\\
397	1.2927367488032\\
398	1.30253525528563\\
399	1.3103838074596\\
400	1.32035255557611\\
401	1.32642556805773\\
402	1.33113695282846\\
403	1.33585279344718\\
404	1.33999365386341\\
405	1.34555223021054\\
406	1.35344564855206\\
407	1.35241887828271\\
408	1.35411028611421\\
409	1.35025044432861\\
410	1.33214478834473\\
411	1.32808276688248\\
412	1.33090312723729\\
413	1.30485057301158\\
414	1.24125245548511\\
415	1.1920770794134\\
416	1.1647476808527\\
417	1.14524979305127\\
418	1.12959707598923\\
419	1.10699012353146\\
420	1.09166771881165\\
421	1.08838690434887\\
422	1.08121019216224\\
423	1.06743141399358\\
424	1.05379940980944\\
425	1.04539414356506\\
426	1.0488322259468\\
427	1.0591822013082\\
428	1.07217731491081\\
429	1.07852417064277\\
430	1.07324199660129\\
431	1.07350725901479\\
432	1.05742696688718\\
433	1.04404092143605\\
434	1.04807661861726\\
435	1.07238648561882\\
436	1.09683228285832\\
437	1.12259061998256\\
438	1.1438825987053\\
439	1.16044019898197\\
440	1.16864909472897\\
441	1.18132983757077\\
442	1.19926599065417\\
443	1.21216421109054\\
444	1.22174349512806\\
445	1.21062998340367\\
446	1.18973818565205\\
447	1.16760600204699\\
448	1.15838493652157\\
449	1.16623843511689\\
450	1.17954555035429\\
451	1.18837698394722\\
452	1.19982140032349\\
453	1.2004170324034\\
454	1.20099879274688\\
455	1.20339269631393\\
456	1.21717934986038\\
457	1.22707104616856\\
458	1.2350779433553\\
459	1.24198649295922\\
460	1.23881841654474\\
461	1.24322880622689\\
462	1.22966322419847\\
463	1.21701230175307\\
464	1.21473768991655\\
465	1.22909379473114\\
466	1.2352356689106\\
467	1.24669150347459\\
468	1.25651271715391\\
469	1.26618143970508\\
470	1.2797239363893\\
471	1.28921865491199\\
472	1.28744719067115\\
473	1.29017214381679\\
474	1.28518132650283\\
475	1.28876809779218\\
476	1.2938867855398\\
477	1.28985684768596\\
478	1.28754930798023\\
479	1.27834067501766\\
480	1.28263414771527\\
481	1.28901290506376\\
482	1.28973976969753\\
483	1.28370601821796\\
484	1.2897470941561\\
485	1.30614527218031\\
486	1.31760394228245\\
487	1.31364207714983\\
488	1.32588639708995\\
489	1.3264434632973\\
490	1.33242772866858\\
491	1.32562489087385\\
492	1.31356504413174\\
493	1.30572543752473\\
494	1.30886477881248\\
495	1.29814565621722\\
496	1.30320406283413\\
497	1.30500092518087\\
};
\addplot [color=darkgray, line width=1.0pt, forget plot]
  table[row sep=crcr]{%
1	1\\
2	1.00001320206112\\
3	1.00003626143431\\
4	1.00004331870354\\
5	1.00003111115936\\
6	1.00000966335803\\
7	0.999987494404428\\
8	0.999942195783741\\
9	0.999906136982445\\
10	0.999885497095118\\
11	0.999894115701582\\
12	0.999947131795666\\
13	1.00000089259395\\
14	1.0000424499231\\
15	1.00005245301952\\
16	1.00008018421469\\
17	1.00013013166136\\
18	1.00018282653053\\
19	1.00020712971297\\
20	1.00015808896766\\
21	1.00020761549747\\
22	1.00027634009732\\
23	1.00036931772259\\
24	1.00040981659241\\
25	1.00054769007351\\
26	1.00082735534991\\
27	1.00108402332166\\
28	1.00131052002904\\
29	1.00158442310557\\
30	1.00196830928961\\
31	1.00221437142655\\
32	1.00255533720747\\
33	1.0028903793279\\
34	1.00332791521599\\
35	1.00352263929332\\
36	1.00370712182909\\
37	1.00393727250781\\
38	1.00428541139256\\
39	1.00455742715578\\
40	1.00539373910527\\
41	1.00688343326751\\
42	1.00787902012741\\
43	1.00926599593823\\
44	1.01095507330719\\
45	1.0123645820672\\
46	1.01396758685093\\
47	1.01537019935522\\
48	1.01770651878603\\
49	1.01915341686649\\
50	1.01883432867081\\
51	1.01780382766675\\
52	1.01810373749758\\
53	1.01833392015158\\
54	1.0203638043792\\
55	1.02181636944678\\
56	1.02299629477326\\
57	1.02380394870978\\
58	1.02382096679255\\
59	1.02353351593814\\
60	1.02654566231133\\
61	1.03138997935408\\
62	1.03457205579276\\
63	1.03869757021368\\
64	1.04713453197167\\
65	1.05545086865855\\
66	1.06356404243255\\
67	1.07235562843282\\
68	1.07912713873332\\
69	1.08675932387586\\
70	1.09620264482557\\
71	1.10914390951538\\
72	1.1183223959462\\
73	1.12789043876826\\
74	1.14247059139178\\
75	1.15608129571991\\
76	1.16677713556556\\
77	1.17214311446042\\
78	1.18007820649415\\
79	1.1839183574965\\
80	1.1858092810718\\
81	1.19642271029645\\
82	1.20001599985797\\
83	1.189369432076\\
84	1.17270366499644\\
85	1.14848178104165\\
86	1.1259904524015\\
87	1.1101946737125\\
88	1.09744822774996\\
89	1.09510099856035\\
90	1.09288570505103\\
91	1.08958508821752\\
92	1.08638490710197\\
93	1.0904464323359\\
94	1.10148694780768\\
95	1.11033081444803\\
96	1.1106745052409\\
97	1.12084560666491\\
98	1.12959038818174\\
99	1.13242213165314\\
100	1.14673932462959\\
101	1.1560033138964\\
102	1.16582250161746\\
103	1.18086598828754\\
104	1.20412088637667\\
105	1.22013969949102\\
106	1.23157608176155\\
107	1.24330310120255\\
108	1.25324069751827\\
109	1.25063453387511\\
110	1.24768746204352\\
111	1.2487892336064\\
112	1.24793417713037\\
113	1.25533877751232\\
114	1.27211868461502\\
115	1.28674460095532\\
116	1.29082270192149\\
117	1.3029690602375\\
118	1.30524301470256\\
119	1.30075064096543\\
120	1.30906177766646\\
121	1.3084524196965\\
122	1.31117628824844\\
123	1.31160919390432\\
124	1.31183978853085\\
125	1.30584491506911\\
126	1.31600075396187\\
127	1.3136501590236\\
128	1.30722668154745\\
129	1.30989478885025\\
130	1.31417588603685\\
131	1.32031132036171\\
132	1.32760624497287\\
133	1.33819583692841\\
134	1.34906231758055\\
135	1.34971418951917\\
136	1.34969503984646\\
137	1.35373640277526\\
138	1.34944895532543\\
139	1.35255525800925\\
140	1.35457184210389\\
141	1.35758891761158\\
142	1.35936675994683\\
143	1.36356640303949\\
144	1.3691526582641\\
145	1.37618709375374\\
146	1.36879612066717\\
147	1.35678712459408\\
148	1.35491669720255\\
149	1.34802653725264\\
150	1.31484625117679\\
151	1.25939421961248\\
152	1.22976242503397\\
153	1.21941859539808\\
154	1.21473888420544\\
155	1.21264286136264\\
156	1.19817881230926\\
157	1.18597391006482\\
158	1.18142708262988\\
159	1.18516871379402\\
160	1.19349363986945\\
161	1.15155654842906\\
162	1.10565393229508\\
163	1.07765727985686\\
164	1.07798588713986\\
165	1.08972286347517\\
166	1.11001251999506\\
167	1.12416711731731\\
168	1.14029691606938\\
169	1.16404122459873\\
170	1.17562059756271\\
171	1.17195719487808\\
172	1.17115660050397\\
173	1.16664807272283\\
174	1.15051238844858\\
175	1.14428520774262\\
176	1.14606413875151\\
177	1.15104098155095\\
178	1.15728457667496\\
179	1.15978745143791\\
180	1.16795112263332\\
181	1.17909640017682\\
182	1.15967888204927\\
183	1.14758952357198\\
184	1.13518069310914\\
185	1.13602152469844\\
186	1.14821349669289\\
187	1.16045200440593\\
188	1.17422888007282\\
189	1.18379727127246\\
190	1.21151859271524\\
191	1.22178870539753\\
192	1.22447498039541\\
193	1.2053896616603\\
194	1.19803889131571\\
195	1.19086806424265\\
196	1.18826925443309\\
197	1.19386138956698\\
198	1.20290306744038\\
199	1.20827913289938\\
200	1.22841462145127\\
201	1.24246472579004\\
202	1.2469608446702\\
203	1.25180355525404\\
204	1.25997383472781\\
205	1.2750535434686\\
206	1.28192016122782\\
207	1.26615881950602\\
208	1.2169730308017\\
209	1.16579604909557\\
210	1.12570196887742\\
211	1.09902730473342\\
212	1.08538275869951\\
213	1.06558170653384\\
214	1.05800528545727\\
215	1.05174486918609\\
216	1.04497816551603\\
217	1.038313728423\\
218	1.01973071752481\\
219	1.00056472446387\\
220	0.995762796278105\\
221	0.992964164995331\\
222	0.999174596844898\\
223	1.00276575922531\\
224	1.0084799545874\\
225	1.00810773029121\\
226	1.01519590852466\\
227	1.02467534057097\\
228	1.04842621289019\\
229	1.07842223727657\\
230	1.09921562877898\\
231	1.11593778210179\\
232	1.13201223197885\\
233	1.14073980538276\\
234	1.15131348454428\\
235	1.15047106687325\\
236	1.14274789574118\\
237	1.12371607037921\\
238	1.10241340508657\\
239	1.08244181342401\\
240	1.0630563339188\\
241	1.04534147089928\\
242	1.00972516551823\\
243	0.973946296715346\\
244	0.960066573540078\\
245	0.957416541485882\\
246	0.96363711324221\\
247	0.974918188855325\\
248	0.994438858266727\\
249	1.01531391045185\\
250	1.02577659000764\\
251	1.01995853756086\\
252	1.02431168995237\\
253	1.02226106201996\\
254	1.02917186365422\\
255	1.03674287307389\\
256	1.03384285344149\\
257	1.00739298781562\\
258	0.984634657349648\\
259	0.971079241207954\\
260	0.971963837993025\\
261	0.983849650971115\\
262	1.01463457804526\\
263	1.03283447959992\\
264	1.04079929321819\\
265	1.05966448089683\\
266	1.0851153803848\\
267	1.10983363342501\\
268	1.13323955126874\\
269	1.1540809440917\\
270	1.16758523248557\\
271	1.13017409567402\\
272	1.06647242503394\\
273	1.01029769070153\\
274	0.989183030036546\\
275	0.984634677328715\\
276	0.990974889809656\\
277	1.00071388547626\\
278	1.01721503128784\\
279	1.04222633369828\\
280	1.05096531472104\\
281	1.06597701113463\\
282	1.07438010034323\\
283	1.07766711009795\\
284	1.09477541027187\\
285	1.11291277280741\\
286	1.13633771896035\\
287	1.16407352187288\\
288	1.18171188686855\\
289	1.19410978671592\\
290	1.20435670411733\\
291	1.22030511627249\\
292	1.23152463185542\\
293	1.24727353582419\\
294	1.26178068991665\\
295	1.28017203339474\\
296	1.29387481996115\\
297	1.2939804971597\\
298	1.29862814996039\\
299	1.30812033918668\\
300	1.31302206745169\\
301	1.32376198286151\\
302	1.33255989035347\\
303	1.33342552729185\\
304	1.33626536090979\\
305	1.33799449616269\\
306	1.35078751538451\\
307	1.31108267804535\\
308	1.22829309375682\\
309	1.15212865759521\\
310	1.09985441257215\\
311	1.07688736440182\\
312	1.06683184778354\\
313	1.06281274136275\\
314	1.06672961067882\\
315	1.0903467322174\\
316	1.10138220548717\\
317	1.10302919370705\\
318	1.10726085237217\\
319	1.11582793477189\\
320	1.10119195888486\\
321	1.08693129237305\\
322	1.10712066918495\\
323	1.12880776295367\\
324	1.1535342324749\\
325	1.17340742437968\\
326	1.19502085874158\\
327	1.20869477184731\\
328	1.20854702486723\\
329	1.2039694624694\\
330	1.16748764336364\\
331	1.12321085146855\\
332	1.08605925229826\\
333	1.05847197060269\\
334	1.04931245733272\\
335	1.04169140495575\\
336	1.03549623503061\\
337	1.03714722274722\\
338	1.045373615747\\
339	1.06809413951107\\
340	1.08719550323228\\
341	1.10699319844229\\
342	1.11959444956197\\
343	1.14450860512977\\
344	1.16250133165331\\
345	1.17599932926316\\
346	1.19085908159065\\
347	1.20759698899443\\
348	1.22198873295973\\
349	1.23286616197821\\
350	1.22871345762672\\
351	1.20108123446065\\
352	1.17616031013544\\
353	1.15974436147419\\
354	1.14662991574802\\
355	1.13708230586279\\
356	1.13537586653827\\
357	1.13652737154241\\
358	1.15497721059468\\
359	1.16727342900685\\
360	1.18518285190076\\
361	1.20049207326252\\
362	1.2071225221443\\
363	1.21747018662612\\
364	1.22176046501332\\
365	1.22510917425212\\
366	1.21600536655643\\
367	1.20526393871619\\
368	1.19151938011653\\
369	1.16134907724986\\
370	1.12793105090159\\
371	1.1084432701794\\
372	1.10023944023289\\
373	1.09554616545782\\
374	1.09315110781503\\
375	1.09611594598812\\
376	1.09615968201342\\
377	1.10183297346572\\
378	1.10040010307063\\
379	1.10866066071624\\
380	1.12156012800492\\
381	1.11358954758615\\
382	1.10726603331151\\
383	1.1114291298704\\
384	1.11730625413735\\
385	1.13457332344802\\
386	1.15627024785368\\
387	1.17892888765272\\
388	1.18880601386216\\
389	1.1923949395666\\
390	1.20441731546178\\
391	1.218084178269\\
392	1.2402693705286\\
393	1.25895674989579\\
394	1.27789116317924\\
395	1.28618919499562\\
396	1.28460360554415\\
397	1.2928116023131\\
398	1.30252171384481\\
399	1.3105501173501\\
400	1.32058790100796\\
401	1.32656882560546\\
402	1.33127206212782\\
403	1.33593252824046\\
404	1.33991655817194\\
405	1.34564214251213\\
406	1.35350787773536\\
407	1.35250549146591\\
408	1.3541712813087\\
409	1.35014958781681\\
410	1.33209013647153\\
411	1.328011774678\\
412	1.33090604506184\\
413	1.30479940213412\\
414	1.24103400351538\\
415	1.19169947256044\\
416	1.16426451321157\\
417	1.14467064953592\\
418	1.12896112362847\\
419	1.10625551877235\\
420	1.09096556301967\\
421	1.08768522861376\\
422	1.08052466688739\\
423	1.06687583274336\\
424	1.05306242007266\\
425	1.04462166422518\\
426	1.04805124169265\\
427	1.05832190054554\\
428	1.07146112692112\\
429	1.07792303401647\\
430	1.07262995961709\\
431	1.07292922366226\\
432	1.05682016414595\\
433	1.04328611087942\\
434	1.04734030093997\\
435	1.0717159482915\\
436	1.09631450819325\\
437	1.12200916254197\\
438	1.1434267984975\\
439	1.15986956564512\\
440	1.16824952458661\\
441	1.18078200293229\\
442	1.1988034716384\\
443	1.21171088400173\\
444	1.22138446949209\\
445	1.21030268232017\\
446	1.1894019325725\\
447	1.16730473591123\\
448	1.15811394070266\\
449	1.1659578434113\\
450	1.17921883585504\\
451	1.1880690910687\\
452	1.19945696513499\\
453	1.20003643257809\\
454	1.20057315618889\\
455	1.20303597906588\\
456	1.21683691812695\\
457	1.22674716668233\\
458	1.23477419145161\\
459	1.24165777458607\\
460	1.23854825537284\\
461	1.24294440721984\\
462	1.22943762256776\\
463	1.21693855722749\\
464	1.21456414927625\\
465	1.22885368918211\\
466	1.23501647550432\\
467	1.24652577965377\\
468	1.25641983724291\\
469	1.26622802117457\\
470	1.27971065003821\\
471	1.2891712106823\\
472	1.28740146182348\\
473	1.29014508667231\\
474	1.28522482700219\\
475	1.28877465371657\\
476	1.29375497230603\\
477	1.28961567443012\\
478	1.28726956266681\\
479	1.27803997491618\\
480	1.28233581806099\\
481	1.28874867370292\\
482	1.28947274399054\\
483	1.28343237384088\\
484	1.2893788945679\\
485	1.30586496306099\\
486	1.31734768081033\\
487	1.31326826683514\\
488	1.32555501342149\\
489	1.326091010318\\
490	1.33210180243243\\
491	1.32528228066662\\
492	1.31311454342103\\
493	1.30505956450319\\
494	1.3082287327075\\
495	1.29755242309587\\
496	1.30269914178883\\
497	1.30447678503334\\
};
\addplot [color=lms_red, line width=1.0pt, forget plot]
  table[row sep=crcr]{%
1	2\\
2	1.93381428987669\\
3	1.88777539005386\\
4	1.81759248438886\\
5	1.72779496747021\\
6	1.63603242515697\\
7	1.55114329504071\\
8	1.45510523574354\\
9	1.3792466586786\\
10	1.32746918372549\\
11	1.30436312400199\\
12	1.3104418244426\\
13	1.31501023204431\\
14	1.30867693095105\\
15	1.28530226465843\\
16	1.27015562835354\\
17	1.26288744717618\\
18	1.25583469975607\\
19	1.23952185798932\\
20	1.20412143529318\\
21	1.19590377913185\\
22	1.19292889094892\\
23	1.19229176958574\\
24	1.18219301070281\\
25	1.18824878447639\\
26	1.20929367408892\\
27	1.22116151491343\\
28	1.22552895653521\\
29	1.23242713129095\\
30	1.2451408514133\\
31	1.24286042367065\\
32	1.24580019401189\\
33	1.24542439014768\\
34	1.24992525511359\\
35	1.23800232235619\\
36	1.22465549904438\\
37	1.21492130709635\\
38	1.20850976737872\\
39	1.19929121524741\\
40	1.20910038699648\\
41	1.23335838348149\\
42	1.23843700473509\\
43	1.25209546542519\\
44	1.26968419978436\\
45	1.275994587177\\
46	1.28133779362193\\
47	1.27901915349146\\
48	1.28894606816361\\
49	1.28434707246963\\
50	1.25617097703452\\
51	1.22340947236402\\
52	1.20779448622464\\
53	1.19293804991179\\
54	1.19522409214969\\
55	1.19147291190496\\
56	1.18409235689532\\
57	1.17378139022928\\
58	1.16038511449107\\
59	1.14756479425795\\
60	1.15228326433776\\
61	1.16167948076882\\
62	1.16343868575233\\
63	1.16709902207179\\
64	1.18643568059809\\
65	1.20086002819434\\
66	1.21350152903577\\
67	1.22391586691115\\
68	1.22816921124497\\
69	1.23458940241746\\
70	1.24563653731848\\
71	1.26178470441319\\
72	1.26746685752998\\
73	1.273028812627\\
74	1.28787436320249\\
75	1.29980303255752\\
76	1.30517441500789\\
77	1.30140473795849\\
78	1.30383832750056\\
79	1.29981813295444\\
80	1.29244601422038\\
81	1.29828094514888\\
82	1.29443743082618\\
83	1.27238653677864\\
84	1.24260510196669\\
85	1.20587493994923\\
86	1.17293282393206\\
87	1.14922982782365\\
88	1.13169929111784\\
89	1.12656984012034\\
90	1.12121398842415\\
91	1.11608207480043\\
92	1.11071907527649\\
93	1.11281637687816\\
94	1.12189851338839\\
95	1.12912823650376\\
96	1.12709161984727\\
97	1.13698068705254\\
98	1.14503113887002\\
99	1.14761555199002\\
100	1.16178929094508\\
101	1.16959834741792\\
102	1.17891414901581\\
103	1.19384880259149\\
104	1.21639913022105\\
105	1.2328658552983\\
106	1.24325617979541\\
107	1.25359204715407\\
108	1.26218772343679\\
109	1.25925058737492\\
110	1.25485902338598\\
111	1.25492817507131\\
112	1.25342508920567\\
113	1.26023495442691\\
114	1.27634548721484\\
115	1.29002350096264\\
116	1.29350859637491\\
117	1.30504340979033\\
118	1.3058188852737\\
119	1.30009065420652\\
120	1.30885269615656\\
121	1.30875552929537\\
122	1.31035162649147\\
123	1.31057625970009\\
124	1.31101977727072\\
125	1.30534578812871\\
126	1.31509228189715\\
127	1.31285721019334\\
128	1.30647519853631\\
129	1.3082102967962\\
130	1.31165928216036\\
131	1.31702059781145\\
132	1.32455353407144\\
133	1.33506696859819\\
134	1.34617206153934\\
135	1.34720113112386\\
136	1.34748519791035\\
137	1.35121488230205\\
138	1.34679776565448\\
139	1.34993661722074\\
140	1.35262921007128\\
141	1.35585581292207\\
142	1.35770742705676\\
143	1.36204251341728\\
144	1.36750945460793\\
145	1.37391411840478\\
146	1.36620313034972\\
147	1.35454923678361\\
148	1.35269911878014\\
149	1.34617275868849\\
150	1.31322720356326\\
151	1.25824814643145\\
152	1.22871664469668\\
153	1.2180527511248\\
154	1.21377575266252\\
155	1.21178876266109\\
156	1.19754498746572\\
157	1.18551072755084\\
158	1.18069209606533\\
159	1.18399888781227\\
160	1.19234177850884\\
161	1.1504309879272\\
162	1.10447639085102\\
163	1.07678136475243\\
164	1.07754549980471\\
165	1.08967235492966\\
166	1.10958021174487\\
167	1.12382127699223\\
168	1.13981044823333\\
169	1.16364441927434\\
170	1.17502501462069\\
171	1.17148899786857\\
172	1.17046825742531\\
173	1.16584946702186\\
174	1.14971970753632\\
175	1.14367310256109\\
176	1.14555215373965\\
177	1.15050098924934\\
178	1.1567328912086\\
179	1.15986679755888\\
180	1.16815476681542\\
181	1.17919190993271\\
182	1.16005541183775\\
183	1.14823814287561\\
184	1.13621422070327\\
185	1.13708907559238\\
186	1.14903981325302\\
187	1.16149131127652\\
188	1.17527959406515\\
189	1.18493604597545\\
190	1.21295014903969\\
191	1.22296494437277\\
192	1.22554865097199\\
193	1.20615547780389\\
194	1.19874285837348\\
195	1.19125732986456\\
196	1.18839900648077\\
197	1.19391065063292\\
198	1.20287919237429\\
199	1.2084392568414\\
200	1.22875202752301\\
201	1.24294368381453\\
202	1.24688038291605\\
203	1.25145911022532\\
204	1.25938355409316\\
205	1.27412545144191\\
206	1.28103169752964\\
207	1.26529220689601\\
208	1.21597783086295\\
209	1.1649165592226\\
210	1.12503136808448\\
211	1.098410138942\\
212	1.08467939565603\\
213	1.06475721312652\\
214	1.05728205455548\\
215	1.05103490898871\\
216	1.04423092010692\\
217	1.03776506071048\\
218	1.01908938162172\\
219	1.00007785560871\\
220	0.995410584054255\\
221	0.992672420282833\\
222	0.998881709816292\\
223	1.00255915655972\\
224	1.00840958797338\\
225	1.00829313968981\\
226	1.01574886715958\\
227	1.02519247601901\\
228	1.04890783008585\\
229	1.07884383505045\\
230	1.09966006531455\\
231	1.11628365389765\\
232	1.13216200532323\\
233	1.14095166301554\\
234	1.15125986941238\\
235	1.15020458628395\\
236	1.14250486366128\\
237	1.12371022906491\\
238	1.10245839837552\\
239	1.08254857208216\\
240	1.0633575711725\\
241	1.04549482330869\\
242	1.00995117236148\\
243	0.974194810494016\\
244	0.960323410223565\\
245	0.957901837967629\\
246	0.964235418846893\\
247	0.975428639560647\\
248	0.995065537011318\\
249	1.01573484504574\\
250	1.02612005025051\\
251	1.02051538673618\\
252	1.02473827999278\\
253	1.02282142881152\\
254	1.02972369394982\\
255	1.03729582340194\\
256	1.03436738457299\\
257	1.00783378688303\\
258	0.985085188052824\\
259	0.971390130915151\\
260	0.972164738732296\\
261	0.9840981215563\\
262	1.01492564165093\\
263	1.03326448646812\\
264	1.04116039216699\\
265	1.06005406318639\\
266	1.08550005961146\\
267	1.11013006384452\\
268	1.13369300683699\\
269	1.15430636390589\\
270	1.16760650807578\\
271	1.13015111368781\\
272	1.06641971572577\\
273	1.01014836186374\\
274	0.98906578003246\\
275	0.984477491476734\\
276	0.991023971256596\\
277	1.00064635236119\\
278	1.01693565576542\\
279	1.04203535044131\\
280	1.05062481351368\\
281	1.06562511895671\\
282	1.07401273800435\\
283	1.07738310632397\\
284	1.09421243739601\\
285	1.11238980552867\\
286	1.13579079788053\\
287	1.1633679745858\\
288	1.18094849576695\\
289	1.19352810982381\\
290	1.20370983549429\\
291	1.21956572420919\\
292	1.23092544993279\\
293	1.24665807089073\\
294	1.26119500862364\\
295	1.27975304102896\\
296	1.29354931887392\\
297	1.2936640728365\\
298	1.29836698554598\\
299	1.30776428589364\\
300	1.31270733691858\\
301	1.32341781962209\\
302	1.33228386120868\\
303	1.33316969240489\\
304	1.33589349537386\\
305	1.33758051102976\\
306	1.35024463192323\\
307	1.31055788128459\\
308	1.22790782076672\\
309	1.15182905400166\\
310	1.09960622815549\\
311	1.07665758461658\\
312	1.06664841942843\\
313	1.06266574167383\\
314	1.06656824068887\\
315	1.09016123356165\\
316	1.10130838378243\\
317	1.10296982630045\\
318	1.10737114278824\\
319	1.11583710353363\\
320	1.10116662193998\\
321	1.08684422609359\\
322	1.10698184275633\\
323	1.12848380313131\\
324	1.15320951972436\\
325	1.17316004477736\\
326	1.19464099561624\\
327	1.20828543072652\\
328	1.20806703019669\\
329	1.20343336740512\\
330	1.16693009411102\\
331	1.12263962876699\\
332	1.08552703611619\\
333	1.0579815588107\\
334	1.0488896327112\\
335	1.04128347170624\\
336	1.03513272565531\\
337	1.03676887529003\\
338	1.04493832707897\\
339	1.06773842228879\\
340	1.08693551000015\\
341	1.10676271741037\\
342	1.11941980972627\\
343	1.14431483680095\\
344	1.16225394586463\\
345	1.17589190464255\\
346	1.19071814468858\\
347	1.2073332889854\\
348	1.22175987897844\\
349	1.23272055686081\\
350	1.22863420530797\\
351	1.200974169636\\
352	1.17612879966888\\
353	1.15973006908252\\
354	1.14664846349956\\
355	1.13715742512636\\
356	1.13544550849548\\
357	1.13654923758126\\
358	1.15494001105763\\
359	1.1672621446071\\
360	1.18512393722022\\
361	1.20047285202866\\
362	1.20697724688446\\
363	1.2173037373927\\
364	1.22165299691621\\
365	1.22507521348804\\
366	1.21598418089409\\
367	1.20530053433036\\
368	1.19154618970139\\
369	1.1614223641132\\
370	1.12804025663335\\
371	1.1085381700571\\
372	1.10041624100938\\
373	1.09579421107523\\
374	1.09337514716997\\
375	1.09632794639191\\
376	1.09637264596967\\
377	1.10194531748913\\
378	1.10042992708891\\
379	1.1087225000421\\
380	1.12160211208591\\
381	1.11361095998113\\
382	1.1073283361453\\
383	1.11141403647144\\
384	1.11732515358242\\
385	1.13458475135033\\
386	1.15629089197307\\
387	1.17909436801962\\
388	1.18900914378684\\
389	1.19252272227991\\
390	1.20455596005826\\
391	1.21823472327664\\
392	1.2403598348649\\
393	1.25899183717627\\
394	1.27787567546446\\
395	1.28610712865366\\
396	1.28449298355296\\
397	1.29271114130105\\
398	1.30250634526613\\
399	1.31040945814134\\
400	1.32040931831503\\
401	1.3264602188197\\
402	1.33113028825722\\
403	1.33581539112235\\
404	1.33987344386474\\
405	1.34546826810023\\
406	1.35336978428055\\
407	1.35234486802411\\
408	1.35402095217783\\
409	1.35007006885396\\
410	1.33196922780578\\
411	1.32789241497027\\
412	1.33073485540675\\
413	1.30466025205867\\
414	1.24093236402021\\
415	1.19163286323562\\
416	1.16428932988352\\
417	1.14476674623166\\
418	1.12903877386701\\
419	1.10636914487426\\
420	1.09103174411332\\
421	1.0877421412183\\
422	1.08058600267656\\
423	1.06687956293311\\
424	1.05321686581445\\
425	1.0447923738768\\
426	1.04825706253606\\
427	1.05861098971921\\
428	1.07170733914998\\
429	1.07813736903774\\
430	1.07286885611228\\
431	1.07314311660016\\
432	1.05701611635878\\
433	1.04351931745286\\
434	1.04759145228844\\
435	1.07193058873658\\
436	1.0964637254721\\
437	1.1222011848826\\
438	1.14356190372194\\
439	1.16011396448146\\
440	1.16840826186977\\
441	1.18107678143019\\
442	1.19905120094119\\
443	1.21193935271387\\
444	1.22154239289692\\
445	1.21043969208544\\
446	1.18954990426872\\
447	1.16738755674854\\
448	1.15816124525747\\
449	1.16601910054037\\
450	1.17928143916245\\
451	1.18814234266597\\
452	1.19960559382971\\
453	1.20019918908027\\
454	1.20079132386721\\
455	1.20321940179501\\
456	1.21700748858064\\
457	1.22687374255782\\
458	1.23493620557437\\
459	1.24186861491253\\
460	1.23869543625777\\
461	1.24308186250509\\
462	1.22956451362521\\
463	1.21698986933544\\
464	1.21467310114382\\
465	1.22903555455593\\
466	1.23517759396245\\
467	1.24667208690431\\
468	1.25653268094933\\
469	1.26625459936064\\
470	1.27979073280557\\
471	1.28924911751191\\
472	1.28744091175253\\
473	1.29015364105824\\
474	1.28516937521073\\
475	1.28873997523394\\
476	1.29378599075111\\
477	1.28971437716976\\
478	1.28741606791963\\
479	1.27820761610377\\
480	1.28249002070113\\
481	1.2888868185251\\
482	1.28960899103358\\
483	1.28357162002794\\
484	1.28957661255509\\
485	1.30599271115943\\
486	1.31745983954008\\
487	1.31342456419126\\
488	1.32570082414664\\
489	1.32627621332126\\
490	1.33226711309014\\
491	1.32544324632929\\
492	1.31334476334931\\
493	1.3054229641727\\
494	1.30857216247744\\
495	1.29787658493907\\
496	1.30296259634914\\
497	1.30473371902577\\
};
\addplot [color=darkgray, line width=1.0pt, forget plot]
  table[row sep=crcr]{%
1	2\\
2	1.99996062718018\\
3	1.9999259422046\\
4	1.99986781231831\\
5	1.99978294561075\\
6	1.99968135996041\\
7	1.99956979819584\\
8	1.99942398091051\\
9	1.99927695019376\\
10	1.99913305829552\\
11	1.99900524975784\\
12	1.99890770314502\\
13	1.99879432182848\\
14	1.9986489035633\\
15	1.99845149195017\\
16	1.99825005093361\\
17	1.99804492796148\\
18	1.99781496083695\\
19	1.99752307746247\\
20	1.99712166634596\\
21	1.99678306506904\\
22	1.99642275468532\\
23	1.99604308971537\\
24	1.99555468429885\\
25	1.99510250376219\\
26	1.99472838965745\\
27	1.99426702237565\\
28	1.99369476217333\\
29	1.99307738566837\\
30	1.99248286856985\\
31	1.99163932539756\\
32	1.99075982682583\\
33	1.98975256953637\\
34	1.98872146078866\\
35	1.98727386274181\\
36	1.98564119823827\\
37	1.98384667593942\\
38	1.98197770347781\\
39	1.97978460278727\\
40	1.97791174718157\\
41	1.97639402198939\\
42	1.97406324398612\\
43	1.97179536298099\\
44	1.96951399483652\\
45	1.96649675649512\\
46	1.96318534422324\\
47	1.95920044869005\\
48	1.95568873847466\\
49	1.95068805373647\\
50	1.94329017472012\\
51	1.93448702335905\\
52	1.92632316447592\\
53	1.91739239083647\\
54	1.90959247585706\\
55	1.90028096782549\\
56	1.88980894888883\\
57	1.87797823359908\\
58	1.86427349610145\\
59	1.84925862308125\\
60	1.83687535664867\\
61	1.82525871476932\\
62	1.81096282311918\\
63	1.79660387515447\\
64	1.78586033574555\\
65	1.77363007632664\\
66	1.76081877383301\\
67	1.74825558305345\\
68	1.73228197392098\\
69	1.71606006470495\\
70	1.70156887604994\\
71	1.68995988905803\\
72	1.67440522286474\\
73	1.6589604903352\\
74	1.6484423417984\\
75	1.6367519639207\\
76	1.62270584733169\\
77	1.60326850874988\\
78	1.5872024191647\\
79	1.56758969704498\\
80	1.5463896444734\\
81	1.53439354323746\\
82	1.51622378168032\\
83	1.48461644786785\\
84	1.44771389614551\\
85	1.40368316225127\\
86	1.3624084463941\\
87	1.32921888821035\\
88	1.30013866655116\\
89	1.28282025041712\\
90	1.26608761014652\\
91	1.24910050909418\\
92	1.23327793835992\\
93	1.22602403319872\\
94	1.22654528007515\\
95	1.22580100152939\\
96	1.21684829629928\\
97	1.21879329159324\\
98	1.21994218531744\\
99	1.21533772228664\\
100	1.22284201467728\\
101	1.22622978862024\\
102	1.23021042634821\\
103	1.24045776705741\\
104	1.25935267891139\\
105	1.27086601356202\\
106	1.27854098622449\\
107	1.28669867412271\\
108	1.29357268899086\\
109	1.28769705733834\\
110	1.2818627361836\\
111	1.28043248207674\\
112	1.27714168143207\\
113	1.28227384391781\\
114	1.29716681023189\\
115	1.3099482391307\\
116	1.31230993797743\\
117	1.32291357162418\\
118	1.32414154227731\\
119	1.31848414611622\\
120	1.32540517471516\\
121	1.32355664591934\\
122	1.32541110353676\\
123	1.32506036411232\\
124	1.32428988256994\\
125	1.31733992877142\\
126	1.3269874331309\\
127	1.32383720370608\\
128	1.31664112805668\\
129	1.31900229121019\\
130	1.32296461806433\\
131	1.32886075080204\\
132	1.3356211762449\\
133	1.34593487232485\\
134	1.3561750057341\\
135	1.35630837789828\\
136	1.35583990144727\\
137	1.35968888552207\\
138	1.35489141127496\\
139	1.35774692165423\\
140	1.35913704852165\\
141	1.36190591021276\\
142	1.36347906529031\\
143	1.36756213238282\\
144	1.37337185649064\\
145	1.38043418174329\\
146	1.37311157307573\\
147	1.36082855282162\\
148	1.35884312989444\\
149	1.35174092019994\\
150	1.31832713845102\\
151	1.26258966365188\\
152	1.23275721949639\\
153	1.22254220612635\\
154	1.21775699177205\\
155	1.21560283483499\\
156	1.20082935786939\\
157	1.18840125715387\\
158	1.18372872351553\\
159	1.18738456703785\\
160	1.1956047645116\\
161	1.15347859763839\\
162	1.10744916799897\\
163	1.07923589506639\\
164	1.07939891060481\\
165	1.09106067346926\\
166	1.11152271868085\\
167	1.12552500523169\\
168	1.14160717661421\\
169	1.16527115251302\\
170	1.17678180280469\\
171	1.17300618875344\\
172	1.17225713787964\\
173	1.16767813341581\\
174	1.15159636973806\\
175	1.14524100213328\\
176	1.14686912386136\\
177	1.15183954835082\\
178	1.15818569748296\\
179	1.16062405588124\\
180	1.1688608263719\\
181	1.1799399119548\\
182	1.16036132612156\\
183	1.14810381545975\\
184	1.13559636090088\\
185	1.13641400035562\\
186	1.14860178628991\\
187	1.16081812121766\\
188	1.17454925091376\\
189	1.18406962044245\\
190	1.21176618620508\\
191	1.22211163144145\\
192	1.22478847276749\\
193	1.20582515708664\\
194	1.19846279155609\\
195	1.1913617377746\\
196	1.18876852316895\\
197	1.19432080019075\\
198	1.20337630068226\\
199	1.20872092836161\\
200	1.22886276901267\\
201	1.24293365042046\\
202	1.24757177821341\\
203	1.25254092892422\\
204	1.26066548212255\\
205	1.27589477494342\\
206	1.28275064388466\\
207	1.26692217706344\\
208	1.21771310935339\\
209	1.16648462493451\\
210	1.1263676759834\\
211	1.09963409640632\\
212	1.0860109442008\\
213	1.06617451560157\\
214	1.05853206423662\\
215	1.05220867551705\\
216	1.04543375289717\\
217	1.0386866960088\\
218	1.02011316809114\\
219	1.00097777247165\\
220	0.996110357705399\\
221	0.993241824205654\\
222	0.99946089154333\\
223	1.00307283345354\\
224	1.00878811119203\\
225	1.00832751717056\\
226	1.01538054603979\\
227	1.02491326275191\\
228	1.04867890333008\\
229	1.07875047970152\\
230	1.09953600748378\\
231	1.11634511553427\\
232	1.13247403652271\\
233	1.14116007104055\\
234	1.15173713074883\\
235	1.15092405892631\\
236	1.14317247408226\\
237	1.12407701807443\\
238	1.10275016202666\\
239	1.08277475910585\\
240	1.06330799157561\\
241	1.04562103048589\\
242	1.00998952556268\\
243	0.974183083214662\\
244	0.960294816058604\\
245	0.957585822432554\\
246	0.963712118016743\\
247	0.975003282924697\\
248	0.994522853526663\\
249	1.01543836461206\\
250	1.0259267620137\\
251	1.02009461834251\\
252	1.02446122639806\\
253	1.02236724686256\\
254	1.02922396757957\\
255	1.03676314368081\\
256	1.03386292167353\\
257	1.00743260951934\\
258	0.984683287732963\\
259	0.971134212670099\\
260	0.972056880670235\\
261	0.983925863435609\\
262	1.0146942185686\\
263	1.03285921961981\\
264	1.04080994393151\\
265	1.05967713256489\\
266	1.08513857223834\\
267	1.10991569016198\\
268	1.13328404982167\\
269	1.15417786162396\\
270	1.16773225082441\\
271	1.13032349440148\\
272	1.06659279474596\\
273	1.01041395936393\\
274	0.989294168426329\\
275	0.984759893001829\\
276	0.991059713123551\\
277	1.00082626604792\\
278	1.01732737193969\\
279	1.042314722317\\
280	1.05108071454816\\
281	1.06610353519126\\
282	1.0745176868931\\
283	1.07778340045132\\
284	1.09494144615521\\
285	1.11307986782163\\
286	1.13649231385153\\
287	1.16429462968756\\
288	1.18194430961774\\
289	1.1943202796085\\
290	1.20461094671507\\
291	1.2205800767675\\
292	1.23178880328027\\
293	1.24752690945817\\
294	1.26203241252035\\
295	1.28039292280037\\
296	1.29407272602539\\
297	1.29415394937231\\
298	1.29877569868925\\
299	1.30830217936042\\
300	1.31319227285016\\
301	1.32394937743433\\
302	1.33275028422248\\
303	1.33360177178801\\
304	1.33645351214099\\
305	1.33819435817499\\
306	1.35102226667832\\
307	1.31131586905091\\
308	1.22849604710863\\
309	1.15230583550298\\
310	1.09999797540455\\
311	1.07703939374316\\
312	1.0669595189359\\
313	1.06292637141314\\
314	1.06685482445774\\
315	1.09046618294169\\
316	1.10147449880423\\
317	1.10312173532949\\
318	1.1073147764647\\
319	1.11591899428021\\
320	1.10127852920895\\
321	1.08702971715047\\
322	1.10721051695489\\
323	1.12892702984454\\
324	1.15368277989895\\
325	1.17353742925657\\
326	1.19515914031324\\
327	1.20885512274429\\
328	1.20871896757334\\
329	1.20412457083734\\
330	1.16763486378276\\
331	1.12331788309766\\
332	1.08614222257705\\
333	1.05855105808275\\
334	1.04935848560098\\
335	1.04173287397776\\
336	1.03552681740163\\
337	1.0371830792206\\
338	1.04541556974963\\
339	1.06811953988173\\
340	1.08720819776538\\
341	1.10701300323527\\
342	1.11961363216291\\
343	1.14454109123444\\
344	1.16256021366898\\
345	1.17603309502444\\
346	1.19089331561851\\
347	1.20763748165827\\
348	1.22205765688302\\
349	1.23290644102181\\
350	1.2287606722169\\
351	1.20113194421851\\
352	1.17619263129108\\
353	1.15976407561232\\
354	1.14661600556686\\
355	1.13705969006313\\
356	1.13537530583202\\
357	1.13655310710958\\
358	1.15503528499847\\
359	1.16733083822583\\
360	1.18525005254428\\
361	1.20056052617371\\
362	1.20721892865219\\
363	1.21757253619938\\
364	1.2218439649704\\
365	1.22519097270991\\
366	1.21608272032261\\
367	1.20533660161091\\
368	1.19158507616168\\
369	1.16139207106885\\
370	1.12796751157978\\
371	1.10850566062777\\
372	1.10028811414082\\
373	1.0955770339722\\
374	1.09317622692836\\
375	1.09613715672036\\
376	1.09619132381727\\
377	1.10185956335722\\
378	1.10043714528544\\
379	1.10867851750676\\
380	1.12157072329958\\
381	1.11361996412996\\
382	1.10729378316396\\
383	1.11146596318467\\
384	1.11732583263802\\
385	1.1345897680508\\
386	1.15629349933854\\
387	1.17891829135547\\
388	1.18879702669311\\
389	1.19239915160931\\
390	1.20442995756608\\
391	1.21808586686291\\
392	1.24027967222128\\
393	1.2589790357498\\
394	1.27792940222527\\
395	1.28625067225529\\
396	1.28466374309824\\
397	1.29288142630088\\
398	1.30259899491635\\
399	1.31064713689036\\
400	1.32067970475127\\
401	1.32664623552451\\
402	1.33135887787861\\
403	1.33600406825845\\
404	1.33998079780215\\
405	1.34572257828875\\
406	1.35358511432774\\
407	1.35258620084216\\
408	1.35424897142081\\
409	1.35020134450829\\
410	1.33214430070593\\
411	1.32807259518478\\
412	1.33096730012447\\
413	1.30485930188139\\
414	1.24111724617596\\
415	1.19179646611738\\
416	1.16434824022874\\
417	1.14475864553678\\
418	1.12906582113632\\
419	1.10636680037594\\
420	1.09108266536069\\
421	1.08780258940991\\
422	1.08065069235346\\
423	1.0669960835015\\
424	1.05314023326569\\
425	1.04470545584194\\
426	1.04812053731688\\
427	1.05836069303628\\
428	1.07150621899922\\
429	1.07796256899069\\
430	1.0726492653479\\
431	1.0729586607542\\
432	1.05684624846724\\
433	1.04331813810516\\
434	1.04736493130448\\
435	1.0717323175056\\
436	1.09633326972993\\
437	1.12201390149249\\
438	1.14345270518502\\
439	1.1598583526127\\
440	1.16824548060155\\
441	1.18075148203894\\
442	1.19876939629582\\
443	1.21169969831582\\
444	1.22138319648204\\
445	1.21030798039567\\
446	1.18940851988054\\
447	1.16732999520325\\
448	1.15815059019444\\
449	1.16599675653431\\
450	1.17926704487519\\
451	1.18811000764735\\
452	1.19946780912077\\
453	1.20004889899232\\
454	1.20058092761836\\
455	1.20303866699976\\
456	1.21684253352408\\
457	1.22675663963738\\
458	1.23480412099515\\
459	1.241677916805\\
460	1.23858341085851\\
461	1.24298113076504\\
462	1.22946864016068\\
463	1.21697469699386\\
464	1.21459380535948\\
465	1.22886838197432\\
466	1.2350448883029\\
467	1.24655926256007\\
468	1.25646187863123\\
469	1.26627970065576\\
470	1.2797380351484\\
471	1.28921204432344\\
472	1.28745263255435\\
473	1.29020970379781\\
474	1.28530123570873\\
475	1.28883524369277\\
476	1.29379485612839\\
477	1.28962122775816\\
478	1.28727197721036\\
479	1.27802940454927\\
480	1.2823242714639\\
481	1.2887488422975\\
482	1.28948039318002\\
483	1.28343355268709\\
484	1.28936862536468\\
485	1.30586858438446\\
486	1.31736568462252\\
487	1.31327059429712\\
488	1.32555462658874\\
489	1.32607503460699\\
490	1.33208786880821\\
491	1.32526743876705\\
492	1.31309530802984\\
493	1.30500811677615\\
494	1.30817266173544\\
495	1.29750101611856\\
496	1.30266483742943\\
497	1.30444091302579\\
};
\addplot [color=lms_red, line width=1.0pt, forget plot]
  table[row sep=crcr]{%
1	5\\
2	4.65641747523114\\
3	4.36052002664593\\
4	4.06518038428369\\
5	3.77173089292584\\
6	3.49588440359782\\
7	3.24152568929542\\
8	2.98850873150747\\
9	2.76946842687099\\
10	2.58804157571533\\
11	2.44810527862822\\
12	2.35045912355946\\
13	2.26144841142171\\
14	2.17107785593584\\
15	2.07124890114208\\
16	1.98604943045381\\
17	1.91434688298925\\
18	1.84915421541818\\
19	1.77943173643026\\
20	1.69470208445873\\
21	1.64165681572754\\
22	1.59860304578927\\
23	1.56202451508541\\
24	1.52014651435548\\
25	1.4955562567349\\
26	1.48922884844677\\
27	1.47682132356151\\
28	1.45891045408496\\
29	1.4446716190145\\
30	1.43867959207418\\
31	1.41976734636532\\
32	1.40724634802219\\
33	1.39273456116218\\
34	1.38479985937092\\
35	1.36032988072258\\
36	1.33640771196514\\
37	1.31659552954138\\
38	1.30123438924168\\
39	1.2837924608659\\
40	1.28607151104237\\
41	1.30392570658531\\
42	1.30294019424938\\
43	1.31120082470535\\
44	1.32390362771957\\
45	1.32541657410313\\
46	1.3262644949915\\
47	1.31993286272378\\
48	1.32639311860432\\
49	1.31821900719825\\
50	1.28732294365913\\
51	1.25182447516481\\
52	1.23394614785194\\
53	1.21669932649709\\
54	1.21668749615975\\
55	1.21073096424476\\
56	1.20148832552316\\
57	1.18976869504367\\
58	1.17471804574536\\
59	1.16042155186705\\
60	1.16418450275587\\
61	1.17324828803957\\
62	1.1740418861083\\
63	1.17723240755037\\
64	1.19574306720078\\
65	1.20934713091111\\
66	1.22107931010755\\
67	1.23145739375454\\
68	1.23526417038167\\
69	1.2409271151999\\
70	1.25144881732671\\
71	1.2672435696603\\
72	1.27246692041182\\
73	1.27772178126981\\
74	1.29213084644373\\
75	1.30387732740094\\
76	1.30921511187256\\
77	1.30552920247674\\
78	1.30771915400996\\
79	1.30319389281101\\
80	1.29561611780554\\
81	1.30103292375159\\
82	1.29698920191084\\
83	1.2746319819584\\
84	1.24460857241181\\
85	1.20755934460159\\
86	1.17449858002907\\
87	1.15063811961778\\
88	1.13284092597224\\
89	1.12762011871701\\
90	1.1222850140768\\
91	1.11684211651071\\
92	1.11145106656912\\
93	1.11356194563118\\
94	1.1227171663465\\
95	1.13001916423242\\
96	1.12805269581643\\
97	1.13790876254854\\
98	1.14583678691329\\
99	1.14821214241068\\
100	1.16224635284299\\
101	1.17018178054349\\
102	1.17943889492973\\
103	1.19430023798335\\
104	1.21739151404489\\
105	1.23371732308363\\
106	1.24412427897177\\
107	1.25458341201408\\
108	1.26327688236898\\
109	1.26021014914866\\
110	1.25608569101866\\
111	1.2561278387899\\
112	1.25462360399642\\
113	1.26143372692282\\
114	1.27749288460231\\
115	1.2911786918617\\
116	1.29452011982027\\
117	1.30610585398596\\
118	1.30701628814191\\
119	1.30142103441802\\
120	1.31024532042618\\
121	1.30988679042938\\
122	1.3115943962872\\
123	1.31186565807201\\
124	1.31202660754975\\
125	1.30620612641171\\
126	1.31600135993347\\
127	1.31369265176844\\
128	1.30716804363585\\
129	1.30899529135284\\
130	1.31262078289126\\
131	1.31820129253769\\
132	1.32572522063333\\
133	1.33620345488492\\
134	1.34715104807921\\
135	1.34805345971093\\
136	1.34834129204849\\
137	1.35207893721506\\
138	1.34766155437688\\
139	1.35075944793535\\
140	1.35311387503339\\
141	1.3562339946457\\
142	1.35809140374008\\
143	1.36238849923736\\
144	1.3679077115353\\
145	1.37445654070824\\
146	1.36680663970069\\
147	1.35499244110152\\
148	1.35318948921878\\
149	1.34666115771398\\
150	1.31364943542377\\
151	1.2585394203441\\
152	1.22902697469389\\
153	1.21845188177336\\
154	1.21406114203822\\
155	1.21202578715837\\
156	1.19774673698975\\
157	1.18574919219055\\
158	1.18097103811255\\
159	1.18431195915916\\
160	1.19275640140999\\
161	1.15092193176139\\
162	1.10502737628506\\
163	1.0772620320447\\
164	1.07787204143824\\
165	1.08984464069319\\
166	1.10982079879943\\
167	1.12404506038799\\
168	1.14007015870604\\
169	1.16389023740297\\
170	1.1752758850003\\
171	1.17167688369368\\
172	1.17065588522446\\
173	1.16605892503907\\
174	1.14985717025828\\
175	1.14375650815499\\
176	1.14558570091022\\
177	1.15053267231866\\
178	1.15672926612912\\
179	1.15973412943352\\
180	1.16799753487505\\
181	1.17912728039908\\
182	1.15990749054458\\
183	1.14804548942203\\
184	1.1359407049366\\
185	1.13682418916546\\
186	1.14885448485626\\
187	1.16118656762814\\
188	1.17496702659245\\
189	1.18454167153116\\
190	1.21244857106613\\
191	1.22251606730946\\
192	1.22512921394431\\
193	1.20592563569171\\
194	1.19850923182852\\
195	1.19102560504713\\
196	1.18825838284089\\
197	1.19375816643695\\
198	1.20273324836926\\
199	1.20822995090188\\
200	1.22854081645038\\
201	1.24271318304384\\
202	1.24679068649959\\
203	1.25144471903663\\
204	1.25940044505045\\
205	1.27424822111323\\
206	1.28112444671547\\
207	1.26536784037695\\
208	1.21612004279783\\
209	1.16503968159838\\
210	1.12505524306679\\
211	1.09845030036643\\
212	1.08474690437542\\
213	1.0648968695861\\
214	1.05737968775243\\
215	1.05113512107713\\
216	1.0443593004163\\
217	1.03783875538596\\
218	1.01917007643994\\
219	1.00009619821838\\
220	0.995379351814729\\
221	0.992636205448896\\
222	0.998848569412455\\
223	1.00248149531673\\
224	1.00830124926414\\
225	1.00809948042267\\
226	1.01539468793329\\
227	1.02479745390491\\
228	1.04853774111795\\
229	1.07847282610469\\
230	1.09928308478993\\
231	1.11601848379295\\
232	1.13194794225074\\
233	1.14074837206586\\
234	1.15117768845741\\
235	1.1502265970017\\
236	1.1425238161564\\
237	1.1236309836702\\
238	1.10236381122735\\
239	1.08237572256991\\
240	1.06312658464144\\
241	1.04530315111307\\
242	1.00974068134546\\
243	0.973994134764198\\
244	0.960122425549312\\
245	0.957653442051226\\
246	0.963919937435486\\
247	0.975130408731202\\
248	0.994748219519507\\
249	1.01547810128979\\
250	1.02588456427\\
251	1.02019435238804\\
252	1.02447002600892\\
253	1.02256101577681\\
254	1.02949303364077\\
255	1.03707730046379\\
256	1.03416944573007\\
257	1.00766431752505\\
258	0.984891315172397\\
259	0.971214269715971\\
260	0.972003688425872\\
261	0.983923826754779\\
262	1.01471017858903\\
263	1.03302857714438\\
264	1.04095236684625\\
265	1.05984280428395\\
266	1.08530450422622\\
267	1.1099512239475\\
268	1.13346975743553\\
269	1.15417466483482\\
270	1.16757189754925\\
271	1.13014224914528\\
272	1.06640551124578\\
273	1.01015319661978\\
274	0.989072517541616\\
275	0.984487425902649\\
276	0.991003920005523\\
277	1.00066165569558\\
278	1.01703230678697\\
279	1.04211274238426\\
280	1.05072117649336\\
281	1.0657326590446\\
282	1.07410600552312\\
283	1.07748909984713\\
284	1.09442644290124\\
285	1.11259295321099\\
286	1.13598617742755\\
287	1.16359156216575\\
288	1.18115813463152\\
289	1.19369684707326\\
290	1.20390851433333\\
291	1.21977895707584\\
292	1.23104895644311\\
293	1.24682084655725\\
294	1.261346962579\\
295	1.27983667464532\\
296	1.29362321194864\\
297	1.29374094877737\\
298	1.29843090396843\\
299	1.3078399029345\\
300	1.31274797142237\\
301	1.32347833274162\\
302	1.3323096090276\\
303	1.33321576925125\\
304	1.33601310420379\\
305	1.33771660488359\\
306	1.35040010303734\\
307	1.31069302083171\\
308	1.22800419513589\\
309	1.15190849919978\\
310	1.09969289412819\\
311	1.07671859428309\\
312	1.06671704991014\\
313	1.06272068076557\\
314	1.06659945380418\\
315	1.09021367486062\\
316	1.10135133719026\\
317	1.10296032783185\\
318	1.10726862553291\\
319	1.11576635679279\\
320	1.10112547717511\\
321	1.08682899665711\\
322	1.10697455945712\\
323	1.12854803908115\\
324	1.15330736241661\\
325	1.17323291286464\\
326	1.19474929058324\\
327	1.20837736529977\\
328	1.20818597833319\\
329	1.20358835100034\\
330	1.16709251908813\\
331	1.1228479052747\\
332	1.08573294906306\\
333	1.05817546762286\\
334	1.04910869927223\\
335	1.04149232620401\\
336	1.03533660444157\\
337	1.03697350646253\\
338	1.04514922527237\\
339	1.06791446652973\\
340	1.08705821067694\\
341	1.10688458755816\\
342	1.11949384132317\\
343	1.14440635308522\\
344	1.16234602316775\\
345	1.17590621212213\\
346	1.19073697376326\\
347	1.20741321889366\\
348	1.22183785297511\\
349	1.23277742092268\\
350	1.22866400633349\\
351	1.20099615617137\\
352	1.17612893987845\\
353	1.15974152262115\\
354	1.14668171668395\\
355	1.13712350719063\\
356	1.13539860277028\\
357	1.1365074440854\\
358	1.15492036238684\\
359	1.16723283034795\\
360	1.18509088957508\\
361	1.2004297639336\\
362	1.20697255914659\\
363	1.21730577179756\\
364	1.22164245064138\\
365	1.22503348048451\\
366	1.21595801312285\\
367	1.20526506714012\\
368	1.19152754100632\\
369	1.16144659509527\\
370	1.1280778155421\\
371	1.10855664740496\\
372	1.10039921994746\\
373	1.09574815583892\\
374	1.09334903572699\\
375	1.09632598481112\\
376	1.09638508694694\\
377	1.10197575799563\\
378	1.10048083376832\\
379	1.10875394081308\\
380	1.12163270957338\\
381	1.11363862987423\\
382	1.10735151181985\\
383	1.11147321543138\\
384	1.1173688313418\\
385	1.13463206901499\\
386	1.15631574333174\\
387	1.17903822963844\\
388	1.18893176213893\\
389	1.19248141023181\\
390	1.20450147046874\\
391	1.2181687373048\\
392	1.24031999920838\\
393	1.25897093880398\\
394	1.27785350121712\\
395	1.28610521390485\\
396	1.28451160231703\\
397	1.29272861944829\\
398	1.30250805779956\\
399	1.31045030422879\\
400	1.32045709771232\\
401	1.32648290447176\\
402	1.33115258259604\\
403	1.3358227809474\\
404	1.33983561915181\\
405	1.34547745969812\\
406	1.35337172962997\\
407	1.35235699043434\\
408	1.35401992669316\\
409	1.35002498025965\\
410	1.33194174925997\\
411	1.32785875093832\\
412	1.33072290464983\\
413	1.3046217211233\\
414	1.24084313686638\\
415	1.1914964439558\\
416	1.1641307527261\\
417	1.14458061062339\\
418	1.12883708331684\\
419	1.10614077655724\\
420	1.09081320139254\\
421	1.08752241207075\\
422	1.0803815882871\\
423	1.06671942147125\\
424	1.0530138032378\\
425	1.0445680400201\\
426	1.04803420089938\\
427	1.05838425310446\\
428	1.07152132520039\\
429	1.07798447675402\\
430	1.07271064849309\\
431	1.0729906443745\\
432	1.05685107156986\\
433	1.04330344583191\\
434	1.04738696601432\\
435	1.07173464797463\\
436	1.0963032389313\\
437	1.12202249471744\\
438	1.14340793627133\\
439	1.15993365912439\\
440	1.16826899604696\\
441	1.18089894599371\\
442	1.19888992435423\\
443	1.21178170591112\\
444	1.22141094912531\\
445	1.21031320024628\\
446	1.18942902402325\\
447	1.16727608001207\\
448	1.15805075875658\\
449	1.16592293384438\\
450	1.17917202944192\\
451	1.18804023266324\\
452	1.1994995676204\\
453	1.20009432907378\\
454	1.20065811488502\\
455	1.20310267226842\\
456	1.21690270342501\\
457	1.22676889454232\\
458	1.23483959292237\\
459	1.24177241660323\\
460	1.23861071608495\\
461	1.24299102752007\\
462	1.22948422475963\\
463	1.21696847896477\\
464	1.21462664083151\\
465	1.22898109473643\\
466	1.23512563563647\\
467	1.24663628693524\\
468	1.25651057717163\\
469	1.26626225008473\\
470	1.27978140713986\\
471	1.28921153937087\\
472	1.2874115923315\\
473	1.29012610309966\\
474	1.28515573521963\\
475	1.28872794854834\\
476	1.29374758384088\\
477	1.28965439990201\\
478	1.28733550002703\\
479	1.27811558997736\\
480	1.28240431450107\\
481	1.28881255106303\\
482	1.28953811928019\\
483	1.28350333157717\\
484	1.28948572480196\\
485	1.30591044689622\\
486	1.31738339232048\\
487	1.31331558297936\\
488	1.32561102675279\\
489	1.32618024430035\\
490	1.3321763156121\\
491	1.32535591821464\\
492	1.31323430367735\\
493	1.30526905232051\\
494	1.30842764799105\\
495	1.29774618647573\\
496	1.30285466786975\\
497	1.30461137738809\\
};
\addplot [color=darkgray, line width=1.0pt, forget plot]
  table[row sep=crcr]{%
1	5\\
2	4.99979560825415\\
3	4.99957817089504\\
4	4.99931803382665\\
5	4.9990102291445\\
6	4.99866441153542\\
7	4.99828181053583\\
8	4.99783410649416\\
9	4.99735187934126\\
10	4.99683440472917\\
11	4.99628731040431\\
12	4.99571747890355\\
13	4.99508172866767\\
14	4.99435091923998\\
15	4.99351432714529\\
16	4.99260021433514\\
17	4.99159616631599\\
18	4.99048300224002\\
19	4.98921486743556\\
20	4.98772977890902\\
21	4.98618482199976\\
22	4.98447771037939\\
23	4.98261896983875\\
24	4.98049206530061\\
25	4.97820492978545\\
26	4.97577022958429\\
27	4.97302699709714\\
28	4.96993027702232\\
29	4.9665155779798\\
30	4.96283310648101\\
31	4.9585844020267\\
32	4.95390122443799\\
33	4.94871578400701\\
34	4.94304764428524\\
35	4.93648085815822\\
36	4.92915559071942\\
37	4.92108027715279\\
38	4.91224682579882\\
39	4.90239389677068\\
40	4.89193181848812\\
41	4.88090072267366\\
42	4.86821517372966\\
43	4.85441642785201\\
44	4.83950942587364\\
45	4.82263368062305\\
46	4.80395071633149\\
47	4.78313307778396\\
48	4.76119596688547\\
49	4.73622025737451\\
50	4.70709775602847\\
51	4.67456124005101\\
52	4.6403212387739\\
53	4.60321808520514\\
54	4.56465159349467\\
55	4.52220273236853\\
56	4.47575124834137\\
57	4.42535884947056\\
58	4.3699218206938\\
59	4.31040776769623\\
60	4.25022559564892\\
61	4.18750324679961\\
62	4.11971599912964\\
63	4.04861006855397\\
64	3.97795574488425\\
65	3.90300659614945\\
66	3.82522648919275\\
67	3.74536310361232\\
68	3.65968816082319\\
69	3.57130440451416\\
70	3.48357219146037\\
71	3.39721515530902\\
72	3.30607500800212\\
73	3.21441230381462\\
74	3.12773066312818\\
75	3.04002932546533\\
76	2.95127129473603\\
77	2.8575909088672\\
78	2.76858535405579\\
79	2.67845553033695\\
80	2.58873875660426\\
81	2.50967915989028\\
82	2.42706127298053\\
83	2.3338335397398\\
84	2.2377741861415\\
85	2.13677478777813\\
86	2.04135599627833\\
87	1.95747342136383\\
88	1.88101415567777\\
89	1.81977092709182\\
90	1.76092566326032\\
91	1.70443449117915\\
92	1.65178608289109\\
93	1.61130297965023\\
94	1.58082335019988\\
95	1.55183872077203\\
96	1.51635019322387\\
97	1.49442445292926\\
98	1.47322990915961\\
99	1.4475901467325\\
100	1.43558786741521\\
101	1.42173992411686\\
102	1.40873874399675\\
103	1.40466513429124\\
104	1.41034492331122\\
105	1.40932179846547\\
106	1.40577732652032\\
107	1.4034045848215\\
108	1.40098053642517\\
109	1.38606416099836\\
110	1.37207416196286\\
111	1.36339294560013\\
112	1.3533378769937\\
113	1.35211678809805\\
114	1.361541203877\\
115	1.36922396029829\\
116	1.36676928514396\\
117	1.37299559351038\\
118	1.37090522535238\\
119	1.36186604287544\\
120	1.36513953992609\\
121	1.36009809104931\\
122	1.35931177515327\\
123	1.35644825089534\\
124	1.35329629985864\\
125	1.34394684918277\\
126	1.35195660493872\\
127	1.34682151112816\\
128	1.33782455335224\\
129	1.33896569600751\\
130	1.34179380473819\\
131	1.34677178514284\\
132	1.35222154212317\\
133	1.36152013106172\\
134	1.37050970786941\\
135	1.36948340827942\\
136	1.36797706816325\\
137	1.37113956718013\\
138	1.36538893630363\\
139	1.36758640998631\\
140	1.36796165692405\\
141	1.37015941740792\\
142	1.37115640304032\\
143	1.37476139737583\\
144	1.38045363092437\\
145	1.38738178812236\\
146	1.3799036676611\\
147	1.36714986714842\\
148	1.36485404766856\\
149	1.35745489533942\\
150	1.32371218369277\\
151	1.26756674829126\\
152	1.23744003923376\\
153	1.22721735514685\\
154	1.22225455780303\\
155	1.2200376084969\\
156	1.20480508826479\\
157	1.19203679782202\\
158	1.18721775439569\\
159	1.19076683180833\\
160	1.19878364762867\\
161	1.1563617677816\\
162	1.1101282551515\\
163	1.08160995936386\\
164	1.08153133093121\\
165	1.09309539266877\\
166	1.1137529209736\\
167	1.12762583043209\\
168	1.1436328786385\\
169	1.16716464563275\\
170	1.17859670524053\\
171	1.1746585643705\\
172	1.17399926085641\\
173	1.1693037664097\\
174	1.1532403413709\\
175	1.14674305409129\\
176	1.14819327216859\\
177	1.15316940872224\\
178	1.15959400609372\\
179	1.16192168313533\\
180	1.17024477608534\\
181	1.18121509809258\\
182	1.16141219784472\\
183	1.14897249731623\\
184	1.13631744176028\\
185	1.13711089466785\\
186	1.14932664621836\\
187	1.16157920944409\\
188	1.1752708176227\\
189	1.1847366013382\\
190	1.21242245661718\\
191	1.22281398392106\\
192	1.22544763327467\\
193	1.20657214661392\\
194	1.19918086454941\\
195	1.19213873635074\\
196	1.18951816288711\\
197	1.19502741972236\\
198	1.20406504516023\\
199	1.20934588712294\\
200	1.22952666480164\\
201	1.24368069182852\\
202	1.24840382544435\\
203	1.25354315931607\\
204	1.26160312630128\\
205	1.2770257141232\\
206	1.28387212104677\\
207	1.26795672466606\\
208	1.21871934546699\\
209	1.16741682833404\\
210	1.12728752158408\\
211	1.1004818659214\\
212	1.08685678092716\\
213	1.0669848294721\\
214	1.05926661717076\\
215	1.05286495848693\\
216	1.04607108544003\\
217	1.03920931644863\\
218	1.02064332353347\\
219	1.00154685901617\\
220	0.996606017276507\\
221	0.993679192527029\\
222	0.999918419697488\\
223	1.00352909556758\\
224	1.00924244240008\\
225	1.00867834682529\\
226	1.01569999779128\\
227	1.02531471374541\\
228	1.04912432126812\\
229	1.07928252316307\\
230	1.10004128572242\\
231	1.11691656812495\\
232	1.13309568421938\\
233	1.14172357828608\\
234	1.15228281482122\\
235	1.15149262004026\\
236	1.14370718652597\\
237	1.12451956236583\\
238	1.10316678980026\\
239	1.08320054305667\\
240	1.0636233420706\\
241	1.04595636670204\\
242	1.01030704775868\\
243	0.97447969224601\\
244	0.960585485077732\\
245	0.957814163795398\\
246	0.963850959727442\\
247	0.975160892188067\\
248	0.994704452609091\\
249	1.0156616259327\\
250	1.02616851757238\\
251	1.02030111193158\\
252	1.02469602407184\\
253	1.02254232654676\\
254	1.02932507408855\\
255	1.03681740964825\\
256	1.03391269943235\\
257	1.00749846508904\\
258	0.984749337224407\\
259	0.97119706157934\\
260	0.97215165412872\\
261	0.983986172651194\\
262	1.01472367632919\\
263	1.03285507116065\\
264	1.04080850801534\\
265	1.05966851260375\\
266	1.08514642229828\\
267	1.11000673930044\\
268	1.13332514251423\\
269	1.1542645856223\\
270	1.16784813844789\\
271	1.13043550696128\\
272	1.06667838132504\\
273	1.01050219549311\\
274	0.989378638974166\\
275	0.984872993815187\\
276	0.991136020444534\\
277	1.00093527721059\\
278	1.0174257747748\\
279	1.04239594444127\\
280	1.05120968967208\\
281	1.06624239479047\\
282	1.07467009309116\\
283	1.07791494098933\\
284	1.09510855863565\\
285	1.1132466687941\\
286	1.13664627097317\\
287	1.16452871316549\\
288	1.18218916355021\\
289	1.19454596785649\\
290	1.2048806345728\\
291	1.22086963479247\\
292	1.23208158733591\\
293	1.24780582655163\\
294	1.2623287206851\\
295	1.28066460230024\\
296	1.29431840168254\\
297	1.29436027790638\\
298	1.29894439028968\\
299	1.30851773899364\\
300	1.31339438125621\\
301	1.3241760683836\\
302	1.33298719938639\\
303	1.33380157152947\\
304	1.33664140191794\\
305	1.33840254378475\\
306	1.35126118312038\\
307	1.31155824059342\\
308	1.22871694626918\\
309	1.15250709758972\\
310	1.10016245332748\\
311	1.07720962752606\\
312	1.06710909616819\\
313	1.06305450591628\\
314	1.06699463282645\\
315	1.09060151455727\\
316	1.10156620566254\\
317	1.10320541869978\\
318	1.1073477208326\\
319	1.11598578655072\\
320	1.10134301607116\\
321	1.08711273682529\\
322	1.10728381269747\\
323	1.12902459368939\\
324	1.15377268795025\\
325	1.17362200557502\\
326	1.1952539903219\\
327	1.20899367325193\\
328	1.20887099185022\\
329	1.20426186091991\\
330	1.16775720975045\\
331	1.12339245003839\\
332	1.08620325159656\\
333	1.05860234498388\\
334	1.0493801613496\\
335	1.04176645552274\\
336	1.03555180054912\\
337	1.03720677058865\\
338	1.04543241053662\\
339	1.06812003429862\\
340	1.08720557938963\\
341	1.10702754728176\\
342	1.11962568942709\\
343	1.14456210190489\\
344	1.16260722558988\\
345	1.17607234498651\\
346	1.19094315436667\\
347	1.20768529358298\\
348	1.22211023831306\\
349	1.23293461647757\\
350	1.22879406919198\\
351	1.20117883962682\\
352	1.17622535795811\\
353	1.15977228087447\\
354	1.14658315503738\\
355	1.13703382061809\\
356	1.13537592895571\\
357	1.13657510417575\\
358	1.15508646267148\\
359	1.16738904482516\\
360	1.18533069155407\\
361	1.20063883423351\\
362	1.20731534977768\\
363	1.2176718301172\\
364	1.22191651595826\\
365	1.22525475392432\\
366	1.21613318914091\\
367	1.20538101385942\\
368	1.19163221674466\\
369	1.16140956966612\\
370	1.12797401266925\\
371	1.10854122804696\\
372	1.1003049173443\\
373	1.09557339896803\\
374	1.09317276768291\\
375	1.09612225723433\\
376	1.09618068006406\\
377	1.10185427370272\\
378	1.1004473499378\\
379	1.10867081769483\\
380	1.12155731126761\\
381	1.11362322466608\\
382	1.10728096531016\\
383	1.1114538242803\\
384	1.117292682163\\
385	1.13454122898961\\
386	1.15625989542645\\
387	1.17887481272191\\
388	1.18876681028486\\
389	1.1923791011901\\
390	1.20442566077579\\
391	1.21807209826371\\
392	1.24026713958953\\
393	1.25897630957689\\
394	1.27793784730728\\
395	1.28628598467025\\
396	1.28470243184913\\
397	1.29292720033869\\
398	1.30264387858449\\
399	1.31070854550371\\
400	1.32072852545256\\
401	1.32668657397658\\
402	1.33141210697143\\
403	1.33605186510219\\
404	1.34003091103327\\
405	1.34578877257628\\
406	1.35364976595428\\
407	1.35264810917914\\
408	1.35431046255889\\
409	1.35026418513549\\
410	1.33220570743379\\
411	1.3281380071547\\
412	1.33103244610753\\
413	1.30490621088257\\
414	1.24117439414071\\
415	1.19186209854001\\
416	1.16439833677876\\
417	1.14481673120073\\
418	1.12913557082126\\
419	1.10644553878917\\
420	1.09116826933168\\
421	1.08788628728621\\
422	1.08073739123698\\
423	1.06707706063752\\
424	1.05318359736858\\
425	1.04475798645979\\
426	1.04816168367837\\
427	1.05837451317336\\
428	1.07152344231717\\
429	1.07797197721981\\
430	1.07264004312564\\
431	1.07296262306161\\
432	1.05684424777366\\
433	1.04331531910354\\
434	1.04735585987644\\
435	1.07173370865007\\
436	1.09634723122354\\
437	1.12201889247822\\
438	1.14348039265747\\
439	1.15983146926463\\
440	1.16823207524697\\
441	1.18071658919665\\
442	1.19873596271424\\
443	1.21168989134957\\
444	1.22139130390043\\
445	1.21031856062386\\
446	1.1894199015394\\
447	1.16734880435167\\
448	1.15817728559834\\
449	1.16602926241145\\
450	1.17930811483379\\
451	1.1881463648976\\
452	1.19947766850145\\
453	1.2000588886881\\
454	1.20058635745804\\
455	1.20304982144314\\
456	1.21684688550151\\
457	1.22675398430603\\
458	1.23482486572305\\
459	1.24168257413419\\
460	1.23859891916475\\
461	1.24300031766878\\
462	1.22948966522683\\
463	1.21699390582528\\
464	1.2146142813735\\
465	1.22887030690495\\
466	1.23505320787221\\
467	1.24657132572159\\
468	1.25648790561744\\
469	1.26632310052821\\
470	1.27976302878931\\
471	1.28926154920703\\
472	1.28749513661374\\
473	1.29025900880635\\
474	1.28535454553259\\
475	1.2888632109491\\
476	1.293795791656\\
477	1.28959223315862\\
478	1.28724285714731\\
479	1.27800222941421\\
480	1.2822959375895\\
481	1.28873407897767\\
482	1.28946328166424\\
483	1.2834172835737\\
484	1.2893405475267\\
485	1.30586013970556\\
486	1.31737132404239\\
487	1.31326440895203\\
488	1.32554843606405\\
489	1.32606085653895\\
490	1.33207069098665\\
491	1.32523901366483\\
492	1.31306191020468\\
493	1.30494484694329\\
494	1.30810410816086\\
495	1.29742487619471\\
496	1.30259937277116\\
497	1.3043846215023\\
};
\end{axis}
\end{tikzpicture}%
	}
	\caption[Estimated Variance over Time]{Estimated Variance over Time: \itshape Curves of the same color only differ in the initial value of the variance $\sigma^{2,(0)}$. The red curve shows the variance estimates of \gls{crem} across time, while the gray curve shows the variance estimates of \gls{trem}. These estimates correspond to the "Crossing Movement"-Scenario with \nolinebreak{$\text{T}_{60}=0.4$~s}, but the convergence behaviour visible here is the same across all other evaluation scenarios.}
	\label{fig:trackVarComp}
\end{figure}

As has been shown in \autoref{sec:algSrcTrack}, the \gls{trem} and \gls{crem} variants of the tracking algorithm only differ in the estimation of the variance parameter. \autoref{fig:trackVarComp} shows, how both algorithms update $\sigma^2$ for different initial values $\sigma^{2,(0)}\ \in\ \{0.1, 0.5, 1, 2, 5\}$ over time. Note, that the time-axis is now given in seconds to make it easier to align the estimated parameters over time with the source signals shown in \autoref{fig:signalRepresentation}. The progression of the variance estimate over time shows, that \gls{crem} converges significantly faster from the initial value to the variance inherent to the data. While \gls{crem} seems to follow the inherent variance from $t=0.5$ onwards, \gls{trem} takes until $t=1$ to fully converge for an initial variance of $\sigma^{2,(0)}=5$. This means, that an initial variance that is far from the inherent variance will lead to inferior localisation performance with the \gls{trem} algorithm in the beginning, while the variance estimation of the \gls{crem} algorithm is more robust to such an initialisation. After both algorithms variance estimates have converged onto the inherent variance, the estimates are the same for both \gls{crem} and \gls{trem}. So the difference in variance estimation only has an effect on the convergence speed. In conclusion, if the initial variance is choosen to be close to the inherent variance, variance estimation yields the same results for both algorithms. Should the initial variance deviate from the variance inherent to the data, \gls{crem} offers variance estimates that converge faster than \gls{trem} in the beginning. After a certain amount of timesteps have been processed, both algorithms have converged onto the same variance estimates, that seem to range from $\sigma^2=1$ to $\sigma^2=1.5$ for the scenario of two sources moving along crossing paths with the reverberation time set to \Tsixty$=0.4$. As both algorithms produce the same variance estimates after they have converged, it can be assumed that the location estimates will be similar as well.

\setlength{\figureheight}{4cm}
\newcommand{\trajOpacity}{0.2}
\newcommand{\trajSize}{2pt}
\newcommand{\trajLinewidth}{0.5pt}
\newcommand{\estOpacity}{0.2}
\newcommand{\estSize}{4pt}
\newcommand{\estLinewidth}{0.5pt}
\newcommand{\trajDashedLinewidth}{2pt}

\subsubsection{Dynamic Scenario Evaluation}

In the following section, the three movement scenarios \emph{parallel}, \emph{crossing} and \emph{arc} defined in \autoref{sec:evalScenariosTracking} are evaluated for both \gls{crem} and \gls{trem}. The results will be presented in individual plots for x- and y-coordinate estimates. The dashed lines indicate the true source trajectories and the red markers indicate the coordinate estimates over time. Additionally, a third plot will show the estimated positions, where time has been coded as a color gradient from blue to red. But first, the two source signals should be examined to see, how their speech activity progresses over time. \autoref{fig:signalRepresentation} in the Appendix shows, that the first source signal has a distinctive pause around $t=3$. Further, the STFT-representation shows, that the lower frequencies are missing around $t=1$. The second source signal has two pauses, one at $t=1.4$ and the other at $t=4.7$ until the end. It is expected that source tracking will exhibit difficulties estimating the true source trajectories in those instances.

\togglefalse{quick}
%% PARALLEL MOVEMENT 
\subsubsection*{Parallel Movement}
In the parallel movement scenario, the first source moves from $\bm p^{(0)}_{s=1}=[4~2]$ up to $\bm p^{(T)}_{s=1}=[4~4]$, while the second source moves from $\bm p^{(0)}_{s=2}=[2~4]$ down to $\bm p^{(T)}_{s=2}=[2~2]$. From a top-down view, the source on the left is moving down while the source on the right is moving up. This is the reason, why the true source trajectories cross in the y-dimension, despite the sources moving parallel to each other. The trajectory length is $2.82$~m, therefore the sources move along the trajectory at a speed of $0.564\frac{m}{s}$. 

\begin{figure}[!htbp]
\iftoggle{quick}{%
    \includegraphics[width=\textwidth, height=\figureheight]{plots/tracking/parallel/results-T60=0.4-crem-xy}
}{%
	\begin{subfigure}{0.49\textwidth}
	     \centering
        \setlength{\figurewidth}{0.8\textwidth}
        % This file was created by matlab2tikz.
%
\definecolor{lms_red}{rgb}{0.80000,0.20780,0.21960}%
\definecolor{mycolor2}{rgb}{0.80000,0.20784,0.21961}%
\definecolor{mycolor3}{rgb}{0.92900,0.69400,0.12500}%
\definecolor{mycolor4}{rgb}{0.49400,0.18400,0.55600}%
%
\begin{tikzpicture}

\begin{axis}[%
width=0.951\figurewidth,
height=\figureheight,
at={(0\figurewidth,0\figureheight)},
scale only axis,
xmin=0,
xmax=496,
xtick={0,99.2,198.4,297.6,396.8,496},
xticklabels={{0},{1},{2},{3},{4},{5}},
xlabel style={font=\color{white!15!black}},
xlabel={t},
ymin=1,
ymax=5,
ylabel style={font=\color{white!15!black}},
ylabel={$p_x^{(t)}$~[m]},
axis background/.style={fill=white},
axis x line*=bottom,
axis y line*=left,
xmajorgrids,
ymajorgrids
]
\addplot [color=mycolor2, draw=none, mark=x, mark options={solid, mycolor2}, forget plot]
  table[row sep=crcr]{%
1	3.9\\
2	3.9\\
3	3.9\\
4	3.9\\
5	3.9\\
6	3.9\\
7	3.9\\
8	4\\
9	4\\
10	4\\
11	3.9\\
12	4\\
13	4\\
14	4\\
15	3.9\\
16	3.9\\
17	3.9\\
18	2\\
19	2\\
20	2\\
21	2\\
22	2\\
23	2\\
24	2\\
25	2\\
26	2\\
27	2\\
28	2\\
29	2\\
30	2\\
31	2\\
32	2\\
33	2\\
34	2\\
35	2\\
36	2\\
37	2\\
38	2\\
39	2\\
40	2\\
41	2\\
42	2\\
43	2\\
44	2\\
45	2\\
46	2\\
47	2\\
48	2\\
49	2\\
50	2\\
51	2\\
52	2\\
53	2\\
54	2\\
55	2\\
56	2\\
57	2\\
58	2\\
59	2\\
60	2\\
61	2\\
62	2\\
63	2\\
64	2\\
65	2\\
66	2\\
67	2\\
68	2\\
69	2\\
70	4\\
71	4\\
72	4\\
73	4\\
74	4\\
75	4\\
76	2\\
77	2\\
78	2\\
79	2\\
80	2\\
81	2\\
82	2\\
83	2\\
84	2\\
85	2\\
86	2\\
87	2\\
88	2\\
89	2\\
90	2\\
91	2\\
92	2\\
93	2\\
94	2\\
95	2\\
96	2\\
97	2\\
98	2\\
99	2\\
100	2\\
101	2\\
102	2\\
103	2\\
104	2\\
105	2\\
106	2\\
107	2\\
108	2\\
109	2\\
110	2\\
111	2\\
112	2\\
113	2\\
114	2\\
115	2\\
116	2\\
117	2\\
118	2\\
119	2\\
120	2\\
121	2\\
122	1.2\\
123	1.2\\
124	1.2\\
125	1.2\\
126	1.2\\
127	1.2\\
128	3.8\\
129	3.8\\
130	3.8\\
131	3.8\\
132	1.2\\
133	1.2\\
134	3.8\\
135	3.8\\
136	3.8\\
137	3.8\\
138	3.8\\
139	3.8\\
140	3.8\\
141	3.8\\
142	3.8\\
143	3.8\\
144	3.8\\
145	3.8\\
146	1.2\\
147	1.2\\
148	1.2\\
149	2\\
150	2\\
151	2\\
152	2\\
153	2\\
154	2\\
155	2\\
156	2\\
157	2\\
158	2\\
159	2\\
160	2\\
161	2\\
162	2\\
163	2\\
164	2\\
165	2\\
166	2\\
167	2\\
168	2\\
169	2\\
170	2\\
171	2\\
172	2\\
173	2\\
174	2\\
175	2\\
176	2\\
177	2\\
178	2\\
179	2\\
180	2\\
181	2\\
182	2\\
183	2\\
184	2\\
185	2\\
186	2\\
187	2\\
188	2\\
189	2\\
190	2\\
191	2\\
192	2\\
193	2\\
194	2\\
195	2\\
196	2\\
197	2\\
198	2\\
199	2\\
200	2\\
201	2\\
202	2\\
203	2\\
204	2\\
205	2\\
206	2\\
207	2\\
208	2\\
209	2\\
210	2\\
211	2\\
212	2\\
213	2\\
214	2\\
215	2\\
216	2\\
217	2\\
218	2\\
219	2\\
220	2\\
221	2\\
222	2\\
223	2\\
224	2\\
225	2\\
226	2.1\\
227	2.1\\
228	2.1\\
229	2.1\\
230	2.1\\
231	2.1\\
232	2.1\\
233	2.1\\
234	2.1\\
235	2.1\\
236	2.1\\
237	2.1\\
238	2.1\\
239	3.9\\
240	3.9\\
241	3.9\\
242	3.9\\
243	3.8\\
244	3.8\\
245	3.8\\
246	3.8\\
247	3.8\\
248	3.8\\
249	3.9\\
250	3.9\\
251	3.9\\
252	3.9\\
253	3.9\\
254	3.9\\
255	3.9\\
256	3.9\\
257	3.9\\
258	3.9\\
259	3.9\\
260	3.9\\
261	3.9\\
262	3.9\\
263	3.9\\
264	3.9\\
265	3.9\\
266	3.9\\
267	3.9\\
268	3.9\\
269	3.9\\
270	3.9\\
271	3.9\\
272	3.9\\
273	2.1\\
274	2.1\\
275	2.1\\
276	2.1\\
277	2.1\\
278	2.1\\
279	2.1\\
280	2.1\\
281	2.1\\
282	2.1\\
283	2.1\\
284	2.1\\
285	2.1\\
286	2.1\\
287	2.1\\
288	2.1\\
289	2.1\\
290	2.1\\
291	2.1\\
292	2.1\\
293	2.1\\
294	2.1\\
295	2.1\\
296	2.1\\
297	2.1\\
298	2.1\\
299	2.1\\
300	2.1\\
301	2.1\\
302	2.1\\
303	2.1\\
304	1.2\\
305	1.2\\
306	2.1\\
307	2\\
308	2.1\\
309	2.1\\
310	2.1\\
311	2\\
312	2\\
313	2\\
314	2\\
315	2\\
316	2\\
317	2\\
318	2.1\\
319	2.1\\
320	2.1\\
321	2.1\\
322	2.1\\
323	2.1\\
324	2.1\\
325	2.1\\
326	2.1\\
327	2.1\\
328	2.1\\
329	2.1\\
330	2.1\\
331	2.1\\
332	2.1\\
333	2.1\\
334	2.1\\
335	2.1\\
336	2.1\\
337	2.1\\
338	2.1\\
339	2.1\\
340	2.1\\
341	2.1\\
342	2.1\\
343	2.1\\
344	2.1\\
345	2.1\\
346	2.1\\
347	2.1\\
348	2.1\\
349	2.1\\
350	2.1\\
351	2.1\\
352	2.1\\
353	2.1\\
354	2.1\\
355	2.1\\
356	2.1\\
357	2.1\\
358	2.1\\
359	2.1\\
360	2.1\\
361	2.1\\
362	2.1\\
363	2.1\\
364	2.1\\
365	2\\
366	2\\
367	2.1\\
368	2\\
369	2.1\\
370	2.1\\
371	2\\
372	2\\
373	2\\
374	2\\
375	2\\
376	2\\
377	2\\
378	2\\
379	2\\
380	2\\
381	2\\
382	2.1\\
383	2.1\\
384	2.1\\
385	2.1\\
386	2.1\\
387	2.1\\
388	2\\
389	2\\
390	2\\
391	2\\
392	2.1\\
393	2.1\\
394	2.1\\
395	2.1\\
396	2.1\\
397	2.1\\
398	2.1\\
399	2.1\\
400	2.1\\
401	2.1\\
402	2.1\\
403	2.1\\
404	2.1\\
405	2.1\\
406	4\\
407	4\\
408	4\\
409	4\\
410	4\\
411	4\\
412	4\\
413	2\\
414	2\\
415	2\\
416	2\\
417	2\\
418	2\\
419	2\\
420	2\\
421	2\\
422	2\\
423	2\\
424	2\\
425	2\\
426	2\\
427	2\\
428	2\\
429	2\\
430	2\\
431	2\\
432	2\\
433	2\\
434	2\\
435	2\\
436	2\\
437	2\\
438	2\\
439	2\\
440	2\\
441	2\\
442	2\\
443	2\\
444	2\\
445	2\\
446	2\\
447	2\\
448	2\\
449	2\\
450	2\\
451	2\\
452	2\\
453	2\\
454	2\\
455	2\\
456	2\\
457	2\\
458	2\\
459	2\\
460	2\\
461	2\\
462	2\\
463	2\\
464	2\\
465	2\\
466	2\\
467	2\\
468	2\\
469	2\\
470	2\\
471	2\\
472	2\\
473	2\\
474	2\\
475	2\\
476	3.9\\
477	3.9\\
478	4\\
479	4\\
480	4\\
481	4\\
482	4\\
483	4\\
484	4\\
485	4\\
486	4\\
487	4\\
488	4\\
489	4\\
490	4\\
491	4\\
492	4\\
493	4\\
494	3.9\\
495	3.9\\
496	4\\
};
\addplot [color=mycolor2, draw=none, mark=x, mark options={solid, mycolor2}, forget plot]
  table[row sep=crcr]{%
1	3.9\\
2	3.9\\
3	3.9\\
4	3.9\\
5	3.9\\
6	3.9\\
7	3.9\\
8	3.9\\
9	3.9\\
10	3.9\\
11	3.9\\
12	3.9\\
13	3.9\\
14	3.9\\
15	3.9\\
16	2\\
17	2\\
18	3.9\\
19	3.9\\
20	3.9\\
21	3.9\\
22	3.9\\
23	3.9\\
24	3.9\\
25	3.9\\
26	3.9\\
27	3.9\\
28	3.9\\
29	3.9\\
30	3.9\\
31	3.9\\
32	3.9\\
33	3.9\\
34	3.9\\
35	3.9\\
36	3.9\\
37	3.9\\
38	3.9\\
39	3.9\\
40	3.9\\
41	3.9\\
42	3.9\\
43	3.9\\
44	3.9\\
45	3.9\\
46	3.9\\
47	3.9\\
48	3.9\\
49	3.9\\
50	3.9\\
51	3.9\\
52	3.9\\
53	3.9\\
54	3.9\\
55	3.9\\
56	4\\
57	4\\
58	4\\
59	4\\
60	4\\
61	4\\
62	4\\
63	4\\
64	4\\
65	4\\
66	4\\
67	3.9\\
68	4\\
69	4\\
70	2\\
71	2\\
72	2\\
73	2\\
74	2\\
75	2\\
76	4\\
77	4\\
78	4\\
79	4\\
80	4\\
81	1.2\\
82	1.2\\
83	1.2\\
84	1.2\\
85	1.2\\
86	1.2\\
87	1.2\\
88	1.2\\
89	1.2\\
90	1.2\\
91	1.2\\
92	1.2\\
93	1.2\\
94	1.2\\
95	1.2\\
96	1.2\\
97	1.2\\
98	1.2\\
99	1.2\\
100	1.2\\
101	1.2\\
102	1.2\\
103	1.2\\
104	1.2\\
105	1.2\\
106	2.2\\
107	2.2\\
108	2.2\\
109	2.2\\
110	2.2\\
111	2.2\\
112	2.2\\
113	2.2\\
114	2.2\\
115	2.2\\
116	2.2\\
117	1.2\\
118	2.2\\
119	2.2\\
120	2.2\\
121	1.2\\
122	2\\
123	2\\
124	2\\
125	3.8\\
126	3.8\\
127	3.8\\
128	1.2\\
129	1.2\\
130	1.2\\
131	1.2\\
132	3.8\\
133	3.8\\
134	1.2\\
135	1.2\\
136	1.2\\
137	1.2\\
138	1.2\\
139	1.2\\
140	1.2\\
141	1.2\\
142	4\\
143	4\\
144	1.2\\
145	1.2\\
146	3.8\\
147	3.8\\
148	3.8\\
149	1.2\\
150	1.2\\
151	1.2\\
152	1.2\\
153	1.2\\
154	1.2\\
155	1.2\\
156	1.2\\
157	1.2\\
158	1.2\\
159	1.2\\
160	1.2\\
161	1.2\\
162	1.2\\
163	1.2\\
164	1.2\\
165	1.2\\
166	1.2\\
167	4\\
168	4\\
169	1.2\\
170	3.9\\
171	3.9\\
172	3.9\\
173	3.9\\
174	1.2\\
175	3.9\\
176	3.9\\
177	3.9\\
178	3.9\\
179	3.9\\
180	3.9\\
181	3.9\\
182	3.9\\
183	3.9\\
184	4\\
185	4\\
186	4\\
187	4\\
188	1.2\\
189	1.2\\
190	1.2\\
191	1.2\\
192	1.3\\
193	1.2\\
194	1.2\\
195	1.2\\
196	1.2\\
197	1.2\\
198	1.2\\
199	1.2\\
200	1.2\\
201	1.2\\
202	1.2\\
203	1.2\\
204	1.2\\
205	1.2\\
206	1.2\\
207	1.2\\
208	1.2\\
209	1.2\\
210	1.2\\
211	4\\
212	4\\
213	4\\
214	4\\
215	4\\
216	4\\
217	4\\
218	3.9\\
219	3.9\\
220	3.9\\
221	3.9\\
222	3.9\\
223	3.9\\
224	3.9\\
225	3.9\\
226	3.9\\
227	3.9\\
228	3.9\\
229	3.9\\
230	3.9\\
231	3.9\\
232	3.9\\
233	3.9\\
234	3.9\\
235	3.8\\
236	3.8\\
237	3.9\\
238	3.9\\
239	2.1\\
240	2.1\\
241	2.1\\
242	2.1\\
243	2.1\\
244	2.1\\
245	2.1\\
246	2.1\\
247	2.1\\
248	2.1\\
249	2.1\\
250	2.1\\
251	2.1\\
252	2.1\\
253	2.1\\
254	2.1\\
255	2.1\\
256	2.1\\
257	2.1\\
258	2.1\\
259	2.1\\
260	2.1\\
261	2.1\\
262	2.1\\
263	2.1\\
264	2.1\\
265	2.1\\
266	2.1\\
267	2.1\\
268	2.1\\
269	2.1\\
270	2.1\\
271	2.1\\
272	2.1\\
273	3.9\\
274	3.9\\
275	3.9\\
276	3.9\\
277	3.9\\
278	3.9\\
279	3.9\\
280	3.9\\
281	3.9\\
282	3.9\\
283	3.9\\
284	3.9\\
285	3.9\\
286	3.9\\
287	3.9\\
288	3.9\\
289	3.9\\
290	3.9\\
291	3.9\\
292	3.9\\
293	3.9\\
294	3.9\\
295	3.9\\
296	3.9\\
297	3.9\\
298	3.9\\
299	3.9\\
300	3.9\\
301	1.2\\
302	1.2\\
303	1.2\\
304	2.1\\
305	2.1\\
306	1.2\\
307	1.2\\
308	1.2\\
309	1.2\\
310	1.2\\
311	1.2\\
312	1.2\\
313	1.2\\
314	1.2\\
315	1.2\\
316	1.2\\
317	1.2\\
318	1.2\\
319	1.2\\
320	3.8\\
321	3.8\\
322	3.8\\
323	3.7\\
324	3.7\\
325	3.7\\
326	3.7\\
327	3.7\\
328	3.7\\
329	3.7\\
330	3.7\\
331	3.7\\
332	3.7\\
333	3.7\\
334	3.7\\
335	3.7\\
336	3.7\\
337	3.9\\
338	3.9\\
339	3.7\\
340	3.7\\
341	3.7\\
342	3.9\\
343	3.9\\
344	3.9\\
345	3.9\\
346	3.9\\
347	3.8\\
348	3.8\\
349	3.9\\
350	3.8\\
351	3.8\\
352	3.8\\
353	3.8\\
354	3.9\\
355	4\\
356	4\\
357	4\\
358	4\\
359	4\\
360	4\\
361	4\\
362	4\\
363	4\\
364	4\\
365	4\\
366	3.9\\
367	3.9\\
368	3.9\\
369	3.9\\
370	3.9\\
371	3.9\\
372	4\\
373	3.9\\
374	3.9\\
375	4\\
376	4\\
377	3.9\\
378	3.9\\
379	3.9\\
380	3.9\\
381	3.9\\
382	3.9\\
383	3.9\\
384	3.9\\
385	3.9\\
386	3.9\\
387	4\\
388	4\\
389	4\\
390	4\\
391	4\\
392	4\\
393	4\\
394	4\\
395	3.9\\
396	3.9\\
397	4\\
398	3.9\\
399	4\\
400	4\\
401	3.9\\
402	4\\
403	4\\
404	4\\
405	4\\
406	2.1\\
407	2.1\\
408	2.1\\
409	2.1\\
410	2.1\\
411	2.1\\
412	2\\
413	4\\
414	4\\
415	4\\
416	4\\
417	4\\
418	4\\
419	3.9\\
420	2.2\\
421	2.2\\
422	3.9\\
423	3.9\\
424	3.9\\
425	3.9\\
426	3.9\\
427	3.9\\
428	3.9\\
429	3.9\\
430	3.9\\
431	3.9\\
432	3.9\\
433	3.9\\
434	3.9\\
435	3.9\\
436	3.9\\
437	3.9\\
438	3.9\\
439	4\\
440	4\\
441	4\\
442	3.9\\
443	3.9\\
444	3.9\\
445	3.9\\
446	3.9\\
447	3.9\\
448	3.9\\
449	3.9\\
450	3.9\\
451	3.9\\
452	3.9\\
453	4\\
454	4\\
455	4\\
456	4\\
457	4\\
458	3.9\\
459	3.9\\
460	3.9\\
461	3.9\\
462	3.9\\
463	3.9\\
464	3.9\\
465	3.9\\
466	3.9\\
467	3.9\\
468	3.9\\
469	3.9\\
470	3.9\\
471	3.9\\
472	3.9\\
473	3.9\\
474	3.9\\
475	3.9\\
476	2\\
477	2\\
478	2\\
479	2\\
480	2\\
481	2\\
482	2\\
483	2\\
484	2\\
485	2\\
486	2\\
487	2\\
488	2\\
489	2\\
490	2\\
491	2\\
492	2\\
493	2\\
494	2\\
495	2\\
496	2\\
};
\addplot [color=mycolor3, dashed, forget plot]
  table[row sep=crcr]{%
1	4\\
2	4\\
3	4\\
4	4\\
5	4\\
6	4\\
7	4\\
8	4\\
9	4\\
10	4\\
11	4\\
12	4\\
13	4\\
14	4\\
15	4\\
16	4\\
17	4\\
18	4\\
19	4\\
20	4\\
21	4\\
22	4\\
23	4\\
24	4\\
25	4\\
26	4\\
27	4\\
28	4\\
29	4\\
30	4\\
31	4\\
32	4\\
33	4\\
34	4\\
35	4\\
36	4\\
37	4\\
38	4\\
39	4\\
40	4\\
41	4\\
42	4\\
43	4\\
44	4\\
45	4\\
46	4\\
47	4\\
48	4\\
49	4\\
50	4\\
51	4\\
52	4\\
53	4\\
54	4\\
55	4\\
56	4\\
57	4\\
58	4\\
59	4\\
60	4\\
61	4\\
62	4\\
63	4\\
64	4\\
65	4\\
66	4\\
67	4\\
68	4\\
69	4\\
70	4\\
71	4\\
72	4\\
73	4\\
74	4\\
75	4\\
76	4\\
77	4\\
78	4\\
79	4\\
80	4\\
81	4\\
82	4\\
83	4\\
84	4\\
85	4\\
86	4\\
87	4\\
88	4\\
89	4\\
90	4\\
91	4\\
92	4\\
93	4\\
94	4\\
95	4\\
96	4\\
97	4\\
98	4\\
99	4\\
100	4\\
101	4\\
102	4\\
103	4\\
104	4\\
105	4\\
106	4\\
107	4\\
108	4\\
109	4\\
110	4\\
111	4\\
112	4\\
113	4\\
114	4\\
115	4\\
116	4\\
117	4\\
118	4\\
119	4\\
120	4\\
121	4\\
122	4\\
123	4\\
124	4\\
125	4\\
126	4\\
127	4\\
128	4\\
129	4\\
130	4\\
131	4\\
132	4\\
133	4\\
134	4\\
135	4\\
136	4\\
137	4\\
138	4\\
139	4\\
140	4\\
141	4\\
142	4\\
143	4\\
144	4\\
145	4\\
146	4\\
147	4\\
148	4\\
149	4\\
150	4\\
151	4\\
152	4\\
153	4\\
154	4\\
155	4\\
156	4\\
157	4\\
158	4\\
159	4\\
160	4\\
161	4\\
162	4\\
163	4\\
164	4\\
165	4\\
166	4\\
167	4\\
168	4\\
169	4\\
170	4\\
171	4\\
172	4\\
173	4\\
174	4\\
175	4\\
176	4\\
177	4\\
178	4\\
179	4\\
180	4\\
181	4\\
182	4\\
183	4\\
184	4\\
185	4\\
186	4\\
187	4\\
188	4\\
189	4\\
190	4\\
191	4\\
192	4\\
193	4\\
194	4\\
195	4\\
196	4\\
197	4\\
198	4\\
199	4\\
200	4\\
201	4\\
202	4\\
203	4\\
204	4\\
205	4\\
206	4\\
207	4\\
208	4\\
209	4\\
210	4\\
211	4\\
212	4\\
213	4\\
214	4\\
215	4\\
216	4\\
217	4\\
218	4\\
219	4\\
220	4\\
221	4\\
222	4\\
223	4\\
224	4\\
225	4\\
226	4\\
227	4\\
228	4\\
229	4\\
230	4\\
231	4\\
232	4\\
233	4\\
234	4\\
235	4\\
236	4\\
237	4\\
238	4\\
239	4\\
240	4\\
241	4\\
242	4\\
243	4\\
244	4\\
245	4\\
246	4\\
247	4\\
248	4\\
249	4\\
250	4\\
251	4\\
252	4\\
253	4\\
254	4\\
255	4\\
256	4\\
257	4\\
258	4\\
259	4\\
260	4\\
261	4\\
262	4\\
263	4\\
264	4\\
265	4\\
266	4\\
267	4\\
268	4\\
269	4\\
270	4\\
271	4\\
272	4\\
273	4\\
274	4\\
275	4\\
276	4\\
277	4\\
278	4\\
279	4\\
280	4\\
281	4\\
282	4\\
283	4\\
284	4\\
285	4\\
286	4\\
287	4\\
288	4\\
289	4\\
290	4\\
291	4\\
292	4\\
293	4\\
294	4\\
295	4\\
296	4\\
297	4\\
298	4\\
299	4\\
300	4\\
301	4\\
302	4\\
303	4\\
304	4\\
305	4\\
306	4\\
307	4\\
308	4\\
309	4\\
310	4\\
311	4\\
312	4\\
313	4\\
314	4\\
315	4\\
316	4\\
317	4\\
318	4\\
319	4\\
320	4\\
321	4\\
322	4\\
323	4\\
324	4\\
325	4\\
326	4\\
327	4\\
328	4\\
329	4\\
330	4\\
331	4\\
332	4\\
333	4\\
334	4\\
335	4\\
336	4\\
337	4\\
338	4\\
339	4\\
340	4\\
341	4\\
342	4\\
343	4\\
344	4\\
345	4\\
346	4\\
347	4\\
348	4\\
349	4\\
350	4\\
351	4\\
352	4\\
353	4\\
354	4\\
355	4\\
356	4\\
357	4\\
358	4\\
359	4\\
360	4\\
361	4\\
362	4\\
363	4\\
364	4\\
365	4\\
366	4\\
367	4\\
368	4\\
369	4\\
370	4\\
371	4\\
372	4\\
373	4\\
374	4\\
375	4\\
376	4\\
377	4\\
378	4\\
379	4\\
380	4\\
381	4\\
382	4\\
383	4\\
384	4\\
385	4\\
386	4\\
387	4\\
388	4\\
389	4\\
390	4\\
391	4\\
392	4\\
393	4\\
394	4\\
395	4\\
396	4\\
397	4\\
398	4\\
399	4\\
400	4\\
401	4\\
402	4\\
403	4\\
404	4\\
405	4\\
406	4\\
407	4\\
408	4\\
409	4\\
410	4\\
411	4\\
412	4\\
413	4\\
414	4\\
415	4\\
416	4\\
417	4\\
418	4\\
419	4\\
420	4\\
421	4\\
422	4\\
423	4\\
424	4\\
425	4\\
426	4\\
427	4\\
428	4\\
429	4\\
430	4\\
431	4\\
432	4\\
433	4\\
434	4\\
435	4\\
436	4\\
437	4\\
438	4\\
439	4\\
440	4\\
441	4\\
442	4\\
443	4\\
444	4\\
445	4\\
446	4\\
447	4\\
448	4\\
449	4\\
450	4\\
451	4\\
452	4\\
453	4\\
454	4\\
455	4\\
456	4\\
457	4\\
458	4\\
459	4\\
460	4\\
461	4\\
462	4\\
463	4\\
464	4\\
465	4\\
466	4\\
467	4\\
468	4\\
469	4\\
470	4\\
471	4\\
472	4\\
473	4\\
474	4\\
475	4\\
476	4\\
477	4\\
478	4\\
479	4\\
480	4\\
481	4\\
482	4\\
483	4\\
484	4\\
485	4\\
486	4\\
487	4\\
488	4\\
489	4\\
490	4\\
491	4\\
492	4\\
493	4\\
494	4\\
495	4\\
496	4\\
};
\addplot [color=mycolor4, dashed, forget plot]
  table[row sep=crcr]{%
1	2\\
2	2\\
3	2\\
4	2\\
5	2\\
6	2\\
7	2\\
8	2\\
9	2\\
10	2\\
11	2\\
12	2\\
13	2\\
14	2\\
15	2\\
16	2\\
17	2\\
18	2\\
19	2\\
20	2\\
21	2\\
22	2\\
23	2\\
24	2\\
25	2\\
26	2\\
27	2\\
28	2\\
29	2\\
30	2\\
31	2\\
32	2\\
33	2\\
34	2\\
35	2\\
36	2\\
37	2\\
38	2\\
39	2\\
40	2\\
41	2\\
42	2\\
43	2\\
44	2\\
45	2\\
46	2\\
47	2\\
48	2\\
49	2\\
50	2\\
51	2\\
52	2\\
53	2\\
54	2\\
55	2\\
56	2\\
57	2\\
58	2\\
59	2\\
60	2\\
61	2\\
62	2\\
63	2\\
64	2\\
65	2\\
66	2\\
67	2\\
68	2\\
69	2\\
70	2\\
71	2\\
72	2\\
73	2\\
74	2\\
75	2\\
76	2\\
77	2\\
78	2\\
79	2\\
80	2\\
81	2\\
82	2\\
83	2\\
84	2\\
85	2\\
86	2\\
87	2\\
88	2\\
89	2\\
90	2\\
91	2\\
92	2\\
93	2\\
94	2\\
95	2\\
96	2\\
97	2\\
98	2\\
99	2\\
100	2\\
101	2\\
102	2\\
103	2\\
104	2\\
105	2\\
106	2\\
107	2\\
108	2\\
109	2\\
110	2\\
111	2\\
112	2\\
113	2\\
114	2\\
115	2\\
116	2\\
117	2\\
118	2\\
119	2\\
120	2\\
121	2\\
122	2\\
123	2\\
124	2\\
125	2\\
126	2\\
127	2\\
128	2\\
129	2\\
130	2\\
131	2\\
132	2\\
133	2\\
134	2\\
135	2\\
136	2\\
137	2\\
138	2\\
139	2\\
140	2\\
141	2\\
142	2\\
143	2\\
144	2\\
145	2\\
146	2\\
147	2\\
148	2\\
149	2\\
150	2\\
151	2\\
152	2\\
153	2\\
154	2\\
155	2\\
156	2\\
157	2\\
158	2\\
159	2\\
160	2\\
161	2\\
162	2\\
163	2\\
164	2\\
165	2\\
166	2\\
167	2\\
168	2\\
169	2\\
170	2\\
171	2\\
172	2\\
173	2\\
174	2\\
175	2\\
176	2\\
177	2\\
178	2\\
179	2\\
180	2\\
181	2\\
182	2\\
183	2\\
184	2\\
185	2\\
186	2\\
187	2\\
188	2\\
189	2\\
190	2\\
191	2\\
192	2\\
193	2\\
194	2\\
195	2\\
196	2\\
197	2\\
198	2\\
199	2\\
200	2\\
201	2\\
202	2\\
203	2\\
204	2\\
205	2\\
206	2\\
207	2\\
208	2\\
209	2\\
210	2\\
211	2\\
212	2\\
213	2\\
214	2\\
215	2\\
216	2\\
217	2\\
218	2\\
219	2\\
220	2\\
221	2\\
222	2\\
223	2\\
224	2\\
225	2\\
226	2\\
227	2\\
228	2\\
229	2\\
230	2\\
231	2\\
232	2\\
233	2\\
234	2\\
235	2\\
236	2\\
237	2\\
238	2\\
239	2\\
240	2\\
241	2\\
242	2\\
243	2\\
244	2\\
245	2\\
246	2\\
247	2\\
248	2\\
249	2\\
250	2\\
251	2\\
252	2\\
253	2\\
254	2\\
255	2\\
256	2\\
257	2\\
258	2\\
259	2\\
260	2\\
261	2\\
262	2\\
263	2\\
264	2\\
265	2\\
266	2\\
267	2\\
268	2\\
269	2\\
270	2\\
271	2\\
272	2\\
273	2\\
274	2\\
275	2\\
276	2\\
277	2\\
278	2\\
279	2\\
280	2\\
281	2\\
282	2\\
283	2\\
284	2\\
285	2\\
286	2\\
287	2\\
288	2\\
289	2\\
290	2\\
291	2\\
292	2\\
293	2\\
294	2\\
295	2\\
296	2\\
297	2\\
298	2\\
299	2\\
300	2\\
301	2\\
302	2\\
303	2\\
304	2\\
305	2\\
306	2\\
307	2\\
308	2\\
309	2\\
310	2\\
311	2\\
312	2\\
313	2\\
314	2\\
315	2\\
316	2\\
317	2\\
318	2\\
319	2\\
320	2\\
321	2\\
322	2\\
323	2\\
324	2\\
325	2\\
326	2\\
327	2\\
328	2\\
329	2\\
330	2\\
331	2\\
332	2\\
333	2\\
334	2\\
335	2\\
336	2\\
337	2\\
338	2\\
339	2\\
340	2\\
341	2\\
342	2\\
343	2\\
344	2\\
345	2\\
346	2\\
347	2\\
348	2\\
349	2\\
350	2\\
351	2\\
352	2\\
353	2\\
354	2\\
355	2\\
356	2\\
357	2\\
358	2\\
359	2\\
360	2\\
361	2\\
362	2\\
363	2\\
364	2\\
365	2\\
366	2\\
367	2\\
368	2\\
369	2\\
370	2\\
371	2\\
372	2\\
373	2\\
374	2\\
375	2\\
376	2\\
377	2\\
378	2\\
379	2\\
380	2\\
381	2\\
382	2\\
383	2\\
384	2\\
385	2\\
386	2\\
387	2\\
388	2\\
389	2\\
390	2\\
391	2\\
392	2\\
393	2\\
394	2\\
395	2\\
396	2\\
397	2\\
398	2\\
399	2\\
400	2\\
401	2\\
402	2\\
403	2\\
404	2\\
405	2\\
406	2\\
407	2\\
408	2\\
409	2\\
410	2\\
411	2\\
412	2\\
413	2\\
414	2\\
415	2\\
416	2\\
417	2\\
418	2\\
419	2\\
420	2\\
421	2\\
422	2\\
423	2\\
424	2\\
425	2\\
426	2\\
427	2\\
428	2\\
429	2\\
430	2\\
431	2\\
432	2\\
433	2\\
434	2\\
435	2\\
436	2\\
437	2\\
438	2\\
439	2\\
440	2\\
441	2\\
442	2\\
443	2\\
444	2\\
445	2\\
446	2\\
447	2\\
448	2\\
449	2\\
450	2\\
451	2\\
452	2\\
453	2\\
454	2\\
455	2\\
456	2\\
457	2\\
458	2\\
459	2\\
460	2\\
461	2\\
462	2\\
463	2\\
464	2\\
465	2\\
466	2\\
467	2\\
468	2\\
469	2\\
470	2\\
471	2\\
472	2\\
473	2\\
474	2\\
475	2\\
476	2\\
477	2\\
478	2\\
479	2\\
480	2\\
481	2\\
482	2\\
483	2\\
484	2\\
485	2\\
486	2\\
487	2\\
488	2\\
489	2\\
490	2\\
491	2\\
492	2\\
493	2\\
494	2\\
495	2\\
496	2\\
};
\end{axis}
\end{tikzpicture}%
        \caption{Estimated x-Axis Positions}
	\end{subfigure}
	\begin{subfigure}{0.49\textwidth}
		 \centering
        \setlength{\figurewidth}{0.8\textwidth}
        % This file was created by matlab2tikz.
%
\definecolor{lms_red}{rgb}{0.80000,0.20780,0.21960}%
\definecolor{mycolor2}{rgb}{0.80000,0.20784,0.21961}%
\definecolor{mycolor3}{rgb}{0.92900,0.69400,0.12500}%
\definecolor{mycolor4}{rgb}{0.49400,0.18400,0.55600}%
%
\begin{tikzpicture}

\begin{axis}[%
width=0.951\figurewidth,
height=\figureheight,
at={(0\figurewidth,0\figureheight)},
scale only axis,
xmin=0,
xmax=496,
xtick={0,99.2,198.4,297.6,396.8,496},
xticklabels={{0},{1},{2},{3},{4},{5}},
xlabel style={font=\color{white!15!black}},
xlabel={$t$~[s]},
ymin=1,
ymax=5,
ylabel style={font=\color{white!15!black}},
ylabel={$p_y^{(t)}$~[m]},
axis background/.style={fill=white},
xmajorgrids,
ymajorgrids,
legend entries={Est.,
                $s=1$,
                $s=2$},
legend columns=-1,
legend style={%
    at={(1.0,1.0)},
    anchor=south east,
    font=\footnotesize,
    fill opacity=0.0, draw opacity=1, text opacity=1,
    draw=none,
    column sep=0.42cm,
    /tikz/every odd column/.append style={column sep=0.15cm}
},
]
% Estimates
\addlegendimage{color=lms_red, mark=x, only marks, mark options={mark size=4pt, opacity=1, line width=1}}
\addplot [color=mycolor2, draw=none, mark=x, mark options={solid, lms_red}, forget plot]
  table[row sep=crcr]{%
1	4.7\\
2	4.7\\
3	4.7\\
4	4.7\\
5	2.1\\
6	2.1\\
7	2.1\\
8	2.1\\
9	2.1\\
10	2.1\\
11	2.1\\
12	2.1\\
13	2.1\\
14	2.1\\
15	2.1\\
16	2.1\\
17	2.1\\
18	4\\
19	4\\
20	4\\
21	4\\
22	4\\
23	4\\
24	4\\
25	4\\
26	4\\
27	4\\
28	4\\
29	4\\
30	4\\
31	4\\
32	4\\
33	4\\
34	3.9\\
35	3.9\\
36	3.9\\
37	3.9\\
38	3.9\\
39	3.9\\
40	3.9\\
41	3.9\\
42	3.9\\
43	3.9\\
44	3.9\\
45	3.9\\
46	3.9\\
47	3.9\\
48	3.9\\
49	3.9\\
50	3.9\\
51	3.9\\
52	3.9\\
53	3.9\\
54	3.9\\
55	3.9\\
56	3.9\\
57	3.9\\
58	3.9\\
59	3.9\\
60	3.9\\
61	3.9\\
62	3.9\\
63	3.9\\
64	3.9\\
65	3.9\\
66	3.9\\
67	3.9\\
68	3.9\\
69	3.9\\
70	2.2\\
71	2.2\\
72	2.2\\
73	2.2\\
74	2.2\\
75	2.2\\
76	3.9\\
77	3.9\\
78	3.9\\
79	3.9\\
80	3.9\\
81	3.9\\
82	3.9\\
83	3.9\\
84	3.8\\
85	3.8\\
86	3.8\\
87	3.8\\
88	3.8\\
89	3.8\\
90	3.8\\
91	3.8\\
92	3.8\\
93	3.8\\
94	3.8\\
95	3.8\\
96	3.8\\
97	3.8\\
98	3.8\\
99	3.8\\
100	3.8\\
101	3.8\\
102	3.8\\
103	3.8\\
104	3.8\\
105	3.8\\
106	3.8\\
107	3.8\\
108	3.8\\
109	3.8\\
110	3.7\\
111	3.7\\
112	3.7\\
113	3.7\\
114	3.7\\
115	3.7\\
116	3.8\\
117	3.8\\
118	3.8\\
119	3.8\\
120	3.7\\
121	3.7\\
122	2.2\\
123	2.2\\
124	2.2\\
125	2.2\\
126	2.2\\
127	2.2\\
128	1.2\\
129	1.2\\
130	1.2\\
131	1.2\\
132	2.2\\
133	2.2\\
134	1.2\\
135	1.2\\
136	1.2\\
137	1.2\\
138	1.2\\
139	1.2\\
140	1.2\\
141	1.2\\
142	1.2\\
143	1.2\\
144	1.2\\
145	1.2\\
146	2.2\\
147	2.2\\
148	2.2\\
149	3.6\\
150	3.6\\
151	3.5\\
152	3.5\\
153	3.5\\
154	3.5\\
155	3.5\\
156	3.5\\
157	3.5\\
158	3.5\\
159	3.5\\
160	3.5\\
161	3.5\\
162	3.5\\
163	3.5\\
164	3.4\\
165	3.4\\
166	3.4\\
167	3.4\\
168	3.4\\
169	3.4\\
170	3.4\\
171	3.4\\
172	3.4\\
173	3.4\\
174	3.4\\
175	3.4\\
176	3.4\\
177	3.4\\
178	3.4\\
179	3.4\\
180	3.4\\
181	3.4\\
182	3.4\\
183	3.4\\
184	3.4\\
185	3.4\\
186	3.4\\
187	3.4\\
188	3.4\\
189	3.4\\
190	3.4\\
191	3.4\\
192	3.3\\
193	3.3\\
194	3.3\\
195	3.3\\
196	3.3\\
197	3.3\\
198	3.3\\
199	3.3\\
200	3.3\\
201	3.3\\
202	3.3\\
203	3.3\\
204	3.3\\
205	3.3\\
206	3.3\\
207	3.3\\
208	3.3\\
209	3.3\\
210	3.3\\
211	3.3\\
212	3.3\\
213	3.3\\
214	3.3\\
215	3.3\\
216	3.3\\
217	3.3\\
218	3.3\\
219	3.3\\
220	3.3\\
221	3.3\\
222	3.3\\
223	3.2\\
224	3.2\\
225	3.2\\
226	3.2\\
227	3.2\\
228	3.2\\
229	3.2\\
230	3.2\\
231	3.2\\
232	3.2\\
233	3.2\\
234	3.2\\
235	3.2\\
236	3.2\\
237	3.2\\
238	3.2\\
239	2.8\\
240	2.8\\
241	2.8\\
242	2.8\\
243	2.9\\
244	2.9\\
245	2.9\\
246	2.9\\
247	2.9\\
248	2.9\\
249	2.9\\
250	2.9\\
251	2.9\\
252	2.9\\
253	2.9\\
254	2.9\\
255	2.9\\
256	2.9\\
257	2.9\\
258	2.9\\
259	2.9\\
260	2.9\\
261	2.9\\
262	2.9\\
263	2.9\\
264	2.9\\
265	2.9\\
266	2.9\\
267	2.9\\
268	2.9\\
269	2.9\\
270	2.9\\
271	2.9\\
272	2.9\\
273	3.1\\
274	3.1\\
275	3\\
276	3\\
277	3\\
278	3\\
279	3\\
280	3\\
281	3\\
282	3\\
283	3\\
284	3\\
285	3\\
286	3\\
287	3\\
288	3\\
289	3\\
290	3\\
291	3\\
292	3\\
293	3\\
294	3\\
295	3\\
296	3\\
297	3\\
298	3\\
299	3\\
300	3\\
301	3\\
302	3\\
303	3\\
304	3\\
305	3\\
306	3\\
307	2.9\\
308	2.9\\
309	2.9\\
310	2.9\\
311	2.9\\
312	2.9\\
313	2.8\\
314	2.8\\
315	2.8\\
316	2.8\\
317	2.8\\
318	2.8\\
319	2.8\\
320	2.8\\
321	2.8\\
322	2.8\\
323	2.8\\
324	2.8\\
325	2.8\\
326	2.8\\
327	2.8\\
328	2.8\\
329	2.8\\
330	2.8\\
331	2.8\\
332	2.8\\
333	2.8\\
334	2.8\\
335	2.8\\
336	2.8\\
337	2.8\\
338	2.8\\
339	2.8\\
340	2.8\\
341	2.8\\
342	2.8\\
343	2.8\\
344	2.8\\
345	2.8\\
346	2.8\\
347	2.8\\
348	2.8\\
349	2.8\\
350	2.8\\
351	2.8\\
352	2.8\\
353	2.8\\
354	2.8\\
355	2.8\\
356	2.8\\
357	2.8\\
358	2.8\\
359	2.8\\
360	2.8\\
361	2.8\\
362	2.8\\
363	2.8\\
364	2.8\\
365	2.8\\
366	2.8\\
367	2.7\\
368	2.7\\
369	2.7\\
370	2.7\\
371	2.7\\
372	2.7\\
373	2.7\\
374	2.7\\
375	2.7\\
376	2.7\\
377	2.7\\
378	2.7\\
379	2.7\\
380	2.7\\
381	2.7\\
382	2.7\\
383	2.7\\
384	2.7\\
385	2.7\\
386	2.7\\
387	2.7\\
388	2.7\\
389	2.7\\
390	2.7\\
391	2.7\\
392	2.7\\
393	2.7\\
394	2.7\\
395	2.7\\
396	2.7\\
397	2.7\\
398	2.7\\
399	2.7\\
400	2.7\\
401	2.7\\
402	2.7\\
403	2.7\\
404	2.7\\
405	2.7\\
406	3.5\\
407	3.5\\
408	3.5\\
409	3.5\\
410	3.5\\
411	3.5\\
412	3.5\\
413	2.7\\
414	2.6\\
415	2.6\\
416	2.6\\
417	2.6\\
418	2.6\\
419	2.6\\
420	2.6\\
421	2.6\\
422	2.5\\
423	2.5\\
424	2.5\\
425	2.5\\
426	2.5\\
427	2.5\\
428	2.5\\
429	2.5\\
430	2.5\\
431	2.5\\
432	2.5\\
433	2.5\\
434	2.5\\
435	2.5\\
436	2.5\\
437	2.5\\
438	2.5\\
439	2.5\\
440	2.5\\
441	2.4\\
442	2.5\\
443	2.5\\
444	2.4\\
445	2.4\\
446	2.4\\
447	2.4\\
448	2.4\\
449	2.4\\
450	2.4\\
451	2.4\\
452	2.4\\
453	2.4\\
454	2.4\\
455	2.4\\
456	2.4\\
457	2.4\\
458	2.4\\
459	2.4\\
460	2.4\\
461	2.4\\
462	2.4\\
463	2.4\\
464	2.4\\
465	2.4\\
466	2.4\\
467	2.4\\
468	2.4\\
469	2.4\\
470	2.4\\
471	2.4\\
472	2.4\\
473	2.4\\
474	2.4\\
475	2.4\\
476	3.5\\
477	3.6\\
478	3.6\\
479	3.6\\
480	3.6\\
481	3.6\\
482	3.6\\
483	3.6\\
484	3.6\\
485	3.6\\
486	3.6\\
487	3.6\\
488	3.6\\
489	3.6\\
490	3.6\\
491	3.6\\
492	3.6\\
493	3.6\\
494	3.5\\
495	3.5\\
496	3.6\\
1	1.3\\
2	1.3\\
3	1.3\\
4	1.3\\
5	1.3\\
6	1.3\\
7	1.4\\
8	1.5\\
9	1.5\\
10	1.5\\
11	1.5\\
12	1.5\\
13	1.5\\
14	1.5\\
15	1.4\\
16	4\\
17	4\\
18	2.1\\
19	2.1\\
20	2.1\\
21	2.1\\
22	2.1\\
23	2.1\\
24	2.1\\
25	2.1\\
26	2.1\\
27	2.1\\
28	2.1\\
29	2.1\\
30	2.1\\
31	2.1\\
32	2.1\\
33	2.1\\
34	2.1\\
35	2.1\\
36	2.1\\
37	2.1\\
38	2.1\\
39	2.1\\
40	2.1\\
41	2.1\\
42	2.1\\
43	2.1\\
44	2.1\\
45	2.1\\
46	2.1\\
47	2.1\\
48	2.1\\
49	2.1\\
50	2.1\\
51	2.1\\
52	2.1\\
53	2.1\\
54	2.1\\
55	2.1\\
56	2.2\\
57	2.2\\
58	2.2\\
59	2.2\\
60	2.2\\
61	2.2\\
62	2.2\\
63	2.2\\
64	2.2\\
65	2.2\\
66	2.2\\
67	2.2\\
68	2.2\\
69	2.2\\
70	3.9\\
71	3.9\\
72	3.9\\
73	3.9\\
74	3.9\\
75	3.9\\
76	2.2\\
77	2.2\\
78	2.2\\
79	2.2\\
80	2.2\\
81	3.8\\
82	3.8\\
83	3.8\\
84	3.8\\
85	3.8\\
86	3.8\\
87	3.8\\
88	3.8\\
89	3.8\\
90	3.8\\
91	3.8\\
92	3.8\\
93	3.8\\
94	3.8\\
95	3.8\\
96	3.8\\
97	3.8\\
98	3.8\\
99	3.8\\
100	3.8\\
101	2.2\\
102	2.2\\
103	2.2\\
104	2.2\\
105	2.2\\
106	1.2\\
107	1.2\\
108	1.2\\
109	1.2\\
110	1.2\\
111	1.2\\
112	1.2\\
113	1.2\\
114	1.2\\
115	1.2\\
116	1.2\\
117	2.2\\
118	1.2\\
119	1.2\\
120	1.2\\
121	2.2\\
122	3.7\\
123	3.7\\
124	3.7\\
125	1.2\\
126	1.2\\
127	1.2\\
128	2.2\\
129	2.2\\
130	2.2\\
131	2.2\\
132	1.2\\
133	1.2\\
134	2.2\\
135	2.2\\
136	2.2\\
137	2.2\\
138	2.2\\
139	2.2\\
140	2.2\\
141	2.2\\
142	2.3\\
143	2.3\\
144	2.2\\
145	2.2\\
146	1.2\\
147	1.2\\
148	1.2\\
149	2.2\\
150	2.2\\
151	2.2\\
152	2.2\\
153	2.2\\
154	3\\
155	3\\
156	3\\
157	3\\
158	3\\
159	3\\
160	3\\
161	3\\
162	3\\
163	3\\
164	3\\
165	3\\
166	3\\
167	2.4\\
168	2.4\\
169	2.2\\
170	2.4\\
171	2.4\\
172	2.4\\
173	2.4\\
174	3\\
175	2.4\\
176	2.4\\
177	2.4\\
178	2.4\\
179	2.4\\
180	2.4\\
181	2.4\\
182	2.4\\
183	2.4\\
184	2.4\\
185	2.4\\
186	2.4\\
187	2.4\\
188	2.2\\
189	2.2\\
190	3\\
191	3\\
192	3.1\\
193	3\\
194	3\\
195	3\\
196	3\\
197	2.2\\
198	3\\
199	2.2\\
200	2.2\\
201	2.2\\
202	2.2\\
203	2.2\\
204	2.2\\
205	2.2\\
206	2.2\\
207	2.2\\
208	2.2\\
209	2.2\\
210	2.2\\
211	2.7\\
212	2.7\\
213	2.7\\
214	2.7\\
215	2.7\\
216	2.7\\
217	2.7\\
218	2.8\\
219	2.8\\
220	2.8\\
221	2.8\\
222	2.8\\
223	2.8\\
224	2.8\\
225	2.8\\
226	2.8\\
227	2.8\\
228	2.8\\
229	2.8\\
230	2.8\\
231	2.8\\
232	2.8\\
233	2.8\\
234	2.8\\
235	2.8\\
236	2.8\\
237	2.8\\
238	2.8\\
239	3.2\\
240	3.2\\
241	3.2\\
242	3.2\\
243	3.2\\
244	3.2\\
245	3.2\\
246	3.2\\
247	3.2\\
248	3.2\\
249	3.2\\
250	3.2\\
251	3.2\\
252	3.2\\
253	3.2\\
254	3.2\\
255	3.2\\
256	3.2\\
257	3.2\\
258	3.2\\
259	3.2\\
260	3.2\\
261	3.2\\
262	3.2\\
263	3.2\\
264	3.2\\
265	3.2\\
266	3.2\\
267	3.2\\
268	3.2\\
269	3.2\\
270	3.1\\
271	3.1\\
272	3.1\\
273	2.9\\
274	2.9\\
275	2.9\\
276	2.9\\
277	2.9\\
278	2.9\\
279	2.9\\
280	2.9\\
281	2.9\\
282	2.9\\
283	2.9\\
284	2.9\\
285	2.9\\
286	2.9\\
287	2.9\\
288	2.9\\
289	2.9\\
290	2.9\\
291	2.9\\
292	2.9\\
293	2.9\\
294	2.9\\
295	2.9\\
296	2.9\\
297	2.9\\
298	2.9\\
299	2.9\\
300	2.9\\
301	3\\
302	3\\
303	3\\
304	3\\
305	3\\
306	3\\
307	3\\
308	3\\
309	3\\
310	3\\
311	3\\
312	3\\
313	3\\
314	3\\
315	3\\
316	3\\
317	3\\
318	3\\
319	3\\
320	2.9\\
321	2.9\\
322	2.9\\
323	2.9\\
324	2.9\\
325	2.9\\
326	2.9\\
327	3\\
328	2.9\\
329	2.9\\
330	2.9\\
331	2.9\\
332	2.9\\
333	2.9\\
334	2.9\\
335	3\\
336	3\\
337	1.2\\
338	1.2\\
339	2.9\\
340	2.9\\
341	2.9\\
342	1.2\\
343	1.2\\
344	1.2\\
345	1.2\\
346	1.2\\
347	4.7\\
348	4.7\\
349	1.2\\
350	3\\
351	3\\
352	3\\
353	3\\
354	3.4\\
355	3.5\\
356	3.5\\
357	3.5\\
358	3.5\\
359	3.5\\
360	3.5\\
361	3.5\\
362	3.5\\
363	3.5\\
364	3.5\\
365	3.5\\
366	3.4\\
367	3.4\\
368	3.5\\
369	3.5\\
370	3.5\\
371	3.5\\
372	3.5\\
373	3.5\\
374	3.5\\
375	3.5\\
376	3.5\\
377	3.5\\
378	3.5\\
379	3.5\\
380	3.5\\
381	3.5\\
382	3.5\\
383	3.5\\
384	3.5\\
385	3.5\\
386	3.5\\
387	3.5\\
388	3.5\\
389	3.5\\
390	3.5\\
391	3.5\\
392	3.5\\
393	3.5\\
394	3.5\\
395	3.4\\
396	3.4\\
397	3.5\\
398	3.4\\
399	3.5\\
400	3.5\\
401	3.4\\
402	3.5\\
403	3.5\\
404	3.5\\
405	3.5\\
406	2.7\\
407	2.7\\
408	2.7\\
409	2.7\\
410	2.7\\
411	2.7\\
412	2.7\\
413	3.5\\
414	3.5\\
415	3.5\\
416	3.5\\
417	3.5\\
418	3.5\\
419	3.5\\
420	2.1\\
421	2.1\\
422	3.5\\
423	3.5\\
424	3.5\\
425	3.5\\
426	3.5\\
427	3.5\\
428	3.5\\
429	3.5\\
430	3.5\\
431	3.5\\
432	3.5\\
433	3.5\\
434	3.5\\
435	3.5\\
436	3.5\\
437	3.5\\
438	3.5\\
439	3.5\\
440	3.5\\
441	3.5\\
442	1.2\\
443	1.2\\
444	1.2\\
445	1.2\\
446	1.2\\
447	1.2\\
448	1.2\\
449	1.2\\
450	1.2\\
451	1.2\\
452	1.2\\
453	2.3\\
454	2.3\\
455	2.3\\
456	2.3\\
457	2.3\\
458	3.5\\
459	3.5\\
460	3.5\\
461	3.5\\
462	3.5\\
463	3.5\\
464	3.5\\
465	3.5\\
466	3.5\\
467	3.5\\
468	3.5\\
469	3.5\\
470	3.5\\
471	3.5\\
472	3.5\\
473	3.5\\
474	3.5\\
475	3.5\\
476	2.4\\
477	2.4\\
478	2.4\\
479	2.4\\
480	2.4\\
481	2.4\\
482	2.4\\
483	2.4\\
484	2.4\\
485	2.4\\
486	2.4\\
487	2.4\\
488	2.4\\
489	2.4\\
490	2.4\\
491	2.4\\
492	2.4\\
493	2.4\\
494	2.4\\
495	2.4\\
496	2.4\\
};

\addplot [color=mycolor3, dashed, line width=\trajDashedLinewidth]
  table[row sep=crcr]{%
1	2\\
496	4\\
};
\addplot [color=mycolor4, dashed, line width=\trajDashedLinewidth]
  table[row sep=crcr]{%
1	4\\
496	2\\
};
\end{axis}
\end{tikzpicture}%
        \caption{Estimated y-Axis Positions}	\end{subfigure}
}
	\caption[Parallel Movement Results for CREM]{Parallel Movement Results for CREM (\Tsixty$=0.4$~s).}
	\label{fig:trackingParallelCREM}
\end{figure}

The \gls{crem} algorithm is able to identify the x- and y-coordinates for both sources successfully for most of the trial. \autoref{fig:trackingParallelCREM} clearly shows, how the missing activity of $s=1$ around $t=1$ and $t=3$ immediately leads to spurious x-coordinate estimates close to the microphone pairs. Otherwise the location estimates closely track the true source trajectories.

\begin{figure}[H]
\iftoggle{quick}{%
    \includegraphics[width=\textwidth]{plots/tracking/parallel/results-T60=0.4-trem-xy.png}
}{%
	\begin{subfigure}{0.49\textwidth}
	     \centering
        \setlength{\figurewidth}{0.8\textwidth}
        % This file was created by matlab2tikz.
%
\definecolor{lms_red}{rgb}{0.80000,0.20780,0.21960}%
\definecolor{mycolor2}{rgb}{0.80000,0.20784,0.21961}%
\definecolor{mycolor3}{rgb}{0.92900,0.69400,0.12500}%
\definecolor{mycolor4}{rgb}{0.49400,0.18400,0.55600}%
%
\begin{tikzpicture}

\begin{axis}[%
width=0.951\figurewidth,
height=\figureheight,
at={(0\figurewidth,0\figureheight)},
scale only axis,
xmin=0,
xmax=496,
xtick={0,99.2,198.4,297.6,396.8,496},
xticklabels={{0},{1},{2},{3},{4},{5}},
xlabel style={font=\color{white!15!black}},
xlabel={t},
ymin=1,
ymax=5,
ylabel style={font=\color{white!15!black}},
ylabel={$p_x^{(t)}$~[m]},
axis background/.style={fill=white},
axis x line*=bottom,
axis y line*=left,
xmajorgrids,
ymajorgrids
%legend style={legend cell align=left, align=left, draw=white!15!black}
]
\addplot [color=mycolor2, draw=none, mark=x, mark options={solid, mycolor2}, forget plot]
  table[row sep=crcr]{%
1	3.9\\
2	4.7\\
3	4.7\\
4	4\\
5	4\\
6	4\\
7	4\\
8	4\\
9	4\\
10	4\\
11	4\\
12	4\\
13	4\\
14	4\\
15	4\\
16	4\\
17	4\\
18	4\\
19	2\\
20	2\\
21	2\\
22	2\\
23	2\\
24	2\\
25	2\\
26	2\\
27	2\\
28	2\\
29	2\\
30	2\\
31	2\\
32	2\\
33	2\\
34	2\\
35	1.9\\
36	2\\
37	2\\
38	2\\
39	2\\
40	2\\
41	2\\
42	1.9\\
43	2\\
44	2\\
45	2.1\\
46	1.2\\
47	1.2\\
48	1.2\\
49	2.1\\
50	2.2\\
51	2.2\\
52	2.2\\
53	2.2\\
54	2.1\\
55	2.2\\
56	2.2\\
57	4.1\\
58	4.1\\
59	4.1\\
60	4.1\\
61	4.1\\
62	2.1\\
63	2.1\\
64	2.1\\
65	2.1\\
66	4.1\\
67	4.1\\
68	4.1\\
69	4.1\\
70	4.1\\
71	4.1\\
72	4.1\\
73	4.7\\
74	4.7\\
75	4.7\\
76	4.7\\
77	1.2\\
78	1.2\\
79	1.2\\
80	1.2\\
81	1.2\\
82	1.8\\
83	1.9\\
84	1.9\\
85	1.9\\
86	1.9\\
87	1.9\\
88	1.9\\
89	1.9\\
90	1.9\\
91	1.9\\
92	1.9\\
93	1.9\\
94	1.9\\
95	1.9\\
96	1.9\\
97	1.9\\
98	1.9\\
99	1.9\\
100	1.9\\
101	1.9\\
102	1.9\\
103	1.9\\
104	1.9\\
105	1.9\\
106	1.9\\
107	2\\
108	2\\
109	1.9\\
110	2\\
111	2\\
112	2\\
113	2\\
114	2\\
115	2\\
116	1.9\\
117	1.9\\
118	1.9\\
119	2\\
120	2\\
121	1.9\\
122	1.9\\
123	1.9\\
124	1.9\\
125	3.8\\
126	3.8\\
127	3.8\\
128	3.8\\
129	3.8\\
130	3.8\\
131	3.8\\
132	3.8\\
133	3.8\\
134	3.8\\
135	3.8\\
136	3.8\\
137	3.8\\
138	3.8\\
139	3.8\\
140	3.8\\
141	3.8\\
142	3.8\\
143	3.8\\
144	3.8\\
145	3.8\\
146	3.8\\
147	3.8\\
148	3.8\\
149	1.9\\
150	1.9\\
151	2\\
152	2\\
153	2\\
154	2\\
155	2\\
156	2\\
157	2\\
158	2\\
159	2\\
160	2\\
161	2\\
162	2\\
163	2\\
164	2\\
165	2\\
166	2\\
167	2\\
168	2\\
169	2\\
170	2\\
171	2\\
172	2\\
173	2\\
174	2\\
175	2\\
176	2\\
177	2\\
178	2\\
179	2\\
180	2\\
181	2\\
182	2\\
183	2\\
184	2\\
185	2\\
186	2\\
187	2\\
188	2\\
189	2\\
190	2\\
191	2\\
192	2\\
193	2\\
194	2\\
195	2\\
196	2\\
197	2\\
198	2\\
199	2\\
200	2\\
201	2\\
202	2\\
203	2\\
204	2\\
205	2\\
206	2\\
207	2\\
208	2\\
209	2\\
210	2\\
211	2\\
212	2\\
213	2\\
214	2\\
215	2\\
216	2\\
217	2\\
218	2\\
219	2\\
220	2\\
221	2\\
222	2\\
223	2\\
224	2\\
225	2\\
226	2\\
227	2\\
228	2\\
229	2\\
230	2.1\\
231	2.1\\
232	2.1\\
233	2.1\\
234	2.1\\
235	2.1\\
236	2.1\\
237	2.1\\
238	3.9\\
239	3.9\\
240	3.9\\
241	3.9\\
242	3.9\\
243	3.9\\
244	3.9\\
245	3.9\\
246	3.9\\
247	3.9\\
248	3.9\\
249	3.9\\
250	3.9\\
251	3.9\\
252	3.9\\
253	3.9\\
254	3.9\\
255	3.9\\
256	3.9\\
257	3.9\\
258	3.9\\
259	3.9\\
260	3.9\\
261	3.9\\
262	3.9\\
263	3.9\\
264	3.9\\
265	3.9\\
266	3.9\\
267	3.9\\
268	3.9\\
269	3.9\\
270	3.9\\
271	3.9\\
272	3.9\\
273	2.1\\
274	2.1\\
275	2.1\\
276	2.1\\
277	2.1\\
278	2.1\\
279	2.1\\
280	2.1\\
281	2.1\\
282	2.1\\
283	2.1\\
284	2.1\\
285	2.1\\
286	2.1\\
287	2.1\\
288	2.1\\
289	2.1\\
290	2.1\\
291	2.1\\
292	2.1\\
293	2.1\\
294	2.1\\
295	2.1\\
296	2.1\\
297	2.1\\
298	2.1\\
299	2.1\\
300	2.1\\
301	2.1\\
302	2.1\\
303	2.1\\
304	2.1\\
305	1.2\\
306	2\\
307	2\\
308	2\\
309	2\\
310	2\\
311	2\\
312	2\\
313	2\\
314	2\\
315	2\\
316	2\\
317	2\\
318	2\\
319	2\\
320	2\\
321	2\\
322	2\\
323	2\\
324	2\\
325	2\\
326	2\\
327	2\\
328	2\\
329	2\\
330	2\\
331	2\\
332	2\\
333	2\\
334	2\\
335	2\\
336	2\\
337	2\\
338	2\\
339	2\\
340	2\\
341	2\\
342	2\\
343	2\\
344	2\\
345	2\\
346	2\\
347	2\\
348	2\\
349	2\\
350	2\\
351	2\\
352	2\\
353	2\\
354	2\\
355	2\\
356	2\\
357	2\\
358	2\\
359	2\\
360	2\\
361	2\\
362	2\\
363	2\\
364	2\\
365	2\\
366	2\\
367	2\\
368	2\\
369	2\\
370	2\\
371	2\\
372	2\\
373	2\\
374	2\\
375	2\\
376	2\\
377	2\\
378	2\\
379	2\\
380	2\\
381	2\\
382	2\\
383	2\\
384	2\\
385	2\\
386	2\\
387	2\\
388	2\\
389	2\\
390	2\\
391	2\\
392	2\\
393	2\\
394	2\\
395	2\\
396	2\\
397	2\\
398	2\\
399	2\\
400	2\\
401	2\\
402	2\\
403	2\\
404	2\\
405	2\\
406	3.9\\
407	3.9\\
408	3.9\\
409	3.9\\
410	3.9\\
411	3.9\\
412	3.9\\
413	2\\
414	2\\
415	2\\
416	2\\
417	2\\
418	2\\
419	2\\
420	2\\
421	2\\
422	2\\
423	2\\
424	2\\
425	2\\
426	2\\
427	2\\
428	2\\
429	2\\
430	2\\
431	2\\
432	2\\
433	2\\
434	2\\
435	2\\
436	2\\
437	2\\
438	2\\
439	2\\
440	2\\
441	2\\
442	2\\
443	2\\
444	2\\
445	2\\
446	2\\
447	2\\
448	2\\
449	2\\
450	2\\
451	2\\
452	2\\
453	2\\
454	2\\
455	2\\
456	2\\
457	2\\
458	2\\
459	2\\
460	2\\
461	2\\
462	2\\
463	2\\
464	2\\
465	2\\
466	2\\
467	2\\
468	2\\
469	2\\
470	2\\
471	2\\
472	3.9\\
473	2\\
474	2\\
475	2\\
476	3.9\\
477	3.9\\
478	3.9\\
479	3.9\\
480	3.9\\
481	3.9\\
482	3.9\\
483	3.9\\
484	3.9\\
485	3.9\\
486	3.9\\
487	3.9\\
488	4\\
489	4\\
490	4\\
491	4\\
492	4\\
493	4\\
494	3.9\\
495	3.9\\
496	3.9\\
};
\addplot [color=mycolor2, draw=none, mark=x, mark options={solid, mycolor2}, forget plot]
  table[row sep=crcr]{%
1	3.9\\
2	3.9\\
3	3.8\\
4	4.6\\
5	4.6\\
6	4.6\\
7	4.6\\
8	4.6\\
9	4.5\\
10	4.6\\
11	4.6\\
12	3.9\\
13	3.9\\
14	3.8\\
15	3.8\\
16	2\\
17	2\\
18	2\\
19	4\\
20	4\\
21	4\\
22	4\\
23	4\\
24	4\\
25	4\\
26	4\\
27	4\\
28	4\\
29	4\\
30	4\\
31	4\\
32	4\\
33	4\\
34	1.4\\
35	1.3\\
36	1.4\\
37	1.4\\
38	1.4\\
39	1.4\\
40	1.4\\
41	1.2\\
42	1.2\\
43	1.2\\
44	1.2\\
45	1.2\\
46	2.1\\
47	2.1\\
48	2.2\\
49	1.2\\
50	1.2\\
51	1.2\\
52	1.2\\
53	1.2\\
54	1.2\\
55	1.2\\
56	1.2\\
57	2.2\\
58	2.1\\
59	2.1\\
60	2.1\\
61	2.1\\
62	4.1\\
63	4.1\\
64	4.1\\
65	4.1\\
66	2.1\\
67	2.1\\
68	4.7\\
69	4.7\\
70	4.7\\
71	4.7\\
72	4.7\\
73	4.2\\
74	1.2\\
75	1.2\\
76	1.2\\
77	4.7\\
78	4.7\\
79	4.7\\
80	1.9\\
81	1.8\\
82	1.2\\
83	1.2\\
84	1.2\\
85	1.2\\
86	1.2\\
87	1.2\\
88	1.2\\
89	1.2\\
90	1.2\\
91	1.2\\
92	1.2\\
93	1.2\\
94	1.2\\
95	1.2\\
96	1.2\\
97	1.2\\
98	1.2\\
99	1.2\\
100	1.2\\
101	1.2\\
102	1.2\\
103	1.2\\
104	1.2\\
105	1.2\\
106	1.2\\
107	1.2\\
108	1.2\\
109	2.2\\
110	2.2\\
111	2.1\\
112	2.1\\
113	2.3\\
114	2.3\\
115	2.3\\
116	2.1\\
117	2.3\\
118	2.3\\
119	2.3\\
120	2.3\\
121	1.2\\
122	1.2\\
123	1.2\\
124	1.2\\
125	1.9\\
126	1.2\\
127	1.2\\
128	1.2\\
129	1.2\\
130	1.2\\
131	1.2\\
132	1.2\\
133	1.2\\
134	1.2\\
135	1.2\\
136	1.2\\
137	1.2\\
138	1.2\\
139	1.2\\
140	1.2\\
141	1.2\\
142	1.2\\
143	1.2\\
144	1.2\\
145	1.2\\
146	1.2\\
147	1.2\\
148	1.2\\
149	3.8\\
150	3.8\\
151	3.8\\
152	1.2\\
153	1.2\\
154	1.2\\
155	1.2\\
156	1.2\\
157	1.2\\
158	1.2\\
159	1.2\\
160	1.2\\
161	1.2\\
162	1.2\\
163	1.2\\
164	1.2\\
165	1.2\\
166	1.2\\
167	4.1\\
168	4.1\\
169	1.2\\
170	1.2\\
171	1.2\\
172	1.2\\
173	1.2\\
174	1.2\\
175	2.8\\
176	2.8\\
177	2.8\\
178	2.8\\
179	2.8\\
180	2.8\\
181	2.8\\
182	2.8\\
183	2.8\\
184	2.8\\
185	2.8\\
186	2.8\\
187	2.8\\
188	2.8\\
189	2.8\\
190	2.8\\
191	2.8\\
192	2.8\\
193	2.8\\
194	2.8\\
195	2.8\\
196	2.8\\
197	2.8\\
198	1.2\\
199	1.2\\
200	1.2\\
201	1.2\\
202	2.8\\
203	2.8\\
204	2.8\\
205	2.8\\
206	2.8\\
207	1.2\\
208	1.2\\
209	1.2\\
210	4.1\\
211	4.1\\
212	4.1\\
213	4\\
214	4\\
215	4\\
216	4\\
217	4\\
218	4\\
219	4\\
220	4\\
221	4\\
222	4\\
223	4\\
224	3.9\\
225	4\\
226	4\\
227	3.9\\
228	3.9\\
229	3.9\\
230	3.9\\
231	3.9\\
232	3.9\\
233	3.9\\
234	3.9\\
235	3.9\\
236	3.9\\
237	3.9\\
238	2.1\\
239	2.1\\
240	2.1\\
241	2.1\\
242	2.1\\
243	2.1\\
244	2.1\\
245	2.1\\
246	2.1\\
247	2.1\\
248	2.1\\
249	2.1\\
250	2.1\\
251	2.1\\
252	2.1\\
253	2.1\\
254	2.1\\
255	2.1\\
256	2.1\\
257	2.1\\
258	2.1\\
259	2.1\\
260	2.1\\
261	2.1\\
262	2.1\\
263	2.1\\
264	2.1\\
265	2.1\\
266	2.1\\
267	2.1\\
268	2.1\\
269	2.1\\
270	2.1\\
271	2.1\\
272	2.1\\
273	3.9\\
274	3.9\\
275	3.9\\
276	3.9\\
277	3.9\\
278	3.9\\
279	3.9\\
280	3.9\\
281	3.9\\
282	3.9\\
283	3.9\\
284	3.9\\
285	3.9\\
286	3.9\\
287	3.9\\
288	3.9\\
289	3.9\\
290	3.9\\
291	3.9\\
292	3.9\\
293	3.9\\
294	3.9\\
295	3.9\\
296	4\\
297	4\\
298	4\\
299	4\\
300	4\\
301	1.2\\
302	1.2\\
303	1.2\\
304	1.2\\
305	2\\
306	1.2\\
307	1.2\\
308	1.2\\
309	1.2\\
310	1.2\\
311	1.2\\
312	1.2\\
313	1.2\\
314	1.2\\
315	1.2\\
316	1.2\\
317	1.2\\
318	1.2\\
319	1.2\\
320	3.8\\
321	3.7\\
322	3.7\\
323	3.7\\
324	3.7\\
325	3.7\\
326	3.7\\
327	3.8\\
328	3.8\\
329	3.8\\
330	3.8\\
331	3.8\\
332	3.8\\
333	3.7\\
334	3.8\\
335	3.8\\
336	3.8\\
337	3.8\\
338	3.8\\
339	3.8\\
340	3.8\\
341	3.8\\
342	3.8\\
343	3.8\\
344	3.8\\
345	3.8\\
346	3.8\\
347	3.8\\
348	3.8\\
349	3.8\\
350	3.8\\
351	3.8\\
352	3.8\\
353	3.9\\
354	3.9\\
355	3.9\\
356	4\\
357	3.9\\
358	3.9\\
359	3.9\\
360	3.9\\
361	3.9\\
362	3.9\\
363	3.8\\
364	3.8\\
365	3.8\\
366	3.8\\
367	3.8\\
368	3.8\\
369	3.8\\
370	3.9\\
371	3.9\\
372	3.9\\
373	4\\
374	3.9\\
375	3.9\\
376	3.9\\
377	3.9\\
378	3.9\\
379	3.9\\
380	3.9\\
381	3.9\\
382	3.9\\
383	3.9\\
384	3.9\\
385	3.9\\
386	3.9\\
387	3.9\\
388	3.9\\
389	3.9\\
390	3.9\\
391	3.9\\
392	3.9\\
393	3.9\\
394	3.9\\
395	3.9\\
396	3.9\\
397	3.9\\
398	3.9\\
399	3.9\\
400	3.9\\
401	3.9\\
402	3.9\\
403	3.9\\
404	3.9\\
405	3.9\\
406	2\\
407	2\\
408	2\\
409	2\\
410	2\\
411	2\\
412	2\\
413	3.9\\
414	3.9\\
415	3.9\\
416	3.9\\
417	3.9\\
418	3.9\\
419	3.9\\
420	2.2\\
421	2.2\\
422	2.2\\
423	3.9\\
424	3.9\\
425	3.9\\
426	3.9\\
427	3.9\\
428	2.2\\
429	2.2\\
430	2.2\\
431	3.9\\
432	3.9\\
433	3.9\\
434	3.9\\
435	2.2\\
436	2.2\\
437	2.2\\
438	2.2\\
439	3.9\\
440	4\\
441	4\\
442	3.8\\
443	3.8\\
444	3.8\\
445	3.8\\
446	2.2\\
447	3.8\\
448	3.8\\
449	3.8\\
450	3.8\\
451	3.8\\
452	3.8\\
453	4.1\\
454	4.1\\
455	4.1\\
456	4.1\\
457	3.9\\
458	3.9\\
459	3.9\\
460	3.9\\
461	3.9\\
462	3.9\\
463	3.9\\
464	3.9\\
465	3.9\\
466	3.9\\
467	3.9\\
468	3.9\\
469	3.9\\
470	4\\
471	3.9\\
472	2\\
473	3.9\\
474	3.9\\
475	3.9\\
476	2\\
477	2\\
478	2\\
479	2\\
480	2\\
481	2\\
482	2\\
483	2\\
484	2\\
485	2\\
486	2\\
487	2\\
488	2\\
489	2\\
490	2\\
491	2\\
492	2\\
493	2\\
494	2\\
495	2\\
496	2\\
};
\addplot [color=mycolor3, dashed, forget plot]
  table[row sep=crcr]{%
1	4\\
2	4\\
3	4\\
4	4\\
5	4\\
6	4\\
7	4\\
8	4\\
9	4\\
10	4\\
11	4\\
12	4\\
13	4\\
14	4\\
15	4\\
16	4\\
17	4\\
18	4\\
19	4\\
20	4\\
21	4\\
22	4\\
23	4\\
24	4\\
25	4\\
26	4\\
27	4\\
28	4\\
29	4\\
30	4\\
31	4\\
32	4\\
33	4\\
34	4\\
35	4\\
36	4\\
37	4\\
38	4\\
39	4\\
40	4\\
41	4\\
42	4\\
43	4\\
44	4\\
45	4\\
46	4\\
47	4\\
48	4\\
49	4\\
50	4\\
51	4\\
52	4\\
53	4\\
54	4\\
55	4\\
56	4\\
57	4\\
58	4\\
59	4\\
60	4\\
61	4\\
62	4\\
63	4\\
64	4\\
65	4\\
66	4\\
67	4\\
68	4\\
69	4\\
70	4\\
71	4\\
72	4\\
73	4\\
74	4\\
75	4\\
76	4\\
77	4\\
78	4\\
79	4\\
80	4\\
81	4\\
82	4\\
83	4\\
84	4\\
85	4\\
86	4\\
87	4\\
88	4\\
89	4\\
90	4\\
91	4\\
92	4\\
93	4\\
94	4\\
95	4\\
96	4\\
97	4\\
98	4\\
99	4\\
100	4\\
101	4\\
102	4\\
103	4\\
104	4\\
105	4\\
106	4\\
107	4\\
108	4\\
109	4\\
110	4\\
111	4\\
112	4\\
113	4\\
114	4\\
115	4\\
116	4\\
117	4\\
118	4\\
119	4\\
120	4\\
121	4\\
122	4\\
123	4\\
124	4\\
125	4\\
126	4\\
127	4\\
128	4\\
129	4\\
130	4\\
131	4\\
132	4\\
133	4\\
134	4\\
135	4\\
136	4\\
137	4\\
138	4\\
139	4\\
140	4\\
141	4\\
142	4\\
143	4\\
144	4\\
145	4\\
146	4\\
147	4\\
148	4\\
149	4\\
150	4\\
151	4\\
152	4\\
153	4\\
154	4\\
155	4\\
156	4\\
157	4\\
158	4\\
159	4\\
160	4\\
161	4\\
162	4\\
163	4\\
164	4\\
165	4\\
166	4\\
167	4\\
168	4\\
169	4\\
170	4\\
171	4\\
172	4\\
173	4\\
174	4\\
175	4\\
176	4\\
177	4\\
178	4\\
179	4\\
180	4\\
181	4\\
182	4\\
183	4\\
184	4\\
185	4\\
186	4\\
187	4\\
188	4\\
189	4\\
190	4\\
191	4\\
192	4\\
193	4\\
194	4\\
195	4\\
196	4\\
197	4\\
198	4\\
199	4\\
200	4\\
201	4\\
202	4\\
203	4\\
204	4\\
205	4\\
206	4\\
207	4\\
208	4\\
209	4\\
210	4\\
211	4\\
212	4\\
213	4\\
214	4\\
215	4\\
216	4\\
217	4\\
218	4\\
219	4\\
220	4\\
221	4\\
222	4\\
223	4\\
224	4\\
225	4\\
226	4\\
227	4\\
228	4\\
229	4\\
230	4\\
231	4\\
232	4\\
233	4\\
234	4\\
235	4\\
236	4\\
237	4\\
238	4\\
239	4\\
240	4\\
241	4\\
242	4\\
243	4\\
244	4\\
245	4\\
246	4\\
247	4\\
248	4\\
249	4\\
250	4\\
251	4\\
252	4\\
253	4\\
254	4\\
255	4\\
256	4\\
257	4\\
258	4\\
259	4\\
260	4\\
261	4\\
262	4\\
263	4\\
264	4\\
265	4\\
266	4\\
267	4\\
268	4\\
269	4\\
270	4\\
271	4\\
272	4\\
273	4\\
274	4\\
275	4\\
276	4\\
277	4\\
278	4\\
279	4\\
280	4\\
281	4\\
282	4\\
283	4\\
284	4\\
285	4\\
286	4\\
287	4\\
288	4\\
289	4\\
290	4\\
291	4\\
292	4\\
293	4\\
294	4\\
295	4\\
296	4\\
297	4\\
298	4\\
299	4\\
300	4\\
301	4\\
302	4\\
303	4\\
304	4\\
305	4\\
306	4\\
307	4\\
308	4\\
309	4\\
310	4\\
311	4\\
312	4\\
313	4\\
314	4\\
315	4\\
316	4\\
317	4\\
318	4\\
319	4\\
320	4\\
321	4\\
322	4\\
323	4\\
324	4\\
325	4\\
326	4\\
327	4\\
328	4\\
329	4\\
330	4\\
331	4\\
332	4\\
333	4\\
334	4\\
335	4\\
336	4\\
337	4\\
338	4\\
339	4\\
340	4\\
341	4\\
342	4\\
343	4\\
344	4\\
345	4\\
346	4\\
347	4\\
348	4\\
349	4\\
350	4\\
351	4\\
352	4\\
353	4\\
354	4\\
355	4\\
356	4\\
357	4\\
358	4\\
359	4\\
360	4\\
361	4\\
362	4\\
363	4\\
364	4\\
365	4\\
366	4\\
367	4\\
368	4\\
369	4\\
370	4\\
371	4\\
372	4\\
373	4\\
374	4\\
375	4\\
376	4\\
377	4\\
378	4\\
379	4\\
380	4\\
381	4\\
382	4\\
383	4\\
384	4\\
385	4\\
386	4\\
387	4\\
388	4\\
389	4\\
390	4\\
391	4\\
392	4\\
393	4\\
394	4\\
395	4\\
396	4\\
397	4\\
398	4\\
399	4\\
400	4\\
401	4\\
402	4\\
403	4\\
404	4\\
405	4\\
406	4\\
407	4\\
408	4\\
409	4\\
410	4\\
411	4\\
412	4\\
413	4\\
414	4\\
415	4\\
416	4\\
417	4\\
418	4\\
419	4\\
420	4\\
421	4\\
422	4\\
423	4\\
424	4\\
425	4\\
426	4\\
427	4\\
428	4\\
429	4\\
430	4\\
431	4\\
432	4\\
433	4\\
434	4\\
435	4\\
436	4\\
437	4\\
438	4\\
439	4\\
440	4\\
441	4\\
442	4\\
443	4\\
444	4\\
445	4\\
446	4\\
447	4\\
448	4\\
449	4\\
450	4\\
451	4\\
452	4\\
453	4\\
454	4\\
455	4\\
456	4\\
457	4\\
458	4\\
459	4\\
460	4\\
461	4\\
462	4\\
463	4\\
464	4\\
465	4\\
466	4\\
467	4\\
468	4\\
469	4\\
470	4\\
471	4\\
472	4\\
473	4\\
474	4\\
475	4\\
476	4\\
477	4\\
478	4\\
479	4\\
480	4\\
481	4\\
482	4\\
483	4\\
484	4\\
485	4\\
486	4\\
487	4\\
488	4\\
489	4\\
490	4\\
491	4\\
492	4\\
493	4\\
494	4\\
495	4\\
496	4\\
};
%\addlegendentry{Traj. $s=1$}
\addplot [color=mycolor4, dashed, forget plot]
  table[row sep=crcr]{%
1	2\\
2	2\\
3	2\\
4	2\\
5	2\\
6	2\\
7	2\\
8	2\\
9	2\\
10	2\\
11	2\\
12	2\\
13	2\\
14	2\\
15	2\\
16	2\\
17	2\\
18	2\\
19	2\\
20	2\\
21	2\\
22	2\\
23	2\\
24	2\\
25	2\\
26	2\\
27	2\\
28	2\\
29	2\\
30	2\\
31	2\\
32	2\\
33	2\\
34	2\\
35	2\\
36	2\\
37	2\\
38	2\\
39	2\\
40	2\\
41	2\\
42	2\\
43	2\\
44	2\\
45	2\\
46	2\\
47	2\\
48	2\\
49	2\\
50	2\\
51	2\\
52	2\\
53	2\\
54	2\\
55	2\\
56	2\\
57	2\\
58	2\\
59	2\\
60	2\\
61	2\\
62	2\\
63	2\\
64	2\\
65	2\\
66	2\\
67	2\\
68	2\\
69	2\\
70	2\\
71	2\\
72	2\\
73	2\\
74	2\\
75	2\\
76	2\\
77	2\\
78	2\\
79	2\\
80	2\\
81	2\\
82	2\\
83	2\\
84	2\\
85	2\\
86	2\\
87	2\\
88	2\\
89	2\\
90	2\\
91	2\\
92	2\\
93	2\\
94	2\\
95	2\\
96	2\\
97	2\\
98	2\\
99	2\\
100	2\\
101	2\\
102	2\\
103	2\\
104	2\\
105	2\\
106	2\\
107	2\\
108	2\\
109	2\\
110	2\\
111	2\\
112	2\\
113	2\\
114	2\\
115	2\\
116	2\\
117	2\\
118	2\\
119	2\\
120	2\\
121	2\\
122	2\\
123	2\\
124	2\\
125	2\\
126	2\\
127	2\\
128	2\\
129	2\\
130	2\\
131	2\\
132	2\\
133	2\\
134	2\\
135	2\\
136	2\\
137	2\\
138	2\\
139	2\\
140	2\\
141	2\\
142	2\\
143	2\\
144	2\\
145	2\\
146	2\\
147	2\\
148	2\\
149	2\\
150	2\\
151	2\\
152	2\\
153	2\\
154	2\\
155	2\\
156	2\\
157	2\\
158	2\\
159	2\\
160	2\\
161	2\\
162	2\\
163	2\\
164	2\\
165	2\\
166	2\\
167	2\\
168	2\\
169	2\\
170	2\\
171	2\\
172	2\\
173	2\\
174	2\\
175	2\\
176	2\\
177	2\\
178	2\\
179	2\\
180	2\\
181	2\\
182	2\\
183	2\\
184	2\\
185	2\\
186	2\\
187	2\\
188	2\\
189	2\\
190	2\\
191	2\\
192	2\\
193	2\\
194	2\\
195	2\\
196	2\\
197	2\\
198	2\\
199	2\\
200	2\\
201	2\\
202	2\\
203	2\\
204	2\\
205	2\\
206	2\\
207	2\\
208	2\\
209	2\\
210	2\\
211	2\\
212	2\\
213	2\\
214	2\\
215	2\\
216	2\\
217	2\\
218	2\\
219	2\\
220	2\\
221	2\\
222	2\\
223	2\\
224	2\\
225	2\\
226	2\\
227	2\\
228	2\\
229	2\\
230	2\\
231	2\\
232	2\\
233	2\\
234	2\\
235	2\\
236	2\\
237	2\\
238	2\\
239	2\\
240	2\\
241	2\\
242	2\\
243	2\\
244	2\\
245	2\\
246	2\\
247	2\\
248	2\\
249	2\\
250	2\\
251	2\\
252	2\\
253	2\\
254	2\\
255	2\\
256	2\\
257	2\\
258	2\\
259	2\\
260	2\\
261	2\\
262	2\\
263	2\\
264	2\\
265	2\\
266	2\\
267	2\\
268	2\\
269	2\\
270	2\\
271	2\\
272	2\\
273	2\\
274	2\\
275	2\\
276	2\\
277	2\\
278	2\\
279	2\\
280	2\\
281	2\\
282	2\\
283	2\\
284	2\\
285	2\\
286	2\\
287	2\\
288	2\\
289	2\\
290	2\\
291	2\\
292	2\\
293	2\\
294	2\\
295	2\\
296	2\\
297	2\\
298	2\\
299	2\\
300	2\\
301	2\\
302	2\\
303	2\\
304	2\\
305	2\\
306	2\\
307	2\\
308	2\\
309	2\\
310	2\\
311	2\\
312	2\\
313	2\\
314	2\\
315	2\\
316	2\\
317	2\\
318	2\\
319	2\\
320	2\\
321	2\\
322	2\\
323	2\\
324	2\\
325	2\\
326	2\\
327	2\\
328	2\\
329	2\\
330	2\\
331	2\\
332	2\\
333	2\\
334	2\\
335	2\\
336	2\\
337	2\\
338	2\\
339	2\\
340	2\\
341	2\\
342	2\\
343	2\\
344	2\\
345	2\\
346	2\\
347	2\\
348	2\\
349	2\\
350	2\\
351	2\\
352	2\\
353	2\\
354	2\\
355	2\\
356	2\\
357	2\\
358	2\\
359	2\\
360	2\\
361	2\\
362	2\\
363	2\\
364	2\\
365	2\\
366	2\\
367	2\\
368	2\\
369	2\\
370	2\\
371	2\\
372	2\\
373	2\\
374	2\\
375	2\\
376	2\\
377	2\\
378	2\\
379	2\\
380	2\\
381	2\\
382	2\\
383	2\\
384	2\\
385	2\\
386	2\\
387	2\\
388	2\\
389	2\\
390	2\\
391	2\\
392	2\\
393	2\\
394	2\\
395	2\\
396	2\\
397	2\\
398	2\\
399	2\\
400	2\\
401	2\\
402	2\\
403	2\\
404	2\\
405	2\\
406	2\\
407	2\\
408	2\\
409	2\\
410	2\\
411	2\\
412	2\\
413	2\\
414	2\\
415	2\\
416	2\\
417	2\\
418	2\\
419	2\\
420	2\\
421	2\\
422	2\\
423	2\\
424	2\\
425	2\\
426	2\\
427	2\\
428	2\\
429	2\\
430	2\\
431	2\\
432	2\\
433	2\\
434	2\\
435	2\\
436	2\\
437	2\\
438	2\\
439	2\\
440	2\\
441	2\\
442	2\\
443	2\\
444	2\\
445	2\\
446	2\\
447	2\\
448	2\\
449	2\\
450	2\\
451	2\\
452	2\\
453	2\\
454	2\\
455	2\\
456	2\\
457	2\\
458	2\\
459	2\\
460	2\\
461	2\\
462	2\\
463	2\\
464	2\\
465	2\\
466	2\\
467	2\\
468	2\\
469	2\\
470	2\\
471	2\\
472	2\\
473	2\\
474	2\\
475	2\\
476	2\\
477	2\\
478	2\\
479	2\\
480	2\\
481	2\\
482	2\\
483	2\\
484	2\\
485	2\\
486	2\\
487	2\\
488	2\\
489	2\\
490	2\\
491	2\\
492	2\\
493	2\\
494	2\\
495	2\\
496	2\\
};
%\addlegendentry{Traj(s=2)}
\end{axis}
\end{tikzpicture}%
    \caption{Estimated x-Axis Positions}
	\end{subfigure}
	\begin{subfigure}{0.49\textwidth}
		 \centering
        \setlength{\figurewidth}{0.8\textwidth}
        % This file was created by matlab2tikz.
%
\definecolor{lms_red}{rgb}{0.80000,0.20780,0.21960}%
\definecolor{mycolor2}{rgb}{0.80000,0.20784,0.21961}%
\definecolor{mycolor3}{rgb}{0.92900,0.69400,0.12500}%
\definecolor{mycolor4}{rgb}{0.49400,0.18400,0.55600}%
%
\begin{tikzpicture}

\begin{axis}[%
width=0.951\figurewidth,
height=\figureheight,
at={(0\figurewidth,0\figureheight)},
scale only axis,
xmin=0,
xmax=496,
xtick={0,99.2,198.4,297.6,396.8,496},
xticklabels={{0},{1},{2},{3},{4},{5}},
xlabel style={font=\color{white!15!black}},
xlabel={t},
ymin=1,
ymax=5,
ylabel style={font=\color{white!15!black}},
ylabel={$p_y^{(t)}$~[m]},
axis background/.style={fill=white},
axis x line*=bottom,
axis y line*=left,
xmajorgrids,
ymajorgrids
]
\addplot [color=mycolor2, draw=none, mark=x, mark options={solid, mycolor2}, forget plot]
  table[row sep=crcr]{%
1	4.7\\
2	2.4\\
3	2.2\\
4	2.1\\
5	2.1\\
6	2\\
7	2\\
8	2\\
9	2\\
10	2\\
11	2\\
12	2\\
13	2\\
14	2\\
15	2\\
16	2\\
17	2\\
18	2\\
19	3.9\\
20	3.9\\
21	3.9\\
22	3.9\\
23	3.9\\
24	3.9\\
25	3.9\\
26	3.9\\
27	3.9\\
28	3.9\\
29	3.9\\
30	3.9\\
31	3.9\\
32	3.9\\
33	3.9\\
34	3.9\\
35	3.8\\
36	3.8\\
37	3.8\\
38	3.8\\
39	3.8\\
40	3.8\\
41	3.8\\
42	3.8\\
43	3.8\\
44	3.8\\
45	3.8\\
46	3.8\\
47	3.8\\
48	3.8\\
49	3.8\\
50	3.9\\
51	4\\
52	4\\
53	4\\
54	3.9\\
55	4\\
56	4\\
57	2.3\\
58	2.3\\
59	2.3\\
60	2.3\\
61	2.3\\
62	3.9\\
63	3.9\\
64	3.9\\
65	3.9\\
66	2.3\\
67	2.3\\
68	2.3\\
69	2.3\\
70	2.3\\
71	2.3\\
72	2.3\\
73	2.3\\
74	2.3\\
75	2.3\\
76	2.3\\
77	3.8\\
78	3.8\\
79	3.8\\
80	3.8\\
81	3.8\\
82	3.6\\
83	3.6\\
84	3.6\\
85	3.6\\
86	3.6\\
87	3.6\\
88	3.6\\
89	3.6\\
90	3.6\\
91	3.6\\
92	3.6\\
93	3.6\\
94	3.6\\
95	3.6\\
96	3.6\\
97	3.6\\
98	3.6\\
99	3.6\\
100	3.6\\
101	3.6\\
102	3.6\\
103	3.6\\
104	3.6\\
105	3.6\\
106	3.6\\
107	3.6\\
108	3.6\\
109	3.6\\
110	3.6\\
111	3.6\\
112	3.6\\
113	3.6\\
114	3.6\\
115	3.6\\
116	3.6\\
117	3.6\\
118	3.6\\
119	3.6\\
120	3.6\\
121	3.6\\
122	3.6\\
123	3.6\\
124	3.6\\
125	1.2\\
126	1.2\\
127	1.2\\
128	1.2\\
129	1.2\\
130	1.2\\
131	1.2\\
132	1.2\\
133	1.2\\
134	1.2\\
135	1.2\\
136	1.2\\
137	1.2\\
138	1.2\\
139	1.2\\
140	1.2\\
141	1.2\\
142	1.2\\
143	1.2\\
144	1.2\\
145	1.2\\
146	1.2\\
147	1.2\\
148	1.2\\
149	3.5\\
150	3.5\\
151	3.5\\
152	3.5\\
153	3.5\\
154	3.5\\
155	3.5\\
156	3.5\\
157	3.5\\
158	3.5\\
159	3.5\\
160	3.5\\
161	3.4\\
162	3.4\\
163	3.4\\
164	3.4\\
165	3.4\\
166	3.4\\
167	3.4\\
168	3.4\\
169	3.4\\
170	3.4\\
171	3.4\\
172	3.4\\
173	3.4\\
174	3.4\\
175	3.4\\
176	3.4\\
177	3.4\\
178	3.4\\
179	3.4\\
180	3.4\\
181	3.4\\
182	3.4\\
183	3.4\\
184	3.4\\
185	3.4\\
186	3.4\\
187	3.4\\
188	3.4\\
189	3.4\\
190	3.4\\
191	3.4\\
192	3.4\\
193	3.4\\
194	3.4\\
195	3.4\\
196	3.4\\
197	3.4\\
198	3.4\\
199	3.4\\
200	3.4\\
201	3.4\\
202	3.4\\
203	3.4\\
204	3.4\\
205	3.4\\
206	3.4\\
207	3.3\\
208	3.3\\
209	3.3\\
210	3.3\\
211	3.3\\
212	3.3\\
213	3.3\\
214	3.3\\
215	3.3\\
216	3.3\\
217	3.3\\
218	3.3\\
219	3.3\\
220	3.3\\
221	3.3\\
222	3.3\\
223	3.3\\
224	3.3\\
225	3.3\\
226	3.3\\
227	3.3\\
228	3.3\\
229	3.3\\
230	3.2\\
231	3.2\\
232	3.3\\
233	3.3\\
234	3.3\\
235	3.3\\
236	3.3\\
237	3.3\\
238	2.8\\
239	2.8\\
240	2.8\\
241	2.8\\
242	2.8\\
243	2.8\\
244	2.8\\
245	2.8\\
246	2.8\\
247	2.8\\
248	2.8\\
249	2.8\\
250	2.8\\
251	2.8\\
252	2.9\\
253	2.9\\
254	2.9\\
255	2.9\\
256	2.9\\
257	2.9\\
258	2.9\\
259	2.9\\
260	2.9\\
261	2.9\\
262	2.9\\
263	2.9\\
264	2.9\\
265	2.9\\
266	2.9\\
267	2.9\\
268	2.9\\
269	2.9\\
270	2.9\\
271	2.9\\
272	2.9\\
273	3\\
274	3\\
275	3\\
276	3\\
277	3\\
278	3\\
279	3\\
280	3\\
281	3\\
282	3\\
283	3\\
284	3\\
285	3\\
286	3\\
287	3\\
288	3\\
289	3\\
290	3\\
291	3\\
292	3\\
293	3\\
294	3\\
295	3\\
296	3\\
297	3\\
298	3\\
299	3\\
300	3\\
301	3\\
302	3\\
303	3\\
304	3\\
305	3\\
306	2.9\\
307	2.9\\
308	2.9\\
309	2.9\\
310	2.8\\
311	2.8\\
312	2.8\\
313	2.8\\
314	2.8\\
315	2.8\\
316	2.8\\
317	2.8\\
318	2.8\\
319	2.8\\
320	2.8\\
321	2.8\\
322	2.8\\
323	2.8\\
324	2.8\\
325	2.8\\
326	2.8\\
327	2.8\\
328	2.8\\
329	2.8\\
330	2.8\\
331	2.8\\
332	2.8\\
333	2.8\\
334	2.8\\
335	2.8\\
336	2.8\\
337	2.8\\
338	2.8\\
339	2.8\\
340	2.8\\
341	2.8\\
342	2.8\\
343	2.8\\
344	2.8\\
345	2.8\\
346	2.8\\
347	2.8\\
348	2.8\\
349	2.8\\
350	2.8\\
351	2.8\\
352	2.8\\
353	2.8\\
354	2.8\\
355	2.8\\
356	2.8\\
357	2.8\\
358	2.8\\
359	2.8\\
360	2.8\\
361	2.8\\
362	2.8\\
363	2.8\\
364	2.8\\
365	2.8\\
366	2.7\\
367	2.7\\
368	2.7\\
369	2.7\\
370	2.7\\
371	2.7\\
372	2.7\\
373	2.7\\
374	2.7\\
375	2.7\\
376	2.7\\
377	2.7\\
378	2.7\\
379	2.7\\
380	2.7\\
381	2.7\\
382	2.7\\
383	2.7\\
384	2.7\\
385	2.7\\
386	2.7\\
387	2.7\\
388	2.7\\
389	2.7\\
390	2.7\\
391	2.7\\
392	2.7\\
393	2.7\\
394	2.7\\
395	2.7\\
396	2.7\\
397	2.7\\
398	2.7\\
399	2.7\\
400	2.7\\
401	2.7\\
402	2.7\\
403	2.7\\
404	2.7\\
405	2.7\\
406	3.5\\
407	3.5\\
408	3.5\\
409	3.5\\
410	3.5\\
411	3.5\\
412	3.5\\
413	2.7\\
414	2.6\\
415	2.6\\
416	2.6\\
417	2.6\\
418	2.6\\
419	2.6\\
420	2.6\\
421	2.6\\
422	2.6\\
423	2.5\\
424	2.5\\
425	2.5\\
426	2.5\\
427	2.5\\
428	2.5\\
429	2.5\\
430	2.5\\
431	2.5\\
432	2.5\\
433	2.5\\
434	2.5\\
435	2.5\\
436	2.5\\
437	2.5\\
438	2.5\\
439	2.5\\
440	2.5\\
441	2.5\\
442	2.5\\
443	2.5\\
444	2.5\\
445	2.5\\
446	2.5\\
447	2.4\\
448	2.4\\
449	2.4\\
450	2.4\\
451	2.4\\
452	2.4\\
453	2.4\\
454	2.4\\
455	2.4\\
456	2.4\\
457	2.4\\
458	2.4\\
459	2.4\\
460	2.4\\
461	2.4\\
462	2.4\\
463	2.4\\
464	2.4\\
465	2.4\\
466	2.4\\
467	2.4\\
468	2.4\\
469	2.4\\
470	2.4\\
471	2.4\\
472	3.6\\
473	2.4\\
474	2.4\\
475	2.4\\
476	3.6\\
477	3.6\\
478	3.6\\
479	3.6\\
480	3.6\\
481	3.6\\
482	3.6\\
483	3.6\\
484	3.6\\
485	3.6\\
486	3.6\\
487	3.6\\
488	3.6\\
489	3.6\\
490	3.6\\
491	3.6\\
492	3.6\\
493	3.6\\
494	3.6\\
495	3.6\\
496	3.6\\
};
\addplot [color=mycolor2, draw=none, mark=x, mark options={solid, mycolor2}, forget plot]
  table[row sep=crcr]{%
1	1.3\\
2	1.2\\
3	2\\
4	2.2\\
5	2.2\\
6	2.2\\
7	2.2\\
8	2.2\\
9	2.2\\
10	2.2\\
11	2.2\\
12	1.4\\
13	1.4\\
14	1.4\\
15	1.4\\
16	3.9\\
17	3.9\\
18	3.9\\
19	2\\
20	2\\
21	2\\
22	2\\
23	2\\
24	2\\
25	2\\
26	2\\
27	2\\
28	2\\
29	2\\
30	2.1\\
31	2.1\\
32	2.1\\
33	2.1\\
34	3.8\\
35	3.8\\
36	3.8\\
37	3.8\\
38	3.8\\
39	3.8\\
40	3.8\\
41	3.8\\
42	3.8\\
43	3.8\\
44	3.8\\
45	3.8\\
46	3.8\\
47	3.8\\
48	3.9\\
49	3.8\\
50	3.8\\
51	3.8\\
52	3.8\\
53	3.8\\
54	3.8\\
55	3.8\\
56	3.8\\
57	4\\
58	3.9\\
59	3.9\\
60	3.9\\
61	3.9\\
62	2.3\\
63	2.3\\
64	2.3\\
65	2.3\\
66	3.9\\
67	3.9\\
68	2.3\\
69	2.3\\
70	2.3\\
71	2.3\\
72	2.3\\
73	2.3\\
74	3.8\\
75	3.8\\
76	3.8\\
77	2.3\\
78	2.3\\
79	2.3\\
80	3.7\\
81	3.6\\
82	3.8\\
83	3.8\\
84	3.8\\
85	3.8\\
86	3.8\\
87	3.8\\
88	3.8\\
89	3.8\\
90	3.8\\
91	3.8\\
92	3.8\\
93	3.8\\
94	3.8\\
95	3.8\\
96	3.8\\
97	3.8\\
98	3.8\\
99	3.8\\
100	3.8\\
101	3.8\\
102	3.8\\
103	3.8\\
104	3.8\\
105	3.8\\
106	3.8\\
107	3.8\\
108	3.8\\
109	1.2\\
110	1.2\\
111	1.2\\
112	1.2\\
113	1.2\\
114	1.2\\
115	1.2\\
116	1.2\\
117	1.2\\
118	1.2\\
119	1.2\\
120	1.2\\
121	2.2\\
122	2.2\\
123	2.2\\
124	2.2\\
125	3.5\\
126	2.2\\
127	2.2\\
128	2.2\\
129	2.2\\
130	2.2\\
131	2.2\\
132	2.2\\
133	2.2\\
134	2.2\\
135	2.2\\
136	2.2\\
137	2.2\\
138	2.2\\
139	2.2\\
140	2.2\\
141	2.2\\
142	2.2\\
143	2.2\\
144	2.2\\
145	2.2\\
146	2.2\\
147	2.2\\
148	2.2\\
149	1.2\\
150	1.2\\
151	1.2\\
152	2.2\\
153	2.2\\
154	3\\
155	3\\
156	3\\
157	3\\
158	3\\
159	3\\
160	3\\
161	3\\
162	3\\
163	3\\
164	3\\
165	3\\
166	3\\
167	2.5\\
168	2.5\\
169	2.2\\
170	2.2\\
171	2.2\\
172	2.2\\
173	2.2\\
174	3\\
175	1.2\\
176	1.2\\
177	1.2\\
178	1.2\\
179	1.2\\
180	1.2\\
181	1.2\\
182	1.2\\
183	1.2\\
184	1.2\\
185	1.2\\
186	1.2\\
187	1.2\\
188	1.2\\
189	1.2\\
190	1.2\\
191	1.2\\
192	1.2\\
193	1.2\\
194	1.2\\
195	1.2\\
196	1.2\\
197	1.2\\
198	2.2\\
199	2.2\\
200	2.2\\
201	2.2\\
202	1.2\\
203	1.2\\
204	1.2\\
205	1.2\\
206	1.2\\
207	2.2\\
208	2.2\\
209	2.2\\
210	2.7\\
211	2.7\\
212	2.7\\
213	2.7\\
214	2.7\\
215	2.7\\
216	2.7\\
217	2.7\\
218	2.7\\
219	2.7\\
220	2.7\\
221	2.7\\
222	2.7\\
223	2.7\\
224	2.8\\
225	2.7\\
226	2.7\\
227	2.8\\
228	2.8\\
229	2.8\\
230	2.8\\
231	2.8\\
232	2.8\\
233	2.8\\
234	2.8\\
235	2.8\\
236	2.8\\
237	2.8\\
238	3.3\\
239	3.3\\
240	3.3\\
241	3.3\\
242	3.3\\
243	3.3\\
244	3.2\\
245	3.2\\
246	3.2\\
247	3.2\\
248	3.2\\
249	3.2\\
250	3.2\\
251	3.2\\
252	3.2\\
253	3.2\\
254	3.2\\
255	3.2\\
256	3.2\\
257	3.2\\
258	3.2\\
259	3.2\\
260	3.2\\
261	3.2\\
262	3.2\\
263	3.2\\
264	3.2\\
265	3.2\\
266	3.2\\
267	3.2\\
268	3.2\\
269	3.2\\
270	3.2\\
271	3.1\\
272	3\\
273	2.9\\
274	2.9\\
275	2.9\\
276	2.9\\
277	2.9\\
278	2.9\\
279	2.9\\
280	2.9\\
281	2.9\\
282	2.9\\
283	2.9\\
284	2.9\\
285	2.9\\
286	2.9\\
287	2.9\\
288	2.9\\
289	2.9\\
290	2.9\\
291	2.9\\
292	2.9\\
293	2.9\\
294	2.9\\
295	2.9\\
296	2.8\\
297	2.8\\
298	2.8\\
299	2.8\\
300	2.8\\
301	3\\
302	3\\
303	3\\
304	3\\
305	2.9\\
306	3\\
307	3\\
308	3\\
309	3\\
310	3\\
311	3\\
312	3\\
313	3\\
314	3\\
315	3\\
316	3\\
317	3\\
318	3\\
319	3\\
320	2.9\\
321	2.9\\
322	2.9\\
323	2.9\\
324	2.9\\
325	2.9\\
326	2.9\\
327	1.2\\
328	1.2\\
329	1.2\\
330	1.2\\
331	1.2\\
332	1.2\\
333	2.9\\
334	1.2\\
335	1.2\\
336	1.2\\
337	1.2\\
338	1.2\\
339	1.2\\
340	1.2\\
341	1.2\\
342	1.2\\
343	1.2\\
344	1.2\\
345	1.2\\
346	1.2\\
347	1.2\\
348	1.2\\
349	1.2\\
350	1.2\\
351	1.2\\
352	1.2\\
353	3.5\\
354	3.5\\
355	3.5\\
356	3.5\\
357	3.5\\
358	3.5\\
359	3.5\\
360	3.5\\
361	3.5\\
362	3.5\\
363	1.2\\
364	1.2\\
365	1.2\\
366	1.2\\
367	1.2\\
368	1.2\\
369	1.2\\
370	3.5\\
371	3.5\\
372	3.5\\
373	3.6\\
374	3.5\\
375	3.5\\
376	3.5\\
377	3.5\\
378	3.5\\
379	3.5\\
380	3.5\\
381	3.5\\
382	3.5\\
383	3.5\\
384	3.5\\
385	3.5\\
386	3.5\\
387	3.5\\
388	3.5\\
389	3.5\\
390	3.5\\
391	3.5\\
392	3.5\\
393	3.5\\
394	3.5\\
395	3.5\\
396	3.5\\
397	3.5\\
398	3.5\\
399	3.5\\
400	3.5\\
401	3.5\\
402	3.5\\
403	3.5\\
404	3.5\\
405	3.5\\
406	2.7\\
407	2.7\\
408	2.7\\
409	2.7\\
410	2.7\\
411	2.7\\
412	2.7\\
413	3.5\\
414	3.5\\
415	3.5\\
416	3.5\\
417	3.5\\
418	3.5\\
419	3.5\\
420	2.1\\
421	2.1\\
422	2.1\\
423	3.5\\
424	3.5\\
425	3.5\\
426	3.5\\
427	3.5\\
428	1.9\\
429	1.9\\
430	1.9\\
431	3.5\\
432	3.5\\
433	3.5\\
434	3.5\\
435	1.9\\
436	1.9\\
437	1.9\\
438	1.9\\
439	3.5\\
440	3.5\\
441	3.5\\
442	1.2\\
443	1.2\\
444	1.2\\
445	1.2\\
446	1.9\\
447	1.2\\
448	1.2\\
449	1.2\\
450	1.2\\
451	1.2\\
452	1.2\\
453	2.4\\
454	2.4\\
455	2.4\\
456	2.4\\
457	3.5\\
458	3.5\\
459	3.5\\
460	3.5\\
461	3.5\\
462	3.5\\
463	3.6\\
464	3.6\\
465	3.6\\
466	3.6\\
467	3.6\\
468	3.6\\
469	3.6\\
470	3.6\\
471	3.6\\
472	2.4\\
473	3.6\\
474	3.6\\
475	3.6\\
476	2.4\\
477	2.4\\
478	2.4\\
479	2.4\\
480	2.4\\
481	2.5\\
482	2.5\\
483	2.4\\
484	2.4\\
485	2.5\\
486	2.5\\
487	2.5\\
488	2.5\\
489	2.5\\
490	2.5\\
491	2.5\\
492	2.5\\
493	2.5\\
494	2.5\\
495	2.5\\
496	2.5\\
};
\addplot [color=mycolor3, dashed, forget plot]
  table[row sep=crcr]{%
1	2\\
2	2.0040404040404\\
3	2.00808080808081\\
4	2.01212121212121\\
5	2.01616161616162\\
6	2.02020202020202\\
7	2.02424242424242\\
8	2.02828282828283\\
9	2.03232323232323\\
10	2.03636363636364\\
11	2.04040404040404\\
12	2.04444444444444\\
13	2.04848484848485\\
14	2.05252525252525\\
15	2.05656565656566\\
16	2.06060606060606\\
17	2.06464646464646\\
18	2.06868686868687\\
19	2.07272727272727\\
20	2.07676767676768\\
21	2.08080808080808\\
22	2.08484848484848\\
23	2.08888888888889\\
24	2.09292929292929\\
25	2.0969696969697\\
26	2.1010101010101\\
27	2.1050505050505\\
28	2.10909090909091\\
29	2.11313131313131\\
30	2.11717171717172\\
31	2.12121212121212\\
32	2.12525252525253\\
33	2.12929292929293\\
34	2.13333333333333\\
35	2.13737373737374\\
36	2.14141414141414\\
37	2.14545454545455\\
38	2.14949494949495\\
39	2.15353535353535\\
40	2.15757575757576\\
41	2.16161616161616\\
42	2.16565656565657\\
43	2.16969696969697\\
44	2.17373737373737\\
45	2.17777777777778\\
46	2.18181818181818\\
47	2.18585858585859\\
48	2.18989898989899\\
49	2.19393939393939\\
50	2.1979797979798\\
51	2.2020202020202\\
52	2.20606060606061\\
53	2.21010101010101\\
54	2.21414141414141\\
55	2.21818181818182\\
56	2.22222222222222\\
57	2.22626262626263\\
58	2.23030303030303\\
59	2.23434343434343\\
60	2.23838383838384\\
61	2.24242424242424\\
62	2.24646464646465\\
63	2.25050505050505\\
64	2.25454545454545\\
65	2.25858585858586\\
66	2.26262626262626\\
67	2.26666666666667\\
68	2.27070707070707\\
69	2.27474747474747\\
70	2.27878787878788\\
71	2.28282828282828\\
72	2.28686868686869\\
73	2.29090909090909\\
74	2.2949494949495\\
75	2.2989898989899\\
76	2.3030303030303\\
77	2.30707070707071\\
78	2.31111111111111\\
79	2.31515151515151\\
80	2.31919191919192\\
81	2.32323232323232\\
82	2.32727272727273\\
83	2.33131313131313\\
84	2.33535353535354\\
85	2.33939393939394\\
86	2.34343434343434\\
87	2.34747474747475\\
88	2.35151515151515\\
89	2.35555555555556\\
90	2.35959595959596\\
91	2.36363636363636\\
92	2.36767676767677\\
93	2.37171717171717\\
94	2.37575757575758\\
95	2.37979797979798\\
96	2.38383838383838\\
97	2.38787878787879\\
98	2.39191919191919\\
99	2.3959595959596\\
100	2.4\\
101	2.4040404040404\\
102	2.40808080808081\\
103	2.41212121212121\\
104	2.41616161616162\\
105	2.42020202020202\\
106	2.42424242424242\\
107	2.42828282828283\\
108	2.43232323232323\\
109	2.43636363636364\\
110	2.44040404040404\\
111	2.44444444444444\\
112	2.44848484848485\\
113	2.45252525252525\\
114	2.45656565656566\\
115	2.46060606060606\\
116	2.46464646464646\\
117	2.46868686868687\\
118	2.47272727272727\\
119	2.47676767676768\\
120	2.48080808080808\\
121	2.48484848484848\\
122	2.48888888888889\\
123	2.49292929292929\\
124	2.4969696969697\\
125	2.5010101010101\\
126	2.50505050505051\\
127	2.50909090909091\\
128	2.51313131313131\\
129	2.51717171717172\\
130	2.52121212121212\\
131	2.52525252525253\\
132	2.52929292929293\\
133	2.53333333333333\\
134	2.53737373737374\\
135	2.54141414141414\\
136	2.54545454545455\\
137	2.54949494949495\\
138	2.55353535353535\\
139	2.55757575757576\\
140	2.56161616161616\\
141	2.56565656565657\\
142	2.56969696969697\\
143	2.57373737373737\\
144	2.57777777777778\\
145	2.58181818181818\\
146	2.58585858585859\\
147	2.58989898989899\\
148	2.59393939393939\\
149	2.5979797979798\\
150	2.6020202020202\\
151	2.60606060606061\\
152	2.61010101010101\\
153	2.61414141414141\\
154	2.61818181818182\\
155	2.62222222222222\\
156	2.62626262626263\\
157	2.63030303030303\\
158	2.63434343434343\\
159	2.63838383838384\\
160	2.64242424242424\\
161	2.64646464646465\\
162	2.65050505050505\\
163	2.65454545454545\\
164	2.65858585858586\\
165	2.66262626262626\\
166	2.66666666666667\\
167	2.67070707070707\\
168	2.67474747474747\\
169	2.67878787878788\\
170	2.68282828282828\\
171	2.68686868686869\\
172	2.69090909090909\\
173	2.69494949494949\\
174	2.6989898989899\\
175	2.7030303030303\\
176	2.70707070707071\\
177	2.71111111111111\\
178	2.71515151515152\\
179	2.71919191919192\\
180	2.72323232323232\\
181	2.72727272727273\\
182	2.73131313131313\\
183	2.73535353535354\\
184	2.73939393939394\\
185	2.74343434343434\\
186	2.74747474747475\\
187	2.75151515151515\\
188	2.75555555555556\\
189	2.75959595959596\\
190	2.76363636363636\\
191	2.76767676767677\\
192	2.77171717171717\\
193	2.77575757575758\\
194	2.77979797979798\\
195	2.78383838383838\\
196	2.78787878787879\\
197	2.79191919191919\\
198	2.7959595959596\\
199	2.8\\
200	2.8040404040404\\
201	2.80808080808081\\
202	2.81212121212121\\
203	2.81616161616162\\
204	2.82020202020202\\
205	2.82424242424242\\
206	2.82828282828283\\
207	2.83232323232323\\
208	2.83636363636364\\
209	2.84040404040404\\
210	2.84444444444444\\
211	2.84848484848485\\
212	2.85252525252525\\
213	2.85656565656566\\
214	2.86060606060606\\
215	2.86464646464646\\
216	2.86868686868687\\
217	2.87272727272727\\
218	2.87676767676768\\
219	2.88080808080808\\
220	2.88484848484848\\
221	2.88888888888889\\
222	2.89292929292929\\
223	2.8969696969697\\
224	2.9010101010101\\
225	2.90505050505051\\
226	2.90909090909091\\
227	2.91313131313131\\
228	2.91717171717172\\
229	2.92121212121212\\
230	2.92525252525253\\
231	2.92929292929293\\
232	2.93333333333333\\
233	2.93737373737374\\
234	2.94141414141414\\
235	2.94545454545455\\
236	2.94949494949495\\
237	2.95353535353535\\
238	2.95757575757576\\
239	2.96161616161616\\
240	2.96565656565657\\
241	2.96969696969697\\
242	2.97373737373737\\
243	2.97777777777778\\
244	2.98181818181818\\
245	2.98585858585859\\
246	2.98989898989899\\
247	2.99393939393939\\
248	2.9979797979798\\
249	3.0020202020202\\
250	3.00606060606061\\
251	3.01010101010101\\
252	3.01414141414141\\
253	3.01818181818182\\
254	3.02222222222222\\
255	3.02626262626263\\
256	3.03030303030303\\
257	3.03434343434343\\
258	3.03838383838384\\
259	3.04242424242424\\
260	3.04646464646465\\
261	3.05050505050505\\
262	3.05454545454545\\
263	3.05858585858586\\
264	3.06262626262626\\
265	3.06666666666667\\
266	3.07070707070707\\
267	3.07474747474747\\
268	3.07878787878788\\
269	3.08282828282828\\
270	3.08686868686869\\
271	3.09090909090909\\
272	3.09494949494949\\
273	3.0989898989899\\
274	3.1030303030303\\
275	3.10707070707071\\
276	3.11111111111111\\
277	3.11515151515152\\
278	3.11919191919192\\
279	3.12323232323232\\
280	3.12727272727273\\
281	3.13131313131313\\
282	3.13535353535354\\
283	3.13939393939394\\
284	3.14343434343434\\
285	3.14747474747475\\
286	3.15151515151515\\
287	3.15555555555556\\
288	3.15959595959596\\
289	3.16363636363636\\
290	3.16767676767677\\
291	3.17171717171717\\
292	3.17575757575758\\
293	3.17979797979798\\
294	3.18383838383838\\
295	3.18787878787879\\
296	3.19191919191919\\
297	3.1959595959596\\
298	3.2\\
299	3.2040404040404\\
300	3.20808080808081\\
301	3.21212121212121\\
302	3.21616161616162\\
303	3.22020202020202\\
304	3.22424242424242\\
305	3.22828282828283\\
306	3.23232323232323\\
307	3.23636363636364\\
308	3.24040404040404\\
309	3.24444444444444\\
310	3.24848484848485\\
311	3.25252525252525\\
312	3.25656565656566\\
313	3.26060606060606\\
314	3.26464646464646\\
315	3.26868686868687\\
316	3.27272727272727\\
317	3.27676767676768\\
318	3.28080808080808\\
319	3.28484848484849\\
320	3.28888888888889\\
321	3.29292929292929\\
322	3.2969696969697\\
323	3.3010101010101\\
324	3.30505050505051\\
325	3.30909090909091\\
326	3.31313131313131\\
327	3.31717171717172\\
328	3.32121212121212\\
329	3.32525252525253\\
330	3.32929292929293\\
331	3.33333333333333\\
332	3.33737373737374\\
333	3.34141414141414\\
334	3.34545454545455\\
335	3.34949494949495\\
336	3.35353535353535\\
337	3.35757575757576\\
338	3.36161616161616\\
339	3.36565656565657\\
340	3.36969696969697\\
341	3.37373737373737\\
342	3.37777777777778\\
343	3.38181818181818\\
344	3.38585858585859\\
345	3.38989898989899\\
346	3.39393939393939\\
347	3.3979797979798\\
348	3.4020202020202\\
349	3.40606060606061\\
350	3.41010101010101\\
351	3.41414141414141\\
352	3.41818181818182\\
353	3.42222222222222\\
354	3.42626262626263\\
355	3.43030303030303\\
356	3.43434343434343\\
357	3.43838383838384\\
358	3.44242424242424\\
359	3.44646464646465\\
360	3.45050505050505\\
361	3.45454545454545\\
362	3.45858585858586\\
363	3.46262626262626\\
364	3.46666666666667\\
365	3.47070707070707\\
366	3.47474747474747\\
367	3.47878787878788\\
368	3.48282828282828\\
369	3.48686868686869\\
370	3.49090909090909\\
371	3.49494949494949\\
372	3.4989898989899\\
373	3.5030303030303\\
374	3.50707070707071\\
375	3.51111111111111\\
376	3.51515151515152\\
377	3.51919191919192\\
378	3.52323232323232\\
379	3.52727272727273\\
380	3.53131313131313\\
381	3.53535353535354\\
382	3.53939393939394\\
383	3.54343434343434\\
384	3.54747474747475\\
385	3.55151515151515\\
386	3.55555555555556\\
387	3.55959595959596\\
388	3.56363636363636\\
389	3.56767676767677\\
390	3.57171717171717\\
391	3.57575757575758\\
392	3.57979797979798\\
393	3.58383838383838\\
394	3.58787878787879\\
395	3.59191919191919\\
396	3.5959595959596\\
397	3.6\\
398	3.6040404040404\\
399	3.60808080808081\\
400	3.61212121212121\\
401	3.61616161616162\\
402	3.62020202020202\\
403	3.62424242424242\\
404	3.62828282828283\\
405	3.63232323232323\\
406	3.63636363636364\\
407	3.64040404040404\\
408	3.64444444444444\\
409	3.64848484848485\\
410	3.65252525252525\\
411	3.65656565656566\\
412	3.66060606060606\\
413	3.66464646464646\\
414	3.66868686868687\\
415	3.67272727272727\\
416	3.67676767676768\\
417	3.68080808080808\\
418	3.68484848484848\\
419	3.68888888888889\\
420	3.69292929292929\\
421	3.6969696969697\\
422	3.7010101010101\\
423	3.70505050505051\\
424	3.70909090909091\\
425	3.71313131313131\\
426	3.71717171717172\\
427	3.72121212121212\\
428	3.72525252525253\\
429	3.72929292929293\\
430	3.73333333333333\\
431	3.73737373737374\\
432	3.74141414141414\\
433	3.74545454545455\\
434	3.74949494949495\\
435	3.75353535353535\\
436	3.75757575757576\\
437	3.76161616161616\\
438	3.76565656565657\\
439	3.76969696969697\\
440	3.77373737373737\\
441	3.77777777777778\\
442	3.78181818181818\\
443	3.78585858585859\\
444	3.78989898989899\\
445	3.79393939393939\\
446	3.7979797979798\\
447	3.8020202020202\\
448	3.80606060606061\\
449	3.81010101010101\\
450	3.81414141414141\\
451	3.81818181818182\\
452	3.82222222222222\\
453	3.82626262626263\\
454	3.83030303030303\\
455	3.83434343434343\\
456	3.83838383838384\\
457	3.84242424242424\\
458	3.84646464646465\\
459	3.85050505050505\\
460	3.85454545454545\\
461	3.85858585858586\\
462	3.86262626262626\\
463	3.86666666666667\\
464	3.87070707070707\\
465	3.87474747474747\\
466	3.87878787878788\\
467	3.88282828282828\\
468	3.88686868686869\\
469	3.89090909090909\\
470	3.8949494949495\\
471	3.8989898989899\\
472	3.9030303030303\\
473	3.90707070707071\\
474	3.91111111111111\\
475	3.91515151515152\\
476	3.91919191919192\\
477	3.92323232323232\\
478	3.92727272727273\\
479	3.93131313131313\\
480	3.93535353535354\\
481	3.93939393939394\\
482	3.94343434343434\\
483	3.94747474747475\\
484	3.95151515151515\\
485	3.95555555555556\\
486	3.95959595959596\\
487	3.96363636363636\\
488	3.96767676767677\\
489	3.97171717171717\\
490	3.97575757575758\\
491	3.97979797979798\\
492	3.98383838383838\\
493	3.98787878787879\\
494	3.99191919191919\\
495	3.9959595959596\\
496	4\\
};
\addplot [color=mycolor4, dashed, forget plot]
  table[row sep=crcr]{%
1	4\\
2	3.9959595959596\\
3	3.99191919191919\\
4	3.98787878787879\\
5	3.98383838383838\\
6	3.97979797979798\\
7	3.97575757575758\\
8	3.97171717171717\\
9	3.96767676767677\\
10	3.96363636363636\\
11	3.95959595959596\\
12	3.95555555555556\\
13	3.95151515151515\\
14	3.94747474747475\\
15	3.94343434343434\\
16	3.93939393939394\\
17	3.93535353535354\\
18	3.93131313131313\\
19	3.92727272727273\\
20	3.92323232323232\\
21	3.91919191919192\\
22	3.91515151515152\\
23	3.91111111111111\\
24	3.90707070707071\\
25	3.9030303030303\\
26	3.8989898989899\\
27	3.8949494949495\\
28	3.89090909090909\\
29	3.88686868686869\\
30	3.88282828282828\\
31	3.87878787878788\\
32	3.87474747474747\\
33	3.87070707070707\\
34	3.86666666666667\\
35	3.86262626262626\\
36	3.85858585858586\\
37	3.85454545454545\\
38	3.85050505050505\\
39	3.84646464646465\\
40	3.84242424242424\\
41	3.83838383838384\\
42	3.83434343434343\\
43	3.83030303030303\\
44	3.82626262626263\\
45	3.82222222222222\\
46	3.81818181818182\\
47	3.81414141414141\\
48	3.81010101010101\\
49	3.80606060606061\\
50	3.8020202020202\\
51	3.7979797979798\\
52	3.79393939393939\\
53	3.78989898989899\\
54	3.78585858585859\\
55	3.78181818181818\\
56	3.77777777777778\\
57	3.77373737373737\\
58	3.76969696969697\\
59	3.76565656565657\\
60	3.76161616161616\\
61	3.75757575757576\\
62	3.75353535353535\\
63	3.74949494949495\\
64	3.74545454545455\\
65	3.74141414141414\\
66	3.73737373737374\\
67	3.73333333333333\\
68	3.72929292929293\\
69	3.72525252525253\\
70	3.72121212121212\\
71	3.71717171717172\\
72	3.71313131313131\\
73	3.70909090909091\\
74	3.7050505050505\\
75	3.7010101010101\\
76	3.6969696969697\\
77	3.69292929292929\\
78	3.68888888888889\\
79	3.68484848484849\\
80	3.68080808080808\\
81	3.67676767676768\\
82	3.67272727272727\\
83	3.66868686868687\\
84	3.66464646464646\\
85	3.66060606060606\\
86	3.65656565656566\\
87	3.65252525252525\\
88	3.64848484848485\\
89	3.64444444444444\\
90	3.64040404040404\\
91	3.63636363636364\\
92	3.63232323232323\\
93	3.62828282828283\\
94	3.62424242424242\\
95	3.62020202020202\\
96	3.61616161616162\\
97	3.61212121212121\\
98	3.60808080808081\\
99	3.6040404040404\\
100	3.6\\
101	3.5959595959596\\
102	3.59191919191919\\
103	3.58787878787879\\
104	3.58383838383838\\
105	3.57979797979798\\
106	3.57575757575758\\
107	3.57171717171717\\
108	3.56767676767677\\
109	3.56363636363636\\
110	3.55959595959596\\
111	3.55555555555556\\
112	3.55151515151515\\
113	3.54747474747475\\
114	3.54343434343434\\
115	3.53939393939394\\
116	3.53535353535354\\
117	3.53131313131313\\
118	3.52727272727273\\
119	3.52323232323232\\
120	3.51919191919192\\
121	3.51515151515152\\
122	3.51111111111111\\
123	3.50707070707071\\
124	3.5030303030303\\
125	3.4989898989899\\
126	3.49494949494949\\
127	3.49090909090909\\
128	3.48686868686869\\
129	3.48282828282828\\
130	3.47878787878788\\
131	3.47474747474747\\
132	3.47070707070707\\
133	3.46666666666667\\
134	3.46262626262626\\
135	3.45858585858586\\
136	3.45454545454545\\
137	3.45050505050505\\
138	3.44646464646465\\
139	3.44242424242424\\
140	3.43838383838384\\
141	3.43434343434343\\
142	3.43030303030303\\
143	3.42626262626263\\
144	3.42222222222222\\
145	3.41818181818182\\
146	3.41414141414141\\
147	3.41010101010101\\
148	3.40606060606061\\
149	3.4020202020202\\
150	3.3979797979798\\
151	3.39393939393939\\
152	3.38989898989899\\
153	3.38585858585859\\
154	3.38181818181818\\
155	3.37777777777778\\
156	3.37373737373737\\
157	3.36969696969697\\
158	3.36565656565657\\
159	3.36161616161616\\
160	3.35757575757576\\
161	3.35353535353535\\
162	3.34949494949495\\
163	3.34545454545455\\
164	3.34141414141414\\
165	3.33737373737374\\
166	3.33333333333333\\
167	3.32929292929293\\
168	3.32525252525253\\
169	3.32121212121212\\
170	3.31717171717172\\
171	3.31313131313131\\
172	3.30909090909091\\
173	3.30505050505051\\
174	3.3010101010101\\
175	3.2969696969697\\
176	3.29292929292929\\
177	3.28888888888889\\
178	3.28484848484848\\
179	3.28080808080808\\
180	3.27676767676768\\
181	3.27272727272727\\
182	3.26868686868687\\
183	3.26464646464646\\
184	3.26060606060606\\
185	3.25656565656566\\
186	3.25252525252525\\
187	3.24848484848485\\
188	3.24444444444444\\
189	3.24040404040404\\
190	3.23636363636364\\
191	3.23232323232323\\
192	3.22828282828283\\
193	3.22424242424242\\
194	3.22020202020202\\
195	3.21616161616162\\
196	3.21212121212121\\
197	3.20808080808081\\
198	3.2040404040404\\
199	3.2\\
200	3.1959595959596\\
201	3.19191919191919\\
202	3.18787878787879\\
203	3.18383838383838\\
204	3.17979797979798\\
205	3.17575757575758\\
206	3.17171717171717\\
207	3.16767676767677\\
208	3.16363636363636\\
209	3.15959595959596\\
210	3.15555555555556\\
211	3.15151515151515\\
212	3.14747474747475\\
213	3.14343434343434\\
214	3.13939393939394\\
215	3.13535353535354\\
216	3.13131313131313\\
217	3.12727272727273\\
218	3.12323232323232\\
219	3.11919191919192\\
220	3.11515151515152\\
221	3.11111111111111\\
222	3.10707070707071\\
223	3.1030303030303\\
224	3.0989898989899\\
225	3.09494949494949\\
226	3.09090909090909\\
227	3.08686868686869\\
228	3.08282828282828\\
229	3.07878787878788\\
230	3.07474747474747\\
231	3.07070707070707\\
232	3.06666666666667\\
233	3.06262626262626\\
234	3.05858585858586\\
235	3.05454545454545\\
236	3.05050505050505\\
237	3.04646464646465\\
238	3.04242424242424\\
239	3.03838383838384\\
240	3.03434343434343\\
241	3.03030303030303\\
242	3.02626262626263\\
243	3.02222222222222\\
244	3.01818181818182\\
245	3.01414141414141\\
246	3.01010101010101\\
247	3.00606060606061\\
248	3.0020202020202\\
249	2.9979797979798\\
250	2.99393939393939\\
251	2.98989898989899\\
252	2.98585858585859\\
253	2.98181818181818\\
254	2.97777777777778\\
255	2.97373737373737\\
256	2.96969696969697\\
257	2.96565656565657\\
258	2.96161616161616\\
259	2.95757575757576\\
260	2.95353535353535\\
261	2.94949494949495\\
262	2.94545454545455\\
263	2.94141414141414\\
264	2.93737373737374\\
265	2.93333333333333\\
266	2.92929292929293\\
267	2.92525252525253\\
268	2.92121212121212\\
269	2.91717171717172\\
270	2.91313131313131\\
271	2.90909090909091\\
272	2.90505050505051\\
273	2.9010101010101\\
274	2.8969696969697\\
275	2.89292929292929\\
276	2.88888888888889\\
277	2.88484848484848\\
278	2.88080808080808\\
279	2.87676767676768\\
280	2.87272727272727\\
281	2.86868686868687\\
282	2.86464646464646\\
283	2.86060606060606\\
284	2.85656565656566\\
285	2.85252525252525\\
286	2.84848484848485\\
287	2.84444444444444\\
288	2.84040404040404\\
289	2.83636363636364\\
290	2.83232323232323\\
291	2.82828282828283\\
292	2.82424242424242\\
293	2.82020202020202\\
294	2.81616161616162\\
295	2.81212121212121\\
296	2.80808080808081\\
297	2.8040404040404\\
298	2.8\\
299	2.7959595959596\\
300	2.79191919191919\\
301	2.78787878787879\\
302	2.78383838383838\\
303	2.77979797979798\\
304	2.77575757575758\\
305	2.77171717171717\\
306	2.76767676767677\\
307	2.76363636363636\\
308	2.75959595959596\\
309	2.75555555555556\\
310	2.75151515151515\\
311	2.74747474747475\\
312	2.74343434343434\\
313	2.73939393939394\\
314	2.73535353535354\\
315	2.73131313131313\\
316	2.72727272727273\\
317	2.72323232323232\\
318	2.71919191919192\\
319	2.71515151515151\\
320	2.71111111111111\\
321	2.70707070707071\\
322	2.7030303030303\\
323	2.6989898989899\\
324	2.69494949494949\\
325	2.69090909090909\\
326	2.68686868686869\\
327	2.68282828282828\\
328	2.67878787878788\\
329	2.67474747474747\\
330	2.67070707070707\\
331	2.66666666666667\\
332	2.66262626262626\\
333	2.65858585858586\\
334	2.65454545454545\\
335	2.65050505050505\\
336	2.64646464646465\\
337	2.64242424242424\\
338	2.63838383838384\\
339	2.63434343434343\\
340	2.63030303030303\\
341	2.62626262626263\\
342	2.62222222222222\\
343	2.61818181818182\\
344	2.61414141414141\\
345	2.61010101010101\\
346	2.60606060606061\\
347	2.6020202020202\\
348	2.5979797979798\\
349	2.59393939393939\\
350	2.58989898989899\\
351	2.58585858585859\\
352	2.58181818181818\\
353	2.57777777777778\\
354	2.57373737373737\\
355	2.56969696969697\\
356	2.56565656565657\\
357	2.56161616161616\\
358	2.55757575757576\\
359	2.55353535353535\\
360	2.54949494949495\\
361	2.54545454545455\\
362	2.54141414141414\\
363	2.53737373737374\\
364	2.53333333333333\\
365	2.52929292929293\\
366	2.52525252525253\\
367	2.52121212121212\\
368	2.51717171717172\\
369	2.51313131313131\\
370	2.50909090909091\\
371	2.50505050505051\\
372	2.5010101010101\\
373	2.4969696969697\\
374	2.49292929292929\\
375	2.48888888888889\\
376	2.48484848484848\\
377	2.48080808080808\\
378	2.47676767676768\\
379	2.47272727272727\\
380	2.46868686868687\\
381	2.46464646464646\\
382	2.46060606060606\\
383	2.45656565656566\\
384	2.45252525252525\\
385	2.44848484848485\\
386	2.44444444444444\\
387	2.44040404040404\\
388	2.43636363636364\\
389	2.43232323232323\\
390	2.42828282828283\\
391	2.42424242424242\\
392	2.42020202020202\\
393	2.41616161616162\\
394	2.41212121212121\\
395	2.40808080808081\\
396	2.4040404040404\\
397	2.4\\
398	2.3959595959596\\
399	2.39191919191919\\
400	2.38787878787879\\
401	2.38383838383838\\
402	2.37979797979798\\
403	2.37575757575758\\
404	2.37171717171717\\
405	2.36767676767677\\
406	2.36363636363636\\
407	2.35959595959596\\
408	2.35555555555556\\
409	2.35151515151515\\
410	2.34747474747475\\
411	2.34343434343434\\
412	2.33939393939394\\
413	2.33535353535354\\
414	2.33131313131313\\
415	2.32727272727273\\
416	2.32323232323232\\
417	2.31919191919192\\
418	2.31515151515152\\
419	2.31111111111111\\
420	2.30707070707071\\
421	2.3030303030303\\
422	2.2989898989899\\
423	2.29494949494949\\
424	2.29090909090909\\
425	2.28686868686869\\
426	2.28282828282828\\
427	2.27878787878788\\
428	2.27474747474747\\
429	2.27070707070707\\
430	2.26666666666667\\
431	2.26262626262626\\
432	2.25858585858586\\
433	2.25454545454545\\
434	2.25050505050505\\
435	2.24646464646465\\
436	2.24242424242424\\
437	2.23838383838384\\
438	2.23434343434343\\
439	2.23030303030303\\
440	2.22626262626263\\
441	2.22222222222222\\
442	2.21818181818182\\
443	2.21414141414141\\
444	2.21010101010101\\
445	2.20606060606061\\
446	2.2020202020202\\
447	2.1979797979798\\
448	2.19393939393939\\
449	2.18989898989899\\
450	2.18585858585859\\
451	2.18181818181818\\
452	2.17777777777778\\
453	2.17373737373737\\
454	2.16969696969697\\
455	2.16565656565657\\
456	2.16161616161616\\
457	2.15757575757576\\
458	2.15353535353535\\
459	2.14949494949495\\
460	2.14545454545455\\
461	2.14141414141414\\
462	2.13737373737374\\
463	2.13333333333333\\
464	2.12929292929293\\
465	2.12525252525253\\
466	2.12121212121212\\
467	2.11717171717172\\
468	2.11313131313131\\
469	2.10909090909091\\
470	2.1050505050505\\
471	2.1010101010101\\
472	2.0969696969697\\
473	2.09292929292929\\
474	2.08888888888889\\
475	2.08484848484848\\
476	2.08080808080808\\
477	2.07676767676768\\
478	2.07272727272727\\
479	2.06868686868687\\
480	2.06464646464646\\
481	2.06060606060606\\
482	2.05656565656566\\
483	2.05252525252525\\
484	2.04848484848485\\
485	2.04444444444444\\
486	2.04040404040404\\
487	2.03636363636364\\
488	2.03232323232323\\
489	2.02828282828283\\
490	2.02424242424242\\
491	2.02020202020202\\
492	2.01616161616162\\
493	2.01212121212121\\
494	2.00808080808081\\
495	2.0040404040404\\
496	2\\
};
\end{axis}
\end{tikzpicture}%
    \caption{Estimated y-Axis Positions}
    \end{subfigure}
}
	\caption[Parallel Movement Results for TREM]{Parallel Movement Results for TREM (\Tsixty$=0.4$~s).}
	\label{fig:assignmentParallelTREM}
\end{figure}


Although not identical, \gls{trem} behaves similar to \gls{crem} as the results in \autoref{fig:trackingParallelTREM} demonstrate. As expected, there are more wrong x-coordinate estimates until $t=1$ and the variance has fully converged. Then, both algorithms display very similar behaviour. From $t=3$ to $t=4$ \gls{trem} accurately estimates the x-coordinate $p_x=2$, whereas \gls{crem} missed the true trajectory by 1 grid point. An overview of the position estimates over time for both \gls{trem} and \gls{crem} is provided in \autoref{fig:trackingParallelRoom}.

\begin{figure}[!htbp]
\iftoggle{quick}{%
    \includegraphics[width=\textwidth]{plots/tracking/parallel/results-T60=0.4-cremtrem-room-sc.png}
}{%
	\begin{subfigure}{0.49\textwidth}
	     \centering
        \setlength{\figurewidth}{0.8\textwidth}
        % This file was created by matlab2tikz.
%
\definecolor{lms_red}{rgb}{0.80000,0.20780,0.21960}%
%
\begin{tikzpicture}

\begin{axis}[%
width=0.951\figurewidth,
height=\figureheight,
at={(0\figurewidth,0\figureheight)},
scale only axis,
xmin=1,
xmax=5,
xlabel style={font=\color{white!15!black}},
xlabel={$p_x^{(t)}$~[m]},
ymin=1,
ymax=5,
ylabel style={font=\color{white!15!black}},
ylabel={$p_y^{(t)}$~[m]},
axis background/.style={fill=white},
axis x line*=bottom,
axis y line*=left,
xmajorgrids,
ymajorgrids,
point meta min = 0,
point meta max = 5,
colorbar horizontal, 
colorbar style={
    at={(0.5,1.03)},anchor=south,
    colormap/jet,
    samples = 25,
    xticklabel pos=upper,
    xtick style={draw=none},
    title style={yshift=0.4cm},
    title=t
},
]

\addplot[
    scatter,%
    scatter/@pre marker code/.code={%
        \edef\temp{\noexpand\definecolor{mapped color}{rgb}{\pgfplotspointmeta}}%
        \temp
        \scope[draw=mapped color!80!black,fill=mapped color]%
    },%
    scatter/@post marker code/.code={%
        \endscope
    },%
    only marks,     
    mark=o,
    mark size=3.0pt,
    line width=1.0pt,
    point meta={TeX code symbolic={%
        \edef\pgfplotspointmeta{\thisrow{R},\thisrow{G},\thisrow{B}}%
    }},
] 
table[row sep=crcr]{
x	y	R	G	B\\
4	2	0	0	0.504\\
4	2.1	0	0	0.704\\
4	2.2	0	0	0.904\\
4	2.3	0	0.104	1\\
4	2.4	0	0.304	1\\
4	2.5	0	0.504	1\\
4	2.6	0	0.704	1\\
4	2.7	0	0.904	1\\
4	2.8	0.104	1	0.896\\
4	2.9	0.304	1	0.696\\
4	3.0	0.504	1	0.496\\
4	3.1	0.704	1	0.296\\
4	3.2	0.904	1	0.096\\
4	3.3	1	0.896	0\\
4	3.4	1	0.696	0\\
4	3.5	1	0.496	0\\
4	3.6	1	0.296	0\\
4	3.7	1	0.096	0\\
4	3.8	0.896	0	0\\
4	3.9	0.696	0	0\\
4	4	0.504	0	0\\
2	2	0	0	0.504\\
2	2.1	0	0	0.704\\
2	2.2	0	0	0.904\\
2	2.3	0	0.104	1\\
2	2.4	0	0.304	1\\
2	2.5	0	0.504	1\\
2	2.6	0	0.704	1\\
2	2.7	0	0.904	1\\
2	2.8	0.104	1	0.896\\
2	2.9	0.304	1	0.696\\
2	3.0	0.504	1	0.496\\
2	3.1	0.704	1	0.296\\
2	3.2	0.904	1	0.096\\
2	3.3	1	0.896	0\\
2	3.4	1	0.696	0\\
2	3.5	1	0.496	0\\
2	3.6	1	0.296	0\\
2	3.7	1	0.096	0\\
2	3.8	0.896	0	0\\
2	3.9	0.696	0	0\\
2	4	0.504	0	0\\
};

\addplot[
    scatter,%
    scatter/@pre marker code/.code={%
        \edef\temp{\noexpand\definecolor{mapped color}{rgb}{\pgfplotspointmeta}}%
        \temp
        \scope[draw=mapped color!80!black,fill=mapped color]%
    },%
    scatter/@post marker code/.code={%
        \endscope
    },%
    only marks,     
    mark=x,
    mark size=3.0pt,
    line width=0.5pt,
    point meta={TeX code symbolic={%
        \edef\pgfplotspointmeta{\thisrow{R},\thisrow{G},\thisrow{B}}%
    }},
] 
table[row sep=crcr]{%
x	y	R	G	B\\
1.3	2.3	0	0	0.508064516129032\\
4.1	2.2	0	0	0.516129032258065\\
3.8	1.8	0	0	0.524193548387097\\
3.9	1.9	0	0	0.532258064516129\\
4	2	0	0	0.540322580645161\\
4	2	0	0	0.548387096774194\\
4	2.1	0	0	0.556451612903226\\
4	2.1	0	0	0.564516129032258\\
4	2.1	0	0	0.57258064516129\\
4	2.1	0	0	0.580645161290323\\
4	2.1	0	0	0.588709677419355\\
4	2.1	0	0	0.596774193548387\\
4	2.1	0	0	0.604838709677419\\
4	2.1	0	0	0.612903225806452\\
4	2.1	0	0	0.620967741935484\\
4	2.1	0	0	0.629032258064516\\
4	2.1	0	0	0.637096774193548\\
4	2.1	0	0	0.645161290322581\\
2	3.9	0	0	0.653225806451613\\
2	3.9	0	0	0.661290322580645\\
2	3.9	0	0	0.669354838709677\\
2	3.9	0	0	0.67741935483871\\
2	3.9	0	0	0.685483870967742\\
2	3.9	0	0	0.693548387096774\\
2	3.9	0	0	0.701612903225806\\
2	3.9	0	0	0.709677419354839\\
2	3.9	0	0	0.717741935483871\\
2	3.9	0	0	0.725806451612903\\
2	3.9	0	0	0.733870967741935\\
2	3.9	0	0	0.741935483870968\\
2	3.9	0	0	0.75\\
2	3.9	0	0	0.758064516129032\\
2	3.9	0	0	0.766129032258065\\
2.1	3.9	0	0	0.774193548387097\\
2.1	3.9	0	0	0.782258064516129\\
2.1	3.8	0	0	0.790322580645161\\
2.1	3.9	0	0	0.798387096774194\\
2.1	3.9	0	0	0.806451612903226\\
2	3.8	0	0	0.814516129032258\\
2	3.8	0	0	0.82258064516129\\
2.1	3.8	0	0	0.830645161290323\\
2	3.8	0	0	0.838709677419355\\
2	3.8	0	0	0.846774193548387\\
2	3.8	0	0	0.854838709677419\\
2.1	3.8	0	0	0.862903225806452\\
2.1	3.8	0	0	0.870967741935484\\
2.1	3.9	0	0	0.879032258064516\\
2.1	3.8	0	0	0.887096774193548\\
2.2	3.9	0	0	0.895161290322581\\
2.3	4	0	0	0.903225806451613\\
2.2	4	0	0	0.911290322580645\\
2.2	4	0	0	0.919354838709677\\
2.2	4	0	0	0.92741935483871\\
2.2	4	0	0	0.935483870967742\\
2.2	4	0	0	0.943548387096774\\
2.2	4	0	0	0.951612903225806\\
4	2.3	0	0	0.959677419354839\\
4	2.3	0	0	0.967741935483871\\
4	2.3	0	0	0.975806451612903\\
4	2.3	0	0	0.983870967741935\\
1.2	2.2	0	0	0.991935483870968\\
1.2	2.2	0	0	1\\
2.1	3.9	0	0.00806451612903226	1\\
2	3.9	0	0.0161290322580645	1\\
2.1	3.9	0	0.0241935483870968	1\\
1.2	2.2	0	0.032258064516129	1\\
1.2	2.2	0	0.0403225806451613	1\\
1.2	2.2	0	0.0483870967741935	1\\
4.7	2.3	0	0.0564516129032258	1\\
4.7	2.3	0	0.0645161290322581	1\\
4.7	2.3	0	0.0725806451612903	1\\
4.7	2.3	0	0.0806451612903226	1\\
4.7	2.3	0	0.0887096774193548	1\\
4.7	2.3	0	0.0967741935483871	1\\
4.7	2.3	0	0.104838709677419	1\\
4.7	2.3	0	0.112903225806452	1\\
2.8	1.2	0	0.120967741935484	1\\
4.7	2.3	0	0.129032258064516	1\\
4.7	2.3	0	0.137096774193548	1\\
4.7	2.3	0	0.145161290322581	1\\
4.7	2.3	0	0.153225806451613	1\\
1.9	3.7	0	0.161290322580645	1\\
1.9	3.7	0	0.169354838709677	1\\
1.9	3.7	0	0.17741935483871	1\\
1.9	3.7	0	0.185483870967742	1\\
2	3.7	0	0.193548387096774	1\\
2	3.7	0	0.201612903225806	1\\
2	3.7	0	0.209677419354839	1\\
2	3.7	0	0.217741935483871	1\\
2	3.7	0	0.225806451612903	1\\
2	3.7	0	0.233870967741935	1\\
2	3.7	0	0.241935483870968	1\\
2	3.7	0	0.25	1\\
2	3.6	0	0.258064516129032	1\\
2	3.6	0	0.266129032258065	1\\
2	3.6	0	0.274193548387097	1\\
2	3.7	0	0.282258064516129	1\\
2	3.7	0	0.290322580645161	1\\
2	3.7	0	0.298387096774194	1\\
2	3.7	0	0.306451612903226	1\\
2	3.7	0	0.314516129032258	1\\
2	3.6	0	0.32258064516129	1\\
2	3.6	0	0.330645161290323	1\\
2	3.6	0	0.338709677419355	1\\
2	3.6	0	0.346774193548387	1\\
2	3.6	0	0.354838709677419	1\\
2	3.6	0	0.362903225806452	1\\
2	3.6	0	0.370967741935484	1\\
2	3.6	0	0.379032258064516	1\\
2	3.6	0	0.387096774193548	1\\
2	3.6	0	0.395161290322581	1\\
2	3.6	0	0.403225806451613	1\\
2	3.6	0	0.411290322580645	1\\
2	3.6	0	0.419354838709677	1\\
2	3.6	0	0.42741935483871	1\\
2	3.6	0	0.435483870967742	1\\
2	3.6	0	0.443548387096774	1\\
2	3.6	0	0.451612903225806	1\\
2	3.6	0	0.459677419354839	1\\
2	3.6	0	0.467741935483871	1\\
2	3.6	0	0.475806451612903	1\\
2	3.6	0	0.483870967741935	1\\
1.2	2.2	0	0.491935483870968	1\\
1.2	2.2	0	0.5	1\\
1.2	2.2	0	0.508064516129032	1\\
1.2	2.2	0	0.516129032258065	1\\
1.2	2.2	0	0.524193548387097	1\\
1.2	2.2	0	0.532258064516129	1\\
1.2	2.2	0	0.540322580645161	1\\
1.2	2.2	0	0.548387096774194	1\\
1.2	2.2	0	0.556451612903226	1\\
1.2	2.2	0	0.564516129032258	1\\
1.2	2.2	0	0.57258064516129	1\\
1.2	2.2	0	0.580645161290323	1\\
1.2	2.2	0	0.588709677419355	1\\
1.2	2.2	0	0.596774193548387	1\\
1.2	2.2	0	0.604838709677419	1\\
1.2	2.2	0	0.612903225806452	1\\
1.2	2.2	0	0.620967741935484	1\\
1.2	2.2	0	0.629032258064516	1\\
1.2	2.2	0	0.637096774193548	1\\
4.7	2.3	0	0.645161290322581	1\\
4.7	2.3	0	0.653225806451613	1\\
4.7	2.3	0	0.661290322580645	1\\
4.7	2.3	0	0.669354838709677	1\\
1.2	3.8	0	0.67741935483871	1\\
1.2	3.8	0	0.685483870967742	1\\
1.2	3.8	0	0.693548387096774	1\\
1.2	3.8	0	0.701612903225806	1\\
2	3.6	0	0.709677419354839	1\\
2	3.6	0	0.717741935483871	1\\
2	3.6	0	0.725806451612903	1\\
2	3.6	0	0.733870967741935	1\\
2	3.5	0	0.741935483870968	1\\
2	3.5	0	0.75	1\\
2	3.5	0	0.758064516129032	1\\
2	3.5	0	0.766129032258065	1\\
2	3.5	0	0.774193548387097	1\\
2	3.5	0	0.782258064516129	1\\
2	3.5	0	0.790322580645161	1\\
2	3.5	0	0.798387096774194	1\\
2	3.4	0	0.806451612903226	1\\
2	3.4	0	0.814516129032258	1\\
2	3.4	0	0.82258064516129	1\\
2	3.4	0	0.830645161290323	1\\
2	3.4	0	0.838709677419355	1\\
2	3.4	0	0.846774193548387	1\\
2	3.4	0	0.854838709677419	1\\
2	3.4	0	0.862903225806452	1\\
2	3.4	0	0.870967741935484	1\\
2	3.4	0	0.879032258064516	1\\
2	3.4	0	0.887096774193548	1\\
2	3.4	0	0.895161290322581	1\\
2	3.4	0	0.903225806451613	1\\
2	3.4	0	0.911290322580645	1\\
2	3.4	0	0.919354838709677	1\\
2	3.4	0	0.92741935483871	1\\
2	3.4	0	0.935483870967742	1\\
2	3.4	0	0.943548387096774	1\\
2	3.4	0	0.951612903225806	1\\
2	3.4	0	0.959677419354839	1\\
2	3.4	0	0.967741935483871	1\\
2	3.4	0	0.975806451612903	1\\
2	3.4	0	0.983870967741935	1\\
2	3.4	0	0.991935483870968	1\\
2	3.4	0	1	1\\
2	3.4	0.00806451612903226	1	0.991935483870968\\
2	3.4	0.0161290322580645	1	0.983870967741935\\
2	3.4	0.0241935483870968	1	0.975806451612903\\
2	3.4	0.032258064516129	1	0.967741935483871\\
2	3.4	0.0403225806451613	1	0.959677419354839\\
2	3.4	0.0483870967741935	1	0.951612903225806\\
2	3.4	0.0564516129032258	1	0.943548387096774\\
2	3.4	0.0645161290322581	1	0.935483870967742\\
2	3.3	0.0725806451612903	1	0.92741935483871\\
2.1	3.3	0.0806451612903226	1	0.919354838709677\\
2.1	3.3	0.0887096774193548	1	0.911290322580645\\
2.1	3.3	0.0967741935483871	1	0.903225806451613\\
2.1	3.3	0.104838709677419	1	0.895161290322581\\
2.1	3.3	0.112903225806452	1	0.887096774193548\\
2.1	3.3	0.120967741935484	1	0.879032258064516\\
2.1	3.3	0.129032258064516	1	0.870967741935484\\
2.1	3.3	0.137096774193548	1	0.862903225806452\\
2.1	3.3	0.145161290322581	1	0.854838709677419\\
2.1	3.3	0.153225806451613	1	0.846774193548387\\
2.1	3.3	0.161290322580645	1	0.838709677419355\\
2.1	3.3	0.169354838709677	1	0.830645161290323\\
2.1	3.3	0.17741935483871	1	0.82258064516129\\
2.1	3.3	0.185483870967742	1	0.814516129032258\\
2.1	3.3	0.193548387096774	1	0.806451612903226\\
2.1	3.3	0.201612903225806	1	0.798387096774194\\
2.1	3.3	0.209677419354839	1	0.790322580645161\\
2.1	3.3	0.217741935483871	1	0.782258064516129\\
2.1	3.3	0.225806451612903	1	0.774193548387097\\
2.1	3.3	0.233870967741935	1	0.766129032258065\\
2.1	3.3	0.241935483870968	1	0.758064516129032\\
2.1	3.3	0.25	1	0.75\\
2.1	3.3	0.258064516129032	1	0.741935483870968\\
3.9	2.8	0.266129032258065	1	0.733870967741935\\
2.1	3.3	0.274193548387097	1	0.725806451612903\\
2	3.3	0.282258064516129	1	0.717741935483871\\
2	3.3	0.290322580645161	1	0.709677419354839\\
2.1	3.3	0.298387096774194	1	0.701612903225806\\
2.1	3.3	0.306451612903226	1	0.693548387096774\\
2.1	3.3	0.314516129032258	1	0.685483870967742\\
2.1	3.2	0.32258064516129	1	0.67741935483871\\
2.1	3.2	0.330645161290323	1	0.669354838709677\\
2.1	3.2	0.338709677419355	1	0.661290322580645\\
2.1	3.2	0.346774193548387	1	0.653225806451613\\
2.1	3.2	0.354838709677419	1	0.645161290322581\\
2.1	3.2	0.362903225806452	1	0.637096774193548\\
2.1	3.2	0.370967741935484	1	0.629032258064516\\
2.1	3.2	0.379032258064516	1	0.620967741935484\\
2.1	3.2	0.387096774193548	1	0.612903225806452\\
2.1	3.2	0.395161290322581	1	0.604838709677419\\
2.1	3.2	0.403225806451613	1	0.596774193548387\\
2.1	3.2	0.411290322580645	1	0.588709677419355\\
3.9	2.8	0.419354838709677	1	0.580645161290323\\
3.9	2.8	0.42741935483871	1	0.57258064516129\\
3.9	2.8	0.435483870967742	1	0.564516129032258\\
3.9	2.8	0.443548387096774	1	0.556451612903226\\
3.9	2.8	0.451612903225806	1	0.548387096774194\\
3.9	2.8	0.459677419354839	1	0.540322580645161\\
3.9	2.8	0.467741935483871	1	0.532258064516129\\
3.9	2.8	0.475806451612903	1	0.524193548387097\\
3.9	2.8	0.483870967741935	1	0.516129032258065\\
3.9	2.9	0.491935483870968	1	0.508064516129032\\
3.9	2.9	0.5	1	0.5\\
3.9	2.9	0.508064516129032	1	0.491935483870968\\
3.9	2.9	0.516129032258065	1	0.483870967741935\\
3.9	2.9	0.524193548387097	1	0.475806451612903\\
3.9	2.9	0.532258064516129	1	0.467741935483871\\
3.9	2.9	0.540322580645161	1	0.459677419354839\\
3.9	2.9	0.548387096774194	1	0.451612903225806\\
3.9	2.9	0.556451612903226	1	0.443548387096774\\
3.9	2.9	0.564516129032258	1	0.435483870967742\\
3.9	2.9	0.57258064516129	1	0.42741935483871\\
3.9	2.9	0.580645161290323	1	0.419354838709677\\
3.9	2.9	0.588709677419355	1	0.411290322580645\\
3.9	2.9	0.596774193548387	1	0.403225806451613\\
3.9	2.9	0.604838709677419	1	0.395161290322581\\
3.9	2.9	0.612903225806452	1	0.387096774193548\\
3.9	2.9	0.620967741935484	1	0.379032258064516\\
3.9	2.9	0.629032258064516	1	0.370967741935484\\
3.9	2.9	0.637096774193548	1	0.362903225806452\\
3.9	2.9	0.645161290322581	1	0.354838709677419\\
3.9	2.9	0.653225806451613	1	0.346774193548387\\
3.9	2.9	0.661290322580645	1	0.338709677419355\\
3.9	2.9	0.669354838709677	1	0.330645161290323\\
3.9	2.9	0.67741935483871	1	0.32258064516129\\
3.9	2.9	0.685483870967742	1	0.314516129032258\\
3.9	2.9	0.693548387096774	1	0.306451612903226\\
2.1	3	0.701612903225806	1	0.298387096774194\\
2.1	3	0.709677419354839	1	0.290322580645161\\
2.1	3	0.717741935483871	1	0.282258064516129\\
2.1	3	0.725806451612903	1	0.274193548387097\\
2.1	3	0.733870967741935	1	0.266129032258065\\
2.1	3	0.741935483870968	1	0.258064516129032\\
2.1	3	0.75	1	0.25\\
2.1	3	0.758064516129032	1	0.241935483870968\\
2.1	3	0.766129032258065	1	0.233870967741935\\
2.1	3	0.774193548387097	1	0.225806451612903\\
2.1	3	0.782258064516129	1	0.217741935483871\\
2.1	3	0.790322580645161	1	0.209677419354839\\
2.1	3	0.798387096774194	1	0.201612903225806\\
2.1	3	0.806451612903226	1	0.193548387096774\\
2.1	3	0.814516129032258	1	0.185483870967742\\
2.1	3	0.82258064516129	1	0.17741935483871\\
2.1	3	0.830645161290323	1	0.169354838709677\\
2.1	3	0.838709677419355	1	0.161290322580645\\
2.1	3	0.846774193548387	1	0.153225806451613\\
2.1	3	0.854838709677419	1	0.145161290322581\\
2.1	3	0.862903225806452	1	0.137096774193548\\
2.1	3	0.870967741935484	1	0.129032258064516\\
2.1	3	0.879032258064516	1	0.120967741935484\\
2.1	3	0.887096774193548	1	0.112903225806452\\
2.1	3	0.895161290322581	1	0.104838709677419\\
2.1	3	0.903225806451613	1	0.0967741935483871\\
2.1	3	0.911290322580645	1	0.0887096774193548\\
2.1	3	0.919354838709677	1	0.0806451612903226\\
2.1	3	0.92741935483871	1	0.0725806451612903\\
2.1	3	0.935483870967742	1	0.0645161290322581\\
2.1	3	0.943548387096774	1	0.0564516129032258\\
2.1	3	0.951612903225806	1	0.0483870967741935\\
2.1	3	0.959677419354839	1	0.0403225806451613\\
2.1	2.9	0.967741935483871	1	0.032258064516129\\
2.1	2.9	0.975806451612903	1	0.0241935483870968\\
2.1	2.9	0.983870967741935	1	0.0161290322580645\\
2.1	2.9	0.991935483870968	1	0.00806451612903226\\
2.1	2.9	1	1	0\\
2	2.8	1	0.991935483870968	0\\
2	2.8	1	0.983870967741935	0\\
2	2.8	1	0.975806451612903	0\\
2	2.8	1	0.967741935483871	0\\
2	2.8	1	0.959677419354839	0\\
2	2.8	1	0.951612903225806	0\\
2	2.8	1	0.943548387096774	0\\
2	2.8	1	0.935483870967742	0\\
2	2.8	1	0.92741935483871	0\\
2	2.8	1	0.919354838709677	0\\
2	2.8	1	0.911290322580645	0\\
2	2.8	1	0.903225806451613	0\\
2	2.8	1	0.895161290322581	0\\
2	2.8	1	0.887096774193548	0\\
2	2.8	1	0.879032258064516	0\\
2	2.8	1	0.870967741935484	0\\
2	2.8	1	0.862903225806452	0\\
2	2.8	1	0.854838709677419	0\\
2	2.8	1	0.846774193548387	0\\
2	2.8	1	0.838709677419355	0\\
2	2.8	1	0.830645161290323	0\\
2	2.8	1	0.82258064516129	0\\
2	2.8	1	0.814516129032258	0\\
2	2.8	1	0.806451612903226	0\\
2	2.8	1	0.798387096774194	0\\
2	2.8	1	0.790322580645161	0\\
2	2.8	1	0.782258064516129	0\\
2	2.8	1	0.774193548387097	0\\
2	2.8	1	0.766129032258065	0\\
2	2.8	1	0.758064516129032	0\\
2	2.8	1	0.75	0\\
2	2.8	1	0.741935483870968	0\\
2	2.8	1	0.733870967741935	0\\
2	2.8	1	0.725806451612903	0\\
2	2.8	1	0.717741935483871	0\\
2	2.8	1	0.709677419354839	0\\
2	2.8	1	0.701612903225806	0\\
2	2.8	1	0.693548387096774	0\\
2	2.8	1	0.685483870967742	0\\
2	2.8	1	0.67741935483871	0\\
2	2.8	1	0.669354838709677	0\\
2	2.8	1	0.661290322580645	0\\
2	2.8	1	0.653225806451613	0\\
2	2.8	1	0.645161290322581	0\\
2	2.8	1	0.637096774193548	0\\
2	2.8	1	0.629032258064516	0\\
2	2.8	1	0.620967741935484	0\\
2	2.8	1	0.612903225806452	0\\
2	2.8	1	0.604838709677419	0\\
2	2.8	1	0.596774193548387	0\\
2	2.8	1	0.588709677419355	0\\
2	2.8	1	0.580645161290323	0\\
2	2.8	1	0.57258064516129	0\\
2	2.7	1	0.564516129032258	0\\
2	2.7	1	0.556451612903226	0\\
2	2.7	1	0.548387096774194	0\\
2	2.7	1	0.540322580645161	0\\
2	2.7	1	0.532258064516129	0\\
2	2.7	1	0.524193548387097	0\\
2	2.7	1	0.516129032258065	0\\
2	2.7	1	0.508064516129032	0\\
2	2.7	1	0.5	0\\
2	2.7	1	0.491935483870968	0\\
2	2.7	1	0.483870967741935	0\\
2	2.7	1	0.475806451612903	0\\
2	2.7	1	0.467741935483871	0\\
2	2.7	1	0.459677419354839	0\\
2	2.7	1	0.451612903225806	0\\
2	2.7	1	0.443548387096774	0\\
2	2.7	1	0.435483870967742	0\\
2	2.7	1	0.42741935483871	0\\
2	2.7	1	0.419354838709677	0\\
2	2.7	1	0.411290322580645	0\\
2	2.7	1	0.403225806451613	0\\
2	2.7	1	0.395161290322581	0\\
2	2.7	1	0.387096774193548	0\\
2	2.7	1	0.379032258064516	0\\
2	2.7	1	0.370967741935484	0\\
2	2.7	1	0.362903225806452	0\\
2	2.7	1	0.354838709677419	0\\
2	2.7	1	0.346774193548387	0\\
2	2.7	1	0.338709677419355	0\\
2	2.7	1	0.330645161290323	0\\
2	2.7	1	0.32258064516129	0\\
2	2.7	1	0.314516129032258	0\\
2	2.7	1	0.306451612903226	0\\
2	2.7	1	0.298387096774194	0\\
2	2.7	1	0.290322580645161	0\\
2	2.7	1	0.282258064516129	0\\
2	2.7	1	0.274193548387097	0\\
2	2.7	1	0.266129032258065	0\\
2	2.7	1	0.258064516129032	0\\
2	2.7	1	0.25	0\\
2	2.7	1	0.241935483870968	0\\
2	2.7	1	0.233870967741935	0\\
2	2.7	1	0.225806451612903	0\\
2	2.7	1	0.217741935483871	0\\
4	3.5	1	0.209677419354839	0\\
4	3.5	1	0.201612903225806	0\\
4	3.5	1	0.193548387096774	0\\
4	3.5	1	0.185483870967742	0\\
2	2.7	1	0.17741935483871	0\\
2	2.7	1	0.169354838709677	0\\
2	2.7	1	0.161290322580645	0\\
2	2.7	1	0.153225806451613	0\\
2	2.6	1	0.145161290322581	0\\
2	2.6	1	0.137096774193548	0\\
2	2.6	1	0.129032258064516	0\\
2	2.6	1	0.120967741935484	0\\
2	2.6	1	0.112903225806452	0\\
2	2.6	1	0.104838709677419	0\\
2	2.6	1	0.0967741935483871	0\\
2	2.6	1	0.0887096774193548	0\\
2	2.6	1	0.0806451612903226	0\\
2	2.6	1	0.0725806451612903	0\\
2	2.6	1	0.0645161290322581	0\\
2	2.6	1	0.0564516129032258	0\\
2	2.6	1	0.0483870967741935	0\\
2	2.6	1	0.0403225806451613	0\\
2	2.6	1	0.032258064516129	0\\
2	2.6	1	0.0241935483870968	0\\
2	2.6	1	0.0161290322580645	0\\
2	2.6	1	0.00806451612903226	0\\
2	2.6	1	0	0\\
2	2.6	0.991935483870968	0	0\\
2	2.6	0.983870967741935	0	0\\
2	2.6	0.975806451612903	0	0\\
2	2.6	0.967741935483871	0	0\\
2	2.6	0.959677419354839	0	0\\
2	2.6	0.951612903225806	0	0\\
2	2.6	0.943548387096774	0	0\\
2	2.6	0.935483870967742	0	0\\
2	2.6	0.92741935483871	0	0\\
2.1	2.5	0.919354838709677	0	0\\
2.1	2.5	0.911290322580645	0	0\\
2.1	2.5	0.903225806451613	0	0\\
2.1	2.5	0.895161290322581	0	0\\
2.1	2.5	0.887096774193548	0	0\\
2.1	2.5	0.879032258064516	0	0\\
2.1	2.5	0.870967741935484	0	0\\
2.1	2.4	0.862903225806452	0	0\\
2.1	2.4	0.854838709677419	0	0\\
2.1	2.4	0.846774193548387	0	0\\
2.1	2.4	0.838709677419355	0	0\\
2.1	2.4	0.830645161290323	0	0\\
2.1	2.4	0.82258064516129	0	0\\
2.1	2.4	0.814516129032258	0	0\\
2.1	2.4	0.806451612903226	0	0\\
2.1	2.4	0.798387096774194	0	0\\
2.1	2.4	0.790322580645161	0	0\\
2.1	2.4	0.782258064516129	0	0\\
2.1	2.4	0.774193548387097	0	0\\
2.1	2.4	0.766129032258065	0	0\\
2.1	2.4	0.758064516129032	0	0\\
2.1	2.4	0.75	0	0\\
2.1	2.4	0.741935483870968	0	0\\
2.1	2.4	0.733870967741935	0	0\\
2.1	2.4	0.725806451612903	0	0\\
2.1	2.4	0.717741935483871	0	0\\
2.1	2.4	0.709677419354839	0	0\\
2.1	2.4	0.701612903225806	0	0\\
2.1	2.4	0.693548387096774	0	0\\
2.1	2.4	0.685483870967742	0	0\\
2.1	2.4	0.67741935483871	0	0\\
2.1	2.4	0.669354838709677	0	0\\
2.1	2.4	0.661290322580645	0	0\\
2.1	2.4	0.653225806451613	0	0\\
2.1	2.4	0.645161290322581	0	0\\
4	3.6	0.637096774193548	0	0\\
4	3.6	0.629032258064516	0	0\\
4	3.6	0.620967741935484	0	0\\
4	3.6	0.612903225806452	0	0\\
4	3.6	0.604838709677419	0	0\\
4	3.6	0.596774193548387	0	0\\
4	3.6	0.588709677419355	0	0\\
4	3.6	0.580645161290323	0	0\\
2	2.5	0.57258064516129	0	0\\
2	2.5	0.564516129032258	0	0\\
2	2.5	0.556451612903226	0	0\\
2	2.5	0.548387096774194	0	0\\
2	2.5	0.540322580645161	0	0\\
2	2.5	0.532258064516129	0	0\\
2	2.5	0.524193548387097	0	0\\
2	2.5	0.516129032258065	0	0\\
2	2.5	0.508064516129032	0	0\\
2	2.5	0.5	0	0\\
};
\addplot[
    scatter,%
    scatter/@pre marker code/.code={%
        \edef\temp{\noexpand\definecolor{mapped color}{rgb}{\pgfplotspointmeta}}%
        \temp
        \scope[draw=mapped color!80!black,fill=mapped color]%
    },%
    scatter/@post marker code/.code={%
        \endscope
    },%
    only marks,     
    mark=x,
    mark size=4.0pt,
    line width=0.5pt,
    point meta={TeX code symbolic={%
        \edef\pgfplotspointmeta{\thisrow{R},\thisrow{G},\thisrow{B}}%
    }},
] 
table[row sep=crcr]{%
x	y	R	G	B\\
2.6	4	0	0	0.508064516129032\\
4.6	2.3	0	0	0.516129032258065\\
4.3	2.2	0	0	0.524193548387097\\
4.5	2.3	0	0	0.532258064516129\\
4.6	2.2	0	0	0.540322580645161\\
4.5	2.2	0	0	0.548387096774194\\
4.6	2.2	0	0	0.556451612903226\\
4.6	2.2	0	0	0.564516129032258\\
4.6	2.2	0	0	0.57258064516129\\
4.5	2.3	0	0	0.580645161290323\\
4.5	2.3	0	0	0.588709677419355\\
4.5	2.3	0	0	0.596774193548387\\
4.5	2.3	0	0	0.604838709677419\\
4.5	2.3	0	0	0.612903225806452\\
4.5	2.3	0	0	0.620967741935484\\
2.1	4	0	0	0.629032258064516\\
2.1	4	0	0	0.637096774193548\\
2.1	4	0	0	0.645161290322581\\
4	2.1	0	0	0.653225806451613\\
4	2.1	0	0	0.661290322580645\\
4	2.1	0	0	0.669354838709677\\
4	2.1	0	0	0.67741935483871\\
4	2.1	0	0	0.685483870967742\\
4	2.1	0	0	0.693548387096774\\
4	2.1	0	0	0.701612903225806\\
4	2.1	0	0	0.709677419354839\\
4	2.1	0	0	0.717741935483871\\
4	2.1	0	0	0.725806451612903\\
4	2.1	0	0	0.733870967741935\\
4	2.1	0	0	0.741935483870968\\
4	2.1	0	0	0.75\\
4	2.1	0	0	0.758064516129032\\
4	2.1	0	0	0.766129032258065\\
4	2.1	0	0	0.774193548387097\\
4	2.1	0	0	0.782258064516129\\
4	2.1	0	0	0.790322580645161\\
1.6	3.7	0	0	0.798387096774194\\
1.5	3.8	0	0	0.806451612903226\\
4.7	2.4	0	0	0.814516129032258\\
4.7	2.4	0	0	0.82258064516129\\
4.7	2.3	0	0	0.830645161290323\\
4.7	2.3	0	0	0.838709677419355\\
4.7	2.3	0	0	0.846774193548387\\
4.7	2.3	0	0	0.854838709677419\\
2.7	3.9	0	0	0.862903225806452\\
1.2	3.8	0	0	0.870967741935484\\
1.2	3.8	0	0	0.879032258064516\\
1.2	3.8	0	0	0.887096774193548\\
1.6	3.8	0	0	0.895161290322581\\
1.7	3.8	0	0	0.903225806451613\\
1.6	3.8	0	0	0.911290322580645\\
1.6	3.8	0	0	0.919354838709677\\
1.2	2.2	0	0	0.92741935483871\\
1.6	3.8	0	0	0.935483870967742\\
1.2	2.2	0	0	0.943548387096774\\
1.2	2.2	0	0	0.951612903225806\\
2.2	4	0	0	0.959677419354839\\
2.2	4	0	0	0.967741935483871\\
4.7	2.3	0	0	0.975806451612903\\
4.7	2.3	0	0	0.983870967741935\\
2.1	3.9	0	0	0.991935483870968\\
2.1	3.9	0	0	1\\
4.7	2.4	0	0.00806451612903226	1\\
1.2	2.2	0	0.0161290322580645	1\\
1.2	2.2	0	0.0241935483870968	1\\
4.7	2.4	0	0.032258064516129	1\\
4.7	2.3	0	0.0403225806451613	1\\
4.7	2.3	0	0.0483870967741935	1\\
1.2	2.2	0	0.0564516129032258	1\\
1.2	2.2	0	0.0645161290322581	1\\
1.2	2.2	0	0.0725806451612903	1\\
1.2	2.2	0	0.0806451612903226	1\\
2.8	1.2	0	0.0887096774193548	1\\
2.8	1.2	0	0.0967741935483871	1\\
2.8	1.2	0	0.104838709677419	1\\
2.8	1.2	0	0.112903225806452	1\\
4.7	2.3	0	0.120967741935484	1\\
2.8	1.2	0	0.129032258064516	1\\
2.8	1.2	0	0.137096774193548	1\\
2.8	1.2	0	0.145161290322581	1\\
2.8	1.2	0	0.153225806451613	1\\
4.7	2.3	0	0.161290322580645	1\\
1.2	3.8	0	0.169354838709677	1\\
1.2	3.8	0	0.17741935483871	1\\
1.2	3.8	0	0.185483870967742	1\\
1.2	3.8	0	0.193548387096774	1\\
1.2	3.8	0	0.201612903225806	1\\
1.2	3.8	0	0.209677419354839	1\\
1.2	3.8	0	0.217741935483871	1\\
1.2	3.8	0	0.225806451612903	1\\
1.2	3.8	0	0.233870967741935	1\\
1.2	3.8	0	0.241935483870968	1\\
1.2	3.1	0	0.25	1\\
1.2	3.8	0	0.258064516129032	1\\
1.2	3.8	0	0.266129032258065	1\\
1.2	3.8	0	0.274193548387097	1\\
1.2	3.8	0	0.282258064516129	1\\
1.2	3.8	0	0.290322580645161	1\\
1.2	3.8	0	0.298387096774194	1\\
1.2	3.8	0	0.306451612903226	1\\
1.2	3.8	0	0.314516129032258	1\\
1.2	3.8	0	0.32258064516129	1\\
1.2	3.8	0	0.330645161290323	1\\
1.2	3.1	0	0.338709677419355	1\\
1.2	2.2	0	0.346774193548387	1\\
1.2	2.2	0	0.354838709677419	1\\
1.2	2.2	0	0.362903225806452	1\\
2.8	1.2	0	0.370967741935484	1\\
2.8	1.2	0	0.379032258064516	1\\
1.2	2.2	0	0.387096774193548	1\\
2.8	1.2	0	0.395161290322581	1\\
4.7	2.3	0	0.403225806451613	1\\
4.7	2.3	0	0.411290322580645	1\\
4.7	2.3	0	0.419354838709677	1\\
4.7	2.3	0	0.42741935483871	1\\
3.8	1.2	0	0.435483870967742	1\\
3.8	1.2	0	0.443548387096774	1\\
3.8	1.2	0	0.451612903225806	1\\
3.8	1.2	0	0.459677419354839	1\\
1.2	3	0	0.467741935483871	1\\
1.2	3	0	0.475806451612903	1\\
1.2	3	0	0.483870967741935	1\\
2	3.6	0	0.491935483870968	1\\
2	3.6	0	0.5	1\\
4.7	2.3	0	0.508064516129032	1\\
4.7	2.3	0	0.516129032258065	1\\
1.2	3	0	0.524193548387097	1\\
1.2	3	0	0.532258064516129	1\\
1.2	3	0	0.540322580645161	1\\
1.2	3	0	0.548387096774194	1\\
1.2	3	0	0.556451612903226	1\\
1.2	3	0	0.564516129032258	1\\
1.2	3	0	0.57258064516129	1\\
1.2	3	0	0.580645161290323	1\\
1.2	3	0	0.588709677419355	1\\
1.2	3	0	0.596774193548387	1\\
1.2	3	0	0.604838709677419	1\\
1.2	3	0	0.612903225806452	1\\
1.2	3	0	0.620967741935484	1\\
1.2	3	0	0.629032258064516	1\\
4.7	2.3	0	0.637096774193548	1\\
1.2	2.2	0	0.645161290322581	1\\
1.2	2.2	0	0.653225806451613	1\\
1.2	2.2	0	0.661290322580645	1\\
1.2	3.8	0	0.669354838709677	1\\
4.7	2.3	0	0.67741935483871	1\\
1.2	2.2	0	0.685483870967742	1\\
4.7	3.9	0	0.693548387096774	1\\
2	3.6	0	0.701612903225806	1\\
1.2	3.8	0	0.709677419354839	1\\
1.2	3.8	0	0.717741935483871	1\\
1.2	3.8	0	0.725806451612903	1\\
1.2	3.8	0	0.733870967741935	1\\
1.2	3.8	0	0.741935483870968	1\\
1.2	3.8	0	0.75	1\\
1.2	3.8	0	0.758064516129032	1\\
1.2	3.8	0	0.766129032258065	1\\
1.2	3.8	0	0.774193548387097	1\\
1.2	3.8	0	0.782258064516129	1\\
1.2	3.8	0	0.790322580645161	1\\
2.4	3.2	0	0.798387096774194	1\\
4.1	2.6	0	0.806451612903226	1\\
4	2.6	0	0.814516129032258	1\\
4	2.6	0	0.82258064516129	1\\
4	2.6	0	0.830645161290323	1\\
4	2.6	0	0.838709677419355	1\\
4	2.5	0	0.846774193548387	1\\
4	2.5	0	0.854838709677419	1\\
1.2	2.2	0	0.862903225806452	1\\
3.9	2.5	0	0.870967741935484	1\\
3.9	2.5	0	0.879032258064516	1\\
1.2	2.2	0	0.887096774193548	1\\
1.2	3.8	0	0.895161290322581	1\\
1.2	3.8	0	0.903225806451613	1\\
4	2.6	0	0.911290322580645	1\\
4	2.6	0	0.919354838709677	1\\
2.8	1.2	0	0.92741935483871	1\\
2.8	1.2	0	0.935483870967742	1\\
2.8	1.2	0	0.943548387096774	1\\
2.8	1.2	0	0.951612903225806	1\\
4	2.6	0	0.959677419354839	1\\
4	2.6	0	0.967741935483871	1\\
4	2.6	0	0.975806451612903	1\\
4	2.6	0	0.983870967741935	1\\
2.8	1.2	0	0.991935483870968	1\\
2.8	1.2	0	1	1\\
2.8	1.2	0.00806451612903226	1	0.991935483870968\\
1.2	2.2	0.0161290322580645	1	0.983870967741935\\
1.2	2.2	0.0241935483870968	1	0.975806451612903\\
1.2	2.2	0.032258064516129	1	0.967741935483871\\
1.2	2.2	0.0403225806451613	1	0.959677419354839\\
1.2	3.1	0.0483870967741935	1	0.951612903225806\\
2.8	1.2	0.0564516129032258	1	0.943548387096774\\
1.2	2.2	0.0645161290322581	1	0.935483870967742\\
1.2	2.2	0.0725806451612903	1	0.92741935483871\\
2.8	1.2	0.0806451612903226	1	0.919354838709677\\
1.2	2.2	0.0887096774193548	1	0.911290322580645\\
1.2	2.2	0.0967741935483871	1	0.903225806451613\\
1.2	2.2	0.104838709677419	1	0.895161290322581\\
1.2	2.2	0.112903225806452	1	0.887096774193548\\
1.2	2.2	0.120967741935484	1	0.879032258064516\\
1.2	2.2	0.129032258064516	1	0.870967741935484\\
1.2	2.2	0.137096774193548	1	0.862903225806452\\
1.2	2.2	0.145161290322581	1	0.854838709677419\\
1.2	2.2	0.153225806451613	1	0.846774193548387\\
1.2	2.2	0.161290322580645	1	0.838709677419355\\
1.2	2.2	0.169354838709677	1	0.830645161290323\\
1.2	2.2	0.17741935483871	1	0.82258064516129\\
4	2.7	0.185483870967742	1	0.814516129032258\\
4	2.7	0.193548387096774	1	0.806451612903226\\
4	2.7	0.201612903225806	1	0.798387096774194\\
4	2.7	0.209677419354839	1	0.790322580645161\\
4	2.7	0.217741935483871	1	0.782258064516129\\
4	2.7	0.225806451612903	1	0.774193548387097\\
4	2.7	0.233870967741935	1	0.766129032258065\\
4	2.7	0.241935483870968	1	0.758064516129032\\
4	2.7	0.25	1	0.75\\
3.9	2.8	0.258064516129032	1	0.741935483870968\\
2.1	3.3	0.266129032258065	1	0.733870967741935\\
3.9	2.8	0.274193548387097	1	0.725806451612903\\
3.9	2.8	0.282258064516129	1	0.717741935483871\\
3.9	2.8	0.290322580645161	1	0.709677419354839\\
3.9	2.8	0.298387096774194	1	0.701612903225806\\
3.9	2.8	0.306451612903226	1	0.693548387096774\\
3.9	2.8	0.314516129032258	1	0.685483870967742\\
3.9	2.8	0.32258064516129	1	0.67741935483871\\
3.9	2.8	0.330645161290323	1	0.669354838709677\\
3.9	2.8	0.338709677419355	1	0.661290322580645\\
3.9	2.8	0.346774193548387	1	0.653225806451613\\
3.9	2.8	0.354838709677419	1	0.645161290322581\\
3.9	2.8	0.362903225806452	1	0.637096774193548\\
3.9	2.8	0.370967741935484	1	0.629032258064516\\
3.9	2.8	0.379032258064516	1	0.620967741935484\\
3.9	2.8	0.387096774193548	1	0.612903225806452\\
3.9	2.8	0.395161290322581	1	0.604838709677419\\
3.9	2.8	0.403225806451613	1	0.596774193548387\\
3.9	2.8	0.411290322580645	1	0.588709677419355\\
2.1	3.2	0.419354838709677	1	0.580645161290323\\
2.1	3.2	0.42741935483871	1	0.57258064516129\\
2.1	3.2	0.435483870967742	1	0.564516129032258\\
2.1	3.2	0.443548387096774	1	0.556451612903226\\
2.1	3.2	0.451612903225806	1	0.548387096774194\\
2.1	3.2	0.459677419354839	1	0.540322580645161\\
2.1	3.2	0.467741935483871	1	0.532258064516129\\
2.1	3.2	0.475806451612903	1	0.524193548387097\\
2.1	3.2	0.483870967741935	1	0.516129032258065\\
2.1	3.2	0.491935483870968	1	0.508064516129032\\
2.1	3.2	0.5	1	0.5\\
2.1	3.2	0.508064516129032	1	0.491935483870968\\
2.1	3.2	0.516129032258065	1	0.483870967741935\\
2.1	3.2	0.524193548387097	1	0.475806451612903\\
2.1	3.2	0.532258064516129	1	0.467741935483871\\
2.1	3.2	0.540322580645161	1	0.459677419354839\\
2.1	3.2	0.548387096774194	1	0.451612903225806\\
2.1	3.2	0.556451612903226	1	0.443548387096774\\
2.1	3.2	0.564516129032258	1	0.435483870967742\\
2.1	3.2	0.57258064516129	1	0.42741935483871\\
2.1	3.2	0.580645161290323	1	0.419354838709677\\
2.1	3.2	0.588709677419355	1	0.411290322580645\\
2.1	3.2	0.596774193548387	1	0.403225806451613\\
2.1	3.2	0.604838709677419	1	0.395161290322581\\
2.1	3.2	0.612903225806452	1	0.387096774193548\\
2.1	3.2	0.620967741935484	1	0.379032258064516\\
2.1	3.2	0.629032258064516	1	0.370967741935484\\
2.1	3.2	0.637096774193548	1	0.362903225806452\\
2.1	3.2	0.645161290322581	1	0.354838709677419\\
2.1	3.2	0.653225806451613	1	0.346774193548387\\
2.1	3.2	0.661290322580645	1	0.338709677419355\\
2.1	3.2	0.669354838709677	1	0.330645161290323\\
2.1	3.1	0.67741935483871	1	0.32258064516129\\
2.1	3.1	0.685483870967742	1	0.314516129032258\\
2.1	3	0.693548387096774	1	0.306451612903226\\
3.9	2.9	0.701612903225806	1	0.298387096774194\\
3.9	2.9	0.709677419354839	1	0.290322580645161\\
3.9	2.9	0.717741935483871	1	0.282258064516129\\
3.9	2.9	0.725806451612903	1	0.274193548387097\\
3.9	2.9	0.733870967741935	1	0.266129032258065\\
3.9	2.9	0.741935483870968	1	0.258064516129032\\
3.9	2.9	0.75	1	0.25\\
3.9	2.9	0.758064516129032	1	0.241935483870968\\
3.9	2.9	0.766129032258065	1	0.233870967741935\\
3.9	2.9	0.774193548387097	1	0.225806451612903\\
3.9	2.9	0.782258064516129	1	0.217741935483871\\
3.9	2.9	0.790322580645161	1	0.209677419354839\\
3.9	2.9	0.798387096774194	1	0.201612903225806\\
3.9	2.9	0.806451612903226	1	0.193548387096774\\
3.9	2.9	0.814516129032258	1	0.185483870967742\\
3.9	2.9	0.82258064516129	1	0.17741935483871\\
3.9	2.9	0.830645161290323	1	0.169354838709677\\
3.9	2.9	0.838709677419355	1	0.161290322580645\\
3.9	2.9	0.846774193548387	1	0.153225806451613\\
3.9	2.9	0.854838709677419	1	0.145161290322581\\
3.9	2.9	0.862903225806452	1	0.137096774193548\\
3.9	2.9	0.870967741935484	1	0.129032258064516\\
3.9	2.9	0.879032258064516	1	0.120967741935484\\
1.2	2.3	0.887096774193548	1	0.112903225806452\\
1.2	2.3	0.895161290322581	1	0.104838709677419\\
1.2	2.3	0.903225806451613	1	0.0967741935483871\\
1.2	2.3	0.911290322580645	1	0.0887096774193548\\
1.2	2.3	0.919354838709677	1	0.0806451612903226\\
1.2	2.3	0.92741935483871	1	0.0725806451612903\\
1.2	2.3	0.935483870967742	1	0.0645161290322581\\
1.2	2.3	0.943548387096774	1	0.0564516129032258\\
1.2	2.3	0.951612903225806	1	0.0483870967741935\\
1.2	2.3	0.959677419354839	1	0.0403225806451613\\
1.2	2.3	0.967741935483871	1	0.032258064516129\\
1.2	2.3	0.975806451612903	1	0.0241935483870968\\
1.2	2.3	0.983870967741935	1	0.0161290322580645\\
1.2	2.3	0.991935483870968	1	0.00806451612903226\\
1.2	2.3	1	1	0\\
1.2	2.3	1	0.991935483870968	0\\
1.2	2.3	1	0.983870967741935	0\\
1.2	2.3	1	0.975806451612903	0\\
1.2	2.3	1	0.967741935483871	0\\
1.2	2.3	1	0.959677419354839	0\\
3.9	2.9	1	0.951612903225806	0\\
3.9	2.9	1	0.943548387096774	0\\
3.9	2.9	1	0.935483870967742	0\\
3.8	2.9	1	0.92741935483871	0\\
3.8	2.9	1	0.919354838709677	0\\
3.8	2.9	1	0.911290322580645	0\\
3.8	2.9	1	0.903225806451613	0\\
3.8	2.9	1	0.895161290322581	0\\
3.8	2.9	1	0.887096774193548	0\\
3.8	2.9	1	0.879032258064516	0\\
3.8	2.9	1	0.870967741935484	0\\
3.8	2.9	1	0.862903225806452	0\\
3.8	2.9	1	0.854838709677419	0\\
3.8	2.9	1	0.846774193548387	0\\
3.8	2.9	1	0.838709677419355	0\\
1.2	2.3	1	0.830645161290323	0\\
1.2	2.3	1	0.82258064516129	0\\
1.2	2.3	1	0.814516129032258	0\\
1.2	2.3	1	0.806451612903226	0\\
1.2	2.3	1	0.798387096774194	0\\
1.2	2.3	1	0.790322580645161	0\\
1.2	2.3	1	0.782258064516129	0\\
1.2	2.3	1	0.774193548387097	0\\
1.2	2.3	1	0.766129032258065	0\\
1.2	2.3	1	0.758064516129032	0\\
1.2	2.3	1	0.75	0\\
1.2	2.3	1	0.741935483870968	0\\
1.2	2.3	1	0.733870967741935	0\\
1.2	2.3	1	0.725806451612903	0\\
1.2	2.3	1	0.717741935483871	0\\
1.2	2.3	1	0.709677419354839	0\\
1.2	2.3	1	0.701612903225806	0\\
1.2	2.3	1	0.693548387096774	0\\
1.2	2.3	1	0.685483870967742	0\\
1.2	2.3	1	0.67741935483871	0\\
1.2	2.3	1	0.669354838709677	0\\
1.2	2.3	1	0.661290322580645	0\\
1.2	2.3	1	0.653225806451613	0\\
1.2	2.3	1	0.645161290322581	0\\
1.2	2.3	1	0.637096774193548	0\\
4.1	3.6	1	0.629032258064516	0\\
1.2	2.3	1	0.620967741935484	0\\
4	3.6	1	0.612903225806452	0\\
1.2	2.3	1	0.604838709677419	0\\
1.2	2.3	1	0.596774193548387	0\\
1.2	2.3	1	0.588709677419355	0\\
4	3.6	1	0.580645161290323	0\\
1.2	2.3	1	0.57258064516129	0\\
1.2	2.3	1	0.564516129032258	0\\
1.2	2.3	1	0.556451612903226	0\\
1.2	2.3	1	0.548387096774194	0\\
1.2	2.3	1	0.540322580645161	0\\
1.2	2.3	1	0.532258064516129	0\\
1.2	2.3	1	0.524193548387097	0\\
1.2	2.3	1	0.516129032258065	0\\
1.2	2.3	1	0.508064516129032	0\\
1.2	2.3	1	0.5	0\\
1.2	2.3	1	0.491935483870968	0\\
4	3.6	1	0.483870967741935	0\\
4	3.6	1	0.475806451612903	0\\
3.9	3.5	1	0.467741935483871	0\\
3.9	3.5	1	0.459677419354839	0\\
3.9	3.5	1	0.451612903225806	0\\
4	3.5	1	0.443548387096774	0\\
4	3.5	1	0.435483870967742	0\\
4	3.5	1	0.42741935483871	0\\
4	3.5	1	0.419354838709677	0\\
3.9	3.5	1	0.411290322580645	0\\
4	3.5	1	0.403225806451613	0\\
3.9	3.5	1	0.395161290322581	0\\
4	3.5	1	0.387096774193548	0\\
4	3.5	1	0.379032258064516	0\\
4	3.5	1	0.370967741935484	0\\
4	3.5	1	0.362903225806452	0\\
4	3.5	1	0.354838709677419	0\\
4	3.5	1	0.346774193548387	0\\
4	3.5	1	0.338709677419355	0\\
4	3.5	1	0.330645161290323	0\\
4	3.5	1	0.32258064516129	0\\
4	3.5	1	0.314516129032258	0\\
4	3.5	1	0.306451612903226	0\\
4	3.5	1	0.298387096774194	0\\
4	3.5	1	0.290322580645161	0\\
4	3.5	1	0.282258064516129	0\\
4	3.5	1	0.274193548387097	0\\
4	3.5	1	0.266129032258065	0\\
4	3.5	1	0.258064516129032	0\\
4	3.5	1	0.25	0\\
4	3.5	1	0.241935483870968	0\\
4	3.5	1	0.233870967741935	0\\
4	3.5	1	0.225806451612903	0\\
4	3.5	1	0.217741935483871	0\\
2	2.7	1	0.209677419354839	0\\
2	2.7	1	0.201612903225806	0\\
2	2.7	1	0.193548387096774	0\\
2	2.7	1	0.185483870967742	0\\
4	3.5	1	0.17741935483871	0\\
4	3.5	1	0.169354838709677	0\\
4	3.5	1	0.161290322580645	0\\
4	3.5	1	0.153225806451613	0\\
4	3.5	1	0.145161290322581	0\\
1.2	2.3	1	0.137096774193548	0\\
1.2	2.3	1	0.129032258064516	0\\
1.2	2.3	1	0.120967741935484	0\\
1.2	2.3	1	0.112903225806452	0\\
1.2	2.3	1	0.104838709677419	0\\
1.2	2.3	1	0.0967741935483871	0\\
1.2	2.3	1	0.0887096774193548	0\\
1.2	2.3	1	0.0806451612903226	0\\
1.2	2.3	1	0.0725806451612903	0\\
1.2	2.3	1	0.0645161290322581	0\\
1.2	2.3	1	0.0564516129032258	0\\
1.2	2.3	1	0.0483870967741935	0\\
2.4	2.2	1	0.0403225806451613	0\\
2.4	2.2	1	0.032258064516129	0\\
2.4	2.2	1	0.0241935483870968	0\\
2.4	2.2	1	0.0161290322580645	0\\
2.4	2.2	1	0.00806451612903226	0\\
2.4	2.2	1	0	0\\
2.4	2.2	0.991935483870968	0	0\\
3.7	1.2	0.983870967741935	0	0\\
3.7	1.2	0.975806451612903	0	0\\
2.4	2.2	0.967741935483871	0	0\\
2.3	2.1	0.959677419354839	0	0\\
2.3	2.1	0.951612903225806	0	0\\
3.7	1.2	0.943548387096774	0	0\\
3.7	1.2	0.935483870967742	0	0\\
3.7	1.2	0.92741935483871	0	0\\
3.7	1.2	0.919354838709677	0	0\\
3.7	1.2	0.911290322580645	0	0\\
3.7	1.2	0.903225806451613	0	0\\
3.7	1.2	0.895161290322581	0	0\\
3.7	1.2	0.887096774193548	0	0\\
3.7	1.2	0.879032258064516	0	0\\
3.8	1.2	0.870967741935484	0	0\\
3.7	1.2	0.862903225806452	0	0\\
3.8	1.2	0.854838709677419	0	0\\
3.8	1.2	0.846774193548387	0	0\\
3.8	1.2	0.838709677419355	0	0\\
3.8	1.2	0.830645161290323	0	0\\
3.8	1.2	0.82258064516129	0	0\\
3.8	1.2	0.814516129032258	0	0\\
1.2	2.3	0.806451612903226	0	0\\
1.2	2.3	0.798387096774194	0	0\\
4	3.6	0.790322580645161	0	0\\
4	3.6	0.782258064516129	0	0\\
4	3.6	0.774193548387097	0	0\\
4	3.6	0.766129032258065	0	0\\
4	3.6	0.758064516129032	0	0\\
4	3.6	0.75	0	0\\
4	3.6	0.741935483870968	0	0\\
4	3.6	0.733870967741935	0	0\\
4	3.6	0.725806451612903	0	0\\
4	3.6	0.717741935483871	0	0\\
4	3.6	0.709677419354839	0	0\\
4	3.6	0.701612903225806	0	0\\
4	3.6	0.693548387096774	0	0\\
4	3.6	0.685483870967742	0	0\\
4	3.6	0.67741935483871	0	0\\
4	3.6	0.669354838709677	0	0\\
4	3.6	0.661290322580645	0	0\\
4	3.6	0.653225806451613	0	0\\
4	3.6	0.645161290322581	0	0\\
2.1	2.4	0.637096774193548	0	0\\
2.1	2.4	0.629032258064516	0	0\\
2.1	2.4	0.620967741935484	0	0\\
2.1	2.4	0.612903225806452	0	0\\
2.1	2.4	0.604838709677419	0	0\\
2.1	2.4	0.596774193548387	0	0\\
2	2.5	0.588709677419355	0	0\\
2	2.5	0.580645161290323	0	0\\
4	3.6	0.57258064516129	0	0\\
4	3.6	0.564516129032258	0	0\\
4	3.6	0.556451612903226	0	0\\
4	3.6	0.548387096774194	0	0\\
4	3.6	0.540322580645161	0	0\\
4	3.6	0.532258064516129	0	0\\
4	3.6	0.524193548387097	0	0\\
4	3.6	0.516129032258065	0	0\\
4	3.7	0.508064516129032	0	0\\
4	3.7	0.5	0	0\\
};
\end{axis}
\end{tikzpicture}%
        \caption{Estimated Positions \glsentryshort{trem}}
	\end{subfigure}
	\begin{subfigure}{0.49\textwidth}
		 \centering
        \setlength{\figurewidth}{0.8\textwidth}
        % This file was created by matlab2tikz.
%
\definecolor{lms_red}{rgb}{0.80000,0.20780,0.21960}%
%
\begin{tikzpicture}

\begin{axis}[%
width=0.951\figurewidth,
height=\figureheight,
at={(0\figurewidth,0\figureheight)},
scale only axis,
xmin=1,
xmax=5,
xlabel style={font=\color{white!15!black}},
xlabel={$p_x^{(t)}$~[m]},
ymin=1,
ymax=5,
ylabel style={font=\color{white!15!black}},
ylabel={$p_y^{(t)}$~[m]},
axis background/.style={fill=white},
xmajorgrids,
ymajorgrids,
point meta min = 0,
point meta max = 5,
colorbar horizontal, 
colorbar style={
    at={(0.5,1.03)},anchor=south,
    colormap/jet,
    samples = 25,
    xticklabel pos=upper,
    xtick style={draw=none},
    title style={yshift=0.4cm},
    title=t
},
]

\addplot[
    scatter,%
    scatter/@pre marker code/.code={%
        \edef\temp{\noexpand\definecolor{mapped color}{rgb}{\pgfplotspointmeta}}%
        \temp
        \scope[draw=mapped color!80!black,fill=mapped color]%
    },%
    scatter/@post marker code/.code={%
        \endscope
    },%
    only marks,     
    mark=*,
    mark size=\trajSize,
    opacity=\trajOpacity,
    line width=\trajLinewidth,
    point meta={TeX code symbolic={%
        \edef\pgfplotspointmeta{\thisrow{R},\thisrow{G},\thisrow{B}}%
    }},
] 
table[row sep=crcr]{
x	y	R	G	B\\
4	2	0	0	0.504\\
4	2.1	0	0	0.704\\
4	2.2	0	0	0.904\\
4	2.3	0	0.104	1\\
4	2.4	0	0.304	1\\
4	2.5	0	0.504	1\\
4	2.6	0	0.704	1\\
4	2.7	0	0.904	1\\
4	2.8	0.104	1	0.896\\
4	2.9	0.304	1	0.696\\
4	3.0	0.504	1	0.496\\
4	3.1	0.704	1	0.296\\
4	3.2	0.904	1	0.096\\
4	3.3	1	0.896	0\\
4	3.4	1	0.696	0\\
4	3.5	1	0.496	0\\
4	3.6	1	0.296	0\\
4	3.7	1	0.096	0\\
4	3.8	0.896	0	0\\
4	3.9	0.696	0	0\\
4	4	0.504	0	0\\
2.000000	4.000000	0.000000	0.000000	0.600000\\
2.000000	3.899800	0.000000	0.000000	0.800000\\
2.000000	3.799599	0.000000	0.000000	1.000000\\
2.000000	3.699399	0.000000	0.200000	1.000000\\
2.000000	3.599198	0.000000	0.400000	1.000000\\
2.000000	3.498998	0.000000	0.600000	1.000000\\
2.000000	3.398798	0.000000	0.800000	1.000000\\
2.000000	3.298597	0.000000	1.000000	1.000000\\
2.000000	3.198397	0.200000	1.000000	0.800000\\
2.000000	3.098196	0.400000	1.000000	0.600000\\
2.000000	2.997996	0.600000	1.000000	0.400000\\
2.000000	2.897796	0.800000	1.000000	0.200000\\
2.000000	2.797595	1.000000	1.000000	0.000000\\
2.000000	2.697395	1.000000	0.800000	0.000000\\
2.000000	2.597194	1.000000	0.600000	0.000000\\
2.000000	2.496994	1.000000	0.400000	0.000000\\
2.000000	2.396794	1.000000	0.200000	0.000000\\
2.000000	2.296593	1.000000	0.000000	0.000000\\
2.000000	2.196393	0.800000	0.000000	0.000000\\
2.000000	2.096192	0.600000	0.000000	0.000000\\
};

\node at (axis cs:4,4) [anchor=south west] {$s_1$};
\node at (axis cs:2,4) [anchor=south west] {$s_2$};

\addplot[
    scatter,%
    scatter/@pre marker code/.code={%
        \edef\temp{\noexpand\definecolor{mapped color}{rgb}{\pgfplotspointmeta}}%
        \temp
        \scope[draw=mapped color!80!black,fill=mapped color]%
    },%
    scatter/@post marker code/.code={%
        \endscope
    },%
    only marks,     
    mark=x,
    opacity=\estOpacity,
    mark size=\estSize,
    line width=\estLinewidth,
    point meta={TeX code symbolic={%
        \edef\pgfplotspointmeta{\thisrow{R},\thisrow{G},\thisrow{B}}%
    }},
] 
table[row sep=crcr]{%
x	y	R	G	B\\
1.3	2.3	0	0	0.508064516129032\\
1.3	2.3	0	0	0.516129032258065\\
4.1	2.2	0	0	0.524193548387097\\
4.1	2.2	0	0	0.532258064516129\\
4.1	2.1	0	0	0.540322580645161\\
4.1	2.1	0	0	0.548387096774194\\
4.1	2.1	0	0	0.556451612903226\\
4	2.1	0	0	0.564516129032258\\
4	2.1	0	0	0.57258064516129\\
4	2.1	0	0	0.580645161290323\\
4	2.1	0	0	0.588709677419355\\
4	2.1	0	0	0.596774193548387\\
4	2.1	0	0	0.604838709677419\\
4	2.1	0	0	0.612903225806452\\
4	2.1	0	0	0.620967741935484\\
4	2.1	0	0	0.629032258064516\\
4	2.1	0	0	0.637096774193548\\
4	2.1	0	0	0.645161290322581\\
2.1	4	0	0	0.653225806451613\\
2.1	4	0	0	0.661290322580645\\
2.1	4	0	0	0.669354838709677\\
2.1	3.9	0	0	0.67741935483871\\
2.1	3.9	0	0	0.685483870967742\\
2.1	3.9	0	0	0.693548387096774\\
2.1	3.9	0	0	0.701612903225806\\
2.1	3.9	0	0	0.709677419354839\\
2.1	3.9	0	0	0.717741935483871\\
2.1	3.9	0	0	0.725806451612903\\
4	2.1	0	0	0.733870967741935\\
4	2.1	0	0	0.741935483870968\\
4	2.1	0	0	0.75\\
4	2.1	0	0	0.758064516129032\\
4	2.1	0	0	0.766129032258065\\
4	2.1	0	0	0.774193548387097\\
4	2.1	0	0	0.782258064516129\\
2.1	3.9	0	0	0.790322580645161\\
2.1	3.9	0	0	0.798387096774194\\
2.1	3.9	0	0	0.806451612903226\\
2.1	3.9	0	0	0.814516129032258\\
2.1	3.9	0	0	0.82258064516129\\
2.1	3.9	0	0	0.830645161290323\\
2.1	3.9	0	0	0.838709677419355\\
2.1	3.9	0	0	0.846774193548387\\
2.1	3.9	0	0	0.854838709677419\\
2.1	3.9	0	0	0.862903225806452\\
2.1	3.9	0	0	0.870967741935484\\
2.1	3.9	0	0	0.879032258064516\\
2.1	3.9	0	0	0.887096774193548\\
2.1	3.9	0	0	0.895161290322581\\
2.1	3.9	0	0	0.903225806451613\\
2.1	3.9	0	0	0.911290322580645\\
2.1	3.9	0	0	0.919354838709677\\
2.1	3.9	0	0	0.92741935483871\\
2.1	3.9	0	0	0.935483870967742\\
2.1	3.9	0	0	0.943548387096774\\
2.1	3.9	0	0	0.951612903225806\\
2.1	3.9	0	0	0.959677419354839\\
2.1	3.9	0	0	0.967741935483871\\
2.1	3.9	0	0	0.975806451612903\\
2.1	3.9	0	0	0.983870967741935\\
2.1	3.9	0	0	0.991935483870968\\
2.1	3.9	0	0	1\\
2.1	3.9	0	0.00806451612903226	1\\
2.1	3.9	0	0.0161290322580645	1\\
2.1	3.9	0	0.0241935483870968	1\\
2.1	3.9	0	0.032258064516129	1\\
2.1	3.9	0	0.0403225806451613	1\\
4.1	2.2	0	0.0483870967741935	1\\
4.1	2.2	0	0.0564516129032258	1\\
4.1	2.2	0	0.0645161290322581	1\\
4.1	2.2	0	0.0725806451612903	1\\
4.1	2.2	0	0.0806451612903226	1\\
4.1	2.2	0	0.0887096774193548	1\\
4.1	2.2	0	0.0967741935483871	1\\
4.1	2.2	0	0.104838709677419	1\\
4.1	2.2	0	0.112903225806452	1\\
4.1	2.2	0	0.120967741935484	1\\
4.1	2.2	0	0.129032258064516	1\\
4.1	2.2	0	0.137096774193548	1\\
2.1	3.9	0	0.145161290322581	1\\
2.1	3.9	0	0.153225806451613	1\\
2.1	3.9	0	0.161290322580645	1\\
2.1	3.9	0	0.169354838709677	1\\
2.1	3.9	0	0.17741935483871	1\\
2	3.8	0	0.185483870967742	1\\
2	3.8	0	0.193548387096774	1\\
2	3.8	0	0.201612903225806	1\\
2	3.8	0	0.209677419354839	1\\
2	3.8	0	0.217741935483871	1\\
2	3.8	0	0.225806451612903	1\\
2	3.8	0	0.233870967741935	1\\
2	3.8	0	0.241935483870968	1\\
2	3.8	0	0.25	1\\
2	3.8	0	0.258064516129032	1\\
2	3.8	0	0.266129032258065	1\\
2.1	3.8	0	0.274193548387097	1\\
2.1	3.8	0	0.282258064516129	1\\
2.1	3.8	0	0.290322580645161	1\\
2.1	3.8	0	0.298387096774194	1\\
2.1	3.8	0	0.306451612903226	1\\
2.1	3.8	0	0.314516129032258	1\\
2.1	3.8	0	0.32258064516129	1\\
2.1	3.8	0	0.330645161290323	1\\
2.1	3.8	0	0.338709677419355	1\\
2.1	3.8	0	0.346774193548387	1\\
2.1	3.8	0	0.354838709677419	1\\
2.1	3.8	0	0.362903225806452	1\\
2.1	3.8	0	0.370967741935484	1\\
2.1	3.8	0	0.379032258064516	1\\
2.1	3.8	0	0.387096774193548	1\\
2.1	3.8	0	0.395161290322581	1\\
2.1	3.8	0	0.403225806451613	1\\
2.1	3.8	0	0.411290322580645	1\\
2.1	3.8	0	0.419354838709677	1\\
2.1	3.8	0	0.42741935483871	1\\
2.1	3.8	0	0.435483870967742	1\\
2.1	3.8	0	0.443548387096774	1\\
2.1	3.8	0	0.451612903225806	1\\
2.1	3.8	0	0.459677419354839	1\\
2.1	3.8	0	0.467741935483871	1\\
2.1	3.8	0	0.475806451612903	1\\
2.1	3.8	0	0.483870967741935	1\\
2.1	3.8	0	0.491935483870968	1\\
1.2	2.2	0	0.5	1\\
1.2	2.2	0	0.508064516129032	1\\
1.2	2.2	0	0.516129032258065	1\\
1.2	2.2	0	0.524193548387097	1\\
1.2	2.3	0	0.532258064516129	1\\
1.2	2.2	0	0.540322580645161	1\\
1.2	2.2	0	0.548387096774194	1\\
1.2	2.2	0	0.556451612903226	1\\
1.2	2.2	0	0.564516129032258	1\\
1.2	2.3	0	0.57258064516129	1\\
1.2	2.3	0	0.580645161290323	1\\
1.2	2.3	0	0.588709677419355	1\\
1.2	2.3	0	0.596774193548387	1\\
1.2	2.3	0	0.604838709677419	1\\
1.2	2.3	0	0.612903225806452	1\\
1.2	2.3	0	0.620967741935484	1\\
1.2	2.3	0	0.629032258064516	1\\
1.2	2.3	0	0.637096774193548	1\\
1.2	2.3	0	0.645161290322581	1\\
1.2	2.3	0	0.653225806451613	1\\
4	2.3	0	0.661290322580645	1\\
4	2.3	0	0.669354838709677	1\\
4	2.3	0	0.67741935483871	1\\
1.2	2.3	0	0.685483870967742	1\\
1.2	2.3	0	0.693548387096774	1\\
2.1	3.8	0	0.701612903225806	1\\
2.1	3.8	0	0.709677419354839	1\\
2.1	3.7	0	0.717741935483871	1\\
2.1	3.7	0	0.725806451612903	1\\
2.1	3.7	0	0.733870967741935	1\\
2.1	3.7	0	0.741935483870968	1\\
2.1	3.7	0	0.75	1\\
2	3.6	0	0.758064516129032	1\\
2	3.6	0	0.766129032258065	1\\
2	3.6	0	0.774193548387097	1\\
2	3.6	0	0.782258064516129	1\\
2	3.5	0	0.790322580645161	1\\
2	3.4	0	0.798387096774194	1\\
2	3.4	0	0.806451612903226	1\\
2	3.4	0	0.814516129032258	1\\
2	3.3	0	0.82258064516129	1\\
2	3.3	0	0.830645161290323	1\\
2	3.3	0	0.838709677419355	1\\
2	3.3	0	0.846774193548387	1\\
2	3.3	0	0.854838709677419	1\\
2	3.4	0	0.862903225806452	1\\
2	3.3	0	0.870967741935484	1\\
2	3.3	0	0.879032258064516	1\\
2	3.3	0	0.887096774193548	1\\
2	3.3	0	0.895161290322581	1\\
2	3.3	0	0.903225806451613	1\\
2	3.3	0	0.911290322580645	1\\
2	3.3	0	0.919354838709677	1\\
2	3.3	0	0.92741935483871	1\\
2	3.3	0	0.935483870967742	1\\
2	3.3	0	0.943548387096774	1\\
2	3.3	0	0.951612903225806	1\\
2	3.3	0	0.959677419354839	1\\
2	3.3	0	0.967741935483871	1\\
2	3.3	0	0.975806451612903	1\\
2	3.3	0	0.983870967741935	1\\
2	3.3	0	0.991935483870968	1\\
2	3.3	0	1	1\\
2	3.3	0.00806451612903226	1	0.991935483870968\\
2	3.3	0.0161290322580645	1	0.983870967741935\\
2	3.3	0.0241935483870968	1	0.975806451612903\\
2	3.3	0.032258064516129	1	0.967741935483871\\
2	3.3	0.0403225806451613	1	0.959677419354839\\
2	3.3	0.0483870967741935	1	0.951612903225806\\
2	3.3	0.0564516129032258	1	0.943548387096774\\
2	3.3	0.0645161290322581	1	0.935483870967742\\
2	3.3	0.0725806451612903	1	0.92741935483871\\
2.1	3.3	0.0806451612903226	1	0.919354838709677\\
2.1	3.3	0.0887096774193548	1	0.911290322580645\\
2.1	3.3	0.0967741935483871	1	0.903225806451613\\
2.1	3.3	0.104838709677419	1	0.895161290322581\\
2.1	3.3	0.112903225806452	1	0.887096774193548\\
2.1	3.3	0.120967741935484	1	0.879032258064516\\
2.1	3.3	0.129032258064516	1	0.870967741935484\\
2.1	3.3	0.137096774193548	1	0.862903225806452\\
2.1	3.3	0.145161290322581	1	0.854838709677419\\
2.1	3.3	0.153225806451613	1	0.846774193548387\\
2.1	3.3	0.161290322580645	1	0.838709677419355\\
2	3.3	0.169354838709677	1	0.830645161290323\\
2.1	3.3	0.17741935483871	1	0.82258064516129\\
2.1	3.3	0.185483870967742	1	0.814516129032258\\
2.1	3.3	0.193548387096774	1	0.806451612903226\\
2.1	3.3	0.201612903225806	1	0.798387096774194\\
2	3.3	0.209677419354839	1	0.790322580645161\\
2.1	3.2	0.217741935483871	1	0.782258064516129\\
2.1	3.2	0.225806451612903	1	0.774193548387097\\
2.1	3.2	0.233870967741935	1	0.766129032258065\\
2.1	3.2	0.241935483870968	1	0.758064516129032\\
2.1	3.2	0.25	1	0.75\\
3.9	2.8	0.258064516129032	1	0.741935483870968\\
3.9	2.8	0.266129032258065	1	0.733870967741935\\
2.1	3.2	0.274193548387097	1	0.725806451612903\\
2.1	3.2	0.282258064516129	1	0.717741935483871\\
2.1	3.2	0.290322580645161	1	0.709677419354839\\
2.1	3.2	0.298387096774194	1	0.701612903225806\\
2.1	3.2	0.306451612903226	1	0.693548387096774\\
2.1	3.2	0.314516129032258	1	0.685483870967742\\
2.1	3.2	0.32258064516129	1	0.67741935483871\\
2.1	3.2	0.330645161290323	1	0.669354838709677\\
2.1	3.2	0.338709677419355	1	0.661290322580645\\
2.1	3.2	0.346774193548387	1	0.653225806451613\\
2.1	3.2	0.354838709677419	1	0.645161290322581\\
2.1	3.2	0.362903225806452	1	0.637096774193548\\
2.1	3.2	0.370967741935484	1	0.629032258064516\\
2.1	3.2	0.379032258064516	1	0.620967741935484\\
2.1	3.2	0.387096774193548	1	0.612903225806452\\
2.1	3.2	0.395161290322581	1	0.604838709677419\\
2.1	3.2	0.403225806451613	1	0.596774193548387\\
2.1	3.2	0.411290322580645	1	0.588709677419355\\
2.1	3.2	0.419354838709677	1	0.580645161290323\\
3.9	2.8	0.42741935483871	1	0.57258064516129\\
3.9	2.8	0.435483870967742	1	0.564516129032258\\
3.8	2.9	0.443548387096774	1	0.556451612903226\\
3.8	2.9	0.451612903225806	1	0.548387096774194\\
3.8	2.9	0.459677419354839	1	0.540322580645161\\
3.8	2.9	0.467741935483871	1	0.532258064516129\\
3.8	2.9	0.475806451612903	1	0.524193548387097\\
3.8	2.9	0.483870967741935	1	0.516129032258065\\
3.8	2.9	0.491935483870968	1	0.508064516129032\\
3.8	2.9	0.5	1	0.5\\
3.8	2.9	0.508064516129032	1	0.491935483870968\\
3.8	2.9	0.516129032258065	1	0.483870967741935\\
3.9	2.9	0.524193548387097	1	0.475806451612903\\
3.9	2.9	0.532258064516129	1	0.467741935483871\\
3.9	2.9	0.540322580645161	1	0.459677419354839\\
3.9	2.9	0.548387096774194	1	0.451612903225806\\
3.9	2.9	0.556451612903226	1	0.443548387096774\\
3.9	2.9	0.564516129032258	1	0.435483870967742\\
3.9	2.9	0.57258064516129	1	0.42741935483871\\
3.9	2.9	0.580645161290323	1	0.419354838709677\\
3.9	2.9	0.588709677419355	1	0.411290322580645\\
3.9	2.9	0.596774193548387	1	0.403225806451613\\
3.9	2.9	0.604838709677419	1	0.395161290322581\\
3.9	2.9	0.612903225806452	1	0.387096774193548\\
3.9	2.9	0.620967741935484	1	0.379032258064516\\
3.9	2.9	0.629032258064516	1	0.370967741935484\\
3.9	2.9	0.637096774193548	1	0.362903225806452\\
3.9	2.9	0.645161290322581	1	0.354838709677419\\
3.9	2.9	0.653225806451613	1	0.346774193548387\\
3.9	2.9	0.661290322580645	1	0.338709677419355\\
3.9	2.9	0.669354838709677	1	0.330645161290323\\
3.9	2.9	0.67741935483871	1	0.32258064516129\\
3.9	2.9	0.685483870967742	1	0.314516129032258\\
3.9	2.9	0.693548387096774	1	0.306451612903226\\
2.1	3.1	0.701612903225806	1	0.298387096774194\\
2.1	3.1	0.709677419354839	1	0.290322580645161\\
2.1	3.1	0.717741935483871	1	0.282258064516129\\
2.1	3.1	0.725806451612903	1	0.274193548387097\\
2.1	3.1	0.733870967741935	1	0.266129032258065\\
2.1	3.1	0.741935483870968	1	0.258064516129032\\
2.1	3	0.75	1	0.25\\
2.1	3	0.758064516129032	1	0.241935483870968\\
2.1	3	0.766129032258065	1	0.233870967741935\\
2.1	3	0.774193548387097	1	0.225806451612903\\
2.1	3	0.782258064516129	1	0.217741935483871\\
2.1	3	0.790322580645161	1	0.209677419354839\\
2.1	3	0.798387096774194	1	0.201612903225806\\
2.1	3	0.806451612903226	1	0.193548387096774\\
2.1	3	0.814516129032258	1	0.185483870967742\\
2.1	3	0.82258064516129	1	0.17741935483871\\
2.1	3	0.830645161290323	1	0.169354838709677\\
2.1	3	0.838709677419355	1	0.161290322580645\\
2.1	3	0.846774193548387	1	0.153225806451613\\
2.1	3	0.854838709677419	1	0.145161290322581\\
2.1	3	0.862903225806452	1	0.137096774193548\\
2.1	3	0.870967741935484	1	0.129032258064516\\
2.1	3	0.879032258064516	1	0.120967741935484\\
2.1	3	0.887096774193548	1	0.112903225806452\\
2.1	3	0.895161290322581	1	0.104838709677419\\
2.1	3	0.903225806451613	1	0.0967741935483871\\
2.1	3	0.911290322580645	1	0.0887096774193548\\
2.1	3	0.919354838709677	1	0.0806451612903226\\
2.1	3	0.92741935483871	1	0.0725806451612903\\
2.1	3	0.935483870967742	1	0.0645161290322581\\
2.1	3	0.943548387096774	1	0.0564516129032258\\
2.1	3.1	0.951612903225806	1	0.0483870967741935\\
2.1	3	0.959677419354839	1	0.0403225806451613\\
2.1	3	0.967741935483871	1	0.032258064516129\\
2.1	3	0.975806451612903	1	0.0241935483870968\\
2.1	2.9	0.983870967741935	1	0.0161290322580645\\
2.1	2.9	0.991935483870968	1	0.00806451612903226\\
2.1	2.9	1	1	0\\
2.1	2.9	1	0.991935483870968	0\\
2.1	2.9	1	0.983870967741935	0\\
2.1	2.9	1	0.975806451612903	0\\
2.1	2.9	1	0.967741935483871	0\\
2.1	2.9	1	0.959677419354839	0\\
2.1	2.9	1	0.951612903225806	0\\
2.1	2.9	1	0.943548387096774	0\\
2.1	2.9	1	0.935483870967742	0\\
2.1	2.9	1	0.92741935483871	0\\
2.1	2.8	1	0.919354838709677	0\\
2.1	2.8	1	0.911290322580645	0\\
2.1	2.9	1	0.903225806451613	0\\
2.1	2.9	1	0.895161290322581	0\\
2.1	2.9	1	0.887096774193548	0\\
2.1	2.9	1	0.879032258064516	0\\
2.1	2.9	1	0.870967741935484	0\\
2.1	2.9	1	0.862903225806452	0\\
2.1	2.9	1	0.854838709677419	0\\
2.1	2.8	1	0.846774193548387	0\\
2.1	2.8	1	0.838709677419355	0\\
2.1	2.8	1	0.830645161290323	0\\
2.1	2.8	1	0.82258064516129	0\\
2.1	2.8	1	0.814516129032258	0\\
2.1	2.8	1	0.806451612903226	0\\
2.1	2.8	1	0.798387096774194	0\\
2.1	2.8	1	0.790322580645161	0\\
2.1	2.8	1	0.782258064516129	0\\
2.1	2.8	1	0.774193548387097	0\\
2.1	2.8	1	0.766129032258065	0\\
2.1	2.8	1	0.758064516129032	0\\
2.1	2.8	1	0.75	0\\
2	2.7	1	0.741935483870968	0\\
2.1	2.8	1	0.733870967741935	0\\
2.1	2.8	1	0.725806451612903	0\\
2.1	2.8	1	0.717741935483871	0\\
2.1	2.8	1	0.709677419354839	0\\
2.1	2.8	1	0.701612903225806	0\\
2.1	2.8	1	0.693548387096774	0\\
2.1	2.8	1	0.685483870967742	0\\
2.1	2.8	1	0.67741935483871	0\\
2	2.7	1	0.669354838709677	0\\
2	2.7	1	0.661290322580645	0\\
2	2.7	1	0.653225806451613	0\\
2	2.7	1	0.645161290322581	0\\
2	2.7	1	0.637096774193548	0\\
2	2.7	1	0.629032258064516	0\\
2	2.7	1	0.620967741935484	0\\
2	2.7	1	0.612903225806452	0\\
2	2.7	1	0.604838709677419	0\\
2	2.7	1	0.596774193548387	0\\
2	2.7	1	0.588709677419355	0\\
2	2.7	1	0.580645161290323	0\\
2	2.7	1	0.57258064516129	0\\
2	2.7	1	0.564516129032258	0\\
2	2.7	1	0.556451612903226	0\\
2	2.7	1	0.548387096774194	0\\
2	2.7	1	0.540322580645161	0\\
2	2.7	1	0.532258064516129	0\\
2	2.7	1	0.524193548387097	0\\
2	2.7	1	0.516129032258065	0\\
2	2.7	1	0.508064516129032	0\\
2	2.7	1	0.5	0\\
2	2.7	1	0.491935483870968	0\\
2	2.7	1	0.483870967741935	0\\
2	2.7	1	0.475806451612903	0\\
2	2.7	1	0.467741935483871	0\\
2	2.7	1	0.459677419354839	0\\
2	2.7	1	0.451612903225806	0\\
2	2.7	1	0.443548387096774	0\\
2	2.7	1	0.435483870967742	0\\
2	2.7	1	0.42741935483871	0\\
2	2.7	1	0.419354838709677	0\\
2	2.7	1	0.411290322580645	0\\
2	2.7	1	0.403225806451613	0\\
2	2.7	1	0.395161290322581	0\\
2	2.7	1	0.387096774193548	0\\
2	2.7	1	0.379032258064516	0\\
2	2.7	1	0.370967741935484	0\\
2	2.7	1	0.362903225806452	0\\
2	2.7	1	0.354838709677419	0\\
2	2.7	1	0.346774193548387	0\\
2	2.7	1	0.338709677419355	0\\
2	2.7	1	0.330645161290323	0\\
2	2.7	1	0.32258064516129	0\\
2	2.7	1	0.314516129032258	0\\
2	2.7	1	0.306451612903226	0\\
2	2.7	1	0.298387096774194	0\\
2	2.7	1	0.290322580645161	0\\
2	2.7	1	0.282258064516129	0\\
2	2.7	1	0.274193548387097	0\\
2	2.7	1	0.266129032258065	0\\
2	2.7	1	0.258064516129032	0\\
2	2.7	1	0.25	0\\
2	2.7	1	0.241935483870968	0\\
2	2.7	1	0.233870967741935	0\\
2	2.7	1	0.225806451612903	0\\
2	2.7	1	0.217741935483871	0\\
2	2.7	1	0.209677419354839	0\\
3.9	3.5	1	0.201612903225806	0\\
3.9	3.5	1	0.193548387096774	0\\
3.9	3.5	1	0.185483870967742	0\\
2	2.7	1	0.17741935483871	0\\
2	2.7	1	0.169354838709677	0\\
2	2.7	1	0.161290322580645	0\\
2	2.6	1	0.153225806451613	0\\
2	2.6	1	0.145161290322581	0\\
2	2.6	1	0.137096774193548	0\\
2	2.6	1	0.129032258064516	0\\
2	2.6	1	0.120967741935484	0\\
2	2.6	1	0.112903225806452	0\\
2	2.6	1	0.104838709677419	0\\
2	2.6	1	0.0967741935483871	0\\
2	2.6	1	0.0887096774193548	0\\
2	2.6	1	0.0806451612903226	0\\
2	2.6	1	0.0725806451612903	0\\
2	2.6	1	0.0645161290322581	0\\
2	2.6	1	0.0564516129032258	0\\
2	2.6	1	0.0483870967741935	0\\
2	2.6	1	0.0403225806451613	0\\
2	2.6	1	0.032258064516129	0\\
2	2.6	1	0.0241935483870968	0\\
2	2.6	1	0.0161290322580645	0\\
2	2.6	1	0.00806451612903226	0\\
2	2.6	1	0	0\\
2	2.6	0.991935483870968	0	0\\
2	2.6	0.983870967741935	0	0\\
2	2.6	0.975806451612903	0	0\\
2	2.6	0.967741935483871	0	0\\
2	2.6	0.959677419354839	0	0\\
2	2.6	0.951612903225806	0	0\\
2	2.6	0.943548387096774	0	0\\
2	2.6	0.935483870967742	0	0\\
2	2.6	0.92741935483871	0	0\\
2.1	2.5	0.919354838709677	0	0\\
2.1	2.5	0.911290322580645	0	0\\
2.1	2.5	0.903225806451613	0	0\\
2.1	2.5	0.895161290322581	0	0\\
2.1	2.5	0.887096774193548	0	0\\
2.1	2.5	0.879032258064516	0	0\\
2.1	2.5	0.870967741935484	0	0\\
2.1	2.5	0.862903225806452	0	0\\
2.1	2.5	0.854838709677419	0	0\\
2.1	2.5	0.846774193548387	0	0\\
2.1	2.4	0.838709677419355	0	0\\
2.1	2.4	0.830645161290323	0	0\\
2.1	2.5	0.82258064516129	0	0\\
2.1	2.4	0.814516129032258	0	0\\
2.1	2.4	0.806451612903226	0	0\\
2.1	2.4	0.798387096774194	0	0\\
2.1	2.4	0.790322580645161	0	0\\
2.1	2.4	0.782258064516129	0	0\\
2.1	2.4	0.774193548387097	0	0\\
2.1	2.4	0.766129032258065	0	0\\
2.1	2.4	0.758064516129032	0	0\\
2.1	2.4	0.75	0	0\\
2.1	2.4	0.741935483870968	0	0\\
2.1	2.4	0.733870967741935	0	0\\
2.1	2.4	0.725806451612903	0	0\\
2.1	2.4	0.717741935483871	0	0\\
2.1	2.4	0.709677419354839	0	0\\
2.1	2.4	0.701612903225806	0	0\\
2.1	2.4	0.693548387096774	0	0\\
2.1	2.4	0.685483870967742	0	0\\
2.1	2.4	0.67741935483871	0	0\\
2.1	2.4	0.669354838709677	0	0\\
2.1	2.4	0.661290322580645	0	0\\
2.1	2.4	0.653225806451613	0	0\\
2.1	2.4	0.645161290322581	0	0\\
4.1	3.7	0.637096774193548	0	0\\
4.1	3.7	0.629032258064516	0	0\\
4	3.7	0.620967741935484	0	0\\
4	3.7	0.612903225806452	0	0\\
4	3.7	0.604838709677419	0	0\\
4	3.7	0.596774193548387	0	0\\
4	3.7	0.588709677419355	0	0\\
4	3.7	0.580645161290323	0	0\\
4	3.7	0.57258064516129	0	0\\
2	2.5	0.564516129032258	0	0\\
2	2.5	0.556451612903226	0	0\\
2	2.5	0.548387096774194	0	0\\
2	2.5	0.540322580645161	0	0\\
2	2.5	0.532258064516129	0	0\\
2	2.5	0.524193548387097	0	0\\
2	2.5	0.516129032258065	0	0\\
2	2.5	0.508064516129032	0	0\\
2	2.5	0.5	0	0\\
2.6	4	0	0	0.508064516129032\\
4.1	2.3	0	0	0.516129032258065\\
1.3	2.3	0	0	0.524193548387097\\
4	1.6	0	0	0.532258064516129\\
3.9	1.5	0	0	0.540322580645161\\
3.9	1.6	0	0	0.548387096774194\\
3.9	1.6	0	0	0.556451612903226\\
3.9	1.5	0	0	0.564516129032258\\
3.9	1.5	0	0	0.57258064516129\\
3.9	1.5	0	0	0.580645161290323\\
3.9	1.5	0	0	0.588709677419355\\
3.9	1.5	0	0	0.596774193548387\\
3.9	1.5	0	0	0.604838709677419\\
3.9	1.5	0	0	0.612903225806452\\
2.2	4	0	0	0.620967741935484\\
2.2	4	0	0	0.629032258064516\\
2.1	4	0	0	0.637096774193548\\
2.1	4	0	0	0.645161290322581\\
4	2.1	0	0	0.653225806451613\\
4	2.1	0	0	0.661290322580645\\
4	2.1	0	0	0.669354838709677\\
4	2.1	0	0	0.67741935483871\\
4	2.1	0	0	0.685483870967742\\
4	2.1	0	0	0.693548387096774\\
4	2.1	0	0	0.701612903225806\\
4	2.1	0	0	0.709677419354839\\
4	2.1	0	0	0.717741935483871\\
4	2.1	0	0	0.725806451612903\\
2.1	3.9	0	0	0.733870967741935\\
2.1	3.9	0	0	0.741935483870968\\
2.1	3.9	0	0	0.75\\
2.1	4	0	0	0.758064516129032\\
2.1	4	0	0	0.766129032258065\\
2.1	3.9	0	0	0.774193548387097\\
2.1	3.9	0	0	0.782258064516129\\
4	2.1	0	0	0.790322580645161\\
4	2.1	0	0	0.798387096774194\\
4	2.1	0	0	0.806451612903226\\
4	2.1	0	0	0.814516129032258\\
4	2.1	0	0	0.82258064516129\\
4	2.1	0	0	0.830645161290323\\
4	2.1	0	0	0.838709677419355\\
4	2.1	0	0	0.846774193548387\\
4	2.1	0	0	0.854838709677419\\
4	2.1	0	0	0.862903225806452\\
4	2.1	0	0	0.870967741935484\\
4	2.1	0	0	0.879032258064516\\
4	2.1	0	0	0.887096774193548\\
1.2	2.3	0	0	0.895161290322581\\
1.2	2.3	0	0	0.903225806451613\\
1.2	2.3	0	0	0.911290322580645\\
1.2	2.3	0	0	0.919354838709677\\
1.2	2.2	0	0	0.92741935483871\\
4	2.1	0	0	0.935483870967742\\
4	2.1	0	0	0.943548387096774\\
4.1	2.2	0	0	0.951612903225806\\
4.1	2.2	0	0	0.959677419354839\\
4.1	2.2	0	0	0.967741935483871\\
4.1	2.2	0	0	0.975806451612903\\
4.1	2.2	0	0	0.983870967741935\\
4.1	2.2	0	0	0.991935483870968\\
4.1	2.2	0	0	1\\
4.1	2.2	0	0.00806451612903226	1\\
4.1	2.2	0	0.0161290322580645	1\\
4.1	2.2	0	0.0241935483870968	1\\
4.1	2.2	0	0.032258064516129	1\\
4.1	2.2	0	0.0403225806451613	1\\
2.1	3.9	0	0.0483870967741935	1\\
2.1	3.9	0	0.0564516129032258	1\\
2.1	3.9	0	0.0645161290322581	1\\
2.1	3.9	0	0.0725806451612903	1\\
2.1	3.9	0	0.0806451612903226	1\\
2.8	1.2	0	0.0887096774193548	1\\
2.8	1.2	0	0.0967741935483871	1\\
2.8	1.2	0	0.104838709677419	1\\
2.8	1.2	0	0.112903225806452	1\\
2.8	1.2	0	0.120967741935484	1\\
2.1	3.9	0	0.129032258064516	1\\
2.1	3.9	0	0.137096774193548	1\\
4.1	2.2	0	0.145161290322581	1\\
4.1	2.2	0	0.153225806451613	1\\
4.1	2.2	0	0.161290322580645	1\\
4.1	2.2	0	0.169354838709677	1\\
4.1	2.2	0	0.17741935483871	1\\
2.8	1.2	0	0.185483870967742	1\\
2.8	1.2	0	0.193548387096774	1\\
2.8	1.2	0	0.201612903225806	1\\
2.8	1.2	0	0.209677419354839	1\\
2.8	1.2	0	0.217741935483871	1\\
2.8	1.2	0	0.225806451612903	1\\
1.2	3.1	0	0.233870967741935	1\\
1.2	3.1	0	0.241935483870968	1\\
2.8	1.2	0	0.25	1\\
1.2	3.1	0	0.258064516129032	1\\
2.8	1.2	0	0.266129032258065	1\\
2.8	1.2	0	0.274193548387097	1\\
2.8	1.2	0	0.282258064516129	1\\
2.8	1.2	0	0.290322580645161	1\\
2.8	1.2	0	0.298387096774194	1\\
2.8	1.2	0	0.306451612903226	1\\
2.8	1.2	0	0.314516129032258	1\\
2.8	1.2	0	0.32258064516129	1\\
2.8	1.2	0	0.330645161290323	1\\
2.1	1.2	0	0.338709677419355	1\\
2.1	1.2	0	0.346774193548387	1\\
2.1	1.2	0	0.354838709677419	1\\
2.1	1.2	0	0.362903225806452	1\\
2.1	1.2	0	0.370967741935484	1\\
2.1	1.2	0	0.379032258064516	1\\
2.1	1.2	0	0.387096774193548	1\\
2.1	1.2	0	0.395161290322581	1\\
2.1	1.2	0	0.403225806451613	1\\
2.1	1.2	0	0.411290322580645	1\\
2.1	1.2	0	0.419354838709677	1\\
2.1	1.2	0	0.42741935483871	1\\
2.1	1.2	0	0.435483870967742	1\\
2.1	1.2	0	0.443548387096774	1\\
2.1	1.2	0	0.451612903225806	1\\
2.1	1.2	0	0.459677419354839	1\\
2.1	1.2	0	0.467741935483871	1\\
2.8	1.2	0	0.475806451612903	1\\
2.8	1.2	0	0.483870967741935	1\\
1.2	2.2	0	0.491935483870968	1\\
2.1	1.2	0	0.5	1\\
2.1	1.2	0	0.508064516129032	1\\
2.1	3.8	0	0.516129032258065	1\\
2.1	3.8	0	0.524193548387097	1\\
2.1	1.2	0	0.532258064516129	1\\
2.1	3.8	0	0.540322580645161	1\\
2.1	3.8	0	0.548387096774194	1\\
2.1	1.2	0	0.556451612903226	1\\
2.1	1.2	0	0.564516129032258	1\\
2.1	1.2	0	0.57258064516129	1\\
2.1	1.2	0	0.580645161290323	1\\
2.1	1.2	0	0.588709677419355	1\\
2.1	1.2	0	0.596774193548387	1\\
2.1	1.2	0	0.604838709677419	1\\
2.1	1.2	0	0.612903225806452	1\\
2.1	1.2	0	0.620967741935484	1\\
2.1	1.2	0	0.629032258064516	1\\
4	2.3	0	0.637096774193548	1\\
4	2.3	0	0.645161290322581	1\\
4	2.3	0	0.653225806451613	1\\
1.2	2.3	0	0.661290322580645	1\\
1.2	2.3	0	0.669354838709677	1\\
1.2	2.3	0	0.67741935483871	1\\
4	2.3	0	0.685483870967742	1\\
1.2	3.8	0	0.693548387096774	1\\
1.2	3.8	0	0.701612903225806	1\\
1.2	3.8	0	0.709677419354839	1\\
1.2	3.1	0	0.717741935483871	1\\
1.2	3.1	0	0.725806451612903	1\\
1.2	3.1	0	0.733870967741935	1\\
1.2	3.1	0	0.741935483870968	1\\
2.2	3.2	0	0.75	1\\
1.2	3.1	0	0.758064516129032	1\\
1.2	3.1	0	0.766129032258065	1\\
1.2	3.1	0	0.774193548387097	1\\
1.2	3.1	0	0.782258064516129	1\\
2.4	3.2	0	0.790322580645161	1\\
1.2	3.1	0	0.798387096774194	1\\
4	2.5	0	0.806451612903226	1\\
4	2.5	0	0.814516129032258	1\\
4	2.5	0	0.82258064516129	1\\
4	2.5	0	0.830645161290323	1\\
4	2.5	0	0.838709677419355	1\\
4	2.5	0	0.846774193548387	1\\
4	2.5	0	0.854838709677419	1\\
1.2	2.3	0	0.862903225806452	1\\
3.9	2.5	0	0.870967741935484	1\\
1.2	2.3	0	0.879032258064516	1\\
1.2	3.1	0	0.887096774193548	1\\
1.2	3.1	0	0.895161290322581	1\\
1.2	3.1	0	0.903225806451613	1\\
4	2.5	0	0.911290322580645	1\\
2.8	1.2	0	0.919354838709677	1\\
2.8	1.2	0	0.92741935483871	1\\
2.8	1.2	0	0.935483870967742	1\\
2.8	1.2	0	0.943548387096774	1\\
2.8	1.2	0	0.951612903225806	1\\
2.8	1.2	0	0.959677419354839	1\\
4	2.6	0	0.967741935483871	1\\
4	2.6	0	0.975806451612903	1\\
2.8	1.2	0	0.983870967741935	1\\
2.8	1.2	0	0.991935483870968	1\\
2.8	1.2	0	1	1\\
2.8	1.2	0.00806451612903226	1	0.991935483870968\\
1.2	3.1	0.0161290322580645	1	0.983870967741935\\
1.2	3.1	0.0241935483870968	1	0.975806451612903\\
1.2	3.1	0.032258064516129	1	0.967741935483871\\
1.2	3.1	0.0403225806451613	1	0.959677419354839\\
1.2	3.1	0.0483870967741935	1	0.951612903225806\\
1.2	3.1	0.0564516129032258	1	0.943548387096774\\
1.2	3.1	0.0645161290322581	1	0.935483870967742\\
1.2	3.1	0.0725806451612903	1	0.92741935483871\\
1.2	3.1	0.0806451612903226	1	0.919354838709677\\
1.2	3.1	0.0887096774193548	1	0.911290322580645\\
1.2	3.1	0.0967741935483871	1	0.903225806451613\\
1.2	3.1	0.104838709677419	1	0.895161290322581\\
1.2	3.1	0.112903225806452	1	0.887096774193548\\
1.2	3.1	0.120967741935484	1	0.879032258064516\\
1.2	3.1	0.129032258064516	1	0.870967741935484\\
1.2	3.1	0.137096774193548	1	0.862903225806452\\
1.2	3.1	0.145161290322581	1	0.854838709677419\\
1.2	3.1	0.153225806451613	1	0.846774193548387\\
1.2	3.1	0.161290322580645	1	0.838709677419355\\
1.2	3.1	0.169354838709677	1	0.830645161290323\\
1.2	3.1	0.17741935483871	1	0.82258064516129\\
3.9	2.8	0.185483870967742	1	0.814516129032258\\
3.9	2.8	0.193548387096774	1	0.806451612903226\\
3.9	2.8	0.201612903225806	1	0.798387096774194\\
3.9	2.8	0.209677419354839	1	0.790322580645161\\
3.9	2.8	0.217741935483871	1	0.782258064516129\\
3.9	2.8	0.225806451612903	1	0.774193548387097\\
3.9	2.8	0.233870967741935	1	0.766129032258065\\
3.9	2.8	0.241935483870968	1	0.758064516129032\\
3.9	2.8	0.25	1	0.75\\
2.1	3.2	0.258064516129032	1	0.741935483870968\\
2.1	3.2	0.266129032258065	1	0.733870967741935\\
3.9	2.8	0.274193548387097	1	0.725806451612903\\
3.9	2.8	0.282258064516129	1	0.717741935483871\\
3.9	2.8	0.290322580645161	1	0.709677419354839\\
3.9	2.8	0.298387096774194	1	0.701612903225806\\
3.9	2.8	0.306451612903226	1	0.693548387096774\\
3.9	2.8	0.314516129032258	1	0.685483870967742\\
3.9	2.8	0.32258064516129	1	0.67741935483871\\
3.9	2.8	0.330645161290323	1	0.669354838709677\\
3.9	2.8	0.338709677419355	1	0.661290322580645\\
3.9	2.8	0.346774193548387	1	0.653225806451613\\
1.2	2.3	0.354838709677419	1	0.645161290322581\\
1.2	2.3	0.362903225806452	1	0.637096774193548\\
3.9	2.8	0.370967741935484	1	0.629032258064516\\
3.9	2.8	0.379032258064516	1	0.620967741935484\\
3.9	2.8	0.387096774193548	1	0.612903225806452\\
3.9	2.8	0.395161290322581	1	0.604838709677419\\
3.9	2.8	0.403225806451613	1	0.596774193548387\\
3.9	2.8	0.411290322580645	1	0.588709677419355\\
3.9	2.8	0.419354838709677	1	0.580645161290323\\
2.1	3.2	0.42741935483871	1	0.57258064516129\\
2.1	3.2	0.435483870967742	1	0.564516129032258\\
2.1	3.2	0.443548387096774	1	0.556451612903226\\
2.1	3.2	0.451612903225806	1	0.548387096774194\\
2.1	3.2	0.459677419354839	1	0.540322580645161\\
2.1	3.2	0.467741935483871	1	0.532258064516129\\
2.1	3.2	0.475806451612903	1	0.524193548387097\\
2.1	3.2	0.483870967741935	1	0.516129032258065\\
2.1	3.2	0.491935483870968	1	0.508064516129032\\
2.1	3.2	0.5	1	0.5\\
2.1	3.2	0.508064516129032	1	0.491935483870968\\
2.1	3.2	0.516129032258065	1	0.483870967741935\\
2.1	3.2	0.524193548387097	1	0.475806451612903\\
2.1	3.2	0.532258064516129	1	0.467741935483871\\
2.1	3.2	0.540322580645161	1	0.459677419354839\\
2.1	3.2	0.548387096774194	1	0.451612903225806\\
2.1	3.2	0.556451612903226	1	0.443548387096774\\
2.1	3.2	0.564516129032258	1	0.435483870967742\\
2.1	3.2	0.57258064516129	1	0.42741935483871\\
2.1	3.2	0.580645161290323	1	0.419354838709677\\
2.1	3.2	0.588709677419355	1	0.411290322580645\\
2.1	3.2	0.596774193548387	1	0.403225806451613\\
2.1	3.2	0.604838709677419	1	0.395161290322581\\
2.1	3.2	0.612903225806452	1	0.387096774193548\\
2.1	3.2	0.620967741935484	1	0.379032258064516\\
2.1	3.2	0.629032258064516	1	0.370967741935484\\
2.1	3.2	0.637096774193548	1	0.362903225806452\\
2.1	3.2	0.645161290322581	1	0.354838709677419\\
2.1	3.2	0.653225806451613	1	0.346774193548387\\
2.1	3.2	0.661290322580645	1	0.338709677419355\\
2.1	3.2	0.669354838709677	1	0.330645161290323\\
2.1	3.1	0.67741935483871	1	0.32258064516129\\
2.1	3.1	0.685483870967742	1	0.314516129032258\\
2.1	3.1	0.693548387096774	1	0.306451612903226\\
3.9	2.9	0.701612903225806	1	0.298387096774194\\
3.9	2.9	0.709677419354839	1	0.290322580645161\\
3.9	2.9	0.717741935483871	1	0.282258064516129\\
3.9	2.9	0.725806451612903	1	0.274193548387097\\
3.9	2.9	0.733870967741935	1	0.266129032258065\\
3.9	2.9	0.741935483870968	1	0.258064516129032\\
3.9	2.9	0.75	1	0.25\\
3.9	2.9	0.758064516129032	1	0.241935483870968\\
3.9	2.9	0.766129032258065	1	0.233870967741935\\
3.9	2.9	0.774193548387097	1	0.225806451612903\\
3.9	2.9	0.782258064516129	1	0.217741935483871\\
3.9	2.9	0.790322580645161	1	0.209677419354839\\
3.9	2.9	0.798387096774194	1	0.201612903225806\\
3.9	2.9	0.806451612903226	1	0.193548387096774\\
3.9	2.9	0.814516129032258	1	0.185483870967742\\
3.9	2.9	0.82258064516129	1	0.17741935483871\\
3.9	2.9	0.830645161290323	1	0.169354838709677\\
3.9	2.9	0.838709677419355	1	0.161290322580645\\
3.9	2.9	0.846774193548387	1	0.153225806451613\\
3.9	2.9	0.854838709677419	1	0.145161290322581\\
3.9	2.9	0.862903225806452	1	0.137096774193548\\
3.9	2.9	0.870967741935484	1	0.129032258064516\\
3.9	2.9	0.879032258064516	1	0.120967741935484\\
1.2	2.3	0.887096774193548	1	0.112903225806452\\
1.2	2.3	0.895161290322581	1	0.104838709677419\\
1.2	2.3	0.903225806451613	1	0.0967741935483871\\
1.2	2.3	0.911290322580645	1	0.0887096774193548\\
1.2	2.3	0.919354838709677	1	0.0806451612903226\\
1.2	2.3	0.92741935483871	1	0.0725806451612903\\
1.2	2.3	0.935483870967742	1	0.0645161290322581\\
1.2	2.3	0.943548387096774	1	0.0564516129032258\\
1.2	2.3	0.951612903225806	1	0.0483870967741935\\
1.2	2.3	0.959677419354839	1	0.0403225806451613\\
1.2	2.3	0.967741935483871	1	0.032258064516129\\
1.2	2.3	0.975806451612903	1	0.0241935483870968\\
1.2	2.3	0.983870967741935	1	0.0161290322580645\\
1.2	2.3	0.991935483870968	1	0.00806451612903226\\
1.2	2.3	1	1	0\\
1.2	2.3	1	0.991935483870968	0\\
1.2	2.3	1	0.983870967741935	0\\
1.2	2.3	1	0.975806451612903	0\\
1.2	2.3	1	0.967741935483871	0\\
1.2	2.3	1	0.959677419354839	0\\
3.9	2.9	1	0.951612903225806	0\\
3.9	2.9	1	0.943548387096774	0\\
3.9	2.9	1	0.935483870967742	0\\
3.8	2.9	1	0.92741935483871	0\\
3.8	2.9	1	0.919354838709677	0\\
3.8	2.9	1	0.911290322580645	0\\
3.8	2.9	1	0.903225806451613	0\\
3.7	3	1	0.895161290322581	0\\
3.7	3	1	0.887096774193548	0\\
3.7	3	1	0.879032258064516	0\\
3.7	3	1	0.870967741935484	0\\
3.7	3	1	0.862903225806452	0\\
3.7	3	1	0.854838709677419	0\\
3.7	3	1	0.846774193548387	0\\
3.7	3	1	0.838709677419355	0\\
3.7	3	1	0.830645161290323	0\\
1.2	2.3	1	0.82258064516129	0\\
1.2	2.3	1	0.814516129032258	0\\
1.2	2.3	1	0.806451612903226	0\\
1.2	2.3	1	0.798387096774194	0\\
1.2	2.3	1	0.790322580645161	0\\
1.2	2.3	1	0.782258064516129	0\\
1.2	2.3	1	0.774193548387097	0\\
1.2	2.3	1	0.766129032258065	0\\
1.2	2.3	1	0.758064516129032	0\\
1.2	2.3	1	0.75	0\\
1.2	2.3	1	0.741935483870968	0\\
1.2	2.3	1	0.733870967741935	0\\
1.2	2.3	1	0.725806451612903	0\\
1.2	2.3	1	0.717741935483871	0\\
1.2	2.3	1	0.709677419354839	0\\
1.2	2.3	1	0.701612903225806	0\\
1.2	2.3	1	0.693548387096774	0\\
1.2	2.3	1	0.685483870967742	0\\
1.2	2.3	1	0.67741935483871	0\\
1.2	2.3	1	0.669354838709677	0\\
1.2	2.3	1	0.661290322580645	0\\
1.2	2.3	1	0.653225806451613	0\\
1.2	2.3	1	0.645161290322581	0\\
1.2	2.3	1	0.637096774193548	0\\
1.2	2.3	1	0.629032258064516	0\\
1.2	2.3	1	0.620967741935484	0\\
1.2	2.3	1	0.612903225806452	0\\
1.2	2.3	1	0.604838709677419	0\\
1.2	2.3	1	0.596774193548387	0\\
1.2	2.3	1	0.588709677419355	0\\
1.2	2.3	1	0.580645161290323	0\\
1.2	2.3	1	0.57258064516129	0\\
1.2	2.3	1	0.564516129032258	0\\
1.2	2.3	1	0.556451612903226	0\\
1.2	2.3	1	0.548387096774194	0\\
1.2	2.3	1	0.540322580645161	0\\
1.2	2.3	1	0.532258064516129	0\\
1.2	2.3	1	0.524193548387097	0\\
1.2	2.3	1	0.516129032258065	0\\
1.2	2.3	1	0.508064516129032	0\\
1.2	2.3	1	0.5	0\\
1.2	2.3	1	0.491935483870968	0\\
1.2	2.3	1	0.483870967741935	0\\
1.2	2.3	1	0.475806451612903	0\\
4	3.6	1	0.467741935483871	0\\
3.9	3.5	1	0.459677419354839	0\\
3.9	3.5	1	0.451612903225806	0\\
3.9	3.5	1	0.443548387096774	0\\
3.9	3.5	1	0.435483870967742	0\\
3.9	3.5	1	0.42741935483871	0\\
3.9	3.5	1	0.419354838709677	0\\
3.9	3.5	1	0.411290322580645	0\\
3.9	3.5	1	0.403225806451613	0\\
3.9	3.5	1	0.395161290322581	0\\
3.9	3.5	1	0.387096774193548	0\\
3.9	3.5	1	0.379032258064516	0\\
3.9	3.5	1	0.370967741935484	0\\
3.9	3.5	1	0.362903225806452	0\\
3.9	3.5	1	0.354838709677419	0\\
3.9	3.5	1	0.346774193548387	0\\
3.9	3.5	1	0.338709677419355	0\\
3.9	3.5	1	0.330645161290323	0\\
3.9	3.5	1	0.32258064516129	0\\
3.9	3.5	1	0.314516129032258	0\\
3.9	3.5	1	0.306451612903226	0\\
3.9	3.5	1	0.298387096774194	0\\
3.9	3.5	1	0.290322580645161	0\\
3.9	3.5	1	0.282258064516129	0\\
3.9	3.5	1	0.274193548387097	0\\
3.9	3.5	1	0.266129032258065	0\\
3.9	3.5	1	0.258064516129032	0\\
3.9	3.5	1	0.25	0\\
3.9	3.5	1	0.241935483870968	0\\
3.9	3.5	1	0.233870967741935	0\\
3.9	3.5	1	0.225806451612903	0\\
3.9	3.5	1	0.217741935483871	0\\
3.9	3.5	1	0.209677419354839	0\\
2	2.7	1	0.201612903225806	0\\
2	2.7	1	0.193548387096774	0\\
2	2.7	1	0.185483870967742	0\\
3.9	3.5	1	0.17741935483871	0\\
3.9	3.5	1	0.169354838709677	0\\
3.9	3.5	1	0.161290322580645	0\\
3.9	3.5	1	0.153225806451613	0\\
1.2	2.3	1	0.145161290322581	0\\
1.2	2.3	1	0.137096774193548	0\\
1.2	2.3	1	0.129032258064516	0\\
1.2	2.3	1	0.120967741935484	0\\
1.2	2.3	1	0.112903225806452	0\\
1.2	2.3	1	0.104838709677419	0\\
1.2	2.3	1	0.0967741935483871	0\\
1.2	2.3	1	0.0887096774193548	0\\
1.2	2.3	1	0.0806451612903226	0\\
1.2	2.3	1	0.0725806451612903	0\\
1.2	2.3	1	0.0645161290322581	0\\
1.2	2.3	1	0.0564516129032258	0\\
2.4	2.3	1	0.0483870967741935	0\\
1.2	2.3	1	0.0403225806451613	0\\
3.7	1.2	1	0.032258064516129	0\\
2.4	2.2	1	0.0241935483870968	0\\
2.4	2.2	1	0.0161290322580645	0\\
2.4	2.2	1	0.00806451612903226	0\\
2.4	2.2	1	0	0\\
3.7	1.2	0.991935483870968	0	0\\
3.7	1.2	0.983870967741935	0	0\\
3.7	1.2	0.975806451612903	0	0\\
3.7	1.2	0.967741935483871	0	0\\
3.7	1.2	0.959677419354839	0	0\\
3.7	1.2	0.951612903225806	0	0\\
3.7	1.2	0.943548387096774	0	0\\
3.7	1.2	0.935483870967742	0	0\\
3.7	1.2	0.92741935483871	0	0\\
3.7	1.2	0.919354838709677	0	0\\
3.7	1.2	0.911290322580645	0	0\\
3.7	1.2	0.903225806451613	0	0\\
3.7	1.2	0.895161290322581	0	0\\
3.7	1.2	0.887096774193548	0	0\\
3.7	1.2	0.879032258064516	0	0\\
3.7	1.2	0.870967741935484	0	0\\
3.7	1.2	0.862903225806452	0	0\\
3.7	1.2	0.854838709677419	0	0\\
3.7	1.2	0.846774193548387	0	0\\
3.7	1.2	0.838709677419355	0	0\\
3.7	1.2	0.830645161290323	0	0\\
3.7	1.2	0.82258064516129	0	0\\
3.7	1.2	0.814516129032258	0	0\\
3.7	1.2	0.806451612903226	0	0\\
3.7	1.2	0.798387096774194	0	0\\
3.7	1.2	0.790322580645161	0	0\\
4.1	3.7	0.782258064516129	0	0\\
4.1	3.7	0.774193548387097	0	0\\
4.1	3.7	0.766129032258065	0	0\\
4.1	3.7	0.758064516129032	0	0\\
4.1	3.7	0.75	0	0\\
4.1	3.7	0.741935483870968	0	0\\
4.1	3.7	0.733870967741935	0	0\\
4.1	3.7	0.725806451612903	0	0\\
4.1	3.7	0.717741935483871	0	0\\
4.1	3.7	0.709677419354839	0	0\\
4.1	3.7	0.701612903225806	0	0\\
4.1	3.7	0.693548387096774	0	0\\
4.1	3.7	0.685483870967742	0	0\\
4.1	3.7	0.67741935483871	0	0\\
4.1	3.7	0.669354838709677	0	0\\
4.1	3.7	0.661290322580645	0	0\\
4.1	3.7	0.653225806451613	0	0\\
4.1	3.7	0.645161290322581	0	0\\
2.1	2.4	0.637096774193548	0	0\\
2	2.5	0.629032258064516	0	0\\
2	2.5	0.620967741935484	0	0\\
2	2.5	0.612903225806452	0	0\\
2	2.5	0.604838709677419	0	0\\
2	2.5	0.596774193548387	0	0\\
2	2.5	0.588709677419355	0	0\\
2	2.5	0.580645161290323	0	0\\
2	2.5	0.57258064516129	0	0\\
4	3.7	0.564516129032258	0	0\\
4	3.7	0.556451612903226	0	0\\
4	3.7	0.548387096774194	0	0\\
4	3.7	0.540322580645161	0	0\\
4	3.7	0.532258064516129	0	0\\
4	3.7	0.524193548387097	0	0\\
4.1	3.7	0.516129032258065	0	0\\
4.1	3.7	0.508064516129032	0	0\\
4.1	3.8	0.5	0	0\\
};
\end{axis}
\end{tikzpicture}%
		\caption{Estimated Positions \glsentryshort{crem}}
	\end{subfigure}
}
	\caption[Estimated Positions for CREM and TREM (Parallel)]{Estimated Positions for CREM and TREM (Parallel, \Tsixty$=0.4$~s).}
	\label{fig:trackingParallelRoom}
\end{figure}

\FloatBarrier


%% CROSSING MOVEMENT 
\subsubsection*{Crossing Movement}
In the crossing movement scenario, the first source moves from $\bm p^{(0)}_{s=1}=[2~2]$ up to $\bm p^{(T)}_{s=1}=[4~4]$, while the second source moves from $\bm p^{(0)}_{s=2}=[4~2]$ up to $\bm p^{(T)}_{s=2}=[2~4]$. From a top-down view, both sources move up diagonally and cross paths in $\bm p_s^{(T/2)}=[3~3]$ at $t=2.5$.

\begin{figure}[!htbp]
    \iftoggle{quick}{%
        \includegraphics[width=\textwidth, height=\figureheight]{plots/tracking/crossing/results-T60=0.4-crem-xy}
    }{%
        \begin{subfigure}{0.49\textwidth}
             \centering
            \setlength{\figurewidth}{0.8\textwidth}
            % This file was created by matlab2tikz.
%
\definecolor{lms_red}{rgb}{0.80000,0.20780,0.21960}%
\definecolor{mycolor2}{rgb}{0.80000,0.20784,0.21961}%
\definecolor{mycolor3}{rgb}{0.92900,0.69400,0.12500}%
\definecolor{mycolor4}{rgb}{0.49400,0.18400,0.55600}%
%
\begin{tikzpicture}

\begin{axis}[%
width=0.951\figurewidth,
height=\figureheight,
at={(0\figurewidth,0\figureheight)},
scale only axis,
xmin=0,
xmax=496,
xtick={0,99.2,198.4,297.6,396.8,496},
xticklabels={{0},{1},{2},{3},{4},{5}},
xlabel style={font=\color{white!15!black}},
xlabel={$t$~[s]},
ymin=1,
ymax=5,
ylabel style={font=\color{white!15!black}},
ylabel={$p_x^{(t)}$~[m]},
axis background/.style={fill=white},
xmajorgrids,
ymajorgrids,
legend entries={Est.,
                $s=1$,
                $s=2$},
legend columns=-1,
legend style={%
    at={(1.0,1.0)},
    anchor=south east,
    font=\footnotesize,
    fill opacity=0.0, draw opacity=1, text opacity=1,
    draw=none,
    column sep=0.42cm,
    /tikz/every odd column/.append style={column sep=0.15cm}
},
]
% Estimates
\addlegendimage{color=lms_red, mark=x, only marks, mark options={mark size=4pt, opacity=1, line width=1}}
\addplot [color=mycolor2, draw=none, mark=x, mark options={solid, mycolor2}, forget plot]
  table[row sep=crcr]{%
1	3.8\\
2	3.8\\
3	3.8\\
4	2.2\\
5	2.2\\
6	2.1\\
7	2.1\\
8	2.1\\
9	2.1\\
10	2.1\\
11	2.1\\
12	2.1\\
13	2.1\\
14	2.1\\
15	2.1\\
16	2.1\\
17	3.9\\
18	3.9\\
19	3.9\\
20	3.9\\
21	3.9\\
22	3.9\\
23	3.9\\
24	3.9\\
25	3.9\\
26	3.9\\
27	3.9\\
28	3.9\\
29	3.9\\
30	2.1\\
31	2.1\\
32	2.1\\
33	2.1\\
34	2.1\\
35	2.1\\
36	2.1\\
37	3.9\\
38	3.9\\
39	3.9\\
40	3.9\\
41	3.9\\
42	3.9\\
43	3.9\\
44	3.9\\
45	3.9\\
46	3.9\\
47	3.9\\
48	3.9\\
49	3.9\\
50	3.9\\
51	3.9\\
52	3.9\\
53	3.9\\
54	3.9\\
55	3.9\\
56	3.9\\
57	3.9\\
58	3.9\\
59	3.9\\
60	3.9\\
61	3.9\\
62	3.9\\
63	3.9\\
64	3.9\\
65	3.9\\
66	3.9\\
67	3.9\\
68	3.9\\
69	3.9\\
70	3.9\\
71	3.9\\
72	3.9\\
73	3.9\\
74	3.9\\
75	3.9\\
76	2.1\\
77	2.1\\
78	3.9\\
79	3.9\\
80	3.9\\
81	3.9\\
82	3.9\\
83	3.9\\
84	3.8\\
85	3.8\\
86	3.8\\
87	3.8\\
88	3.8\\
89	3.8\\
90	3.8\\
91	3.8\\
92	3.8\\
93	3.8\\
94	3.8\\
95	3.8\\
96	3.8\\
97	3.8\\
98	3.8\\
99	3.8\\
100	3.8\\
101	3.8\\
102	3.8\\
103	3.8\\
104	3.8\\
105	3.8\\
106	3.8\\
107	3.8\\
108	3.8\\
109	3.8\\
110	3.8\\
111	3.8\\
112	3.8\\
113	3.8\\
114	3.8\\
115	3.8\\
116	3.8\\
117	3.8\\
118	3.8\\
119	3.8\\
120	3.8\\
121	3.8\\
122	3.8\\
123	3.8\\
124	3.8\\
125	3.8\\
126	3.8\\
127	3.8\\
128	3.8\\
129	3.8\\
130	3.8\\
131	3.8\\
132	3.8\\
133	1.2\\
134	3.8\\
135	3.8\\
136	3.8\\
137	3.8\\
138	3.8\\
139	3.8\\
140	1.2\\
141	3.8\\
142	2.2\\
143	2.2\\
144	2.2\\
145	1.2\\
146	1.2\\
147	1.2\\
148	1.2\\
149	1.2\\
150	1.2\\
151	1.2\\
152	3.6\\
153	3.5\\
154	3.5\\
155	3.5\\
156	3.4\\
157	3.4\\
158	3.4\\
159	3.4\\
160	3.4\\
161	3.4\\
162	3.4\\
163	3.4\\
164	3.3\\
165	3.4\\
166	3.3\\
167	3.3\\
168	3.3\\
169	3.3\\
170	3.3\\
171	3.3\\
172	3.3\\
173	3.3\\
174	3.3\\
175	3.3\\
176	3.3\\
177	3.3\\
178	3.3\\
179	3.3\\
180	3.3\\
181	3.3\\
182	3.3\\
183	3.3\\
184	3.3\\
185	3.3\\
186	3.3\\
187	3.3\\
188	3.3\\
189	3.3\\
190	3.3\\
191	3.3\\
192	3.3\\
193	3.3\\
194	3.3\\
195	3.3\\
196	3.3\\
197	3.3\\
198	3.3\\
199	3.3\\
200	3.3\\
201	3.3\\
202	3.3\\
203	3.3\\
204	3.3\\
205	3.3\\
206	3.3\\
207	3.3\\
208	3.3\\
209	3.3\\
210	3.2\\
211	3.2\\
212	3.1\\
213	3.1\\
214	3.1\\
215	3.1\\
216	3.1\\
217	3.1\\
218	3.1\\
219	3.1\\
220	3.1\\
221	3.1\\
222	3.1\\
223	3.1\\
224	3.1\\
225	3.1\\
226	3.1\\
227	3.1\\
228	3.1\\
229	3.1\\
230	3.1\\
231	3.1\\
232	3.1\\
233	3.1\\
234	3.1\\
235	3.1\\
236	3.1\\
237	3.1\\
238	3.1\\
239	3.1\\
240	3.1\\
241	3.1\\
242	3.1\\
243	3.1\\
244	3\\
245	3\\
246	3\\
247	3\\
248	3\\
249	3\\
250	3\\
251	3\\
252	3\\
253	3\\
254	3\\
255	3\\
256	3\\
257	3\\
258	3\\
259	3\\
260	3\\
261	3\\
262	3\\
263	3\\
264	3\\
265	3\\
266	3\\
267	3\\
268	3\\
269	3\\
270	3\\
271	3\\
272	3\\
273	3\\
274	3\\
275	3\\
276	3\\
277	3\\
278	3\\
279	3\\
280	3\\
281	3\\
282	3\\
283	3\\
284	3\\
285	3\\
286	3\\
287	3\\
288	3\\
289	3\\
290	3\\
291	3\\
292	3\\
293	3\\
294	3\\
295	3\\
296	3\\
297	3\\
298	3\\
299	3\\
300	3\\
301	3\\
302	3\\
303	3\\
304	3\\
305	3\\
306	3\\
307	3\\
308	2.9\\
309	2.9\\
310	2.9\\
311	2.9\\
312	2.9\\
313	2.9\\
314	2.9\\
315	2.9\\
316	2.9\\
317	2.9\\
318	2.9\\
319	2.9\\
320	2.9\\
321	2.9\\
322	2.9\\
323	2.9\\
324	2.9\\
325	2.9\\
326	2.9\\
327	2.9\\
328	2.9\\
329	2.9\\
330	2.9\\
331	2.9\\
332	2.9\\
333	2.9\\
334	2.9\\
335	2.9\\
336	2.8\\
337	2.8\\
338	2.8\\
339	2.8\\
340	2.9\\
341	2.9\\
342	2.9\\
343	2.9\\
344	2.9\\
345	2.9\\
346	2.9\\
347	2.9\\
348	2.9\\
349	2.9\\
350	2.9\\
351	2.9\\
352	2.9\\
353	2.9\\
354	2.9\\
355	2.9\\
356	2.9\\
357	2.9\\
358	2.9\\
359	2.9\\
360	2.9\\
361	2.9\\
362	2.9\\
363	2.9\\
364	2.9\\
365	2.9\\
366	2.9\\
367	2.9\\
368	2.6\\
369	2.5\\
370	2.5\\
371	2.5\\
372	2.5\\
373	2.5\\
374	2.5\\
375	2.5\\
376	2.5\\
377	2.5\\
378	2.5\\
379	2.5\\
380	2.5\\
381	2.5\\
382	2.5\\
383	2.5\\
384	2.5\\
385	2.5\\
386	2.5\\
387	2.5\\
388	2.5\\
389	2.5\\
390	2.5\\
391	2.5\\
392	2.5\\
393	2.5\\
394	2.5\\
395	2.5\\
396	2.5\\
397	2.5\\
398	2.5\\
399	2.5\\
400	2.5\\
401	2.5\\
402	2.5\\
403	3\\
404	3\\
405	3\\
406	3\\
407	3\\
408	3\\
409	3\\
410	3\\
411	3\\
412	2.5\\
413	2.5\\
414	2.5\\
415	2.5\\
416	2.5\\
417	2.5\\
418	2.4\\
419	2.4\\
420	2.4\\
421	2.4\\
422	2.4\\
423	2.4\\
424	2.4\\
425	2.4\\
426	2.4\\
427	2.4\\
428	2.4\\
429	2.4\\
430	2.4\\
431	2.4\\
432	2.4\\
433	2.4\\
434	2.4\\
435	2.4\\
436	2.4\\
437	2.4\\
438	2.4\\
439	2.4\\
440	2.4\\
441	2.4\\
442	2.4\\
443	2.4\\
444	2.4\\
445	2.4\\
446	2.4\\
447	2.4\\
448	2.4\\
449	2.4\\
450	2.4\\
451	2.4\\
452	2.4\\
453	2.4\\
454	2.4\\
455	2.4\\
456	2.4\\
457	2.4\\
458	2.4\\
459	2.4\\
460	2.4\\
461	2.4\\
462	2.4\\
463	2.4\\
464	2.4\\
465	2.4\\
466	2.4\\
467	2.4\\
468	2.4\\
469	2.4\\
470	2.4\\
471	2.4\\
472	2.4\\
473	2.4\\
474	2.4\\
475	2.4\\
476	2.4\\
477	3.8\\
478	3.8\\
479	3.8\\
480	3.8\\
481	3.8\\
482	3.8\\
483	3.8\\
484	3.8\\
485	3.8\\
486	3.8\\
487	3.8\\
488	3.8\\
489	3.8\\
490	3.8\\
491	3.8\\
492	3.8\\
493	3.8\\
494	3.8\\
495	3.8\\
496	3.8\\
1	2.3\\
2	2.3\\
3	2.2\\
4	3.8\\
5	1.8\\
6	2.2\\
7	2.2\\
8	1.6\\
9	1.6\\
10	2.2\\
11	2.2\\
12	2.2\\
13	3.9\\
14	3.9\\
15	3.9\\
16	3.9\\
17	2.1\\
18	2.1\\
19	2.1\\
20	2.1\\
21	2.1\\
22	2.1\\
23	2.1\\
24	2.1\\
25	2.1\\
26	2.1\\
27	2.1\\
28	2.1\\
29	2.1\\
30	3.9\\
31	3.9\\
32	3.9\\
33	3.9\\
34	3.9\\
35	3.9\\
36	3.9\\
37	2.1\\
38	2.1\\
39	2.1\\
40	2.1\\
41	2.1\\
42	2.1\\
43	2.1\\
44	2.1\\
45	2.1\\
46	2.1\\
47	2.1\\
48	2.1\\
49	2.1\\
50	4.7\\
51	4.7\\
52	4.7\\
53	2.1\\
54	2.1\\
55	2.1\\
56	2.1\\
57	2.1\\
58	2.1\\
59	2.1\\
60	2.1\\
61	2.1\\
62	2.1\\
63	2.1\\
64	2.1\\
65	2.1\\
66	2.1\\
67	2.1\\
68	2.1\\
69	2.1\\
70	2.1\\
71	2.1\\
72	2.1\\
73	2.1\\
74	2.1\\
75	2.1\\
76	3.9\\
77	3.9\\
78	2.1\\
79	2.1\\
80	2.1\\
81	2.1\\
82	2.1\\
83	2.1\\
84	2.1\\
85	2.1\\
86	2.1\\
87	2.1\\
88	2.1\\
89	2.1\\
90	2.1\\
91	2.1\\
92	2.1\\
93	2.1\\
94	2.1\\
95	2.1\\
96	2.1\\
97	1.2\\
98	1.2\\
99	1.2\\
100	3.2\\
101	3.2\\
102	2.2\\
103	2.2\\
104	3.8\\
105	1.2\\
106	2.2\\
107	2.2\\
108	2.3\\
109	1.2\\
110	1.2\\
111	1.2\\
112	1.2\\
113	1.2\\
114	1.2\\
115	1.2\\
116	1.2\\
117	1.2\\
118	1.2\\
119	1.2\\
120	1.2\\
121	1.2\\
122	1.2\\
123	1.2\\
124	1.2\\
125	1.2\\
126	1.2\\
127	1.2\\
128	1.2\\
129	1.2\\
130	1.2\\
131	1.2\\
132	1.2\\
133	3.8\\
134	1.2\\
135	2.2\\
136	2.2\\
137	1.2\\
138	1.2\\
139	2.2\\
140	2.2\\
141	2.2\\
142	3.8\\
143	1.2\\
144	1.2\\
145	2.2\\
146	2.2\\
147	2.2\\
148	2.2\\
149	2.2\\
150	3.6\\
151	3.6\\
152	1.2\\
153	1.2\\
154	1.2\\
155	1.2\\
156	1.2\\
157	1.2\\
158	1.2\\
159	1.2\\
160	1.2\\
161	1.2\\
162	1.2\\
163	1.2\\
164	1.2\\
165	1.2\\
166	1.2\\
167	1.2\\
168	1.2\\
169	1.2\\
170	1.2\\
171	1.2\\
172	1.2\\
173	1.2\\
174	1.2\\
175	2.7\\
176	2.7\\
177	2.7\\
178	2.7\\
179	2.7\\
180	2.7\\
181	2.7\\
182	2.7\\
183	2.7\\
184	1.2\\
185	1.2\\
186	1.2\\
187	1.2\\
188	1.2\\
189	1.2\\
190	1.2\\
191	1.2\\
192	1.2\\
193	1.2\\
194	1.2\\
195	1.2\\
196	2.7\\
197	2.7\\
198	2.7\\
199	2.7\\
200	2.7\\
201	4.7\\
202	4.7\\
203	4.7\\
204	3.7\\
205	3.7\\
206	3.7\\
207	3.7\\
208	2.7\\
209	2.7\\
210	2.6\\
211	2.6\\
212	2.5\\
213	2.5\\
214	2.5\\
215	2.5\\
216	2.6\\
217	2.6\\
218	2.6\\
219	2.6\\
220	2.6\\
221	2.6\\
222	2.6\\
223	2.6\\
224	2.6\\
225	4.7\\
226	4.7\\
227	4.7\\
228	4.7\\
229	4.7\\
230	4.7\\
231	4.7\\
232	4.7\\
233	4.7\\
234	4.7\\
235	2.6\\
236	2.6\\
237	2.6\\
238	2.6\\
239	4.7\\
240	4.7\\
241	4.7\\
242	4.7\\
243	4.7\\
244	3.4\\
245	3.4\\
246	3.4\\
247	1.2\\
248	1.2\\
249	1.2\\
250	1.2\\
251	1.2\\
252	1.2\\
253	1.2\\
254	1.2\\
255	1.2\\
256	1.2\\
257	1.2\\
258	1.2\\
259	1.2\\
260	1.2\\
261	1.2\\
262	1.2\\
263	1.2\\
264	1.2\\
265	1.2\\
266	1.2\\
267	1.2\\
268	1.2\\
269	1.2\\
270	1.2\\
271	1.2\\
272	1.2\\
273	1.2\\
274	1.2\\
275	1.2\\
276	1.2\\
277	1.2\\
278	1.2\\
279	1.2\\
280	1.2\\
281	1.2\\
282	1.2\\
283	1.2\\
284	1.2\\
285	1.2\\
286	1.2\\
287	1.2\\
288	1.2\\
289	1.2\\
290	1.2\\
291	1.2\\
292	1.2\\
293	1.2\\
294	1.2\\
295	1.2\\
296	1.2\\
297	1.2\\
298	1.2\\
299	1.2\\
300	1.2\\
301	1.2\\
302	1.2\\
303	1.2\\
304	1.2\\
305	1.2\\
306	1.2\\
307	1.2\\
308	1.2\\
309	1.2\\
310	1.2\\
311	1.2\\
312	1.2\\
313	1.2\\
314	1.2\\
315	2.9\\
316	2.9\\
317	2.9\\
318	2.9\\
319	2.3\\
320	2.3\\
321	2.3\\
322	1.2\\
323	1.2\\
324	1.2\\
325	1.2\\
326	1.2\\
327	2.9\\
328	2.9\\
329	2.9\\
330	2.5\\
331	2.5\\
332	2.5\\
333	2.5\\
334	2.5\\
335	2.5\\
336	2.2\\
337	2.2\\
338	2.2\\
339	2.2\\
340	2.5\\
341	2.5\\
342	2.5\\
343	2.4\\
344	2.5\\
345	2.5\\
346	2.5\\
347	2.5\\
348	2.5\\
349	2.5\\
350	2.9\\
351	2.3\\
352	2.3\\
353	2.3\\
354	2.3\\
355	3.6\\
356	3.6\\
357	3.6\\
358	3.6\\
359	3.6\\
360	2.3\\
361	2.3\\
362	2.3\\
363	2.3\\
364	2.3\\
365	2.3\\
366	2.4\\
367	2.4\\
368	3.1\\
369	2.9\\
370	2.9\\
371	2.9\\
372	2.9\\
373	2.9\\
374	2.9\\
375	3.1\\
376	3.1\\
377	3.1\\
378	3.1\\
379	3.1\\
380	3.1\\
381	3.1\\
382	3.1\\
383	3.1\\
384	3.1\\
385	3.1\\
386	3.1\\
387	3\\
388	3\\
389	3\\
390	3\\
391	3\\
392	3\\
393	3.1\\
394	3\\
395	3\\
396	3\\
397	3.1\\
398	3\\
399	3\\
400	3\\
401	3\\
402	3\\
403	2.5\\
404	2.5\\
405	2.5\\
406	2.5\\
407	2.5\\
408	2.5\\
409	2.5\\
410	2.5\\
411	3.7\\
412	3\\
413	3\\
414	2.5\\
415	2.5\\
416	2.4\\
417	2.4\\
418	2.4\\
419	2.4\\
420	2.4\\
421	2.4\\
422	2.4\\
423	2.4\\
424	2.4\\
425	2.4\\
426	2.4\\
427	2.4\\
428	2.4\\
429	2.4\\
430	2.4\\
431	2.4\\
432	2.4\\
433	2.4\\
434	2.4\\
435	2.4\\
436	2.4\\
437	2.4\\
438	2.4\\
439	2.4\\
440	2.4\\
441	2.4\\
442	2.4\\
443	2.4\\
444	2.4\\
445	2.4\\
446	2.4\\
447	2.4\\
448	2.4\\
449	2.4\\
450	2.4\\
451	2.4\\
452	2.4\\
453	2.4\\
454	2.4\\
455	2.4\\
456	2.4\\
457	2.4\\
458	2.4\\
459	3.7\\
460	3.7\\
461	3.8\\
462	3.8\\
463	3.8\\
464	3.8\\
465	3.8\\
466	3.8\\
467	3.8\\
468	3.8\\
469	3.8\\
470	3.8\\
471	3.8\\
472	3.8\\
473	3.8\\
474	3.8\\
475	3.8\\
476	3.8\\
477	2.4\\
478	2.4\\
479	2.4\\
480	2.4\\
481	2.4\\
482	2.4\\
483	2.4\\
484	2.4\\
485	2.4\\
486	3.7\\
487	3.7\\
488	3.7\\
489	3.7\\
490	3.7\\
491	3.7\\
492	3.7\\
493	3.7\\
494	3.7\\
495	3.7\\
496	3.7\\
};
\addplot [color=mycolor3, dashed, line width=\trajDashedLinewidth]
  table[row sep=crcr]{%
1	2\\
496	4\\
};
\addplot [color=mycolor4, dashed, line width=\trajDashedLinewidth]
  table[row sep=crcr]{%
1	4\\
496	2\\
};
\end{axis}
\end{tikzpicture}%  % tikz
    %        \includegraphics[width=\textwidth]{plots/tracking/crossing/results-T60=0.4-crem-x}  %png
            \caption{Estimated x-Axis Positions}
        \end{subfigure}
        \begin{subfigure}{0.49\textwidth}
             \centering
            \setlength{\figurewidth}{0.8\textwidth}
            % This file was created by matlab2tikz.
%
\definecolor{lms_red}{rgb}{0.80000,0.20780,0.21960}%
\definecolor{mycolor2}{rgb}{0.80000,0.20784,0.21961}%
\definecolor{mycolor3}{rgb}{0.92900,0.69400,0.12500}%
\definecolor{mycolor4}{rgb}{0.49400,0.18400,0.55600}%
%
\begin{tikzpicture}

\begin{axis}[%
width=0.951\figurewidth,
height=\figureheight,
at={(0\figurewidth,0\figureheight)},
scale only axis,
xmin=0,
xmax=496,
xtick={0,99.2,198.4,297.6,396.8,496},
xticklabels={{0},{1},{2},{3},{4},{5}},
xlabel style={font=\color{white!15!black}},
xlabel={$t$~[s]},
ymin=1,
ymax=5,
ylabel style={font=\color{white!15!black}},
ylabel={$p_y^{(t)}$~[m]},
axis background/.style={fill=white},
xmajorgrids,
ymajorgrids,
legend entries={Est.,
                $s=1$,
                $s=2$},
legend columns=-1,
legend style={%
    at={(1.0,1.0)},
    anchor=south east,
    font=\footnotesize,
    fill opacity=0.0, draw opacity=1, text opacity=1,
    draw=none,
    column sep=0.42cm,
    /tikz/every odd column/.append style={column sep=0.15cm}
},
]
% Estimates
\addlegendimage{color=lms_red, mark=x, only marks, mark options={mark size=4pt, opacity=1, line width=1}}
\addplot [color=mycolor2, draw=none, mark=x, mark options={solid, mycolor2}, forget plot]
  table[row sep=crcr]{%
1	4.7\\
2	4.7\\
3	4.7\\
4	1.6\\
5	1.7\\
6	1.9\\
7	1.9\\
8	1.9\\
9	1.9\\
10	1.9\\
11	1.9\\
12	1.9\\
13	1.9\\
14	1.9\\
15	1.9\\
16	1.9\\
17	2.1\\
18	2.1\\
19	2.1\\
20	2.1\\
21	2.1\\
22	2.1\\
23	2.1\\
24	2.1\\
25	2.1\\
26	2.1\\
27	2.1\\
28	2.1\\
29	2.1\\
30	1.9\\
31	1.9\\
32	1.9\\
33	1.9\\
34	1.9\\
35	1.9\\
36	1.9\\
37	2.1\\
38	2.1\\
39	2.1\\
40	2.1\\
41	2.1\\
42	2.1\\
43	2.1\\
44	2.1\\
45	2.1\\
46	2.1\\
47	2.1\\
48	2.1\\
49	2.1\\
50	2.1\\
51	2.1\\
52	2.1\\
53	2.1\\
54	2.1\\
55	2.1\\
56	2.1\\
57	2.1\\
58	2.1\\
59	2.1\\
60	2.1\\
61	2.1\\
62	2.1\\
63	2.1\\
64	2.1\\
65	2.1\\
66	2.1\\
67	2.1\\
68	2.1\\
69	2.1\\
70	2.1\\
71	2.1\\
72	2.1\\
73	2.1\\
74	2.1\\
75	2.1\\
76	2\\
77	2\\
78	2.1\\
79	2.2\\
80	2.2\\
81	2.2\\
82	2.2\\
83	2.2\\
84	2.2\\
85	2.2\\
86	2.2\\
87	2.2\\
88	2.2\\
89	2.2\\
90	2.2\\
91	2.2\\
92	2.2\\
93	2.2\\
94	2.2\\
95	2.2\\
96	2.2\\
97	2.2\\
98	2.2\\
99	2.2\\
100	2.2\\
101	2.2\\
102	2.2\\
103	2.2\\
104	2.2\\
105	2.2\\
106	2.2\\
107	2.2\\
108	2.2\\
109	2.2\\
110	2.2\\
111	2.2\\
112	2.2\\
113	2.2\\
114	2.2\\
115	2.2\\
116	2.2\\
117	2.2\\
118	2.2\\
119	2.2\\
120	2.2\\
121	2.2\\
122	2.2\\
123	2.2\\
124	2.2\\
125	2.2\\
126	2.2\\
127	2.2\\
128	2.2\\
129	2.2\\
130	2.2\\
131	2.2\\
132	2.2\\
133	2.2\\
134	2.2\\
135	2.2\\
136	2.2\\
137	2.2\\
138	2.2\\
139	2.2\\
140	2.2\\
141	2.2\\
142	1.2\\
143	1.2\\
144	1.2\\
145	3\\
146	3\\
147	3\\
148	3\\
149	3\\
150	3\\
151	3\\
152	2.4\\
153	2.4\\
154	2.4\\
155	2.4\\
156	2.5\\
157	2.5\\
158	2.5\\
159	2.5\\
160	2.5\\
161	2.5\\
162	2.5\\
163	2.5\\
164	2.5\\
165	2.5\\
166	2.5\\
167	2.5\\
168	2.5\\
169	2.5\\
170	2.5\\
171	2.5\\
172	2.5\\
173	2.5\\
174	2.5\\
175	2.6\\
176	2.6\\
177	2.6\\
178	2.6\\
179	2.6\\
180	2.6\\
181	2.6\\
182	2.6\\
183	2.6\\
184	2.6\\
185	2.6\\
186	2.6\\
187	2.6\\
188	2.6\\
189	2.6\\
190	2.6\\
191	2.6\\
192	2.6\\
193	2.6\\
194	2.6\\
195	2.6\\
196	2.6\\
197	2.6\\
198	2.6\\
199	2.6\\
200	2.6\\
201	2.6\\
202	2.6\\
203	2.6\\
204	2.6\\
205	2.6\\
206	2.6\\
207	2.6\\
208	2.6\\
209	2.6\\
210	2.7\\
211	2.7\\
212	2.7\\
213	2.7\\
214	2.7\\
215	2.7\\
216	2.7\\
217	2.7\\
218	2.7\\
219	2.7\\
220	2.7\\
221	2.7\\
222	2.7\\
223	2.7\\
224	2.7\\
225	2.7\\
226	2.7\\
227	2.7\\
228	2.7\\
229	2.7\\
230	2.7\\
231	2.7\\
232	2.7\\
233	2.7\\
234	2.7\\
235	2.8\\
236	2.8\\
237	2.8\\
238	2.8\\
239	2.8\\
240	2.8\\
241	2.8\\
242	2.8\\
243	2.8\\
244	2.9\\
245	2.9\\
246	2.9\\
247	2.9\\
248	2.9\\
249	2.9\\
250	2.9\\
251	2.9\\
252	2.9\\
253	2.9\\
254	2.9\\
255	2.9\\
256	2.9\\
257	2.9\\
258	2.9\\
259	2.9\\
260	2.9\\
261	3\\
262	3\\
263	3\\
264	3\\
265	3\\
266	3\\
267	2.9\\
268	2.9\\
269	2.9\\
270	3\\
271	3\\
272	3\\
273	3\\
274	3\\
275	3\\
276	3\\
277	3\\
278	3\\
279	3\\
280	3\\
281	3\\
282	3\\
283	3\\
284	3\\
285	3\\
286	3\\
287	3\\
288	3\\
289	3\\
290	3\\
291	3\\
292	3\\
293	3\\
294	3\\
295	3\\
296	3\\
297	3\\
298	3\\
299	3\\
300	3\\
301	3\\
302	3\\
303	3\\
304	3\\
305	3\\
306	3\\
307	3.1\\
308	3.1\\
309	3.1\\
310	3.1\\
311	3.1\\
312	3.1\\
313	3.1\\
314	3.1\\
315	3.1\\
316	3.1\\
317	3.1\\
318	3.1\\
319	3.1\\
320	3.1\\
321	3.1\\
322	3.1\\
323	3.1\\
324	3.1\\
325	3.1\\
326	3.1\\
327	3.1\\
328	3.1\\
329	3.1\\
330	3.1\\
331	3.1\\
332	3.1\\
333	3.1\\
334	3.1\\
335	3.1\\
336	3.2\\
337	3.2\\
338	3.2\\
339	3.2\\
340	3.2\\
341	3.2\\
342	3.1\\
343	3.1\\
344	3.1\\
345	3.1\\
346	3.1\\
347	3.1\\
348	3.1\\
349	3.1\\
350	3.1\\
351	3.1\\
352	3.2\\
353	3.2\\
354	3.2\\
355	3.2\\
356	3.2\\
357	3.2\\
358	3.2\\
359	3.2\\
360	3.2\\
361	3.2\\
362	3.2\\
363	3.2\\
364	3.2\\
365	3.2\\
366	3.2\\
367	3.2\\
368	3.3\\
369	3.4\\
370	3.4\\
371	3.4\\
372	3.4\\
373	3.4\\
374	3.4\\
375	3.4\\
376	3.4\\
377	3.4\\
378	3.4\\
379	3.4\\
380	3.4\\
381	3.4\\
382	3.4\\
383	3.4\\
384	3.4\\
385	3.4\\
386	3.4\\
387	3.4\\
388	3.4\\
389	3.4\\
390	3.4\\
391	3.4\\
392	3.4\\
393	3.4\\
394	3.4\\
395	3.4\\
396	3.4\\
397	3.4\\
398	3.4\\
399	3.4\\
400	3.4\\
401	3.4\\
402	3.4\\
403	3.2\\
404	3.2\\
405	3.2\\
406	3.2\\
407	3.2\\
408	3.2\\
409	3.2\\
410	3.2\\
411	3.2\\
412	3.4\\
413	3.4\\
414	3.4\\
415	3.4\\
416	3.5\\
417	3.5\\
418	3.5\\
419	3.5\\
420	3.5\\
421	3.5\\
422	3.5\\
423	3.5\\
424	3.5\\
425	3.5\\
426	3.5\\
427	3.5\\
428	3.5\\
429	3.5\\
430	3.5\\
431	3.5\\
432	3.5\\
433	3.5\\
434	3.5\\
435	3.5\\
436	3.5\\
437	3.5\\
438	3.5\\
439	3.5\\
440	3.5\\
441	3.5\\
442	3.5\\
443	3.5\\
444	3.5\\
445	3.5\\
446	3.5\\
447	3.5\\
448	3.5\\
449	3.5\\
450	3.5\\
451	3.5\\
452	3.5\\
453	3.5\\
454	3.5\\
455	3.5\\
456	3.5\\
457	3.5\\
458	3.5\\
459	3.5\\
460	3.5\\
461	3.5\\
462	3.5\\
463	3.5\\
464	3.5\\
465	3.5\\
466	3.5\\
467	3.5\\
468	3.5\\
469	3.5\\
470	3.5\\
471	3.5\\
472	3.5\\
473	3.5\\
474	3.5\\
475	3.5\\
476	3.5\\
477	3.9\\
478	3.9\\
479	3.9\\
480	3.9\\
481	3.9\\
482	3.9\\
483	3.9\\
484	3.9\\
485	3.9\\
486	3.9\\
487	3.9\\
488	3.9\\
489	3.9\\
490	3.9\\
491	3.9\\
492	3.9\\
493	4\\
494	3.9\\
495	3.9\\
496	3.9\\
1	1.4\\
2	1.4\\
3	1.5\\
4	4.7\\
5	2.1\\
6	1.3\\
7	1.3\\
8	2.2\\
9	2.2\\
10	1.3\\
11	1.3\\
12	1.3\\
13	2.1\\
14	2.1\\
15	2.1\\
16	2.1\\
17	1.9\\
18	1.9\\
19	1.9\\
20	1.9\\
21	1.9\\
22	1.9\\
23	1.9\\
24	1.9\\
25	1.9\\
26	1.9\\
27	1.9\\
28	1.9\\
29	1.9\\
30	2.1\\
31	2.1\\
32	2.1\\
33	2.1\\
34	2.1\\
35	2.1\\
36	2.1\\
37	1.9\\
38	1.9\\
39	1.9\\
40	1.9\\
41	1.9\\
42	1.9\\
43	1.9\\
44	1.9\\
45	1.9\\
46	1.9\\
47	1.9\\
48	1.9\\
49	1.9\\
50	2.4\\
51	2.4\\
52	2.4\\
53	2\\
54	2\\
55	2\\
56	2\\
57	2\\
58	2\\
59	2\\
60	2\\
61	2\\
62	2\\
63	2\\
64	2\\
65	2\\
66	2\\
67	2\\
68	2\\
69	2\\
70	2\\
71	2\\
72	2\\
73	2\\
74	2\\
75	2\\
76	2.1\\
77	2.1\\
78	2\\
79	2\\
80	2\\
81	2\\
82	2\\
83	2\\
84	2\\
85	2\\
86	2\\
87	2\\
88	2\\
89	2\\
90	2\\
91	2\\
92	2\\
93	2\\
94	2\\
95	2\\
96	2\\
97	3.8\\
98	3.8\\
99	3.8\\
100	2.3\\
101	2.3\\
102	1.2\\
103	1.2\\
104	1.2\\
105	2.2\\
106	1.2\\
107	1.2\\
108	4.7\\
109	3.8\\
110	3.8\\
111	3.8\\
112	3.8\\
113	3.8\\
114	3.8\\
115	3.8\\
116	3.8\\
117	3.8\\
118	3.8\\
119	3.8\\
120	3.8\\
121	3.8\\
122	3.8\\
123	3.8\\
124	3.8\\
125	2.2\\
126	2.2\\
127	2.2\\
128	3.8\\
129	3.8\\
130	3.8\\
131	3.8\\
132	2.2\\
133	2.2\\
134	2.1\\
135	1.2\\
136	1.2\\
137	2.2\\
138	2.2\\
139	1.2\\
140	1.2\\
141	1.2\\
142	2.2\\
143	3\\
144	3\\
145	1.2\\
146	1.2\\
147	1.2\\
148	1.2\\
149	1.2\\
150	2.4\\
151	2.4\\
152	3\\
153	3\\
154	3\\
155	3\\
156	3\\
157	3\\
158	3\\
159	3\\
160	3\\
161	3\\
162	3\\
163	3\\
164	3\\
165	3\\
166	3\\
167	3\\
168	3\\
169	3\\
170	3\\
171	3\\
172	3\\
173	3\\
174	3\\
175	2.6\\
176	2.6\\
177	2.6\\
178	2.6\\
179	2.6\\
180	2.6\\
181	2.6\\
182	2.6\\
183	2.6\\
184	3\\
185	3\\
186	3\\
187	3\\
188	3\\
189	3\\
190	3\\
191	3\\
192	3\\
193	3\\
194	3\\
195	3\\
196	2.6\\
197	2.6\\
198	2.6\\
199	2.6\\
200	2.6\\
201	3.9\\
202	3.9\\
203	3.9\\
204	1.2\\
205	1.2\\
206	1.2\\
207	1.2\\
208	2.6\\
209	2.7\\
210	2.6\\
211	2.6\\
212	2.6\\
213	2.6\\
214	2.6\\
215	2.6\\
216	2.5\\
217	2.5\\
218	2.5\\
219	2.5\\
220	2.5\\
221	2.5\\
222	2.5\\
223	2.5\\
224	2.5\\
225	3.9\\
226	3.9\\
227	3.9\\
228	3.9\\
229	3.9\\
230	3.9\\
231	3.9\\
232	3.9\\
233	3.9\\
234	3.9\\
235	2.6\\
236	2.6\\
237	2.6\\
238	2.6\\
239	3.9\\
240	3.9\\
241	3.9\\
242	3.9\\
243	3.9\\
244	2.5\\
245	2.5\\
246	2.5\\
247	3\\
248	3\\
249	3\\
250	3\\
251	3\\
252	3\\
253	3\\
254	3\\
255	3\\
256	3\\
257	3\\
258	3\\
259	3\\
260	3\\
261	3\\
262	3\\
263	3\\
264	3\\
265	3\\
266	3\\
267	3\\
268	3\\
269	3\\
270	3\\
271	3\\
272	3\\
273	3\\
274	3\\
275	3\\
276	3\\
277	3\\
278	3\\
279	3\\
280	3\\
281	3\\
282	3\\
283	3\\
284	3\\
285	3\\
286	3.8\\
287	3.8\\
288	3.8\\
289	3.8\\
290	3.8\\
291	3.8\\
292	3.8\\
293	3.8\\
294	3.8\\
295	3.8\\
296	3.8\\
297	3.8\\
298	3.8\\
299	3.8\\
300	3.8\\
301	3.8\\
302	3.8\\
303	3.8\\
304	3.8\\
305	3.8\\
306	3.8\\
307	3.8\\
308	3.8\\
309	3.8\\
310	3.8\\
311	3.8\\
312	3.8\\
313	3.8\\
314	3.8\\
315	1.2\\
316	1.2\\
317	1.2\\
318	1.2\\
319	3.3\\
320	3.3\\
321	3.3\\
322	3.8\\
323	3.8\\
324	3.8\\
325	3.8\\
326	3.8\\
327	1.2\\
328	1.2\\
329	1.2\\
330	3.5\\
331	3.5\\
332	3.5\\
333	3.5\\
334	3.5\\
335	3.5\\
336	3.3\\
337	3.3\\
338	3.3\\
339	3.3\\
340	3.5\\
341	3.5\\
342	3.5\\
343	3.3\\
344	3.5\\
345	3.5\\
346	3.5\\
347	3.5\\
348	3.5\\
349	3.5\\
350	1.2\\
351	3.3\\
352	3.3\\
353	3.3\\
354	3.3\\
355	3.7\\
356	3.7\\
357	3.7\\
358	3.7\\
359	3.7\\
360	3.3\\
361	3.3\\
362	3.3\\
363	3.3\\
364	3.3\\
365	3.3\\
366	3.3\\
367	3.3\\
368	3.2\\
369	3\\
370	3\\
371	3\\
372	3\\
373	3\\
374	3\\
375	3.2\\
376	3.2\\
377	3.2\\
378	3.2\\
379	3.2\\
380	3.2\\
381	3.2\\
382	3.2\\
383	3.2\\
384	3.2\\
385	3.2\\
386	3.2\\
387	3.2\\
388	3.2\\
389	3.2\\
390	3.2\\
391	3.2\\
392	3.2\\
393	3.2\\
394	3.2\\
395	3.2\\
396	3.2\\
397	3.2\\
398	3.2\\
399	3.2\\
400	3.2\\
401	3.2\\
402	3.2\\
403	3.4\\
404	3.4\\
405	3.4\\
406	3.4\\
407	3.4\\
408	3.4\\
409	3.4\\
410	3.4\\
411	3.8\\
412	3.2\\
413	3.2\\
414	4\\
415	4\\
416	4\\
417	4\\
418	4\\
419	4\\
420	4.1\\
421	4.1\\
422	4.1\\
423	4.1\\
424	4.1\\
425	4.1\\
426	4.1\\
427	4.1\\
428	4.1\\
429	4.1\\
430	4.1\\
431	4.1\\
432	4.1\\
433	4.1\\
434	4.1\\
435	4.1\\
436	4.1\\
437	4.1\\
438	4.1\\
439	4.1\\
440	4.1\\
441	4.1\\
442	4.1\\
443	4.1\\
444	4.1\\
445	4.1\\
446	4.1\\
447	4.1\\
448	4.1\\
449	4.1\\
450	4.1\\
451	4.1\\
452	4.1\\
453	4.1\\
454	4.1\\
455	4.1\\
456	4.1\\
457	4.1\\
458	4.1\\
459	1.2\\
460	1.2\\
461	3.9\\
462	3.9\\
463	3.9\\
464	3.9\\
465	3.9\\
466	3.9\\
467	3.9\\
468	3.9\\
469	3.9\\
470	3.9\\
471	3.9\\
472	3.9\\
473	3.9\\
474	3.9\\
475	3.9\\
476	3.9\\
477	3.5\\
478	3.5\\
479	3.5\\
480	3.5\\
481	3.5\\
482	3.5\\
483	3.5\\
484	3.5\\
485	3.5\\
486	1.2\\
487	1.2\\
488	1.2\\
489	1.2\\
490	1.2\\
491	1.2\\
492	1.2\\
493	1.2\\
494	1.2\\
495	1.2\\
496	1.2\\
};
\addplot [color=mycolor3, dashed, line width=\trajDashedLinewidth]
  table[row sep=crcr]{%
1	2\\
496	4\\
};
\addplot [color=mycolor4, dashed, line width=\trajDashedLinewidth]
  table[row sep=crcr]{%
1	2\\
496	4\\
};
\end{axis}
\end{tikzpicture}%  % tikz
    %        \includegraphics[width=\textwidth]{plots/tracking/crossing/results-T60=0.4-crem-y}  % png
            \caption{Estimated y-Axis Positions}	\end{subfigure}
    }
        \caption[Crossing Movement Results for CREM]{Crossing Movement Results for CREM (\Tsixty$=0.4$s).}
        \label{fig:trackingCrossingCREM}
    \end{figure}
\begin{figure}[!htbp]
    \iftoggle{quick}{%
        \includegraphics[width=\textwidth, height=\figureheight]{plots/tracking/crossing/results-T60=0.4-trem-xy}
    }{%
        \begin{subfigure}{0.49\textwidth}
             \centering
            \setlength{\figurewidth}{0.8\textwidth}
            % This file was created by matlab2tikz.
%
\definecolor{lms_red}{rgb}{0.80000,0.20780,0.21960}%
\definecolor{mycolor2}{rgb}{0.80000,0.20784,0.21961}%
\definecolor{mycolor3}{rgb}{0.92900,0.69400,0.12500}%
\definecolor{mycolor4}{rgb}{0.49400,0.18400,0.55600}%
%
\begin{tikzpicture}

\begin{axis}[%
width=0.951\figurewidth,
height=\figureheight,
at={(0\figurewidth,0\figureheight)},
scale only axis,
xmin=0,
xmax=496,
xtick={0,99.2,198.4,297.6,396.8,496},
xticklabels={{0},{1},{2},{3},{4},{5}},
xlabel style={font=\color{white!15!black}},
xlabel={$t$~[s]},
ymin=1,
ymax=5,
ylabel style={font=\color{white!15!black}},
ylabel={$p_x^{(t)}$~[m]},
axis background/.style={fill=white},
axis x line*=bottom,
axis y line*=left
]
\addplot [color=mycolor2, draw=none, mark=x, mark options={solid, mycolor2}, forget plot]
  table[row sep=crcr]{%
1	3.8\\
2	2.2\\
3	2.1\\
4	2.1\\
5	2.1\\
6	2.1\\
7	2.1\\
8	2.1\\
9	2.1\\
10	2.1\\
11	2.1\\
12	2.1\\
13	2.1\\
14	2.1\\
15	2.1\\
16	2.1\\
17	2.1\\
18	2.1\\
19	4\\
20	4\\
21	4\\
22	4\\
23	4\\
24	4\\
25	4\\
26	2.1\\
27	2.1\\
28	2.1\\
29	2.1\\
30	2.1\\
31	2.1\\
32	2.1\\
33	2.1\\
34	2.1\\
35	2\\
36	2\\
37	3.8\\
38	3.8\\
39	3.8\\
40	3.8\\
41	4.7\\
42	4.7\\
43	4.7\\
44	4.7\\
45	4.7\\
46	4.7\\
47	4.7\\
48	3.8\\
49	3.9\\
50	3.9\\
51	3.9\\
52	3.9\\
53	3.9\\
54	3.9\\
55	3.9\\
56	3.9\\
57	2.2\\
58	2.2\\
59	2.2\\
60	3.8\\
61	3.8\\
62	3.8\\
63	3.8\\
64	2.2\\
65	2.2\\
66	2.2\\
67	2.2\\
68	2.2\\
69	2.2\\
70	2.2\\
71	2.2\\
72	2.2\\
73	2.2\\
74	3.7\\
75	3.7\\
76	2.2\\
77	2.2\\
78	2.2\\
79	2.2\\
80	3.9\\
81	3.9\\
82	3.8\\
83	3.8\\
84	3.8\\
85	3.8\\
86	3.7\\
87	3.7\\
88	3.7\\
89	3.7\\
90	3.7\\
91	3.7\\
92	3.7\\
93	3.7\\
94	3.7\\
95	3.7\\
96	3.7\\
97	3.7\\
98	3.7\\
99	3.7\\
100	3.7\\
101	3.7\\
102	3.7\\
103	3.7\\
104	3.7\\
105	3.7\\
106	3.7\\
107	3.7\\
108	3.7\\
109	3.7\\
110	3.7\\
111	3.7\\
112	3.7\\
113	3.7\\
114	3.7\\
115	3.7\\
116	3.7\\
117	3.7\\
118	3.7\\
119	3.7\\
120	3.7\\
121	3.7\\
122	3.7\\
123	3.7\\
124	1.2\\
125	1.2\\
126	1.2\\
127	1.2\\
128	1.2\\
129	1.2\\
130	1.2\\
131	1.2\\
132	1.2\\
133	1.2\\
134	1.2\\
135	1.2\\
136	1.2\\
137	1.2\\
138	1.2\\
139	1.2\\
140	1.2\\
141	1.2\\
142	1.2\\
143	2.2\\
144	2.2\\
145	2.2\\
146	2.2\\
147	1.2\\
148	1.2\\
149	1.2\\
150	1.2\\
151	3.5\\
152	3.5\\
153	3.5\\
154	3.5\\
155	3.4\\
156	3.4\\
157	3.4\\
158	3.4\\
159	3.4\\
160	3.4\\
161	3.4\\
162	3.3\\
163	3.3\\
164	3.3\\
165	3.3\\
166	3.3\\
167	3.3\\
168	3.3\\
169	3.3\\
170	3.3\\
171	3.3\\
172	3.3\\
173	3.3\\
174	3.3\\
175	3.3\\
176	3.3\\
177	3.3\\
178	3.3\\
179	3.3\\
180	3.3\\
181	3.3\\
182	3.3\\
183	3.3\\
184	3.3\\
185	3.3\\
186	3.3\\
187	3.3\\
188	3.3\\
189	3.3\\
190	3.3\\
191	3.3\\
192	3.3\\
193	3.3\\
194	3.3\\
195	3.3\\
196	3.3\\
197	3.3\\
198	3.3\\
199	3.3\\
200	3.3\\
201	3.3\\
202	3.3\\
203	3.3\\
204	3.3\\
205	3.3\\
206	3.3\\
207	3.3\\
208	3.3\\
209	3.3\\
210	3.2\\
211	3.2\\
212	3.2\\
213	3.2\\
214	3.2\\
215	3.1\\
216	3.1\\
217	3.1\\
218	3.1\\
219	3.1\\
220	3.1\\
221	3.1\\
222	3.1\\
223	3.1\\
224	3.1\\
225	3.1\\
226	3.1\\
227	3.1\\
228	3.1\\
229	3.1\\
230	3.1\\
231	3.1\\
232	3.1\\
233	3.1\\
234	3.1\\
235	3.1\\
236	3.1\\
237	3.1\\
238	3.1\\
239	3.1\\
240	3.1\\
241	3.1\\
242	3.1\\
243	3\\
244	3\\
245	3\\
246	3\\
247	3\\
248	3\\
249	3\\
250	3\\
251	3\\
252	3\\
253	3\\
254	3\\
255	3\\
256	3\\
257	3\\
258	3\\
259	3\\
260	3\\
261	3\\
262	3\\
263	3\\
264	3\\
265	3\\
266	3\\
267	3\\
268	3\\
269	3\\
270	3\\
271	3\\
272	3\\
273	3\\
274	3\\
275	3\\
276	3\\
277	3\\
278	3\\
279	3\\
280	3\\
281	3\\
282	3\\
283	3\\
284	3\\
285	3\\
286	3\\
287	3\\
288	3\\
289	3\\
290	3\\
291	3\\
292	3\\
293	3\\
294	3\\
295	3\\
296	3\\
297	3\\
298	3\\
299	3\\
300	3\\
301	3\\
302	3\\
303	3\\
304	3\\
305	3\\
306	2.9\\
307	2.9\\
308	2.9\\
309	2.9\\
310	2.9\\
311	2.9\\
312	2.9\\
313	2.9\\
314	2.9\\
315	2.9\\
316	2.9\\
317	2.9\\
318	2.9\\
319	2.9\\
320	2.9\\
321	2.9\\
322	2.9\\
323	2.9\\
324	2.9\\
325	2.9\\
326	2.9\\
327	2.9\\
328	2.9\\
329	2.9\\
330	2.9\\
331	2.9\\
332	2.9\\
333	2.9\\
334	2.8\\
335	2.8\\
336	2.8\\
337	2.8\\
338	2.8\\
339	2.8\\
340	2.8\\
341	2.8\\
342	2.9\\
343	2.9\\
344	2.9\\
345	2.8\\
346	2.8\\
347	2.8\\
348	2.8\\
349	2.9\\
350	2.9\\
351	2.9\\
352	2.9\\
353	2.9\\
354	2.9\\
355	2.9\\
356	2.9\\
357	2.9\\
358	2.9\\
359	2.9\\
360	2.9\\
361	2.9\\
362	2.9\\
363	2.9\\
364	2.9\\
365	2.9\\
366	2.9\\
367	2.8\\
368	2.8\\
369	2.6\\
370	2.6\\
371	2.5\\
372	2.5\\
373	2.5\\
374	2.5\\
375	2.5\\
376	2.5\\
377	2.5\\
378	2.5\\
379	2.5\\
380	2.5\\
381	2.5\\
382	2.9\\
383	2.5\\
384	2.5\\
385	2.5\\
386	2.9\\
387	2.9\\
388	2.9\\
389	2.9\\
390	2.9\\
391	2.9\\
392	2.9\\
393	2.9\\
394	2.9\\
395	2.9\\
396	2.9\\
397	2.9\\
398	2.9\\
399	2.9\\
400	2.9\\
401	2.9\\
402	2.9\\
403	2.9\\
404	2.9\\
405	2.9\\
406	3\\
407	3\\
408	3\\
409	3\\
410	3\\
411	3\\
412	2.5\\
413	2.5\\
414	2.5\\
415	2.4\\
416	2.4\\
417	2.4\\
418	2.4\\
419	2.4\\
420	2.4\\
421	2.4\\
422	2.4\\
423	2.4\\
424	2.4\\
425	2.4\\
426	2.4\\
427	2.4\\
428	2.4\\
429	2.4\\
430	2.4\\
431	2.4\\
432	2.4\\
433	2.4\\
434	2.4\\
435	2.4\\
436	2.4\\
437	2.4\\
438	2.4\\
439	2.4\\
440	2.4\\
441	2.4\\
442	2.4\\
443	2.4\\
444	2.4\\
445	2.4\\
446	2.4\\
447	2.4\\
448	2.4\\
449	2.4\\
450	2.4\\
451	2.4\\
452	2.4\\
453	2.4\\
454	2.4\\
455	2.4\\
456	2.4\\
457	2.4\\
458	2.4\\
459	2.4\\
460	2.4\\
461	2.4\\
462	2.4\\
463	2.4\\
464	2.4\\
465	2.4\\
466	2.4\\
467	2.4\\
468	2.4\\
469	2.4\\
470	2.4\\
471	2.4\\
472	2.4\\
473	2.4\\
474	2.4\\
475	2.4\\
476	3.9\\
477	3.9\\
478	3.9\\
479	3.9\\
480	3.9\\
481	3.9\\
482	3.9\\
483	3.9\\
484	3.9\\
485	3.9\\
486	3.9\\
487	3.9\\
488	3.9\\
489	3.9\\
490	3.9\\
491	3.9\\
492	3.9\\
493	3.9\\
494	3.9\\
495	3.9\\
496	3.9\\
};
\addplot [color=mycolor2, draw=none, mark=x, mark options={solid, mycolor2}, forget plot]
  table[row sep=crcr]{%
1	2.3\\
2	1.3\\
3	2.2\\
4	1.6\\
5	2.2\\
6	2.2\\
7	1.5\\
8	1.5\\
9	1.5\\
10	1.5\\
11	2.2\\
12	2.2\\
13	2.2\\
14	3.9\\
15	4\\
16	3.9\\
17	3.9\\
18	3.9\\
19	2.1\\
20	2.1\\
21	2.1\\
22	2.1\\
23	2.1\\
24	2.1\\
25	2.1\\
26	4\\
27	4\\
28	4\\
29	4\\
30	4\\
31	4\\
32	2.2\\
33	2.2\\
34	2.2\\
35	2.2\\
36	3.8\\
37	2\\
38	2\\
39	2\\
40	4.7\\
41	3.8\\
42	3.9\\
43	3.9\\
44	3.8\\
45	3.8\\
46	3.8\\
47	3.8\\
48	4.7\\
49	4.7\\
50	4.7\\
51	4.7\\
52	4.7\\
53	3.8\\
54	3.7\\
55	2.3\\
56	2.2\\
57	3.9\\
58	3.8\\
59	3.8\\
60	2.2\\
61	2.2\\
62	2.2\\
63	2.2\\
64	3.8\\
65	3.8\\
66	3.8\\
67	3.8\\
68	3.7\\
69	3.7\\
70	3.7\\
71	3.7\\
72	3.7\\
73	3.7\\
74	2.2\\
75	2.2\\
76	3.7\\
77	3.7\\
78	3.7\\
79	3.9\\
80	2.2\\
81	2.2\\
82	2.2\\
83	2.2\\
84	3.5\\
85	3.6\\
86	2.2\\
87	2.2\\
88	2.2\\
89	2.2\\
90	3.4\\
91	3.4\\
92	3.4\\
93	3.4\\
94	3.4\\
95	3.4\\
96	3.4\\
97	3.4\\
98	3.4\\
99	3.4\\
100	3.4\\
101	3.2\\
102	1.2\\
103	1.2\\
104	3.8\\
105	1.2\\
106	1.2\\
107	2.2\\
108	1.2\\
109	3.1\\
110	1.2\\
111	1.2\\
112	1.2\\
113	1.2\\
114	1.2\\
115	1.2\\
116	1.2\\
117	1.2\\
118	1.2\\
119	1.2\\
120	1.2\\
121	1.2\\
122	1.2\\
123	1.2\\
124	3.7\\
125	3.7\\
126	3.7\\
127	3.7\\
128	3.7\\
129	1.2\\
130	1.2\\
131	3.7\\
132	2.2\\
133	2.2\\
134	2.2\\
135	2.2\\
136	2.2\\
137	2.2\\
138	2.2\\
139	2.2\\
140	2.2\\
141	2.2\\
142	2.2\\
143	1.2\\
144	1.2\\
145	1.2\\
146	1.2\\
147	2.2\\
148	2.2\\
149	2.2\\
150	1.2\\
151	1.2\\
152	1.2\\
153	1.2\\
154	1.2\\
155	1.2\\
156	1.2\\
157	1.2\\
158	1.2\\
159	1.2\\
160	1.2\\
161	1.2\\
162	1.2\\
163	1.2\\
164	1.2\\
165	1.2\\
166	1.2\\
167	1.2\\
168	1.2\\
169	1.2\\
170	1.2\\
171	3.7\\
172	3.7\\
173	3.7\\
174	3.7\\
175	2.7\\
176	2.7\\
177	2.7\\
178	2.7\\
179	4.7\\
180	2.7\\
181	2.7\\
182	2.7\\
183	1.2\\
184	1.2\\
185	1.2\\
186	1.2\\
187	1.2\\
188	1.2\\
189	1.2\\
190	1.2\\
191	1.2\\
192	1.2\\
193	1.2\\
194	1.2\\
195	1.2\\
196	2.7\\
197	2.7\\
198	2.7\\
199	2.7\\
200	2.7\\
201	1.2\\
202	4.7\\
203	3.7\\
204	3.7\\
205	3.7\\
206	3.7\\
207	3.7\\
208	2.8\\
209	2.7\\
210	2.8\\
211	2.6\\
212	2.6\\
213	2.6\\
214	2.6\\
215	2.6\\
216	2.6\\
217	2.6\\
218	2.6\\
219	2.6\\
220	2.6\\
221	2.6\\
222	2.6\\
223	2.8\\
224	4.7\\
225	4.7\\
226	4.7\\
227	4.7\\
228	4.7\\
229	2.9\\
230	4.7\\
231	4.7\\
232	4.7\\
233	4.7\\
234	4.7\\
235	3.7\\
236	2.8\\
237	2.6\\
238	4.7\\
239	4.7\\
240	4.7\\
241	2.8\\
242	2.8\\
243	3.5\\
244	3.5\\
245	3.5\\
246	3.5\\
247	3.3\\
248	1.2\\
249	1.2\\
250	1.2\\
251	1.2\\
252	1.2\\
253	1.2\\
254	1.2\\
255	1.2\\
256	1.2\\
257	1.2\\
258	1.2\\
259	1.2\\
260	1.2\\
261	1.2\\
262	1.2\\
263	1.2\\
264	1.2\\
265	1.2\\
266	1.2\\
267	1.2\\
268	1.2\\
269	1.2\\
270	1.2\\
271	1.2\\
272	1.2\\
273	1.2\\
274	1.2\\
275	1.2\\
276	1.2\\
277	1.2\\
278	1.2\\
279	1.2\\
280	1.2\\
281	1.2\\
282	1.2\\
283	1.2\\
284	1.2\\
285	1.2\\
286	1.2\\
287	1.2\\
288	1.2\\
289	1.2\\
290	1.2\\
291	1.2\\
292	1.2\\
293	1.2\\
294	1.2\\
295	1.2\\
296	1.2\\
297	1.2\\
298	1.2\\
299	1.2\\
300	1.2\\
301	1.2\\
302	1.2\\
303	1.2\\
304	1.2\\
305	1.2\\
306	1.2\\
307	1.2\\
308	1.2\\
309	1.2\\
310	1.2\\
311	1.2\\
312	1.2\\
313	1.2\\
314	1.2\\
315	1.2\\
316	1.2\\
317	1.2\\
318	1.2\\
319	1.2\\
320	1.2\\
321	2.5\\
322	2.5\\
323	2.5\\
324	2.5\\
325	2.5\\
326	1.2\\
327	1.2\\
328	1.2\\
329	1.2\\
330	1.2\\
331	2.5\\
332	2.5\\
333	2.5\\
334	2.3\\
335	2.3\\
336	2.3\\
337	2.3\\
338	2.3\\
339	2.3\\
340	2.3\\
341	2.3\\
342	2.5\\
343	2.5\\
344	2.5\\
345	2.3\\
346	2.3\\
347	2.3\\
348	2.3\\
349	2.3\\
350	2.3\\
351	2.3\\
352	2.3\\
353	2.3\\
354	2.3\\
355	3.5\\
356	3.5\\
357	3.5\\
358	3.4\\
359	3.5\\
360	2.3\\
361	2.3\\
362	2.3\\
363	2.3\\
364	2.3\\
365	2.3\\
366	2.3\\
367	2.3\\
368	2.3\\
369	3.1\\
370	3.1\\
371	2.9\\
372	2.9\\
373	2.9\\
374	2.9\\
375	2.9\\
376	3.1\\
377	3.1\\
378	3.1\\
379	3.1\\
380	3.1\\
381	3.1\\
382	2.3\\
383	3.1\\
384	3.1\\
385	3.1\\
386	3.5\\
387	3.4\\
388	3.4\\
389	3.4\\
390	3.4\\
391	3.4\\
392	3.4\\
393	3.4\\
394	3.4\\
395	3.4\\
396	3.4\\
397	3.4\\
398	3.4\\
399	3.4\\
400	3.4\\
401	3.4\\
402	3.5\\
403	3.5\\
404	3.4\\
405	3.4\\
406	2.5\\
407	2.5\\
408	2.5\\
409	2.5\\
410	3.5\\
411	3.7\\
412	3.1\\
413	3.1\\
414	2.5\\
415	3\\
416	3\\
417	2.5\\
418	2.5\\
419	2.5\\
420	2.5\\
421	2.5\\
422	2.5\\
423	2.9\\
424	2.5\\
425	2.5\\
426	2.5\\
427	2.5\\
428	2.5\\
429	2.5\\
430	2.9\\
431	2.9\\
432	2.9\\
433	3\\
434	2.4\\
435	2.4\\
436	2.4\\
437	2.4\\
438	2.4\\
439	2.4\\
440	1.2\\
441	1.2\\
442	1.2\\
443	1.2\\
444	2.4\\
445	3.7\\
446	3.7\\
447	3.7\\
448	3.7\\
449	1.2\\
450	1.2\\
451	1.2\\
452	1.2\\
453	1.2\\
454	1.2\\
455	1.2\\
456	3.7\\
457	3.7\\
458	3.7\\
459	3.7\\
460	4\\
461	4\\
462	4\\
463	3.9\\
464	3.9\\
465	3.9\\
466	3.9\\
467	3.9\\
468	3.9\\
469	3.9\\
470	3.9\\
471	3.9\\
472	3.9\\
473	3.9\\
474	3.9\\
475	3.9\\
476	2.4\\
477	2.4\\
478	2.4\\
479	2.4\\
480	2.4\\
481	2.4\\
482	2.4\\
483	2.4\\
484	2.4\\
485	2.4\\
486	2.4\\
487	2.4\\
488	2.4\\
489	2.4\\
490	2.4\\
491	3.7\\
492	2.4\\
493	3.7\\
494	2.4\\
495	2.4\\
496	2.4\\
};
\addplot [color=mycolor3, dashed, forget plot]
  table[row sep=crcr]{%
1	2\\
2	2.0040404040404\\
3	2.00808080808081\\
4	2.01212121212121\\
5	2.01616161616162\\
6	2.02020202020202\\
7	2.02424242424242\\
8	2.02828282828283\\
9	2.03232323232323\\
10	2.03636363636364\\
11	2.04040404040404\\
12	2.04444444444444\\
13	2.04848484848485\\
14	2.05252525252525\\
15	2.05656565656566\\
16	2.06060606060606\\
17	2.06464646464646\\
18	2.06868686868687\\
19	2.07272727272727\\
20	2.07676767676768\\
21	2.08080808080808\\
22	2.08484848484848\\
23	2.08888888888889\\
24	2.09292929292929\\
25	2.0969696969697\\
26	2.1010101010101\\
27	2.1050505050505\\
28	2.10909090909091\\
29	2.11313131313131\\
30	2.11717171717172\\
31	2.12121212121212\\
32	2.12525252525253\\
33	2.12929292929293\\
34	2.13333333333333\\
35	2.13737373737374\\
36	2.14141414141414\\
37	2.14545454545455\\
38	2.14949494949495\\
39	2.15353535353535\\
40	2.15757575757576\\
41	2.16161616161616\\
42	2.16565656565657\\
43	2.16969696969697\\
44	2.17373737373737\\
45	2.17777777777778\\
46	2.18181818181818\\
47	2.18585858585859\\
48	2.18989898989899\\
49	2.19393939393939\\
50	2.1979797979798\\
51	2.2020202020202\\
52	2.20606060606061\\
53	2.21010101010101\\
54	2.21414141414141\\
55	2.21818181818182\\
56	2.22222222222222\\
57	2.22626262626263\\
58	2.23030303030303\\
59	2.23434343434343\\
60	2.23838383838384\\
61	2.24242424242424\\
62	2.24646464646465\\
63	2.25050505050505\\
64	2.25454545454545\\
65	2.25858585858586\\
66	2.26262626262626\\
67	2.26666666666667\\
68	2.27070707070707\\
69	2.27474747474747\\
70	2.27878787878788\\
71	2.28282828282828\\
72	2.28686868686869\\
73	2.29090909090909\\
74	2.2949494949495\\
75	2.2989898989899\\
76	2.3030303030303\\
77	2.30707070707071\\
78	2.31111111111111\\
79	2.31515151515151\\
80	2.31919191919192\\
81	2.32323232323232\\
82	2.32727272727273\\
83	2.33131313131313\\
84	2.33535353535354\\
85	2.33939393939394\\
86	2.34343434343434\\
87	2.34747474747475\\
88	2.35151515151515\\
89	2.35555555555556\\
90	2.35959595959596\\
91	2.36363636363636\\
92	2.36767676767677\\
93	2.37171717171717\\
94	2.37575757575758\\
95	2.37979797979798\\
96	2.38383838383838\\
97	2.38787878787879\\
98	2.39191919191919\\
99	2.3959595959596\\
100	2.4\\
101	2.4040404040404\\
102	2.40808080808081\\
103	2.41212121212121\\
104	2.41616161616162\\
105	2.42020202020202\\
106	2.42424242424242\\
107	2.42828282828283\\
108	2.43232323232323\\
109	2.43636363636364\\
110	2.44040404040404\\
111	2.44444444444444\\
112	2.44848484848485\\
113	2.45252525252525\\
114	2.45656565656566\\
115	2.46060606060606\\
116	2.46464646464646\\
117	2.46868686868687\\
118	2.47272727272727\\
119	2.47676767676768\\
120	2.48080808080808\\
121	2.48484848484848\\
122	2.48888888888889\\
123	2.49292929292929\\
124	2.4969696969697\\
125	2.5010101010101\\
126	2.50505050505051\\
127	2.50909090909091\\
128	2.51313131313131\\
129	2.51717171717172\\
130	2.52121212121212\\
131	2.52525252525253\\
132	2.52929292929293\\
133	2.53333333333333\\
134	2.53737373737374\\
135	2.54141414141414\\
136	2.54545454545455\\
137	2.54949494949495\\
138	2.55353535353535\\
139	2.55757575757576\\
140	2.56161616161616\\
141	2.56565656565657\\
142	2.56969696969697\\
143	2.57373737373737\\
144	2.57777777777778\\
145	2.58181818181818\\
146	2.58585858585859\\
147	2.58989898989899\\
148	2.59393939393939\\
149	2.5979797979798\\
150	2.6020202020202\\
151	2.60606060606061\\
152	2.61010101010101\\
153	2.61414141414141\\
154	2.61818181818182\\
155	2.62222222222222\\
156	2.62626262626263\\
157	2.63030303030303\\
158	2.63434343434343\\
159	2.63838383838384\\
160	2.64242424242424\\
161	2.64646464646465\\
162	2.65050505050505\\
163	2.65454545454545\\
164	2.65858585858586\\
165	2.66262626262626\\
166	2.66666666666667\\
167	2.67070707070707\\
168	2.67474747474747\\
169	2.67878787878788\\
170	2.68282828282828\\
171	2.68686868686869\\
172	2.69090909090909\\
173	2.69494949494949\\
174	2.6989898989899\\
175	2.7030303030303\\
176	2.70707070707071\\
177	2.71111111111111\\
178	2.71515151515152\\
179	2.71919191919192\\
180	2.72323232323232\\
181	2.72727272727273\\
182	2.73131313131313\\
183	2.73535353535354\\
184	2.73939393939394\\
185	2.74343434343434\\
186	2.74747474747475\\
187	2.75151515151515\\
188	2.75555555555556\\
189	2.75959595959596\\
190	2.76363636363636\\
191	2.76767676767677\\
192	2.77171717171717\\
193	2.77575757575758\\
194	2.77979797979798\\
195	2.78383838383838\\
196	2.78787878787879\\
197	2.79191919191919\\
198	2.7959595959596\\
199	2.8\\
200	2.8040404040404\\
201	2.80808080808081\\
202	2.81212121212121\\
203	2.81616161616162\\
204	2.82020202020202\\
205	2.82424242424242\\
206	2.82828282828283\\
207	2.83232323232323\\
208	2.83636363636364\\
209	2.84040404040404\\
210	2.84444444444444\\
211	2.84848484848485\\
212	2.85252525252525\\
213	2.85656565656566\\
214	2.86060606060606\\
215	2.86464646464646\\
216	2.86868686868687\\
217	2.87272727272727\\
218	2.87676767676768\\
219	2.88080808080808\\
220	2.88484848484848\\
221	2.88888888888889\\
222	2.89292929292929\\
223	2.8969696969697\\
224	2.9010101010101\\
225	2.90505050505051\\
226	2.90909090909091\\
227	2.91313131313131\\
228	2.91717171717172\\
229	2.92121212121212\\
230	2.92525252525253\\
231	2.92929292929293\\
232	2.93333333333333\\
233	2.93737373737374\\
234	2.94141414141414\\
235	2.94545454545455\\
236	2.94949494949495\\
237	2.95353535353535\\
238	2.95757575757576\\
239	2.96161616161616\\
240	2.96565656565657\\
241	2.96969696969697\\
242	2.97373737373737\\
243	2.97777777777778\\
244	2.98181818181818\\
245	2.98585858585859\\
246	2.98989898989899\\
247	2.99393939393939\\
248	2.9979797979798\\
249	3.0020202020202\\
250	3.00606060606061\\
251	3.01010101010101\\
252	3.01414141414141\\
253	3.01818181818182\\
254	3.02222222222222\\
255	3.02626262626263\\
256	3.03030303030303\\
257	3.03434343434343\\
258	3.03838383838384\\
259	3.04242424242424\\
260	3.04646464646465\\
261	3.05050505050505\\
262	3.05454545454545\\
263	3.05858585858586\\
264	3.06262626262626\\
265	3.06666666666667\\
266	3.07070707070707\\
267	3.07474747474747\\
268	3.07878787878788\\
269	3.08282828282828\\
270	3.08686868686869\\
271	3.09090909090909\\
272	3.09494949494949\\
273	3.0989898989899\\
274	3.1030303030303\\
275	3.10707070707071\\
276	3.11111111111111\\
277	3.11515151515152\\
278	3.11919191919192\\
279	3.12323232323232\\
280	3.12727272727273\\
281	3.13131313131313\\
282	3.13535353535354\\
283	3.13939393939394\\
284	3.14343434343434\\
285	3.14747474747475\\
286	3.15151515151515\\
287	3.15555555555556\\
288	3.15959595959596\\
289	3.16363636363636\\
290	3.16767676767677\\
291	3.17171717171717\\
292	3.17575757575758\\
293	3.17979797979798\\
294	3.18383838383838\\
295	3.18787878787879\\
296	3.19191919191919\\
297	3.1959595959596\\
298	3.2\\
299	3.2040404040404\\
300	3.20808080808081\\
301	3.21212121212121\\
302	3.21616161616162\\
303	3.22020202020202\\
304	3.22424242424242\\
305	3.22828282828283\\
306	3.23232323232323\\
307	3.23636363636364\\
308	3.24040404040404\\
309	3.24444444444444\\
310	3.24848484848485\\
311	3.25252525252525\\
312	3.25656565656566\\
313	3.26060606060606\\
314	3.26464646464646\\
315	3.26868686868687\\
316	3.27272727272727\\
317	3.27676767676768\\
318	3.28080808080808\\
319	3.28484848484849\\
320	3.28888888888889\\
321	3.29292929292929\\
322	3.2969696969697\\
323	3.3010101010101\\
324	3.30505050505051\\
325	3.30909090909091\\
326	3.31313131313131\\
327	3.31717171717172\\
328	3.32121212121212\\
329	3.32525252525253\\
330	3.32929292929293\\
331	3.33333333333333\\
332	3.33737373737374\\
333	3.34141414141414\\
334	3.34545454545455\\
335	3.34949494949495\\
336	3.35353535353535\\
337	3.35757575757576\\
338	3.36161616161616\\
339	3.36565656565657\\
340	3.36969696969697\\
341	3.37373737373737\\
342	3.37777777777778\\
343	3.38181818181818\\
344	3.38585858585859\\
345	3.38989898989899\\
346	3.39393939393939\\
347	3.3979797979798\\
348	3.4020202020202\\
349	3.40606060606061\\
350	3.41010101010101\\
351	3.41414141414141\\
352	3.41818181818182\\
353	3.42222222222222\\
354	3.42626262626263\\
355	3.43030303030303\\
356	3.43434343434343\\
357	3.43838383838384\\
358	3.44242424242424\\
359	3.44646464646465\\
360	3.45050505050505\\
361	3.45454545454545\\
362	3.45858585858586\\
363	3.46262626262626\\
364	3.46666666666667\\
365	3.47070707070707\\
366	3.47474747474747\\
367	3.47878787878788\\
368	3.48282828282828\\
369	3.48686868686869\\
370	3.49090909090909\\
371	3.49494949494949\\
372	3.4989898989899\\
373	3.5030303030303\\
374	3.50707070707071\\
375	3.51111111111111\\
376	3.51515151515152\\
377	3.51919191919192\\
378	3.52323232323232\\
379	3.52727272727273\\
380	3.53131313131313\\
381	3.53535353535354\\
382	3.53939393939394\\
383	3.54343434343434\\
384	3.54747474747475\\
385	3.55151515151515\\
386	3.55555555555556\\
387	3.55959595959596\\
388	3.56363636363636\\
389	3.56767676767677\\
390	3.57171717171717\\
391	3.57575757575758\\
392	3.57979797979798\\
393	3.58383838383838\\
394	3.58787878787879\\
395	3.59191919191919\\
396	3.5959595959596\\
397	3.6\\
398	3.6040404040404\\
399	3.60808080808081\\
400	3.61212121212121\\
401	3.61616161616162\\
402	3.62020202020202\\
403	3.62424242424242\\
404	3.62828282828283\\
405	3.63232323232323\\
406	3.63636363636364\\
407	3.64040404040404\\
408	3.64444444444444\\
409	3.64848484848485\\
410	3.65252525252525\\
411	3.65656565656566\\
412	3.66060606060606\\
413	3.66464646464646\\
414	3.66868686868687\\
415	3.67272727272727\\
416	3.67676767676768\\
417	3.68080808080808\\
418	3.68484848484848\\
419	3.68888888888889\\
420	3.69292929292929\\
421	3.6969696969697\\
422	3.7010101010101\\
423	3.70505050505051\\
424	3.70909090909091\\
425	3.71313131313131\\
426	3.71717171717172\\
427	3.72121212121212\\
428	3.72525252525253\\
429	3.72929292929293\\
430	3.73333333333333\\
431	3.73737373737374\\
432	3.74141414141414\\
433	3.74545454545455\\
434	3.74949494949495\\
435	3.75353535353535\\
436	3.75757575757576\\
437	3.76161616161616\\
438	3.76565656565657\\
439	3.76969696969697\\
440	3.77373737373737\\
441	3.77777777777778\\
442	3.78181818181818\\
443	3.78585858585859\\
444	3.78989898989899\\
445	3.79393939393939\\
446	3.7979797979798\\
447	3.8020202020202\\
448	3.80606060606061\\
449	3.81010101010101\\
450	3.81414141414141\\
451	3.81818181818182\\
452	3.82222222222222\\
453	3.82626262626263\\
454	3.83030303030303\\
455	3.83434343434343\\
456	3.83838383838384\\
457	3.84242424242424\\
458	3.84646464646465\\
459	3.85050505050505\\
460	3.85454545454545\\
461	3.85858585858586\\
462	3.86262626262626\\
463	3.86666666666667\\
464	3.87070707070707\\
465	3.87474747474747\\
466	3.87878787878788\\
467	3.88282828282828\\
468	3.88686868686869\\
469	3.89090909090909\\
470	3.8949494949495\\
471	3.8989898989899\\
472	3.9030303030303\\
473	3.90707070707071\\
474	3.91111111111111\\
475	3.91515151515152\\
476	3.91919191919192\\
477	3.92323232323232\\
478	3.92727272727273\\
479	3.93131313131313\\
480	3.93535353535354\\
481	3.93939393939394\\
482	3.94343434343434\\
483	3.94747474747475\\
484	3.95151515151515\\
485	3.95555555555556\\
486	3.95959595959596\\
487	3.96363636363636\\
488	3.96767676767677\\
489	3.97171717171717\\
490	3.97575757575758\\
491	3.97979797979798\\
492	3.98383838383838\\
493	3.98787878787879\\
494	3.99191919191919\\
495	3.9959595959596\\
496	4\\
};
\addplot [color=mycolor4, dashed, forget plot]
  table[row sep=crcr]{%
1	4\\
2	3.9959595959596\\
3	3.99191919191919\\
4	3.98787878787879\\
5	3.98383838383838\\
6	3.97979797979798\\
7	3.97575757575758\\
8	3.97171717171717\\
9	3.96767676767677\\
10	3.96363636363636\\
11	3.95959595959596\\
12	3.95555555555556\\
13	3.95151515151515\\
14	3.94747474747475\\
15	3.94343434343434\\
16	3.93939393939394\\
17	3.93535353535354\\
18	3.93131313131313\\
19	3.92727272727273\\
20	3.92323232323232\\
21	3.91919191919192\\
22	3.91515151515152\\
23	3.91111111111111\\
24	3.90707070707071\\
25	3.9030303030303\\
26	3.8989898989899\\
27	3.8949494949495\\
28	3.89090909090909\\
29	3.88686868686869\\
30	3.88282828282828\\
31	3.87878787878788\\
32	3.87474747474747\\
33	3.87070707070707\\
34	3.86666666666667\\
35	3.86262626262626\\
36	3.85858585858586\\
37	3.85454545454545\\
38	3.85050505050505\\
39	3.84646464646465\\
40	3.84242424242424\\
41	3.83838383838384\\
42	3.83434343434343\\
43	3.83030303030303\\
44	3.82626262626263\\
45	3.82222222222222\\
46	3.81818181818182\\
47	3.81414141414141\\
48	3.81010101010101\\
49	3.80606060606061\\
50	3.8020202020202\\
51	3.7979797979798\\
52	3.79393939393939\\
53	3.78989898989899\\
54	3.78585858585859\\
55	3.78181818181818\\
56	3.77777777777778\\
57	3.77373737373737\\
58	3.76969696969697\\
59	3.76565656565657\\
60	3.76161616161616\\
61	3.75757575757576\\
62	3.75353535353535\\
63	3.74949494949495\\
64	3.74545454545455\\
65	3.74141414141414\\
66	3.73737373737374\\
67	3.73333333333333\\
68	3.72929292929293\\
69	3.72525252525253\\
70	3.72121212121212\\
71	3.71717171717172\\
72	3.71313131313131\\
73	3.70909090909091\\
74	3.7050505050505\\
75	3.7010101010101\\
76	3.6969696969697\\
77	3.69292929292929\\
78	3.68888888888889\\
79	3.68484848484849\\
80	3.68080808080808\\
81	3.67676767676768\\
82	3.67272727272727\\
83	3.66868686868687\\
84	3.66464646464646\\
85	3.66060606060606\\
86	3.65656565656566\\
87	3.65252525252525\\
88	3.64848484848485\\
89	3.64444444444444\\
90	3.64040404040404\\
91	3.63636363636364\\
92	3.63232323232323\\
93	3.62828282828283\\
94	3.62424242424242\\
95	3.62020202020202\\
96	3.61616161616162\\
97	3.61212121212121\\
98	3.60808080808081\\
99	3.6040404040404\\
100	3.6\\
101	3.5959595959596\\
102	3.59191919191919\\
103	3.58787878787879\\
104	3.58383838383838\\
105	3.57979797979798\\
106	3.57575757575758\\
107	3.57171717171717\\
108	3.56767676767677\\
109	3.56363636363636\\
110	3.55959595959596\\
111	3.55555555555556\\
112	3.55151515151515\\
113	3.54747474747475\\
114	3.54343434343434\\
115	3.53939393939394\\
116	3.53535353535354\\
117	3.53131313131313\\
118	3.52727272727273\\
119	3.52323232323232\\
120	3.51919191919192\\
121	3.51515151515152\\
122	3.51111111111111\\
123	3.50707070707071\\
124	3.5030303030303\\
125	3.4989898989899\\
126	3.49494949494949\\
127	3.49090909090909\\
128	3.48686868686869\\
129	3.48282828282828\\
130	3.47878787878788\\
131	3.47474747474747\\
132	3.47070707070707\\
133	3.46666666666667\\
134	3.46262626262626\\
135	3.45858585858586\\
136	3.45454545454545\\
137	3.45050505050505\\
138	3.44646464646465\\
139	3.44242424242424\\
140	3.43838383838384\\
141	3.43434343434343\\
142	3.43030303030303\\
143	3.42626262626263\\
144	3.42222222222222\\
145	3.41818181818182\\
146	3.41414141414141\\
147	3.41010101010101\\
148	3.40606060606061\\
149	3.4020202020202\\
150	3.3979797979798\\
151	3.39393939393939\\
152	3.38989898989899\\
153	3.38585858585859\\
154	3.38181818181818\\
155	3.37777777777778\\
156	3.37373737373737\\
157	3.36969696969697\\
158	3.36565656565657\\
159	3.36161616161616\\
160	3.35757575757576\\
161	3.35353535353535\\
162	3.34949494949495\\
163	3.34545454545455\\
164	3.34141414141414\\
165	3.33737373737374\\
166	3.33333333333333\\
167	3.32929292929293\\
168	3.32525252525253\\
169	3.32121212121212\\
170	3.31717171717172\\
171	3.31313131313131\\
172	3.30909090909091\\
173	3.30505050505051\\
174	3.3010101010101\\
175	3.2969696969697\\
176	3.29292929292929\\
177	3.28888888888889\\
178	3.28484848484848\\
179	3.28080808080808\\
180	3.27676767676768\\
181	3.27272727272727\\
182	3.26868686868687\\
183	3.26464646464646\\
184	3.26060606060606\\
185	3.25656565656566\\
186	3.25252525252525\\
187	3.24848484848485\\
188	3.24444444444444\\
189	3.24040404040404\\
190	3.23636363636364\\
191	3.23232323232323\\
192	3.22828282828283\\
193	3.22424242424242\\
194	3.22020202020202\\
195	3.21616161616162\\
196	3.21212121212121\\
197	3.20808080808081\\
198	3.2040404040404\\
199	3.2\\
200	3.1959595959596\\
201	3.19191919191919\\
202	3.18787878787879\\
203	3.18383838383838\\
204	3.17979797979798\\
205	3.17575757575758\\
206	3.17171717171717\\
207	3.16767676767677\\
208	3.16363636363636\\
209	3.15959595959596\\
210	3.15555555555556\\
211	3.15151515151515\\
212	3.14747474747475\\
213	3.14343434343434\\
214	3.13939393939394\\
215	3.13535353535354\\
216	3.13131313131313\\
217	3.12727272727273\\
218	3.12323232323232\\
219	3.11919191919192\\
220	3.11515151515152\\
221	3.11111111111111\\
222	3.10707070707071\\
223	3.1030303030303\\
224	3.0989898989899\\
225	3.09494949494949\\
226	3.09090909090909\\
227	3.08686868686869\\
228	3.08282828282828\\
229	3.07878787878788\\
230	3.07474747474747\\
231	3.07070707070707\\
232	3.06666666666667\\
233	3.06262626262626\\
234	3.05858585858586\\
235	3.05454545454545\\
236	3.05050505050505\\
237	3.04646464646465\\
238	3.04242424242424\\
239	3.03838383838384\\
240	3.03434343434343\\
241	3.03030303030303\\
242	3.02626262626263\\
243	3.02222222222222\\
244	3.01818181818182\\
245	3.01414141414141\\
246	3.01010101010101\\
247	3.00606060606061\\
248	3.0020202020202\\
249	2.9979797979798\\
250	2.99393939393939\\
251	2.98989898989899\\
252	2.98585858585859\\
253	2.98181818181818\\
254	2.97777777777778\\
255	2.97373737373737\\
256	2.96969696969697\\
257	2.96565656565657\\
258	2.96161616161616\\
259	2.95757575757576\\
260	2.95353535353535\\
261	2.94949494949495\\
262	2.94545454545455\\
263	2.94141414141414\\
264	2.93737373737374\\
265	2.93333333333333\\
266	2.92929292929293\\
267	2.92525252525253\\
268	2.92121212121212\\
269	2.91717171717172\\
270	2.91313131313131\\
271	2.90909090909091\\
272	2.90505050505051\\
273	2.9010101010101\\
274	2.8969696969697\\
275	2.89292929292929\\
276	2.88888888888889\\
277	2.88484848484848\\
278	2.88080808080808\\
279	2.87676767676768\\
280	2.87272727272727\\
281	2.86868686868687\\
282	2.86464646464646\\
283	2.86060606060606\\
284	2.85656565656566\\
285	2.85252525252525\\
286	2.84848484848485\\
287	2.84444444444444\\
288	2.84040404040404\\
289	2.83636363636364\\
290	2.83232323232323\\
291	2.82828282828283\\
292	2.82424242424242\\
293	2.82020202020202\\
294	2.81616161616162\\
295	2.81212121212121\\
296	2.80808080808081\\
297	2.8040404040404\\
298	2.8\\
299	2.7959595959596\\
300	2.79191919191919\\
301	2.78787878787879\\
302	2.78383838383838\\
303	2.77979797979798\\
304	2.77575757575758\\
305	2.77171717171717\\
306	2.76767676767677\\
307	2.76363636363636\\
308	2.75959595959596\\
309	2.75555555555556\\
310	2.75151515151515\\
311	2.74747474747475\\
312	2.74343434343434\\
313	2.73939393939394\\
314	2.73535353535354\\
315	2.73131313131313\\
316	2.72727272727273\\
317	2.72323232323232\\
318	2.71919191919192\\
319	2.71515151515151\\
320	2.71111111111111\\
321	2.70707070707071\\
322	2.7030303030303\\
323	2.6989898989899\\
324	2.69494949494949\\
325	2.69090909090909\\
326	2.68686868686869\\
327	2.68282828282828\\
328	2.67878787878788\\
329	2.67474747474747\\
330	2.67070707070707\\
331	2.66666666666667\\
332	2.66262626262626\\
333	2.65858585858586\\
334	2.65454545454545\\
335	2.65050505050505\\
336	2.64646464646465\\
337	2.64242424242424\\
338	2.63838383838384\\
339	2.63434343434343\\
340	2.63030303030303\\
341	2.62626262626263\\
342	2.62222222222222\\
343	2.61818181818182\\
344	2.61414141414141\\
345	2.61010101010101\\
346	2.60606060606061\\
347	2.6020202020202\\
348	2.5979797979798\\
349	2.59393939393939\\
350	2.58989898989899\\
351	2.58585858585859\\
352	2.58181818181818\\
353	2.57777777777778\\
354	2.57373737373737\\
355	2.56969696969697\\
356	2.56565656565657\\
357	2.56161616161616\\
358	2.55757575757576\\
359	2.55353535353535\\
360	2.54949494949495\\
361	2.54545454545455\\
362	2.54141414141414\\
363	2.53737373737374\\
364	2.53333333333333\\
365	2.52929292929293\\
366	2.52525252525253\\
367	2.52121212121212\\
368	2.51717171717172\\
369	2.51313131313131\\
370	2.50909090909091\\
371	2.50505050505051\\
372	2.5010101010101\\
373	2.4969696969697\\
374	2.49292929292929\\
375	2.48888888888889\\
376	2.48484848484848\\
377	2.48080808080808\\
378	2.47676767676768\\
379	2.47272727272727\\
380	2.46868686868687\\
381	2.46464646464646\\
382	2.46060606060606\\
383	2.45656565656566\\
384	2.45252525252525\\
385	2.44848484848485\\
386	2.44444444444444\\
387	2.44040404040404\\
388	2.43636363636364\\
389	2.43232323232323\\
390	2.42828282828283\\
391	2.42424242424242\\
392	2.42020202020202\\
393	2.41616161616162\\
394	2.41212121212121\\
395	2.40808080808081\\
396	2.4040404040404\\
397	2.4\\
398	2.3959595959596\\
399	2.39191919191919\\
400	2.38787878787879\\
401	2.38383838383838\\
402	2.37979797979798\\
403	2.37575757575758\\
404	2.37171717171717\\
405	2.36767676767677\\
406	2.36363636363636\\
407	2.35959595959596\\
408	2.35555555555556\\
409	2.35151515151515\\
410	2.34747474747475\\
411	2.34343434343434\\
412	2.33939393939394\\
413	2.33535353535354\\
414	2.33131313131313\\
415	2.32727272727273\\
416	2.32323232323232\\
417	2.31919191919192\\
418	2.31515151515152\\
419	2.31111111111111\\
420	2.30707070707071\\
421	2.3030303030303\\
422	2.2989898989899\\
423	2.29494949494949\\
424	2.29090909090909\\
425	2.28686868686869\\
426	2.28282828282828\\
427	2.27878787878788\\
428	2.27474747474747\\
429	2.27070707070707\\
430	2.26666666666667\\
431	2.26262626262626\\
432	2.25858585858586\\
433	2.25454545454545\\
434	2.25050505050505\\
435	2.24646464646465\\
436	2.24242424242424\\
437	2.23838383838384\\
438	2.23434343434343\\
439	2.23030303030303\\
440	2.22626262626263\\
441	2.22222222222222\\
442	2.21818181818182\\
443	2.21414141414141\\
444	2.21010101010101\\
445	2.20606060606061\\
446	2.2020202020202\\
447	2.1979797979798\\
448	2.19393939393939\\
449	2.18989898989899\\
450	2.18585858585859\\
451	2.18181818181818\\
452	2.17777777777778\\
453	2.17373737373737\\
454	2.16969696969697\\
455	2.16565656565657\\
456	2.16161616161616\\
457	2.15757575757576\\
458	2.15353535353535\\
459	2.14949494949495\\
460	2.14545454545455\\
461	2.14141414141414\\
462	2.13737373737374\\
463	2.13333333333333\\
464	2.12929292929293\\
465	2.12525252525253\\
466	2.12121212121212\\
467	2.11717171717172\\
468	2.11313131313131\\
469	2.10909090909091\\
470	2.1050505050505\\
471	2.1010101010101\\
472	2.0969696969697\\
473	2.09292929292929\\
474	2.08888888888889\\
475	2.08484848484848\\
476	2.08080808080808\\
477	2.07676767676768\\
478	2.07272727272727\\
479	2.06868686868687\\
480	2.06464646464646\\
481	2.06060606060606\\
482	2.05656565656566\\
483	2.05252525252525\\
484	2.04848484848485\\
485	2.04444444444444\\
486	2.04040404040404\\
487	2.03636363636364\\
488	2.03232323232323\\
489	2.02828282828283\\
490	2.02424242424242\\
491	2.02020202020202\\
492	2.01616161616162\\
493	2.01212121212121\\
494	2.00808080808081\\
495	2.0040404040404\\
496	2\\
};
\end{axis}
\end{tikzpicture}%  % tikz
            \caption{Estimated x-Axis Positions}
        \end{subfigure}
        \begin{subfigure}{0.49\textwidth}
             \centering
            \setlength{\figurewidth}{0.8\textwidth}
            % This file was created by matlab2tikz.
%
\definecolor{lms_red}{rgb}{0.80000,0.20780,0.21960}%
\definecolor{mycolor2}{rgb}{0.80000,0.20784,0.21961}%
\definecolor{mycolor3}{rgb}{0.92900,0.69400,0.12500}%
\definecolor{mycolor4}{rgb}{0.49400,0.18400,0.55600}%
%
\begin{tikzpicture}

\begin{axis}[%
width=0.951\figurewidth,
height=\figureheight,
at={(0\figurewidth,0\figureheight)},
scale only axis,
xmin=0,
xmax=496,
xtick={0,99.2,198.4,297.6,396.8,496},
xticklabels={{0},{1},{2},{3},{4},{5}},
xlabel style={font=\color{white!15!black}},
xlabel={$t$~[s]},
ymin=1,
ymax=5,
ylabel style={font=\color{white!15!black}},
ylabel={$p_y^{(t)}$~[m]},
axis background/.style={fill=white},
axis x line*=bottom,
axis y line*=left
]
\addplot [color=mycolor2, draw=none, mark=x, mark options={solid, mycolor2}, forget plot]
  table[row sep=crcr]{%
1	4.7\\
2	1.5\\
3	1.9\\
4	1.9\\
5	2\\
6	2\\
7	2\\
8	2\\
9	2\\
10	2\\
11	2\\
12	2\\
13	2\\
14	2\\
15	2\\
16	2\\
17	2\\
18	2\\
19	2.1\\
20	2.1\\
21	2.1\\
22	2.1\\
23	2.1\\
24	2.1\\
25	2.1\\
26	1.9\\
27	2\\
28	2\\
29	2\\
30	2\\
31	2\\
32	2\\
33	1.9\\
34	2\\
35	2\\
36	2\\
37	2.2\\
38	2.2\\
39	2.2\\
40	2.2\\
41	2.4\\
42	2.3\\
43	2.3\\
44	2.3\\
45	2.3\\
46	2.3\\
47	2.3\\
48	2.1\\
49	2.2\\
50	2.2\\
51	2.2\\
52	2.2\\
53	2.2\\
54	2.2\\
55	2.2\\
56	2.2\\
57	2.3\\
58	2.2\\
59	2.2\\
60	2.2\\
61	2.1\\
62	2.1\\
63	2.1\\
64	2.2\\
65	2.2\\
66	2.2\\
67	2.2\\
68	2.2\\
69	2.2\\
70	2.1\\
71	2.1\\
72	2.1\\
73	2.1\\
74	1.2\\
75	1.2\\
76	2\\
77	2\\
78	2\\
79	2\\
80	2.3\\
81	2.3\\
82	2.2\\
83	2.2\\
84	2.3\\
85	2.3\\
86	2.3\\
87	2.3\\
88	2.3\\
89	2.3\\
90	2.3\\
91	2.3\\
92	2.3\\
93	2.3\\
94	2.3\\
95	2.3\\
96	2.3\\
97	2.3\\
98	2.3\\
99	2.3\\
100	2.3\\
101	2.3\\
102	2.3\\
103	2.3\\
104	2.3\\
105	2.3\\
106	2.3\\
107	2.3\\
108	2.3\\
109	2.3\\
110	2.3\\
111	2.3\\
112	2.3\\
113	2.3\\
114	2.3\\
115	2.3\\
116	2.3\\
117	2.3\\
118	2.3\\
119	2.3\\
120	2.3\\
121	2.3\\
122	2.3\\
123	2.3\\
124	2.2\\
125	2.2\\
126	2.2\\
127	2.2\\
128	2.2\\
129	2.2\\
130	2.2\\
131	2.2\\
132	2.2\\
133	2.2\\
134	2.1\\
135	2.1\\
136	2.2\\
137	2.2\\
138	2.2\\
139	2.2\\
140	2.2\\
141	2.2\\
142	2.2\\
143	1.2\\
144	1.2\\
145	1.2\\
146	1.2\\
147	3\\
148	3\\
149	3\\
150	3\\
151	2.4\\
152	2.4\\
153	2.5\\
154	2.5\\
155	2.5\\
156	2.5\\
157	2.5\\
158	2.5\\
159	2.5\\
160	2.5\\
161	2.5\\
162	2.6\\
163	2.6\\
164	2.6\\
165	2.6\\
166	2.6\\
167	2.6\\
168	2.6\\
169	2.6\\
170	2.6\\
171	2.6\\
172	2.6\\
173	2.6\\
174	2.6\\
175	2.6\\
176	2.6\\
177	2.6\\
178	2.6\\
179	2.6\\
180	2.6\\
181	2.6\\
182	2.6\\
183	2.6\\
184	2.6\\
185	2.6\\
186	2.6\\
187	2.6\\
188	2.6\\
189	2.6\\
190	2.6\\
191	2.6\\
192	2.6\\
193	2.6\\
194	2.6\\
195	2.6\\
196	2.6\\
197	2.6\\
198	2.6\\
199	2.6\\
200	2.6\\
201	2.6\\
202	2.6\\
203	2.6\\
204	2.6\\
205	2.6\\
206	2.6\\
207	2.6\\
208	2.6\\
209	2.6\\
210	2.6\\
211	2.7\\
212	2.7\\
213	2.7\\
214	2.7\\
215	2.7\\
216	2.7\\
217	2.7\\
218	2.7\\
219	2.7\\
220	2.7\\
221	2.7\\
222	2.7\\
223	2.7\\
224	2.7\\
225	2.7\\
226	2.7\\
227	2.7\\
228	2.7\\
229	2.7\\
230	2.7\\
231	2.7\\
232	2.7\\
233	2.7\\
234	2.7\\
235	2.7\\
236	2.7\\
237	2.8\\
238	2.8\\
239	2.8\\
240	2.8\\
241	2.8\\
242	2.8\\
243	2.8\\
244	2.9\\
245	2.9\\
246	2.9\\
247	2.9\\
248	2.9\\
249	2.9\\
250	2.9\\
251	2.9\\
252	2.9\\
253	2.9\\
254	2.9\\
255	2.9\\
256	2.9\\
257	2.9\\
258	2.9\\
259	2.9\\
260	3\\
261	3\\
262	3\\
263	3\\
264	3\\
265	3\\
266	3\\
267	3\\
268	3\\
269	3\\
270	3\\
271	3\\
272	3\\
273	3\\
274	3\\
275	3\\
276	3\\
277	3\\
278	3\\
279	3\\
280	3\\
281	3\\
282	3\\
283	3\\
284	3\\
285	3\\
286	3\\
287	3\\
288	3\\
289	3\\
290	3\\
291	3\\
292	3\\
293	3\\
294	3\\
295	3\\
296	3\\
297	3\\
298	3\\
299	3\\
300	3\\
301	3\\
302	3\\
303	3\\
304	3\\
305	3\\
306	3\\
307	3.1\\
308	3.1\\
309	3.1\\
310	3.1\\
311	3.1\\
312	3.1\\
313	3.1\\
314	3.1\\
315	3.1\\
316	3.1\\
317	3.1\\
318	3.1\\
319	3.1\\
320	3.1\\
321	3.1\\
322	3.1\\
323	3.1\\
324	3.1\\
325	3.1\\
326	3.1\\
327	3.1\\
328	3.1\\
329	3.1\\
330	3.1\\
331	3.1\\
332	3.1\\
333	3.1\\
334	3.2\\
335	3.2\\
336	3.2\\
337	3.2\\
338	3.2\\
339	3.2\\
340	3.2\\
341	3.2\\
342	3.2\\
343	3.2\\
344	3.2\\
345	3.2\\
346	3.2\\
347	3.2\\
348	3.2\\
349	3.1\\
350	3.1\\
351	3.1\\
352	3.2\\
353	3.2\\
354	3.2\\
355	3.2\\
356	3.2\\
357	3.2\\
358	3.2\\
359	3.2\\
360	3.2\\
361	3.2\\
362	3.2\\
363	3.2\\
364	3.2\\
365	3.2\\
366	3.2\\
367	3.2\\
368	3.2\\
369	3.3\\
370	3.3\\
371	3.4\\
372	3.4\\
373	3.4\\
374	3.4\\
375	3.4\\
376	3.4\\
377	3.4\\
378	3.4\\
379	3.4\\
380	3.4\\
381	3.4\\
382	3.2\\
383	3.4\\
384	3.4\\
385	3.4\\
386	3.2\\
387	3.2\\
388	3.2\\
389	3.2\\
390	3.2\\
391	3.2\\
392	3.2\\
393	3.2\\
394	3.2\\
395	3.2\\
396	3.2\\
397	3.2\\
398	3.2\\
399	3.2\\
400	3.2\\
401	3.2\\
402	3.2\\
403	3.2\\
404	3.2\\
405	3.2\\
406	3.2\\
407	3.2\\
408	3.2\\
409	3.2\\
410	3.2\\
411	3.2\\
412	3.4\\
413	3.4\\
414	3.4\\
415	3.5\\
416	3.5\\
417	3.5\\
418	3.5\\
419	3.5\\
420	3.5\\
421	3.5\\
422	3.5\\
423	3.5\\
424	3.5\\
425	3.5\\
426	3.5\\
427	3.5\\
428	3.5\\
429	3.5\\
430	3.5\\
431	3.5\\
432	3.5\\
433	3.6\\
434	3.5\\
435	3.5\\
436	3.5\\
437	3.5\\
438	3.5\\
439	3.5\\
440	3.5\\
441	3.5\\
442	3.5\\
443	3.5\\
444	3.5\\
445	3.6\\
446	3.6\\
447	3.6\\
448	3.6\\
449	3.6\\
450	3.6\\
451	3.6\\
452	3.6\\
453	3.6\\
454	3.6\\
455	3.6\\
456	3.6\\
457	3.6\\
458	3.6\\
459	3.6\\
460	3.6\\
461	3.6\\
462	3.6\\
463	3.6\\
464	3.6\\
465	3.6\\
466	3.6\\
467	3.6\\
468	3.6\\
469	3.6\\
470	3.6\\
471	3.6\\
472	3.6\\
473	3.6\\
474	3.6\\
475	3.6\\
476	3.9\\
477	3.9\\
478	3.9\\
479	3.9\\
480	3.9\\
481	3.9\\
482	3.9\\
483	3.9\\
484	3.9\\
485	3.9\\
486	3.9\\
487	3.9\\
488	3.9\\
489	3.9\\
490	3.9\\
491	3.9\\
492	3.9\\
493	3.9\\
494	3.9\\
495	3.9\\
496	3.9\\
};
\addplot [color=mycolor2, draw=none, mark=x, mark options={solid, mycolor2}, forget plot]
  table[row sep=crcr]{%
1	1.4\\
2	3.7\\
3	1.3\\
4	2.2\\
5	1.4\\
6	1.4\\
7	2.2\\
8	2.2\\
9	2.2\\
10	2.2\\
11	1.4\\
12	1.4\\
13	1.4\\
14	2.1\\
15	2.1\\
16	2\\
17	2\\
18	2.1\\
19	2\\
20	2\\
21	2\\
22	2\\
23	2\\
24	2\\
25	1.9\\
26	2.1\\
27	2.1\\
28	2.1\\
29	2.1\\
30	2.1\\
31	2.1\\
32	1.4\\
33	1.2\\
34	1.2\\
35	1.2\\
36	2.2\\
37	2.1\\
38	2.1\\
39	2.1\\
40	2.4\\
41	2.2\\
42	2.2\\
43	2.2\\
44	2.2\\
45	2.2\\
46	2.1\\
47	2.1\\
48	2.3\\
49	2.3\\
50	2.3\\
51	2.3\\
52	2.3\\
53	1.7\\
54	1.7\\
55	2.3\\
56	2.3\\
57	2.2\\
58	2.2\\
59	2.2\\
60	2.2\\
61	2.2\\
62	2.2\\
63	2.2\\
64	2.1\\
65	2.1\\
66	2.1\\
67	2.1\\
68	1.2\\
69	1.2\\
70	1.2\\
71	1.2\\
72	1.2\\
73	1.2\\
74	2.1\\
75	2\\
76	1.2\\
77	1.2\\
78	1.2\\
79	2.3\\
80	2\\
81	2\\
82	1.2\\
83	1.2\\
84	1.9\\
85	1.8\\
86	1.2\\
87	1.2\\
88	1.2\\
89	1.2\\
90	2.7\\
91	2.7\\
92	2.7\\
93	2.7\\
94	2.7\\
95	2.7\\
96	2.7\\
97	2.7\\
98	2.7\\
99	2.7\\
100	2.7\\
101	2.5\\
102	2.1\\
103	2.1\\
104	1.2\\
105	2.1\\
106	2.2\\
107	1.2\\
108	2.2\\
109	4.7\\
110	2.1\\
111	2.1\\
112	2.1\\
113	2.1\\
114	2.1\\
115	2.1\\
116	2.1\\
117	2.2\\
118	3.8\\
119	3.8\\
120	2.2\\
121	3.8\\
122	3.8\\
123	2.2\\
124	2.3\\
125	2.3\\
126	2.3\\
127	2.3\\
128	2.3\\
129	3.8\\
130	3.8\\
131	2.3\\
132	1.2\\
133	1.2\\
134	1.2\\
135	1.2\\
136	1.2\\
137	1.2\\
138	1.2\\
139	1.2\\
140	1.2\\
141	1.2\\
142	1.2\\
143	2.2\\
144	2.2\\
145	3\\
146	3\\
147	1.2\\
148	1.2\\
149	1.2\\
150	2.2\\
151	3\\
152	3\\
153	3\\
154	3\\
155	3\\
156	3\\
157	3\\
158	3\\
159	3\\
160	3\\
161	3\\
162	3\\
163	3\\
164	3\\
165	3\\
166	3\\
167	3\\
168	3\\
169	3\\
170	3\\
171	1.2\\
172	1.2\\
173	1.2\\
174	1.2\\
175	2.6\\
176	2.5\\
177	2.5\\
178	2.5\\
179	2.2\\
180	2.5\\
181	2.5\\
182	2.5\\
183	3\\
184	3\\
185	3\\
186	3\\
187	3\\
188	3\\
189	3\\
190	3\\
191	3\\
192	3\\
193	3\\
194	3\\
195	3\\
196	2.6\\
197	2.6\\
198	2.6\\
199	2.6\\
200	2.6\\
201	3\\
202	2.2\\
203	1.2\\
204	1.2\\
205	1.2\\
206	1.2\\
207	1.2\\
208	2.8\\
209	2.7\\
210	2.9\\
211	2.6\\
212	2.6\\
213	2.6\\
214	2.6\\
215	2.5\\
216	2.5\\
217	2.5\\
218	2.5\\
219	2.5\\
220	2.5\\
221	2.5\\
222	2.5\\
223	3.1\\
224	3.9\\
225	3.9\\
226	3.9\\
227	3.9\\
228	3.9\\
229	3.2\\
230	3.9\\
231	3.9\\
232	3.9\\
233	3.9\\
234	3.9\\
235	1.2\\
236	3.1\\
237	2.6\\
238	3.9\\
239	3.9\\
240	3.9\\
241	3.2\\
242	3.2\\
243	2.6\\
244	2.7\\
245	2.7\\
246	2.7\\
247	2.5\\
248	3\\
249	3\\
250	3\\
251	3\\
252	3\\
253	3\\
254	3\\
255	3\\
256	3\\
257	3\\
258	3\\
259	3\\
260	3\\
261	3\\
262	3\\
263	3\\
264	3\\
265	3\\
266	3\\
267	3\\
268	3\\
269	3\\
270	3\\
271	3\\
272	3\\
273	3\\
274	3\\
275	3\\
276	3\\
277	3\\
278	3\\
279	3\\
280	3\\
281	3\\
282	3\\
283	3\\
284	3\\
285	3\\
286	3.8\\
287	3.8\\
288	3.8\\
289	3.8\\
290	3.8\\
291	3.8\\
292	3.8\\
293	3.8\\
294	3.8\\
295	3.8\\
296	3.8\\
297	3.8\\
298	3.8\\
299	3.8\\
300	3.8\\
301	3.8\\
302	3.8\\
303	3.8\\
304	3.8\\
305	3.8\\
306	3.8\\
307	3.8\\
308	3.8\\
309	3.8\\
310	3.8\\
311	3.8\\
312	3.8\\
313	3.8\\
314	3.8\\
315	3.8\\
316	3.8\\
317	3.8\\
318	3.8\\
319	3.8\\
320	3.8\\
321	3.4\\
322	3.4\\
323	3.4\\
324	3.4\\
325	3.4\\
326	3.8\\
327	3.8\\
328	3.8\\
329	3.8\\
330	3.8\\
331	3.5\\
332	3.5\\
333	3.5\\
334	3.4\\
335	3.4\\
336	3.4\\
337	3.4\\
338	3.4\\
339	3.4\\
340	3.4\\
341	3.4\\
342	3.5\\
343	3.5\\
344	3.5\\
345	3.4\\
346	3.4\\
347	3.4\\
348	3.4\\
349	3.3\\
350	3.3\\
351	3.3\\
352	3.3\\
353	3.3\\
354	3.3\\
355	3.5\\
356	3.5\\
357	3.5\\
358	3.4\\
359	3.5\\
360	3.3\\
361	3.3\\
362	3.3\\
363	3.3\\
364	3.3\\
365	3.3\\
366	3.3\\
367	3.5\\
368	3.5\\
369	3\\
370	3\\
371	3\\
372	3\\
373	3\\
374	3\\
375	3\\
376	3.2\\
377	3.2\\
378	3.2\\
379	3.2\\
380	3.2\\
381	3.2\\
382	3.4\\
383	3.2\\
384	3.2\\
385	3.2\\
386	3.5\\
387	3.4\\
388	3.4\\
389	3.4\\
390	3.4\\
391	3.4\\
392	3.4\\
393	3.4\\
394	3.4\\
395	3.4\\
396	3.4\\
397	3.4\\
398	3.4\\
399	3.4\\
400	3.4\\
401	3.4\\
402	3.5\\
403	3.5\\
404	3.4\\
405	3.4\\
406	3.4\\
407	3.4\\
408	3.4\\
409	3.4\\
410	3.6\\
411	3.9\\
412	3.2\\
413	3.2\\
414	4\\
415	3.2\\
416	3.2\\
417	4.1\\
418	4.1\\
419	4.1\\
420	4.1\\
421	4.1\\
422	4.1\\
423	3.2\\
424	4.1\\
425	4.1\\
426	4.1\\
427	4.1\\
428	4.1\\
429	4.1\\
430	3.2\\
431	3.2\\
432	3.2\\
433	3.2\\
434	4.1\\
435	4.1\\
436	4.1\\
437	4.1\\
438	4.1\\
439	4.1\\
440	2.1\\
441	2.1\\
442	2.1\\
443	2.1\\
444	4.1\\
445	1.2\\
446	1.2\\
447	1.2\\
448	1.2\\
449	2.1\\
450	2.1\\
451	2.1\\
452	2.1\\
453	2.1\\
454	2.1\\
455	2.1\\
456	1.2\\
457	1.2\\
458	1.2\\
459	1.2\\
460	3.9\\
461	3.9\\
462	3.9\\
463	3.9\\
464	3.9\\
465	3.9\\
466	3.9\\
467	3.9\\
468	3.9\\
469	3.9\\
470	3.9\\
471	3.9\\
472	3.9\\
473	3.9\\
474	3.9\\
475	3.9\\
476	3.6\\
477	3.6\\
478	3.6\\
479	3.6\\
480	3.6\\
481	3.6\\
482	3.6\\
483	3.6\\
484	3.6\\
485	3.6\\
486	3.6\\
487	3.6\\
488	3.6\\
489	3.6\\
490	3.6\\
491	1.2\\
492	3.6\\
493	1.2\\
494	3.6\\
495	3.6\\
496	3.6\\
};
\addplot [color=mycolor3, dashed, forget plot]
  table[row sep=crcr]{%
1	2\\
2	2.0040404040404\\
3	2.00808080808081\\
4	2.01212121212121\\
5	2.01616161616162\\
6	2.02020202020202\\
7	2.02424242424242\\
8	2.02828282828283\\
9	2.03232323232323\\
10	2.03636363636364\\
11	2.04040404040404\\
12	2.04444444444444\\
13	2.04848484848485\\
14	2.05252525252525\\
15	2.05656565656566\\
16	2.06060606060606\\
17	2.06464646464646\\
18	2.06868686868687\\
19	2.07272727272727\\
20	2.07676767676768\\
21	2.08080808080808\\
22	2.08484848484848\\
23	2.08888888888889\\
24	2.09292929292929\\
25	2.0969696969697\\
26	2.1010101010101\\
27	2.1050505050505\\
28	2.10909090909091\\
29	2.11313131313131\\
30	2.11717171717172\\
31	2.12121212121212\\
32	2.12525252525253\\
33	2.12929292929293\\
34	2.13333333333333\\
35	2.13737373737374\\
36	2.14141414141414\\
37	2.14545454545455\\
38	2.14949494949495\\
39	2.15353535353535\\
40	2.15757575757576\\
41	2.16161616161616\\
42	2.16565656565657\\
43	2.16969696969697\\
44	2.17373737373737\\
45	2.17777777777778\\
46	2.18181818181818\\
47	2.18585858585859\\
48	2.18989898989899\\
49	2.19393939393939\\
50	2.1979797979798\\
51	2.2020202020202\\
52	2.20606060606061\\
53	2.21010101010101\\
54	2.21414141414141\\
55	2.21818181818182\\
56	2.22222222222222\\
57	2.22626262626263\\
58	2.23030303030303\\
59	2.23434343434343\\
60	2.23838383838384\\
61	2.24242424242424\\
62	2.24646464646465\\
63	2.25050505050505\\
64	2.25454545454545\\
65	2.25858585858586\\
66	2.26262626262626\\
67	2.26666666666667\\
68	2.27070707070707\\
69	2.27474747474747\\
70	2.27878787878788\\
71	2.28282828282828\\
72	2.28686868686869\\
73	2.29090909090909\\
74	2.2949494949495\\
75	2.2989898989899\\
76	2.3030303030303\\
77	2.30707070707071\\
78	2.31111111111111\\
79	2.31515151515151\\
80	2.31919191919192\\
81	2.32323232323232\\
82	2.32727272727273\\
83	2.33131313131313\\
84	2.33535353535354\\
85	2.33939393939394\\
86	2.34343434343434\\
87	2.34747474747475\\
88	2.35151515151515\\
89	2.35555555555556\\
90	2.35959595959596\\
91	2.36363636363636\\
92	2.36767676767677\\
93	2.37171717171717\\
94	2.37575757575758\\
95	2.37979797979798\\
96	2.38383838383838\\
97	2.38787878787879\\
98	2.39191919191919\\
99	2.3959595959596\\
100	2.4\\
101	2.4040404040404\\
102	2.40808080808081\\
103	2.41212121212121\\
104	2.41616161616162\\
105	2.42020202020202\\
106	2.42424242424242\\
107	2.42828282828283\\
108	2.43232323232323\\
109	2.43636363636364\\
110	2.44040404040404\\
111	2.44444444444444\\
112	2.44848484848485\\
113	2.45252525252525\\
114	2.45656565656566\\
115	2.46060606060606\\
116	2.46464646464646\\
117	2.46868686868687\\
118	2.47272727272727\\
119	2.47676767676768\\
120	2.48080808080808\\
121	2.48484848484848\\
122	2.48888888888889\\
123	2.49292929292929\\
124	2.4969696969697\\
125	2.5010101010101\\
126	2.50505050505051\\
127	2.50909090909091\\
128	2.51313131313131\\
129	2.51717171717172\\
130	2.52121212121212\\
131	2.52525252525253\\
132	2.52929292929293\\
133	2.53333333333333\\
134	2.53737373737374\\
135	2.54141414141414\\
136	2.54545454545455\\
137	2.54949494949495\\
138	2.55353535353535\\
139	2.55757575757576\\
140	2.56161616161616\\
141	2.56565656565657\\
142	2.56969696969697\\
143	2.57373737373737\\
144	2.57777777777778\\
145	2.58181818181818\\
146	2.58585858585859\\
147	2.58989898989899\\
148	2.59393939393939\\
149	2.5979797979798\\
150	2.6020202020202\\
151	2.60606060606061\\
152	2.61010101010101\\
153	2.61414141414141\\
154	2.61818181818182\\
155	2.62222222222222\\
156	2.62626262626263\\
157	2.63030303030303\\
158	2.63434343434343\\
159	2.63838383838384\\
160	2.64242424242424\\
161	2.64646464646465\\
162	2.65050505050505\\
163	2.65454545454545\\
164	2.65858585858586\\
165	2.66262626262626\\
166	2.66666666666667\\
167	2.67070707070707\\
168	2.67474747474747\\
169	2.67878787878788\\
170	2.68282828282828\\
171	2.68686868686869\\
172	2.69090909090909\\
173	2.69494949494949\\
174	2.6989898989899\\
175	2.7030303030303\\
176	2.70707070707071\\
177	2.71111111111111\\
178	2.71515151515152\\
179	2.71919191919192\\
180	2.72323232323232\\
181	2.72727272727273\\
182	2.73131313131313\\
183	2.73535353535354\\
184	2.73939393939394\\
185	2.74343434343434\\
186	2.74747474747475\\
187	2.75151515151515\\
188	2.75555555555556\\
189	2.75959595959596\\
190	2.76363636363636\\
191	2.76767676767677\\
192	2.77171717171717\\
193	2.77575757575758\\
194	2.77979797979798\\
195	2.78383838383838\\
196	2.78787878787879\\
197	2.79191919191919\\
198	2.7959595959596\\
199	2.8\\
200	2.8040404040404\\
201	2.80808080808081\\
202	2.81212121212121\\
203	2.81616161616162\\
204	2.82020202020202\\
205	2.82424242424242\\
206	2.82828282828283\\
207	2.83232323232323\\
208	2.83636363636364\\
209	2.84040404040404\\
210	2.84444444444444\\
211	2.84848484848485\\
212	2.85252525252525\\
213	2.85656565656566\\
214	2.86060606060606\\
215	2.86464646464646\\
216	2.86868686868687\\
217	2.87272727272727\\
218	2.87676767676768\\
219	2.88080808080808\\
220	2.88484848484848\\
221	2.88888888888889\\
222	2.89292929292929\\
223	2.8969696969697\\
224	2.9010101010101\\
225	2.90505050505051\\
226	2.90909090909091\\
227	2.91313131313131\\
228	2.91717171717172\\
229	2.92121212121212\\
230	2.92525252525253\\
231	2.92929292929293\\
232	2.93333333333333\\
233	2.93737373737374\\
234	2.94141414141414\\
235	2.94545454545455\\
236	2.94949494949495\\
237	2.95353535353535\\
238	2.95757575757576\\
239	2.96161616161616\\
240	2.96565656565657\\
241	2.96969696969697\\
242	2.97373737373737\\
243	2.97777777777778\\
244	2.98181818181818\\
245	2.98585858585859\\
246	2.98989898989899\\
247	2.99393939393939\\
248	2.9979797979798\\
249	3.0020202020202\\
250	3.00606060606061\\
251	3.01010101010101\\
252	3.01414141414141\\
253	3.01818181818182\\
254	3.02222222222222\\
255	3.02626262626263\\
256	3.03030303030303\\
257	3.03434343434343\\
258	3.03838383838384\\
259	3.04242424242424\\
260	3.04646464646465\\
261	3.05050505050505\\
262	3.05454545454545\\
263	3.05858585858586\\
264	3.06262626262626\\
265	3.06666666666667\\
266	3.07070707070707\\
267	3.07474747474747\\
268	3.07878787878788\\
269	3.08282828282828\\
270	3.08686868686869\\
271	3.09090909090909\\
272	3.09494949494949\\
273	3.0989898989899\\
274	3.1030303030303\\
275	3.10707070707071\\
276	3.11111111111111\\
277	3.11515151515152\\
278	3.11919191919192\\
279	3.12323232323232\\
280	3.12727272727273\\
281	3.13131313131313\\
282	3.13535353535354\\
283	3.13939393939394\\
284	3.14343434343434\\
285	3.14747474747475\\
286	3.15151515151515\\
287	3.15555555555556\\
288	3.15959595959596\\
289	3.16363636363636\\
290	3.16767676767677\\
291	3.17171717171717\\
292	3.17575757575758\\
293	3.17979797979798\\
294	3.18383838383838\\
295	3.18787878787879\\
296	3.19191919191919\\
297	3.1959595959596\\
298	3.2\\
299	3.2040404040404\\
300	3.20808080808081\\
301	3.21212121212121\\
302	3.21616161616162\\
303	3.22020202020202\\
304	3.22424242424242\\
305	3.22828282828283\\
306	3.23232323232323\\
307	3.23636363636364\\
308	3.24040404040404\\
309	3.24444444444444\\
310	3.24848484848485\\
311	3.25252525252525\\
312	3.25656565656566\\
313	3.26060606060606\\
314	3.26464646464646\\
315	3.26868686868687\\
316	3.27272727272727\\
317	3.27676767676768\\
318	3.28080808080808\\
319	3.28484848484849\\
320	3.28888888888889\\
321	3.29292929292929\\
322	3.2969696969697\\
323	3.3010101010101\\
324	3.30505050505051\\
325	3.30909090909091\\
326	3.31313131313131\\
327	3.31717171717172\\
328	3.32121212121212\\
329	3.32525252525253\\
330	3.32929292929293\\
331	3.33333333333333\\
332	3.33737373737374\\
333	3.34141414141414\\
334	3.34545454545455\\
335	3.34949494949495\\
336	3.35353535353535\\
337	3.35757575757576\\
338	3.36161616161616\\
339	3.36565656565657\\
340	3.36969696969697\\
341	3.37373737373737\\
342	3.37777777777778\\
343	3.38181818181818\\
344	3.38585858585859\\
345	3.38989898989899\\
346	3.39393939393939\\
347	3.3979797979798\\
348	3.4020202020202\\
349	3.40606060606061\\
350	3.41010101010101\\
351	3.41414141414141\\
352	3.41818181818182\\
353	3.42222222222222\\
354	3.42626262626263\\
355	3.43030303030303\\
356	3.43434343434343\\
357	3.43838383838384\\
358	3.44242424242424\\
359	3.44646464646465\\
360	3.45050505050505\\
361	3.45454545454545\\
362	3.45858585858586\\
363	3.46262626262626\\
364	3.46666666666667\\
365	3.47070707070707\\
366	3.47474747474747\\
367	3.47878787878788\\
368	3.48282828282828\\
369	3.48686868686869\\
370	3.49090909090909\\
371	3.49494949494949\\
372	3.4989898989899\\
373	3.5030303030303\\
374	3.50707070707071\\
375	3.51111111111111\\
376	3.51515151515152\\
377	3.51919191919192\\
378	3.52323232323232\\
379	3.52727272727273\\
380	3.53131313131313\\
381	3.53535353535354\\
382	3.53939393939394\\
383	3.54343434343434\\
384	3.54747474747475\\
385	3.55151515151515\\
386	3.55555555555556\\
387	3.55959595959596\\
388	3.56363636363636\\
389	3.56767676767677\\
390	3.57171717171717\\
391	3.57575757575758\\
392	3.57979797979798\\
393	3.58383838383838\\
394	3.58787878787879\\
395	3.59191919191919\\
396	3.5959595959596\\
397	3.6\\
398	3.6040404040404\\
399	3.60808080808081\\
400	3.61212121212121\\
401	3.61616161616162\\
402	3.62020202020202\\
403	3.62424242424242\\
404	3.62828282828283\\
405	3.63232323232323\\
406	3.63636363636364\\
407	3.64040404040404\\
408	3.64444444444444\\
409	3.64848484848485\\
410	3.65252525252525\\
411	3.65656565656566\\
412	3.66060606060606\\
413	3.66464646464646\\
414	3.66868686868687\\
415	3.67272727272727\\
416	3.67676767676768\\
417	3.68080808080808\\
418	3.68484848484848\\
419	3.68888888888889\\
420	3.69292929292929\\
421	3.6969696969697\\
422	3.7010101010101\\
423	3.70505050505051\\
424	3.70909090909091\\
425	3.71313131313131\\
426	3.71717171717172\\
427	3.72121212121212\\
428	3.72525252525253\\
429	3.72929292929293\\
430	3.73333333333333\\
431	3.73737373737374\\
432	3.74141414141414\\
433	3.74545454545455\\
434	3.74949494949495\\
435	3.75353535353535\\
436	3.75757575757576\\
437	3.76161616161616\\
438	3.76565656565657\\
439	3.76969696969697\\
440	3.77373737373737\\
441	3.77777777777778\\
442	3.78181818181818\\
443	3.78585858585859\\
444	3.78989898989899\\
445	3.79393939393939\\
446	3.7979797979798\\
447	3.8020202020202\\
448	3.80606060606061\\
449	3.81010101010101\\
450	3.81414141414141\\
451	3.81818181818182\\
452	3.82222222222222\\
453	3.82626262626263\\
454	3.83030303030303\\
455	3.83434343434343\\
456	3.83838383838384\\
457	3.84242424242424\\
458	3.84646464646465\\
459	3.85050505050505\\
460	3.85454545454545\\
461	3.85858585858586\\
462	3.86262626262626\\
463	3.86666666666667\\
464	3.87070707070707\\
465	3.87474747474747\\
466	3.87878787878788\\
467	3.88282828282828\\
468	3.88686868686869\\
469	3.89090909090909\\
470	3.8949494949495\\
471	3.8989898989899\\
472	3.9030303030303\\
473	3.90707070707071\\
474	3.91111111111111\\
475	3.91515151515152\\
476	3.91919191919192\\
477	3.92323232323232\\
478	3.92727272727273\\
479	3.93131313131313\\
480	3.93535353535354\\
481	3.93939393939394\\
482	3.94343434343434\\
483	3.94747474747475\\
484	3.95151515151515\\
485	3.95555555555556\\
486	3.95959595959596\\
487	3.96363636363636\\
488	3.96767676767677\\
489	3.97171717171717\\
490	3.97575757575758\\
491	3.97979797979798\\
492	3.98383838383838\\
493	3.98787878787879\\
494	3.99191919191919\\
495	3.9959595959596\\
496	4\\
};
\addplot [color=mycolor4, dashed, forget plot]
  table[row sep=crcr]{%
1	2\\
2	2.0040404040404\\
3	2.00808080808081\\
4	2.01212121212121\\
5	2.01616161616162\\
6	2.02020202020202\\
7	2.02424242424242\\
8	2.02828282828283\\
9	2.03232323232323\\
10	2.03636363636364\\
11	2.04040404040404\\
12	2.04444444444444\\
13	2.04848484848485\\
14	2.05252525252525\\
15	2.05656565656566\\
16	2.06060606060606\\
17	2.06464646464646\\
18	2.06868686868687\\
19	2.07272727272727\\
20	2.07676767676768\\
21	2.08080808080808\\
22	2.08484848484848\\
23	2.08888888888889\\
24	2.09292929292929\\
25	2.0969696969697\\
26	2.1010101010101\\
27	2.1050505050505\\
28	2.10909090909091\\
29	2.11313131313131\\
30	2.11717171717172\\
31	2.12121212121212\\
32	2.12525252525253\\
33	2.12929292929293\\
34	2.13333333333333\\
35	2.13737373737374\\
36	2.14141414141414\\
37	2.14545454545455\\
38	2.14949494949495\\
39	2.15353535353535\\
40	2.15757575757576\\
41	2.16161616161616\\
42	2.16565656565657\\
43	2.16969696969697\\
44	2.17373737373737\\
45	2.17777777777778\\
46	2.18181818181818\\
47	2.18585858585859\\
48	2.18989898989899\\
49	2.19393939393939\\
50	2.1979797979798\\
51	2.2020202020202\\
52	2.20606060606061\\
53	2.21010101010101\\
54	2.21414141414141\\
55	2.21818181818182\\
56	2.22222222222222\\
57	2.22626262626263\\
58	2.23030303030303\\
59	2.23434343434343\\
60	2.23838383838384\\
61	2.24242424242424\\
62	2.24646464646465\\
63	2.25050505050505\\
64	2.25454545454545\\
65	2.25858585858586\\
66	2.26262626262626\\
67	2.26666666666667\\
68	2.27070707070707\\
69	2.27474747474747\\
70	2.27878787878788\\
71	2.28282828282828\\
72	2.28686868686869\\
73	2.29090909090909\\
74	2.2949494949495\\
75	2.2989898989899\\
76	2.3030303030303\\
77	2.30707070707071\\
78	2.31111111111111\\
79	2.31515151515151\\
80	2.31919191919192\\
81	2.32323232323232\\
82	2.32727272727273\\
83	2.33131313131313\\
84	2.33535353535354\\
85	2.33939393939394\\
86	2.34343434343434\\
87	2.34747474747475\\
88	2.35151515151515\\
89	2.35555555555556\\
90	2.35959595959596\\
91	2.36363636363636\\
92	2.36767676767677\\
93	2.37171717171717\\
94	2.37575757575758\\
95	2.37979797979798\\
96	2.38383838383838\\
97	2.38787878787879\\
98	2.39191919191919\\
99	2.3959595959596\\
100	2.4\\
101	2.4040404040404\\
102	2.40808080808081\\
103	2.41212121212121\\
104	2.41616161616162\\
105	2.42020202020202\\
106	2.42424242424242\\
107	2.42828282828283\\
108	2.43232323232323\\
109	2.43636363636364\\
110	2.44040404040404\\
111	2.44444444444444\\
112	2.44848484848485\\
113	2.45252525252525\\
114	2.45656565656566\\
115	2.46060606060606\\
116	2.46464646464646\\
117	2.46868686868687\\
118	2.47272727272727\\
119	2.47676767676768\\
120	2.48080808080808\\
121	2.48484848484848\\
122	2.48888888888889\\
123	2.49292929292929\\
124	2.4969696969697\\
125	2.5010101010101\\
126	2.50505050505051\\
127	2.50909090909091\\
128	2.51313131313131\\
129	2.51717171717172\\
130	2.52121212121212\\
131	2.52525252525253\\
132	2.52929292929293\\
133	2.53333333333333\\
134	2.53737373737374\\
135	2.54141414141414\\
136	2.54545454545455\\
137	2.54949494949495\\
138	2.55353535353535\\
139	2.55757575757576\\
140	2.56161616161616\\
141	2.56565656565657\\
142	2.56969696969697\\
143	2.57373737373737\\
144	2.57777777777778\\
145	2.58181818181818\\
146	2.58585858585859\\
147	2.58989898989899\\
148	2.59393939393939\\
149	2.5979797979798\\
150	2.6020202020202\\
151	2.60606060606061\\
152	2.61010101010101\\
153	2.61414141414141\\
154	2.61818181818182\\
155	2.62222222222222\\
156	2.62626262626263\\
157	2.63030303030303\\
158	2.63434343434343\\
159	2.63838383838384\\
160	2.64242424242424\\
161	2.64646464646465\\
162	2.65050505050505\\
163	2.65454545454545\\
164	2.65858585858586\\
165	2.66262626262626\\
166	2.66666666666667\\
167	2.67070707070707\\
168	2.67474747474747\\
169	2.67878787878788\\
170	2.68282828282828\\
171	2.68686868686869\\
172	2.69090909090909\\
173	2.69494949494949\\
174	2.6989898989899\\
175	2.7030303030303\\
176	2.70707070707071\\
177	2.71111111111111\\
178	2.71515151515152\\
179	2.71919191919192\\
180	2.72323232323232\\
181	2.72727272727273\\
182	2.73131313131313\\
183	2.73535353535354\\
184	2.73939393939394\\
185	2.74343434343434\\
186	2.74747474747475\\
187	2.75151515151515\\
188	2.75555555555556\\
189	2.75959595959596\\
190	2.76363636363636\\
191	2.76767676767677\\
192	2.77171717171717\\
193	2.77575757575758\\
194	2.77979797979798\\
195	2.78383838383838\\
196	2.78787878787879\\
197	2.79191919191919\\
198	2.7959595959596\\
199	2.8\\
200	2.8040404040404\\
201	2.80808080808081\\
202	2.81212121212121\\
203	2.81616161616162\\
204	2.82020202020202\\
205	2.82424242424242\\
206	2.82828282828283\\
207	2.83232323232323\\
208	2.83636363636364\\
209	2.84040404040404\\
210	2.84444444444444\\
211	2.84848484848485\\
212	2.85252525252525\\
213	2.85656565656566\\
214	2.86060606060606\\
215	2.86464646464646\\
216	2.86868686868687\\
217	2.87272727272727\\
218	2.87676767676768\\
219	2.88080808080808\\
220	2.88484848484848\\
221	2.88888888888889\\
222	2.89292929292929\\
223	2.8969696969697\\
224	2.9010101010101\\
225	2.90505050505051\\
226	2.90909090909091\\
227	2.91313131313131\\
228	2.91717171717172\\
229	2.92121212121212\\
230	2.92525252525253\\
231	2.92929292929293\\
232	2.93333333333333\\
233	2.93737373737374\\
234	2.94141414141414\\
235	2.94545454545455\\
236	2.94949494949495\\
237	2.95353535353535\\
238	2.95757575757576\\
239	2.96161616161616\\
240	2.96565656565657\\
241	2.96969696969697\\
242	2.97373737373737\\
243	2.97777777777778\\
244	2.98181818181818\\
245	2.98585858585859\\
246	2.98989898989899\\
247	2.99393939393939\\
248	2.9979797979798\\
249	3.0020202020202\\
250	3.00606060606061\\
251	3.01010101010101\\
252	3.01414141414141\\
253	3.01818181818182\\
254	3.02222222222222\\
255	3.02626262626263\\
256	3.03030303030303\\
257	3.03434343434343\\
258	3.03838383838384\\
259	3.04242424242424\\
260	3.04646464646465\\
261	3.05050505050505\\
262	3.05454545454545\\
263	3.05858585858586\\
264	3.06262626262626\\
265	3.06666666666667\\
266	3.07070707070707\\
267	3.07474747474747\\
268	3.07878787878788\\
269	3.08282828282828\\
270	3.08686868686869\\
271	3.09090909090909\\
272	3.09494949494949\\
273	3.0989898989899\\
274	3.1030303030303\\
275	3.10707070707071\\
276	3.11111111111111\\
277	3.11515151515152\\
278	3.11919191919192\\
279	3.12323232323232\\
280	3.12727272727273\\
281	3.13131313131313\\
282	3.13535353535354\\
283	3.13939393939394\\
284	3.14343434343434\\
285	3.14747474747475\\
286	3.15151515151515\\
287	3.15555555555556\\
288	3.15959595959596\\
289	3.16363636363636\\
290	3.16767676767677\\
291	3.17171717171717\\
292	3.17575757575758\\
293	3.17979797979798\\
294	3.18383838383838\\
295	3.18787878787879\\
296	3.19191919191919\\
297	3.1959595959596\\
298	3.2\\
299	3.2040404040404\\
300	3.20808080808081\\
301	3.21212121212121\\
302	3.21616161616162\\
303	3.22020202020202\\
304	3.22424242424242\\
305	3.22828282828283\\
306	3.23232323232323\\
307	3.23636363636364\\
308	3.24040404040404\\
309	3.24444444444444\\
310	3.24848484848485\\
311	3.25252525252525\\
312	3.25656565656566\\
313	3.26060606060606\\
314	3.26464646464646\\
315	3.26868686868687\\
316	3.27272727272727\\
317	3.27676767676768\\
318	3.28080808080808\\
319	3.28484848484849\\
320	3.28888888888889\\
321	3.29292929292929\\
322	3.2969696969697\\
323	3.3010101010101\\
324	3.30505050505051\\
325	3.30909090909091\\
326	3.31313131313131\\
327	3.31717171717172\\
328	3.32121212121212\\
329	3.32525252525253\\
330	3.32929292929293\\
331	3.33333333333333\\
332	3.33737373737374\\
333	3.34141414141414\\
334	3.34545454545455\\
335	3.34949494949495\\
336	3.35353535353535\\
337	3.35757575757576\\
338	3.36161616161616\\
339	3.36565656565657\\
340	3.36969696969697\\
341	3.37373737373737\\
342	3.37777777777778\\
343	3.38181818181818\\
344	3.38585858585859\\
345	3.38989898989899\\
346	3.39393939393939\\
347	3.3979797979798\\
348	3.4020202020202\\
349	3.40606060606061\\
350	3.41010101010101\\
351	3.41414141414141\\
352	3.41818181818182\\
353	3.42222222222222\\
354	3.42626262626263\\
355	3.43030303030303\\
356	3.43434343434343\\
357	3.43838383838384\\
358	3.44242424242424\\
359	3.44646464646465\\
360	3.45050505050505\\
361	3.45454545454545\\
362	3.45858585858586\\
363	3.46262626262626\\
364	3.46666666666667\\
365	3.47070707070707\\
366	3.47474747474747\\
367	3.47878787878788\\
368	3.48282828282828\\
369	3.48686868686869\\
370	3.49090909090909\\
371	3.49494949494949\\
372	3.4989898989899\\
373	3.5030303030303\\
374	3.50707070707071\\
375	3.51111111111111\\
376	3.51515151515152\\
377	3.51919191919192\\
378	3.52323232323232\\
379	3.52727272727273\\
380	3.53131313131313\\
381	3.53535353535354\\
382	3.53939393939394\\
383	3.54343434343434\\
384	3.54747474747475\\
385	3.55151515151515\\
386	3.55555555555556\\
387	3.55959595959596\\
388	3.56363636363636\\
389	3.56767676767677\\
390	3.57171717171717\\
391	3.57575757575758\\
392	3.57979797979798\\
393	3.58383838383838\\
394	3.58787878787879\\
395	3.59191919191919\\
396	3.5959595959596\\
397	3.6\\
398	3.6040404040404\\
399	3.60808080808081\\
400	3.61212121212121\\
401	3.61616161616162\\
402	3.62020202020202\\
403	3.62424242424242\\
404	3.62828282828283\\
405	3.63232323232323\\
406	3.63636363636364\\
407	3.64040404040404\\
408	3.64444444444444\\
409	3.64848484848485\\
410	3.65252525252525\\
411	3.65656565656566\\
412	3.66060606060606\\
413	3.66464646464646\\
414	3.66868686868687\\
415	3.67272727272727\\
416	3.67676767676768\\
417	3.68080808080808\\
418	3.68484848484848\\
419	3.68888888888889\\
420	3.69292929292929\\
421	3.6969696969697\\
422	3.7010101010101\\
423	3.70505050505051\\
424	3.70909090909091\\
425	3.71313131313131\\
426	3.71717171717172\\
427	3.72121212121212\\
428	3.72525252525253\\
429	3.72929292929293\\
430	3.73333333333333\\
431	3.73737373737374\\
432	3.74141414141414\\
433	3.74545454545455\\
434	3.74949494949495\\
435	3.75353535353535\\
436	3.75757575757576\\
437	3.76161616161616\\
438	3.76565656565657\\
439	3.76969696969697\\
440	3.77373737373737\\
441	3.77777777777778\\
442	3.78181818181818\\
443	3.78585858585859\\
444	3.78989898989899\\
445	3.79393939393939\\
446	3.7979797979798\\
447	3.8020202020202\\
448	3.80606060606061\\
449	3.81010101010101\\
450	3.81414141414141\\
451	3.81818181818182\\
452	3.82222222222222\\
453	3.82626262626263\\
454	3.83030303030303\\
455	3.83434343434343\\
456	3.83838383838384\\
457	3.84242424242424\\
458	3.84646464646465\\
459	3.85050505050505\\
460	3.85454545454545\\
461	3.85858585858586\\
462	3.86262626262626\\
463	3.86666666666667\\
464	3.87070707070707\\
465	3.87474747474747\\
466	3.87878787878788\\
467	3.88282828282828\\
468	3.88686868686869\\
469	3.89090909090909\\
470	3.8949494949495\\
471	3.8989898989899\\
472	3.9030303030303\\
473	3.90707070707071\\
474	3.91111111111111\\
475	3.91515151515152\\
476	3.91919191919192\\
477	3.92323232323232\\
478	3.92727272727273\\
479	3.93131313131313\\
480	3.93535353535354\\
481	3.93939393939394\\
482	3.94343434343434\\
483	3.94747474747475\\
484	3.95151515151515\\
485	3.95555555555556\\
486	3.95959595959596\\
487	3.96363636363636\\
488	3.96767676767677\\
489	3.97171717171717\\
490	3.97575757575758\\
491	3.97979797979798\\
492	3.98383838383838\\
493	3.98787878787879\\
494	3.99191919191919\\
495	3.9959595959596\\
496	4\\
};
\end{axis}
\end{tikzpicture}%  % tikz
             \caption{Estimated y-Axis Positions}	\end{subfigure}
    }
        \caption[Crossing Movement Results for TREM]{Crossing Movement Results for TREM (\Tsixty$=0.4$s).}
        \label{fig:trackingCrossingTREM}
    \end{figure}

As with the parallel movement scenario, both algorithms give erroneous estimates between $t=1$ and $t=2$ and around $t=3$, due to the inactivity of the first source. In \autoref{fig:trackingCrossingCREM} we can see, that the location estimates seem to have collapsed onto the second source between $t=2.5$ and $t=4.5$, as only two location estimates are close to the first source during this time. Here, \gls{trem} performs better, as it estimated more positions close to the first source in this time frame. The \gls{trem} algorithm also more closely tracks the second source in the first second, while \gls{crem} seems to be stuck in a local optima at $\p_x=3.9$ for almost the entire second. An overview of the position estimates over time for both \gls{trem} and \gls{crem} for the crossing movement scenario is provided in \autoref{fig:trackingCrossingRoom}.
\begin{figure}[!htbp]
\iftoggle{quick}{%
    \includegraphics[width=\textwidth]{plots/tracking/crossing/results-T60=0.4-cremtrem-room-sc.png}
}{%
	\begin{subfigure}{0.49\textwidth}
	     \centering
        \setlength{\figurewidth}{0.8\textwidth}
        % This file was created by matlab2tikz.
%
\definecolor{lms_red}{rgb}{0.80000,0.20780,0.21960}%
%
\begin{tikzpicture}

\begin{axis}[%
width=0.951\figurewidth,
height=\figureheight,
at={(0\figurewidth,0\figureheight)},
scale only axis,
xmin=1,
xmax=5,
xlabel style={font=\color{white!15!black}},
xlabel={$p_x^{(t)}$~[m]},
ymin=1,
ymax=5,
ylabel style={font=\color{white!15!black}},
ylabel={$p_y^{(t)}$~[m]},
axis background/.style={fill=white},
xmajorgrids,
ymajorgrids,
point meta min = 0,
point meta max = 5,
colorbar horizontal, 
colorbar style={
    at={(0.5,1.03)},anchor=south,
    colormap/jet,
    samples = 25,
    xticklabel pos=upper,
    xtick style={draw=none},
    title style={yshift=0.4cm},
    title=t
},
]

\addplot[
    scatter,%
    scatter/@pre marker code/.code={%
        \edef\temp{\noexpand\definecolor{mapped color}{rgb}{\pgfplotspointmeta}}%
        \temp
        \scope[draw=mapped color!80!black,fill=mapped color]%
    },%
    scatter/@post marker code/.code={%
        \endscope
    },%
    only marks,     
    mark=*,
    mark size=\trajSize,
    opacity=\trajOpacity,
    line width=\trajLinewidth,
    point meta={TeX code symbolic={%
        \edef\pgfplotspointmeta{\thisrow{R},\thisrow{G},\thisrow{B}}%
    }},
] 
table[row sep=crcr]{
x	y	R	G	B\\
2.000000	2.000000	0.000000	0.000000	0.600000\\
2.100200	2.100200	0.000000	0.000000	0.800000\\
2.200401	2.200401	0.000000	0.000000	1.000000\\
2.300601	2.300601	0.000000	0.200000	1.000000\\
2.400802	2.400802	0.000000	0.400000	1.000000\\
2.501002	2.501002	0.000000	0.600000	1.000000\\
2.601202	2.601202	0.000000	0.800000	1.000000\\
2.701403	2.701403	0.000000	1.000000	1.000000\\
2.801603	2.801603	0.200000	1.000000	0.800000\\
2.901804	2.901804	0.400000	1.000000	0.600000\\
3.002004	3.002004	0.600000	1.000000	0.400000\\
3.102204	3.102204	0.800000	1.000000	0.200000\\
3.202405	3.202405	1.000000	1.000000	0.000000\\
3.302605	3.302605	1.000000	0.800000	0.000000\\
3.402806	3.402806	1.000000	0.600000	0.000000\\
3.503006	3.503006	1.000000	0.400000	0.000000\\
3.603206	3.603206	1.000000	0.200000	0.000000\\
3.703407	3.703407	1.000000	0.000000	0.000000\\
3.803607	3.803607	0.800000	0.000000	0.000000\\
3.903808	3.903808	0.600000	0.000000	0.000000\\
4.000000	2.000000	0.000000	0.000000	0.600000\\
3.899800	2.100200	0.000000	0.000000	0.800000\\
3.799599	2.200401	0.000000	0.000000	1.000000\\
3.699399	2.300601	0.000000	0.200000	1.000000\\
3.599198	2.400802	0.000000	0.400000	1.000000\\
3.498998	2.501002	0.000000	0.600000	1.000000\\
3.398798	2.601202	0.000000	0.800000	1.000000\\
3.298597	2.701403	0.000000	1.000000	1.000000\\
3.198397	2.801603	0.200000	1.000000	0.800000\\
3.098196	2.901804	0.400000	1.000000	0.600000\\
2.997996	3.002004	0.600000	1.000000	0.400000\\
2.897796	3.102204	0.800000	1.000000	0.200000\\
2.797595	3.202405	1.000000	1.000000	0.000000\\
2.697395	3.302605	1.000000	0.800000	0.000000\\
2.597194	3.402806	1.000000	0.600000	0.000000\\
2.496994	3.503006	1.000000	0.400000	0.000000\\
2.396794	3.603206	1.000000	0.200000	0.000000\\
2.296593	3.703407	1.000000	0.000000	0.000000\\
2.196393	3.803607	0.800000	0.000000	0.000000\\
2.096192	3.903808	0.600000	0.000000	0.000000\\
};

\node at (axis cs:4,4) [anchor=south west] {$s_1$};
\node at (axis cs:2,4) [anchor=south west] {$s_2$};

\addplot[
    scatter,%
    scatter/@pre marker code/.code={%
        \edef\temp{\noexpand\definecolor{mapped color}{rgb}{\pgfplotspointmeta}}%
        \temp
        \scope[draw=mapped color!80!black,fill=mapped color]%
    },%
    scatter/@post marker code/.code={%
        \endscope
    },%
    only marks,     
    mark=x,
    opacity=\estOpacity,
    mark size=\estSize,
    line width=\estLinewidth,
    point meta={TeX code symbolic={%
        \edef\pgfplotspointmeta{\thisrow{R},\thisrow{G},\thisrow{B}}%
    }},
] 
table[row sep=crcr]{%
x	y	R	G	B\\
2.5	1.8	0	0	0.508064516129032\\
2	2	0	0	0.516129032258065\\
1.9	2	0	0	0.524193548387097\\
2	2	0	0	0.532258064516129\\
2	2	0	0	0.540322580645161\\
2.1	2	0	0	0.548387096774194\\
2.1	2	0	0	0.556451612903226\\
2.1	2	0	0	0.564516129032258\\
2.1	2	0	0	0.57258064516129\\
2.1	2	0	0	0.580645161290323\\
2.1	2	0	0	0.588709677419355\\
2.1	2	0	0	0.596774193548387\\
2.1	2	0	0	0.604838709677419\\
2.1	2	0	0	0.612903225806452\\
2.1	2	0	0	0.620967741935484\\
2.1	2	0	0	0.629032258064516\\
2.1	2	0	0	0.637096774193548\\
4	2.1	0	0	0.645161290322581\\
4	2.1	0	0	0.653225806451613\\
4	2.1	0	0	0.661290322580645\\
4	2.1	0	0	0.669354838709677\\
4	2.1	0	0	0.67741935483871\\
4	2.1	0	0	0.685483870967742\\
4	2.1	0	0	0.693548387096774\\
4	2.1	0	0	0.701612903225806\\
3.9	2.1	0	0	0.709677419354839\\
2.1	2	0	0	0.717741935483871\\
2.1	2	0	0	0.725806451612903\\
2.1	2	0	0	0.733870967741935\\
2.1	2	0	0	0.741935483870968\\
2.1	2	0	0	0.75\\
2.1	2	0	0	0.758064516129032\\
2.1	1.9	0	0	0.766129032258065\\
2.1	1.9	0	0	0.774193548387097\\
2.1	2	0	0	0.782258064516129\\
2.1	2	0	0	0.790322580645161\\
3.9	2.2	0	0	0.798387096774194\\
3.9	2.2	0	0	0.806451612903226\\
3.9	2.2	0	0	0.814516129032258\\
3.9	2.2	0	0	0.82258064516129\\
3.9	2.2	0	0	0.830645161290323\\
3.9	2.2	0	0	0.838709677419355\\
3.9	2.2	0	0	0.846774193548387\\
3.9	2.2	0	0	0.854838709677419\\
3.9	2.2	0	0	0.862903225806452\\
3.9	2.2	0	0	0.870967741935484\\
3.9	2.2	0	0	0.879032258064516\\
3.9	2.2	0	0	0.887096774193548\\
3.9	2.2	0	0	0.895161290322581\\
3.9	2.1	0	0	0.903225806451613\\
3.9	2.1	0	0	0.911290322580645\\
3.9	2.1	0	0	0.919354838709677\\
3.9	2.1	0	0	0.92741935483871\\
3.9	2.1	0	0	0.935483870967742\\
3.9	2.1	0	0	0.943548387096774\\
3.9	2.1	0	0	0.951612903225806\\
3.9	2.1	0	0	0.959677419354839\\
3.9	2.1	0	0	0.967741935483871\\
3.8	2.1	0	0	0.975806451612903\\
3.8	2.1	0	0	0.983870967741935\\
3.8	2.1	0	0	0.991935483870968\\
3.9	2.1	0	0	1\\
3.9	2.1	0	0.00806451612903226	1\\
3.9	2.1	0	0.0161290322580645	1\\
3.9	2.1	0	0.0241935483870968	1\\
3.9	2.1	0	0.032258064516129	1\\
3.9	2.1	0	0.0403225806451613	1\\
2.2	2.1	0	0.0483870967741935	1\\
2.2	2.1	0	0.0564516129032258	1\\
2.2	2.1	0	0.0645161290322581	1\\
2.2	2.1	0	0.0725806451612903	1\\
2.2	2.1	0	0.0806451612903226	1\\
2.2	2.2	0	0.0887096774193548	1\\
2.2	2.1	0	0.0967741935483871	1\\
2.2	2.1	0	0.104838709677419	1\\
2.2	2.1	0	0.112903225806452	1\\
3.9	2.2	0	0.120967741935484	1\\
3.9	2.2	0	0.129032258064516	1\\
3.9	2.2	0	0.137096774193548	1\\
3.9	2.2	0	0.145161290322581	1\\
3.9	2.2	0	0.153225806451613	1\\
3.8	2.2	0	0.161290322580645	1\\
3.7	2.3	0	0.169354838709677	1\\
3.7	2.3	0	0.17741935483871	1\\
3.6	2.3	0	0.185483870967742	1\\
3.6	2.3	0	0.193548387096774	1\\
3.6	2.3	0	0.201612903225806	1\\
3.6	2.3	0	0.209677419354839	1\\
3.6	2.3	0	0.217741935483871	1\\
3.6	2.3	0	0.225806451612903	1\\
3.6	2.3	0	0.233870967741935	1\\
3.6	2.3	0	0.241935483870968	1\\
3.6	2.3	0	0.25	1\\
3.6	2.3	0	0.258064516129032	1\\
3.6	2.3	0	0.266129032258065	1\\
3.6	2.3	0	0.274193548387097	1\\
3.6	2.3	0	0.282258064516129	1\\
3.6	2.3	0	0.290322580645161	1\\
3.6	2.3	0	0.298387096774194	1\\
3.6	2.3	0	0.306451612903226	1\\
3.6	2.3	0	0.314516129032258	1\\
3.6	2.3	0	0.32258064516129	1\\
3.6	2.3	0	0.330645161290323	1\\
3.6	2.3	0	0.338709677419355	1\\
3.6	2.3	0	0.346774193548387	1\\
3.6	2.3	0	0.354838709677419	1\\
3.6	2.3	0	0.362903225806452	1\\
3.6	2.3	0	0.370967741935484	1\\
3.6	2.3	0	0.379032258064516	1\\
3.6	2.3	0	0.387096774193548	1\\
3.6	2.3	0	0.395161290322581	1\\
3.6	2.3	0	0.403225806451613	1\\
3.6	2.3	0	0.411290322580645	1\\
3.6	2.3	0	0.419354838709677	1\\
3.6	2.3	0	0.42741935483871	1\\
3.6	2.3	0	0.435483870967742	1\\
3.6	2.3	0	0.443548387096774	1\\
3.6	2.3	0	0.451612903225806	1\\
1.2	3.8	0	0.459677419354839	1\\
1.2	3.8	0	0.467741935483871	1\\
1.2	3.8	0	0.475806451612903	1\\
1.2	3.8	0	0.483870967741935	1\\
1.2	3.8	0	0.491935483870968	1\\
3.6	2.3	0	0.5	1\\
3.6	2.3	0	0.508064516129032	1\\
3.6	2.3	0	0.516129032258065	1\\
3.6	2.3	0	0.524193548387097	1\\
1.2	2.2	0	0.532258064516129	1\\
3.6	2.3	0	0.540322580645161	1\\
3.6	2.3	0	0.548387096774194	1\\
1.2	2.2	0	0.556451612903226	1\\
1.2	2.2	0	0.564516129032258	1\\
1.2	2.2	0	0.57258064516129	1\\
1.2	2.2	0	0.580645161290323	1\\
1.2	2.2	0	0.588709677419355	1\\
1.2	2.2	0	0.596774193548387	1\\
1.2	2.2	0	0.604838709677419	1\\
1.2	2.2	0	0.612903225806452	1\\
1.2	2.2	0	0.620967741935484	1\\
1.2	2.2	0	0.629032258064516	1\\
1.2	2.2	0	0.637096774193548	1\\
2.8	1.2	0	0.645161290322581	1\\
2.8	1.2	0	0.653225806451613	1\\
2.8	1.2	0	0.661290322580645	1\\
2.8	1.2	0	0.669354838709677	1\\
2.8	1.2	0	0.67741935483871	1\\
2.8	1.2	0	0.685483870967742	1\\
2.8	1.2	0	0.693548387096774	1\\
2.8	1.2	0	0.701612903225806	1\\
3.6	2.4	0	0.709677419354839	1\\
3.6	2.4	0	0.717741935483871	1\\
3.6	2.4	0	0.725806451612903	1\\
3.5	2.4	0	0.733870967741935	1\\
3.5	2.4	0	0.741935483870968	1\\
3.5	2.5	0	0.75	1\\
3.5	2.5	0	0.758064516129032	1\\
3.5	2.5	0	0.766129032258065	1\\
3.5	2.5	0	0.774193548387097	1\\
3.5	2.5	0	0.782258064516129	1\\
3.4	2.5	0	0.790322580645161	1\\
3.4	2.5	0	0.798387096774194	1\\
3.4	2.5	0	0.806451612903226	1\\
3.4	2.5	0	0.814516129032258	1\\
3.4	2.5	0	0.82258064516129	1\\
3.4	2.5	0	0.830645161290323	1\\
3.4	2.5	0	0.838709677419355	1\\
3.4	2.5	0	0.846774193548387	1\\
3.4	2.5	0	0.854838709677419	1\\
3.4	2.5	0	0.862903225806452	1\\
3.4	2.5	0	0.870967741935484	1\\
3.4	2.5	0	0.879032258064516	1\\
3.4	2.5	0	0.887096774193548	1\\
3.4	2.5	0	0.895161290322581	1\\
3.3	2.6	0	0.903225806451613	1\\
3.3	2.6	0	0.911290322580645	1\\
3.3	2.6	0	0.919354838709677	1\\
3.3	2.6	0	0.92741935483871	1\\
3.3	2.6	0	0.935483870967742	1\\
3.3	2.6	0	0.943548387096774	1\\
3.3	2.6	0	0.951612903225806	1\\
3.3	2.6	0	0.959677419354839	1\\
3.3	2.6	0	0.967741935483871	1\\
3.3	2.6	0	0.975806451612903	1\\
3.3	2.6	0	0.983870967741935	1\\
3.3	2.6	0	0.991935483870968	1\\
3.3	2.6	0	1	1\\
3.3	2.6	0.00806451612903226	1	0.991935483870968\\
3.3	2.6	0.0161290322580645	1	0.983870967741935\\
3.3	2.6	0.0241935483870968	1	0.975806451612903\\
3.3	2.6	0.032258064516129	1	0.967741935483871\\
3.3	2.6	0.0403225806451613	1	0.959677419354839\\
3.3	2.6	0.0483870967741935	1	0.951612903225806\\
3.3	2.6	0.0564516129032258	1	0.943548387096774\\
3.3	2.6	0.0645161290322581	1	0.935483870967742\\
3.3	2.6	0.0725806451612903	1	0.92741935483871\\
3.3	2.6	0.0806451612903226	1	0.919354838709677\\
3.3	2.6	0.0887096774193548	1	0.911290322580645\\
3.3	2.6	0.0967741935483871	1	0.903225806451613\\
3.3	2.6	0.104838709677419	1	0.895161290322581\\
3.3	2.6	0.112903225806452	1	0.887096774193548\\
3.3	2.6	0.120967741935484	1	0.879032258064516\\
3.3	2.6	0.129032258064516	1	0.870967741935484\\
3.3	2.6	0.137096774193548	1	0.862903225806452\\
3.3	2.6	0.145161290322581	1	0.854838709677419\\
3.3	2.6	0.153225806451613	1	0.846774193548387\\
3.3	2.6	0.161290322580645	1	0.838709677419355\\
3.3	2.6	0.169354838709677	1	0.830645161290323\\
3.2	2.6	0.17741935483871	1	0.82258064516129\\
3.2	2.6	0.185483870967742	1	0.814516129032258\\
3.2	2.6	0.193548387096774	1	0.806451612903226\\
3.1	2.7	0.201612903225806	1	0.798387096774194\\
3.1	2.7	0.209677419354839	1	0.790322580645161\\
3.1	2.7	0.217741935483871	1	0.782258064516129\\
3.1	2.7	0.225806451612903	1	0.774193548387097\\
3.1	2.7	0.233870967741935	1	0.766129032258065\\
3.1	2.7	0.241935483870968	1	0.758064516129032\\
3.1	2.7	0.25	1	0.75\\
3.1	2.7	0.258064516129032	1	0.741935483870968\\
3.1	2.7	0.266129032258065	1	0.733870967741935\\
3.1	2.7	0.274193548387097	1	0.725806451612903\\
3.1	2.7	0.282258064516129	1	0.717741935483871\\
3.1	2.7	0.290322580645161	1	0.709677419354839\\
3.1	2.7	0.298387096774194	1	0.701612903225806\\
3.1	2.7	0.306451612903226	1	0.693548387096774\\
3.1	2.7	0.314516129032258	1	0.685483870967742\\
3.1	2.7	0.32258064516129	1	0.67741935483871\\
3.1	2.7	0.330645161290323	1	0.669354838709677\\
3.1	2.7	0.338709677419355	1	0.661290322580645\\
3.1	2.7	0.346774193548387	1	0.653225806451613\\
3.1	2.7	0.354838709677419	1	0.645161290322581\\
3.1	2.7	0.362903225806452	1	0.637096774193548\\
3.1	2.7	0.370967741935484	1	0.629032258064516\\
3.1	2.7	0.379032258064516	1	0.620967741935484\\
3.1	2.7	0.387096774193548	1	0.612903225806452\\
3.1	2.7	0.395161290322581	1	0.604838709677419\\
3.1	2.7	0.403225806451613	1	0.596774193548387\\
3.1	2.7	0.411290322580645	1	0.588709677419355\\
3	2.8	0.419354838709677	1	0.580645161290323\\
3	2.8	0.42741935483871	1	0.57258064516129\\
3	2.8	0.435483870967742	1	0.564516129032258\\
3	2.8	0.443548387096774	1	0.556451612903226\\
3	2.8	0.451612903225806	1	0.548387096774194\\
3	2.8	0.459677419354839	1	0.540322580645161\\
3	2.8	0.467741935483871	1	0.532258064516129\\
3	2.9	0.475806451612903	1	0.524193548387097\\
3	2.9	0.483870967741935	1	0.516129032258065\\
3	2.9	0.491935483870968	1	0.508064516129032\\
3	2.9	0.5	1	0.5\\
3	2.9	0.508064516129032	1	0.491935483870968\\
3	2.9	0.516129032258065	1	0.483870967741935\\
3	2.9	0.524193548387097	1	0.475806451612903\\
3	2.9	0.532258064516129	1	0.467741935483871\\
3	2.9	0.540322580645161	1	0.459677419354839\\
3	2.9	0.548387096774194	1	0.451612903225806\\
3	2.9	0.556451612903226	1	0.443548387096774\\
3	2.9	0.564516129032258	1	0.435483870967742\\
3	2.9	0.57258064516129	1	0.42741935483871\\
3	2.9	0.580645161290323	1	0.419354838709677\\
3	2.9	0.588709677419355	1	0.411290322580645\\
3	2.9	0.596774193548387	1	0.403225806451613\\
3	3	0.604838709677419	1	0.395161290322581\\
3	3	0.612903225806452	1	0.387096774193548\\
3	3	0.620967741935484	1	0.379032258064516\\
3	2.9	0.629032258064516	1	0.370967741935484\\
3	3	0.637096774193548	1	0.362903225806452\\
3	3	0.645161290322581	1	0.354838709677419\\
3	3	0.653225806451613	1	0.346774193548387\\
3	2.9	0.661290322580645	1	0.338709677419355\\
3	2.9	0.669354838709677	1	0.330645161290323\\
3	3	0.67741935483871	1	0.32258064516129\\
3	3	0.685483870967742	1	0.314516129032258\\
3	3	0.693548387096774	1	0.306451612903226\\
3	3	0.701612903225806	1	0.298387096774194\\
3	3	0.709677419354839	1	0.290322580645161\\
3	3	0.717741935483871	1	0.282258064516129\\
3	3	0.725806451612903	1	0.274193548387097\\
3	3	0.733870967741935	1	0.266129032258065\\
3	3	0.741935483870968	1	0.258064516129032\\
3	3	0.75	1	0.25\\
3	3	0.758064516129032	1	0.241935483870968\\
2.9	3	0.766129032258065	1	0.233870967741935\\
2.9	3	0.774193548387097	1	0.225806451612903\\
3	3	0.782258064516129	1	0.217741935483871\\
3	3	0.790322580645161	1	0.209677419354839\\
3	3	0.798387096774194	1	0.201612903225806\\
3	3	0.806451612903226	1	0.193548387096774\\
3	3	0.814516129032258	1	0.185483870967742\\
3	3	0.82258064516129	1	0.17741935483871\\
3	3	0.830645161290323	1	0.169354838709677\\
3	3	0.838709677419355	1	0.161290322580645\\
3	3	0.846774193548387	1	0.153225806451613\\
2.9	3	0.854838709677419	1	0.145161290322581\\
2.9	3	0.862903225806452	1	0.137096774193548\\
2.9	3	0.870967741935484	1	0.129032258064516\\
2.9	3	0.879032258064516	1	0.120967741935484\\
2.9	3	0.887096774193548	1	0.112903225806452\\
2.9	3	0.895161290322581	1	0.104838709677419\\
2.9	3	0.903225806451613	1	0.0967741935483871\\
2.9	3	0.911290322580645	1	0.0887096774193548\\
2.9	3	0.919354838709677	1	0.0806451612903226\\
2.9	3	0.92741935483871	1	0.0725806451612903\\
2.9	3	0.935483870967742	1	0.0645161290322581\\
2.9	3	0.943548387096774	1	0.0564516129032258\\
2.9	3	0.951612903225806	1	0.0483870967741935\\
2.9	3	0.959677419354839	1	0.0403225806451613\\
2.9	3	0.967741935483871	1	0.032258064516129\\
2.9	3.1	0.975806451612903	1	0.0241935483870968\\
2.9	3.1	0.983870967741935	1	0.0161290322580645\\
2.9	3.1	0.991935483870968	1	0.00806451612903226\\
2.9	3.1	1	1	0\\
2.9	3.1	1	0.991935483870968	0\\
2.9	3.1	1	0.983870967741935	0\\
2.9	3.1	1	0.975806451612903	0\\
2.9	3.1	1	0.967741935483871	0\\
2.9	3.1	1	0.959677419354839	0\\
2.9	3.1	1	0.951612903225806	0\\
2.9	3.1	1	0.943548387096774	0\\
2.9	3.1	1	0.935483870967742	0\\
2.9	3.1	1	0.92741935483871	0\\
2.9	3.1	1	0.919354838709677	0\\
2.9	3.1	1	0.911290322580645	0\\
2.9	3.1	1	0.903225806451613	0\\
2.9	3.1	1	0.895161290322581	0\\
2.9	3.1	1	0.887096774193548	0\\
2.9	3.1	1	0.879032258064516	0\\
2.9	3.1	1	0.870967741935484	0\\
2.9	3.1	1	0.862903225806452	0\\
2.9	3.1	1	0.854838709677419	0\\
2.9	3.1	1	0.846774193548387	0\\
2.9	3.1	1	0.838709677419355	0\\
2.9	3.1	1	0.830645161290323	0\\
2.9	3.1	1	0.82258064516129	0\\
2.9	3.2	1	0.814516129032258	0\\
2.9	3.2	1	0.806451612903226	0\\
2.9	3.2	1	0.798387096774194	0\\
2.9	3.2	1	0.790322580645161	0\\
2.9	3.2	1	0.782258064516129	0\\
2.9	3.2	1	0.774193548387097	0\\
2.9	3.2	1	0.766129032258065	0\\
2.9	3.2	1	0.758064516129032	0\\
2.9	3.2	1	0.75	0\\
2.9	3.2	1	0.741935483870968	0\\
2.9	3.2	1	0.733870967741935	0\\
2.9	3.2	1	0.725806451612903	0\\
2.9	3.2	1	0.717741935483871	0\\
2.9	3.2	1	0.709677419354839	0\\
2.9	3.2	1	0.701612903225806	0\\
2.9	3.2	1	0.693548387096774	0\\
2.9	3.2	1	0.685483870967742	0\\
2.9	3.2	1	0.67741935483871	0\\
2.9	3.2	1	0.669354838709677	0\\
2.9	3.2	1	0.661290322580645	0\\
2.9	3.2	1	0.653225806451613	0\\
2.9	3.2	1	0.645161290322581	0\\
2.9	3.2	1	0.637096774193548	0\\
2.9	3.2	1	0.629032258064516	0\\
2.9	3.2	1	0.620967741935484	0\\
2.9	3.2	1	0.612903225806452	0\\
2.9	3.2	1	0.604838709677419	0\\
2.9	3.2	1	0.596774193548387	0\\
2.9	3.2	1	0.588709677419355	0\\
2.9	3.2	1	0.580645161290323	0\\
2.9	3.2	1	0.57258064516129	0\\
2.9	3.2	1	0.564516129032258	0\\
2.9	3.2	1	0.556451612903226	0\\
2.9	3.2	1	0.548387096774194	0\\
2.9	3.2	1	0.540322580645161	0\\
2.8	3.3	1	0.532258064516129	0\\
2.8	3.3	1	0.524193548387097	0\\
2.8	3.3	1	0.516129032258065	0\\
2.8	3.3	1	0.508064516129032	0\\
2.8	3.3	1	0.5	0\\
2.6	3.4	1	0.491935483870968	0\\
2.6	3.4	1	0.483870967741935	0\\
2.6	3.4	1	0.475806451612903	0\\
2.6	3.4	1	0.467741935483871	0\\
2.6	3.4	1	0.459677419354839	0\\
2.6	3.4	1	0.451612903225806	0\\
2.8	3.3	1	0.443548387096774	0\\
2.8	3.3	1	0.435483870967742	0\\
2.8	3.3	1	0.42741935483871	0\\
2.8	3.3	1	0.419354838709677	0\\
2.8	3.3	1	0.411290322580645	0\\
2.8	3.3	1	0.403225806451613	0\\
2.8	3.3	1	0.395161290322581	0\\
2.8	3.3	1	0.387096774193548	0\\
2.8	3.3	1	0.379032258064516	0\\
2.8	3.3	1	0.370967741935484	0\\
2.8	3.3	1	0.362903225806452	0\\
2.8	3.3	1	0.354838709677419	0\\
2.8	3.3	1	0.346774193548387	0\\
2.8	3.3	1	0.338709677419355	0\\
2.8	3.3	1	0.330645161290323	0\\
2.8	3.3	1	0.32258064516129	0\\
2.8	3.3	1	0.314516129032258	0\\
2.8	3.3	1	0.306451612903226	0\\
2.8	3.3	1	0.298387096774194	0\\
2.8	3.3	1	0.290322580645161	0\\
2.8	3.3	1	0.282258064516129	0\\
2.8	3.3	1	0.274193548387097	0\\
2.6	3.4	1	0.266129032258065	0\\
2.6	3.4	1	0.258064516129032	0\\
2.8	3.3	1	0.25	0\\
2.8	3.3	1	0.241935483870968	0\\
2.8	3.3	1	0.233870967741935	0\\
3.5	3.5	1	0.225806451612903	0\\
3.5	3.5	1	0.217741935483871	0\\
3.5	3.5	1	0.209677419354839	0\\
3.5	3.5	1	0.201612903225806	0\\
3.5	3.5	1	0.193548387096774	0\\
3.5	3.5	1	0.185483870967742	0\\
2.6	3.4	1	0.17741935483871	0\\
2.5	3.5	1	0.169354838709677	0\\
2.4	3.5	1	0.161290322580645	0\\
2.4	3.5	1	0.153225806451613	0\\
2.4	3.5	1	0.145161290322581	0\\
2.4	3.5	1	0.137096774193548	0\\
2.4	3.5	1	0.129032258064516	0\\
2.4	3.5	1	0.120967741935484	0\\
2.4	3.5	1	0.112903225806452	0\\
2.4	3.5	1	0.104838709677419	0\\
2.4	3.5	1	0.0967741935483871	0\\
2.4	3.6	1	0.0887096774193548	0\\
2.4	3.6	1	0.0806451612903226	0\\
2.4	3.6	1	0.0725806451612903	0\\
2.4	3.6	1	0.0645161290322581	0\\
2.4	3.6	1	0.0564516129032258	0\\
2.4	3.6	1	0.0483870967741935	0\\
2.4	3.6	1	0.0403225806451613	0\\
2.4	3.6	1	0.032258064516129	0\\
2.4	3.6	1	0.0241935483870968	0\\
2.4	3.6	1	0.0161290322580645	0\\
2.4	3.6	1	0.00806451612903226	0\\
2.4	3.6	1	0	0\\
2.4	3.6	0.991935483870968	0	0\\
2.4	3.6	0.983870967741935	0	0\\
2.4	3.6	0.975806451612903	0	0\\
2.4	3.6	0.967741935483871	0	0\\
2.4	3.6	0.959677419354839	0	0\\
2.4	3.6	0.951612903225806	0	0\\
2.4	3.6	0.943548387096774	0	0\\
2.4	3.6	0.935483870967742	0	0\\
2.4	3.6	0.92741935483871	0	0\\
2.4	3.6	0.919354838709677	0	0\\
2.4	3.6	0.911290322580645	0	0\\
2.4	3.6	0.903225806451613	0	0\\
2.4	3.6	0.895161290322581	0	0\\
2.4	3.6	0.887096774193548	0	0\\
2.4	3.6	0.879032258064516	0	0\\
2.4	3.6	0.870967741935484	0	0\\
2.4	3.6	0.862903225806452	0	0\\
2.3	3.6	0.854838709677419	0	0\\
2.3	3.6	0.846774193548387	0	0\\
2.3	3.6	0.838709677419355	0	0\\
2.3	3.6	0.830645161290323	0	0\\
2.3	3.6	0.82258064516129	0	0\\
2.3	3.6	0.814516129032258	0	0\\
2.3	3.6	0.806451612903226	0	0\\
2.3	3.6	0.798387096774194	0	0\\
2.3	3.6	0.790322580645161	0	0\\
2.3	3.6	0.782258064516129	0	0\\
2.3	3.6	0.774193548387097	0	0\\
2.3	3.6	0.766129032258065	0	0\\
2.3	3.6	0.758064516129032	0	0\\
2.3	3.6	0.75	0	0\\
2.3	3.6	0.741935483870968	0	0\\
2.3	3.6	0.733870967741935	0	0\\
2.3	3.6	0.725806451612903	0	0\\
2.3	3.6	0.717741935483871	0	0\\
2.3	3.6	0.709677419354839	0	0\\
2.3	3.6	0.701612903225806	0	0\\
2.3	3.6	0.693548387096774	0	0\\
2.3	3.6	0.685483870967742	0	0\\
2.3	3.6	0.67741935483871	0	0\\
2.3	3.6	0.669354838709677	0	0\\
2.3	3.6	0.661290322580645	0	0\\
2.3	3.6	0.653225806451613	0	0\\
2.3	3.6	0.645161290322581	0	0\\
3.9	3.9	0.637096774193548	0	0\\
3.9	3.9	0.629032258064516	0	0\\
3.9	3.9	0.620967741935484	0	0\\
3.9	3.9	0.612903225806452	0	0\\
3.9	3.9	0.604838709677419	0	0\\
3.9	3.9	0.596774193548387	0	0\\
3.9	3.9	0.588709677419355	0	0\\
3.9	3.9	0.580645161290323	0	0\\
3.9	3.9	0.57258064516129	0	0\\
3.9	3.9	0.564516129032258	0	0\\
3.9	3.9	0.556451612903226	0	0\\
3.9	3.9	0.548387096774194	0	0\\
3.9	3.9	0.540322580645161	0	0\\
3.9	3.9	0.532258064516129	0	0\\
3.9	3.9	0.524193548387097	0	0\\
3.9	3.9	0.516129032258065	0	0\\
3.9	3.9	0.508064516129032	0	0\\
3.9	3.9	0.5	0	0\\
1.2	3	0	0	0.508064516129032\\
2.5	1.9	0	0	0.516129032258065\\
1.3	2.2	0	0	0.524193548387097\\
1.4	2.2	0	0	0.532258064516129\\
1.4	2.2	0	0	0.540322580645161\\
1.5	2.1	0	0	0.548387096774194\\
1.5	2.1	0	0	0.556451612903226\\
1.5	2.1	0	0	0.564516129032258\\
1.5	2.1	0	0	0.57258064516129\\
1.5	2.1	0	0	0.580645161290323\\
1.5	2.1	0	0	0.588709677419355\\
1.5	2.1	0	0	0.596774193548387\\
1.5	2.1	0	0	0.604838709677419\\
3.9	2.1	0	0	0.612903225806452\\
4	2.2	0	0	0.620967741935484\\
4	2.1	0	0	0.629032258064516\\
3.9	2.1	0	0	0.637096774193548\\
2.1	2	0	0	0.645161290322581\\
2.1	2	0	0	0.653225806451613\\
2.1	2	0	0	0.661290322580645\\
2.1	2	0	0	0.669354838709677\\
2.1	2	0	0	0.67741935483871\\
2.1	2	0	0	0.685483870967742\\
2.1	2	0	0	0.693548387096774\\
2.1	2	0	0	0.701612903225806\\
2.1	2	0	0	0.709677419354839\\
3.9	2.1	0	0	0.717741935483871\\
3.9	2.1	0	0	0.725806451612903\\
3.9	2.1	0	0	0.733870967741935\\
3.9	2.1	0	0	0.741935483870968\\
2.2	1.4	0	0	0.75\\
2.2	1.4	0	0	0.758064516129032\\
2.2	1.3	0	0	0.766129032258065\\
2.2	1.2	0	0	0.774193548387097\\
2.2	1.2	0	0	0.782258064516129\\
2.2	1.2	0	0	0.790322580645161\\
2.1	2	0	0	0.798387096774194\\
2.1	2	0	0	0.806451612903226\\
2.1	2	0	0	0.814516129032258\\
2.1	2	0	0	0.82258064516129\\
2.1	2	0	0	0.830645161290323\\
2.2	1.2	0	0	0.838709677419355\\
2.2	1.2	0	0	0.846774193548387\\
2.2	1.2	0	0	0.854838709677419\\
3.8	1.2	0	0	0.862903225806452\\
3.8	1.2	0	0	0.870967741935484\\
3.8	1.2	0	0	0.879032258064516\\
3.8	1.2	0	0	0.887096774193548\\
3.8	1.2	0	0	0.895161290322581\\
3.8	1.2	0	0	0.903225806451613\\
3.7	1.5	0	0	0.911290322580645\\
3.6	1.7	0	0	0.919354838709677\\
3.7	1.5	0	0	0.92741935483871\\
3.7	1.5	0	0	0.935483870967742\\
2.3	2.2	0	0	0.943548387096774\\
2.3	2.2	0	0	0.951612903225806\\
2.2	2.2	0	0	0.959677419354839\\
2.2	2.2	0	0	0.967741935483871\\
2.2	2.1	0	0	0.975806451612903\\
2.2	2.1	0	0	0.983870967741935\\
2.2	2.1	0	0	0.991935483870968\\
2.2	2.1	0	0	1\\
2.2	2.1	0	0.00806451612903226	1\\
2.2	2	0	0.0161290322580645	1\\
2.2	2	0	0.0241935483870968	1\\
2.2	2	0	0.032258064516129	1\\
2.2	2.1	0	0.0403225806451613	1\\
3.9	2.1	0	0.0483870967741935	1\\
3.9	2.1	0	0.0564516129032258	1\\
3.8	2.1	0	0.0645161290322581	1\\
3.9	2.1	0	0.0725806451612903	1\\
3.9	2.1	0	0.0806451612903226	1\\
3.9	2.1	0	0.0887096774193548	1\\
3.9	2.1	0	0.0967741935483871	1\\
3.9	2.2	0	0.104838709677419	1\\
3.9	2.2	0	0.112903225806452	1\\
2.3	2.2	0	0.120967741935484	1\\
2.2	2.1	0	0.129032258064516	1\\
2.2	2.1	0	0.137096774193548	1\\
2.2	2.1	0	0.145161290322581	1\\
2.2	2.1	0	0.153225806451613	1\\
2.2	2.1	0	0.161290322580645	1\\
4.3	2.3	0	0.169354838709677	1\\
4.3	2.3	0	0.17741935483871	1\\
4.2	2.3	0	0.185483870967742	1\\
4.2	2.3	0	0.193548387096774	1\\
3.9	2.7	0	0.201612903225806	1\\
4	2.6	0	0.209677419354839	1\\
4	2.6	0	0.217741935483871	1\\
4	2.6	0	0.225806451612903	1\\
4	2.6	0	0.233870967741935	1\\
4	2.6	0	0.241935483870968	1\\
4	2.6	0	0.25	1\\
4	2.6	0	0.258064516129032	1\\
4.2	2.3	0	0.266129032258065	1\\
4.2	2.3	0	0.274193548387097	1\\
4.2	2.3	0	0.282258064516129	1\\
4.2	2.3	0	0.290322580645161	1\\
4.2	2.3	0	0.298387096774194	1\\
4.2	2.3	0	0.306451612903226	1\\
4.2	2.3	0	0.314516129032258	1\\
4.2	2.3	0	0.32258064516129	1\\
4.2	2.3	0	0.330645161290323	1\\
4.2	2.3	0	0.338709677419355	1\\
2.2	1.2	0	0.346774193548387	1\\
2.2	1.2	0	0.354838709677419	1\\
2.2	1.2	0	0.362903225806452	1\\
1.2	3.8	0	0.370967741935484	1\\
1.2	3.8	0	0.379032258064516	1\\
1.2	3.8	0	0.387096774193548	1\\
1.2	3.8	0	0.395161290322581	1\\
1.2	3.8	0	0.403225806451613	1\\
1.2	3.8	0	0.411290322580645	1\\
1.2	3.8	0	0.419354838709677	1\\
1.2	3.8	0	0.42741935483871	1\\
1.2	3.8	0	0.435483870967742	1\\
1.2	3.8	0	0.443548387096774	1\\
1.2	3.8	0	0.451612903225806	1\\
3.6	2.3	0	0.459677419354839	1\\
3.6	2.3	0	0.467741935483871	1\\
3.6	2.3	0	0.475806451612903	1\\
3.6	2.3	0	0.483870967741935	1\\
3.6	2.3	0	0.491935483870968	1\\
1.2	3.8	0	0.5	1\\
1.2	3.8	0	0.508064516129032	1\\
1.2	3.8	0	0.516129032258065	1\\
1.2	2.2	0	0.524193548387097	1\\
3.6	2.3	0	0.532258064516129	1\\
1.2	2.2	0	0.540322580645161	1\\
1.2	2.2	0	0.548387096774194	1\\
3.6	2.3	0	0.556451612903226	1\\
3.6	2.3	0	0.564516129032258	1\\
3.6	2.3	0	0.57258064516129	1\\
3.6	2.3	0	0.580645161290323	1\\
3.6	2.3	0	0.588709677419355	1\\
3.6	2.3	0	0.596774193548387	1\\
2.8	1.2	0	0.604838709677419	1\\
2.8	1.2	0	0.612903225806452	1\\
2.8	1.2	0	0.620967741935484	1\\
2.8	1.2	0	0.629032258064516	1\\
2.8	1.2	0	0.637096774193548	1\\
1.2	2.2	0	0.645161290322581	1\\
1.2	2.2	0	0.653225806451613	1\\
1.2	2.2	0	0.661290322580645	1\\
1.2	2.2	0	0.669354838709677	1\\
1.2	2.2	0	0.67741935483871	1\\
1.2	2.2	0	0.685483870967742	1\\
1.2	2.2	0	0.693548387096774	1\\
3.6	2.4	0	0.701612903225806	1\\
2.8	1.2	0	0.709677419354839	1\\
2.8	1.2	0	0.717741935483871	1\\
2.8	1.2	0	0.725806451612903	1\\
2.8	1.2	0	0.733870967741935	1\\
2.8	1.2	0	0.741935483870968	1\\
2.8	1.2	0	0.75	1\\
2.8	1.2	0	0.758064516129032	1\\
2.8	1.2	0	0.766129032258065	1\\
2.8	1.2	0	0.774193548387097	1\\
2.8	1.2	0	0.782258064516129	1\\
2.8	1.2	0	0.790322580645161	1\\
3	2.8	0	0.798387096774194	1\\
3	2.8	0	0.806451612903226	1\\
1.2	2.2	0	0.814516129032258	1\\
1.2	2.2	0	0.82258064516129	1\\
3.8	1.2	0	0.830645161290323	1\\
1.2	2.2	0	0.838709677419355	1\\
1.2	2.2	0	0.846774193548387	1\\
1.2	2.2	0	0.854838709677419	1\\
1.2	2.2	0	0.862903225806452	1\\
1.2	2.2	0	0.870967741935484	1\\
1.2	2.2	0	0.879032258064516	1\\
1.2	2.2	0	0.887096774193548	1\\
1.2	2.2	0	0.895161290322581	1\\
1.2	2.2	0	0.903225806451613	1\\
1.2	2.2	0	0.911290322580645	1\\
1.2	2.2	0	0.919354838709677	1\\
1.2	2.2	0	0.92741935483871	1\\
1.2	2.2	0	0.935483870967742	1\\
1.2	2.2	0	0.943548387096774	1\\
1.2	2.2	0	0.951612903225806	1\\
1.2	2.2	0	0.959677419354839	1\\
1.2	2.2	0	0.967741935483871	1\\
1.2	2.2	0	0.975806451612903	1\\
1.2	2.2	0	0.983870967741935	1\\
1.2	3	0	0.991935483870968	1\\
1.2	3	0	1	1\\
1.2	3	0.00806451612903226	1	0.991935483870968\\
1.2	3	0.0161290322580645	1	0.983870967741935\\
1.2	3	0.0241935483870968	1	0.975806451612903\\
1.2	3	0.032258064516129	1	0.967741935483871\\
1.2	3	0.0403225806451613	1	0.959677419354839\\
1.2	3	0.0483870967741935	1	0.951612903225806\\
1.2	3	0.0564516129032258	1	0.943548387096774\\
2.8	1.2	0.0645161290322581	1	0.935483870967742\\
1.2	3	0.0725806451612903	1	0.92741935483871\\
1.2	3	0.0806451612903226	1	0.919354838709677\\
1.2	3	0.0887096774193548	1	0.911290322580645\\
1.2	3	0.0967741935483871	1	0.903225806451613\\
1.2	3	0.104838709677419	1	0.895161290322581\\
1.2	3	0.112903225806452	1	0.887096774193548\\
1.2	3	0.120967741935484	1	0.879032258064516\\
1.2	3	0.129032258064516	1	0.870967741935484\\
4.7	3.8	0.137096774193548	1	0.862903225806452\\
4.7	3.8	0.145161290322581	1	0.854838709677419\\
4.7	3.8	0.153225806451613	1	0.846774193548387\\
4.7	3.8	0.161290322580645	1	0.838709677419355\\
4.7	3.8	0.169354838709677	1	0.830645161290323\\
2.8	2.9	0.17741935483871	1	0.82258064516129\\
2.8	2.9	0.185483870967742	1	0.814516129032258\\
2.8	2.9	0.193548387096774	1	0.806451612903226\\
2.6	2.9	0.201612903225806	1	0.798387096774194\\
2.8	3.1	0.209677419354839	1	0.790322580645161\\
2.8	3.1	0.217741935483871	1	0.782258064516129\\
2.8	3.1	0.225806451612903	1	0.774193548387097\\
2.8	3.1	0.233870967741935	1	0.766129032258065\\
2.8	3.1	0.241935483870968	1	0.758064516129032\\
2.7	3.1	0.25	1	0.75\\
2.7	3.1	0.258064516129032	1	0.741935483870968\\
2.8	3.1	0.266129032258065	1	0.733870967741935\\
2.7	3.1	0.274193548387097	1	0.725806451612903\\
2.7	3.1	0.282258064516129	1	0.717741935483871\\
2.8	3.1	0.290322580645161	1	0.709677419354839\\
2.8	3.1	0.298387096774194	1	0.701612903225806\\
2.8	3.1	0.306451612903226	1	0.693548387096774\\
2.8	3.1	0.314516129032258	1	0.685483870967742\\
2.9	3.2	0.32258064516129	1	0.67741935483871\\
2.9	3.2	0.330645161290323	1	0.669354838709677\\
2.8	3.1	0.338709677419355	1	0.661290322580645\\
2.8	3.1	0.346774193548387	1	0.653225806451613\\
2.8	3.1	0.354838709677419	1	0.645161290322581\\
2.8	3.1	0.362903225806452	1	0.637096774193548\\
2.9	3.2	0.370967741935484	1	0.629032258064516\\
2.9	3.2	0.379032258064516	1	0.620967741935484\\
2.8	3.1	0.387096774193548	1	0.612903225806452\\
2.8	3.1	0.395161290322581	1	0.604838709677419\\
2.8	3.1	0.403225806451613	1	0.596774193548387\\
2.8	3.1	0.411290322580645	1	0.588709677419355\\
3.4	2.5	0.419354838709677	1	0.580645161290323\\
3.4	2.5	0.42741935483871	1	0.57258064516129\\
3.4	2.5	0.435483870967742	1	0.564516129032258\\
3.4	2.5	0.443548387096774	1	0.556451612903226\\
3.4	2.5	0.451612903225806	1	0.548387096774194\\
3	3.4	0.459677419354839	1	0.540322580645161\\
2.8	3.3	0.467741935483871	1	0.532258064516129\\
3.4	2.5	0.475806451612903	1	0.524193548387097\\
3.4	2.5	0.483870967741935	1	0.516129032258065\\
1.2	3	0.491935483870968	1	0.508064516129032\\
2.8	1.2	0.5	1	0.5\\
2.8	1.2	0.508064516129032	1	0.491935483870968\\
2.8	1.2	0.516129032258065	1	0.483870967741935\\
2.7	1.2	0.524193548387097	1	0.475806451612903\\
1.2	3	0.532258064516129	1	0.467741935483871\\
1.2	3	0.540322580645161	1	0.459677419354839\\
1.2	3	0.548387096774194	1	0.451612903225806\\
1.2	3	0.556451612903226	1	0.443548387096774\\
1.2	3	0.564516129032258	1	0.435483870967742\\
1.2	3	0.57258064516129	1	0.42741935483871\\
1.2	3	0.580645161290323	1	0.419354838709677\\
1.2	3	0.588709677419355	1	0.411290322580645\\
1.2	3	0.596774193548387	1	0.403225806451613\\
1.2	3	0.604838709677419	1	0.395161290322581\\
1.2	3	0.612903225806452	1	0.387096774193548\\
1.2	3	0.620967741935484	1	0.379032258064516\\
1.2	3	0.629032258064516	1	0.370967741935484\\
1.2	3	0.637096774193548	1	0.362903225806452\\
1.2	3	0.645161290322581	1	0.354838709677419\\
1.2	3	0.653225806451613	1	0.346774193548387\\
1.2	3	0.661290322580645	1	0.338709677419355\\
1.2	3	0.669354838709677	1	0.330645161290323\\
1.2	3	0.67741935483871	1	0.32258064516129\\
1.2	3	0.685483870967742	1	0.314516129032258\\
1.2	3	0.693548387096774	1	0.306451612903226\\
1.2	3	0.701612903225806	1	0.298387096774194\\
1.2	3	0.709677419354839	1	0.290322580645161\\
1.2	3	0.717741935483871	1	0.282258064516129\\
1.2	3	0.725806451612903	1	0.274193548387097\\
1.2	3	0.733870967741935	1	0.266129032258065\\
1.2	3	0.741935483870968	1	0.258064516129032\\
1.2	3	0.75	1	0.25\\
1.2	3	0.758064516129032	1	0.241935483870968\\
1.2	3	0.766129032258065	1	0.233870967741935\\
1.2	3	0.774193548387097	1	0.225806451612903\\
2.7	1.2	0.782258064516129	1	0.217741935483871\\
2.7	1.2	0.790322580645161	1	0.209677419354839\\
2.7	1.2	0.798387096774194	1	0.201612903225806\\
2.7	1.2	0.806451612903226	1	0.193548387096774\\
2.7	1.2	0.814516129032258	1	0.185483870967742\\
2.7	1.2	0.82258064516129	1	0.17741935483871\\
2.7	1.2	0.830645161290323	1	0.169354838709677\\
1.3	3.7	0.838709677419355	1	0.161290322580645\\
1.3	3.7	0.846774193548387	1	0.153225806451613\\
1.2	3.8	0.854838709677419	1	0.145161290322581\\
1.2	3.8	0.862903225806452	1	0.137096774193548\\
1.2	3.8	0.870967741935484	1	0.129032258064516\\
1.3	3.7	0.879032258064516	1	0.120967741935484\\
1.2	3.8	0.887096774193548	1	0.112903225806452\\
1.2	3.8	0.895161290322581	1	0.104838709677419\\
1.2	3.8	0.903225806451613	1	0.0967741935483871\\
1.2	3.8	0.911290322580645	1	0.0887096774193548\\
1.2	3.8	0.919354838709677	1	0.0806451612903226\\
1.3	3.7	0.92741935483871	1	0.0725806451612903\\
1.3	3.7	0.935483870967742	1	0.0645161290322581\\
1.3	3.7	0.943548387096774	1	0.0564516129032258\\
1.3	3.7	0.951612903225806	1	0.0483870967741935\\
1.2	3.8	0.959677419354839	1	0.0403225806451613\\
1.2	3.8	0.967741935483871	1	0.032258064516129\\
1.2	3.8	0.975806451612903	1	0.0241935483870968\\
1.2	3.8	0.983870967741935	1	0.0161290322580645\\
1.2	3.8	0.991935483870968	1	0.00806451612903226\\
1.2	3.8	1	1	0\\
3.8	1.2	1	0.991935483870968	0\\
3.8	1.2	1	0.983870967741935	0\\
1.3	3.7	1	0.975806451612903	0\\
1.3	3.7	1	0.967741935483871	0\\
3.7	1.2	1	0.959677419354839	0\\
3.7	1.2	1	0.951612903225806	0\\
3.7	1.2	1	0.943548387096774	0\\
3.7	1.2	1	0.935483870967742	0\\
3.7	1.2	1	0.92741935483871	0\\
3.7	1.2	1	0.919354838709677	0\\
3.7	1.2	1	0.911290322580645	0\\
3.7	1.2	1	0.903225806451613	0\\
3.7	1.2	1	0.895161290322581	0\\
3.7	1.2	1	0.887096774193548	0\\
3.7	1.2	1	0.879032258064516	0\\
3.7	1.2	1	0.870967741935484	0\\
3.7	1.2	1	0.862903225806452	0\\
3.7	1.2	1	0.854838709677419	0\\
3.7	1.2	1	0.846774193548387	0\\
3.7	1.2	1	0.838709677419355	0\\
3.7	1.2	1	0.830645161290323	0\\
2.6	3.5	1	0.82258064516129	0\\
2.4	3.4	1	0.814516129032258	0\\
2.4	3.4	1	0.806451612903226	0\\
2.4	3.4	1	0.798387096774194	0\\
2.4	3.4	1	0.790322580645161	0\\
2.4	3.4	1	0.782258064516129	0\\
2.4	3.4	1	0.774193548387097	0\\
2.4	3.4	1	0.766129032258065	0\\
2.4	3.4	1	0.758064516129032	0\\
2.4	3.4	1	0.75	0\\
2.4	3.4	1	0.741935483870968	0\\
2.4	3.4	1	0.733870967741935	0\\
2.4	3.4	1	0.725806451612903	0\\
2.7	1.2	1	0.717741935483871	0\\
2.7	1.2	1	0.709677419354839	0\\
2.7	1.2	1	0.701612903225806	0\\
2.7	1.2	1	0.693548387096774	0\\
2.7	1.2	1	0.685483870967742	0\\
3.7	1.2	1	0.67741935483871	0\\
3.7	1.2	1	0.669354838709677	0\\
2.4	3.4	1	0.661290322580645	0\\
2.4	3.4	1	0.653225806451613	0\\
2.7	1.2	1	0.645161290322581	0\\
2.7	1.2	1	0.637096774193548	0\\
3.7	1.2	1	0.629032258064516	0\\
3.7	1.2	1	0.620967741935484	0\\
3.7	1.2	1	0.612903225806452	0\\
3.7	1.2	1	0.604838709677419	0\\
3.7	1.2	1	0.596774193548387	0\\
3.7	1.2	1	0.588709677419355	0\\
3.7	1.2	1	0.580645161290323	0\\
3.7	1.2	1	0.57258064516129	0\\
2.3	3.4	1	0.564516129032258	0\\
2.3	3.4	1	0.556451612903226	0\\
2.3	3.4	1	0.548387096774194	0\\
2.3	3.4	1	0.540322580645161	0\\
2.2	3.4	1	0.532258064516129	0\\
2.3	3.6	1	0.524193548387097	0\\
2.3	3.6	1	0.516129032258065	0\\
2.2	3.5	1	0.508064516129032	0\\
2.2	3.5	1	0.5	0\\
3.1	3.1	1	0.491935483870968	0\\
3.1	3.1	1	0.483870967741935	0\\
3.1	3.1	1	0.475806451612903	0\\
3.1	3.1	1	0.467741935483871	0\\
3.2	3.3	1	0.459677419354839	0\\
3.2	3.3	1	0.451612903225806	0\\
3.4	3.4	1	0.443548387096774	0\\
3.4	3.4	1	0.435483870967742	0\\
3.4	3.4	1	0.42741935483871	0\\
3.4	3.4	1	0.419354838709677	0\\
3.4	3.4	1	0.411290322580645	0\\
3.4	3.4	1	0.403225806451613	0\\
3.4	3.4	1	0.395161290322581	0\\
3.4	3.4	1	0.387096774193548	0\\
3.4	3.4	1	0.379032258064516	0\\
3.4	3.4	1	0.370967741935484	0\\
3.4	3.4	1	0.362903225806452	0\\
3.4	3.4	1	0.354838709677419	0\\
3.4	3.4	1	0.346774193548387	0\\
3.4	3.4	1	0.338709677419355	0\\
3.4	3.4	1	0.330645161290323	0\\
3.4	3.4	1	0.32258064516129	0\\
3.4	3.4	1	0.314516129032258	0\\
3.4	3.4	1	0.306451612903226	0\\
3.4	3.4	1	0.298387096774194	0\\
3.4	3.4	1	0.290322580645161	0\\
3.4	3.4	1	0.282258064516129	0\\
3.4	3.4	1	0.274193548387097	0\\
3.4	3.4	1	0.266129032258065	0\\
3.5	3.5	1	0.258064516129032	0\\
3.5	3.5	1	0.25	0\\
3.5	3.5	1	0.241935483870968	0\\
3.5	3.5	1	0.233870967741935	0\\
2.8	3.3	1	0.225806451612903	0\\
2.8	3.3	1	0.217741935483871	0\\
2.8	3.3	1	0.209677419354839	0\\
2.8	3.3	1	0.201612903225806	0\\
2.8	3.3	1	0.193548387096774	0\\
2.8	3.3	1	0.185483870967742	0\\
3.5	3.5	1	0.17741935483871	0\\
3.5	3.5	1	0.169354838709677	0\\
3.5	3.5	1	0.161290322580645	0\\
3.5	3.5	1	0.153225806451613	0\\
3.5	3.5	1	0.145161290322581	0\\
3.5	3.5	1	0.137096774193548	0\\
3.5	3.5	1	0.129032258064516	0\\
3.5	3.5	1	0.120967741935484	0\\
3.5	3.5	1	0.112903225806452	0\\
3.5	3.5	1	0.104838709677419	0\\
3.5	3.5	1	0.0967741935483871	0\\
2.8	3.2	1	0.0887096774193548	0\\
2.8	3.2	1	0.0806451612903226	0\\
2.8	3.2	1	0.0725806451612903	0\\
2.8	3.2	1	0.0645161290322581	0\\
2.8	3.2	1	0.0564516129032258	0\\
2.8	3.2	1	0.0483870967741935	0\\
2.8	3.2	1	0.0403225806451613	0\\
2.8	3.2	1	0.032258064516129	0\\
2.8	3.2	1	0.0241935483870968	0\\
2.8	3.2	1	0.0161290322580645	0\\
2.8	3.2	1	0.00806451612903226	0\\
1.2	3.7	1	0	0\\
1.2	3.7	0.991935483870968	0	0\\
1.2	3.7	0.983870967741935	0	0\\
1.2	3.7	0.975806451612903	0	0\\
1.2	3.7	0.967741935483871	0	0\\
1.2	3.7	0.959677419354839	0	0\\
1.2	3.7	0.951612903225806	0	0\\
1.2	3.7	0.943548387096774	0	0\\
1.2	3.7	0.935483870967742	0	0\\
1.2	3.7	0.92741935483871	0	0\\
1.2	3.7	0.919354838709677	0	0\\
1.2	3.7	0.911290322580645	0	0\\
1.2	3.7	0.903225806451613	0	0\\
1.2	3.7	0.895161290322581	0	0\\
1.2	3.7	0.887096774193548	0	0\\
1.2	3.7	0.879032258064516	0	0\\
1.2	3.7	0.870967741935484	0	0\\
1.2	3.7	0.862903225806452	0	0\\
1.2	3.7	0.854838709677419	0	0\\
1.2	3.7	0.846774193548387	0	0\\
1.2	3.7	0.838709677419355	0	0\\
1.2	3.7	0.830645161290323	0	0\\
1.2	3.7	0.82258064516129	0	0\\
1.2	3.7	0.814516129032258	0	0\\
1.2	3.7	0.806451612903226	0	0\\
1.2	3.7	0.798387096774194	0	0\\
1.2	3.7	0.790322580645161	0	0\\
3.9	3.8	0.782258064516129	0	0\\
3.9	3.8	0.774193548387097	0	0\\
3.9	3.9	0.766129032258065	0	0\\
3.9	3.9	0.758064516129032	0	0\\
3.9	3.9	0.75	0	0\\
3.9	3.9	0.741935483870968	0	0\\
3.9	3.9	0.733870967741935	0	0\\
3.9	3.9	0.725806451612903	0	0\\
3.9	3.9	0.717741935483871	0	0\\
3.9	3.9	0.709677419354839	0	0\\
3.9	3.9	0.701612903225806	0	0\\
3.9	3.9	0.693548387096774	0	0\\
3.9	3.9	0.685483870967742	0	0\\
3.9	3.9	0.67741935483871	0	0\\
3.9	3.9	0.669354838709677	0	0\\
3.9	3.9	0.661290322580645	0	0\\
3.9	3.9	0.653225806451613	0	0\\
3.9	3.9	0.645161290322581	0	0\\
2.3	3.6	0.637096774193548	0	0\\
2.3	3.6	0.629032258064516	0	0\\
2.3	3.6	0.620967741935484	0	0\\
2.3	3.6	0.612903225806452	0	0\\
2.3	3.6	0.604838709677419	0	0\\
2.3	3.6	0.596774193548387	0	0\\
2.3	3.6	0.588709677419355	0	0\\
2.3	3.6	0.580645161290323	0	0\\
2.3	3.6	0.57258064516129	0	0\\
2.3	3.6	0.564516129032258	0	0\\
2.3	3.6	0.556451612903226	0	0\\
2.3	3.6	0.548387096774194	0	0\\
2.3	3.6	0.540322580645161	0	0\\
2.3	3.6	0.532258064516129	0	0\\
2.3	3.6	0.524193548387097	0	0\\
2.3	3.6	0.516129032258065	0	0\\
2.3	3.6	0.508064516129032	0	0\\
2.3	3.6	0.5	0	0\\
};
\end{axis}
\end{tikzpicture}%
        \caption{Estimated Positions \glsentryshort{trem}}
	\end{subfigure}
	\begin{subfigure}{0.49\textwidth}
		 \centering
        \setlength{\figurewidth}{0.8\textwidth}
        % This file was created by matlab2tikz.
%
\definecolor{lms_red}{rgb}{0.80000,0.20780,0.21960}%
%
\begin{tikzpicture}

\begin{axis}[%
width=0.951\figurewidth,
height=\figureheight,
at={(0\figurewidth,0\figureheight)},
scale only axis,
xmin=1,
xmax=5,
xlabel style={font=\color{white!15!black}},
xlabel={$p_x^{(t)}$~[m]},
ymin=1,
ymax=5,
ylabel style={font=\color{white!15!black}},
ylabel={$p_y^{(t)}$~[m]},
axis background/.style={fill=white},
axis x line*=bottom,
axis y line*=left,
xmajorgrids,
ymajorgrids
]
\addplot[scatter, only marks, mark=o] table[row sep=crcr]{%
x	y	R	G	B\\
2	2	0	0	0.504\\
2.00400801603206	2.00400801603206	0	0	0.512\\
2.00801603206413	2.00801603206413	0	0	0.52\\
2.01202404809619	2.01202404809619	0	0	0.528\\
2.01603206412826	2.01603206412826	0	0	0.536\\
2.02004008016032	2.02004008016032	0	0	0.544\\
2.02404809619238	2.02404809619238	0	0	0.552\\
2.02805611222445	2.02805611222445	0	0	0.56\\
2.03206412825651	2.03206412825651	0	0	0.568\\
2.03607214428858	2.03607214428858	0	0	0.576\\
2.04008016032064	2.04008016032064	0	0	0.584\\
2.04408817635271	2.04408817635271	0	0	0.592\\
2.04809619238477	2.04809619238477	0	0	0.6\\
2.05210420841683	2.05210420841683	0	0	0.608\\
2.0561122244489	2.0561122244489	0	0	0.616\\
2.06012024048096	2.06012024048096	0	0	0.624\\
2.06412825651303	2.06412825651303	0	0	0.632\\
2.06813627254509	2.06813627254509	0	0	0.64\\
2.07214428857715	2.07214428857715	0	0	0.648\\
2.07615230460922	2.07615230460922	0	0	0.656\\
2.08016032064128	2.08016032064128	0	0	0.664\\
2.08416833667335	2.08416833667335	0	0	0.672\\
2.08817635270541	2.08817635270541	0	0	0.68\\
2.09218436873747	2.09218436873747	0	0	0.688\\
2.09619238476954	2.09619238476954	0	0	0.696\\
2.1002004008016	2.1002004008016	0	0	0.704\\
2.10420841683367	2.10420841683367	0	0	0.712\\
2.10821643286573	2.10821643286573	0	0	0.72\\
2.1122244488978	2.1122244488978	0	0	0.728\\
2.11623246492986	2.11623246492986	0	0	0.736\\
2.12024048096192	2.12024048096192	0	0	0.744\\
2.12424849699399	2.12424849699399	0	0	0.752\\
2.12825651302605	2.12825651302605	0	0	0.76\\
2.13226452905812	2.13226452905812	0	0	0.768\\
2.13627254509018	2.13627254509018	0	0	0.776\\
2.14028056112224	2.14028056112224	0	0	0.784\\
2.14428857715431	2.14428857715431	0	0	0.792\\
2.14829659318637	2.14829659318637	0	0	0.8\\
2.15230460921844	2.15230460921844	0	0	0.808\\
2.1563126252505	2.1563126252505	0	0	0.816\\
2.16032064128257	2.16032064128257	0	0	0.824\\
2.16432865731463	2.16432865731463	0	0	0.832\\
2.16833667334669	2.16833667334669	0	0	0.84\\
2.17234468937876	2.17234468937876	0	0	0.848\\
2.17635270541082	2.17635270541082	0	0	0.856\\
2.18036072144289	2.18036072144289	0	0	0.864\\
2.18436873747495	2.18436873747495	0	0	0.872\\
2.18837675350701	2.18837675350701	0	0	0.88\\
2.19238476953908	2.19238476953908	0	0	0.888\\
2.19639278557114	2.19639278557114	0	0	0.896\\
2.20040080160321	2.20040080160321	0	0	0.904\\
2.20440881763527	2.20440881763527	0	0	0.912\\
2.20841683366733	2.20841683366733	0	0	0.92\\
2.2124248496994	2.2124248496994	0	0	0.928\\
2.21643286573146	2.21643286573146	0	0	0.936\\
2.22044088176353	2.22044088176353	0	0	0.944\\
2.22444889779559	2.22444889779559	0	0	0.952\\
2.22845691382766	2.22845691382766	0	0	0.96\\
2.23246492985972	2.23246492985972	0	0	0.968\\
2.23647294589178	2.23647294589178	0	0	0.976\\
2.24048096192385	2.24048096192385	0	0	0.984\\
2.24448897795591	2.24448897795591	0	0	0.992\\
2.24849699398798	2.24849699398798	0	0	1\\
2.25250501002004	2.25250501002004	0	0.008	1\\
2.2565130260521	2.2565130260521	0	0.016	1\\
2.26052104208417	2.26052104208417	0	0.024	1\\
2.26452905811623	2.26452905811623	0	0.032	1\\
2.2685370741483	2.2685370741483	0	0.04	1\\
2.27254509018036	2.27254509018036	0	0.048	1\\
2.27655310621242	2.27655310621242	0	0.056	1\\
2.28056112224449	2.28056112224449	0	0.064	1\\
2.28456913827655	2.28456913827655	0	0.072	1\\
2.28857715430862	2.28857715430862	0	0.08	1\\
2.29258517034068	2.29258517034068	0	0.088	1\\
2.29659318637275	2.29659318637275	0	0.096	1\\
2.30060120240481	2.30060120240481	0	0.104	1\\
2.30460921843687	2.30460921843687	0	0.112	1\\
2.30861723446894	2.30861723446894	0	0.12	1\\
2.312625250501	2.312625250501	0	0.128	1\\
2.31663326653307	2.31663326653307	0	0.136	1\\
2.32064128256513	2.32064128256513	0	0.144	1\\
2.32464929859719	2.32464929859719	0	0.152	1\\
2.32865731462926	2.32865731462926	0	0.16	1\\
2.33266533066132	2.33266533066132	0	0.168	1\\
2.33667334669339	2.33667334669339	0	0.176	1\\
2.34068136272545	2.34068136272545	0	0.184	1\\
2.34468937875751	2.34468937875751	0	0.192	1\\
2.34869739478958	2.34869739478958	0	0.2	1\\
2.35270541082164	2.35270541082164	0	0.208	1\\
2.35671342685371	2.35671342685371	0	0.216	1\\
2.36072144288577	2.36072144288577	0	0.224	1\\
2.36472945891784	2.36472945891784	0	0.232	1\\
2.3687374749499	2.3687374749499	0	0.24	1\\
2.37274549098196	2.37274549098196	0	0.248	1\\
2.37675350701403	2.37675350701403	0	0.256	1\\
2.38076152304609	2.38076152304609	0	0.264	1\\
2.38476953907816	2.38476953907816	0	0.272	1\\
2.38877755511022	2.38877755511022	0	0.28	1\\
2.39278557114228	2.39278557114228	0	0.288	1\\
2.39679358717435	2.39679358717435	0	0.296	1\\
2.40080160320641	2.40080160320641	0	0.304	1\\
2.40480961923848	2.40480961923848	0	0.312	1\\
2.40881763527054	2.40881763527054	0	0.32	1\\
2.41282565130261	2.41282565130261	0	0.328	1\\
2.41683366733467	2.41683366733467	0	0.336	1\\
2.42084168336673	2.42084168336673	0	0.344	1\\
2.4248496993988	2.4248496993988	0	0.352	1\\
2.42885771543086	2.42885771543086	0	0.36	1\\
2.43286573146293	2.43286573146293	0	0.368	1\\
2.43687374749499	2.43687374749499	0	0.376	1\\
2.44088176352705	2.44088176352705	0	0.384	1\\
2.44488977955912	2.44488977955912	0	0.392	1\\
2.44889779559118	2.44889779559118	0	0.4	1\\
2.45290581162325	2.45290581162325	0	0.408	1\\
2.45691382765531	2.45691382765531	0	0.416	1\\
2.46092184368737	2.46092184368737	0	0.424	1\\
2.46492985971944	2.46492985971944	0	0.432	1\\
2.4689378757515	2.4689378757515	0	0.44	1\\
2.47294589178357	2.47294589178357	0	0.448	1\\
2.47695390781563	2.47695390781563	0	0.456	1\\
2.4809619238477	2.4809619238477	0	0.464	1\\
2.48496993987976	2.48496993987976	0	0.472	1\\
2.48897795591182	2.48897795591182	0	0.48	1\\
2.49298597194389	2.49298597194389	0	0.488	1\\
2.49699398797595	2.49699398797595	0	0.496	1\\
2.50100200400802	2.50100200400802	0	0.504	1\\
2.50501002004008	2.50501002004008	0	0.512	1\\
2.50901803607214	2.50901803607214	0	0.52	1\\
2.51302605210421	2.51302605210421	0	0.528	1\\
2.51703406813627	2.51703406813627	0	0.536	1\\
2.52104208416834	2.52104208416834	0	0.544	1\\
2.5250501002004	2.5250501002004	0	0.552	1\\
2.52905811623247	2.52905811623247	0	0.56	1\\
2.53306613226453	2.53306613226453	0	0.568	1\\
2.53707414829659	2.53707414829659	0	0.576	1\\
2.54108216432866	2.54108216432866	0	0.584	1\\
2.54509018036072	2.54509018036072	0	0.592	1\\
2.54909819639279	2.54909819639279	0	0.6	1\\
2.55310621242485	2.55310621242485	0	0.608	1\\
2.55711422845691	2.55711422845691	0	0.616	1\\
2.56112224448898	2.56112224448898	0	0.624	1\\
2.56513026052104	2.56513026052104	0	0.632	1\\
2.56913827655311	2.56913827655311	0	0.64	1\\
2.57314629258517	2.57314629258517	0	0.648	1\\
2.57715430861723	2.57715430861723	0	0.656	1\\
2.5811623246493	2.5811623246493	0	0.664	1\\
2.58517034068136	2.58517034068136	0	0.672	1\\
2.58917835671343	2.58917835671343	0	0.68	1\\
2.59318637274549	2.59318637274549	0	0.688	1\\
2.59719438877756	2.59719438877756	0	0.696	1\\
2.60120240480962	2.60120240480962	0	0.704	1\\
2.60521042084168	2.60521042084168	0	0.712	1\\
2.60921843687375	2.60921843687375	0	0.72	1\\
2.61322645290581	2.61322645290581	0	0.728	1\\
2.61723446893788	2.61723446893788	0	0.736	1\\
2.62124248496994	2.62124248496994	0	0.744	1\\
2.625250501002	2.625250501002	0	0.752	1\\
2.62925851703407	2.62925851703407	0	0.76	1\\
2.63326653306613	2.63326653306613	0	0.768	1\\
2.6372745490982	2.6372745490982	0	0.776	1\\
2.64128256513026	2.64128256513026	0	0.784	1\\
2.64529058116232	2.64529058116232	0	0.792	1\\
2.64929859719439	2.64929859719439	0	0.8	1\\
2.65330661322645	2.65330661322645	0	0.808	1\\
2.65731462925852	2.65731462925852	0	0.816	1\\
2.66132264529058	2.66132264529058	0	0.824	1\\
2.66533066132265	2.66533066132265	0	0.832	1\\
2.66933867735471	2.66933867735471	0	0.84	1\\
2.67334669338677	2.67334669338677	0	0.848	1\\
2.67735470941884	2.67735470941884	0	0.856	1\\
2.6813627254509	2.6813627254509	0	0.864	1\\
2.68537074148297	2.68537074148297	0	0.872	1\\
2.68937875751503	2.68937875751503	0	0.88	1\\
2.69338677354709	2.69338677354709	0	0.888	1\\
2.69739478957916	2.69739478957916	0	0.896	1\\
2.70140280561122	2.70140280561122	0	0.904	1\\
2.70541082164329	2.70541082164329	0	0.912	1\\
2.70941883767535	2.70941883767535	0	0.92	1\\
2.71342685370742	2.71342685370742	0	0.928	1\\
2.71743486973948	2.71743486973948	0	0.936	1\\
2.72144288577154	2.72144288577154	0	0.944	1\\
2.72545090180361	2.72545090180361	0	0.952	1\\
2.72945891783567	2.72945891783567	0	0.96	1\\
2.73346693386774	2.73346693386774	0	0.968	1\\
2.7374749498998	2.7374749498998	0	0.976	1\\
2.74148296593186	2.74148296593186	0	0.984	1\\
2.74549098196393	2.74549098196393	0	0.992	1\\
2.74949899799599	2.74949899799599	0	1	1\\
2.75350701402806	2.75350701402806	0.008	1	0.992\\
2.75751503006012	2.75751503006012	0.016	1	0.984\\
2.76152304609218	2.76152304609218	0.024	1	0.976\\
2.76553106212425	2.76553106212425	0.032	1	0.968\\
2.76953907815631	2.76953907815631	0.04	1	0.96\\
2.77354709418838	2.77354709418838	0.048	1	0.952\\
2.77755511022044	2.77755511022044	0.056	1	0.944\\
2.78156312625251	2.78156312625251	0.064	1	0.936\\
2.78557114228457	2.78557114228457	0.072	1	0.928\\
2.78957915831663	2.78957915831663	0.08	1	0.92\\
2.7935871743487	2.7935871743487	0.088	1	0.912\\
2.79759519038076	2.79759519038076	0.096	1	0.904\\
2.80160320641283	2.80160320641283	0.104	1	0.896\\
2.80561122244489	2.80561122244489	0.112	1	0.888\\
2.80961923847695	2.80961923847695	0.12	1	0.88\\
2.81362725450902	2.81362725450902	0.128	1	0.872\\
2.81763527054108	2.81763527054108	0.136	1	0.864\\
2.82164328657315	2.82164328657315	0.144	1	0.856\\
2.82565130260521	2.82565130260521	0.152	1	0.848\\
2.82965931863727	2.82965931863727	0.16	1	0.84\\
2.83366733466934	2.83366733466934	0.168	1	0.832\\
2.8376753507014	2.8376753507014	0.176	1	0.824\\
2.84168336673347	2.84168336673347	0.184	1	0.816\\
2.84569138276553	2.84569138276553	0.192	1	0.808\\
2.8496993987976	2.8496993987976	0.2	1	0.8\\
2.85370741482966	2.85370741482966	0.208	1	0.792\\
2.85771543086172	2.85771543086172	0.216	1	0.784\\
2.86172344689379	2.86172344689379	0.224	1	0.776\\
2.86573146292585	2.86573146292585	0.232	1	0.768\\
2.86973947895792	2.86973947895792	0.24	1	0.76\\
2.87374749498998	2.87374749498998	0.248	1	0.752\\
2.87775551102204	2.87775551102204	0.256	1	0.744\\
2.88176352705411	2.88176352705411	0.264	1	0.736\\
2.88577154308617	2.88577154308617	0.272	1	0.728\\
2.88977955911824	2.88977955911824	0.28	1	0.72\\
2.8937875751503	2.8937875751503	0.288	1	0.712\\
2.89779559118236	2.89779559118236	0.296	1	0.704\\
2.90180360721443	2.90180360721443	0.304	1	0.696\\
2.90581162324649	2.90581162324649	0.312	1	0.688\\
2.90981963927856	2.90981963927856	0.32	1	0.68\\
2.91382765531062	2.91382765531062	0.328	1	0.672\\
2.91783567134269	2.91783567134269	0.336	1	0.664\\
2.92184368737475	2.92184368737475	0.344	1	0.656\\
2.92585170340681	2.92585170340681	0.352	1	0.648\\
2.92985971943888	2.92985971943888	0.36	1	0.64\\
2.93386773547094	2.93386773547094	0.368	1	0.632\\
2.93787575150301	2.93787575150301	0.376	1	0.624\\
2.94188376753507	2.94188376753507	0.384	1	0.616\\
2.94589178356713	2.94589178356713	0.392	1	0.608\\
2.9498997995992	2.9498997995992	0.4	1	0.6\\
2.95390781563126	2.95390781563126	0.408	1	0.592\\
2.95791583166333	2.95791583166333	0.416	1	0.584\\
2.96192384769539	2.96192384769539	0.424	1	0.576\\
2.96593186372745	2.96593186372745	0.432	1	0.568\\
2.96993987975952	2.96993987975952	0.44	1	0.56\\
2.97394789579158	2.97394789579158	0.448	1	0.552\\
2.97795591182365	2.97795591182365	0.456	1	0.544\\
2.98196392785571	2.98196392785571	0.464	1	0.536\\
2.98597194388778	2.98597194388778	0.472	1	0.528\\
2.98997995991984	2.98997995991984	0.48	1	0.52\\
2.9939879759519	2.9939879759519	0.488	1	0.512\\
2.99799599198397	2.99799599198397	0.496	1	0.504\\
3.00200400801603	3.00200400801603	0.504	1	0.496\\
3.0060120240481	3.0060120240481	0.512	1	0.488\\
3.01002004008016	3.01002004008016	0.52	1	0.48\\
3.01402805611222	3.01402805611222	0.528	1	0.472\\
3.01803607214429	3.01803607214429	0.536	1	0.464\\
3.02204408817635	3.02204408817635	0.544	1	0.456\\
3.02605210420842	3.02605210420842	0.552	1	0.448\\
3.03006012024048	3.03006012024048	0.56	1	0.44\\
3.03406813627254	3.03406813627254	0.568	1	0.432\\
3.03807615230461	3.03807615230461	0.576	1	0.424\\
3.04208416833667	3.04208416833667	0.584	1	0.416\\
3.04609218436874	3.04609218436874	0.592	1	0.408\\
3.0501002004008	3.0501002004008	0.6	1	0.4\\
3.05410821643287	3.05410821643287	0.608	1	0.392\\
3.05811623246493	3.05811623246493	0.616	1	0.384\\
3.06212424849699	3.06212424849699	0.624	1	0.376\\
3.06613226452906	3.06613226452906	0.632	1	0.368\\
3.07014028056112	3.07014028056112	0.64	1	0.36\\
3.07414829659319	3.07414829659319	0.648	1	0.352\\
3.07815631262525	3.07815631262525	0.656	1	0.344\\
3.08216432865731	3.08216432865731	0.664	1	0.336\\
3.08617234468938	3.08617234468938	0.672	1	0.328\\
3.09018036072144	3.09018036072144	0.68	1	0.32\\
3.09418837675351	3.09418837675351	0.688	1	0.312\\
3.09819639278557	3.09819639278557	0.696	1	0.304\\
3.10220440881764	3.10220440881764	0.704	1	0.296\\
3.1062124248497	3.1062124248497	0.712	1	0.288\\
3.11022044088176	3.11022044088176	0.72	1	0.28\\
3.11422845691383	3.11422845691383	0.728	1	0.272\\
3.11823647294589	3.11823647294589	0.736	1	0.264\\
3.12224448897796	3.12224448897796	0.744	1	0.256\\
3.12625250501002	3.12625250501002	0.752	1	0.248\\
3.13026052104208	3.13026052104208	0.76	1	0.24\\
3.13426853707415	3.13426853707415	0.768	1	0.232\\
3.13827655310621	3.13827655310621	0.776	1	0.224\\
3.14228456913828	3.14228456913828	0.784	1	0.216\\
3.14629258517034	3.14629258517034	0.792	1	0.208\\
3.1503006012024	3.1503006012024	0.8	1	0.2\\
3.15430861723447	3.15430861723447	0.808	1	0.192\\
3.15831663326653	3.15831663326653	0.816	1	0.184\\
3.1623246492986	3.1623246492986	0.824	1	0.176\\
3.16633266533066	3.16633266533066	0.832	1	0.168\\
3.17034068136273	3.17034068136273	0.84	1	0.16\\
3.17434869739479	3.17434869739479	0.848	1	0.152\\
3.17835671342685	3.17835671342685	0.856	1	0.144\\
3.18236472945892	3.18236472945892	0.864	1	0.136\\
3.18637274549098	3.18637274549098	0.872	1	0.128\\
3.19038076152305	3.19038076152305	0.88	1	0.12\\
3.19438877755511	3.19438877755511	0.888	1	0.112\\
3.19839679358717	3.19839679358717	0.896	1	0.104\\
3.20240480961924	3.20240480961924	0.904	1	0.096\\
3.2064128256513	3.2064128256513	0.912	1	0.088\\
3.21042084168337	3.21042084168337	0.92	1	0.08\\
3.21442885771543	3.21442885771543	0.928	1	0.072\\
3.2184368737475	3.2184368737475	0.936	1	0.064\\
3.22244488977956	3.22244488977956	0.944	1	0.056\\
3.22645290581162	3.22645290581162	0.952	1	0.048\\
3.23046092184369	3.23046092184369	0.96	1	0.04\\
3.23446893787575	3.23446893787575	0.968	1	0.032\\
3.23847695390782	3.23847695390782	0.976	1	0.024\\
3.24248496993988	3.24248496993988	0.984	1	0.016\\
3.24649298597194	3.24649298597194	0.992	1	0.008\\
3.25050100200401	3.25050100200401	1	1	0\\
3.25450901803607	3.25450901803607	1	0.992	0\\
3.25851703406814	3.25851703406814	1	0.984	0\\
3.2625250501002	3.2625250501002	1	0.976	0\\
3.26653306613226	3.26653306613226	1	0.968	0\\
3.27054108216433	3.27054108216433	1	0.96	0\\
3.27454909819639	3.27454909819639	1	0.952	0\\
3.27855711422846	3.27855711422846	1	0.944	0\\
3.28256513026052	3.28256513026052	1	0.936	0\\
3.28657314629258	3.28657314629258	1	0.928	0\\
3.29058116232465	3.29058116232465	1	0.92	0\\
3.29458917835671	3.29458917835671	1	0.912	0\\
3.29859719438878	3.29859719438878	1	0.904	0\\
3.30260521042084	3.30260521042084	1	0.896	0\\
3.30661322645291	3.30661322645291	1	0.888	0\\
3.31062124248497	3.31062124248497	1	0.88	0\\
3.31462925851703	3.31462925851703	1	0.872	0\\
3.3186372745491	3.3186372745491	1	0.864	0\\
3.32264529058116	3.32264529058116	1	0.856	0\\
3.32665330661323	3.32665330661323	1	0.848	0\\
3.33066132264529	3.33066132264529	1	0.84	0\\
3.33466933867735	3.33466933867735	1	0.832	0\\
3.33867735470942	3.33867735470942	1	0.824	0\\
3.34268537074148	3.34268537074148	1	0.816	0\\
3.34669338677355	3.34669338677355	1	0.808	0\\
3.35070140280561	3.35070140280561	1	0.8	0\\
3.35470941883768	3.35470941883768	1	0.792	0\\
3.35871743486974	3.35871743486974	1	0.784	0\\
3.3627254509018	3.3627254509018	1	0.776	0\\
3.36673346693387	3.36673346693387	1	0.768	0\\
3.37074148296593	3.37074148296593	1	0.76	0\\
3.374749498998	3.374749498998	1	0.752	0\\
3.37875751503006	3.37875751503006	1	0.744	0\\
3.38276553106212	3.38276553106212	1	0.736	0\\
3.38677354709419	3.38677354709419	1	0.728	0\\
3.39078156312625	3.39078156312625	1	0.72	0\\
3.39478957915832	3.39478957915832	1	0.712	0\\
3.39879759519038	3.39879759519038	1	0.704	0\\
3.40280561122244	3.40280561122244	1	0.696	0\\
3.40681362725451	3.40681362725451	1	0.688	0\\
3.41082164328657	3.41082164328657	1	0.68	0\\
3.41482965931864	3.41482965931864	1	0.672	0\\
3.4188376753507	3.4188376753507	1	0.664	0\\
3.42284569138277	3.42284569138277	1	0.656	0\\
3.42685370741483	3.42685370741483	1	0.648	0\\
3.43086172344689	3.43086172344689	1	0.64	0\\
3.43486973947896	3.43486973947896	1	0.632	0\\
3.43887775551102	3.43887775551102	1	0.624	0\\
3.44288577154309	3.44288577154309	1	0.616	0\\
3.44689378757515	3.44689378757515	1	0.608	0\\
3.45090180360721	3.45090180360721	1	0.6	0\\
3.45490981963928	3.45490981963928	1	0.592	0\\
3.45891783567134	3.45891783567134	1	0.584	0\\
3.46292585170341	3.46292585170341	1	0.576	0\\
3.46693386773547	3.46693386773547	1	0.568	0\\
3.47094188376753	3.47094188376753	1	0.56	0\\
3.4749498997996	3.4749498997996	1	0.552	0\\
3.47895791583166	3.47895791583166	1	0.544	0\\
3.48296593186373	3.48296593186373	1	0.536	0\\
3.48697394789579	3.48697394789579	1	0.528	0\\
3.49098196392786	3.49098196392786	1	0.52	0\\
3.49498997995992	3.49498997995992	1	0.512	0\\
3.49899799599198	3.49899799599198	1	0.504	0\\
3.50300601202405	3.50300601202405	1	0.496	0\\
3.50701402805611	3.50701402805611	1	0.488	0\\
3.51102204408818	3.51102204408818	1	0.48	0\\
3.51503006012024	3.51503006012024	1	0.472	0\\
3.5190380761523	3.5190380761523	1	0.464	0\\
3.52304609218437	3.52304609218437	1	0.456	0\\
3.52705410821643	3.52705410821643	1	0.448	0\\
3.5310621242485	3.5310621242485	1	0.44	0\\
3.53507014028056	3.53507014028056	1	0.432	0\\
3.53907815631263	3.53907815631263	1	0.424	0\\
3.54308617234469	3.54308617234469	1	0.416	0\\
3.54709418837675	3.54709418837675	1	0.408	0\\
3.55110220440882	3.55110220440882	1	0.4	0\\
3.55511022044088	3.55511022044088	1	0.392	0\\
3.55911823647295	3.55911823647295	1	0.384	0\\
3.56312625250501	3.56312625250501	1	0.376	0\\
3.56713426853707	3.56713426853707	1	0.368	0\\
3.57114228456914	3.57114228456914	1	0.36	0\\
3.5751503006012	3.5751503006012	1	0.352	0\\
3.57915831663327	3.57915831663327	1	0.344	0\\
3.58316633266533	3.58316633266533	1	0.336	0\\
3.58717434869739	3.58717434869739	1	0.328	0\\
3.59118236472946	3.59118236472946	1	0.32	0\\
3.59519038076152	3.59519038076152	1	0.312	0\\
3.59919839679359	3.59919839679359	1	0.304	0\\
3.60320641282565	3.60320641282565	1	0.296	0\\
3.60721442885772	3.60721442885772	1	0.288	0\\
3.61122244488978	3.61122244488978	1	0.28	0\\
3.61523046092184	3.61523046092184	1	0.272	0\\
3.61923847695391	3.61923847695391	1	0.264	0\\
3.62324649298597	3.62324649298597	1	0.256	0\\
3.62725450901804	3.62725450901804	1	0.248	0\\
3.6312625250501	3.6312625250501	1	0.24	0\\
3.63527054108216	3.63527054108216	1	0.232	0\\
3.63927855711423	3.63927855711423	1	0.224	0\\
3.64328657314629	3.64328657314629	1	0.216	0\\
3.64729458917836	3.64729458917836	1	0.208	0\\
3.65130260521042	3.65130260521042	1	0.2	0\\
3.65531062124249	3.65531062124249	1	0.192	0\\
3.65931863727455	3.65931863727455	1	0.184	0\\
3.66332665330661	3.66332665330661	1	0.176	0\\
3.66733466933868	3.66733466933868	1	0.168	0\\
3.67134268537074	3.67134268537074	1	0.16	0\\
3.67535070140281	3.67535070140281	1	0.152	0\\
3.67935871743487	3.67935871743487	1	0.144	0\\
3.68336673346693	3.68336673346693	1	0.136	0\\
3.687374749499	3.687374749499	1	0.128	0\\
3.69138276553106	3.69138276553106	1	0.12	0\\
3.69539078156313	3.69539078156313	1	0.112	0\\
3.69939879759519	3.69939879759519	1	0.104	0\\
3.70340681362725	3.70340681362725	1	0.096	0\\
3.70741482965932	3.70741482965932	1	0.088	0\\
3.71142284569138	3.71142284569138	1	0.08	0\\
3.71543086172345	3.71543086172345	1	0.072	0\\
3.71943887775551	3.71943887775551	1	0.064	0\\
3.72344689378758	3.72344689378758	1	0.056	0\\
3.72745490981964	3.72745490981964	1	0.048	0\\
3.7314629258517	3.7314629258517	1	0.04	0\\
3.73547094188377	3.73547094188377	1	0.032	0\\
3.73947895791583	3.73947895791583	1	0.024	0\\
3.7434869739479	3.7434869739479	1	0.016	0\\
3.74749498997996	3.74749498997996	1	0.008	0\\
3.75150300601202	3.75150300601202	1	0	0\\
3.75551102204409	3.75551102204409	0.992	0	0\\
3.75951903807615	3.75951903807615	0.984	0	0\\
3.76352705410822	3.76352705410822	0.976	0	0\\
3.76753507014028	3.76753507014028	0.968	0	0\\
3.77154308617234	3.77154308617234	0.96	0	0\\
3.77555110220441	3.77555110220441	0.952	0	0\\
3.77955911823647	3.77955911823647	0.944	0	0\\
3.78356713426854	3.78356713426854	0.936	0	0\\
3.7875751503006	3.7875751503006	0.928	0	0\\
3.79158316633267	3.79158316633267	0.92	0	0\\
3.79559118236473	3.79559118236473	0.912	0	0\\
3.79959919839679	3.79959919839679	0.904	0	0\\
3.80360721442886	3.80360721442886	0.896	0	0\\
3.80761523046092	3.80761523046092	0.888	0	0\\
3.81162324649299	3.81162324649299	0.88	0	0\\
3.81563126252505	3.81563126252505	0.872	0	0\\
3.81963927855711	3.81963927855711	0.864	0	0\\
3.82364729458918	3.82364729458918	0.856	0	0\\
3.82765531062124	3.82765531062124	0.848	0	0\\
3.83166332665331	3.83166332665331	0.84	0	0\\
3.83567134268537	3.83567134268537	0.832	0	0\\
3.83967935871743	3.83967935871743	0.824	0	0\\
3.8436873747495	3.8436873747495	0.816	0	0\\
3.84769539078156	3.84769539078156	0.808	0	0\\
3.85170340681363	3.85170340681363	0.8	0	0\\
3.85571142284569	3.85571142284569	0.792	0	0\\
3.85971943887776	3.85971943887776	0.784	0	0\\
3.86372745490982	3.86372745490982	0.776	0	0\\
3.86773547094188	3.86773547094188	0.768	0	0\\
3.87174348697395	3.87174348697395	0.76	0	0\\
3.87575150300601	3.87575150300601	0.752	0	0\\
3.87975951903808	3.87975951903808	0.744	0	0\\
3.88376753507014	3.88376753507014	0.736	0	0\\
3.8877755511022	3.8877755511022	0.728	0	0\\
3.89178356713427	3.89178356713427	0.72	0	0\\
3.89579158316633	3.89579158316633	0.712	0	0\\
3.8997995991984	3.8997995991984	0.704	0	0\\
3.90380761523046	3.90380761523046	0.696	0	0\\
3.90781563126253	3.90781563126253	0.688	0	0\\
3.91182364729459	3.91182364729459	0.68	0	0\\
3.91583166332665	3.91583166332665	0.672	0	0\\
3.91983967935872	3.91983967935872	0.664	0	0\\
3.92384769539078	3.92384769539078	0.656	0	0\\
3.92785571142285	3.92785571142285	0.648	0	0\\
3.93186372745491	3.93186372745491	0.64	0	0\\
3.93587174348697	3.93587174348697	0.632	0	0\\
3.93987975951904	3.93987975951904	0.624	0	0\\
3.9438877755511	3.9438877755511	0.616	0	0\\
3.94789579158317	3.94789579158317	0.608	0	0\\
3.95190380761523	3.95190380761523	0.6	0	0\\
3.95591182364729	3.95591182364729	0.592	0	0\\
3.95991983967936	3.95991983967936	0.584	0	0\\
3.96392785571142	3.96392785571142	0.576	0	0\\
3.96793587174349	3.96793587174349	0.568	0	0\\
3.97194388777555	3.97194388777555	0.56	0	0\\
3.97595190380762	3.97595190380762	0.552	0	0\\
3.97995991983968	3.97995991983968	0.544	0	0\\
3.98396793587174	3.98396793587174	0.536	0	0\\
3.98797595190381	3.98797595190381	0.528	0	0\\
3.99198396793587	3.99198396793587	0.52	0	0\\
3.99599198396794	3.99599198396794	0.512	0	0\\
4	4	0.504	0	0\\
};
\addplot[scatter, only marks, mark=o] table[row sep=crcr]{%
x	y	R	G	B\\
4	2	0	0	0.504\\
3.99599198396794	2.00400801603206	0	0	0.512\\
3.99198396793587	2.00801603206413	0	0	0.52\\
3.98797595190381	2.01202404809619	0	0	0.528\\
3.98396793587174	2.01603206412826	0	0	0.536\\
3.97995991983968	2.02004008016032	0	0	0.544\\
3.97595190380762	2.02404809619238	0	0	0.552\\
3.97194388777555	2.02805611222445	0	0	0.56\\
3.96793587174349	2.03206412825651	0	0	0.568\\
3.96392785571142	2.03607214428858	0	0	0.576\\
3.95991983967936	2.04008016032064	0	0	0.584\\
3.95591182364729	2.04408817635271	0	0	0.592\\
3.95190380761523	2.04809619238477	0	0	0.6\\
3.94789579158317	2.05210420841683	0	0	0.608\\
3.9438877755511	2.0561122244489	0	0	0.616\\
3.93987975951904	2.06012024048096	0	0	0.624\\
3.93587174348697	2.06412825651303	0	0	0.632\\
3.93186372745491	2.06813627254509	0	0	0.64\\
3.92785571142285	2.07214428857715	0	0	0.648\\
3.92384769539078	2.07615230460922	0	0	0.656\\
3.91983967935872	2.08016032064128	0	0	0.664\\
3.91583166332665	2.08416833667335	0	0	0.672\\
3.91182364729459	2.08817635270541	0	0	0.68\\
3.90781563126253	2.09218436873747	0	0	0.688\\
3.90380761523046	2.09619238476954	0	0	0.696\\
3.8997995991984	2.1002004008016	0	0	0.704\\
3.89579158316633	2.10420841683367	0	0	0.712\\
3.89178356713427	2.10821643286573	0	0	0.72\\
3.8877755511022	2.1122244488978	0	0	0.728\\
3.88376753507014	2.11623246492986	0	0	0.736\\
3.87975951903808	2.12024048096192	0	0	0.744\\
3.87575150300601	2.12424849699399	0	0	0.752\\
3.87174348697395	2.12825651302605	0	0	0.76\\
3.86773547094188	2.13226452905812	0	0	0.768\\
3.86372745490982	2.13627254509018	0	0	0.776\\
3.85971943887776	2.14028056112224	0	0	0.784\\
3.85571142284569	2.14428857715431	0	0	0.792\\
3.85170340681363	2.14829659318637	0	0	0.8\\
3.84769539078156	2.15230460921844	0	0	0.808\\
3.8436873747495	2.1563126252505	0	0	0.816\\
3.83967935871743	2.16032064128257	0	0	0.824\\
3.83567134268537	2.16432865731463	0	0	0.832\\
3.83166332665331	2.16833667334669	0	0	0.84\\
3.82765531062124	2.17234468937876	0	0	0.848\\
3.82364729458918	2.17635270541082	0	0	0.856\\
3.81963927855711	2.18036072144289	0	0	0.864\\
3.81563126252505	2.18436873747495	0	0	0.872\\
3.81162324649299	2.18837675350701	0	0	0.88\\
3.80761523046092	2.19238476953908	0	0	0.888\\
3.80360721442886	2.19639278557114	0	0	0.896\\
3.79959919839679	2.20040080160321	0	0	0.904\\
3.79559118236473	2.20440881763527	0	0	0.912\\
3.79158316633267	2.20841683366733	0	0	0.92\\
3.7875751503006	2.2124248496994	0	0	0.928\\
3.78356713426854	2.21643286573146	0	0	0.936\\
3.77955911823647	2.22044088176353	0	0	0.944\\
3.77555110220441	2.22444889779559	0	0	0.952\\
3.77154308617234	2.22845691382766	0	0	0.96\\
3.76753507014028	2.23246492985972	0	0	0.968\\
3.76352705410822	2.23647294589178	0	0	0.976\\
3.75951903807615	2.24048096192385	0	0	0.984\\
3.75551102204409	2.24448897795591	0	0	0.992\\
3.75150300601202	2.24849699398798	0	0	1\\
3.74749498997996	2.25250501002004	0	0.008	1\\
3.7434869739479	2.2565130260521	0	0.016	1\\
3.73947895791583	2.26052104208417	0	0.024	1\\
3.73547094188377	2.26452905811623	0	0.032	1\\
3.7314629258517	2.2685370741483	0	0.04	1\\
3.72745490981964	2.27254509018036	0	0.048	1\\
3.72344689378758	2.27655310621242	0	0.056	1\\
3.71943887775551	2.28056112224449	0	0.064	1\\
3.71543086172345	2.28456913827655	0	0.072	1\\
3.71142284569138	2.28857715430862	0	0.08	1\\
3.70741482965932	2.29258517034068	0	0.088	1\\
3.70340681362725	2.29659318637275	0	0.096	1\\
3.69939879759519	2.30060120240481	0	0.104	1\\
3.69539078156313	2.30460921843687	0	0.112	1\\
3.69138276553106	2.30861723446894	0	0.12	1\\
3.687374749499	2.312625250501	0	0.128	1\\
3.68336673346693	2.31663326653307	0	0.136	1\\
3.67935871743487	2.32064128256513	0	0.144	1\\
3.67535070140281	2.32464929859719	0	0.152	1\\
3.67134268537074	2.32865731462926	0	0.16	1\\
3.66733466933868	2.33266533066132	0	0.168	1\\
3.66332665330661	2.33667334669339	0	0.176	1\\
3.65931863727455	2.34068136272545	0	0.184	1\\
3.65531062124249	2.34468937875751	0	0.192	1\\
3.65130260521042	2.34869739478958	0	0.2	1\\
3.64729458917836	2.35270541082164	0	0.208	1\\
3.64328657314629	2.35671342685371	0	0.216	1\\
3.63927855711423	2.36072144288577	0	0.224	1\\
3.63527054108216	2.36472945891784	0	0.232	1\\
3.6312625250501	2.3687374749499	0	0.24	1\\
3.62725450901804	2.37274549098196	0	0.248	1\\
3.62324649298597	2.37675350701403	0	0.256	1\\
3.61923847695391	2.38076152304609	0	0.264	1\\
3.61523046092184	2.38476953907816	0	0.272	1\\
3.61122244488978	2.38877755511022	0	0.28	1\\
3.60721442885772	2.39278557114228	0	0.288	1\\
3.60320641282565	2.39679358717435	0	0.296	1\\
3.59919839679359	2.40080160320641	0	0.304	1\\
3.59519038076152	2.40480961923848	0	0.312	1\\
3.59118236472946	2.40881763527054	0	0.32	1\\
3.58717434869739	2.41282565130261	0	0.328	1\\
3.58316633266533	2.41683366733467	0	0.336	1\\
3.57915831663327	2.42084168336673	0	0.344	1\\
3.5751503006012	2.4248496993988	0	0.352	1\\
3.57114228456914	2.42885771543086	0	0.36	1\\
3.56713426853707	2.43286573146293	0	0.368	1\\
3.56312625250501	2.43687374749499	0	0.376	1\\
3.55911823647295	2.44088176352705	0	0.384	1\\
3.55511022044088	2.44488977955912	0	0.392	1\\
3.55110220440882	2.44889779559118	0	0.4	1\\
3.54709418837675	2.45290581162325	0	0.408	1\\
3.54308617234469	2.45691382765531	0	0.416	1\\
3.53907815631263	2.46092184368737	0	0.424	1\\
3.53507014028056	2.46492985971944	0	0.432	1\\
3.5310621242485	2.4689378757515	0	0.44	1\\
3.52705410821643	2.47294589178357	0	0.448	1\\
3.52304609218437	2.47695390781563	0	0.456	1\\
3.5190380761523	2.4809619238477	0	0.464	1\\
3.51503006012024	2.48496993987976	0	0.472	1\\
3.51102204408818	2.48897795591182	0	0.48	1\\
3.50701402805611	2.49298597194389	0	0.488	1\\
3.50300601202405	2.49699398797595	0	0.496	1\\
3.49899799599198	2.50100200400802	0	0.504	1\\
3.49498997995992	2.50501002004008	0	0.512	1\\
3.49098196392786	2.50901803607214	0	0.52	1\\
3.48697394789579	2.51302605210421	0	0.528	1\\
3.48296593186373	2.51703406813627	0	0.536	1\\
3.47895791583166	2.52104208416834	0	0.544	1\\
3.4749498997996	2.5250501002004	0	0.552	1\\
3.47094188376753	2.52905811623247	0	0.56	1\\
3.46693386773547	2.53306613226453	0	0.568	1\\
3.46292585170341	2.53707414829659	0	0.576	1\\
3.45891783567134	2.54108216432866	0	0.584	1\\
3.45490981963928	2.54509018036072	0	0.592	1\\
3.45090180360721	2.54909819639279	0	0.6	1\\
3.44689378757515	2.55310621242485	0	0.608	1\\
3.44288577154309	2.55711422845691	0	0.616	1\\
3.43887775551102	2.56112224448898	0	0.624	1\\
3.43486973947896	2.56513026052104	0	0.632	1\\
3.43086172344689	2.56913827655311	0	0.64	1\\
3.42685370741483	2.57314629258517	0	0.648	1\\
3.42284569138277	2.57715430861723	0	0.656	1\\
3.4188376753507	2.5811623246493	0	0.664	1\\
3.41482965931864	2.58517034068136	0	0.672	1\\
3.41082164328657	2.58917835671343	0	0.68	1\\
3.40681362725451	2.59318637274549	0	0.688	1\\
3.40280561122244	2.59719438877756	0	0.696	1\\
3.39879759519038	2.60120240480962	0	0.704	1\\
3.39478957915832	2.60521042084168	0	0.712	1\\
3.39078156312625	2.60921843687375	0	0.72	1\\
3.38677354709419	2.61322645290581	0	0.728	1\\
3.38276553106212	2.61723446893788	0	0.736	1\\
3.37875751503006	2.62124248496994	0	0.744	1\\
3.374749498998	2.625250501002	0	0.752	1\\
3.37074148296593	2.62925851703407	0	0.76	1\\
3.36673346693387	2.63326653306613	0	0.768	1\\
3.3627254509018	2.6372745490982	0	0.776	1\\
3.35871743486974	2.64128256513026	0	0.784	1\\
3.35470941883768	2.64529058116232	0	0.792	1\\
3.35070140280561	2.64929859719439	0	0.8	1\\
3.34669338677355	2.65330661322645	0	0.808	1\\
3.34268537074148	2.65731462925852	0	0.816	1\\
3.33867735470942	2.66132264529058	0	0.824	1\\
3.33466933867735	2.66533066132265	0	0.832	1\\
3.33066132264529	2.66933867735471	0	0.84	1\\
3.32665330661323	2.67334669338677	0	0.848	1\\
3.32264529058116	2.67735470941884	0	0.856	1\\
3.3186372745491	2.6813627254509	0	0.864	1\\
3.31462925851703	2.68537074148297	0	0.872	1\\
3.31062124248497	2.68937875751503	0	0.88	1\\
3.30661322645291	2.69338677354709	0	0.888	1\\
3.30260521042084	2.69739478957916	0	0.896	1\\
3.29859719438878	2.70140280561122	0	0.904	1\\
3.29458917835671	2.70541082164329	0	0.912	1\\
3.29058116232465	2.70941883767535	0	0.92	1\\
3.28657314629258	2.71342685370742	0	0.928	1\\
3.28256513026052	2.71743486973948	0	0.936	1\\
3.27855711422846	2.72144288577154	0	0.944	1\\
3.27454909819639	2.72545090180361	0	0.952	1\\
3.27054108216433	2.72945891783567	0	0.96	1\\
3.26653306613226	2.73346693386774	0	0.968	1\\
3.2625250501002	2.7374749498998	0	0.976	1\\
3.25851703406814	2.74148296593186	0	0.984	1\\
3.25450901803607	2.74549098196393	0	0.992	1\\
3.25050100200401	2.74949899799599	0	1	1\\
3.24649298597194	2.75350701402806	0.008	1	0.992\\
3.24248496993988	2.75751503006012	0.016	1	0.984\\
3.23847695390782	2.76152304609218	0.024	1	0.976\\
3.23446893787575	2.76553106212425	0.032	1	0.968\\
3.23046092184369	2.76953907815631	0.04	1	0.96\\
3.22645290581162	2.77354709418838	0.048	1	0.952\\
3.22244488977956	2.77755511022044	0.056	1	0.944\\
3.21843687374749	2.78156312625251	0.064	1	0.936\\
3.21442885771543	2.78557114228457	0.072	1	0.928\\
3.21042084168337	2.78957915831663	0.08	1	0.92\\
3.2064128256513	2.7935871743487	0.088	1	0.912\\
3.20240480961924	2.79759519038076	0.096	1	0.904\\
3.19839679358717	2.80160320641283	0.104	1	0.896\\
3.19438877755511	2.80561122244489	0.112	1	0.888\\
3.19038076152305	2.80961923847695	0.12	1	0.88\\
3.18637274549098	2.81362725450902	0.128	1	0.872\\
3.18236472945892	2.81763527054108	0.136	1	0.864\\
3.17835671342685	2.82164328657315	0.144	1	0.856\\
3.17434869739479	2.82565130260521	0.152	1	0.848\\
3.17034068136273	2.82965931863727	0.16	1	0.84\\
3.16633266533066	2.83366733466934	0.168	1	0.832\\
3.1623246492986	2.8376753507014	0.176	1	0.824\\
3.15831663326653	2.84168336673347	0.184	1	0.816\\
3.15430861723447	2.84569138276553	0.192	1	0.808\\
3.1503006012024	2.8496993987976	0.2	1	0.8\\
3.14629258517034	2.85370741482966	0.208	1	0.792\\
3.14228456913828	2.85771543086172	0.216	1	0.784\\
3.13827655310621	2.86172344689379	0.224	1	0.776\\
3.13426853707415	2.86573146292585	0.232	1	0.768\\
3.13026052104208	2.86973947895792	0.24	1	0.76\\
3.12625250501002	2.87374749498998	0.248	1	0.752\\
3.12224448897796	2.87775551102204	0.256	1	0.744\\
3.11823647294589	2.88176352705411	0.264	1	0.736\\
3.11422845691383	2.88577154308617	0.272	1	0.728\\
3.11022044088176	2.88977955911824	0.28	1	0.72\\
3.1062124248497	2.8937875751503	0.288	1	0.712\\
3.10220440881764	2.89779559118236	0.296	1	0.704\\
3.09819639278557	2.90180360721443	0.304	1	0.696\\
3.09418837675351	2.90581162324649	0.312	1	0.688\\
3.09018036072144	2.90981963927856	0.32	1	0.68\\
3.08617234468938	2.91382765531062	0.328	1	0.672\\
3.08216432865731	2.91783567134269	0.336	1	0.664\\
3.07815631262525	2.92184368737475	0.344	1	0.656\\
3.07414829659319	2.92585170340681	0.352	1	0.648\\
3.07014028056112	2.92985971943888	0.36	1	0.64\\
3.06613226452906	2.93386773547094	0.368	1	0.632\\
3.06212424849699	2.93787575150301	0.376	1	0.624\\
3.05811623246493	2.94188376753507	0.384	1	0.616\\
3.05410821643287	2.94589178356713	0.392	1	0.608\\
3.0501002004008	2.9498997995992	0.4	1	0.6\\
3.04609218436874	2.95390781563126	0.408	1	0.592\\
3.04208416833667	2.95791583166333	0.416	1	0.584\\
3.03807615230461	2.96192384769539	0.424	1	0.576\\
3.03406813627255	2.96593186372745	0.432	1	0.568\\
3.03006012024048	2.96993987975952	0.44	1	0.56\\
3.02605210420842	2.97394789579158	0.448	1	0.552\\
3.02204408817635	2.97795591182365	0.456	1	0.544\\
3.01803607214429	2.98196392785571	0.464	1	0.536\\
3.01402805611222	2.98597194388778	0.472	1	0.528\\
3.01002004008016	2.98997995991984	0.48	1	0.52\\
3.0060120240481	2.9939879759519	0.488	1	0.512\\
3.00200400801603	2.99799599198397	0.496	1	0.504\\
2.99799599198397	3.00200400801603	0.504	1	0.496\\
2.9939879759519	3.0060120240481	0.512	1	0.488\\
2.98997995991984	3.01002004008016	0.52	1	0.48\\
2.98597194388778	3.01402805611222	0.528	1	0.472\\
2.98196392785571	3.01803607214429	0.536	1	0.464\\
2.97795591182365	3.02204408817635	0.544	1	0.456\\
2.97394789579158	3.02605210420842	0.552	1	0.448\\
2.96993987975952	3.03006012024048	0.56	1	0.44\\
2.96593186372746	3.03406813627254	0.568	1	0.432\\
2.96192384769539	3.03807615230461	0.576	1	0.424\\
2.95791583166333	3.04208416833667	0.584	1	0.416\\
2.95390781563126	3.04609218436874	0.592	1	0.408\\
2.9498997995992	3.0501002004008	0.6	1	0.4\\
2.94589178356713	3.05410821643287	0.608	1	0.392\\
2.94188376753507	3.05811623246493	0.616	1	0.384\\
2.93787575150301	3.06212424849699	0.624	1	0.376\\
2.93386773547094	3.06613226452906	0.632	1	0.368\\
2.92985971943888	3.07014028056112	0.64	1	0.36\\
2.92585170340681	3.07414829659319	0.648	1	0.352\\
2.92184368737475	3.07815631262525	0.656	1	0.344\\
2.91783567134269	3.08216432865731	0.664	1	0.336\\
2.91382765531062	3.08617234468938	0.672	1	0.328\\
2.90981963927856	3.09018036072144	0.68	1	0.32\\
2.90581162324649	3.09418837675351	0.688	1	0.312\\
2.90180360721443	3.09819639278557	0.696	1	0.304\\
2.89779559118236	3.10220440881764	0.704	1	0.296\\
2.8937875751503	3.1062124248497	0.712	1	0.288\\
2.88977955911824	3.11022044088176	0.72	1	0.28\\
2.88577154308617	3.11422845691383	0.728	1	0.272\\
2.88176352705411	3.11823647294589	0.736	1	0.264\\
2.87775551102204	3.12224448897796	0.744	1	0.256\\
2.87374749498998	3.12625250501002	0.752	1	0.248\\
2.86973947895792	3.13026052104208	0.76	1	0.24\\
2.86573146292585	3.13426853707415	0.768	1	0.232\\
2.86172344689379	3.13827655310621	0.776	1	0.224\\
2.85771543086172	3.14228456913828	0.784	1	0.216\\
2.85370741482966	3.14629258517034	0.792	1	0.208\\
2.8496993987976	3.1503006012024	0.8	1	0.2\\
2.84569138276553	3.15430861723447	0.808	1	0.192\\
2.84168336673347	3.15831663326653	0.816	1	0.184\\
2.8376753507014	3.1623246492986	0.824	1	0.176\\
2.83366733466934	3.16633266533066	0.832	1	0.168\\
2.82965931863727	3.17034068136273	0.84	1	0.16\\
2.82565130260521	3.17434869739479	0.848	1	0.152\\
2.82164328657315	3.17835671342685	0.856	1	0.144\\
2.81763527054108	3.18236472945892	0.864	1	0.136\\
2.81362725450902	3.18637274549098	0.872	1	0.128\\
2.80961923847695	3.19038076152305	0.88	1	0.12\\
2.80561122244489	3.19438877755511	0.888	1	0.112\\
2.80160320641283	3.19839679358717	0.896	1	0.104\\
2.79759519038076	3.20240480961924	0.904	1	0.096\\
2.7935871743487	3.2064128256513	0.912	1	0.088\\
2.78957915831663	3.21042084168337	0.92	1	0.08\\
2.78557114228457	3.21442885771543	0.928	1	0.072\\
2.7815631262525	3.2184368737475	0.936	1	0.064\\
2.77755511022044	3.22244488977956	0.944	1	0.056\\
2.77354709418838	3.22645290581162	0.952	1	0.048\\
2.76953907815631	3.23046092184369	0.96	1	0.04\\
2.76553106212425	3.23446893787575	0.968	1	0.032\\
2.76152304609218	3.23847695390782	0.976	1	0.024\\
2.75751503006012	3.24248496993988	0.984	1	0.016\\
2.75350701402806	3.24649298597194	0.992	1	0.008\\
2.74949899799599	3.25050100200401	1	1	0\\
2.74549098196393	3.25450901803607	1	0.992	0\\
2.74148296593186	3.25851703406814	1	0.984	0\\
2.7374749498998	3.2625250501002	1	0.976	0\\
2.73346693386774	3.26653306613226	1	0.968	0\\
2.72945891783567	3.27054108216433	1	0.96	0\\
2.72545090180361	3.27454909819639	1	0.952	0\\
2.72144288577154	3.27855711422846	1	0.944	0\\
2.71743486973948	3.28256513026052	1	0.936	0\\
2.71342685370742	3.28657314629258	1	0.928	0\\
2.70941883767535	3.29058116232465	1	0.92	0\\
2.70541082164329	3.29458917835671	1	0.912	0\\
2.70140280561122	3.29859719438878	1	0.904	0\\
2.69739478957916	3.30260521042084	1	0.896	0\\
2.69338677354709	3.30661322645291	1	0.888	0\\
2.68937875751503	3.31062124248497	1	0.88	0\\
2.68537074148297	3.31462925851703	1	0.872	0\\
2.6813627254509	3.3186372745491	1	0.864	0\\
2.67735470941884	3.32264529058116	1	0.856	0\\
2.67334669338677	3.32665330661323	1	0.848	0\\
2.66933867735471	3.33066132264529	1	0.84	0\\
2.66533066132265	3.33466933867735	1	0.832	0\\
2.66132264529058	3.33867735470942	1	0.824	0\\
2.65731462925852	3.34268537074148	1	0.816	0\\
2.65330661322645	3.34669338677355	1	0.808	0\\
2.64929859719439	3.35070140280561	1	0.8	0\\
2.64529058116232	3.35470941883768	1	0.792	0\\
2.64128256513026	3.35871743486974	1	0.784	0\\
2.6372745490982	3.3627254509018	1	0.776	0\\
2.63326653306613	3.36673346693387	1	0.768	0\\
2.62925851703407	3.37074148296593	1	0.76	0\\
2.625250501002	3.374749498998	1	0.752	0\\
2.62124248496994	3.37875751503006	1	0.744	0\\
2.61723446893788	3.38276553106212	1	0.736	0\\
2.61322645290581	3.38677354709419	1	0.728	0\\
2.60921843687375	3.39078156312625	1	0.72	0\\
2.60521042084168	3.39478957915832	1	0.712	0\\
2.60120240480962	3.39879759519038	1	0.704	0\\
2.59719438877756	3.40280561122244	1	0.696	0\\
2.59318637274549	3.40681362725451	1	0.688	0\\
2.58917835671343	3.41082164328657	1	0.68	0\\
2.58517034068136	3.41482965931864	1	0.672	0\\
2.5811623246493	3.4188376753507	1	0.664	0\\
2.57715430861723	3.42284569138277	1	0.656	0\\
2.57314629258517	3.42685370741483	1	0.648	0\\
2.56913827655311	3.43086172344689	1	0.64	0\\
2.56513026052104	3.43486973947896	1	0.632	0\\
2.56112224448898	3.43887775551102	1	0.624	0\\
2.55711422845691	3.44288577154309	1	0.616	0\\
2.55310621242485	3.44689378757515	1	0.608	0\\
2.54909819639279	3.45090180360721	1	0.6	0\\
2.54509018036072	3.45490981963928	1	0.592	0\\
2.54108216432866	3.45891783567134	1	0.584	0\\
2.53707414829659	3.46292585170341	1	0.576	0\\
2.53306613226453	3.46693386773547	1	0.568	0\\
2.52905811623247	3.47094188376753	1	0.56	0\\
2.5250501002004	3.4749498997996	1	0.552	0\\
2.52104208416834	3.47895791583166	1	0.544	0\\
2.51703406813627	3.48296593186373	1	0.536	0\\
2.51302605210421	3.48697394789579	1	0.528	0\\
2.50901803607214	3.49098196392786	1	0.52	0\\
2.50501002004008	3.49498997995992	1	0.512	0\\
2.50100200400802	3.49899799599198	1	0.504	0\\
2.49699398797595	3.50300601202405	1	0.496	0\\
2.49298597194389	3.50701402805611	1	0.488	0\\
2.48897795591182	3.51102204408818	1	0.48	0\\
2.48496993987976	3.51503006012024	1	0.472	0\\
2.4809619238477	3.5190380761523	1	0.464	0\\
2.47695390781563	3.52304609218437	1	0.456	0\\
2.47294589178357	3.52705410821643	1	0.448	0\\
2.4689378757515	3.5310621242485	1	0.44	0\\
2.46492985971944	3.53507014028056	1	0.432	0\\
2.46092184368737	3.53907815631263	1	0.424	0\\
2.45691382765531	3.54308617234469	1	0.416	0\\
2.45290581162325	3.54709418837675	1	0.408	0\\
2.44889779559118	3.55110220440882	1	0.4	0\\
2.44488977955912	3.55511022044088	1	0.392	0\\
2.44088176352705	3.55911823647295	1	0.384	0\\
2.43687374749499	3.56312625250501	1	0.376	0\\
2.43286573146293	3.56713426853707	1	0.368	0\\
2.42885771543086	3.57114228456914	1	0.36	0\\
2.4248496993988	3.5751503006012	1	0.352	0\\
2.42084168336673	3.57915831663327	1	0.344	0\\
2.41683366733467	3.58316633266533	1	0.336	0\\
2.41282565130261	3.58717434869739	1	0.328	0\\
2.40881763527054	3.59118236472946	1	0.32	0\\
2.40480961923848	3.59519038076152	1	0.312	0\\
2.40080160320641	3.59919839679359	1	0.304	0\\
2.39679358717435	3.60320641282565	1	0.296	0\\
2.39278557114228	3.60721442885772	1	0.288	0\\
2.38877755511022	3.61122244488978	1	0.28	0\\
2.38476953907816	3.61523046092184	1	0.272	0\\
2.38076152304609	3.61923847695391	1	0.264	0\\
2.37675350701403	3.62324649298597	1	0.256	0\\
2.37274549098196	3.62725450901804	1	0.248	0\\
2.3687374749499	3.6312625250501	1	0.24	0\\
2.36472945891784	3.63527054108216	1	0.232	0\\
2.36072144288577	3.63927855711423	1	0.224	0\\
2.35671342685371	3.64328657314629	1	0.216	0\\
2.35270541082164	3.64729458917836	1	0.208	0\\
2.34869739478958	3.65130260521042	1	0.2	0\\
2.34468937875751	3.65531062124249	1	0.192	0\\
2.34068136272545	3.65931863727455	1	0.184	0\\
2.33667334669339	3.66332665330661	1	0.176	0\\
2.33266533066132	3.66733466933868	1	0.168	0\\
2.32865731462926	3.67134268537074	1	0.16	0\\
2.32464929859719	3.67535070140281	1	0.152	0\\
2.32064128256513	3.67935871743487	1	0.144	0\\
2.31663326653307	3.68336673346693	1	0.136	0\\
2.312625250501	3.687374749499	1	0.128	0\\
2.30861723446894	3.69138276553106	1	0.12	0\\
2.30460921843687	3.69539078156313	1	0.112	0\\
2.30060120240481	3.69939879759519	1	0.104	0\\
2.29659318637275	3.70340681362725	1	0.096	0\\
2.29258517034068	3.70741482965932	1	0.088	0\\
2.28857715430862	3.71142284569138	1	0.08	0\\
2.28456913827655	3.71543086172345	1	0.072	0\\
2.28056112224449	3.71943887775551	1	0.064	0\\
2.27655310621242	3.72344689378758	1	0.056	0\\
2.27254509018036	3.72745490981964	1	0.048	0\\
2.2685370741483	3.7314629258517	1	0.04	0\\
2.26452905811623	3.73547094188377	1	0.032	0\\
2.26052104208417	3.73947895791583	1	0.024	0\\
2.2565130260521	3.7434869739479	1	0.016	0\\
2.25250501002004	3.74749498997996	1	0.008	0\\
2.24849699398798	3.75150300601202	1	0	0\\
2.24448897795591	3.75551102204409	0.992	0	0\\
2.24048096192385	3.75951903807615	0.984	0	0\\
2.23647294589178	3.76352705410822	0.976	0	0\\
2.23246492985972	3.76753507014028	0.968	0	0\\
2.22845691382766	3.77154308617234	0.96	0	0\\
2.22444889779559	3.77555110220441	0.952	0	0\\
2.22044088176353	3.77955911823647	0.944	0	0\\
2.21643286573146	3.78356713426854	0.936	0	0\\
2.2124248496994	3.7875751503006	0.928	0	0\\
2.20841683366733	3.79158316633267	0.92	0	0\\
2.20440881763527	3.79559118236473	0.912	0	0\\
2.20040080160321	3.79959919839679	0.904	0	0\\
2.19639278557114	3.80360721442886	0.896	0	0\\
2.19238476953908	3.80761523046092	0.888	0	0\\
2.18837675350701	3.81162324649299	0.88	0	0\\
2.18436873747495	3.81563126252505	0.872	0	0\\
2.18036072144289	3.81963927855711	0.864	0	0\\
2.17635270541082	3.82364729458918	0.856	0	0\\
2.17234468937876	3.82765531062124	0.848	0	0\\
2.16833667334669	3.83166332665331	0.84	0	0\\
2.16432865731463	3.83567134268537	0.832	0	0\\
2.16032064128257	3.83967935871743	0.824	0	0\\
2.1563126252505	3.8436873747495	0.816	0	0\\
2.15230460921844	3.84769539078156	0.808	0	0\\
2.14829659318637	3.85170340681363	0.8	0	0\\
2.14428857715431	3.85571142284569	0.792	0	0\\
2.14028056112224	3.85971943887776	0.784	0	0\\
2.13627254509018	3.86372745490982	0.776	0	0\\
2.13226452905812	3.86773547094188	0.768	0	0\\
2.12825651302605	3.87174348697395	0.76	0	0\\
2.12424849699399	3.87575150300601	0.752	0	0\\
2.12024048096192	3.87975951903808	0.744	0	0\\
2.11623246492986	3.88376753507014	0.736	0	0\\
2.1122244488978	3.8877755511022	0.728	0	0\\
2.10821643286573	3.89178356713427	0.72	0	0\\
2.10420841683367	3.89579158316633	0.712	0	0\\
2.1002004008016	3.8997995991984	0.704	0	0\\
2.09619238476954	3.90380761523046	0.696	0	0\\
2.09218436873747	3.90781563126253	0.688	0	0\\
2.08817635270541	3.91182364729459	0.68	0	0\\
2.08416833667335	3.91583166332665	0.672	0	0\\
2.08016032064128	3.91983967935872	0.664	0	0\\
2.07615230460922	3.92384769539078	0.656	0	0\\
2.07214428857715	3.92785571142285	0.648	0	0\\
2.06813627254509	3.93186372745491	0.64	0	0\\
2.06412825651303	3.93587174348697	0.632	0	0\\
2.06012024048096	3.93987975951904	0.624	0	0\\
2.0561122244489	3.9438877755511	0.616	0	0\\
2.05210420841683	3.94789579158317	0.608	0	0\\
2.04809619238477	3.95190380761523	0.6	0	0\\
2.04408817635271	3.95591182364729	0.592	0	0\\
2.04008016032064	3.95991983967936	0.584	0	0\\
2.03607214428858	3.96392785571142	0.576	0	0\\
2.03206412825651	3.96793587174349	0.568	0	0\\
2.02805611222445	3.97194388777555	0.56	0	0\\
2.02404809619238	3.97595190380762	0.552	0	0\\
2.02004008016032	3.97995991983968	0.544	0	0\\
2.01603206412826	3.98396793587174	0.536	0	0\\
2.01202404809619	3.98797595190381	0.528	0	0\\
2.00801603206413	3.99198396793587	0.52	0	0\\
2.00400801603206	3.99599198396794	0.512	0	0\\
2	4	0.504	0	0\\
};
\addplot[scatter, only marks, mark=x] table[row sep=crcr]{%
x	y	R	G	B\\
2.5	1.8	0	0	0.508064516129032\\
2.5	1.8	0	0	0.516129032258065\\
2.5	1.8	0	0	0.524193548387097\\
2.1	1.9	0	0	0.532258064516129\\
2.1	1.9	0	0	0.540322580645161\\
2.1	1.9	0	0	0.548387096774194\\
2.1	2	0	0	0.556451612903226\\
2.1	2	0	0	0.564516129032258\\
2.1	2	0	0	0.57258064516129\\
2.1	2	0	0	0.580645161290323\\
2.1	2	0	0	0.588709677419355\\
2.1	2	0	0	0.596774193548387\\
2.1	2	0	0	0.604838709677419\\
2.1	2	0	0	0.612903225806452\\
2.1	2	0	0	0.620967741935484\\
2.1	2	0	0	0.629032258064516\\
2.1	2	0	0	0.637096774193548\\
3.9	2	0	0	0.645161290322581\\
3.9	2	0	0	0.653225806451613\\
3.9	2	0	0	0.661290322580645\\
3.9	2	0	0	0.669354838709677\\
3.9	2	0	0	0.67741935483871\\
3.9	2	0	0	0.685483870967742\\
3.9	2	0	0	0.693548387096774\\
3.9	2	0	0	0.701612903225806\\
3.9	2	0	0	0.709677419354839\\
3.9	2	0	0	0.717741935483871\\
2.1	2	0	0	0.725806451612903\\
2.1	2	0	0	0.733870967741935\\
2.1	2	0	0	0.741935483870968\\
2.1	2	0	0	0.75\\
2.1	2	0	0	0.758064516129032\\
2.1	2	0	0	0.766129032258065\\
2.1	2	0	0	0.774193548387097\\
2.1	2	0	0	0.782258064516129\\
2.1	2	0	0	0.790322580645161\\
2.1	2	0	0	0.798387096774194\\
3.9	2.1	0	0	0.806451612903226\\
3.9	2.1	0	0	0.814516129032258\\
3.9	2.1	0	0	0.82258064516129\\
3.9	2.1	0	0	0.830645161290323\\
3.9	2.1	0	0	0.838709677419355\\
3.9	2.1	0	0	0.846774193548387\\
3.9	2.1	0	0	0.854838709677419\\
3.9	2.1	0	0	0.862903225806452\\
3.9	2.1	0	0	0.870967741935484\\
3.9	2.1	0	0	0.879032258064516\\
3.9	2.1	0	0	0.887096774193548\\
3.9	2.1	0	0	0.895161290322581\\
3.9	2.1	0	0	0.903225806451613\\
3.9	2.1	0	0	0.911290322580645\\
3.9	2.1	0	0	0.919354838709677\\
3.9	2.1	0	0	0.92741935483871\\
3.9	2.1	0	0	0.935483870967742\\
3.9	2.1	0	0	0.943548387096774\\
3.9	2.1	0	0	0.951612903225806\\
3.9	2.1	0	0	0.959677419354839\\
3.9	2.1	0	0	0.967741935483871\\
3.9	2.1	0	0	0.975806451612903\\
3.9	2.1	0	0	0.983870967741935\\
3.9	2.1	0	0	0.991935483870968\\
3.9	2.1	0	0	1\\
3.9	2.1	0	0.00806451612903226	1\\
3.9	2.1	0	0.0161290322580645	1\\
3.9	2.1	0	0.0241935483870968	1\\
3.9	2.1	0	0.032258064516129	1\\
3.9	2.1	0	0.0403225806451613	1\\
3.9	2.1	0	0.0483870967741935	1\\
3.9	2.1	0	0.0564516129032258	1\\
3.9	2.1	0	0.0645161290322581	1\\
3.9	2.1	0	0.0725806451612903	1\\
3.9	2.1	0	0.0806451612903226	1\\
3.9	2.1	0	0.0887096774193548	1\\
3.9	2.1	0	0.0967741935483871	1\\
3.9	2.1	0	0.104838709677419	1\\
3.9	2.1	0	0.112903225806452	1\\
3.9	2.1	0	0.120967741935484	1\\
3.9	2.1	0	0.129032258064516	1\\
3.9	2.1	0	0.137096774193548	1\\
3.9	2.1	0	0.145161290322581	1\\
3.9	2.1	0	0.153225806451613	1\\
3.9	2.1	0	0.161290322580645	1\\
3.9	2.1	0	0.169354838709677	1\\
3.9	2.1	0	0.17741935483871	1\\
3.9	2.2	0	0.185483870967742	1\\
3.9	2.2	0	0.193548387096774	1\\
3.9	2.2	0	0.201612903225806	1\\
3.8	2.2	0	0.209677419354839	1\\
3.8	2.2	0	0.217741935483871	1\\
3.8	2.2	0	0.225806451612903	1\\
3.8	2.2	0	0.233870967741935	1\\
3.8	2.2	0	0.241935483870968	1\\
3.8	2.2	0	0.25	1\\
3.8	2.2	0	0.258064516129032	1\\
3.8	2.2	0	0.266129032258065	1\\
3.8	2.2	0	0.274193548387097	1\\
3.8	2.2	0	0.282258064516129	1\\
3.8	2.2	0	0.290322580645161	1\\
3.8	2.2	0	0.298387096774194	1\\
3.8	2.2	0	0.306451612903226	1\\
3.8	2.2	0	0.314516129032258	1\\
3.8	2.2	0	0.32258064516129	1\\
3.8	2.2	0	0.330645161290323	1\\
3.8	2.2	0	0.338709677419355	1\\
3.8	2.2	0	0.346774193548387	1\\
3.8	2.2	0	0.354838709677419	1\\
3.8	2.2	0	0.362903225806452	1\\
3.8	2.2	0	0.370967741935484	1\\
3.8	2.2	0	0.379032258064516	1\\
3.8	2.2	0	0.387096774193548	1\\
3.8	2.2	0	0.395161290322581	1\\
3.8	2.2	0	0.403225806451613	1\\
3.8	2.2	0	0.411290322580645	1\\
3.8	2.2	0	0.419354838709677	1\\
3.8	2.2	0	0.42741935483871	1\\
3.8	2.2	0	0.435483870967742	1\\
3.8	2.2	0	0.443548387096774	1\\
3.8	2.2	0	0.451612903225806	1\\
3.8	2.2	0	0.459677419354839	1\\
1.2	3.8	0	0.467741935483871	1\\
1.2	3.8	0	0.475806451612903	1\\
1.2	3.8	0	0.483870967741935	1\\
3.8	2.2	0	0.491935483870968	1\\
3.8	2.2	0	0.5	1\\
3.8	2.2	0	0.508064516129032	1\\
3.8	2.2	0	0.516129032258065	1\\
3.8	2.2	0	0.524193548387097	1\\
3.8	2.2	0	0.532258064516129	1\\
3.8	2.2	0	0.540322580645161	1\\
3.8	2.2	0	0.548387096774194	1\\
1.2	2.2	0	0.556451612903226	1\\
1.2	2.2	0	0.564516129032258	1\\
1.2	2.2	0	0.57258064516129	1\\
1.2	2.2	0	0.580645161290323	1\\
1.2	2.2	0	0.588709677419355	1\\
3.9	2.2	0	0.596774193548387	1\\
3.9	2.2	0	0.604838709677419	1\\
2.9	1.2	0	0.612903225806452	1\\
2.9	1.2	0	0.620967741935484	1\\
1.2	2.2	0	0.629032258064516	1\\
1.2	2.2	0	0.637096774193548	1\\
1.2	2.2	0	0.645161290322581	1\\
1.2	2.2	0	0.653225806451613	1\\
1.2	2.2	0	0.661290322580645	1\\
1.2	2.2	0	0.669354838709677	1\\
2.8	1.2	0	0.67741935483871	1\\
2.8	1.2	0	0.685483870967742	1\\
2.8	1.2	0	0.693548387096774	1\\
2.8	1.2	0	0.701612903225806	1\\
3.6	2.4	0	0.709677419354839	1\\
3.6	2.4	0	0.717741935483871	1\\
3.6	2.4	0	0.725806451612903	1\\
3.5	2.5	0	0.733870967741935	1\\
3.5	2.5	0	0.741935483870968	1\\
3.5	2.5	0	0.75	1\\
3.5	2.5	0	0.758064516129032	1\\
3.5	2.5	0	0.766129032258065	1\\
3.5	2.5	0	0.774193548387097	1\\
3.4	2.5	0	0.782258064516129	1\\
3.4	2.5	0	0.790322580645161	1\\
3.4	2.5	0	0.798387096774194	1\\
3.4	2.5	0	0.806451612903226	1\\
3.4	2.5	0	0.814516129032258	1\\
3.4	2.5	0	0.82258064516129	1\\
3.4	2.5	0	0.830645161290323	1\\
3.4	2.5	0	0.838709677419355	1\\
3.4	2.5	0	0.846774193548387	1\\
3.4	2.5	0	0.854838709677419	1\\
3.4	2.5	0	0.862903225806452	1\\
3.4	2.5	0	0.870967741935484	1\\
3.3	2.6	0	0.879032258064516	1\\
3.3	2.6	0	0.887096774193548	1\\
3.3	2.6	0	0.895161290322581	1\\
3.3	2.6	0	0.903225806451613	1\\
3.3	2.6	0	0.911290322580645	1\\
3.3	2.6	0	0.919354838709677	1\\
3.3	2.6	0	0.92741935483871	1\\
3.3	2.6	0	0.935483870967742	1\\
3.3	2.6	0	0.943548387096774	1\\
3.3	2.6	0	0.951612903225806	1\\
3.3	2.6	0	0.959677419354839	1\\
3.3	2.6	0	0.967741935483871	1\\
3.3	2.6	0	0.975806451612903	1\\
3.3	2.6	0	0.983870967741935	1\\
3.3	2.6	0	0.991935483870968	1\\
3.3	2.6	0	1	1\\
3.3	2.6	0.00806451612903226	1	0.991935483870968\\
3.3	2.6	0.0161290322580645	1	0.983870967741935\\
3.3	2.6	0.0241935483870968	1	0.975806451612903\\
3.3	2.6	0.032258064516129	1	0.967741935483871\\
3.3	2.6	0.0403225806451613	1	0.959677419354839\\
3.3	2.6	0.0483870967741935	1	0.951612903225806\\
3.3	2.6	0.0564516129032258	1	0.943548387096774\\
3.3	2.6	0.0645161290322581	1	0.935483870967742\\
3.3	2.6	0.0725806451612903	1	0.92741935483871\\
3.3	2.6	0.0806451612903226	1	0.919354838709677\\
3.3	2.6	0.0887096774193548	1	0.911290322580645\\
3.3	2.6	0.0967741935483871	1	0.903225806451613\\
3.3	2.6	0.104838709677419	1	0.895161290322581\\
3.3	2.6	0.112903225806452	1	0.887096774193548\\
3.3	2.6	0.120967741935484	1	0.879032258064516\\
3.3	2.6	0.129032258064516	1	0.870967741935484\\
3.3	2.6	0.137096774193548	1	0.862903225806452\\
3.3	2.6	0.145161290322581	1	0.854838709677419\\
3.3	2.6	0.153225806451613	1	0.846774193548387\\
3.3	2.6	0.161290322580645	1	0.838709677419355\\
3.3	2.6	0.169354838709677	1	0.830645161290323\\
3.2	2.6	0.17741935483871	1	0.82258064516129\\
3.2	2.6	0.185483870967742	1	0.814516129032258\\
3.2	2.6	0.193548387096774	1	0.806451612903226\\
3.1	2.7	0.201612903225806	1	0.798387096774194\\
3.1	2.7	0.209677419354839	1	0.790322580645161\\
3.1	2.7	0.217741935483871	1	0.782258064516129\\
3.1	2.7	0.225806451612903	1	0.774193548387097\\
3.1	2.7	0.233870967741935	1	0.766129032258065\\
3.1	2.7	0.241935483870968	1	0.758064516129032\\
3.1	2.7	0.25	1	0.75\\
3.1	2.7	0.258064516129032	1	0.741935483870968\\
3.1	2.7	0.266129032258065	1	0.733870967741935\\
3	2.7	0.274193548387097	1	0.725806451612903\\
3	2.7	0.282258064516129	1	0.717741935483871\\
3	2.7	0.290322580645161	1	0.709677419354839\\
3.1	2.7	0.298387096774194	1	0.701612903225806\\
3.1	2.7	0.306451612903226	1	0.693548387096774\\
3.1	2.7	0.314516129032258	1	0.685483870967742\\
3.1	2.7	0.32258064516129	1	0.67741935483871\\
3.1	2.7	0.330645161290323	1	0.669354838709677\\
3	2.8	0.338709677419355	1	0.661290322580645\\
3.1	2.7	0.346774193548387	1	0.653225806451613\\
3.1	2.7	0.354838709677419	1	0.645161290322581\\
3.1	2.7	0.362903225806452	1	0.637096774193548\\
3	2.8	0.370967741935484	1	0.629032258064516\\
3	2.8	0.379032258064516	1	0.620967741935484\\
3	2.8	0.387096774193548	1	0.612903225806452\\
3	2.8	0.395161290322581	1	0.604838709677419\\
3	2.8	0.403225806451613	1	0.596774193548387\\
3	2.8	0.411290322580645	1	0.588709677419355\\
3	2.8	0.419354838709677	1	0.580645161290323\\
3	2.8	0.42741935483871	1	0.57258064516129\\
3	2.8	0.435483870967742	1	0.564516129032258\\
3	2.8	0.443548387096774	1	0.556451612903226\\
3	2.8	0.451612903225806	1	0.548387096774194\\
3	2.8	0.459677419354839	1	0.540322580645161\\
3	2.8	0.467741935483871	1	0.532258064516129\\
3	2.8	0.475806451612903	1	0.524193548387097\\
3	2.8	0.483870967741935	1	0.516129032258065\\
3	2.8	0.491935483870968	1	0.508064516129032\\
3	2.8	0.5	1	0.5\\
3	2.8	0.508064516129032	1	0.491935483870968\\
3	2.9	0.516129032258065	1	0.483870967741935\\
3	2.9	0.524193548387097	1	0.475806451612903\\
3	2.9	0.532258064516129	1	0.467741935483871\\
3	2.9	0.540322580645161	1	0.459677419354839\\
3	2.9	0.548387096774194	1	0.451612903225806\\
3	2.9	0.556451612903226	1	0.443548387096774\\
3	2.9	0.564516129032258	1	0.435483870967742\\
3	2.9	0.57258064516129	1	0.42741935483871\\
3	2.9	0.580645161290323	1	0.419354838709677\\
3	2.9	0.588709677419355	1	0.411290322580645\\
3	2.9	0.596774193548387	1	0.403225806451613\\
3	2.9	0.604838709677419	1	0.395161290322581\\
3	2.9	0.612903225806452	1	0.387096774193548\\
3	2.9	0.620967741935484	1	0.379032258064516\\
3	2.9	0.629032258064516	1	0.370967741935484\\
3	2.9	0.637096774193548	1	0.362903225806452\\
3	2.9	0.645161290322581	1	0.354838709677419\\
3	2.9	0.653225806451613	1	0.346774193548387\\
3	2.9	0.661290322580645	1	0.338709677419355\\
3	2.9	0.669354838709677	1	0.330645161290323\\
3	2.9	0.67741935483871	1	0.32258064516129\\
2.9	2.9	0.685483870967742	1	0.314516129032258\\
2.9	2.9	0.693548387096774	1	0.306451612903226\\
2.9	2.9	0.701612903225806	1	0.298387096774194\\
2.9	2.9	0.709677419354839	1	0.290322580645161\\
2.9	3	0.717741935483871	1	0.282258064516129\\
2.9	3	0.725806451612903	1	0.274193548387097\\
2.9	3	0.733870967741935	1	0.266129032258065\\
2.9	3	0.741935483870968	1	0.258064516129032\\
2.9	3	0.75	1	0.25\\
2.9	3	0.758064516129032	1	0.241935483870968\\
2.9	3	0.766129032258065	1	0.233870967741935\\
2.9	3	0.774193548387097	1	0.225806451612903\\
2.9	3	0.782258064516129	1	0.217741935483871\\
2.9	3	0.790322580645161	1	0.209677419354839\\
2.9	3	0.798387096774194	1	0.201612903225806\\
2.9	3	0.806451612903226	1	0.193548387096774\\
2.9	3	0.814516129032258	1	0.185483870967742\\
2.9	3	0.82258064516129	1	0.17741935483871\\
2.9	3	0.830645161290323	1	0.169354838709677\\
2.9	3	0.838709677419355	1	0.161290322580645\\
2.9	3	0.846774193548387	1	0.153225806451613\\
2.9	3	0.854838709677419	1	0.145161290322581\\
2.9	3	0.862903225806452	1	0.137096774193548\\
2.9	3	0.870967741935484	1	0.129032258064516\\
2.9	3	0.879032258064516	1	0.120967741935484\\
2.9	3	0.887096774193548	1	0.112903225806452\\
2.9	3	0.895161290322581	1	0.104838709677419\\
2.9	3	0.903225806451613	1	0.0967741935483871\\
2.9	3	0.911290322580645	1	0.0887096774193548\\
2.9	3	0.919354838709677	1	0.0806451612903226\\
2.9	3	0.92741935483871	1	0.0725806451612903\\
2.9	3	0.935483870967742	1	0.0645161290322581\\
2.9	3	0.943548387096774	1	0.0564516129032258\\
2.9	3	0.951612903225806	1	0.0483870967741935\\
2.9	3	0.959677419354839	1	0.0403225806451613\\
2.9	3	0.967741935483871	1	0.032258064516129\\
2.9	3	0.975806451612903	1	0.0241935483870968\\
2.9	3	0.983870967741935	1	0.0161290322580645\\
2.9	3	0.991935483870968	1	0.00806451612903226\\
2.9	3.1	1	1	0\\
2.9	3.1	1	0.991935483870968	0\\
2.9	3.1	1	0.983870967741935	0\\
2.9	3.1	1	0.975806451612903	0\\
2.9	3.1	1	0.967741935483871	0\\
2.9	3.1	1	0.959677419354839	0\\
2.9	3.1	1	0.951612903225806	0\\
2.9	3.1	1	0.943548387096774	0\\
2.9	3.1	1	0.935483870967742	0\\
2.9	3.1	1	0.92741935483871	0\\
2.9	3.1	1	0.919354838709677	0\\
2.9	3.1	1	0.911290322580645	0\\
2.9	3.1	1	0.903225806451613	0\\
2.9	3.1	1	0.895161290322581	0\\
2.9	3.1	1	0.887096774193548	0\\
2.9	3.1	1	0.879032258064516	0\\
2.9	3.1	1	0.870967741935484	0\\
2.9	3.1	1	0.862903225806452	0\\
2.9	3.1	1	0.854838709677419	0\\
2.9	3.1	1	0.846774193548387	0\\
2.9	3.1	1	0.838709677419355	0\\
2.9	3.1	1	0.830645161290323	0\\
2.9	3.1	1	0.82258064516129	0\\
2.9	3.1	1	0.814516129032258	0\\
2.9	3.1	1	0.806451612903226	0\\
2.9	3.1	1	0.798387096774194	0\\
2.8	3.2	1	0.790322580645161	0\\
2.8	3.2	1	0.782258064516129	0\\
2.8	3.2	1	0.774193548387097	0\\
2.8	3.2	1	0.766129032258065	0\\
2.8	3.2	1	0.758064516129032	0\\
2.8	3.2	1	0.75	0\\
2.8	3.2	1	0.741935483870968	0\\
2.8	3.2	1	0.733870967741935	0\\
2.8	3.2	1	0.725806451612903	0\\
2.8	3.2	1	0.717741935483871	0\\
2.8	3.2	1	0.709677419354839	0\\
2.8	3.2	1	0.701612903225806	0\\
2.8	3.2	1	0.693548387096774	0\\
2.9	3.1	1	0.685483870967742	0\\
2.9	3.1	1	0.67741935483871	0\\
2.9	3.1	1	0.669354838709677	0\\
2.9	3.1	1	0.661290322580645	0\\
2.9	3.1	1	0.653225806451613	0\\
2.9	3.1	1	0.645161290322581	0\\
2.9	3.1	1	0.637096774193548	0\\
2.9	3.2	1	0.629032258064516	0\\
2.9	3.2	1	0.620967741935484	0\\
2.9	3.2	1	0.612903225806452	0\\
2.9	3.2	1	0.604838709677419	0\\
2.9	3.2	1	0.596774193548387	0\\
2.9	3.2	1	0.588709677419355	0\\
2.9	3.2	1	0.580645161290323	0\\
2.9	3.2	1	0.57258064516129	0\\
2.8	3.2	1	0.564516129032258	0\\
2.8	3.2	1	0.556451612903226	0\\
2.8	3.2	1	0.548387096774194	0\\
2.8	3.2	1	0.540322580645161	0\\
2.8	3.2	1	0.532258064516129	0\\
2.8	3.2	1	0.524193548387097	0\\
2.8	3.2	1	0.516129032258065	0\\
2.7	3.3	1	0.508064516129032	0\\
2.7	3.3	1	0.5	0\\
2.7	3.3	1	0.491935483870968	0\\
2.6	3.4	1	0.483870967741935	0\\
2.6	3.4	1	0.475806451612903	0\\
2.6	3.4	1	0.467741935483871	0\\
2.6	3.4	1	0.459677419354839	0\\
2.6	3.4	1	0.451612903225806	0\\
2.8	3.3	1	0.443548387096774	0\\
2.8	3.3	1	0.435483870967742	0\\
2.8	3.3	1	0.42741935483871	0\\
2.8	3.3	1	0.419354838709677	0\\
2.8	3.3	1	0.411290322580645	0\\
2.8	3.3	1	0.403225806451613	0\\
2.8	3.3	1	0.395161290322581	0\\
3.5	3.4	1	0.387096774193548	0\\
3.5	3.4	1	0.379032258064516	0\\
3.5	3.4	1	0.370967741935484	0\\
3.5	3.4	1	0.362903225806452	0\\
3.5	3.4	1	0.354838709677419	0\\
3.5	3.4	1	0.346774193548387	0\\
3.5	3.4	1	0.338709677419355	0\\
3.5	3.4	1	0.330645161290323	0\\
3.5	3.4	1	0.32258064516129	0\\
2.8	3.3	1	0.314516129032258	0\\
2.8	3.3	1	0.306451612903226	0\\
3.5	3.4	1	0.298387096774194	0\\
3.5	3.4	1	0.290322580645161	0\\
3.5	3.4	1	0.282258064516129	0\\
3.5	3.4	1	0.274193548387097	0\\
3.5	3.4	1	0.266129032258065	0\\
3.6	3.5	1	0.258064516129032	0\\
3.6	3.5	1	0.25	0\\
3.6	3.5	1	0.241935483870968	0\\
3.6	3.5	1	0.233870967741935	0\\
3.6	3.5	1	0.225806451612903	0\\
3.6	3.5	1	0.217741935483871	0\\
3.6	3.5	1	0.209677419354839	0\\
3.6	3.5	1	0.201612903225806	0\\
3.6	3.5	1	0.193548387096774	0\\
3.6	3.5	1	0.185483870967742	0\\
3.6	3.5	1	0.17741935483871	0\\
2.4	3.5	1	0.169354838709677	0\\
2.4	3.5	1	0.161290322580645	0\\
2.4	3.5	1	0.153225806451613	0\\
2.4	3.5	1	0.145161290322581	0\\
2.4	3.5	1	0.137096774193548	0\\
2.4	3.5	1	0.129032258064516	0\\
2.4	3.5	1	0.120967741935484	0\\
2.4	3.5	1	0.112903225806452	0\\
2.4	3.6	1	0.104838709677419	0\\
2.4	3.6	1	0.0967741935483871	0\\
2.4	3.6	1	0.0887096774193548	0\\
2.4	3.6	1	0.0806451612903226	0\\
2.4	3.6	1	0.0725806451612903	0\\
2.4	3.6	1	0.0645161290322581	0\\
2.4	3.6	1	0.0564516129032258	0\\
2.4	3.6	1	0.0483870967741935	0\\
2.4	3.6	1	0.0403225806451613	0\\
2.4	3.6	1	0.032258064516129	0\\
2.4	3.6	1	0.0241935483870968	0\\
2.4	3.6	1	0.0161290322580645	0\\
2.4	3.6	1	0.00806451612903226	0\\
2.4	3.6	1	0	0\\
2.4	3.6	0.991935483870968	0	0\\
2.4	3.6	0.983870967741935	0	0\\
2.4	3.6	0.975806451612903	0	0\\
2.4	3.6	0.967741935483871	0	0\\
2.4	3.6	0.959677419354839	0	0\\
2.4	3.6	0.951612903225806	0	0\\
2.4	3.6	0.943548387096774	0	0\\
2.4	3.6	0.935483870967742	0	0\\
2.4	3.6	0.92741935483871	0	0\\
2.4	3.6	0.919354838709677	0	0\\
2.4	3.6	0.911290322580645	0	0\\
2.4	3.6	0.903225806451613	0	0\\
2.4	3.6	0.895161290322581	0	0\\
2.3	3.6	0.887096774193548	0	0\\
2.3	3.6	0.879032258064516	0	0\\
2.3	3.6	0.870967741935484	0	0\\
2.3	3.6	0.862903225806452	0	0\\
2.3	3.6	0.854838709677419	0	0\\
2.3	3.6	0.846774193548387	0	0\\
2.3	3.6	0.838709677419355	0	0\\
2.3	3.6	0.830645161290323	0	0\\
2.3	3.6	0.82258064516129	0	0\\
2.3	3.6	0.814516129032258	0	0\\
2.3	3.6	0.806451612903226	0	0\\
2.3	3.6	0.798387096774194	0	0\\
2.3	3.6	0.790322580645161	0	0\\
2.3	3.6	0.782258064516129	0	0\\
2.3	3.6	0.774193548387097	0	0\\
2.3	3.6	0.766129032258065	0	0\\
2.3	3.6	0.758064516129032	0	0\\
2.3	3.6	0.75	0	0\\
2.3	3.6	0.741935483870968	0	0\\
2.3	3.6	0.733870967741935	0	0\\
2.3	3.6	0.725806451612903	0	0\\
2.3	3.6	0.717741935483871	0	0\\
2.3	3.6	0.709677419354839	0	0\\
2.3	3.6	0.701612903225806	0	0\\
2.3	3.6	0.693548387096774	0	0\\
2.3	3.6	0.685483870967742	0	0\\
2.3	3.6	0.67741935483871	0	0\\
2.3	3.6	0.669354838709677	0	0\\
2.3	3.6	0.661290322580645	0	0\\
2.3	3.6	0.653225806451613	0	0\\
2.3	3.6	0.645161290322581	0	0\\
3.9	3.9	0.637096774193548	0	0\\
3.9	3.9	0.629032258064516	0	0\\
3.9	3.9	0.620967741935484	0	0\\
3.9	3.9	0.612903225806452	0	0\\
3.9	3.9	0.604838709677419	0	0\\
3.9	3.9	0.596774193548387	0	0\\
3.9	3.9	0.588709677419355	0	0\\
3.9	3.9	0.580645161290323	0	0\\
3.9	3.9	0.57258064516129	0	0\\
3.9	3.9	0.564516129032258	0	0\\
3.9	3.9	0.556451612903226	0	0\\
3.9	3.9	0.548387096774194	0	0\\
3.9	3.9	0.540322580645161	0	0\\
3.9	3.9	0.532258064516129	0	0\\
3.9	3.9	0.524193548387097	0	0\\
3.9	3.9	0.516129032258065	0	0\\
3.9	3.9	0.508064516129032	0	0\\
3.9	3.9	0.5	0	0\\
};
\addplot[scatter, only marks, mark=x] table[row sep=crcr]{%
x	y	R	G	B\\
1.2	3	0	0	0.508064516129032\\
1.2	3	0	0	0.516129032258065\\
1.2	3	0	0	0.524193548387097\\
1.2	3.8	0	0	0.532258064516129\\
1.2	3.8	0	0	0.540322580645161\\
1.5	2.1	0	0	0.548387096774194\\
1.5	2.1	0	0	0.556451612903226\\
1.5	2.1	0	0	0.564516129032258\\
1.5	2.1	0	0	0.57258064516129\\
1.5	2.1	0	0	0.580645161290323\\
1.5	2.1	0	0	0.588709677419355\\
1.5	2.1	0	0	0.596774193548387\\
2.2	1.4	0	0	0.604838709677419\\
3.9	2	0	0	0.612903225806452\\
3.9	2	0	0	0.620967741935484\\
3.9	2	0	0	0.629032258064516\\
3.9	2	0	0	0.637096774193548\\
2.1	2	0	0	0.645161290322581\\
2.1	2	0	0	0.653225806451613\\
2.1	2	0	0	0.661290322580645\\
2.1	2	0	0	0.669354838709677\\
2.1	2	0	0	0.67741935483871\\
2.1	2	0	0	0.685483870967742\\
2.1	2	0	0	0.693548387096774\\
2.1	2	0	0	0.701612903225806\\
2.1	2	0	0	0.709677419354839\\
2.1	2	0	0	0.717741935483871\\
3.9	2	0	0	0.725806451612903\\
3.9	2	0	0	0.733870967741935\\
3.9	2	0	0	0.741935483870968\\
3.9	2	0	0	0.75\\
3.9	2	0	0	0.758064516129032\\
3.9	2	0	0	0.766129032258065\\
3.9	2	0	0	0.774193548387097\\
3.9	2.1	0	0	0.782258064516129\\
3.9	2.1	0	0	0.790322580645161\\
3.9	2.1	0	0	0.798387096774194\\
2.1	2	0	0	0.806451612903226\\
2.1	2	0	0	0.814516129032258\\
2.1	2	0	0	0.82258064516129\\
2.1	2	0	0	0.830645161290323\\
2.1	2	0	0	0.838709677419355\\
2.1	2	0	0	0.846774193548387\\
2.1	2	0	0	0.854838709677419\\
2.1	2	0	0	0.862903225806452\\
2.1	2	0	0	0.870967741935484\\
2.1	2	0	0	0.879032258064516\\
2.1	2	0	0	0.887096774193548\\
2.1	2	0	0	0.895161290322581\\
2.1	2	0	0	0.903225806451613\\
2.1	2	0	0	0.911290322580645\\
2.1	2	0	0	0.919354838709677\\
2.1	2	0	0	0.92741935483871\\
2.1	2	0	0	0.935483870967742\\
2.1	2	0	0	0.943548387096774\\
2.1	2	0	0	0.951612903225806\\
2.1	2	0	0	0.959677419354839\\
2.1	2	0	0	0.967741935483871\\
2.1	2	0	0	0.975806451612903\\
2.1	2	0	0	0.983870967741935\\
2.1	2	0	0	0.991935483870968\\
2.1	2	0	0	1\\
2.1	2	0	0.00806451612903226	1\\
2.1	2	0	0.0161290322580645	1\\
2.1	2	0	0.0241935483870968	1\\
2.1	2	0	0.032258064516129	1\\
2.1	2	0	0.0403225806451613	1\\
2.1	2	0	0.0483870967741935	1\\
2.1	2	0	0.0564516129032258	1\\
2.1	2	0	0.0645161290322581	1\\
2.1	2	0	0.0725806451612903	1\\
2.1	2	0	0.0806451612903226	1\\
2.1	2	0	0.0887096774193548	1\\
2.1	2	0	0.0967741935483871	1\\
2.1	2	0	0.104838709677419	1\\
2.2	2	0	0.112903225806452	1\\
2.1	2	0	0.120967741935484	1\\
2.1	2	0	0.129032258064516	1\\
2.1	2	0	0.137096774193548	1\\
2.1	2	0	0.145161290322581	1\\
2.1	2	0	0.153225806451613	1\\
2.1	2	0	0.161290322580645	1\\
2.1	2	0	0.169354838709677	1\\
2.1	2	0	0.17741935483871	1\\
2.1	2	0	0.185483870967742	1\\
2.1	2	0	0.193548387096774	1\\
3.5	2.5	0	0.201612903225806	1\\
2.1	2	0	0.209677419354839	1\\
2.1	2	0	0.217741935483871	1\\
2.1	2	0	0.225806451612903	1\\
2.9	1.2	0	0.233870967741935	1\\
2.9	1.2	0	0.241935483870968	1\\
2.9	1.2	0	0.25	1\\
3.4	2.6	0	0.258064516129032	1\\
2.9	1.2	0	0.266129032258065	1\\
2.9	1.2	0	0.274193548387097	1\\
2.9	1.2	0	0.282258064516129	1\\
2.9	1.2	0	0.290322580645161	1\\
2.9	1.2	0	0.298387096774194	1\\
2.9	1.2	0	0.306451612903226	1\\
2.9	1.2	0	0.314516129032258	1\\
2.9	1.2	0	0.32258064516129	1\\
2.9	1.2	0	0.330645161290323	1\\
2.9	1.2	0	0.338709677419355	1\\
2.9	1.2	0	0.346774193548387	1\\
2.2	1.2	0	0.354838709677419	1\\
1.2	3.8	0	0.362903225806452	1\\
2.9	1.2	0	0.370967741935484	1\\
1.2	3.8	0	0.379032258064516	1\\
1.2	3.8	0	0.387096774193548	1\\
1.2	3.8	0	0.395161290322581	1\\
1.2	3.8	0	0.403225806451613	1\\
1.2	3.8	0	0.411290322580645	1\\
1.2	3.8	0	0.419354838709677	1\\
1.2	3.8	0	0.42741935483871	1\\
1.2	3.8	0	0.435483870967742	1\\
1.2	3.8	0	0.443548387096774	1\\
1.2	3.8	0	0.451612903225806	1\\
1.2	3.8	0	0.459677419354839	1\\
3.8	2.2	0	0.467741935483871	1\\
3.8	2.2	0	0.475806451612903	1\\
3.8	2.2	0	0.483870967741935	1\\
1.2	3.8	0	0.491935483870968	1\\
1.2	3.8	0	0.5	1\\
1.2	3.8	0	0.508064516129032	1\\
1.2	3.8	0	0.516129032258065	1\\
1.2	2.2	0	0.524193548387097	1\\
1.2	2.2	0	0.532258064516129	1\\
1.2	2.2	0	0.540322580645161	1\\
1.2	2.2	0	0.548387096774194	1\\
3.8	2.2	0	0.556451612903226	1\\
3.8	2.2	0	0.564516129032258	1\\
3.9	2.2	0	0.57258064516129	1\\
3.9	2.2	0	0.580645161290323	1\\
3.9	2.2	0	0.588709677419355	1\\
1.2	2.2	0	0.596774193548387	1\\
2.9	1.2	0	0.604838709677419	1\\
1.2	2.2	0	0.612903225806452	1\\
1.2	2.2	0	0.620967741935484	1\\
2.9	1.2	0	0.629032258064516	1\\
2.9	1.2	0	0.637096774193548	1\\
2.8	1.2	0	0.645161290322581	1\\
2.8	1.2	0	0.653225806451613	1\\
2.8	1.2	0	0.661290322580645	1\\
2.8	1.2	0	0.669354838709677	1\\
1.2	2.2	0	0.67741935483871	1\\
1.2	2.2	0	0.685483870967742	1\\
1.2	2.2	0	0.693548387096774	1\\
3.8	2.2	0	0.701612903225806	1\\
2.8	1.2	0	0.709677419354839	1\\
2.8	1.2	0	0.717741935483871	1\\
2.8	1.2	0	0.725806451612903	1\\
2.8	1.2	0	0.733870967741935	1\\
1.2	2.2	0	0.741935483870968	1\\
2.8	1.2	0	0.75	1\\
2.8	1.2	0	0.758064516129032	1\\
4.7	3.8	0	0.766129032258065	1\\
2.8	1.2	0	0.774193548387097	1\\
3.9	2.2	0	0.782258064516129	1\\
3	2.8	0	0.790322580645161	1\\
2.8	2.7	0	0.798387096774194	1\\
3	2.8	0	0.806451612903226	1\\
3	2.8	0	0.814516129032258	1\\
3	2.8	0	0.82258064516129	1\\
3	2.8	0	0.830645161290323	1\\
3	2.8	0	0.838709677419355	1\\
1.2	2.2	0	0.846774193548387	1\\
1.2	2.2	0	0.854838709677419	1\\
1.2	2.2	0	0.862903225806452	1\\
1.2	2.2	0	0.870967741935484	1\\
1.2	2.2	0	0.879032258064516	1\\
1.2	2.2	0	0.887096774193548	1\\
1.2	2.2	0	0.895161290322581	1\\
1.2	2.2	0	0.903225806451613	1\\
2.7	2.7	0	0.911290322580645	1\\
2.7	2.7	0	0.919354838709677	1\\
1.2	2.2	0	0.92741935483871	1\\
2.7	2.7	0	0.935483870967742	1\\
2.7	2.7	0	0.943548387096774	1\\
2.7	2.7	0	0.951612903225806	1\\
2.7	2.7	0	0.959677419354839	1\\
2.7	2.7	0	0.967741935483871	1\\
2.7	2.7	0	0.975806451612903	1\\
2.7	2.7	0	0.983870967741935	1\\
2.7	2.7	0	0.991935483870968	1\\
2.7	2.7	0	1	1\\
1.2	3	0.00806451612903226	1	0.991935483870968\\
3.8	4.7	0.0161290322580645	1	0.983870967741935\\
3.8	4.7	0.0241935483870968	1	0.975806451612903\\
3.8	4.7	0.032258064516129	1	0.967741935483871\\
3.8	4.7	0.0403225806451613	1	0.959677419354839\\
2.9	1.2	0.0483870967741935	1	0.951612903225806\\
3.8	4.7	0.0564516129032258	1	0.943548387096774\\
2.9	1.2	0.0645161290322581	1	0.935483870967742\\
2.9	1.2	0.0725806451612903	1	0.92741935483871\\
2.8	2.8	0.0806451612903226	1	0.919354838709677\\
3.8	4.7	0.0887096774193548	1	0.911290322580645\\
3.8	4.7	0.0967741935483871	1	0.903225806451613\\
3.8	4.7	0.104838709677419	1	0.895161290322581\\
3.8	4.7	0.112903225806452	1	0.887096774193548\\
3.8	4.7	0.120967741935484	1	0.879032258064516\\
3.8	4.7	0.129032258064516	1	0.870967741935484\\
3.8	4.7	0.137096774193548	1	0.862903225806452\\
3.8	4.7	0.145161290322581	1	0.854838709677419\\
3.8	4.7	0.153225806451613	1	0.846774193548387\\
3.8	4.7	0.161290322580645	1	0.838709677419355\\
3.8	4.7	0.169354838709677	1	0.830645161290323\\
3.8	4.7	0.17741935483871	1	0.82258064516129\\
2.8	3	0.185483870967742	1	0.814516129032258\\
2.8	3	0.193548387096774	1	0.806451612903226\\
3.8	4.7	0.201612903225806	1	0.798387096774194\\
3.8	4.7	0.209677419354839	1	0.790322580645161\\
3.8	4.7	0.217741935483871	1	0.782258064516129\\
3.8	4.7	0.225806451612903	1	0.774193548387097\\
3.8	4.7	0.233870967741935	1	0.766129032258065\\
2.9	1.2	0.241935483870968	1	0.758064516129032\\
2.9	1.2	0.25	1	0.75\\
2.9	1.2	0.258064516129032	1	0.741935483870968\\
2.9	1.2	0.266129032258065	1	0.733870967741935\\
3.8	4.7	0.274193548387097	1	0.725806451612903\\
3.8	4.7	0.282258064516129	1	0.717741935483871\\
3.8	4.7	0.290322580645161	1	0.709677419354839\\
3.8	4.7	0.298387096774194	1	0.701612903225806\\
3.8	4.7	0.306451612903226	1	0.693548387096774\\
3.8	4.7	0.314516129032258	1	0.685483870967742\\
3.8	4.7	0.32258064516129	1	0.67741935483871\\
3.8	4.7	0.330645161290323	1	0.669354838709677\\
3.8	4.7	0.338709677419355	1	0.661290322580645\\
3.8	4.7	0.346774193548387	1	0.653225806451613\\
3.8	4.7	0.354838709677419	1	0.645161290322581\\
3.8	4.7	0.362903225806452	1	0.637096774193548\\
3.4	2.5	0.370967741935484	1	0.629032258064516\\
3.4	2.5	0.379032258064516	1	0.620967741935484\\
3.4	2.5	0.387096774193548	1	0.612903225806452\\
3.4	2.5	0.395161290322581	1	0.604838709677419\\
3.4	2.5	0.403225806451613	1	0.596774193548387\\
3.4	2.5	0.411290322580645	1	0.588709677419355\\
3.4	2.5	0.419354838709677	1	0.580645161290323\\
3.4	2.5	0.42741935483871	1	0.57258064516129\\
3.4	2.5	0.435483870967742	1	0.564516129032258\\
3.4	2.5	0.443548387096774	1	0.556451612903226\\
3.4	2.5	0.451612903225806	1	0.548387096774194\\
3.4	2.5	0.459677419354839	1	0.540322580645161\\
3.4	2.5	0.467741935483871	1	0.532258064516129\\
2.7	3.2	0.475806451612903	1	0.524193548387097\\
2.7	3.2	0.483870967741935	1	0.516129032258065\\
2.7	3.2	0.491935483870968	1	0.508064516129032\\
2.7	3.2	0.5	1	0.5\\
2.7	3.2	0.508064516129032	1	0.491935483870968\\
3.3	2.5	0.516129032258065	1	0.483870967741935\\
3.3	2.5	0.524193548387097	1	0.475806451612903\\
1.2	3	0.532258064516129	1	0.467741935483871\\
1.2	3	0.540322580645161	1	0.459677419354839\\
1.2	3	0.548387096774194	1	0.451612903225806\\
1.2	3	0.556451612903226	1	0.443548387096774\\
1.2	3	0.564516129032258	1	0.435483870967742\\
1.2	3	0.57258064516129	1	0.42741935483871\\
1.2	3	0.580645161290323	1	0.419354838709677\\
1.2	3	0.588709677419355	1	0.411290322580645\\
1.2	3	0.596774193548387	1	0.403225806451613\\
1.2	3	0.604838709677419	1	0.395161290322581\\
1.2	3	0.612903225806452	1	0.387096774193548\\
1.2	3	0.620967741935484	1	0.379032258064516\\
1.2	3	0.629032258064516	1	0.370967741935484\\
1.2	3	0.637096774193548	1	0.362903225806452\\
1.2	3	0.645161290322581	1	0.354838709677419\\
1.2	3	0.653225806451613	1	0.346774193548387\\
1.2	3	0.661290322580645	1	0.338709677419355\\
1.2	3	0.669354838709677	1	0.330645161290323\\
1.2	3	0.67741935483871	1	0.32258064516129\\
1.2	3	0.685483870967742	1	0.314516129032258\\
1.2	3	0.693548387096774	1	0.306451612903226\\
1.2	3	0.701612903225806	1	0.298387096774194\\
1.2	3	0.709677419354839	1	0.290322580645161\\
1.2	3	0.717741935483871	1	0.282258064516129\\
1.2	3	0.725806451612903	1	0.274193548387097\\
1.2	3	0.733870967741935	1	0.266129032258065\\
1.2	3	0.741935483870968	1	0.258064516129032\\
1.2	3	0.75	1	0.25\\
1.2	3.7	0.758064516129032	1	0.241935483870968\\
1.2	3.7	0.766129032258065	1	0.233870967741935\\
1.2	3.7	0.774193548387097	1	0.225806451612903\\
1.2	3.7	0.782258064516129	1	0.217741935483871\\
1.2	3.7	0.790322580645161	1	0.209677419354839\\
1.2	3.7	0.798387096774194	1	0.201612903225806\\
1.2	3.7	0.806451612903226	1	0.193548387096774\\
1.2	3.7	0.814516129032258	1	0.185483870967742\\
1.2	3.7	0.82258064516129	1	0.17741935483871\\
1.2	3.7	0.830645161290323	1	0.169354838709677\\
1.2	3.7	0.838709677419355	1	0.161290322580645\\
1.2	3.7	0.846774193548387	1	0.153225806451613\\
1.2	3.7	0.854838709677419	1	0.145161290322581\\
1.2	3.7	0.862903225806452	1	0.137096774193548\\
1.2	3.7	0.870967741935484	1	0.129032258064516\\
1.2	3.7	0.879032258064516	1	0.120967741935484\\
3.8	4.7	0.887096774193548	1	0.112903225806452\\
3.8	4.7	0.895161290322581	1	0.104838709677419\\
3.8	4.7	0.903225806451613	1	0.0967741935483871\\
3.8	4.7	0.911290322580645	1	0.0887096774193548\\
3.8	4.7	0.919354838709677	1	0.0806451612903226\\
1.2	3.7	0.92741935483871	1	0.0725806451612903\\
1.2	3.7	0.935483870967742	1	0.0645161290322581\\
1.2	3.7	0.943548387096774	1	0.0564516129032258\\
1.2	3.7	0.951612903225806	1	0.0483870967741935\\
1.2	3.7	0.959677419354839	1	0.0403225806451613\\
1.2	3.7	0.967741935483871	1	0.032258064516129\\
1.2	3.7	0.975806451612903	1	0.0241935483870968\\
1.2	3.7	0.983870967741935	1	0.0161290322580645\\
1.2	3.7	0.991935483870968	1	0.00806451612903226\\
1.2	3.7	1	1	0\\
1.2	3.7	1	0.991935483870968	0\\
1.2	3.7	1	0.983870967741935	0\\
1.2	3.7	1	0.975806451612903	0\\
1.2	3.7	1	0.967741935483871	0\\
1.2	3.7	1	0.959677419354839	0\\
3.7	1.2	1	0.951612903225806	0\\
3.7	1.2	1	0.943548387096774	0\\
3.7	1.2	1	0.935483870967742	0\\
3.7	1.2	1	0.92741935483871	0\\
3.7	1.2	1	0.919354838709677	0\\
3.7	1.2	1	0.911290322580645	0\\
3.7	1.2	1	0.903225806451613	0\\
1.2	3.7	1	0.895161290322581	0\\
1.2	3.7	1	0.887096774193548	0\\
3.7	1.2	1	0.879032258064516	0\\
3.7	1.2	1	0.870967741935484	0\\
3.7	1.2	1	0.862903225806452	0\\
3.7	1.2	1	0.854838709677419	0\\
3.7	1.2	1	0.846774193548387	0\\
3.7	1.2	1	0.838709677419355	0\\
2.4	3.4	1	0.830645161290323	0\\
2.4	3.4	1	0.82258064516129	0\\
2.4	3.4	1	0.814516129032258	0\\
2.4	3.4	1	0.806451612903226	0\\
2.4	3.4	1	0.798387096774194	0\\
1.2	3.7	1	0.790322580645161	0\\
3.7	1.2	1	0.782258064516129	0\\
2.7	1.2	1	0.774193548387097	0\\
2.7	1.2	1	0.766129032258065	0\\
3.1	3.6	1	0.758064516129032	0\\
2.7	1.2	1	0.75	0\\
3.7	1.2	1	0.741935483870968	0\\
3.7	1.2	1	0.733870967741935	0\\
2.7	1.2	1	0.725806451612903	0\\
2.7	1.2	1	0.717741935483871	0\\
2.7	1.2	1	0.709677419354839	0\\
3.7	1.2	1	0.701612903225806	0\\
2.7	1.2	1	0.693548387096774	0\\
3.7	1.2	1	0.685483870967742	0\\
3.7	1.2	1	0.67741935483871	0\\
3.7	1.2	1	0.669354838709677	0\\
2.4	3.4	1	0.661290322580645	0\\
2.4	3.4	1	0.653225806451613	0\\
2.4	3.4	1	0.645161290322581	0\\
2.5	3.4	1	0.637096774193548	0\\
3.9	2.1	1	0.629032258064516	0\\
3.7	1.2	1	0.620967741935484	0\\
3.9	2.1	1	0.612903225806452	0\\
3.9	2.1	1	0.604838709677419	0\\
3.9	2.1	1	0.596774193548387	0\\
3.9	2.1	1	0.588709677419355	0\\
3.9	2.1	1	0.580645161290323	0\\
3.9	2.1	1	0.57258064516129	0\\
3.9	2.1	1	0.564516129032258	0\\
3.9	2.1	1	0.556451612903226	0\\
2.4	3.6	1	0.548387096774194	0\\
2.4	3.6	1	0.540322580645161	0\\
2.4	3.6	1	0.532258064516129	0\\
2.4	3.6	1	0.524193548387097	0\\
2.2	3.5	1	0.516129032258065	0\\
2.3	3.7	1	0.508064516129032	0\\
2.3	3.7	1	0.5	0\\
2.3	3.7	1	0.491935483870968	0\\
2.9	2.9	1	0.483870967741935	0\\
3	3	1	0.475806451612903	0\\
3.2	3.4	1	0.467741935483871	0\\
3.2	3.4	1	0.459677419354839	0\\
3.2	3.4	1	0.451612903225806	0\\
3.4	3.4	1	0.443548387096774	0\\
3.4	3.4	1	0.435483870967742	0\\
3.4	3.4	1	0.42741935483871	0\\
3.4	3.4	1	0.419354838709677	0\\
3.5	3.4	1	0.411290322580645	0\\
3.5	3.4	1	0.403225806451613	0\\
3.5	3.4	1	0.395161290322581	0\\
2.8	3.3	1	0.387096774193548	0\\
2.8	3.3	1	0.379032258064516	0\\
2.8	3.3	1	0.370967741935484	0\\
2.8	3.3	1	0.362903225806452	0\\
2.8	3.3	1	0.354838709677419	0\\
2.8	3.3	1	0.346774193548387	0\\
2.8	3.3	1	0.338709677419355	0\\
2.8	3.3	1	0.330645161290323	0\\
2.8	3.3	1	0.32258064516129	0\\
3.5	3.4	1	0.314516129032258	0\\
3.5	3.4	1	0.306451612903226	0\\
2.8	3.3	1	0.298387096774194	0\\
2.8	3.3	1	0.290322580645161	0\\
2.8	3.3	1	0.282258064516129	0\\
2.8	3.3	1	0.274193548387097	0\\
2.8	3.3	1	0.266129032258065	0\\
2.5	3.4	1	0.258064516129032	0\\
2.8	3.3	1	0.25	0\\
2.8	3.3	1	0.241935483870968	0\\
2.8	3.3	1	0.233870967741935	0\\
2.8	3.3	1	0.225806451612903	0\\
2.8	3.3	1	0.217741935483871	0\\
2.8	3.3	1	0.209677419354839	0\\
2.8	3.3	1	0.201612903225806	0\\
2.8	3.3	1	0.193548387096774	0\\
2.8	3.3	1	0.185483870967742	0\\
2.4	3.5	1	0.17741935483871	0\\
3.6	3.5	1	0.169354838709677	0\\
3.6	3.5	1	0.161290322580645	0\\
3.6	3.5	1	0.153225806451613	0\\
3.6	3.5	1	0.145161290322581	0\\
3.6	3.5	1	0.137096774193548	0\\
1.2	3.7	1	0.129032258064516	0\\
1.2	3.7	1	0.120967741935484	0\\
3.6	3.5	1	0.112903225806452	0\\
1.2	3.7	1	0.104838709677419	0\\
1.2	3.7	1	0.0967741935483871	0\\
1.2	3.7	1	0.0887096774193548	0\\
1.2	3.7	1	0.0806451612903226	0\\
1.2	3.7	1	0.0725806451612903	0\\
1.2	3.7	1	0.0645161290322581	0\\
1.2	3.7	1	0.0564516129032258	0\\
1.2	3.7	1	0.0483870967741935	0\\
1.2	3.7	1	0.0403225806451613	0\\
1.2	3.7	1	0.032258064516129	0\\
1.2	3.7	1	0.0241935483870968	0\\
1.2	3.7	1	0.0161290322580645	0\\
1.2	3.7	1	0.00806451612903226	0\\
1.2	3.7	1	0	0\\
1.2	3.7	0.991935483870968	0	0\\
1.2	3.7	0.983870967741935	0	0\\
1.2	3.7	0.975806451612903	0	0\\
1.2	3.7	0.967741935483871	0	0\\
1.2	3.7	0.959677419354839	0	0\\
1.2	3.7	0.951612903225806	0	0\\
1.2	3.7	0.943548387096774	0	0\\
1.2	3.7	0.935483870967742	0	0\\
1.2	3.7	0.92741935483871	0	0\\
1.2	3.7	0.919354838709677	0	0\\
1.2	3.7	0.911290322580645	0	0\\
1.2	3.7	0.903225806451613	0	0\\
1.2	3.7	0.895161290322581	0	0\\
1.2	3.7	0.887096774193548	0	0\\
1.2	3.7	0.879032258064516	0	0\\
1.2	3.7	0.870967741935484	0	0\\
1.2	3.7	0.862903225806452	0	0\\
1.2	3.7	0.854838709677419	0	0\\
1.2	3.7	0.846774193548387	0	0\\
1.2	3.7	0.838709677419355	0	0\\
1.2	3.7	0.830645161290323	0	0\\
1.2	3.7	0.82258064516129	0	0\\
1.2	3.7	0.814516129032258	0	0\\
1.2	3.7	0.806451612903226	0	0\\
1.2	3.7	0.798387096774194	0	0\\
1.2	3.7	0.790322580645161	0	0\\
3.9	3.8	0.782258064516129	0	0\\
3.9	3.8	0.774193548387097	0	0\\
3.9	3.8	0.766129032258065	0	0\\
3.9	3.9	0.758064516129032	0	0\\
3.9	3.9	0.75	0	0\\
3.9	3.9	0.741935483870968	0	0\\
3.9	3.9	0.733870967741935	0	0\\
3.9	3.9	0.725806451612903	0	0\\
3.9	3.9	0.717741935483871	0	0\\
3.9	3.9	0.709677419354839	0	0\\
3.9	3.9	0.701612903225806	0	0\\
3.9	3.9	0.693548387096774	0	0\\
3.9	3.9	0.685483870967742	0	0\\
3.9	3.9	0.67741935483871	0	0\\
3.9	3.9	0.669354838709677	0	0\\
3.9	3.9	0.661290322580645	0	0\\
3.9	3.9	0.653225806451613	0	0\\
3.9	3.9	0.645161290322581	0	0\\
2.3	3.6	0.637096774193548	0	0\\
2.3	3.6	0.629032258064516	0	0\\
2.3	3.6	0.620967741935484	0	0\\
2.3	3.6	0.612903225806452	0	0\\
2.3	3.6	0.604838709677419	0	0\\
2.3	3.6	0.596774193548387	0	0\\
2.3	3.6	0.588709677419355	0	0\\
2.3	3.6	0.580645161290323	0	0\\
2.3	3.6	0.57258064516129	0	0\\
2.3	3.6	0.564516129032258	0	0\\
2.3	3.6	0.556451612903226	0	0\\
2.3	3.6	0.548387096774194	0	0\\
2.3	3.6	0.540322580645161	0	0\\
2.3	3.6	0.532258064516129	0	0\\
2.3	3.6	0.524193548387097	0	0\\
2.3	3.6	0.516129032258065	0	0\\
2.3	3.6	0.508064516129032	0	0\\
2.3	3.6	0.5	0	0\\
};
\end{axis}
\end{tikzpicture}%
		\caption{Estimated Positions \glsentryshort{crem}}
	\end{subfigure}
}
	\caption[Estimated Positions for CREM and TREM (Crossing)]{Estimated Position for CREM and TREM (Crossing, \Tsixty$=0.4$~s).}
	\label{fig:trackingCrossingRoom}
\end{figure}



\FloatBarrier

%% ARC MOVEMENT 
\subsubsection*{Arc Movement}
In the arc movement scenario, the first source moves from $\bm p^{(0)}_{s=1}=[3~2]$ up to $\bm p^{(T)}_{s=1}=[3~4]$, while the second source moves from $\bm p^{(0)}_{s=2}=[3~4]$ down to $\bm p^{(T)}_{s=2}=[3~2]$. From a top-down view, both sources move along a half-circle, until they reach the starting point of the respective other source.

\begin{figure}[!htbp]
	\iftoggle{quick}{%
		\includegraphics[width=\textwidth,height=\figureheight]{plots/tracking/arc/results-T60=0.4-crem-xy}
	}{%
		\begin{subfigure}{0.49\textwidth}
			\centering
			\setlength{\figurewidth}{0.8\textwidth}
			% This file was created by matlab2tikz.
%
\definecolor{lms_red}{rgb}{0.80000,0.20780,0.21960}%
\definecolor{mycolor2}{rgb}{0.80000,0.20784,0.21961}%
\definecolor{mycolor3}{rgb}{0.92900,0.69400,0.12500}%
\definecolor{mycolor4}{rgb}{0.49400,0.18400,0.55600}%
%
\begin{tikzpicture}

\begin{axis}[%
width=0.951\figurewidth,
height=\figureheight,
at={(0\figurewidth,0\figureheight)},
scale only axis,
xmin=0,
xmax=496,
xtick={0,99.2,198.4,297.6,396.8,496},
xticklabels={{0},{1},{2},{3},{4},{5}},
xlabel style={font=\color{white!15!black}},
xlabel={$t$~[s]},
ymin=1,
ymax=5,
ylabel style={font=\color{white!15!black}},
ylabel={$p_x^{(t)}$~[m]},
axis background/.style={fill=white},
xmajorgrids,
ymajorgrids,
legend entries={Est.,
                $s=1$,
                $s=2$},
legend columns=-1,
legend style={%
    at={(1.0,1.0)},
    anchor=south east,
    font=\footnotesize,
    fill opacity=0.0, draw opacity=1, text opacity=1,
    draw=none,
    column sep=0.42cm,
    /tikz/every odd column/.append style={column sep=0.15cm}
},
]
% Estimates
\addlegendimage{color=lms_red, mark=x, only marks, mark options={mark size=4pt, opacity=1, line width=1}}
\addplot [color=mycolor2, draw=none, mark=x, mark options={solid, mycolor2}, forget plot]
  table[row sep=crcr]{%
1	3.7\\
2	2.9\\
3	2.9\\
4	2.9\\
5	2.9\\
6	2.9\\
7	3\\
8	3\\
9	3\\
10	3\\
11	3\\
12	3\\
13	3\\
14	3\\
15	3\\
16	3\\
17	3\\
18	3\\
19	3\\
20	2.9\\
21	2.9\\
22	3\\
23	3\\
24	3\\
25	3\\
26	3\\
27	3\\
28	3\\
29	3\\
30	3\\
31	3\\
32	3\\
33	3\\
34	3\\
35	2.9\\
36	2.9\\
37	2.9\\
38	2.9\\
39	2.9\\
40	2.9\\
41	2.9\\
42	2.9\\
43	2.9\\
44	2.9\\
45	2.9\\
46	2.9\\
47	2.9\\
48	2.9\\
49	2.9\\
50	2.9\\
51	2.9\\
52	2.9\\
53	2.9\\
54	2.9\\
55	2.9\\
56	2.9\\
57	2.9\\
58	2.9\\
59	2.9\\
60	2.9\\
61	2.8\\
62	2.8\\
63	2.8\\
64	2.8\\
65	2.8\\
66	2.8\\
67	2.8\\
68	2.8\\
69	2.8\\
70	2.8\\
71	2.8\\
72	2.8\\
73	2.8\\
74	2.8\\
75	2.8\\
76	2.8\\
77	2.8\\
78	2.8\\
79	2.8\\
80	2.8\\
81	2.8\\
82	2.8\\
83	2.8\\
84	2.8\\
85	2.8\\
86	2.8\\
87	2.8\\
88	2.8\\
89	2.7\\
90	2.7\\
91	2.7\\
92	2.7\\
93	2.7\\
94	2.7\\
95	2.7\\
96	2.7\\
97	2.7\\
98	2.7\\
99	2.7\\
100	2.7\\
101	2.7\\
102	2.7\\
103	2.7\\
104	2.7\\
105	2.7\\
106	2.7\\
107	2.7\\
108	2.7\\
109	2.7\\
110	2.7\\
111	2.7\\
112	2.7\\
113	2.7\\
114	2.7\\
115	2.7\\
116	2.7\\
117	2.7\\
118	2.7\\
119	2.7\\
120	2.7\\
121	2.7\\
122	1.2\\
123	1.2\\
124	1.2\\
125	2.7\\
126	2.7\\
127	1.2\\
128	1.2\\
129	1.2\\
130	1.2\\
131	1.2\\
132	1.2\\
133	1.2\\
134	1.2\\
135	1.2\\
136	2.7\\
137	2.7\\
138	1.2\\
139	1.2\\
140	1.2\\
141	1.2\\
142	1.2\\
143	1.2\\
144	1.2\\
145	1.2\\
146	1.2\\
147	1.2\\
148	1.2\\
149	1.2\\
150	2.7\\
151	2.7\\
152	2.7\\
153	2.5\\
154	2.5\\
155	2.4\\
156	2.3\\
157	2.3\\
158	2.3\\
159	2.3\\
160	2.3\\
161	2.3\\
162	2.3\\
163	2.3\\
164	2.3\\
165	2.3\\
166	2.3\\
167	2.3\\
168	2.3\\
169	2.3\\
170	2.3\\
171	2.3\\
172	2.3\\
173	2.3\\
174	2.3\\
175	2.2\\
176	2.2\\
177	2.2\\
178	2.2\\
179	2.2\\
180	2.2\\
181	2.2\\
182	2.2\\
183	2.2\\
184	2.2\\
185	2.2\\
186	2.2\\
187	2.2\\
188	2.2\\
189	2.2\\
190	2.2\\
191	2.2\\
192	2.2\\
193	2.2\\
194	2.2\\
195	2.2\\
196	2.2\\
197	2.2\\
198	2.2\\
199	2.2\\
200	2.2\\
201	2.2\\
202	2.2\\
203	2.1\\
204	2.1\\
205	2.1\\
206	2.2\\
207	2.2\\
208	2.2\\
209	2.2\\
210	2.2\\
211	2.2\\
212	2.2\\
213	2.2\\
214	4\\
215	4\\
216	4\\
217	4\\
218	4\\
219	4\\
220	4\\
221	4\\
222	4\\
223	2.1\\
224	2.1\\
225	2.1\\
226	2.1\\
227	2.1\\
228	2.2\\
229	2.2\\
230	2.1\\
231	2.1\\
232	2.1\\
233	2.1\\
234	2.2\\
235	2.2\\
236	4\\
237	4\\
238	4\\
239	4\\
240	4\\
241	4\\
242	4\\
243	4\\
244	4\\
245	4\\
246	4\\
247	4\\
248	4\\
249	4\\
250	4\\
251	4\\
252	4\\
253	4\\
254	4\\
255	4\\
256	4\\
257	4\\
258	3.9\\
259	3.9\\
260	3.9\\
261	3.9\\
262	3.9\\
263	3.9\\
264	3.9\\
265	3.9\\
266	3.9\\
267	3.9\\
268	3.9\\
269	3.9\\
270	3.9\\
271	3.9\\
272	3.9\\
273	2\\
274	2\\
275	2\\
276	2\\
277	2\\
278	2\\
279	2\\
280	2\\
281	2\\
282	2\\
283	2\\
284	2\\
285	2\\
286	2\\
287	2\\
288	2\\
289	2\\
290	2\\
291	2\\
292	2\\
293	2\\
294	2\\
295	2\\
296	2\\
297	2\\
298	2\\
299	2\\
300	1.2\\
301	2.9\\
302	1.2\\
303	1.2\\
304	1.2\\
305	2.9\\
306	2\\
307	2\\
308	2\\
309	2\\
310	2\\
311	2\\
312	2\\
313	2\\
314	2\\
315	2\\
316	2\\
317	2\\
318	2\\
319	2\\
320	2\\
321	2\\
322	2\\
323	2\\
324	2\\
325	2\\
326	2\\
327	2\\
328	2\\
329	2\\
330	2\\
331	2\\
332	2\\
333	2.1\\
334	2.1\\
335	2.1\\
336	2.1\\
337	2.1\\
338	2.1\\
339	2.1\\
340	2.1\\
341	2.1\\
342	2.1\\
343	2.1\\
344	2.1\\
345	2.1\\
346	2.1\\
347	2.1\\
348	2.1\\
349	2.1\\
350	2.1\\
351	2.1\\
352	2.1\\
353	2.1\\
354	2.1\\
355	2.1\\
356	2.1\\
357	2.1\\
358	2.1\\
359	2.1\\
360	2.1\\
361	2.1\\
362	2.1\\
363	2.1\\
364	2.1\\
365	2.1\\
366	2.1\\
367	2.1\\
368	2.1\\
369	2.1\\
370	2.2\\
371	2.2\\
372	2.2\\
373	2.2\\
374	2.2\\
375	2.2\\
376	2.2\\
377	2.2\\
378	2.2\\
379	2.2\\
380	2.2\\
381	2.2\\
382	2.2\\
383	2.2\\
384	2.2\\
385	2.2\\
386	2.2\\
387	2.2\\
388	2.2\\
389	2.2\\
390	2.2\\
391	2.2\\
392	2.2\\
393	2.2\\
394	2.2\\
395	2.2\\
396	2.2\\
397	2.2\\
398	2.2\\
399	2.2\\
400	2.2\\
401	2.2\\
402	2.2\\
403	3.7\\
404	3.7\\
405	3.7\\
406	3.7\\
407	3.7\\
408	3.7\\
409	3.7\\
410	3.7\\
411	3.7\\
412	2.3\\
413	2.3\\
414	2.3\\
415	2.3\\
416	2.3\\
417	2.4\\
418	2.4\\
419	2.4\\
420	2.4\\
421	2.4\\
422	2.4\\
423	2.4\\
424	2.4\\
425	2.4\\
426	2.4\\
427	2.4\\
428	2.4\\
429	2.4\\
430	2.4\\
431	2.4\\
432	2.4\\
433	2.4\\
434	2.4\\
435	2.4\\
436	2.4\\
437	2.4\\
438	2.4\\
439	2.4\\
440	2.4\\
441	2.4\\
442	2.5\\
443	2.5\\
444	2.5\\
445	2.5\\
446	2.5\\
447	2.5\\
448	2.5\\
449	2.5\\
450	2.5\\
451	2.5\\
452	2.5\\
453	2.5\\
454	2.5\\
455	2.5\\
456	2.5\\
457	2.5\\
458	2.5\\
459	2.5\\
460	2.5\\
461	2.5\\
462	2.5\\
463	2.5\\
464	2.5\\
465	2.5\\
466	2.5\\
467	2.5\\
468	2.5\\
469	2.5\\
470	2.5\\
471	2.5\\
472	2.5\\
473	2.5\\
474	2.5\\
475	2.5\\
476	2.5\\
477	2.5\\
478	2.5\\
479	2.5\\
480	2.5\\
481	2.5\\
482	2.5\\
483	2.5\\
484	2.5\\
485	2.5\\
486	2.5\\
487	2.5\\
488	2.5\\
489	2.5\\
490	2.5\\
491	2.5\\
492	2.5\\
493	2.5\\
494	2.5\\
495	2.5\\
496	2.5\\
1	2.9\\
2	3.7\\
3	3.7\\
4	3.7\\
5	3.7\\
6	3\\
7	2.9\\
8	2.9\\
9	2.9\\
10	2.9\\
11	2.9\\
12	2.9\\
13	2.9\\
14	2.9\\
15	3\\
16	3\\
17	3\\
18	3\\
19	3\\
20	3\\
21	3\\
22	3\\
23	3\\
24	3\\
25	3\\
26	3\\
27	3\\
28	3\\
29	3\\
30	3\\
31	3\\
32	3\\
33	3\\
34	2.9\\
35	3\\
36	3\\
37	3\\
38	3\\
39	3\\
40	3\\
41	3\\
42	3\\
43	3\\
44	3\\
45	3\\
46	3\\
47	3\\
48	3\\
49	3\\
50	1.2\\
51	1.2\\
52	1.2\\
53	1.2\\
54	1.2\\
55	3\\
56	3.1\\
57	3.1\\
58	3.1\\
59	3.1\\
60	3.1\\
61	3.1\\
62	3.1\\
63	3.1\\
64	3.1\\
65	3.1\\
66	3.1\\
67	3.1\\
68	3.1\\
69	3.1\\
70	3.1\\
71	3.1\\
72	3.1\\
73	3.1\\
74	3.1\\
75	3.1\\
76	3.1\\
77	1.2\\
78	1.2\\
79	1.2\\
80	1.2\\
81	1.2\\
82	1.2\\
83	1.2\\
84	1.2\\
85	1.2\\
86	1.2\\
87	1.2\\
88	1.2\\
89	1.2\\
90	1.2\\
91	1.2\\
92	1.2\\
93	1.2\\
94	1.2\\
95	1.2\\
96	1.2\\
97	1.2\\
98	1.2\\
99	1.2\\
100	1.2\\
101	1.2\\
102	1.2\\
103	1.2\\
104	1.2\\
105	1.2\\
106	1.2\\
107	1.2\\
108	1.2\\
109	1.2\\
110	1.2\\
111	1.2\\
112	1.2\\
113	1.2\\
114	1.2\\
115	1.2\\
116	1.2\\
117	1.2\\
118	1.2\\
119	1.2\\
120	1.2\\
121	1.2\\
122	2.7\\
123	2.7\\
124	2.7\\
125	1.2\\
126	1.2\\
127	2.7\\
128	2.7\\
129	2.7\\
130	2.7\\
131	2.7\\
132	2.7\\
133	2.7\\
134	2.7\\
135	2.7\\
136	1.2\\
137	1.2\\
138	2.7\\
139	2.7\\
140	2.8\\
141	2.8\\
142	2.7\\
143	2.7\\
144	2.7\\
145	2.7\\
146	3.8\\
147	3.8\\
148	3.8\\
149	2.7\\
150	1.2\\
151	1.2\\
152	1.2\\
153	1.2\\
154	1.2\\
155	1.2\\
156	1.2\\
157	1.2\\
158	1.2\\
159	1.2\\
160	1.2\\
161	1.2\\
162	2.7\\
163	2.8\\
164	1.2\\
165	2.8\\
166	2.8\\
167	3.7\\
168	3.7\\
169	2.7\\
170	2.7\\
171	3.8\\
172	3.9\\
173	3.9\\
174	3.9\\
175	3.8\\
176	3.8\\
177	3.8\\
178	3.8\\
179	3.9\\
180	3.9\\
181	3.9\\
182	3.9\\
183	3.9\\
184	3.9\\
185	3.9\\
186	3.9\\
187	3.9\\
188	3.9\\
189	3.9\\
190	3.9\\
191	3.9\\
192	3.9\\
193	3.9\\
194	3.9\\
195	3.9\\
196	3.9\\
197	1.2\\
198	1.2\\
199	1.2\\
200	1.2\\
201	1.2\\
202	1.2\\
203	1.2\\
204	1.2\\
205	1.2\\
206	1.2\\
207	3.9\\
208	3.9\\
209	4\\
210	4\\
211	4\\
212	4\\
213	4\\
214	2.2\\
215	2.2\\
216	2.2\\
217	2.2\\
218	2.2\\
219	2.2\\
220	2.2\\
221	2.2\\
222	2.1\\
223	4\\
224	4\\
225	4\\
226	4\\
227	4\\
228	4\\
229	4\\
230	4\\
231	4\\
232	4\\
233	4\\
234	4\\
235	4\\
236	2.2\\
237	2.2\\
238	2.2\\
239	2.2\\
240	2.2\\
241	2.2\\
242	2.2\\
243	2.2\\
244	2.2\\
245	2.2\\
246	2.2\\
247	2.2\\
248	2.2\\
249	3.6\\
250	3.6\\
251	3.6\\
252	3.6\\
253	3.5\\
254	3.5\\
255	3.5\\
256	3.5\\
257	3.5\\
258	2.2\\
259	3.5\\
260	3.5\\
261	3.5\\
262	1.2\\
263	1.2\\
264	1.2\\
265	1.2\\
266	1.2\\
267	1.2\\
268	1.2\\
269	1.2\\
270	1.2\\
271	2\\
272	2\\
273	3.9\\
274	3.9\\
275	3.9\\
276	3.9\\
277	3.9\\
278	3.9\\
279	3.9\\
280	3.9\\
281	3.9\\
282	3.9\\
283	3.9\\
284	2.9\\
285	2.9\\
286	2.9\\
287	2.9\\
288	2.9\\
289	2.9\\
290	2.9\\
291	2.9\\
292	2.9\\
293	2.9\\
294	2.9\\
295	2.9\\
296	2.9\\
297	2.9\\
298	2.9\\
299	1.2\\
300	2\\
301	1.2\\
302	2.9\\
303	2\\
304	2\\
305	2\\
306	2.9\\
307	2.9\\
308	2.9\\
309	2.9\\
310	2.9\\
311	2.9\\
312	2.9\\
313	2.9\\
314	2.9\\
315	2.9\\
316	2.9\\
317	2.9\\
318	2.9\\
319	2.9\\
320	2.9\\
321	2.9\\
322	2.9\\
323	2.9\\
324	2.9\\
325	2.9\\
326	2.9\\
327	2.9\\
328	2.9\\
329	2.9\\
330	2.9\\
331	2.9\\
332	2.9\\
333	2.9\\
334	2.9\\
335	2.9\\
336	2.9\\
337	2.9\\
338	2.9\\
339	2.9\\
340	2.9\\
341	2.9\\
342	2.9\\
343	2.9\\
344	2.9\\
345	2.9\\
346	2.9\\
347	2.9\\
348	2.9\\
349	2.9\\
350	2.9\\
351	2.9\\
352	2.9\\
353	3.6\\
354	3.6\\
355	3.6\\
356	3.6\\
357	3.6\\
358	3.6\\
359	3.6\\
360	3.6\\
361	3.6\\
362	3.6\\
363	3.6\\
364	3.6\\
365	3.6\\
366	3.6\\
367	3.6\\
368	3.6\\
369	3.6\\
370	3.6\\
371	2.9\\
372	2.9\\
373	2.9\\
374	2.9\\
375	3.7\\
376	3.7\\
377	3.7\\
378	3.7\\
379	3.7\\
380	3.7\\
381	3.7\\
382	3.7\\
383	3.7\\
384	3.7\\
385	3.7\\
386	3.7\\
387	3.7\\
388	3.7\\
389	3.7\\
390	3.7\\
391	3.7\\
392	3.7\\
393	3.7\\
394	3.7\\
395	3.7\\
396	3.7\\
397	3.7\\
398	3.7\\
399	3.7\\
400	3.7\\
401	3.7\\
402	3.7\\
403	2.2\\
404	2.2\\
405	2.2\\
406	2.2\\
407	2.2\\
408	2.2\\
409	2.2\\
410	2.2\\
411	2.2\\
412	3.7\\
413	3.7\\
414	3.7\\
415	3.7\\
416	3.7\\
417	3.7\\
418	3.7\\
419	3.7\\
420	3.7\\
421	3.7\\
422	3.7\\
423	3.7\\
424	3.7\\
425	2.9\\
426	2.9\\
427	2.9\\
428	2.9\\
429	2.9\\
430	2.9\\
431	2.9\\
432	2.9\\
433	2.1\\
434	2.1\\
435	2.2\\
436	2.1\\
437	3.7\\
438	3.7\\
439	3.7\\
440	3.7\\
441	3.7\\
442	3.7\\
443	3.7\\
444	3.7\\
445	3.7\\
446	3.7\\
447	3.7\\
448	3.7\\
449	3.7\\
450	3.7\\
451	3.7\\
452	3.7\\
453	3.7\\
454	3.7\\
455	3.7\\
456	3.7\\
457	3.7\\
458	3.7\\
459	3.7\\
460	3.7\\
461	3.7\\
462	3.7\\
463	3\\
464	3\\
465	3\\
466	3\\
467	3\\
468	3\\
469	3\\
470	3\\
471	3\\
472	3\\
473	3\\
474	3\\
475	3\\
476	3\\
477	3\\
478	3\\
479	3\\
480	3\\
481	3\\
482	3\\
483	3\\
484	3\\
485	3\\
486	3\\
487	3\\
488	3\\
489	3\\
490	3\\
491	3\\
492	3\\
493	3\\
494	3\\
495	3\\
496	3\\
};
\addplot [color=mycolor3, dashed, line width=\trajDashedLinewidth]
  table[row sep=crcr]{%
1	3\\
2	3.00634660921834\\
3	3.01269296279619\\
4	3.01903880510335\\
5	3.02538388053021\\
6	3.03172793349807\\
7	3.03807070846939\\
8	3.04441194995813\\
9	3.05075140253999\\
10	3.05708881086277\\
11	3.06342391965656\\
12	3.06975647374413\\
13	3.0760862180511\\
14	3.0824128976163\\
15	3.088736257602\\
16	3.09505604330418\\
17	3.1013720001628\\
18	3.10768387377204\\
19	3.11399140989054\\
20	3.12029435445167\\
21	3.12659245357375\\
22	3.13288545357025\\
23	3.13917310096007\\
24	3.14545514247766\\
25	3.15173132508333\\
26	3.15800139597335\\
27	3.16426510259018\\
28	3.17052219263262\\
29	3.17677241406602\\
30	3.18301551513234\\
31	3.18925124436041\\
32	3.19547935057595\\
33	3.20169958291175\\
34	3.20791169081776\\
35	3.21411542407118\\
36	3.22031053278654\\
37	3.22649676742576\\
38	3.23267387880822\\
39	3.23884161812077\\
40	3.24499973692776\\
41	3.25114798718108\\
42	3.25728612123009\\
43	3.26341389183166\\
44	3.26953105216007\\
45	3.275637355817\\
46	3.28173255684143\\
47	3.28781640971955\\
48	3.29388866939466\\
49	3.29994909127701\\
50	3.30599743125371\\
51	3.31203344569849\\
52	3.31805689148158\\
53	3.32406752597945\\
54	3.33006510708463\\
55	3.33604939321543\\
56	3.34202014332567\\
57	3.34797711691441\\
58	3.35392007403561\\
59	3.35984877530784\\
60	3.36576298192387\\
61	3.37166245566033\\
62	3.37754695888726\\
63	3.38341625457774\\
64	3.38927010631739\\
65	3.39510827831392\\
66	3.40093053540661\\
67	3.4067366430758\\
68	3.41252636745231\\
69	3.41829947532689\\
70	3.4240557341596\\
71	3.42979491208917\\
72	3.43551677794235\\
73	3.44122110124322\\
74	3.44690765222247\\
75	3.45257620182665\\
76	3.45822652172741\\
77	3.46385838433069\\
78	3.46947156278589\\
79	3.47506583099499\\
80	3.4806409636217\\
81	3.48619673610047\\
82	3.4917329246456\\
83	3.49724930626024\\
84	3.50274565874532\\
85	3.50822176070857\\
86	3.51367739157341\\
87	3.51911233158781\\
88	3.52452636183319\\
89	3.5299192642332\\
90	3.53529082156252\\
91	3.5406408174556\\
92	3.54596903641538\\
93	3.55127526382198\\
94	3.55655928594134\\
95	3.56182088993382\\
96	3.56705986386277\\
97	3.57227599670309\\
98	3.57746907834971\\
99	3.58263889962605\\
100	3.58778525229247\\
101	3.59290792905464\\
102	3.59800672357188\\
103	3.60308143046549\\
104	3.60813184532702\\
105	3.61315776472649\\
106	3.61815898622061\\
107	3.62313530836089\\
108	3.62808653070182\\
109	3.63301245380887\\
110	3.63791287926659\\
111	3.64278760968654\\
112	3.6476364487153\\
113	3.65245920104234\\
114	3.65725567240791\\
115	3.66202566961082\\
116	3.66676900051629\\
117	3.67148547406365\\
118	3.67617490027402\\
119	3.680837090258\\
120	3.68547185622327\\
121	3.69007901148211\\
122	3.694658370459\\
123	3.69920974869801\\
124	3.7037329628703\\
125	3.70822783078146\\
126	3.71269417137886\\
127	3.71713180475896\\
128	3.72154055217454\\
129	3.72592023604188\\
130	3.73027067994796\\
131	3.73459170865753\\
132	3.7388831481202\\
133	3.74314482547739\\
134	3.74737656906938\\
135	3.75157820844214\\
136	3.75574957435426\\
137	3.75989049878372\\
138	3.76400081493469\\
139	3.76808035724423\\
140	3.77212896138898\\
141	3.77614646429176\\
142	3.78013270412812\\
143	3.78408752033292\\
144	3.78801075360672\\
145	3.79190224592227\\
146	3.79576184053083\\
147	3.79958938196848\\
148	3.80338471606242\\
149	3.80714768993714\\
150	3.8108781520206\\
151	3.81457595205034\\
152	3.8182409410795\\
153	3.82187297148286\\
154	3.82547189696277\\
155	3.82903757255504\\
156	3.83256985463477\\
157	3.83606860092216\\
158	3.8395336704882\\
159	3.84296492376042\\
160	3.84636222252842\\
161	3.84972542994951\\
162	3.85305441055419\\
163	3.85634903025159\\
164	3.85960915633492\\
165	3.86283465748677\\
166	3.86602540378444\\
167	3.86918126670512\\
168	3.87230211913111\\
169	3.87538783535494\\
170	3.8784382910844\\
171	3.88145336344758\\
172	3.88443293099781\\
173	3.88737687371855\\
174	3.8902850730282\\
175	3.89315741178492\\
176	3.89599377429134\\
177	3.89879404629917\\
178	3.90155811501387\\
179	3.90428586909916\\
180	3.90697719868149\\
181	3.90963199535452\\
182	3.91225015218341\\
183	3.91483156370918\\
184	3.91737612595296\\
185	3.91988373642016\\
186	3.92235429410458\\
187	3.92478769949253\\
188	3.92718385456679\\
189	3.92954266281058\\
190	3.93186402921145\\
191	3.93414786026511\\
192	3.93639406397916\\
193	3.93860254987686\\
194	3.9407732290007\\
195	3.94290601391606\\
196	3.94500081871467\\
197	3.94705755901809\\
198	3.94907615198113\\
199	3.95105651629515\\
200	3.95299857219138\\
201	3.95490224144407\\
202	3.95676744737372\\
203	3.9585941148501\\
204	3.9603821702953\\
205	3.96213154168673\\
206	3.96384215855994\\
207	3.96551395201155\\
208	3.96714685470196\\
209	3.96874080085808\\
210	3.970295726276\\
211	3.97181156832354\\
212	3.97328826594282\\
213	3.97472575965266\\
214	3.97612399155103\\
215	3.97748290531735\\
216	3.97880244621478\\
217	3.98008256109239\\
218	3.98132319838736\\
219	3.98252430812698\\
220	3.98368584193073\\
221	3.98480775301221\\
222	3.98588999618099\\
223	3.98693252784448\\
224	3.98793530600966\\
225	3.98889829028477\\
226	3.98982144188093\\
227	3.99070472361375\\
228	3.99154809990476\\
229	3.99235153678288\\
230	3.9931150018858\\
231	3.99383846446125\\
232	3.99452189536827\\
233	3.99516526707836\\
234	3.9957685536766\\
235	3.99633173086269\\
236	3.99685477595194\\
237	3.99733766787617\\
238	3.99778038718456\\
239	3.99818291604445\\
240	3.99854523824203\\
241	3.99886733918301\\
242	3.99914920589321\\
243	3.9993908270191\\
244	3.99959219282819\\
245	3.99975329520951\\
246	3.99987412767388\\
247	3.99995468535417\\
248	3.99999496500555\\
249	3.99999496500555\\
250	3.99995468535417\\
251	3.99987412767388\\
252	3.99975329520951\\
253	3.99959219282819\\
254	3.9993908270191\\
255	3.99914920589321\\
256	3.99886733918301\\
257	3.99854523824203\\
258	3.99818291604445\\
259	3.99778038718456\\
260	3.99733766787617\\
261	3.99685477595194\\
262	3.99633173086269\\
263	3.9957685536766\\
264	3.99516526707836\\
265	3.99452189536827\\
266	3.99383846446125\\
267	3.9931150018858\\
268	3.99235153678288\\
269	3.99154809990476\\
270	3.99070472361375\\
271	3.98982144188093\\
272	3.98889829028477\\
273	3.98793530600966\\
274	3.98693252784448\\
275	3.98588999618099\\
276	3.98480775301221\\
277	3.98368584193073\\
278	3.98252430812698\\
279	3.98132319838736\\
280	3.98008256109239\\
281	3.97880244621478\\
282	3.97748290531735\\
283	3.97612399155103\\
284	3.97472575965266\\
285	3.97328826594282\\
286	3.97181156832354\\
287	3.970295726276\\
288	3.96874080085808\\
289	3.96714685470196\\
290	3.96551395201155\\
291	3.96384215855994\\
292	3.96213154168673\\
293	3.9603821702953\\
294	3.9585941148501\\
295	3.95676744737372\\
296	3.95490224144407\\
297	3.95299857219138\\
298	3.95105651629515\\
299	3.94907615198113\\
300	3.94705755901809\\
301	3.94500081871467\\
302	3.94290601391606\\
303	3.9407732290007\\
304	3.93860254987686\\
305	3.93639406397916\\
306	3.93414786026511\\
307	3.93186402921145\\
308	3.92954266281058\\
309	3.92718385456679\\
310	3.92478769949253\\
311	3.92235429410458\\
312	3.91988373642016\\
313	3.91737612595296\\
314	3.91483156370918\\
315	3.91225015218341\\
316	3.90963199535452\\
317	3.90697719868149\\
318	3.90428586909916\\
319	3.90155811501387\\
320	3.89879404629917\\
321	3.89599377429134\\
322	3.89315741178492\\
323	3.8902850730282\\
324	3.88737687371855\\
325	3.88443293099781\\
326	3.88145336344758\\
327	3.8784382910844\\
328	3.87538783535494\\
329	3.87230211913111\\
330	3.86918126670512\\
331	3.86602540378444\\
332	3.86283465748677\\
333	3.85960915633492\\
334	3.85634903025159\\
335	3.85305441055419\\
336	3.84972542994951\\
337	3.84636222252842\\
338	3.84296492376042\\
339	3.8395336704882\\
340	3.83606860092216\\
341	3.83256985463477\\
342	3.82903757255504\\
343	3.82547189696277\\
344	3.82187297148286\\
345	3.8182409410795\\
346	3.81457595205034\\
347	3.8108781520206\\
348	3.80714768993714\\
349	3.80338471606242\\
350	3.79958938196848\\
351	3.79576184053083\\
352	3.79190224592227\\
353	3.78801075360672\\
354	3.78408752033292\\
355	3.78013270412812\\
356	3.77614646429176\\
357	3.77212896138898\\
358	3.76808035724423\\
359	3.76400081493469\\
360	3.75989049878372\\
361	3.75574957435426\\
362	3.75157820844214\\
363	3.74737656906938\\
364	3.74314482547739\\
365	3.7388831481202\\
366	3.73459170865753\\
367	3.73027067994796\\
368	3.72592023604188\\
369	3.72154055217454\\
370	3.71713180475896\\
371	3.71269417137886\\
372	3.70822783078146\\
373	3.7037329628703\\
374	3.69920974869801\\
375	3.694658370459\\
376	3.69007901148211\\
377	3.68547185622327\\
378	3.680837090258\\
379	3.67617490027402\\
380	3.67148547406365\\
381	3.66676900051629\\
382	3.66202566961082\\
383	3.65725567240791\\
384	3.65245920104234\\
385	3.6476364487153\\
386	3.64278760968654\\
387	3.63791287926659\\
388	3.63301245380887\\
389	3.62808653070182\\
390	3.62313530836089\\
391	3.61815898622061\\
392	3.61315776472649\\
393	3.60813184532702\\
394	3.60308143046549\\
395	3.59800672357188\\
396	3.59290792905464\\
397	3.58778525229247\\
398	3.58263889962605\\
399	3.57746907834971\\
400	3.57227599670309\\
401	3.56705986386277\\
402	3.56182088993382\\
403	3.55655928594134\\
404	3.55127526382198\\
405	3.54596903641538\\
406	3.5406408174556\\
407	3.53529082156252\\
408	3.5299192642332\\
409	3.52452636183319\\
410	3.51911233158781\\
411	3.51367739157341\\
412	3.50822176070857\\
413	3.50274565874532\\
414	3.49724930626024\\
415	3.4917329246456\\
416	3.48619673610047\\
417	3.4806409636217\\
418	3.47506583099499\\
419	3.46947156278589\\
420	3.46385838433069\\
421	3.45822652172741\\
422	3.45257620182665\\
423	3.44690765222247\\
424	3.44122110124322\\
425	3.43551677794235\\
426	3.42979491208917\\
427	3.4240557341596\\
428	3.41829947532689\\
429	3.41252636745231\\
430	3.4067366430758\\
431	3.40093053540661\\
432	3.39510827831392\\
433	3.38927010631739\\
434	3.38341625457774\\
435	3.37754695888726\\
436	3.37166245566033\\
437	3.36576298192387\\
438	3.35984877530784\\
439	3.35392007403561\\
440	3.34797711691441\\
441	3.34202014332567\\
442	3.33604939321543\\
443	3.33006510708463\\
444	3.32406752597945\\
445	3.31805689148158\\
446	3.31203344569849\\
447	3.3059974312537\\
448	3.29994909127701\\
449	3.29388866939466\\
450	3.28781640971955\\
451	3.28173255684143\\
452	3.275637355817\\
453	3.26953105216007\\
454	3.26341389183166\\
455	3.25728612123009\\
456	3.25114798718108\\
457	3.24499973692776\\
458	3.23884161812077\\
459	3.23267387880822\\
460	3.22649676742576\\
461	3.22031053278654\\
462	3.21411542407118\\
463	3.20791169081776\\
464	3.20169958291175\\
465	3.19547935057595\\
466	3.18925124436041\\
467	3.18301551513234\\
468	3.17677241406602\\
469	3.17052219263262\\
470	3.16426510259018\\
471	3.15800139597335\\
472	3.15173132508333\\
473	3.14545514247766\\
474	3.13917310096007\\
475	3.13288545357025\\
476	3.12659245357375\\
477	3.12029435445167\\
478	3.11399140989054\\
479	3.10768387377204\\
480	3.1013720001628\\
481	3.09505604330418\\
482	3.088736257602\\
483	3.0824128976163\\
484	3.0760862180511\\
485	3.06975647374413\\
486	3.06342391965656\\
487	3.05708881086277\\
488	3.05075140253999\\
489	3.04441194995813\\
490	3.03807070846939\\
491	3.03172793349807\\
492	3.02538388053021\\
493	3.01903880510335\\
494	3.01269296279619\\
495	3.00634660921834\\
496	3\\
};
\addplot [color=mycolor4, dashed, line width=\trajDashedLinewidth]
  table[row sep=crcr]{%
1	3\\
2	2.99365339078166\\
3	2.98730703720381\\
4	2.98096119489665\\
5	2.97461611946979\\
6	2.96827206650193\\
7	2.96192929153061\\
8	2.95558805004187\\
9	2.94924859746001\\
10	2.94291118913723\\
11	2.93657608034344\\
12	2.93024352625587\\
13	2.9239137819489\\
14	2.9175871023837\\
15	2.911263742398\\
16	2.90494395669582\\
17	2.8986279998372\\
18	2.89231612622796\\
19	2.88600859010946\\
20	2.87970564554833\\
21	2.87340754642625\\
22	2.86711454642975\\
23	2.86082689903993\\
24	2.85454485752234\\
25	2.84826867491667\\
26	2.84199860402665\\
27	2.83573489740982\\
28	2.82947780736738\\
29	2.82322758593398\\
30	2.81698448486766\\
31	2.81074875563959\\
32	2.80452064942405\\
33	2.79830041708825\\
34	2.79208830918224\\
35	2.78588457592882\\
36	2.77968946721346\\
37	2.77350323257424\\
38	2.76732612119178\\
39	2.76115838187923\\
40	2.75500026307224\\
41	2.74885201281892\\
42	2.74271387876991\\
43	2.73658610816834\\
44	2.73046894783993\\
45	2.724362644183\\
46	2.71826744315857\\
47	2.71218359028045\\
48	2.70611133060534\\
49	2.70005090872299\\
50	2.69400256874629\\
51	2.68796655430151\\
52	2.68194310851842\\
53	2.67593247402055\\
54	2.66993489291537\\
55	2.66395060678457\\
56	2.65797985667433\\
57	2.65202288308559\\
58	2.64607992596439\\
59	2.64015122469216\\
60	2.63423701807613\\
61	2.62833754433967\\
62	2.62245304111274\\
63	2.61658374542226\\
64	2.61072989368261\\
65	2.60489172168608\\
66	2.59906946459339\\
67	2.5932633569242\\
68	2.58747363254769\\
69	2.58170052467311\\
70	2.5759442658404\\
71	2.57020508791083\\
72	2.56448322205765\\
73	2.55877889875678\\
74	2.55309234777753\\
75	2.54742379817335\\
76	2.54177347827259\\
77	2.53614161566931\\
78	2.53052843721411\\
79	2.524934169005\\
80	2.5193590363783\\
81	2.51380326389953\\
82	2.5082670753544\\
83	2.50275069373976\\
84	2.49725434125468\\
85	2.49177823929143\\
86	2.48632260842659\\
87	2.48088766841219\\
88	2.47547363816681\\
89	2.47008073576679\\
90	2.46470917843748\\
91	2.4593591825444\\
92	2.45403096358462\\
93	2.44872473617802\\
94	2.44344071405866\\
95	2.43817911006618\\
96	2.43294013613723\\
97	2.42772400329691\\
98	2.42253092165029\\
99	2.41736110037395\\
100	2.41221474770753\\
101	2.40709207094536\\
102	2.40199327642812\\
103	2.39691856953451\\
104	2.39186815467298\\
105	2.38684223527351\\
106	2.38184101377939\\
107	2.37686469163911\\
108	2.37191346929818\\
109	2.36698754619113\\
110	2.36208712073341\\
111	2.35721239031346\\
112	2.3523635512847\\
113	2.34754079895766\\
114	2.34274432759209\\
115	2.33797433038918\\
116	2.33323099948371\\
117	2.32851452593635\\
118	2.32382509972598\\
119	2.319162909742\\
120	2.31452814377673\\
121	2.30992098851789\\
122	2.305341629541\\
123	2.30079025130199\\
124	2.2962670371297\\
125	2.29177216921854\\
126	2.28730582862114\\
127	2.28286819524104\\
128	2.27845944782546\\
129	2.27407976395812\\
130	2.26972932005204\\
131	2.26540829134247\\
132	2.2611168518798\\
133	2.25685517452261\\
134	2.25262343093062\\
135	2.24842179155786\\
136	2.24425042564574\\
137	2.24010950121628\\
138	2.23599918506531\\
139	2.23191964275577\\
140	2.22787103861102\\
141	2.22385353570824\\
142	2.21986729587188\\
143	2.21591247966708\\
144	2.21198924639328\\
145	2.20809775407773\\
146	2.20423815946917\\
147	2.20041061803151\\
148	2.19661528393758\\
149	2.19285231006286\\
150	2.1891218479794\\
151	2.18542404794966\\
152	2.1817590589205\\
153	2.17812702851714\\
154	2.17452810303723\\
155	2.17096242744496\\
156	2.16743014536523\\
157	2.16393139907784\\
158	2.1604663295118\\
159	2.15703507623958\\
160	2.15363777747158\\
161	2.15027457005049\\
162	2.14694558944581\\
163	2.14365096974841\\
164	2.14039084366508\\
165	2.13716534251323\\
166	2.13397459621556\\
167	2.13081873329488\\
168	2.12769788086889\\
169	2.12461216464506\\
170	2.1215617089156\\
171	2.11854663655242\\
172	2.11556706900219\\
173	2.11262312628145\\
174	2.1097149269718\\
175	2.10684258821508\\
176	2.10400622570866\\
177	2.10120595370083\\
178	2.09844188498613\\
179	2.09571413090084\\
180	2.09302280131851\\
181	2.09036800464548\\
182	2.08774984781659\\
183	2.08516843629082\\
184	2.08262387404704\\
185	2.08011626357984\\
186	2.07764570589542\\
187	2.07521230050747\\
188	2.07281614543321\\
189	2.07045733718942\\
190	2.06813597078855\\
191	2.06585213973489\\
192	2.06360593602084\\
193	2.06139745012314\\
194	2.0592267709993\\
195	2.05709398608394\\
196	2.05499918128533\\
197	2.05294244098191\\
198	2.05092384801887\\
199	2.04894348370485\\
200	2.04700142780862\\
201	2.04509775855593\\
202	2.04323255262628\\
203	2.0414058851499\\
204	2.0396178297047\\
205	2.03786845831327\\
206	2.03615784144006\\
207	2.03448604798845\\
208	2.03285314529804\\
209	2.03125919914192\\
210	2.029704273724\\
211	2.02818843167646\\
212	2.02671173405718\\
213	2.02527424034734\\
214	2.02387600844897\\
215	2.02251709468265\\
216	2.02119755378522\\
217	2.01991743890761\\
218	2.01867680161264\\
219	2.01747569187302\\
220	2.01631415806927\\
221	2.01519224698779\\
222	2.01411000381901\\
223	2.01306747215552\\
224	2.01206469399034\\
225	2.01110170971523\\
226	2.01017855811907\\
227	2.00929527638625\\
228	2.00845190009524\\
229	2.00764846321712\\
230	2.0068849981142\\
231	2.00616153553875\\
232	2.00547810463173\\
233	2.00483473292164\\
234	2.0042314463234\\
235	2.00366826913731\\
236	2.00314522404806\\
237	2.00266233212383\\
238	2.00221961281544\\
239	2.00181708395555\\
240	2.00145476175797\\
241	2.00113266081699\\
242	2.00085079410679\\
243	2.0006091729809\\
244	2.00040780717181\\
245	2.00024670479049\\
246	2.00012587232612\\
247	2.00004531464583\\
248	2.00000503499445\\
249	2.00000503499445\\
250	2.00004531464583\\
251	2.00012587232612\\
252	2.00024670479049\\
253	2.00040780717181\\
254	2.0006091729809\\
255	2.00085079410679\\
256	2.00113266081699\\
257	2.00145476175797\\
258	2.00181708395555\\
259	2.00221961281544\\
260	2.00266233212383\\
261	2.00314522404806\\
262	2.00366826913731\\
263	2.0042314463234\\
264	2.00483473292164\\
265	2.00547810463173\\
266	2.00616153553875\\
267	2.0068849981142\\
268	2.00764846321712\\
269	2.00845190009524\\
270	2.00929527638625\\
271	2.01017855811907\\
272	2.01110170971523\\
273	2.01206469399034\\
274	2.01306747215552\\
275	2.01411000381901\\
276	2.01519224698779\\
277	2.01631415806927\\
278	2.01747569187302\\
279	2.01867680161264\\
280	2.01991743890761\\
281	2.02119755378522\\
282	2.02251709468265\\
283	2.02387600844897\\
284	2.02527424034734\\
285	2.02671173405718\\
286	2.02818843167646\\
287	2.029704273724\\
288	2.03125919914192\\
289	2.03285314529804\\
290	2.03448604798845\\
291	2.03615784144006\\
292	2.03786845831327\\
293	2.0396178297047\\
294	2.0414058851499\\
295	2.04323255262628\\
296	2.04509775855593\\
297	2.04700142780862\\
298	2.04894348370485\\
299	2.05092384801887\\
300	2.05294244098191\\
301	2.05499918128533\\
302	2.05709398608394\\
303	2.0592267709993\\
304	2.06139745012314\\
305	2.06360593602084\\
306	2.06585213973489\\
307	2.06813597078855\\
308	2.07045733718942\\
309	2.07281614543321\\
310	2.07521230050747\\
311	2.07764570589542\\
312	2.08011626357984\\
313	2.08262387404704\\
314	2.08516843629082\\
315	2.08774984781659\\
316	2.09036800464548\\
317	2.09302280131851\\
318	2.09571413090084\\
319	2.09844188498613\\
320	2.10120595370083\\
321	2.10400622570866\\
322	2.10684258821508\\
323	2.1097149269718\\
324	2.11262312628145\\
325	2.11556706900219\\
326	2.11854663655242\\
327	2.1215617089156\\
328	2.12461216464506\\
329	2.12769788086889\\
330	2.13081873329488\\
331	2.13397459621556\\
332	2.13716534251323\\
333	2.14039084366508\\
334	2.14365096974841\\
335	2.14694558944581\\
336	2.15027457005049\\
337	2.15363777747158\\
338	2.15703507623958\\
339	2.1604663295118\\
340	2.16393139907784\\
341	2.16743014536523\\
342	2.17096242744496\\
343	2.17452810303723\\
344	2.17812702851714\\
345	2.1817590589205\\
346	2.18542404794966\\
347	2.1891218479794\\
348	2.19285231006286\\
349	2.19661528393758\\
350	2.20041061803152\\
351	2.20423815946917\\
352	2.20809775407773\\
353	2.21198924639328\\
354	2.21591247966708\\
355	2.21986729587188\\
356	2.22385353570824\\
357	2.22787103861102\\
358	2.23191964275577\\
359	2.23599918506531\\
360	2.24010950121628\\
361	2.24425042564574\\
362	2.24842179155786\\
363	2.25262343093062\\
364	2.25685517452261\\
365	2.2611168518798\\
366	2.26540829134247\\
367	2.26972932005204\\
368	2.27407976395812\\
369	2.27845944782546\\
370	2.28286819524104\\
371	2.28730582862114\\
372	2.29177216921854\\
373	2.2962670371297\\
374	2.30079025130199\\
375	2.305341629541\\
376	2.30992098851789\\
377	2.31452814377673\\
378	2.319162909742\\
379	2.32382509972598\\
380	2.32851452593635\\
381	2.33323099948371\\
382	2.33797433038918\\
383	2.34274432759209\\
384	2.34754079895766\\
385	2.3523635512847\\
386	2.35721239031346\\
387	2.36208712073341\\
388	2.36698754619113\\
389	2.37191346929818\\
390	2.37686469163911\\
391	2.38184101377939\\
392	2.38684223527351\\
393	2.39186815467298\\
394	2.39691856953451\\
395	2.40199327642812\\
396	2.40709207094536\\
397	2.41221474770753\\
398	2.41736110037395\\
399	2.42253092165029\\
400	2.42772400329691\\
401	2.43294013613723\\
402	2.43817911006618\\
403	2.44344071405866\\
404	2.44872473617802\\
405	2.45403096358462\\
406	2.4593591825444\\
407	2.46470917843748\\
408	2.4700807357668\\
409	2.47547363816681\\
410	2.48088766841219\\
411	2.48632260842659\\
412	2.49177823929143\\
413	2.49725434125468\\
414	2.50275069373976\\
415	2.5082670753544\\
416	2.51380326389953\\
417	2.5193590363783\\
418	2.52493416900501\\
419	2.53052843721411\\
420	2.53614161566931\\
421	2.54177347827259\\
422	2.54742379817335\\
423	2.55309234777753\\
424	2.55877889875678\\
425	2.56448322205765\\
426	2.57020508791083\\
427	2.5759442658404\\
428	2.58170052467311\\
429	2.58747363254769\\
430	2.5932633569242\\
431	2.59906946459339\\
432	2.60489172168608\\
433	2.61072989368261\\
434	2.61658374542226\\
435	2.62245304111274\\
436	2.62833754433967\\
437	2.63423701807613\\
438	2.64015122469216\\
439	2.64607992596439\\
440	2.65202288308559\\
441	2.65797985667433\\
442	2.66395060678457\\
443	2.66993489291537\\
444	2.67593247402055\\
445	2.68194310851842\\
446	2.68796655430151\\
447	2.6940025687463\\
448	2.70005090872299\\
449	2.70611133060534\\
450	2.71218359028045\\
451	2.71826744315857\\
452	2.724362644183\\
453	2.73046894783993\\
454	2.73658610816834\\
455	2.74271387876991\\
456	2.74885201281892\\
457	2.75500026307224\\
458	2.76115838187923\\
459	2.76732612119178\\
460	2.77350323257424\\
461	2.77968946721346\\
462	2.78588457592882\\
463	2.79208830918224\\
464	2.79830041708825\\
465	2.80452064942405\\
466	2.81074875563959\\
467	2.81698448486766\\
468	2.82322758593398\\
469	2.82947780736738\\
470	2.83573489740982\\
471	2.84199860402665\\
472	2.84826867491667\\
473	2.85454485752234\\
474	2.86082689903993\\
475	2.86711454642975\\
476	2.87340754642625\\
477	2.87970564554833\\
478	2.88600859010946\\
479	2.89231612622796\\
480	2.8986279998372\\
481	2.90494395669582\\
482	2.911263742398\\
483	2.9175871023837\\
484	2.9239137819489\\
485	2.93024352625588\\
486	2.93657608034344\\
487	2.94291118913723\\
488	2.94924859746001\\
489	2.95558805004187\\
490	2.96192929153061\\
491	2.96827206650193\\
492	2.97461611946979\\
493	2.98096119489665\\
494	2.98730703720381\\
495	2.99365339078166\\
496	3\\
};
\end{axis}
\end{tikzpicture}%  % tikz
			\caption{Estimated x-Axis Positions}
		\end{subfigure}
		\begin{subfigure}{0.49\textwidth}
			\centering
			\setlength{\figurewidth}{0.8\textwidth}
			% This file was created by matlab2tikz.
%
\definecolor{lms_red}{rgb}{0.80000,0.20780,0.21960}%
\definecolor{mycolor2}{rgb}{0.80000,0.20784,0.21961}%
\definecolor{mycolor3}{rgb}{0.92900,0.69400,0.12500}%
\definecolor{mycolor4}{rgb}{0.49400,0.18400,0.55600}%
%
\begin{tikzpicture}

\begin{axis}[%
width=0.951\figurewidth,
height=\figureheight,
at={(0\figurewidth,0\figureheight)},
scale only axis,
xmin=0,
xmax=496,
xtick={0,99.2,198.4,297.6,396.8,496},
xticklabels={{0},{1},{2},{3},{4},{5}},
xlabel style={font=\color{white!15!black}},
xlabel={$t$~[s]},
ymin=1,
ymax=5,
ylabel style={font=\color{white!15!black}},
ylabel={$p_y^{(t)}$~[m]},
axis background/.style={fill=white},
axis x line*=bottom,
axis y line*=left
]
\addplot [color=mycolor2, draw=none, mark=x, mark options={solid, mycolor2}, forget plot]
  table[row sep=crcr]{%
1	4.5\\
2	1.3\\
3	1.4\\
4	1.6\\
5	1.7\\
6	1.7\\
7	1.9\\
8	2\\
9	2\\
10	2\\
11	2\\
12	2\\
13	2\\
14	2\\
15	2\\
16	2\\
17	2\\
18	3.9\\
19	3.9\\
20	3.9\\
21	3.9\\
22	3.9\\
23	3.9\\
24	3.9\\
25	3.9\\
26	3.9\\
27	3.9\\
28	3.9\\
29	2\\
30	2\\
31	2\\
32	2\\
33	2\\
34	2\\
35	3.9\\
36	3.9\\
37	3.9\\
38	3.9\\
39	3.9\\
40	3.9\\
41	3.9\\
42	3.9\\
43	3.9\\
44	3.9\\
45	3.9\\
46	3.9\\
47	3.9\\
48	3.9\\
49	3.9\\
50	3.9\\
51	3.9\\
52	3.9\\
53	3.9\\
54	3.9\\
55	3.9\\
56	3.9\\
57	3.9\\
58	3.9\\
59	3.9\\
60	3.9\\
61	3.9\\
62	3.9\\
63	3.9\\
64	3.9\\
65	3.9\\
66	3.9\\
67	3.9\\
68	3.9\\
69	3.9\\
70	3.9\\
71	3.9\\
72	3.9\\
73	3.9\\
74	3.9\\
75	3.9\\
76	3.8\\
77	3.9\\
78	3.9\\
79	3.9\\
80	3.9\\
81	3.9\\
82	3.9\\
83	3.9\\
84	3.9\\
85	3.9\\
86	3.9\\
87	3.9\\
88	3.9\\
89	3.9\\
90	3.9\\
91	3.9\\
92	3.8\\
93	3.8\\
94	3.8\\
95	3.8\\
96	3.8\\
97	3.8\\
98	3.8\\
99	3.8\\
100	3.8\\
101	3.8\\
102	3.8\\
103	3.8\\
104	3.8\\
105	3.8\\
106	3.8\\
107	3.8\\
108	3.8\\
109	3.8\\
110	3.8\\
111	3.8\\
112	3.8\\
113	3.8\\
114	3.8\\
115	3.8\\
116	3.8\\
117	3.8\\
118	3.8\\
119	3.8\\
120	3.8\\
121	3.8\\
122	3\\
123	3\\
124	3\\
125	3.8\\
126	3.8\\
127	3\\
128	3\\
129	3\\
130	3\\
131	3\\
132	3\\
133	3\\
134	3\\
135	3\\
136	3.8\\
137	3.8\\
138	2.2\\
139	2.2\\
140	2.2\\
141	2.2\\
142	2.2\\
143	2.2\\
144	2.2\\
145	2.2\\
146	2.2\\
147	2.2\\
148	2.2\\
149	2.2\\
150	3.8\\
151	3.8\\
152	3.8\\
153	3.7\\
154	3.7\\
155	3.6\\
156	3.6\\
157	3.6\\
158	3.6\\
159	3.6\\
160	3.6\\
161	3.6\\
162	3.5\\
163	3.6\\
164	3.6\\
165	3.6\\
166	3.6\\
167	3.6\\
168	3.6\\
169	3.5\\
170	3.5\\
171	3.5\\
172	3.5\\
173	3.5\\
174	3.5\\
175	3.5\\
176	3.5\\
177	3.5\\
178	3.5\\
179	3.5\\
180	3.5\\
181	3.5\\
182	3.5\\
183	3.5\\
184	3.5\\
185	3.5\\
186	3.5\\
187	3.5\\
188	3.5\\
189	3.5\\
190	3.5\\
191	3.5\\
192	3.5\\
193	3.5\\
194	3.4\\
195	3.4\\
196	3.4\\
197	3.4\\
198	3.4\\
199	3.4\\
200	3.4\\
201	3.4\\
202	3.4\\
203	3.4\\
204	3.4\\
205	3.4\\
206	3.5\\
207	3.4\\
208	3.4\\
209	3.4\\
210	3.4\\
211	3.4\\
212	3.4\\
213	3.4\\
214	2.6\\
215	2.6\\
216	2.6\\
217	2.6\\
218	2.6\\
219	2.6\\
220	2.6\\
221	2.6\\
222	2.6\\
223	3.3\\
224	3.3\\
225	3.3\\
226	3.3\\
227	3.3\\
228	3.3\\
229	3.3\\
230	3.3\\
231	3.3\\
232	3.3\\
233	3.3\\
234	3.3\\
235	3.3\\
236	2.6\\
237	2.6\\
238	2.7\\
239	2.7\\
240	2.7\\
241	2.7\\
242	2.7\\
243	2.7\\
244	2.7\\
245	2.7\\
246	2.7\\
247	2.7\\
248	2.7\\
249	2.7\\
250	2.7\\
251	2.7\\
252	2.8\\
253	2.8\\
254	2.8\\
255	2.8\\
256	2.8\\
257	2.8\\
258	2.9\\
259	2.9\\
260	2.9\\
261	2.9\\
262	2.9\\
263	2.9\\
264	2.9\\
265	2.9\\
266	2.9\\
267	2.9\\
268	2.9\\
269	2.9\\
270	2.9\\
271	2.9\\
272	2.9\\
273	2.9\\
274	2.9\\
275	2.9\\
276	2.9\\
277	2.9\\
278	2.9\\
279	2.9\\
280	2.9\\
281	2.9\\
282	2.9\\
283	2.9\\
284	2.9\\
285	2.9\\
286	2.9\\
287	2.9\\
288	2.9\\
289	2.9\\
290	2.9\\
291	2.9\\
292	2.9\\
293	2.9\\
294	2.9\\
295	2.9\\
296	2.9\\
297	2.9\\
298	2.9\\
299	2.9\\
300	2.3\\
301	1.2\\
302	2.3\\
303	2.3\\
304	2.3\\
305	1.2\\
306	2.8\\
307	2.8\\
308	2.8\\
309	2.8\\
310	2.8\\
311	2.8\\
312	2.8\\
313	2.8\\
314	2.8\\
315	2.8\\
316	2.8\\
317	2.8\\
318	2.8\\
319	2.8\\
320	2.8\\
321	2.8\\
322	2.8\\
323	2.8\\
324	2.8\\
325	2.8\\
326	2.8\\
327	2.8\\
328	2.8\\
329	2.8\\
330	2.7\\
331	2.7\\
332	2.7\\
333	2.7\\
334	2.7\\
335	2.7\\
336	2.7\\
337	2.7\\
338	2.7\\
339	2.7\\
340	2.7\\
341	2.7\\
342	2.7\\
343	2.7\\
344	2.7\\
345	2.7\\
346	2.7\\
347	2.7\\
348	2.7\\
349	2.7\\
350	2.7\\
351	2.7\\
352	2.7\\
353	2.7\\
354	2.7\\
355	2.7\\
356	2.7\\
357	2.6\\
358	2.6\\
359	2.6\\
360	2.7\\
361	2.6\\
362	2.6\\
363	2.6\\
364	2.6\\
365	2.6\\
366	2.6\\
367	2.6\\
368	2.6\\
369	2.6\\
370	2.5\\
371	2.5\\
372	2.5\\
373	2.5\\
374	2.5\\
375	2.5\\
376	2.5\\
377	2.5\\
378	2.5\\
379	2.5\\
380	2.5\\
381	2.5\\
382	2.5\\
383	2.5\\
384	2.5\\
385	2.5\\
386	2.5\\
387	2.5\\
388	2.5\\
389	2.5\\
390	2.5\\
391	2.5\\
392	2.5\\
393	2.5\\
394	2.5\\
395	2.5\\
396	2.5\\
397	2.5\\
398	2.5\\
399	2.5\\
400	2.5\\
401	2.5\\
402	2.5\\
403	3.8\\
404	3.8\\
405	3.8\\
406	3.8\\
407	3.8\\
408	3.8\\
409	3.8\\
410	3.8\\
411	3.8\\
412	2.3\\
413	2.3\\
414	2.3\\
415	2.3\\
416	2.3\\
417	2.3\\
418	2.2\\
419	2.2\\
420	2.2\\
421	2.2\\
422	2.2\\
423	2.2\\
424	2.2\\
425	2.2\\
426	2.2\\
427	2.2\\
428	2.2\\
429	2.2\\
430	2.2\\
431	2.2\\
432	2.2\\
433	2.2\\
434	2.2\\
435	2.2\\
436	2.2\\
437	2.2\\
438	2.2\\
439	2.2\\
440	2.2\\
441	2.2\\
442	2.2\\
443	2.2\\
444	2.2\\
445	2.2\\
446	2.2\\
447	2.2\\
448	2.2\\
449	2.2\\
450	2.2\\
451	2.2\\
452	2.2\\
453	2.2\\
454	2.2\\
455	2.2\\
456	2.2\\
457	2.2\\
458	2.2\\
459	2.2\\
460	2.2\\
461	2.2\\
462	2.2\\
463	2.2\\
464	2.2\\
465	2.2\\
466	2.2\\
467	2.2\\
468	2.2\\
469	2.2\\
470	2.2\\
471	2.2\\
472	2.2\\
473	2.2\\
474	2.2\\
475	2.2\\
476	2.2\\
477	2.2\\
478	2.2\\
479	2.2\\
480	2.2\\
481	2.2\\
482	2.2\\
483	2.2\\
484	2.2\\
485	2.2\\
486	2.2\\
487	2.2\\
488	2.2\\
489	2.2\\
490	2.2\\
491	2.2\\
492	2.2\\
493	2.2\\
494	2.2\\
495	2.2\\
496	2.2\\
};
\addplot [color=mycolor2, draw=none, mark=x, mark options={solid, mycolor2}, forget plot]
  table[row sep=crcr]{%
1	1.3\\
2	4.5\\
3	4.5\\
4	4.5\\
5	4.5\\
6	2.3\\
7	1.3\\
8	1.4\\
9	1.4\\
10	1.4\\
11	1.4\\
12	1.4\\
13	1.4\\
14	1.4\\
15	3.9\\
16	3.9\\
17	3.9\\
18	2\\
19	2\\
20	2\\
21	2\\
22	2\\
23	2\\
24	2\\
25	2\\
26	2\\
27	2\\
28	2\\
29	3.9\\
30	3.9\\
31	3.9\\
32	3.9\\
33	3.9\\
34	3.9\\
35	2\\
36	2\\
37	2\\
38	2\\
39	2\\
40	2\\
41	2\\
42	2\\
43	2\\
44	2\\
45	2\\
46	2\\
47	2\\
48	2\\
49	2\\
50	3\\
51	3\\
52	3\\
53	3\\
54	3\\
55	2\\
56	2.1\\
57	2.1\\
58	2.1\\
59	2.1\\
60	2.1\\
61	2.1\\
62	2.1\\
63	2.1\\
64	2.1\\
65	2.1\\
66	2.1\\
67	2.1\\
68	2.1\\
69	2.1\\
70	2.1\\
71	2.1\\
72	2.1\\
73	2.1\\
74	2.1\\
75	2.1\\
76	2.1\\
77	3\\
78	3\\
79	3\\
80	3\\
81	3\\
82	3\\
83	3\\
84	3\\
85	3.8\\
86	3.8\\
87	3.8\\
88	3.8\\
89	3.8\\
90	3.8\\
91	3.8\\
92	3.8\\
93	3.8\\
94	3.8\\
95	3.8\\
96	3.8\\
97	3.8\\
98	3\\
99	3.8\\
100	3.8\\
101	3.8\\
102	3.8\\
103	3.8\\
104	3.8\\
105	3.8\\
106	3.8\\
107	3.8\\
108	3.8\\
109	3\\
110	3\\
111	3\\
112	3\\
113	3\\
114	3\\
115	3\\
116	3\\
117	3\\
118	3\\
119	3\\
120	3\\
121	3\\
122	3.8\\
123	3.8\\
124	3.8\\
125	3\\
126	3\\
127	3.8\\
128	3.8\\
129	3.8\\
130	3.8\\
131	3.8\\
132	3.8\\
133	3.8\\
134	3.8\\
135	3.8\\
136	3\\
137	2.2\\
138	3.8\\
139	3.8\\
140	3.8\\
141	3.8\\
142	3.8\\
143	3.8\\
144	3.8\\
145	3.8\\
146	1.2\\
147	1.2\\
148	1.2\\
149	3.8\\
150	2.2\\
151	2.2\\
152	2.2\\
153	2.2\\
154	2.2\\
155	2.2\\
156	2.2\\
157	2.2\\
158	3\\
159	3\\
160	3\\
161	3.8\\
162	3.8\\
163	3.8\\
164	3.8\\
165	3.8\\
166	3.8\\
167	2.4\\
168	2.4\\
169	3.8\\
170	3.8\\
171	2.4\\
172	2.5\\
173	2.5\\
174	2.5\\
175	2.5\\
176	2.5\\
177	2.5\\
178	2.5\\
179	2.5\\
180	2.5\\
181	2.5\\
182	2.5\\
183	2.5\\
184	2.5\\
185	2.5\\
186	2.5\\
187	2.5\\
188	2.5\\
189	2.5\\
190	2.5\\
191	2.5\\
192	2.5\\
193	2.5\\
194	2.5\\
195	2.5\\
196	2.5\\
197	3\\
198	3\\
199	3.1\\
200	3.1\\
201	3.1\\
202	3.1\\
203	3.1\\
204	3.8\\
205	3.8\\
206	3.8\\
207	2.5\\
208	2.5\\
209	2.6\\
210	2.6\\
211	2.6\\
212	2.6\\
213	2.6\\
214	3.4\\
215	3.4\\
216	3.4\\
217	3.4\\
218	3.4\\
219	3.4\\
220	3.4\\
221	3.4\\
222	3.3\\
223	2.6\\
224	2.6\\
225	2.6\\
226	2.6\\
227	2.6\\
228	2.6\\
229	2.6\\
230	2.6\\
231	2.6\\
232	2.6\\
233	2.6\\
234	2.6\\
235	2.6\\
236	3.3\\
237	3.3\\
238	3.3\\
239	3.3\\
240	3.3\\
241	3.3\\
242	3.3\\
243	3.3\\
244	3.3\\
245	3.3\\
246	3.3\\
247	3.3\\
248	3.3\\
249	3.2\\
250	3.2\\
251	3.2\\
252	3.2\\
253	3.1\\
254	3.1\\
255	3.1\\
256	3.1\\
257	3.1\\
258	3.3\\
259	3.3\\
260	3.3\\
261	3.3\\
262	2.3\\
263	2.3\\
264	2.3\\
265	2.3\\
266	2.3\\
267	2.3\\
268	2.3\\
269	2.3\\
270	2.3\\
271	2.9\\
272	2.9\\
273	2.9\\
274	2.9\\
275	2.9\\
276	2.9\\
277	2.9\\
278	2.9\\
279	2.9\\
280	2.9\\
281	2.9\\
282	2.9\\
283	2.9\\
284	1.2\\
285	1.2\\
286	1.2\\
287	1.2\\
288	1.2\\
289	1.2\\
290	1.2\\
291	1.2\\
292	1.2\\
293	1.2\\
294	1.2\\
295	1.2\\
296	1.2\\
297	1.2\\
298	1.2\\
299	2.3\\
300	2.9\\
301	2.3\\
302	1.2\\
303	2.9\\
304	2.9\\
305	2.9\\
306	1.2\\
307	1.2\\
308	1.2\\
309	1.2\\
310	1.2\\
311	1.2\\
312	1.2\\
313	1.2\\
314	1.2\\
315	1.2\\
316	1.2\\
317	1.2\\
318	1.2\\
319	1.2\\
320	1.2\\
321	1.2\\
322	1.2\\
323	1.2\\
324	1.2\\
325	1.2\\
326	1.2\\
327	1.2\\
328	1.2\\
329	1.2\\
330	1.2\\
331	1.2\\
332	1.2\\
333	1.2\\
334	1.2\\
335	1.2\\
336	1.2\\
337	1.2\\
338	1.2\\
339	1.2\\
340	1.2\\
341	1.2\\
342	1.2\\
343	1.2\\
344	1.2\\
345	1.2\\
346	1.2\\
347	1.2\\
348	1.2\\
349	1.2\\
350	1.2\\
351	1.2\\
352	1.2\\
353	3.3\\
354	3.3\\
355	3.3\\
356	3.3\\
357	3.3\\
358	3.3\\
359	3.3\\
360	3.3\\
361	3.3\\
362	3.3\\
363	3.3\\
364	3.3\\
365	3.3\\
366	3.3\\
367	3.3\\
368	3.3\\
369	3.3\\
370	3.3\\
371	1.2\\
372	1.2\\
373	1.2\\
374	1.2\\
375	3.8\\
376	3.8\\
377	3.8\\
378	3.8\\
379	3.8\\
380	3.7\\
381	3.8\\
382	3.8\\
383	3.8\\
384	3.8\\
385	3.8\\
386	3.8\\
387	3.8\\
388	3.8\\
389	3.8\\
390	3.8\\
391	3.8\\
392	3.8\\
393	3.8\\
394	3.8\\
395	3.8\\
396	3.8\\
397	3.8\\
398	3.8\\
399	3.8\\
400	3.8\\
401	3.8\\
402	3.8\\
403	2.5\\
404	2.5\\
405	2.5\\
406	2.5\\
407	2.5\\
408	2.5\\
409	2.5\\
410	2.5\\
411	2.5\\
412	3.8\\
413	3.8\\
414	3.8\\
415	3.8\\
416	3.8\\
417	3.8\\
418	3.8\\
419	3.8\\
420	3.8\\
421	3.8\\
422	3.8\\
423	3.8\\
424	3.8\\
425	1.2\\
426	1.2\\
427	1.2\\
428	1.2\\
429	1.2\\
430	1.2\\
431	1.2\\
432	1.2\\
433	2.6\\
434	2.6\\
435	2.8\\
436	2.6\\
437	3.8\\
438	3.8\\
439	3.8\\
440	3.8\\
441	3.8\\
442	3.8\\
443	3.8\\
444	3.8\\
445	3.8\\
446	3.8\\
447	3.8\\
448	3.8\\
449	3.8\\
450	3.8\\
451	3.8\\
452	3.8\\
453	3.8\\
454	3.8\\
455	3.8\\
456	3.8\\
457	3.8\\
458	3.8\\
459	3.8\\
460	3.8\\
461	3.8\\
462	3.8\\
463	3.9\\
464	3.9\\
465	3.9\\
466	3.9\\
467	3.9\\
468	3.9\\
469	3.9\\
470	3.9\\
471	3.9\\
472	3.9\\
473	3.9\\
474	3.9\\
475	3.9\\
476	3.9\\
477	3.9\\
478	3.9\\
479	3.9\\
480	3.9\\
481	3.9\\
482	3.9\\
483	3.9\\
484	3.9\\
485	3.9\\
486	3.9\\
487	3.9\\
488	3.9\\
489	3.9\\
490	3.9\\
491	3.9\\
492	3.9\\
493	3.9\\
494	3.9\\
495	3.9\\
496	3.9\\
};
\addplot [color=mycolor3, dashed, forget plot]
  table[row sep=crcr]{%
1	2\\
2	2.00002013992709\\
3	2.00008055889714\\
4	2.00018125447647\\
5	2.00032222260909\\
6	2.00050345761681\\
7	2.00072495219953\\
8	2.00098669743546\\
9	2.00128868278154\\
10	2.00163089607386\\
11	2.00201332352812\\
12	2.00243594974018\\
13	2.00289875768672\\
14	2.00340172872592\\
15	2.00394484259817\\
16	2.00452807742692\\
17	2.00515140971955\\
18	2.00581481436835\\
19	2.00651826465145\\
20	2.00726173223399\\
21	2.0080451871692\\
22	2.00886859789964\\
23	2.00973193125843\\
24	2.01063515247064\\
25	2.01157822515464\\
26	2.01256111132361\\
27	2.01358377138703\\
28	2.01464616415231\\
29	2.01574824682642\\
30	2.01688997501763\\
31	2.01807130273729\\
32	2.01929218240171\\
33	2.02055256483401\\
34	2.02185239926619\\
35	2.0231916333411\\
36	2.02457021311459\\
37	2.02598808305767\\
38	2.02744518605873\\
39	2.02894146342589\\
40	2.03047685488931\\
41	2.03205129860364\\
42	2.03366473115053\\
43	2.03531708754114\\
44	2.03700830121881\\
45	2.03873830406168\\
46	2.0405070263855\\
47	2.04231439694639\\
48	2.04416034294373\\
49	2.04604479002309\\
50	2.0479676622792\\
51	2.04992888225905\\
52	2.051928370965\\
53	2.05396604785792\\
54	2.05604183086049\\
55	2.05815563636048\\
56	2.06030737921409\\
57	2.06249697274945\\
58	2.06472432877005\\
59	2.06698935755831\\
60	2.06929196787921\\
61	2.07163206698393\\
62	2.07400956061362\\
63	2.07642435300319\\
64	2.07887634688515\\
65	2.08136544349355\\
66	2.08389154256793\\
67	2.0864545423574\\
68	2.0890543396247\\
69	2.09169082965036\\
70	2.09436390623697\\
71	2.09707346171338\\
72	2.09981938693909\\
73	2.10260157130864\\
74	2.10541990275605\\
75	2.10827426775933\\
76	2.11116455134508\\
77	2.11409063709309\\
78	2.11705240714107\\
79	2.12004974218935\\
80	2.1230825215057\\
81	2.12615062293021\\
82	2.12925392288022\\
83	2.13239229635525\\
84	2.13556561694207\\
85	2.1387737568198\\
86	2.14201658676502\\
87	2.14529397615703\\
88	2.14860579298304\\
89	2.15195190384357\\
90	2.15533217395776\\
91	2.15874646716882\\
92	2.16219464594951\\
93	2.1656765714077\\
94	2.16919210329195\\
95	2.17274109999712\\
96	2.17632341857017\\
97	2.17993891471581\\
98	2.18358744280239\\
99	2.18726885586773\\
100	2.19098300562505\\
101	2.19472974246894\\
102	2.19850891548138\\
103	2.20232037243784\\
104	2.20616395981339\\
105	2.21003952278888\\
106	2.21394690525721\\
107	2.21788594982958\\
108	2.22185649784185\\
109	2.22585838936092\\
110	2.22989146319118\\
111	2.23395555688102\\
112	2.23805050672933\\
113	2.24217614779212\\
114	2.24633231388918\\
115	2.25051883761075\\
116	2.25473555032425\\
117	2.25898228218111\\
118	2.2632588621236\\
119	2.26756511789169\\
120	2.27190087603004\\
121	2.27626596189493\\
122	2.28066019966135\\
123	2.28508341233004\\
124	2.28953542173463\\
125	2.29401604854884\\
126	2.29852511229368\\
127	2.30306243134471\\
128	2.30762782293938\\
129	2.31222110318438\\
130	2.31684208706307\\
131	2.32149058844287\\
132	2.32616642008283\\
133	2.33086939364114\\
134	2.33559931968271\\
135	2.3403560076868\\
136	2.34513926605471\\
137	2.34994890211751\\
138	2.35478472214374\\
139	2.35964653134728\\
140	2.36453413389516\\
141	2.36944733291548\\
142	2.37438593050528\\
143	2.37934972773858\\
144	2.38433852467434\\
145	2.38935212036456\\
146	2.39439031286233\\
147	2.39945289923\\
148	2.40453967554731\\
149	2.40965043691967\\
150	2.41478497748635\\
151	2.4199430904288\\
152	2.42512456797899\\
153	2.43032920142776\\
154	2.43555678113323\\
155	2.44080709652925\\
156	2.44607993613389\\
157	2.45137508755793\\
158	2.45669233751345\\
159	2.46203147182239\\
160	2.4673922754252\\
161	2.4727745323895\\
162	2.47817802591875\\
163	2.48360253836104\\
164	2.48904785121778\\
165	2.49451374515257\\
166	2.5\\
167	2.50550639477452\\
168	2.51103270767936\\
169	2.51657871611544\\
170	2.52214419669034\\
171	2.52772892522732\\
172	2.53333267677433\\
173	2.53895522561307\\
174	2.5445963452681\\
175	2.55025580851593\\
176	2.55593338739423\\
177	2.56162885321092\\
178	2.56734197655349\\
179	2.57307252729816\\
180	2.57882027461917\\
181	2.58458498699811\\
182	2.5903664322332\\
183	2.59616437744868\\
184	2.60197858910414\\
185	2.60780883300399\\
186	2.61365487430687\\
187	2.61951647753508\\
188	2.62539340658409\\
189	2.63128542473206\\
190	2.63719229464936\\
191	2.64311377840813\\
192	2.64904963749186\\
193	2.65499963280503\\
194	2.66096352468268\\
195	2.66694107290012\\
196	2.67293203668258\\
197	2.67893617471491\\
198	2.68495324515131\\
199	2.69098300562505\\
200	2.69702521325827\\
201	2.70307962467173\\
202	2.7091459959946\\
203	2.71522408287435\\
204	2.72131364048651\\
205	2.7274144235446\\
206	2.73352618630997\\
207	2.73964868260169\\
208	2.74578166580651\\
209	2.75192488888877\\
210	2.75807810440033\\
211	2.76424106449057\\
212	2.77041352091636\\
213	2.77659522505205\\
214	2.78278592789949\\
215	2.78898538009809\\
216	2.79519333193481\\
217	2.80140953335425\\
218	2.80763373396874\\
219	2.81386568306837\\
220	2.82010512963115\\
221	2.82635182233307\\
222	2.83260550955827\\
223	2.83886593940914\\
224	2.84513285971647\\
225	2.85140601804963\\
226	2.85768516172671\\
227	2.86397003782474\\
228	2.8702603931898\\
229	2.87655597444731\\
230	2.88285652801216\\
231	2.88916180009899\\
232	2.89547153673235\\
233	2.90178548375696\\
234	2.90810338684797\\
235	2.91442499152116\\
236	2.92075004314321\\
237	2.92707828694197\\
238	2.9334094680167\\
239	2.93974333134834\\
240	2.94607962180981\\
241	2.95241808417626\\
242	2.95875846313533\\
243	2.9651005032975\\
244	2.9714439492063\\
245	2.97778854534867\\
246	2.98413403616519\\
247	2.99048016606042\\
248	2.99682667941318\\
249	3.00317332058682\\
250	3.00951983393958\\
251	3.01586596383481\\
252	3.02221145465133\\
253	3.0285560507937\\
254	3.0348994967025\\
255	3.04124153686467\\
256	3.04758191582374\\
257	3.05392037819019\\
258	3.06025666865166\\
259	3.0665905319833\\
260	3.07292171305803\\
261	3.07924995685679\\
262	3.08557500847884\\
263	3.09189661315203\\
264	3.09821451624304\\
265	3.10452846326765\\
266	3.11083819990101\\
267	3.11714347198784\\
268	3.12344402555269\\
269	3.1297396068102\\
270	3.13602996217526\\
271	3.14231483827329\\
272	3.14859398195037\\
273	3.15486714028353\\
274	3.16113406059086\\
275	3.16739449044173\\
276	3.17364817766693\\
277	3.17989487036885\\
278	3.18613431693163\\
279	3.19236626603126\\
280	3.19859046664575\\
281	3.20480666806519\\
282	3.21101461990191\\
283	3.21721407210051\\
284	3.22340477494795\\
285	3.22958647908364\\
286	3.23575893550943\\
287	3.24192189559967\\
288	3.24807511111123\\
289	3.25421833419349\\
290	3.26035131739831\\
291	3.26647381369003\\
292	3.2725855764554\\
293	3.27868635951349\\
294	3.28477591712565\\
295	3.2908540040054\\
296	3.29692037532827\\
297	3.30297478674173\\
298	3.30901699437495\\
299	3.31504675484869\\
300	3.32106382528509\\
301	3.32706796331742\\
302	3.33305892709988\\
303	3.33903647531732\\
304	3.34500036719497\\
305	3.35095036250814\\
306	3.35688622159187\\
307	3.36280770535064\\
308	3.36871457526794\\
309	3.37460659341591\\
310	3.38048352246492\\
311	3.38634512569313\\
312	3.39219116699601\\
313	3.39802141089586\\
314	3.40383562255132\\
315	3.4096335677668\\
316	3.41541501300189\\
317	3.42117972538083\\
318	3.42692747270184\\
319	3.43265802344651\\
320	3.43837114678908\\
321	3.44406661260577\\
322	3.44974419148407\\
323	3.4554036547319\\
324	3.46104477438693\\
325	3.46666732322567\\
326	3.47227107477268\\
327	3.47785580330966\\
328	3.48342128388456\\
329	3.48896729232064\\
330	3.49449360522548\\
331	3.5\\
332	3.50548625484743\\
333	3.51095214878222\\
334	3.51639746163896\\
335	3.52182197408125\\
336	3.5272254676105\\
337	3.5326077245748\\
338	3.53796852817761\\
339	3.54330766248655\\
340	3.54862491244207\\
341	3.55392006386611\\
342	3.55919290347075\\
343	3.56444321886677\\
344	3.56967079857224\\
345	3.57487543202101\\
346	3.5800569095712\\
347	3.58521502251365\\
348	3.59034956308033\\
349	3.59546032445269\\
350	3.60054710077\\
351	3.60560968713767\\
352	3.61064787963544\\
353	3.61566147532566\\
354	3.62065027226142\\
355	3.62561406949472\\
356	3.63055266708452\\
357	3.63546586610484\\
358	3.64035346865272\\
359	3.64521527785626\\
360	3.65005109788249\\
361	3.65486073394529\\
362	3.6596439923132\\
363	3.66440068031729\\
364	3.66913060635886\\
365	3.67383357991717\\
366	3.67850941155713\\
367	3.68315791293693\\
368	3.68777889681562\\
369	3.69237217706062\\
370	3.69693756865529\\
371	3.70147488770632\\
372	3.70598395145116\\
373	3.71046457826537\\
374	3.71491658766996\\
375	3.71933980033865\\
376	3.72373403810507\\
377	3.72809912396996\\
378	3.73243488210831\\
379	3.7367411378764\\
380	3.74101771781889\\
381	3.74526444967575\\
382	3.74948116238925\\
383	3.75366768611082\\
384	3.75782385220788\\
385	3.76194949327067\\
386	3.76604444311898\\
387	3.77010853680882\\
388	3.77414161063908\\
389	3.77814350215815\\
390	3.78211405017042\\
391	3.78605309474279\\
392	3.78996047721112\\
393	3.79383604018661\\
394	3.79767962756216\\
395	3.80149108451862\\
396	3.80527025753106\\
397	3.80901699437495\\
398	3.81273114413227\\
399	3.81641255719761\\
400	3.82006108528419\\
401	3.82367658142983\\
402	3.82725890000288\\
403	3.83080789670805\\
404	3.83432342859229\\
405	3.83780535405049\\
406	3.84125353283118\\
407	3.84466782604224\\
408	3.84804809615643\\
409	3.85139420701696\\
410	3.85470602384297\\
411	3.85798341323498\\
412	3.8612262431802\\
413	3.86443438305793\\
414	3.86760770364475\\
415	3.87074607711978\\
416	3.87384937706979\\
417	3.8769174784943\\
418	3.87995025781065\\
419	3.88294759285893\\
420	3.88590936290691\\
421	3.88883544865492\\
422	3.89172573224067\\
423	3.89458009724395\\
424	3.89739842869136\\
425	3.90018061306091\\
426	3.90292653828662\\
427	3.90563609376303\\
428	3.90830917034964\\
429	3.9109456603753\\
430	3.9135454576426\\
431	3.91610845743207\\
432	3.91863455650645\\
433	3.92112365311485\\
434	3.92357564699681\\
435	3.92599043938638\\
436	3.92836793301607\\
437	3.93070803212079\\
438	3.93301064244169\\
439	3.93527567122995\\
440	3.93750302725055\\
441	3.93969262078591\\
442	3.94184436363952\\
443	3.94395816913951\\
444	3.94603395214208\\
445	3.948071629035\\
446	3.95007111774095\\
447	3.9520323377208\\
448	3.95395520997691\\
449	3.95583965705627\\
450	3.95768560305361\\
451	3.9594929736145\\
452	3.96126169593832\\
453	3.96299169878119\\
454	3.96468291245886\\
455	3.96633526884947\\
456	3.96794870139636\\
457	3.96952314511069\\
458	3.97105853657411\\
459	3.97255481394127\\
460	3.97401191694233\\
461	3.97542978688541\\
462	3.9768083666589\\
463	3.97814760073381\\
464	3.97944743516599\\
465	3.98070781759829\\
466	3.98192869726271\\
467	3.98311002498237\\
468	3.98425175317358\\
469	3.98535383584769\\
470	3.98641622861297\\
471	3.98743888867639\\
472	3.98842177484536\\
473	3.98936484752936\\
474	3.99026806874157\\
475	3.99113140210036\\
476	3.9919548128308\\
477	3.99273826776601\\
478	3.99348173534855\\
479	3.99418518563165\\
480	3.99484859028045\\
481	3.99547192257308\\
482	3.99605515740183\\
483	3.99659827127408\\
484	3.99710124231328\\
485	3.99756405025982\\
486	3.99798667647188\\
487	3.99836910392614\\
488	3.99871131721846\\
489	3.99901330256454\\
490	3.99927504780047\\
491	3.99949654238319\\
492	3.99967777739091\\
493	3.99981874552353\\
494	3.99991944110286\\
495	3.99997986007291\\
496	4\\
};
\addplot [color=mycolor4, dashed, forget plot]
  table[row sep=crcr]{%
1	4\\
2	3.99997986007291\\
3	3.99991944110286\\
4	3.99981874552353\\
5	3.99967777739091\\
6	3.99949654238319\\
7	3.99927504780047\\
8	3.99901330256454\\
9	3.99871131721846\\
10	3.99836910392614\\
11	3.99798667647188\\
12	3.99756405025982\\
13	3.99710124231328\\
14	3.99659827127408\\
15	3.99605515740183\\
16	3.99547192257308\\
17	3.99484859028045\\
18	3.99418518563165\\
19	3.99348173534855\\
20	3.99273826776601\\
21	3.9919548128308\\
22	3.99113140210036\\
23	3.99026806874157\\
24	3.98936484752936\\
25	3.98842177484536\\
26	3.98743888867639\\
27	3.98641622861297\\
28	3.98535383584769\\
29	3.98425175317358\\
30	3.98311002498237\\
31	3.98192869726271\\
32	3.98070781759829\\
33	3.97944743516599\\
34	3.97814760073381\\
35	3.9768083666589\\
36	3.97542978688541\\
37	3.97401191694233\\
38	3.97255481394127\\
39	3.97105853657411\\
40	3.96952314511069\\
41	3.96794870139636\\
42	3.96633526884947\\
43	3.96468291245886\\
44	3.96299169878119\\
45	3.96126169593832\\
46	3.9594929736145\\
47	3.9576856030536\\
48	3.95583965705627\\
49	3.95395520997691\\
50	3.9520323377208\\
51	3.95007111774095\\
52	3.948071629035\\
53	3.94603395214208\\
54	3.94395816913951\\
55	3.94184436363952\\
56	3.93969262078591\\
57	3.93750302725055\\
58	3.93527567122995\\
59	3.93301064244169\\
60	3.93070803212079\\
61	3.92836793301607\\
62	3.92599043938638\\
63	3.92357564699681\\
64	3.92112365311485\\
65	3.91863455650645\\
66	3.91610845743207\\
67	3.9135454576426\\
68	3.9109456603753\\
69	3.90830917034964\\
70	3.90563609376303\\
71	3.90292653828662\\
72	3.90018061306091\\
73	3.89739842869136\\
74	3.89458009724395\\
75	3.89172573224067\\
76	3.88883544865492\\
77	3.88590936290691\\
78	3.88294759285893\\
79	3.87995025781065\\
80	3.8769174784943\\
81	3.87384937706979\\
82	3.87074607711978\\
83	3.86760770364475\\
84	3.86443438305793\\
85	3.8612262431802\\
86	3.85798341323498\\
87	3.85470602384297\\
88	3.85139420701696\\
89	3.84804809615643\\
90	3.84466782604224\\
91	3.84125353283118\\
92	3.83780535405049\\
93	3.83432342859229\\
94	3.83080789670805\\
95	3.82725890000288\\
96	3.82367658142983\\
97	3.82006108528419\\
98	3.81641255719761\\
99	3.81273114413227\\
100	3.80901699437495\\
101	3.80527025753106\\
102	3.80149108451862\\
103	3.79767962756216\\
104	3.79383604018661\\
105	3.78996047721112\\
106	3.78605309474279\\
107	3.78211405017042\\
108	3.77814350215815\\
109	3.77414161063908\\
110	3.77010853680882\\
111	3.76604444311898\\
112	3.76194949327067\\
113	3.75782385220788\\
114	3.75366768611082\\
115	3.74948116238925\\
116	3.74526444967575\\
117	3.74101771781889\\
118	3.7367411378764\\
119	3.73243488210831\\
120	3.72809912396996\\
121	3.72373403810507\\
122	3.71933980033865\\
123	3.71491658766996\\
124	3.71046457826537\\
125	3.70598395145116\\
126	3.70147488770632\\
127	3.69693756865529\\
128	3.69237217706062\\
129	3.68777889681562\\
130	3.68315791293693\\
131	3.67850941155713\\
132	3.67383357991717\\
133	3.66913060635886\\
134	3.66440068031729\\
135	3.6596439923132\\
136	3.65486073394529\\
137	3.65005109788249\\
138	3.64521527785626\\
139	3.64035346865272\\
140	3.63546586610484\\
141	3.63055266708452\\
142	3.62561406949472\\
143	3.62065027226142\\
144	3.61566147532566\\
145	3.61064787963544\\
146	3.60560968713767\\
147	3.60054710077\\
148	3.59546032445269\\
149	3.59034956308033\\
150	3.58521502251365\\
151	3.5800569095712\\
152	3.57487543202101\\
153	3.56967079857224\\
154	3.56444321886677\\
155	3.55919290347075\\
156	3.55392006386611\\
157	3.54862491244207\\
158	3.54330766248655\\
159	3.53796852817761\\
160	3.5326077245748\\
161	3.5272254676105\\
162	3.52182197408125\\
163	3.51639746163896\\
164	3.51095214878222\\
165	3.50548625484743\\
166	3.5\\
167	3.49449360522548\\
168	3.48896729232064\\
169	3.48342128388456\\
170	3.47785580330966\\
171	3.47227107477268\\
172	3.46666732322567\\
173	3.46104477438693\\
174	3.4554036547319\\
175	3.44974419148407\\
176	3.44406661260577\\
177	3.43837114678908\\
178	3.43265802344651\\
179	3.42692747270184\\
180	3.42117972538083\\
181	3.41541501300189\\
182	3.4096335677668\\
183	3.40383562255132\\
184	3.39802141089586\\
185	3.39219116699601\\
186	3.38634512569313\\
187	3.38048352246492\\
188	3.37460659341591\\
189	3.36871457526794\\
190	3.36280770535064\\
191	3.35688622159187\\
192	3.35095036250814\\
193	3.34500036719497\\
194	3.33903647531732\\
195	3.33305892709988\\
196	3.32706796331742\\
197	3.32106382528509\\
198	3.31504675484869\\
199	3.30901699437495\\
200	3.30297478674173\\
201	3.29692037532828\\
202	3.2908540040054\\
203	3.28477591712565\\
204	3.27868635951349\\
205	3.2725855764554\\
206	3.26647381369004\\
207	3.26035131739831\\
208	3.25421833419349\\
209	3.24807511111123\\
210	3.24192189559967\\
211	3.23575893550943\\
212	3.22958647908364\\
213	3.22340477494795\\
214	3.21721407210051\\
215	3.21101461990191\\
216	3.20480666806519\\
217	3.19859046664575\\
218	3.19236626603126\\
219	3.18613431693163\\
220	3.17989487036885\\
221	3.17364817766693\\
222	3.16739449044173\\
223	3.16113406059086\\
224	3.15486714028353\\
225	3.14859398195037\\
226	3.14231483827328\\
227	3.13602996217526\\
228	3.1297396068102\\
229	3.12344402555269\\
230	3.11714347198784\\
231	3.11083819990101\\
232	3.10452846326765\\
233	3.09821451624304\\
234	3.09189661315203\\
235	3.08557500847884\\
236	3.07924995685679\\
237	3.07292171305803\\
238	3.0665905319833\\
239	3.06025666865166\\
240	3.05392037819019\\
241	3.04758191582374\\
242	3.04124153686467\\
243	3.0348994967025\\
244	3.0285560507937\\
245	3.02221145465133\\
246	3.01586596383481\\
247	3.00951983393958\\
248	3.00317332058682\\
249	2.99682667941318\\
250	2.99048016606042\\
251	2.98413403616519\\
252	2.97778854534867\\
253	2.9714439492063\\
254	2.9651005032975\\
255	2.95875846313533\\
256	2.95241808417626\\
257	2.94607962180981\\
258	2.93974333134834\\
259	2.9334094680167\\
260	2.92707828694197\\
261	2.92075004314321\\
262	2.91442499152116\\
263	2.90810338684797\\
264	2.90178548375696\\
265	2.89547153673235\\
266	2.88916180009899\\
267	2.88285652801216\\
268	2.87655597444731\\
269	2.8702603931898\\
270	2.86397003782474\\
271	2.85768516172671\\
272	2.85140601804963\\
273	2.84513285971647\\
274	2.83886593940914\\
275	2.83260550955827\\
276	2.82635182233307\\
277	2.82010512963115\\
278	2.81386568306837\\
279	2.80763373396874\\
280	2.80140953335425\\
281	2.79519333193481\\
282	2.78898538009809\\
283	2.78278592789949\\
284	2.77659522505205\\
285	2.77041352091636\\
286	2.76424106449057\\
287	2.75807810440033\\
288	2.75192488888877\\
289	2.74578166580651\\
290	2.73964868260169\\
291	2.73352618630997\\
292	2.7274144235446\\
293	2.72131364048651\\
294	2.71522408287435\\
295	2.7091459959946\\
296	2.70307962467172\\
297	2.69702521325827\\
298	2.69098300562505\\
299	2.68495324515131\\
300	2.67893617471491\\
301	2.67293203668258\\
302	2.66694107290012\\
303	2.66096352468268\\
304	2.65499963280503\\
305	2.64904963749186\\
306	2.64311377840813\\
307	2.63719229464936\\
308	2.63128542473206\\
309	2.62539340658409\\
310	2.61951647753508\\
311	2.61365487430687\\
312	2.60780883300399\\
313	2.60197858910414\\
314	2.59616437744867\\
315	2.5903664322332\\
316	2.58458498699811\\
317	2.57882027461917\\
318	2.57307252729816\\
319	2.56734197655349\\
320	2.56162885321092\\
321	2.55593338739423\\
322	2.55025580851593\\
323	2.5445963452681\\
324	2.53895522561307\\
325	2.53333267677433\\
326	2.52772892522732\\
327	2.52214419669034\\
328	2.51657871611544\\
329	2.51103270767936\\
330	2.50550639477452\\
331	2.5\\
332	2.49451374515257\\
333	2.48904785121778\\
334	2.48360253836104\\
335	2.47817802591875\\
336	2.4727745323895\\
337	2.4673922754252\\
338	2.46203147182239\\
339	2.45669233751345\\
340	2.45137508755793\\
341	2.44607993613389\\
342	2.44080709652925\\
343	2.43555678113323\\
344	2.43032920142776\\
345	2.42512456797899\\
346	2.4199430904288\\
347	2.41478497748635\\
348	2.40965043691967\\
349	2.40453967554731\\
350	2.39945289923\\
351	2.39439031286233\\
352	2.38935212036456\\
353	2.38433852467434\\
354	2.37934972773858\\
355	2.37438593050528\\
356	2.36944733291548\\
357	2.36453413389516\\
358	2.35964653134728\\
359	2.35478472214374\\
360	2.34994890211751\\
361	2.34513926605471\\
362	2.3403560076868\\
363	2.33559931968271\\
364	2.33086939364114\\
365	2.32616642008283\\
366	2.32149058844287\\
367	2.31684208706307\\
368	2.31222110318438\\
369	2.30762782293938\\
370	2.30306243134471\\
371	2.29852511229368\\
372	2.29401604854884\\
373	2.28953542173463\\
374	2.28508341233004\\
375	2.28066019966135\\
376	2.27626596189493\\
377	2.27190087603004\\
378	2.26756511789169\\
379	2.26325886212359\\
380	2.25898228218111\\
381	2.25473555032424\\
382	2.25051883761075\\
383	2.24633231388918\\
384	2.24217614779212\\
385	2.23805050672933\\
386	2.23395555688102\\
387	2.22989146319118\\
388	2.22585838936092\\
389	2.22185649784185\\
390	2.21788594982958\\
391	2.21394690525721\\
392	2.21003952278888\\
393	2.20616395981339\\
394	2.20232037243784\\
395	2.19850891548138\\
396	2.19472974246894\\
397	2.19098300562505\\
398	2.18726885586773\\
399	2.18358744280239\\
400	2.17993891471581\\
401	2.17632341857017\\
402	2.17274109999712\\
403	2.16919210329195\\
404	2.16567657140771\\
405	2.16219464594951\\
406	2.15874646716882\\
407	2.15533217395776\\
408	2.15195190384357\\
409	2.14860579298304\\
410	2.14529397615703\\
411	2.14201658676502\\
412	2.1387737568198\\
413	2.13556561694207\\
414	2.13239229635525\\
415	2.12925392288022\\
416	2.12615062293021\\
417	2.1230825215057\\
418	2.12004974218935\\
419	2.11705240714107\\
420	2.11409063709309\\
421	2.11116455134508\\
422	2.10827426775933\\
423	2.10541990275605\\
424	2.10260157130864\\
425	2.09981938693909\\
426	2.09707346171338\\
427	2.09436390623697\\
428	2.09169082965036\\
429	2.0890543396247\\
430	2.0864545423574\\
431	2.08389154256793\\
432	2.08136544349355\\
433	2.07887634688515\\
434	2.07642435300319\\
435	2.07400956061362\\
436	2.07163206698393\\
437	2.06929196787921\\
438	2.06698935755831\\
439	2.06472432877005\\
440	2.06249697274945\\
441	2.06030737921409\\
442	2.05815563636048\\
443	2.05604183086049\\
444	2.05396604785792\\
445	2.051928370965\\
446	2.04992888225905\\
447	2.0479676622792\\
448	2.04604479002309\\
449	2.04416034294373\\
450	2.04231439694639\\
451	2.0405070263855\\
452	2.03873830406168\\
453	2.03700830121881\\
454	2.03531708754114\\
455	2.03366473115053\\
456	2.03205129860364\\
457	2.03047685488931\\
458	2.02894146342589\\
459	2.02744518605873\\
460	2.02598808305767\\
461	2.02457021311459\\
462	2.0231916333411\\
463	2.02185239926619\\
464	2.02055256483401\\
465	2.01929218240171\\
466	2.01807130273729\\
467	2.01688997501763\\
468	2.01574824682642\\
469	2.01464616415231\\
470	2.01358377138703\\
471	2.01256111132361\\
472	2.01157822515464\\
473	2.01063515247064\\
474	2.00973193125843\\
475	2.00886859789964\\
476	2.0080451871692\\
477	2.00726173223399\\
478	2.00651826465145\\
479	2.00581481436835\\
480	2.00515140971955\\
481	2.00452807742692\\
482	2.00394484259817\\
483	2.00340172872592\\
484	2.00289875768672\\
485	2.00243594974018\\
486	2.00201332352812\\
487	2.00163089607386\\
488	2.00128868278154\\
489	2.00098669743546\\
490	2.00072495219953\\
491	2.00050345761681\\
492	2.00032222260909\\
493	2.00018125447647\\
494	2.00008055889714\\
495	2.00002013992709\\
496	2\\
};
\end{axis}
\end{tikzpicture}%  % tikz
			\caption{Estimated y-Axis Positions}
		\end{subfigure}
	}
	\caption[Arc Movement Results for CREM]{Arc Movement Results for CREM (\Tsixty$=0.4$s).}
	\label{fig:trackingArcCREM}
\end{figure}
\begin{figure}[!htbp]
	\iftoggle{quick}{%
		\includegraphics[width=\textwidth,height=\figureheight]{plots/tracking/arc/results-T60=0.4-trem-xy.png}
	}{%
		\begin{subfigure}{0.49\textwidth}
			\centering
			\setlength{\figurewidth}{0.8\textwidth}
			% This file was created by matlab2tikz.
%
\definecolor{lms_red}{rgb}{0.80000,0.20780,0.21960}%
\definecolor{mycolor2}{rgb}{0.80000,0.20784,0.21961}%
\definecolor{mycolor3}{rgb}{0.92900,0.69400,0.12500}%
\definecolor{mycolor4}{rgb}{0.49400,0.18400,0.55600}%
%
\begin{tikzpicture}

\begin{axis}[%
width=0.951\figurewidth,
height=\figureheight,
at={(0\figurewidth,0\figureheight)},
scale only axis,
xmin=0,
xmax=496,
xtick={0,99.2,198.4,297.6,396.8,496},
xticklabels={{0},{1},{2},{3},{4},{5}},
xlabel style={font=\color{white!15!black}},
xlabel={$t$~[s]},
ymin=1,
ymax=5,
ylabel style={font=\color{white!15!black}},
ylabel={$p_x^{(t)}$~[m]},
axis background/.style={fill=white},
axis x line*=bottom,
axis y line*=left
]
\addplot [color=mycolor2, draw=none, mark=x, mark options={solid, mycolor2}, forget plot]
  table[row sep=crcr]{%
1	3.7\\
2	2.9\\
3	2.9\\
4	2.9\\
5	2.9\\
6	3\\
7	3\\
8	3\\
9	3\\
10	3\\
11	3\\
12	3\\
13	3\\
14	3\\
15	3\\
16	3\\
17	3\\
18	2.9\\
19	2.9\\
20	2.9\\
21	2.9\\
22	2.9\\
23	2.9\\
24	2.9\\
25	2.9\\
26	2.9\\
27	3.1\\
28	3.1\\
29	3.1\\
30	3.1\\
31	3.1\\
32	3.1\\
33	3.1\\
34	3.1\\
35	2.9\\
36	2.8\\
37	2.8\\
38	2.8\\
39	2.8\\
40	2.8\\
41	1.2\\
42	1.2\\
43	1.2\\
44	1.2\\
45	1.2\\
46	1.2\\
47	2.8\\
48	2.7\\
49	2.7\\
50	2.7\\
51	2.7\\
52	2.7\\
53	2.7\\
54	2.7\\
55	2.7\\
56	2.7\\
57	2.7\\
58	2.7\\
59	2.7\\
60	2.7\\
61	2.7\\
62	2.7\\
63	2.7\\
64	2.7\\
65	2.7\\
66	2.7\\
67	2.7\\
68	2.7\\
69	2.7\\
70	2.7\\
71	2.3\\
72	2.3\\
73	2.3\\
74	2.3\\
75	2.3\\
76	2.3\\
77	2.3\\
78	2.7\\
79	2.7\\
80	2.7\\
81	2.7\\
82	2.7\\
83	2.6\\
84	2.6\\
85	2.6\\
86	2.6\\
87	2.6\\
88	2.6\\
89	2.6\\
90	2.6\\
91	2.6\\
92	2.6\\
93	2.5\\
94	2.5\\
95	2.5\\
96	2.5\\
97	2.5\\
98	2.5\\
99	2.5\\
100	2.5\\
101	2.5\\
102	2.5\\
103	2.5\\
104	2.5\\
105	2.5\\
106	2.5\\
107	2.6\\
108	2.6\\
109	2.6\\
110	2.6\\
111	2.6\\
112	2.6\\
113	2.6\\
114	2.6\\
115	2.6\\
116	2.6\\
117	2.6\\
118	2.6\\
119	2.6\\
120	2.6\\
121	1.2\\
122	1.2\\
123	1.2\\
124	1.2\\
125	1.2\\
126	1.2\\
127	1.2\\
128	1.2\\
129	1.2\\
130	1.2\\
131	1.2\\
132	1.2\\
133	1.2\\
134	1.2\\
135	1.2\\
136	1.2\\
137	1.2\\
138	1.2\\
139	1.2\\
140	1.2\\
141	1.2\\
142	1.2\\
143	1.2\\
144	1.2\\
145	1.2\\
146	1.2\\
147	1.2\\
148	1.2\\
149	1.2\\
150	1.2\\
151	2.5\\
152	2.5\\
153	2.5\\
154	2.5\\
155	2.4\\
156	2.4\\
157	2.3\\
158	2.3\\
159	2.3\\
160	2.3\\
161	2.3\\
162	2.3\\
163	2.3\\
164	2.3\\
165	2.3\\
166	2.3\\
167	2.3\\
168	2.3\\
169	2.3\\
170	2.3\\
171	2.3\\
172	2.3\\
173	2.3\\
174	2.3\\
175	2.3\\
176	2.2\\
177	2.2\\
178	2.2\\
179	2.2\\
180	2.2\\
181	2.2\\
182	2.2\\
183	2.2\\
184	2.2\\
185	2.2\\
186	2.2\\
187	2.2\\
188	2.2\\
189	2.2\\
190	2.2\\
191	2.2\\
192	2.2\\
193	2.2\\
194	2.2\\
195	2.2\\
196	2.2\\
197	2.2\\
198	2.2\\
199	2.2\\
200	2.2\\
201	2.2\\
202	2.2\\
203	2.2\\
204	2.2\\
205	2.2\\
206	2.2\\
207	2.2\\
208	2.2\\
209	2.2\\
210	2.2\\
211	2.2\\
212	2.2\\
213	2.2\\
214	4\\
215	4\\
216	4\\
217	4\\
218	4\\
219	4\\
220	4\\
221	4\\
222	4\\
223	2.2\\
224	2.2\\
225	2.2\\
226	2.2\\
227	2.2\\
228	2.2\\
229	2.2\\
230	2.2\\
231	2.2\\
232	2.2\\
233	2.2\\
234	2.2\\
235	4\\
236	4\\
237	4\\
238	4\\
239	4\\
240	4\\
241	4\\
242	4\\
243	4\\
244	4\\
245	4\\
246	4\\
247	4\\
248	4\\
249	4\\
250	4\\
251	4\\
252	4\\
253	4\\
254	4\\
255	4\\
256	4\\
257	4\\
258	4\\
259	4\\
260	4\\
261	4\\
262	4\\
263	4\\
264	4\\
265	4\\
266	4\\
267	4\\
268	4\\
269	4\\
270	4\\
271	4\\
272	4\\
273	2\\
274	2\\
275	2\\
276	2\\
277	2\\
278	2\\
279	2\\
280	2\\
281	2\\
282	2\\
283	2\\
284	2\\
285	2\\
286	2\\
287	2\\
288	2\\
289	2\\
290	2\\
291	2\\
292	2\\
293	2\\
294	2\\
295	2\\
296	2\\
297	2\\
298	2\\
299	1.2\\
300	1.2\\
301	1.2\\
302	1.2\\
303	1.2\\
304	1.2\\
305	1.2\\
306	2\\
307	2\\
308	2\\
309	2\\
310	2\\
311	2\\
312	2\\
313	2\\
314	2\\
315	2\\
316	2\\
317	2\\
318	2\\
319	2\\
320	2\\
321	2\\
322	2\\
323	2\\
324	2\\
325	2\\
326	2\\
327	2\\
328	2\\
329	2\\
330	2.1\\
331	2.1\\
332	2.1\\
333	2.1\\
334	2.1\\
335	2.1\\
336	2.1\\
337	2.1\\
338	2.1\\
339	2.1\\
340	2.1\\
341	2.1\\
342	2.1\\
343	2.1\\
344	2.1\\
345	2.1\\
346	2.1\\
347	2.1\\
348	2.1\\
349	2.1\\
350	2.1\\
351	2.1\\
352	2.1\\
353	2.1\\
354	2.1\\
355	2.1\\
356	2.1\\
357	2.1\\
358	2.1\\
359	2.1\\
360	2.1\\
361	2.1\\
362	2.1\\
363	2.1\\
364	2.1\\
365	2.1\\
366	2.1\\
367	2.1\\
368	2.1\\
369	2.1\\
370	2.2\\
371	2.2\\
372	2.2\\
373	2.2\\
374	2.2\\
375	2.2\\
376	2.2\\
377	2.2\\
378	2.2\\
379	2.2\\
380	2.2\\
381	2.2\\
382	2.2\\
383	2.2\\
384	2.2\\
385	2.2\\
386	2.2\\
387	2.2\\
388	2.2\\
389	2.2\\
390	2.2\\
391	2.2\\
392	2.2\\
393	2.2\\
394	2.2\\
395	2.2\\
396	2.2\\
397	2.2\\
398	2.2\\
399	2.2\\
400	2.2\\
401	2.2\\
402	2.2\\
403	3.2\\
404	3.2\\
405	3.2\\
406	3.2\\
407	3.2\\
408	3.2\\
409	3.2\\
410	3.2\\
411	3.2\\
412	2.3\\
413	2.3\\
414	2.4\\
415	2.4\\
416	2.4\\
417	2.4\\
418	2.4\\
419	2.4\\
420	2.4\\
421	2.4\\
422	2.4\\
423	2.4\\
424	2.4\\
425	2.4\\
426	2.4\\
427	2.4\\
428	2.4\\
429	2.4\\
430	2.4\\
431	2.5\\
432	2.5\\
433	2.5\\
434	2.5\\
435	2.5\\
436	2.5\\
437	2.5\\
438	2.5\\
439	2.5\\
440	2.5\\
441	2.5\\
442	2.5\\
443	2.5\\
444	2.5\\
445	2.5\\
446	2.5\\
447	2.5\\
448	2.5\\
449	2.5\\
450	2.5\\
451	2.5\\
452	2.5\\
453	2.5\\
454	2.5\\
455	2.5\\
456	2.5\\
457	2.5\\
458	2.5\\
459	2.5\\
460	2.5\\
461	2.5\\
462	2.5\\
463	2.5\\
464	2.5\\
465	2.5\\
466	2.5\\
467	2.5\\
468	2.5\\
469	2.5\\
470	2.5\\
471	2.5\\
472	2.5\\
473	2.5\\
474	2.5\\
475	2.5\\
476	2.5\\
477	2.5\\
478	2.5\\
479	2.5\\
480	2.5\\
481	2.5\\
482	2.5\\
483	2.5\\
484	2.5\\
485	2.5\\
486	2.5\\
487	2.5\\
488	2.5\\
489	2.5\\
490	2.5\\
491	2.5\\
492	2.5\\
493	2.5\\
494	2.5\\
495	2.5\\
496	2.5\\
};
\addplot [color=mycolor2, draw=none, mark=x, mark options={solid, mycolor2}, forget plot]
  table[row sep=crcr]{%
1	2.9\\
2	1.2\\
3	3.1\\
4	3.1\\
5	2.9\\
6	2.9\\
7	2.9\\
8	3.2\\
9	3.3\\
10	3.3\\
11	3.3\\
12	3.3\\
13	3.3\\
14	3.3\\
15	3\\
16	3\\
17	3\\
18	3\\
19	3\\
20	3\\
21	3\\
22	3\\
23	3\\
24	3\\
25	3\\
26	3\\
27	2.9\\
28	2.9\\
29	2.9\\
30	2.9\\
31	3.4\\
32	3.4\\
33	3.4\\
34	2.9\\
35	3.1\\
36	3.1\\
37	3.2\\
38	3.1\\
39	3.1\\
40	3.1\\
41	2.8\\
42	2.8\\
43	2.8\\
44	1.2\\
45	2.8\\
46	2.8\\
47	1.2\\
48	1.2\\
49	1.2\\
50	1.2\\
51	1.2\\
52	1.2\\
53	2.8\\
54	2.8\\
55	2.8\\
56	3.4\\
57	3.4\\
58	3.4\\
59	3.4\\
60	3.4\\
61	3.4\\
62	3.4\\
63	3.4\\
64	3.4\\
65	3.4\\
66	3.4\\
67	3.4\\
68	3.4\\
69	3.4\\
70	3.4\\
71	2.7\\
72	3.4\\
73	3.5\\
74	3.5\\
75	3.5\\
76	3.5\\
77	2.7\\
78	2.3\\
79	1.2\\
80	1.2\\
81	1.2\\
82	1.2\\
83	1.2\\
84	1.2\\
85	2.2\\
86	2.5\\
87	2.2\\
88	2.2\\
89	2.2\\
90	2.2\\
91	1.3\\
92	2.2\\
93	2.2\\
94	2.2\\
95	2.2\\
96	2.2\\
97	2.2\\
98	2.2\\
99	2.2\\
100	2.2\\
101	2.2\\
102	2.2\\
103	1.2\\
104	1.2\\
105	2.2\\
106	2.2\\
107	2.2\\
108	2.2\\
109	1.2\\
110	1.2\\
111	1.2\\
112	1.2\\
113	1.2\\
114	1.2\\
115	1.2\\
116	1.2\\
117	3.9\\
118	1.2\\
119	1.2\\
120	1.2\\
121	2.6\\
122	2.6\\
123	2.5\\
124	2.6\\
125	2.6\\
126	2.6\\
127	2.6\\
128	2.6\\
129	1.2\\
130	1.2\\
131	1.2\\
132	2.6\\
133	2.6\\
134	1.2\\
135	1.2\\
136	1.2\\
137	1.2\\
138	1.2\\
139	1.2\\
140	2.8\\
141	2.8\\
142	2.8\\
143	1.2\\
144	1.2\\
145	1.2\\
146	1.2\\
147	1.2\\
148	1.2\\
149	1.2\\
150	2.5\\
151	1.2\\
152	1.2\\
153	1.2\\
154	1.2\\
155	1.2\\
156	1.2\\
157	1.2\\
158	1.2\\
159	1.2\\
160	1.2\\
161	1.2\\
162	1.2\\
163	1.2\\
164	1.2\\
165	1.2\\
166	2.2\\
167	2.2\\
168	2.2\\
169	1.2\\
170	3.8\\
171	3.8\\
172	3.9\\
173	3.9\\
174	3.9\\
175	3.8\\
176	3.8\\
177	3.8\\
178	3.8\\
179	3.9\\
180	3.9\\
181	3.9\\
182	3.9\\
183	3.9\\
184	3.9\\
185	3.9\\
186	3.9\\
187	3.9\\
188	3.9\\
189	3.9\\
190	3.9\\
191	3.9\\
192	3.9\\
193	4.7\\
194	1.2\\
195	1.2\\
196	1.2\\
197	1.2\\
198	1.2\\
199	1.2\\
200	1.2\\
201	1.2\\
202	1.2\\
203	1.2\\
204	2.8\\
205	2.8\\
206	2.8\\
207	1.2\\
208	4\\
209	4\\
210	4\\
211	4\\
212	4\\
213	4\\
214	2.2\\
215	2.2\\
216	2.2\\
217	2.2\\
218	2.2\\
219	2.2\\
220	2.2\\
221	2.2\\
222	2.2\\
223	4\\
224	4\\
225	4\\
226	4\\
227	4\\
228	4\\
229	4\\
230	4\\
231	4\\
232	4\\
233	4\\
234	4\\
235	2.2\\
236	2.2\\
237	2.2\\
238	2.2\\
239	2.2\\
240	2.2\\
241	2.2\\
242	2.2\\
243	2.2\\
244	2.2\\
245	2.2\\
246	2.2\\
247	2.2\\
248	2.2\\
249	3.6\\
250	3.6\\
251	3.6\\
252	2.2\\
253	3.5\\
254	3.5\\
255	3.5\\
256	3.6\\
257	3.6\\
258	3.6\\
259	3.6\\
260	3.6\\
261	3.6\\
262	3.6\\
263	3.6\\
264	1.2\\
265	1.2\\
266	1.2\\
267	1.2\\
268	1.2\\
269	1.2\\
270	1.2\\
271	1.9\\
272	1.9\\
273	4\\
274	4\\
275	4\\
276	4\\
277	4\\
278	4\\
279	4\\
280	4\\
281	4\\
282	4\\
283	4\\
284	4\\
285	2.9\\
286	2.9\\
287	2.9\\
288	4\\
289	1.2\\
290	2.9\\
291	2.9\\
292	2.9\\
293	2.9\\
294	2.9\\
295	2.9\\
296	2.9\\
297	2.9\\
298	1.2\\
299	1.9\\
300	1.9\\
301	2.9\\
302	1.9\\
303	1.9\\
304	1.9\\
305	2.9\\
306	1.2\\
307	1.2\\
308	1.2\\
309	1.2\\
310	1.2\\
311	1.2\\
312	2.9\\
313	2.9\\
314	2.9\\
315	2.9\\
316	2.9\\
317	2.9\\
318	2.9\\
319	2.9\\
320	2.9\\
321	2.9\\
322	2.9\\
323	2.9\\
324	2.9\\
325	2.9\\
326	2.9\\
327	2.9\\
328	2.9\\
329	2.9\\
330	2.9\\
331	2.9\\
332	2.9\\
333	2.9\\
334	2.9\\
335	2.9\\
336	2.9\\
337	2.9\\
338	2.9\\
339	2.9\\
340	2.9\\
341	2.9\\
342	2.9\\
343	2.9\\
344	2.9\\
345	2.9\\
346	2.9\\
347	2.9\\
348	2.9\\
349	2.9\\
350	2.9\\
351	2.9\\
352	2.9\\
353	2.9\\
354	3.6\\
355	3.6\\
356	3.6\\
357	3.6\\
358	3.6\\
359	3.6\\
360	3.6\\
361	3.6\\
362	3.6\\
363	3.6\\
364	3.6\\
365	3.6\\
366	2.9\\
367	2.9\\
368	2.5\\
369	2.5\\
370	1.2\\
371	2.9\\
372	2.9\\
373	2.9\\
374	2.9\\
375	3.6\\
376	3.7\\
377	3.7\\
378	3.6\\
379	3.6\\
380	3.6\\
381	3.6\\
382	3.6\\
383	3.6\\
384	3.6\\
385	3.7\\
386	3.6\\
387	3.6\\
388	3.6\\
389	3.7\\
390	3.7\\
391	3.7\\
392	3.7\\
393	3.7\\
394	3.7\\
395	3.7\\
396	3.7\\
397	3.7\\
398	3.7\\
399	3.7\\
400	3.6\\
401	3.6\\
402	3.2\\
403	2.2\\
404	2.2\\
405	2.2\\
406	2.2\\
407	2.2\\
408	2.3\\
409	2.3\\
410	2.3\\
411	2.3\\
412	3.2\\
413	3.2\\
414	3.2\\
415	3.2\\
416	3.2\\
417	3.2\\
418	3.2\\
419	3.2\\
420	3.2\\
421	3.2\\
422	3.2\\
423	3.2\\
424	3.2\\
425	3.2\\
426	3.2\\
427	3.2\\
428	3.2\\
429	3.2\\
430	3.2\\
431	3.2\\
432	3.2\\
433	3.2\\
434	2.2\\
435	2.2\\
436	2.2\\
437	2.2\\
438	3.2\\
439	3.2\\
440	3.2\\
441	3.2\\
442	3.2\\
443	3.2\\
444	3.2\\
445	3.2\\
446	3.2\\
447	3.2\\
448	3.2\\
449	3.2\\
450	3.2\\
451	3.2\\
452	3.7\\
453	3.2\\
454	3.2\\
455	3.2\\
456	3.2\\
457	3.2\\
458	3.2\\
459	3.2\\
460	3.2\\
461	3.2\\
462	3.2\\
463	3.2\\
464	3.2\\
465	3.2\\
466	3.6\\
467	3.2\\
468	3.2\\
469	3.2\\
470	3.2\\
471	3.2\\
472	3.2\\
473	3.2\\
474	3.2\\
475	3.2\\
476	3.2\\
477	3.2\\
478	2.9\\
479	3\\
480	3\\
481	3\\
482	3\\
483	3\\
484	3\\
485	3\\
486	3\\
487	3\\
488	3\\
489	3\\
490	3.6\\
491	3.6\\
492	3.6\\
493	3.6\\
494	3.6\\
495	3.6\\
496	3.6\\
};
\addplot [color=mycolor3, dashed, forget plot]
  table[row sep=crcr]{%
1	3\\
2	3.00634660921834\\
3	3.01269296279619\\
4	3.01903880510335\\
5	3.02538388053021\\
6	3.03172793349807\\
7	3.03807070846939\\
8	3.04441194995813\\
9	3.05075140253999\\
10	3.05708881086277\\
11	3.06342391965656\\
12	3.06975647374413\\
13	3.0760862180511\\
14	3.0824128976163\\
15	3.088736257602\\
16	3.09505604330418\\
17	3.1013720001628\\
18	3.10768387377204\\
19	3.11399140989054\\
20	3.12029435445167\\
21	3.12659245357375\\
22	3.13288545357025\\
23	3.13917310096007\\
24	3.14545514247766\\
25	3.15173132508333\\
26	3.15800139597335\\
27	3.16426510259018\\
28	3.17052219263262\\
29	3.17677241406602\\
30	3.18301551513234\\
31	3.18925124436041\\
32	3.19547935057595\\
33	3.20169958291175\\
34	3.20791169081776\\
35	3.21411542407118\\
36	3.22031053278654\\
37	3.22649676742576\\
38	3.23267387880822\\
39	3.23884161812077\\
40	3.24499973692776\\
41	3.25114798718108\\
42	3.25728612123009\\
43	3.26341389183166\\
44	3.26953105216007\\
45	3.275637355817\\
46	3.28173255684143\\
47	3.28781640971955\\
48	3.29388866939466\\
49	3.29994909127701\\
50	3.30599743125371\\
51	3.31203344569849\\
52	3.31805689148158\\
53	3.32406752597945\\
54	3.33006510708463\\
55	3.33604939321543\\
56	3.34202014332567\\
57	3.34797711691441\\
58	3.35392007403561\\
59	3.35984877530784\\
60	3.36576298192387\\
61	3.37166245566033\\
62	3.37754695888726\\
63	3.38341625457774\\
64	3.38927010631739\\
65	3.39510827831392\\
66	3.40093053540661\\
67	3.4067366430758\\
68	3.41252636745231\\
69	3.41829947532689\\
70	3.4240557341596\\
71	3.42979491208917\\
72	3.43551677794235\\
73	3.44122110124322\\
74	3.44690765222247\\
75	3.45257620182665\\
76	3.45822652172741\\
77	3.46385838433069\\
78	3.46947156278589\\
79	3.47506583099499\\
80	3.4806409636217\\
81	3.48619673610047\\
82	3.4917329246456\\
83	3.49724930626024\\
84	3.50274565874532\\
85	3.50822176070857\\
86	3.51367739157341\\
87	3.51911233158781\\
88	3.52452636183319\\
89	3.5299192642332\\
90	3.53529082156252\\
91	3.5406408174556\\
92	3.54596903641538\\
93	3.55127526382198\\
94	3.55655928594134\\
95	3.56182088993382\\
96	3.56705986386277\\
97	3.57227599670309\\
98	3.57746907834971\\
99	3.58263889962605\\
100	3.58778525229247\\
101	3.59290792905464\\
102	3.59800672357188\\
103	3.60308143046549\\
104	3.60813184532702\\
105	3.61315776472649\\
106	3.61815898622061\\
107	3.62313530836089\\
108	3.62808653070182\\
109	3.63301245380887\\
110	3.63791287926659\\
111	3.64278760968654\\
112	3.6476364487153\\
113	3.65245920104234\\
114	3.65725567240791\\
115	3.66202566961082\\
116	3.66676900051629\\
117	3.67148547406365\\
118	3.67617490027402\\
119	3.680837090258\\
120	3.68547185622327\\
121	3.69007901148211\\
122	3.694658370459\\
123	3.69920974869801\\
124	3.7037329628703\\
125	3.70822783078146\\
126	3.71269417137886\\
127	3.71713180475896\\
128	3.72154055217454\\
129	3.72592023604188\\
130	3.73027067994796\\
131	3.73459170865753\\
132	3.7388831481202\\
133	3.74314482547739\\
134	3.74737656906938\\
135	3.75157820844214\\
136	3.75574957435426\\
137	3.75989049878372\\
138	3.76400081493469\\
139	3.76808035724423\\
140	3.77212896138898\\
141	3.77614646429176\\
142	3.78013270412812\\
143	3.78408752033292\\
144	3.78801075360672\\
145	3.79190224592227\\
146	3.79576184053083\\
147	3.79958938196848\\
148	3.80338471606242\\
149	3.80714768993714\\
150	3.8108781520206\\
151	3.81457595205034\\
152	3.8182409410795\\
153	3.82187297148286\\
154	3.82547189696277\\
155	3.82903757255504\\
156	3.83256985463477\\
157	3.83606860092216\\
158	3.8395336704882\\
159	3.84296492376042\\
160	3.84636222252842\\
161	3.84972542994951\\
162	3.85305441055419\\
163	3.85634903025159\\
164	3.85960915633492\\
165	3.86283465748677\\
166	3.86602540378444\\
167	3.86918126670512\\
168	3.87230211913111\\
169	3.87538783535494\\
170	3.8784382910844\\
171	3.88145336344758\\
172	3.88443293099781\\
173	3.88737687371855\\
174	3.8902850730282\\
175	3.89315741178492\\
176	3.89599377429134\\
177	3.89879404629917\\
178	3.90155811501387\\
179	3.90428586909916\\
180	3.90697719868149\\
181	3.90963199535452\\
182	3.91225015218341\\
183	3.91483156370918\\
184	3.91737612595296\\
185	3.91988373642016\\
186	3.92235429410458\\
187	3.92478769949253\\
188	3.92718385456679\\
189	3.92954266281058\\
190	3.93186402921145\\
191	3.93414786026511\\
192	3.93639406397916\\
193	3.93860254987686\\
194	3.9407732290007\\
195	3.94290601391606\\
196	3.94500081871467\\
197	3.94705755901809\\
198	3.94907615198113\\
199	3.95105651629515\\
200	3.95299857219138\\
201	3.95490224144407\\
202	3.95676744737372\\
203	3.9585941148501\\
204	3.9603821702953\\
205	3.96213154168673\\
206	3.96384215855994\\
207	3.96551395201155\\
208	3.96714685470196\\
209	3.96874080085808\\
210	3.970295726276\\
211	3.97181156832354\\
212	3.97328826594282\\
213	3.97472575965266\\
214	3.97612399155103\\
215	3.97748290531735\\
216	3.97880244621478\\
217	3.98008256109239\\
218	3.98132319838736\\
219	3.98252430812698\\
220	3.98368584193073\\
221	3.98480775301221\\
222	3.98588999618099\\
223	3.98693252784448\\
224	3.98793530600966\\
225	3.98889829028477\\
226	3.98982144188093\\
227	3.99070472361375\\
228	3.99154809990476\\
229	3.99235153678288\\
230	3.9931150018858\\
231	3.99383846446125\\
232	3.99452189536827\\
233	3.99516526707836\\
234	3.9957685536766\\
235	3.99633173086269\\
236	3.99685477595194\\
237	3.99733766787617\\
238	3.99778038718456\\
239	3.99818291604445\\
240	3.99854523824203\\
241	3.99886733918301\\
242	3.99914920589321\\
243	3.9993908270191\\
244	3.99959219282819\\
245	3.99975329520951\\
246	3.99987412767388\\
247	3.99995468535417\\
248	3.99999496500555\\
249	3.99999496500555\\
250	3.99995468535417\\
251	3.99987412767388\\
252	3.99975329520951\\
253	3.99959219282819\\
254	3.9993908270191\\
255	3.99914920589321\\
256	3.99886733918301\\
257	3.99854523824203\\
258	3.99818291604445\\
259	3.99778038718456\\
260	3.99733766787617\\
261	3.99685477595194\\
262	3.99633173086269\\
263	3.9957685536766\\
264	3.99516526707836\\
265	3.99452189536827\\
266	3.99383846446125\\
267	3.9931150018858\\
268	3.99235153678288\\
269	3.99154809990476\\
270	3.99070472361375\\
271	3.98982144188093\\
272	3.98889829028477\\
273	3.98793530600966\\
274	3.98693252784448\\
275	3.98588999618099\\
276	3.98480775301221\\
277	3.98368584193073\\
278	3.98252430812698\\
279	3.98132319838736\\
280	3.98008256109239\\
281	3.97880244621478\\
282	3.97748290531735\\
283	3.97612399155103\\
284	3.97472575965266\\
285	3.97328826594282\\
286	3.97181156832354\\
287	3.970295726276\\
288	3.96874080085808\\
289	3.96714685470196\\
290	3.96551395201155\\
291	3.96384215855994\\
292	3.96213154168673\\
293	3.9603821702953\\
294	3.9585941148501\\
295	3.95676744737372\\
296	3.95490224144407\\
297	3.95299857219138\\
298	3.95105651629515\\
299	3.94907615198113\\
300	3.94705755901809\\
301	3.94500081871467\\
302	3.94290601391606\\
303	3.9407732290007\\
304	3.93860254987686\\
305	3.93639406397916\\
306	3.93414786026511\\
307	3.93186402921145\\
308	3.92954266281058\\
309	3.92718385456679\\
310	3.92478769949253\\
311	3.92235429410458\\
312	3.91988373642016\\
313	3.91737612595296\\
314	3.91483156370918\\
315	3.91225015218341\\
316	3.90963199535452\\
317	3.90697719868149\\
318	3.90428586909916\\
319	3.90155811501387\\
320	3.89879404629917\\
321	3.89599377429134\\
322	3.89315741178492\\
323	3.8902850730282\\
324	3.88737687371855\\
325	3.88443293099781\\
326	3.88145336344758\\
327	3.8784382910844\\
328	3.87538783535494\\
329	3.87230211913111\\
330	3.86918126670512\\
331	3.86602540378444\\
332	3.86283465748677\\
333	3.85960915633492\\
334	3.85634903025159\\
335	3.85305441055419\\
336	3.84972542994951\\
337	3.84636222252842\\
338	3.84296492376042\\
339	3.8395336704882\\
340	3.83606860092216\\
341	3.83256985463477\\
342	3.82903757255504\\
343	3.82547189696277\\
344	3.82187297148286\\
345	3.8182409410795\\
346	3.81457595205034\\
347	3.8108781520206\\
348	3.80714768993714\\
349	3.80338471606242\\
350	3.79958938196848\\
351	3.79576184053083\\
352	3.79190224592227\\
353	3.78801075360672\\
354	3.78408752033292\\
355	3.78013270412812\\
356	3.77614646429176\\
357	3.77212896138898\\
358	3.76808035724423\\
359	3.76400081493469\\
360	3.75989049878372\\
361	3.75574957435426\\
362	3.75157820844214\\
363	3.74737656906938\\
364	3.74314482547739\\
365	3.7388831481202\\
366	3.73459170865753\\
367	3.73027067994796\\
368	3.72592023604188\\
369	3.72154055217454\\
370	3.71713180475896\\
371	3.71269417137886\\
372	3.70822783078146\\
373	3.7037329628703\\
374	3.69920974869801\\
375	3.694658370459\\
376	3.69007901148211\\
377	3.68547185622327\\
378	3.680837090258\\
379	3.67617490027402\\
380	3.67148547406365\\
381	3.66676900051629\\
382	3.66202566961082\\
383	3.65725567240791\\
384	3.65245920104234\\
385	3.6476364487153\\
386	3.64278760968654\\
387	3.63791287926659\\
388	3.63301245380887\\
389	3.62808653070182\\
390	3.62313530836089\\
391	3.61815898622061\\
392	3.61315776472649\\
393	3.60813184532702\\
394	3.60308143046549\\
395	3.59800672357188\\
396	3.59290792905464\\
397	3.58778525229247\\
398	3.58263889962605\\
399	3.57746907834971\\
400	3.57227599670309\\
401	3.56705986386277\\
402	3.56182088993382\\
403	3.55655928594134\\
404	3.55127526382198\\
405	3.54596903641538\\
406	3.5406408174556\\
407	3.53529082156252\\
408	3.5299192642332\\
409	3.52452636183319\\
410	3.51911233158781\\
411	3.51367739157341\\
412	3.50822176070857\\
413	3.50274565874532\\
414	3.49724930626024\\
415	3.4917329246456\\
416	3.48619673610047\\
417	3.4806409636217\\
418	3.47506583099499\\
419	3.46947156278589\\
420	3.46385838433069\\
421	3.45822652172741\\
422	3.45257620182665\\
423	3.44690765222247\\
424	3.44122110124322\\
425	3.43551677794235\\
426	3.42979491208917\\
427	3.4240557341596\\
428	3.41829947532689\\
429	3.41252636745231\\
430	3.4067366430758\\
431	3.40093053540661\\
432	3.39510827831392\\
433	3.38927010631739\\
434	3.38341625457774\\
435	3.37754695888726\\
436	3.37166245566033\\
437	3.36576298192387\\
438	3.35984877530784\\
439	3.35392007403561\\
440	3.34797711691441\\
441	3.34202014332567\\
442	3.33604939321543\\
443	3.33006510708463\\
444	3.32406752597945\\
445	3.31805689148158\\
446	3.31203344569849\\
447	3.3059974312537\\
448	3.29994909127701\\
449	3.29388866939466\\
450	3.28781640971955\\
451	3.28173255684143\\
452	3.275637355817\\
453	3.26953105216007\\
454	3.26341389183166\\
455	3.25728612123009\\
456	3.25114798718108\\
457	3.24499973692776\\
458	3.23884161812077\\
459	3.23267387880822\\
460	3.22649676742576\\
461	3.22031053278654\\
462	3.21411542407118\\
463	3.20791169081776\\
464	3.20169958291175\\
465	3.19547935057595\\
466	3.18925124436041\\
467	3.18301551513234\\
468	3.17677241406602\\
469	3.17052219263262\\
470	3.16426510259018\\
471	3.15800139597335\\
472	3.15173132508333\\
473	3.14545514247766\\
474	3.13917310096007\\
475	3.13288545357025\\
476	3.12659245357375\\
477	3.12029435445167\\
478	3.11399140989054\\
479	3.10768387377204\\
480	3.1013720001628\\
481	3.09505604330418\\
482	3.088736257602\\
483	3.0824128976163\\
484	3.0760862180511\\
485	3.06975647374413\\
486	3.06342391965656\\
487	3.05708881086277\\
488	3.05075140253999\\
489	3.04441194995813\\
490	3.03807070846939\\
491	3.03172793349807\\
492	3.02538388053021\\
493	3.01903880510335\\
494	3.01269296279619\\
495	3.00634660921834\\
496	3\\
};
\addplot [color=mycolor4, dashed, forget plot]
  table[row sep=crcr]{%
1	3\\
2	2.99365339078166\\
3	2.98730703720381\\
4	2.98096119489665\\
5	2.97461611946979\\
6	2.96827206650193\\
7	2.96192929153061\\
8	2.95558805004187\\
9	2.94924859746001\\
10	2.94291118913723\\
11	2.93657608034344\\
12	2.93024352625587\\
13	2.9239137819489\\
14	2.9175871023837\\
15	2.911263742398\\
16	2.90494395669582\\
17	2.8986279998372\\
18	2.89231612622796\\
19	2.88600859010946\\
20	2.87970564554833\\
21	2.87340754642625\\
22	2.86711454642975\\
23	2.86082689903993\\
24	2.85454485752234\\
25	2.84826867491667\\
26	2.84199860402665\\
27	2.83573489740982\\
28	2.82947780736738\\
29	2.82322758593398\\
30	2.81698448486766\\
31	2.81074875563959\\
32	2.80452064942405\\
33	2.79830041708825\\
34	2.79208830918224\\
35	2.78588457592882\\
36	2.77968946721346\\
37	2.77350323257424\\
38	2.76732612119178\\
39	2.76115838187923\\
40	2.75500026307224\\
41	2.74885201281892\\
42	2.74271387876991\\
43	2.73658610816834\\
44	2.73046894783993\\
45	2.724362644183\\
46	2.71826744315857\\
47	2.71218359028045\\
48	2.70611133060534\\
49	2.70005090872299\\
50	2.69400256874629\\
51	2.68796655430151\\
52	2.68194310851842\\
53	2.67593247402055\\
54	2.66993489291537\\
55	2.66395060678457\\
56	2.65797985667433\\
57	2.65202288308559\\
58	2.64607992596439\\
59	2.64015122469216\\
60	2.63423701807613\\
61	2.62833754433967\\
62	2.62245304111274\\
63	2.61658374542226\\
64	2.61072989368261\\
65	2.60489172168608\\
66	2.59906946459339\\
67	2.5932633569242\\
68	2.58747363254769\\
69	2.58170052467311\\
70	2.5759442658404\\
71	2.57020508791083\\
72	2.56448322205765\\
73	2.55877889875678\\
74	2.55309234777753\\
75	2.54742379817335\\
76	2.54177347827259\\
77	2.53614161566931\\
78	2.53052843721411\\
79	2.524934169005\\
80	2.5193590363783\\
81	2.51380326389953\\
82	2.5082670753544\\
83	2.50275069373976\\
84	2.49725434125468\\
85	2.49177823929143\\
86	2.48632260842659\\
87	2.48088766841219\\
88	2.47547363816681\\
89	2.47008073576679\\
90	2.46470917843748\\
91	2.4593591825444\\
92	2.45403096358462\\
93	2.44872473617802\\
94	2.44344071405866\\
95	2.43817911006618\\
96	2.43294013613723\\
97	2.42772400329691\\
98	2.42253092165029\\
99	2.41736110037395\\
100	2.41221474770753\\
101	2.40709207094536\\
102	2.40199327642812\\
103	2.39691856953451\\
104	2.39186815467298\\
105	2.38684223527351\\
106	2.38184101377939\\
107	2.37686469163911\\
108	2.37191346929818\\
109	2.36698754619113\\
110	2.36208712073341\\
111	2.35721239031346\\
112	2.3523635512847\\
113	2.34754079895766\\
114	2.34274432759209\\
115	2.33797433038918\\
116	2.33323099948371\\
117	2.32851452593635\\
118	2.32382509972598\\
119	2.319162909742\\
120	2.31452814377673\\
121	2.30992098851789\\
122	2.305341629541\\
123	2.30079025130199\\
124	2.2962670371297\\
125	2.29177216921854\\
126	2.28730582862114\\
127	2.28286819524104\\
128	2.27845944782546\\
129	2.27407976395812\\
130	2.26972932005204\\
131	2.26540829134247\\
132	2.2611168518798\\
133	2.25685517452261\\
134	2.25262343093062\\
135	2.24842179155786\\
136	2.24425042564574\\
137	2.24010950121628\\
138	2.23599918506531\\
139	2.23191964275577\\
140	2.22787103861102\\
141	2.22385353570824\\
142	2.21986729587188\\
143	2.21591247966708\\
144	2.21198924639328\\
145	2.20809775407773\\
146	2.20423815946917\\
147	2.20041061803151\\
148	2.19661528393758\\
149	2.19285231006286\\
150	2.1891218479794\\
151	2.18542404794966\\
152	2.1817590589205\\
153	2.17812702851714\\
154	2.17452810303723\\
155	2.17096242744496\\
156	2.16743014536523\\
157	2.16393139907784\\
158	2.1604663295118\\
159	2.15703507623958\\
160	2.15363777747158\\
161	2.15027457005049\\
162	2.14694558944581\\
163	2.14365096974841\\
164	2.14039084366508\\
165	2.13716534251323\\
166	2.13397459621556\\
167	2.13081873329488\\
168	2.12769788086889\\
169	2.12461216464506\\
170	2.1215617089156\\
171	2.11854663655242\\
172	2.11556706900219\\
173	2.11262312628145\\
174	2.1097149269718\\
175	2.10684258821508\\
176	2.10400622570866\\
177	2.10120595370083\\
178	2.09844188498613\\
179	2.09571413090084\\
180	2.09302280131851\\
181	2.09036800464548\\
182	2.08774984781659\\
183	2.08516843629082\\
184	2.08262387404704\\
185	2.08011626357984\\
186	2.07764570589542\\
187	2.07521230050747\\
188	2.07281614543321\\
189	2.07045733718942\\
190	2.06813597078855\\
191	2.06585213973489\\
192	2.06360593602084\\
193	2.06139745012314\\
194	2.0592267709993\\
195	2.05709398608394\\
196	2.05499918128533\\
197	2.05294244098191\\
198	2.05092384801887\\
199	2.04894348370485\\
200	2.04700142780862\\
201	2.04509775855593\\
202	2.04323255262628\\
203	2.0414058851499\\
204	2.0396178297047\\
205	2.03786845831327\\
206	2.03615784144006\\
207	2.03448604798845\\
208	2.03285314529804\\
209	2.03125919914192\\
210	2.029704273724\\
211	2.02818843167646\\
212	2.02671173405718\\
213	2.02527424034734\\
214	2.02387600844897\\
215	2.02251709468265\\
216	2.02119755378522\\
217	2.01991743890761\\
218	2.01867680161264\\
219	2.01747569187302\\
220	2.01631415806927\\
221	2.01519224698779\\
222	2.01411000381901\\
223	2.01306747215552\\
224	2.01206469399034\\
225	2.01110170971523\\
226	2.01017855811907\\
227	2.00929527638625\\
228	2.00845190009524\\
229	2.00764846321712\\
230	2.0068849981142\\
231	2.00616153553875\\
232	2.00547810463173\\
233	2.00483473292164\\
234	2.0042314463234\\
235	2.00366826913731\\
236	2.00314522404806\\
237	2.00266233212383\\
238	2.00221961281544\\
239	2.00181708395555\\
240	2.00145476175797\\
241	2.00113266081699\\
242	2.00085079410679\\
243	2.0006091729809\\
244	2.00040780717181\\
245	2.00024670479049\\
246	2.00012587232612\\
247	2.00004531464583\\
248	2.00000503499445\\
249	2.00000503499445\\
250	2.00004531464583\\
251	2.00012587232612\\
252	2.00024670479049\\
253	2.00040780717181\\
254	2.0006091729809\\
255	2.00085079410679\\
256	2.00113266081699\\
257	2.00145476175797\\
258	2.00181708395555\\
259	2.00221961281544\\
260	2.00266233212383\\
261	2.00314522404806\\
262	2.00366826913731\\
263	2.0042314463234\\
264	2.00483473292164\\
265	2.00547810463173\\
266	2.00616153553875\\
267	2.0068849981142\\
268	2.00764846321712\\
269	2.00845190009524\\
270	2.00929527638625\\
271	2.01017855811907\\
272	2.01110170971523\\
273	2.01206469399034\\
274	2.01306747215552\\
275	2.01411000381901\\
276	2.01519224698779\\
277	2.01631415806927\\
278	2.01747569187302\\
279	2.01867680161264\\
280	2.01991743890761\\
281	2.02119755378522\\
282	2.02251709468265\\
283	2.02387600844897\\
284	2.02527424034734\\
285	2.02671173405718\\
286	2.02818843167646\\
287	2.029704273724\\
288	2.03125919914192\\
289	2.03285314529804\\
290	2.03448604798845\\
291	2.03615784144006\\
292	2.03786845831327\\
293	2.0396178297047\\
294	2.0414058851499\\
295	2.04323255262628\\
296	2.04509775855593\\
297	2.04700142780862\\
298	2.04894348370485\\
299	2.05092384801887\\
300	2.05294244098191\\
301	2.05499918128533\\
302	2.05709398608394\\
303	2.0592267709993\\
304	2.06139745012314\\
305	2.06360593602084\\
306	2.06585213973489\\
307	2.06813597078855\\
308	2.07045733718942\\
309	2.07281614543321\\
310	2.07521230050747\\
311	2.07764570589542\\
312	2.08011626357984\\
313	2.08262387404704\\
314	2.08516843629082\\
315	2.08774984781659\\
316	2.09036800464548\\
317	2.09302280131851\\
318	2.09571413090084\\
319	2.09844188498613\\
320	2.10120595370083\\
321	2.10400622570866\\
322	2.10684258821508\\
323	2.1097149269718\\
324	2.11262312628145\\
325	2.11556706900219\\
326	2.11854663655242\\
327	2.1215617089156\\
328	2.12461216464506\\
329	2.12769788086889\\
330	2.13081873329488\\
331	2.13397459621556\\
332	2.13716534251323\\
333	2.14039084366508\\
334	2.14365096974841\\
335	2.14694558944581\\
336	2.15027457005049\\
337	2.15363777747158\\
338	2.15703507623958\\
339	2.1604663295118\\
340	2.16393139907784\\
341	2.16743014536523\\
342	2.17096242744496\\
343	2.17452810303723\\
344	2.17812702851714\\
345	2.1817590589205\\
346	2.18542404794966\\
347	2.1891218479794\\
348	2.19285231006286\\
349	2.19661528393758\\
350	2.20041061803152\\
351	2.20423815946917\\
352	2.20809775407773\\
353	2.21198924639328\\
354	2.21591247966708\\
355	2.21986729587188\\
356	2.22385353570824\\
357	2.22787103861102\\
358	2.23191964275577\\
359	2.23599918506531\\
360	2.24010950121628\\
361	2.24425042564574\\
362	2.24842179155786\\
363	2.25262343093062\\
364	2.25685517452261\\
365	2.2611168518798\\
366	2.26540829134247\\
367	2.26972932005204\\
368	2.27407976395812\\
369	2.27845944782546\\
370	2.28286819524104\\
371	2.28730582862114\\
372	2.29177216921854\\
373	2.2962670371297\\
374	2.30079025130199\\
375	2.305341629541\\
376	2.30992098851789\\
377	2.31452814377673\\
378	2.319162909742\\
379	2.32382509972598\\
380	2.32851452593635\\
381	2.33323099948371\\
382	2.33797433038918\\
383	2.34274432759209\\
384	2.34754079895766\\
385	2.3523635512847\\
386	2.35721239031346\\
387	2.36208712073341\\
388	2.36698754619113\\
389	2.37191346929818\\
390	2.37686469163911\\
391	2.38184101377939\\
392	2.38684223527351\\
393	2.39186815467298\\
394	2.39691856953451\\
395	2.40199327642812\\
396	2.40709207094536\\
397	2.41221474770753\\
398	2.41736110037395\\
399	2.42253092165029\\
400	2.42772400329691\\
401	2.43294013613723\\
402	2.43817911006618\\
403	2.44344071405866\\
404	2.44872473617802\\
405	2.45403096358462\\
406	2.4593591825444\\
407	2.46470917843748\\
408	2.4700807357668\\
409	2.47547363816681\\
410	2.48088766841219\\
411	2.48632260842659\\
412	2.49177823929143\\
413	2.49725434125468\\
414	2.50275069373976\\
415	2.5082670753544\\
416	2.51380326389953\\
417	2.5193590363783\\
418	2.52493416900501\\
419	2.53052843721411\\
420	2.53614161566931\\
421	2.54177347827259\\
422	2.54742379817335\\
423	2.55309234777753\\
424	2.55877889875678\\
425	2.56448322205765\\
426	2.57020508791083\\
427	2.5759442658404\\
428	2.58170052467311\\
429	2.58747363254769\\
430	2.5932633569242\\
431	2.59906946459339\\
432	2.60489172168608\\
433	2.61072989368261\\
434	2.61658374542226\\
435	2.62245304111274\\
436	2.62833754433967\\
437	2.63423701807613\\
438	2.64015122469216\\
439	2.64607992596439\\
440	2.65202288308559\\
441	2.65797985667433\\
442	2.66395060678457\\
443	2.66993489291537\\
444	2.67593247402055\\
445	2.68194310851842\\
446	2.68796655430151\\
447	2.6940025687463\\
448	2.70005090872299\\
449	2.70611133060534\\
450	2.71218359028045\\
451	2.71826744315857\\
452	2.724362644183\\
453	2.73046894783993\\
454	2.73658610816834\\
455	2.74271387876991\\
456	2.74885201281892\\
457	2.75500026307224\\
458	2.76115838187923\\
459	2.76732612119178\\
460	2.77350323257424\\
461	2.77968946721346\\
462	2.78588457592882\\
463	2.79208830918224\\
464	2.79830041708825\\
465	2.80452064942405\\
466	2.81074875563959\\
467	2.81698448486766\\
468	2.82322758593398\\
469	2.82947780736738\\
470	2.83573489740982\\
471	2.84199860402665\\
472	2.84826867491667\\
473	2.85454485752234\\
474	2.86082689903993\\
475	2.86711454642975\\
476	2.87340754642625\\
477	2.87970564554833\\
478	2.88600859010946\\
479	2.89231612622796\\
480	2.8986279998372\\
481	2.90494395669582\\
482	2.911263742398\\
483	2.9175871023837\\
484	2.9239137819489\\
485	2.93024352625588\\
486	2.93657608034344\\
487	2.94291118913723\\
488	2.94924859746001\\
489	2.95558805004187\\
490	2.96192929153061\\
491	2.96827206650193\\
492	2.97461611946979\\
493	2.98096119489665\\
494	2.98730703720381\\
495	2.99365339078166\\
496	3\\
};
\end{axis}
\end{tikzpicture}%  % tikz
			\caption{Estimated x-Axis Positions}
		\end{subfigure}
		\begin{subfigure}{0.49\textwidth}
			\centering
			\setlength{\figurewidth}{0.8\textwidth}
			% This file was created by matlab2tikz.
%
\definecolor{lms_red}{rgb}{0.80000,0.20780,0.21960}%
\definecolor{mycolor2}{rgb}{0.80000,0.20784,0.21961}%
\definecolor{mycolor3}{rgb}{0.92900,0.69400,0.12500}%
\definecolor{mycolor4}{rgb}{0.49400,0.18400,0.55600}%
%
\begin{tikzpicture}

\begin{axis}[%
width=0.951\figurewidth,
height=\figureheight,
at={(0\figurewidth,0\figureheight)},
scale only axis,
xmin=0,
xmax=496,
xtick={0,99.2,198.4,297.6,396.8,496},
xticklabels={{0},{1},{2},{3},{4},{5}},
xlabel style={font=\color{white!15!black}},
xlabel={$t$~[s]},
ymin=1,
ymax=5,
ylabel style={font=\color{white!15!black}},
ylabel={$p_y^{(t)}$~[m]},
axis background/.style={fill=white},
xmajorgrids,
ymajorgrids,
legend entries={Est.,
                $s=1$,
                $s=2$},
legend columns=-1,
legend style={%
    at={(1.0,1.0)},
    anchor=south east,
    font=\footnotesize,
    fill opacity=0.0, draw opacity=1, text opacity=1,
    draw=none,
    column sep=0.42cm,
    /tikz/every odd column/.append style={column sep=0.15cm}
},
]
% Estimates
\addlegendimage{color=lms_red, mark=x, only marks, mark options={mark size=4pt, opacity=1, line width=1}}
\addplot [color=mycolor2, draw=none, mark=x, mark options={solid, mycolor2}, forget plot]
  table[row sep=crcr]{%
1	4.5\\
2	1.2\\
3	1.7\\
4	1.9\\
5	2\\
6	2\\
7	2\\
8	2\\
9	2\\
10	2\\
11	2\\
12	2\\
13	2\\
14	2\\
15	2\\
16	2\\
17	2\\
18	3.9\\
19	3.9\\
20	3.9\\
21	3.9\\
22	3.9\\
23	3.9\\
24	3.9\\
25	3.9\\
26	3.9\\
27	2.1\\
28	2\\
29	2\\
30	2.1\\
31	2\\
32	2\\
33	2\\
34	2\\
35	3.8\\
36	3.9\\
37	3.9\\
38	3.9\\
39	3.9\\
40	3.9\\
41	3\\
42	3\\
43	3\\
44	3\\
45	3\\
46	3\\
47	4\\
48	4\\
49	3.9\\
50	3.9\\
51	3.9\\
52	3.9\\
53	3.8\\
54	3.8\\
55	3.8\\
56	3.8\\
57	3.8\\
58	3.9\\
59	3.8\\
60	3.8\\
61	3.8\\
62	3.8\\
63	3.8\\
64	3.8\\
65	3.8\\
66	3.8\\
67	3.8\\
68	3.8\\
69	3.8\\
70	3.8\\
71	1.2\\
72	1.2\\
73	1.2\\
74	1.2\\
75	1.2\\
76	1.2\\
77	1.2\\
78	3.8\\
79	3.8\\
80	3.8\\
81	3.8\\
82	3.8\\
83	3.9\\
84	3.8\\
85	3.8\\
86	3.8\\
87	3.9\\
88	3.9\\
89	3.9\\
90	3.8\\
91	3.8\\
92	3.8\\
93	3.8\\
94	3.8\\
95	3.8\\
96	3.8\\
97	3.8\\
98	3.8\\
99	3.8\\
100	3.8\\
101	3.8\\
102	3.8\\
103	3.8\\
104	3.8\\
105	3.8\\
106	3.8\\
107	3.8\\
108	3.8\\
109	3.8\\
110	3.8\\
111	3.8\\
112	3.8\\
113	3.8\\
114	3.8\\
115	3.8\\
116	3.8\\
117	3.8\\
118	3.8\\
119	3.8\\
120	3.8\\
121	3\\
122	3\\
123	3\\
124	3\\
125	3\\
126	3\\
127	3\\
128	3\\
129	3\\
130	3\\
131	3\\
132	3\\
133	3\\
134	3\\
135	3\\
136	3\\
137	2.2\\
138	2.2\\
139	2.2\\
140	2.2\\
141	2.2\\
142	2.2\\
143	2.2\\
144	2.2\\
145	2.2\\
146	2.2\\
147	2.2\\
148	2.2\\
149	2.2\\
150	2.2\\
151	3.7\\
152	3.7\\
153	3.7\\
154	3.7\\
155	3.7\\
156	3.6\\
157	3.6\\
158	3.6\\
159	3.6\\
160	3.6\\
161	3.6\\
162	3.6\\
163	3.6\\
164	3.6\\
165	3.6\\
166	3.6\\
167	3.6\\
168	3.6\\
169	3.6\\
170	3.6\\
171	3.6\\
172	3.6\\
173	3.6\\
174	3.6\\
175	3.6\\
176	3.5\\
177	3.5\\
178	3.5\\
179	3.5\\
180	3.5\\
181	3.5\\
182	3.5\\
183	3.5\\
184	3.5\\
185	3.5\\
186	3.5\\
187	3.5\\
188	3.5\\
189	3.5\\
190	3.5\\
191	3.5\\
192	3.5\\
193	3.5\\
194	3.5\\
195	3.5\\
196	3.5\\
197	3.4\\
198	3.5\\
199	3.5\\
200	3.5\\
201	3.5\\
202	3.5\\
203	3.5\\
204	3.5\\
205	3.5\\
206	3.5\\
207	3.5\\
208	3.5\\
209	3.5\\
210	3.4\\
211	3.4\\
212	3.4\\
213	3.4\\
214	2.6\\
215	2.6\\
216	2.6\\
217	2.6\\
218	2.6\\
219	2.6\\
220	2.6\\
221	2.6\\
222	2.6\\
223	3.4\\
224	3.4\\
225	3.4\\
226	3.4\\
227	3.4\\
228	3.3\\
229	3.3\\
230	3.3\\
231	3.3\\
232	3.3\\
233	3.3\\
234	3.3\\
235	2.7\\
236	2.7\\
237	2.7\\
238	2.7\\
239	2.7\\
240	2.7\\
241	2.7\\
242	2.7\\
243	2.7\\
244	2.7\\
245	2.7\\
246	2.7\\
247	2.7\\
248	2.7\\
249	2.8\\
250	2.8\\
251	2.8\\
252	2.8\\
253	2.8\\
254	2.8\\
255	2.8\\
256	2.8\\
257	2.8\\
258	2.8\\
259	2.8\\
260	2.8\\
261	2.8\\
262	2.8\\
263	2.8\\
264	2.8\\
265	2.8\\
266	2.8\\
267	2.8\\
268	2.8\\
269	2.8\\
270	2.8\\
271	2.8\\
272	2.8\\
273	2.9\\
274	2.9\\
275	2.9\\
276	2.9\\
277	2.9\\
278	2.9\\
279	2.9\\
280	2.9\\
281	2.9\\
282	2.9\\
283	2.9\\
284	2.9\\
285	2.9\\
286	2.9\\
287	2.9\\
288	2.9\\
289	2.9\\
290	2.9\\
291	2.9\\
292	2.9\\
293	2.9\\
294	2.9\\
295	2.9\\
296	2.9\\
297	2.8\\
298	2.8\\
299	2.3\\
300	2.3\\
301	2.3\\
302	2.3\\
303	2.3\\
304	2.3\\
305	2.3\\
306	2.8\\
307	2.8\\
308	2.8\\
309	2.8\\
310	2.8\\
311	2.8\\
312	2.8\\
313	2.8\\
314	2.8\\
315	2.8\\
316	2.8\\
317	2.8\\
318	2.8\\
319	2.8\\
320	2.8\\
321	2.8\\
322	2.8\\
323	2.8\\
324	2.8\\
325	2.8\\
326	2.8\\
327	2.8\\
328	2.8\\
329	2.8\\
330	2.7\\
331	2.7\\
332	2.7\\
333	2.7\\
334	2.7\\
335	2.7\\
336	2.7\\
337	2.7\\
338	2.7\\
339	2.7\\
340	2.7\\
341	2.7\\
342	2.7\\
343	2.7\\
344	2.7\\
345	2.7\\
346	2.7\\
347	2.7\\
348	2.7\\
349	2.7\\
350	2.7\\
351	2.7\\
352	2.7\\
353	2.7\\
354	2.7\\
355	2.7\\
356	2.7\\
357	2.7\\
358	2.7\\
359	2.7\\
360	2.7\\
361	2.7\\
362	2.7\\
363	2.7\\
364	2.7\\
365	2.7\\
366	2.6\\
367	2.6\\
368	2.6\\
369	2.6\\
370	2.5\\
371	2.5\\
372	2.5\\
373	2.5\\
374	2.5\\
375	2.5\\
376	2.5\\
377	2.5\\
378	2.5\\
379	2.5\\
380	2.5\\
381	2.5\\
382	2.5\\
383	2.5\\
384	2.5\\
385	2.5\\
386	2.5\\
387	2.5\\
388	2.5\\
389	2.5\\
390	2.5\\
391	2.5\\
392	2.5\\
393	2.5\\
394	2.5\\
395	2.5\\
396	2.5\\
397	2.5\\
398	2.5\\
399	2.5\\
400	2.5\\
401	2.5\\
402	2.5\\
403	4.7\\
404	4.7\\
405	4.7\\
406	4.7\\
407	4.7\\
408	4.7\\
409	4.7\\
410	4.7\\
411	4.7\\
412	2.4\\
413	2.3\\
414	2.3\\
415	2.3\\
416	2.3\\
417	2.3\\
418	2.3\\
419	2.3\\
420	2.3\\
421	2.2\\
422	2.2\\
423	2.2\\
424	2.2\\
425	2.2\\
426	2.2\\
427	2.2\\
428	2.2\\
429	2.2\\
430	2.2\\
431	2.2\\
432	2.2\\
433	2.2\\
434	2.2\\
435	2.2\\
436	2.2\\
437	2.2\\
438	2.2\\
439	2.2\\
440	2.2\\
441	2.2\\
442	2.2\\
443	2.2\\
444	2.2\\
445	2.2\\
446	2.2\\
447	2.2\\
448	2.2\\
449	2.2\\
450	2.2\\
451	2.2\\
452	2.2\\
453	2.2\\
454	2.2\\
455	2.2\\
456	2.2\\
457	2.2\\
458	2.2\\
459	2.2\\
460	2.2\\
461	2.2\\
462	2.2\\
463	2.2\\
464	2.2\\
465	2.2\\
466	2.2\\
467	2.2\\
468	2.2\\
469	2.2\\
470	2.2\\
471	2.2\\
472	2.2\\
473	2.2\\
474	2.2\\
475	2.2\\
476	2.2\\
477	2.2\\
478	2.2\\
479	2.2\\
480	2.2\\
481	2.2\\
482	2.2\\
483	2.2\\
484	2.2\\
485	2.2\\
486	2.2\\
487	2.2\\
488	2.2\\
489	2.2\\
490	2.2\\
491	2.2\\
492	2.2\\
493	2.2\\
494	2.2\\
495	2.2\\
496	2.2\\
1	1.3\\
2	2.2\\
3	2.2\\
4	2.4\\
5	1.4\\
6	1.4\\
7	1.4\\
8	2.5\\
9	2.4\\
10	2.4\\
11	2.4\\
12	2.4\\
13	2.4\\
14	2.4\\
15	3.9\\
16	3.9\\
17	3.9\\
18	2\\
19	2\\
20	2\\
21	2\\
22	2\\
23	2\\
24	2\\
25	2\\
26	2\\
27	3.9\\
28	3.9\\
29	3.9\\
30	3.9\\
31	2.4\\
32	2.4\\
33	2.4\\
34	3.9\\
35	2\\
36	2\\
37	2.1\\
38	2\\
39	2.1\\
40	2.1\\
41	3.9\\
42	3.9\\
43	3.9\\
44	2.2\\
45	3.9\\
46	4\\
47	3\\
48	3\\
49	3\\
50	3\\
51	3\\
52	3\\
53	4.4\\
54	4.4\\
55	4.4\\
56	2.1\\
57	2\\
58	2\\
59	2\\
60	2\\
61	2\\
62	2\\
63	2\\
64	2\\
65	2\\
66	2\\
67	2\\
68	2\\
69	2\\
70	2\\
71	3.8\\
72	2\\
73	2\\
74	2\\
75	2\\
76	2\\
77	3.7\\
78	1.2\\
79	3\\
80	3\\
81	3\\
82	3\\
83	3\\
84	3\\
85	3.5\\
86	4.4\\
87	3.6\\
88	3.6\\
89	3.6\\
90	1.2\\
91	3.7\\
92	1.2\\
93	1.2\\
94	1.2\\
95	1.2\\
96	1.2\\
97	1.2\\
98	1.2\\
99	1.2\\
100	1.2\\
101	1.2\\
102	1.2\\
103	3\\
104	3\\
105	1.2\\
106	1.2\\
107	1.2\\
108	1.2\\
109	3\\
110	3\\
111	3\\
112	3\\
113	3\\
114	3\\
115	3\\
116	3\\
117	4.7\\
118	3\\
119	3\\
120	3\\
121	3.8\\
122	3.8\\
123	3.8\\
124	3.8\\
125	3.8\\
126	3.8\\
127	3.8\\
128	3.8\\
129	2.2\\
130	2.2\\
131	2.2\\
132	3.8\\
133	3.8\\
134	2.2\\
135	2.2\\
136	2.2\\
137	3\\
138	3\\
139	3\\
140	1.2\\
141	1.2\\
142	1.2\\
143	2.9\\
144	2.9\\
145	2.9\\
146	2.9\\
147	2.9\\
148	2.9\\
149	2.9\\
150	3.8\\
151	2.2\\
152	2.2\\
153	2.2\\
154	2.2\\
155	2.2\\
156	2.2\\
157	2.9\\
158	2.9\\
159	2.9\\
160	2.9\\
161	2.9\\
162	2.9\\
163	2.9\\
164	2.2\\
165	2.2\\
166	1.2\\
167	1.2\\
168	1.2\\
169	2.9\\
170	2.4\\
171	2.4\\
172	2.5\\
173	2.5\\
174	2.5\\
175	2.5\\
176	2.5\\
177	2.5\\
178	2.5\\
179	2.5\\
180	2.5\\
181	2.5\\
182	2.5\\
183	2.5\\
184	2.5\\
185	2.5\\
186	2.5\\
187	2.5\\
188	2.5\\
189	2.5\\
190	2.5\\
191	2.5\\
192	2.5\\
193	3\\
194	3\\
195	3\\
196	3\\
197	3\\
198	3\\
199	3\\
200	3\\
201	3\\
202	3\\
203	3\\
204	1.2\\
205	1.2\\
206	1.2\\
207	2.3\\
208	2.6\\
209	2.6\\
210	2.6\\
211	2.6\\
212	2.6\\
213	2.6\\
214	3.4\\
215	3.4\\
216	3.4\\
217	3.4\\
218	3.4\\
219	3.4\\
220	3.4\\
221	3.4\\
222	3.4\\
223	2.6\\
224	2.6\\
225	2.6\\
226	2.6\\
227	2.6\\
228	2.6\\
229	2.6\\
230	2.6\\
231	2.6\\
232	2.6\\
233	2.6\\
234	2.7\\
235	3.4\\
236	3.4\\
237	3.4\\
238	3.4\\
239	3.3\\
240	3.3\\
241	3.3\\
242	3.3\\
243	3.3\\
244	3.3\\
245	3.3\\
246	3.3\\
247	3.3\\
248	3.3\\
249	3.1\\
250	3.1\\
251	3.1\\
252	3.3\\
253	3\\
254	3\\
255	3.1\\
256	3.2\\
257	3.2\\
258	3.2\\
259	3.2\\
260	3.2\\
261	3.2\\
262	3.2\\
263	3.2\\
264	2.3\\
265	2.3\\
266	2.3\\
267	2.3\\
268	2.3\\
269	2.3\\
270	2.3\\
271	2.9\\
272	2.9\\
273	2.8\\
274	2.8\\
275	2.8\\
276	2.8\\
277	2.8\\
278	2.8\\
279	2.8\\
280	2.8\\
281	2.8\\
282	2.8\\
283	2.8\\
284	2.8\\
285	1.2\\
286	1.2\\
287	1.2\\
288	2.8\\
289	2.3\\
290	1.2\\
291	1.2\\
292	1.2\\
293	1.2\\
294	1.2\\
295	1.2\\
296	1.2\\
297	1.2\\
298	2.3\\
299	2.8\\
300	2.8\\
301	1.2\\
302	2.8\\
303	2.8\\
304	2.8\\
305	1.2\\
306	2.3\\
307	2.3\\
308	2.3\\
309	2.3\\
310	2.3\\
311	2.3\\
312	1.2\\
313	1.2\\
314	1.2\\
315	1.2\\
316	1.2\\
317	1.2\\
318	1.2\\
319	1.2\\
320	1.2\\
321	1.2\\
322	1.2\\
323	1.2\\
324	1.2\\
325	1.2\\
326	1.2\\
327	1.2\\
328	1.2\\
329	1.2\\
330	1.2\\
331	1.2\\
332	1.2\\
333	1.2\\
334	1.2\\
335	1.2\\
336	1.2\\
337	1.2\\
338	1.2\\
339	1.2\\
340	1.2\\
341	1.2\\
342	1.2\\
343	1.2\\
344	1.2\\
345	1.2\\
346	1.2\\
347	1.2\\
348	1.2\\
349	1.2\\
350	1.2\\
351	1.2\\
352	1.2\\
353	1.2\\
354	3.4\\
355	3.4\\
356	3.4\\
357	3.4\\
358	3.4\\
359	3.4\\
360	3.4\\
361	3.4\\
362	3.4\\
363	3.4\\
364	3.4\\
365	3.4\\
366	1.2\\
367	1.2\\
368	2.3\\
369	2.3\\
370	2.3\\
371	1.2\\
372	1.2\\
373	1.2\\
374	1.2\\
375	3.4\\
376	3.8\\
377	3.8\\
378	3.8\\
379	3.7\\
380	3.7\\
381	3.7\\
382	3.7\\
383	3.7\\
384	3.7\\
385	3.8\\
386	3.7\\
387	3.7\\
388	3.7\\
389	3.8\\
390	3.8\\
391	3.8\\
392	3.8\\
393	3.8\\
394	3.8\\
395	3.8\\
396	3.8\\
397	3.8\\
398	3.8\\
399	3.8\\
400	3.7\\
401	3.8\\
402	4.7\\
403	2.5\\
404	2.5\\
405	2.5\\
406	2.5\\
407	2.5\\
408	2.4\\
409	2.4\\
410	2.4\\
411	2.4\\
412	4.7\\
413	4.7\\
414	4.7\\
415	4.7\\
416	4.7\\
417	4.7\\
418	4.7\\
419	4.7\\
420	4.7\\
421	4.7\\
422	4.7\\
423	4.7\\
424	4.7\\
425	4.7\\
426	4.7\\
427	4.7\\
428	4.7\\
429	4.7\\
430	4.7\\
431	4.7\\
432	4.7\\
433	4.7\\
434	2.7\\
435	2.7\\
436	2.7\\
437	2.7\\
438	4.7\\
439	4.7\\
440	4.7\\
441	4.7\\
442	4.7\\
443	4.7\\
444	4.7\\
445	4.7\\
446	4.7\\
447	4.7\\
448	4.7\\
449	4.7\\
450	4.7\\
451	4.7\\
452	3.8\\
453	4.7\\
454	4.7\\
455	4.7\\
456	4.7\\
457	4.7\\
458	4.7\\
459	4.7\\
460	4.7\\
461	4.7\\
462	4.7\\
463	4.7\\
464	4.7\\
465	4.7\\
466	3.8\\
467	4.7\\
468	4.7\\
469	4.7\\
470	4.7\\
471	4.7\\
472	4.7\\
473	4.7\\
474	4.7\\
475	4.7\\
476	4.7\\
477	4.7\\
478	3.9\\
479	3.9\\
480	3.9\\
481	3.9\\
482	3.9\\
483	3.9\\
484	3.9\\
485	3.9\\
486	3.9\\
487	3.9\\
488	3.9\\
489	3.9\\
490	3.8\\
491	3.8\\
492	3.8\\
493	3.8\\
494	3.8\\
495	3.7\\
496	3.7\\
};
\addplot [color=mycolor3, dashed, line width=\trajDashedLinewidth]
  table[row sep=crcr]{%
1	2\\
2	2.00002013992709\\
3	2.00008055889714\\
4	2.00018125447647\\
5	2.00032222260909\\
6	2.00050345761681\\
7	2.00072495219953\\
8	2.00098669743546\\
9	2.00128868278154\\
10	2.00163089607386\\
11	2.00201332352812\\
12	2.00243594974018\\
13	2.00289875768672\\
14	2.00340172872592\\
15	2.00394484259817\\
16	2.00452807742692\\
17	2.00515140971955\\
18	2.00581481436835\\
19	2.00651826465145\\
20	2.00726173223399\\
21	2.0080451871692\\
22	2.00886859789964\\
23	2.00973193125843\\
24	2.01063515247064\\
25	2.01157822515464\\
26	2.01256111132361\\
27	2.01358377138703\\
28	2.01464616415231\\
29	2.01574824682642\\
30	2.01688997501763\\
31	2.01807130273729\\
32	2.01929218240171\\
33	2.02055256483401\\
34	2.02185239926619\\
35	2.0231916333411\\
36	2.02457021311459\\
37	2.02598808305767\\
38	2.02744518605873\\
39	2.02894146342589\\
40	2.03047685488931\\
41	2.03205129860364\\
42	2.03366473115053\\
43	2.03531708754114\\
44	2.03700830121881\\
45	2.03873830406168\\
46	2.0405070263855\\
47	2.04231439694639\\
48	2.04416034294373\\
49	2.04604479002309\\
50	2.0479676622792\\
51	2.04992888225905\\
52	2.051928370965\\
53	2.05396604785792\\
54	2.05604183086049\\
55	2.05815563636048\\
56	2.06030737921409\\
57	2.06249697274945\\
58	2.06472432877005\\
59	2.06698935755831\\
60	2.06929196787921\\
61	2.07163206698393\\
62	2.07400956061362\\
63	2.07642435300319\\
64	2.07887634688515\\
65	2.08136544349355\\
66	2.08389154256793\\
67	2.0864545423574\\
68	2.0890543396247\\
69	2.09169082965036\\
70	2.09436390623697\\
71	2.09707346171338\\
72	2.09981938693909\\
73	2.10260157130864\\
74	2.10541990275605\\
75	2.10827426775933\\
76	2.11116455134508\\
77	2.11409063709309\\
78	2.11705240714107\\
79	2.12004974218935\\
80	2.1230825215057\\
81	2.12615062293021\\
82	2.12925392288022\\
83	2.13239229635525\\
84	2.13556561694207\\
85	2.1387737568198\\
86	2.14201658676502\\
87	2.14529397615703\\
88	2.14860579298304\\
89	2.15195190384357\\
90	2.15533217395776\\
91	2.15874646716882\\
92	2.16219464594951\\
93	2.1656765714077\\
94	2.16919210329195\\
95	2.17274109999712\\
96	2.17632341857017\\
97	2.17993891471581\\
98	2.18358744280239\\
99	2.18726885586773\\
100	2.19098300562505\\
101	2.19472974246894\\
102	2.19850891548138\\
103	2.20232037243784\\
104	2.20616395981339\\
105	2.21003952278888\\
106	2.21394690525721\\
107	2.21788594982958\\
108	2.22185649784185\\
109	2.22585838936092\\
110	2.22989146319118\\
111	2.23395555688102\\
112	2.23805050672933\\
113	2.24217614779212\\
114	2.24633231388918\\
115	2.25051883761075\\
116	2.25473555032425\\
117	2.25898228218111\\
118	2.2632588621236\\
119	2.26756511789169\\
120	2.27190087603004\\
121	2.27626596189493\\
122	2.28066019966135\\
123	2.28508341233004\\
124	2.28953542173463\\
125	2.29401604854884\\
126	2.29852511229368\\
127	2.30306243134471\\
128	2.30762782293938\\
129	2.31222110318438\\
130	2.31684208706307\\
131	2.32149058844287\\
132	2.32616642008283\\
133	2.33086939364114\\
134	2.33559931968271\\
135	2.3403560076868\\
136	2.34513926605471\\
137	2.34994890211751\\
138	2.35478472214374\\
139	2.35964653134728\\
140	2.36453413389516\\
141	2.36944733291548\\
142	2.37438593050528\\
143	2.37934972773858\\
144	2.38433852467434\\
145	2.38935212036456\\
146	2.39439031286233\\
147	2.39945289923\\
148	2.40453967554731\\
149	2.40965043691967\\
150	2.41478497748635\\
151	2.4199430904288\\
152	2.42512456797899\\
153	2.43032920142776\\
154	2.43555678113323\\
155	2.44080709652925\\
156	2.44607993613389\\
157	2.45137508755793\\
158	2.45669233751345\\
159	2.46203147182239\\
160	2.4673922754252\\
161	2.4727745323895\\
162	2.47817802591875\\
163	2.48360253836104\\
164	2.48904785121778\\
165	2.49451374515257\\
166	2.5\\
167	2.50550639477452\\
168	2.51103270767936\\
169	2.51657871611544\\
170	2.52214419669034\\
171	2.52772892522732\\
172	2.53333267677433\\
173	2.53895522561307\\
174	2.5445963452681\\
175	2.55025580851593\\
176	2.55593338739423\\
177	2.56162885321092\\
178	2.56734197655349\\
179	2.57307252729816\\
180	2.57882027461917\\
181	2.58458498699811\\
182	2.5903664322332\\
183	2.59616437744868\\
184	2.60197858910414\\
185	2.60780883300399\\
186	2.61365487430687\\
187	2.61951647753508\\
188	2.62539340658409\\
189	2.63128542473206\\
190	2.63719229464936\\
191	2.64311377840813\\
192	2.64904963749186\\
193	2.65499963280503\\
194	2.66096352468268\\
195	2.66694107290012\\
196	2.67293203668258\\
197	2.67893617471491\\
198	2.68495324515131\\
199	2.69098300562505\\
200	2.69702521325827\\
201	2.70307962467173\\
202	2.7091459959946\\
203	2.71522408287435\\
204	2.72131364048651\\
205	2.7274144235446\\
206	2.73352618630997\\
207	2.73964868260169\\
208	2.74578166580651\\
209	2.75192488888877\\
210	2.75807810440033\\
211	2.76424106449057\\
212	2.77041352091636\\
213	2.77659522505205\\
214	2.78278592789949\\
215	2.78898538009809\\
216	2.79519333193481\\
217	2.80140953335425\\
218	2.80763373396874\\
219	2.81386568306837\\
220	2.82010512963115\\
221	2.82635182233307\\
222	2.83260550955827\\
223	2.83886593940914\\
224	2.84513285971647\\
225	2.85140601804963\\
226	2.85768516172671\\
227	2.86397003782474\\
228	2.8702603931898\\
229	2.87655597444731\\
230	2.88285652801216\\
231	2.88916180009899\\
232	2.89547153673235\\
233	2.90178548375696\\
234	2.90810338684797\\
235	2.91442499152116\\
236	2.92075004314321\\
237	2.92707828694197\\
238	2.9334094680167\\
239	2.93974333134834\\
240	2.94607962180981\\
241	2.95241808417626\\
242	2.95875846313533\\
243	2.9651005032975\\
244	2.9714439492063\\
245	2.97778854534867\\
246	2.98413403616519\\
247	2.99048016606042\\
248	2.99682667941318\\
249	3.00317332058682\\
250	3.00951983393958\\
251	3.01586596383481\\
252	3.02221145465133\\
253	3.0285560507937\\
254	3.0348994967025\\
255	3.04124153686467\\
256	3.04758191582374\\
257	3.05392037819019\\
258	3.06025666865166\\
259	3.0665905319833\\
260	3.07292171305803\\
261	3.07924995685679\\
262	3.08557500847884\\
263	3.09189661315203\\
264	3.09821451624304\\
265	3.10452846326765\\
266	3.11083819990101\\
267	3.11714347198784\\
268	3.12344402555269\\
269	3.1297396068102\\
270	3.13602996217526\\
271	3.14231483827329\\
272	3.14859398195037\\
273	3.15486714028353\\
274	3.16113406059086\\
275	3.16739449044173\\
276	3.17364817766693\\
277	3.17989487036885\\
278	3.18613431693163\\
279	3.19236626603126\\
280	3.19859046664575\\
281	3.20480666806519\\
282	3.21101461990191\\
283	3.21721407210051\\
284	3.22340477494795\\
285	3.22958647908364\\
286	3.23575893550943\\
287	3.24192189559967\\
288	3.24807511111123\\
289	3.25421833419349\\
290	3.26035131739831\\
291	3.26647381369003\\
292	3.2725855764554\\
293	3.27868635951349\\
294	3.28477591712565\\
295	3.2908540040054\\
296	3.29692037532827\\
297	3.30297478674173\\
298	3.30901699437495\\
299	3.31504675484869\\
300	3.32106382528509\\
301	3.32706796331742\\
302	3.33305892709988\\
303	3.33903647531732\\
304	3.34500036719497\\
305	3.35095036250814\\
306	3.35688622159187\\
307	3.36280770535064\\
308	3.36871457526794\\
309	3.37460659341591\\
310	3.38048352246492\\
311	3.38634512569313\\
312	3.39219116699601\\
313	3.39802141089586\\
314	3.40383562255132\\
315	3.4096335677668\\
316	3.41541501300189\\
317	3.42117972538083\\
318	3.42692747270184\\
319	3.43265802344651\\
320	3.43837114678908\\
321	3.44406661260577\\
322	3.44974419148407\\
323	3.4554036547319\\
324	3.46104477438693\\
325	3.46666732322567\\
326	3.47227107477268\\
327	3.47785580330966\\
328	3.48342128388456\\
329	3.48896729232064\\
330	3.49449360522548\\
331	3.5\\
332	3.50548625484743\\
333	3.51095214878222\\
334	3.51639746163896\\
335	3.52182197408125\\
336	3.5272254676105\\
337	3.5326077245748\\
338	3.53796852817761\\
339	3.54330766248655\\
340	3.54862491244207\\
341	3.55392006386611\\
342	3.55919290347075\\
343	3.56444321886677\\
344	3.56967079857224\\
345	3.57487543202101\\
346	3.5800569095712\\
347	3.58521502251365\\
348	3.59034956308033\\
349	3.59546032445269\\
350	3.60054710077\\
351	3.60560968713767\\
352	3.61064787963544\\
353	3.61566147532566\\
354	3.62065027226142\\
355	3.62561406949472\\
356	3.63055266708452\\
357	3.63546586610484\\
358	3.64035346865272\\
359	3.64521527785626\\
360	3.65005109788249\\
361	3.65486073394529\\
362	3.6596439923132\\
363	3.66440068031729\\
364	3.66913060635886\\
365	3.67383357991717\\
366	3.67850941155713\\
367	3.68315791293693\\
368	3.68777889681562\\
369	3.69237217706062\\
370	3.69693756865529\\
371	3.70147488770632\\
372	3.70598395145116\\
373	3.71046457826537\\
374	3.71491658766996\\
375	3.71933980033865\\
376	3.72373403810507\\
377	3.72809912396996\\
378	3.73243488210831\\
379	3.7367411378764\\
380	3.74101771781889\\
381	3.74526444967575\\
382	3.74948116238925\\
383	3.75366768611082\\
384	3.75782385220788\\
385	3.76194949327067\\
386	3.76604444311898\\
387	3.77010853680882\\
388	3.77414161063908\\
389	3.77814350215815\\
390	3.78211405017042\\
391	3.78605309474279\\
392	3.78996047721112\\
393	3.79383604018661\\
394	3.79767962756216\\
395	3.80149108451862\\
396	3.80527025753106\\
397	3.80901699437495\\
398	3.81273114413227\\
399	3.81641255719761\\
400	3.82006108528419\\
401	3.82367658142983\\
402	3.82725890000288\\
403	3.83080789670805\\
404	3.83432342859229\\
405	3.83780535405049\\
406	3.84125353283118\\
407	3.84466782604224\\
408	3.84804809615643\\
409	3.85139420701696\\
410	3.85470602384297\\
411	3.85798341323498\\
412	3.8612262431802\\
413	3.86443438305793\\
414	3.86760770364475\\
415	3.87074607711978\\
416	3.87384937706979\\
417	3.8769174784943\\
418	3.87995025781065\\
419	3.88294759285893\\
420	3.88590936290691\\
421	3.88883544865492\\
422	3.89172573224067\\
423	3.89458009724395\\
424	3.89739842869136\\
425	3.90018061306091\\
426	3.90292653828662\\
427	3.90563609376303\\
428	3.90830917034964\\
429	3.9109456603753\\
430	3.9135454576426\\
431	3.91610845743207\\
432	3.91863455650645\\
433	3.92112365311485\\
434	3.92357564699681\\
435	3.92599043938638\\
436	3.92836793301607\\
437	3.93070803212079\\
438	3.93301064244169\\
439	3.93527567122995\\
440	3.93750302725055\\
441	3.93969262078591\\
442	3.94184436363952\\
443	3.94395816913951\\
444	3.94603395214208\\
445	3.948071629035\\
446	3.95007111774095\\
447	3.9520323377208\\
448	3.95395520997691\\
449	3.95583965705627\\
450	3.95768560305361\\
451	3.9594929736145\\
452	3.96126169593832\\
453	3.96299169878119\\
454	3.96468291245886\\
455	3.96633526884947\\
456	3.96794870139636\\
457	3.96952314511069\\
458	3.97105853657411\\
459	3.97255481394127\\
460	3.97401191694233\\
461	3.97542978688541\\
462	3.9768083666589\\
463	3.97814760073381\\
464	3.97944743516599\\
465	3.98070781759829\\
466	3.98192869726271\\
467	3.98311002498237\\
468	3.98425175317358\\
469	3.98535383584769\\
470	3.98641622861297\\
471	3.98743888867639\\
472	3.98842177484536\\
473	3.98936484752936\\
474	3.99026806874157\\
475	3.99113140210036\\
476	3.9919548128308\\
477	3.99273826776601\\
478	3.99348173534855\\
479	3.99418518563165\\
480	3.99484859028045\\
481	3.99547192257308\\
482	3.99605515740183\\
483	3.99659827127408\\
484	3.99710124231328\\
485	3.99756405025982\\
486	3.99798667647188\\
487	3.99836910392614\\
488	3.99871131721846\\
489	3.99901330256454\\
490	3.99927504780047\\
491	3.99949654238319\\
492	3.99967777739091\\
493	3.99981874552353\\
494	3.99991944110286\\
495	3.99997986007291\\
496	4\\
};

\addplot [color=mycolor4, dashed, line width=\trajDashedLinewidth]
  table[row sep=crcr]{%
1	4\\
2	3.99997986007291\\
3	3.99991944110286\\
4	3.99981874552353\\
5	3.99967777739091\\
6	3.99949654238319\\
7	3.99927504780047\\
8	3.99901330256454\\
9	3.99871131721846\\
10	3.99836910392614\\
11	3.99798667647188\\
12	3.99756405025982\\
13	3.99710124231328\\
14	3.99659827127408\\
15	3.99605515740183\\
16	3.99547192257308\\
17	3.99484859028045\\
18	3.99418518563165\\
19	3.99348173534855\\
20	3.99273826776601\\
21	3.9919548128308\\
22	3.99113140210036\\
23	3.99026806874157\\
24	3.98936484752936\\
25	3.98842177484536\\
26	3.98743888867639\\
27	3.98641622861297\\
28	3.98535383584769\\
29	3.98425175317358\\
30	3.98311002498237\\
31	3.98192869726271\\
32	3.98070781759829\\
33	3.97944743516599\\
34	3.97814760073381\\
35	3.9768083666589\\
36	3.97542978688541\\
37	3.97401191694233\\
38	3.97255481394127\\
39	3.97105853657411\\
40	3.96952314511069\\
41	3.96794870139636\\
42	3.96633526884947\\
43	3.96468291245886\\
44	3.96299169878119\\
45	3.96126169593832\\
46	3.9594929736145\\
47	3.9576856030536\\
48	3.95583965705627\\
49	3.95395520997691\\
50	3.9520323377208\\
51	3.95007111774095\\
52	3.948071629035\\
53	3.94603395214208\\
54	3.94395816913951\\
55	3.94184436363952\\
56	3.93969262078591\\
57	3.93750302725055\\
58	3.93527567122995\\
59	3.93301064244169\\
60	3.93070803212079\\
61	3.92836793301607\\
62	3.92599043938638\\
63	3.92357564699681\\
64	3.92112365311485\\
65	3.91863455650645\\
66	3.91610845743207\\
67	3.9135454576426\\
68	3.9109456603753\\
69	3.90830917034964\\
70	3.90563609376303\\
71	3.90292653828662\\
72	3.90018061306091\\
73	3.89739842869136\\
74	3.89458009724395\\
75	3.89172573224067\\
76	3.88883544865492\\
77	3.88590936290691\\
78	3.88294759285893\\
79	3.87995025781065\\
80	3.8769174784943\\
81	3.87384937706979\\
82	3.87074607711978\\
83	3.86760770364475\\
84	3.86443438305793\\
85	3.8612262431802\\
86	3.85798341323498\\
87	3.85470602384297\\
88	3.85139420701696\\
89	3.84804809615643\\
90	3.84466782604224\\
91	3.84125353283118\\
92	3.83780535405049\\
93	3.83432342859229\\
94	3.83080789670805\\
95	3.82725890000288\\
96	3.82367658142983\\
97	3.82006108528419\\
98	3.81641255719761\\
99	3.81273114413227\\
100	3.80901699437495\\
101	3.80527025753106\\
102	3.80149108451862\\
103	3.79767962756216\\
104	3.79383604018661\\
105	3.78996047721112\\
106	3.78605309474279\\
107	3.78211405017042\\
108	3.77814350215815\\
109	3.77414161063908\\
110	3.77010853680882\\
111	3.76604444311898\\
112	3.76194949327067\\
113	3.75782385220788\\
114	3.75366768611082\\
115	3.74948116238925\\
116	3.74526444967575\\
117	3.74101771781889\\
118	3.7367411378764\\
119	3.73243488210831\\
120	3.72809912396996\\
121	3.72373403810507\\
122	3.71933980033865\\
123	3.71491658766996\\
124	3.71046457826537\\
125	3.70598395145116\\
126	3.70147488770632\\
127	3.69693756865529\\
128	3.69237217706062\\
129	3.68777889681562\\
130	3.68315791293693\\
131	3.67850941155713\\
132	3.67383357991717\\
133	3.66913060635886\\
134	3.66440068031729\\
135	3.6596439923132\\
136	3.65486073394529\\
137	3.65005109788249\\
138	3.64521527785626\\
139	3.64035346865272\\
140	3.63546586610484\\
141	3.63055266708452\\
142	3.62561406949472\\
143	3.62065027226142\\
144	3.61566147532566\\
145	3.61064787963544\\
146	3.60560968713767\\
147	3.60054710077\\
148	3.59546032445269\\
149	3.59034956308033\\
150	3.58521502251365\\
151	3.5800569095712\\
152	3.57487543202101\\
153	3.56967079857224\\
154	3.56444321886677\\
155	3.55919290347075\\
156	3.55392006386611\\
157	3.54862491244207\\
158	3.54330766248655\\
159	3.53796852817761\\
160	3.5326077245748\\
161	3.5272254676105\\
162	3.52182197408125\\
163	3.51639746163896\\
164	3.51095214878222\\
165	3.50548625484743\\
166	3.5\\
167	3.49449360522548\\
168	3.48896729232064\\
169	3.48342128388456\\
170	3.47785580330966\\
171	3.47227107477268\\
172	3.46666732322567\\
173	3.46104477438693\\
174	3.4554036547319\\
175	3.44974419148407\\
176	3.44406661260577\\
177	3.43837114678908\\
178	3.43265802344651\\
179	3.42692747270184\\
180	3.42117972538083\\
181	3.41541501300189\\
182	3.4096335677668\\
183	3.40383562255132\\
184	3.39802141089586\\
185	3.39219116699601\\
186	3.38634512569313\\
187	3.38048352246492\\
188	3.37460659341591\\
189	3.36871457526794\\
190	3.36280770535064\\
191	3.35688622159187\\
192	3.35095036250814\\
193	3.34500036719497\\
194	3.33903647531732\\
195	3.33305892709988\\
196	3.32706796331742\\
197	3.32106382528509\\
198	3.31504675484869\\
199	3.30901699437495\\
200	3.30297478674173\\
201	3.29692037532828\\
202	3.2908540040054\\
203	3.28477591712565\\
204	3.27868635951349\\
205	3.2725855764554\\
206	3.26647381369004\\
207	3.26035131739831\\
208	3.25421833419349\\
209	3.24807511111123\\
210	3.24192189559967\\
211	3.23575893550943\\
212	3.22958647908364\\
213	3.22340477494795\\
214	3.21721407210051\\
215	3.21101461990191\\
216	3.20480666806519\\
217	3.19859046664575\\
218	3.19236626603126\\
219	3.18613431693163\\
220	3.17989487036885\\
221	3.17364817766693\\
222	3.16739449044173\\
223	3.16113406059086\\
224	3.15486714028353\\
225	3.14859398195037\\
226	3.14231483827328\\
227	3.13602996217526\\
228	3.1297396068102\\
229	3.12344402555269\\
230	3.11714347198784\\
231	3.11083819990101\\
232	3.10452846326765\\
233	3.09821451624304\\
234	3.09189661315203\\
235	3.08557500847884\\
236	3.07924995685679\\
237	3.07292171305803\\
238	3.0665905319833\\
239	3.06025666865166\\
240	3.05392037819019\\
241	3.04758191582374\\
242	3.04124153686467\\
243	3.0348994967025\\
244	3.0285560507937\\
245	3.02221145465133\\
246	3.01586596383481\\
247	3.00951983393958\\
248	3.00317332058682\\
249	2.99682667941318\\
250	2.99048016606042\\
251	2.98413403616519\\
252	2.97778854534867\\
253	2.9714439492063\\
254	2.9651005032975\\
255	2.95875846313533\\
256	2.95241808417626\\
257	2.94607962180981\\
258	2.93974333134834\\
259	2.9334094680167\\
260	2.92707828694197\\
261	2.92075004314321\\
262	2.91442499152116\\
263	2.90810338684797\\
264	2.90178548375696\\
265	2.89547153673235\\
266	2.88916180009899\\
267	2.88285652801216\\
268	2.87655597444731\\
269	2.8702603931898\\
270	2.86397003782474\\
271	2.85768516172671\\
272	2.85140601804963\\
273	2.84513285971647\\
274	2.83886593940914\\
275	2.83260550955827\\
276	2.82635182233307\\
277	2.82010512963115\\
278	2.81386568306837\\
279	2.80763373396874\\
280	2.80140953335425\\
281	2.79519333193481\\
282	2.78898538009809\\
283	2.78278592789949\\
284	2.77659522505205\\
285	2.77041352091636\\
286	2.76424106449057\\
287	2.75807810440033\\
288	2.75192488888877\\
289	2.74578166580651\\
290	2.73964868260169\\
291	2.73352618630997\\
292	2.7274144235446\\
293	2.72131364048651\\
294	2.71522408287435\\
295	2.7091459959946\\
296	2.70307962467172\\
297	2.69702521325827\\
298	2.69098300562505\\
299	2.68495324515131\\
300	2.67893617471491\\
301	2.67293203668258\\
302	2.66694107290012\\
303	2.66096352468268\\
304	2.65499963280503\\
305	2.64904963749186\\
306	2.64311377840813\\
307	2.63719229464936\\
308	2.63128542473206\\
309	2.62539340658409\\
310	2.61951647753508\\
311	2.61365487430687\\
312	2.60780883300399\\
313	2.60197858910414\\
314	2.59616437744867\\
315	2.5903664322332\\
316	2.58458498699811\\
317	2.57882027461917\\
318	2.57307252729816\\
319	2.56734197655349\\
320	2.56162885321092\\
321	2.55593338739423\\
322	2.55025580851593\\
323	2.5445963452681\\
324	2.53895522561307\\
325	2.53333267677433\\
326	2.52772892522732\\
327	2.52214419669034\\
328	2.51657871611544\\
329	2.51103270767936\\
330	2.50550639477452\\
331	2.5\\
332	2.49451374515257\\
333	2.48904785121778\\
334	2.48360253836104\\
335	2.47817802591875\\
336	2.4727745323895\\
337	2.4673922754252\\
338	2.46203147182239\\
339	2.45669233751345\\
340	2.45137508755793\\
341	2.44607993613389\\
342	2.44080709652925\\
343	2.43555678113323\\
344	2.43032920142776\\
345	2.42512456797899\\
346	2.4199430904288\\
347	2.41478497748635\\
348	2.40965043691967\\
349	2.40453967554731\\
350	2.39945289923\\
351	2.39439031286233\\
352	2.38935212036456\\
353	2.38433852467434\\
354	2.37934972773858\\
355	2.37438593050528\\
356	2.36944733291548\\
357	2.36453413389516\\
358	2.35964653134728\\
359	2.35478472214374\\
360	2.34994890211751\\
361	2.34513926605471\\
362	2.3403560076868\\
363	2.33559931968271\\
364	2.33086939364114\\
365	2.32616642008283\\
366	2.32149058844287\\
367	2.31684208706307\\
368	2.31222110318438\\
369	2.30762782293938\\
370	2.30306243134471\\
371	2.29852511229368\\
372	2.29401604854884\\
373	2.28953542173463\\
374	2.28508341233004\\
375	2.28066019966135\\
376	2.27626596189493\\
377	2.27190087603004\\
378	2.26756511789169\\
379	2.26325886212359\\
380	2.25898228218111\\
381	2.25473555032424\\
382	2.25051883761075\\
383	2.24633231388918\\
384	2.24217614779212\\
385	2.23805050672933\\
386	2.23395555688102\\
387	2.22989146319118\\
388	2.22585838936092\\
389	2.22185649784185\\
390	2.21788594982958\\
391	2.21394690525721\\
392	2.21003952278888\\
393	2.20616395981339\\
394	2.20232037243784\\
395	2.19850891548138\\
396	2.19472974246894\\
397	2.19098300562505\\
398	2.18726885586773\\
399	2.18358744280239\\
400	2.17993891471581\\
401	2.17632341857017\\
402	2.17274109999712\\
403	2.16919210329195\\
404	2.16567657140771\\
405	2.16219464594951\\
406	2.15874646716882\\
407	2.15533217395776\\
408	2.15195190384357\\
409	2.14860579298304\\
410	2.14529397615703\\
411	2.14201658676502\\
412	2.1387737568198\\
413	2.13556561694207\\
414	2.13239229635525\\
415	2.12925392288022\\
416	2.12615062293021\\
417	2.1230825215057\\
418	2.12004974218935\\
419	2.11705240714107\\
420	2.11409063709309\\
421	2.11116455134508\\
422	2.10827426775933\\
423	2.10541990275605\\
424	2.10260157130864\\
425	2.09981938693909\\
426	2.09707346171338\\
427	2.09436390623697\\
428	2.09169082965036\\
429	2.0890543396247\\
430	2.0864545423574\\
431	2.08389154256793\\
432	2.08136544349355\\
433	2.07887634688515\\
434	2.07642435300319\\
435	2.07400956061362\\
436	2.07163206698393\\
437	2.06929196787921\\
438	2.06698935755831\\
439	2.06472432877005\\
440	2.06249697274945\\
441	2.06030737921409\\
442	2.05815563636048\\
443	2.05604183086049\\
444	2.05396604785792\\
445	2.051928370965\\
446	2.04992888225905\\
447	2.0479676622792\\
448	2.04604479002309\\
449	2.04416034294373\\
450	2.04231439694639\\
451	2.0405070263855\\
452	2.03873830406168\\
453	2.03700830121881\\
454	2.03531708754114\\
455	2.03366473115053\\
456	2.03205129860364\\
457	2.03047685488931\\
458	2.02894146342589\\
459	2.02744518605873\\
460	2.02598808305767\\
461	2.02457021311459\\
462	2.0231916333411\\
463	2.02185239926619\\
464	2.02055256483401\\
465	2.01929218240171\\
466	2.01807130273729\\
467	2.01688997501763\\
468	2.01574824682642\\
469	2.01464616415231\\
470	2.01358377138703\\
471	2.01256111132361\\
472	2.01157822515464\\
473	2.01063515247064\\
474	2.00973193125843\\
475	2.00886859789964\\
476	2.0080451871692\\
477	2.00726173223399\\
478	2.00651826465145\\
479	2.00581481436835\\
480	2.00515140971955\\
481	2.00452807742692\\
482	2.00394484259817\\
483	2.00340172872592\\
484	2.00289875768672\\
485	2.00243594974018\\
486	2.00201332352812\\
487	2.00163089607386\\
488	2.00128868278154\\
489	2.00098669743546\\
490	2.00072495219953\\
491	2.00050345761681\\
492	2.00032222260909\\
493	2.00018125447647\\
494	2.00008055889714\\
495	2.00002013992709\\
496	2\\
};
\end{axis}
\end{tikzpicture}%  % tikz
			\caption{Estimated y-Axis Positions}
		\end{subfigure}
	}
	\caption[Arc Movement Results for TREM]{Arc Movement Results for TREM (\Tsixty$=0.4$s).}
	\label{fig:trackingArcTREM}
\end{figure}

For this scenario, both algorithms were able to reliably track the second source shown in violett in \autoref{fig:trackingArcCREM} and \autoref{fig:trackingArcTREM}. Again, \gls{crem} seems to lag behind the actual x-trajectory of the second source until $t=1$, but after that both algorithms yield similar x-coordinate estimates. After $t=4$, \gls{trem} was unable to identify the y-coordinates for the first source. In this case, \gls{crem} performed more reliably, correctly identifying y-coordinates around the true y-trajectory of the first source from $t=3.7$ onwards. The overview of the position estimates over time for both \gls{trem} and \gls{crem} in \autoref{fig:trackingArcRoom} shows, that overall, \gls{trem} seems to more different position estimates than \gls{crem}. While some of these correspond to the source trajectories, it produces are also more false position estimates compared to \gls{crem}, which yields less erroneous estimates, but also less estimates close to the true source trajectories.
\begin{figure}[!htbp]
\iftoggle{quick}{%
    \includegraphics[width=\textwidth]{plots/tracking/arc/results-T60=0.4-cremtrem-room-sc.png}
}{%
	\begin{subfigure}{0.49\textwidth}
	     \centering
        \setlength{\figurewidth}{0.8\textwidth}
        % This file was created by matlab2tikz.
%
\definecolor{lms_red}{rgb}{0.80000,0.20780,0.21960}%
%
\begin{tikzpicture}

\begin{axis}[%
width=0.951\figurewidth,
height=\figureheight,
at={(0\figurewidth,0\figureheight)},
scale only axis,
xmin=1,
xmax=5,
xlabel style={font=\color{white!15!black}},
xlabel={$p_x^{(t)}$~[m]},
ymin=1,
ymax=5,
ylabel style={font=\color{white!15!black}},
ylabel={$p_y^{(t)}$~[m]},
axis background/.style={fill=white},
xmajorgrids,
ymajorgrids,
point meta min = 0,
point meta max = 5,
colorbar horizontal, 
colorbar style={
    at={(0.5,1.03)},anchor=south,
    colormap/jet,
    samples = 25,
    xticklabel pos=upper,
    xtick style={draw=none},
    title style={yshift=0.4cm},
    title=t
},
]

\addplot[
    scatter,%
    scatter/@pre marker code/.code={%
        \edef\temp{\noexpand\definecolor{mapped color}{rgb}{\pgfplotspointmeta}}%
        \temp
        \scope[draw=mapped color!80!black,fill=mapped color]%
    },%
    scatter/@post marker code/.code={%
        \endscope
    },%
    only marks,     
    mark=*,
    mark size=\trajSize,
    opacity=\trajOpacity,
    line width=\trajLinewidth,
    point meta={TeX code symbolic={%
        \edef\pgfplotspointmeta{\thisrow{R},\thisrow{G},\thisrow{B}}%
    }},
] 
table[row sep=crcr]{
x	y	R	G	B\\
3.000000	2.000000	0.000000	0.000000	0.600000\\
3.156745	2.012361	0.000000	0.000000	0.800000\\
3.309616	2.049138	0.000000	0.000000	1.000000\\
3.454832	2.109423	0.000000	0.200000	1.000000\\
3.588803	2.191724	0.000000	0.400000	1.000000\\
3.708219	2.294007	0.000000	0.600000	1.000000\\
3.810126	2.413744	0.000000	0.800000	1.000000\\
3.892005	2.547974	0.000000	1.000000	1.000000\\
3.951832	2.693379	0.200000	1.000000	0.800000\\
3.988128	2.846364	0.400000	1.000000	0.600000\\
3.999995	3.003148	0.600000	1.000000	0.400000\\
3.987141	3.159854	0.800000	1.000000	0.200000\\
3.949882	3.312607	1.000000	1.000000	0.000000\\
3.889141	3.457633	1.000000	0.800000	0.000000\\
3.806419	3.591345	1.000000	0.600000	0.000000\\
3.703760	3.710438	1.000000	0.400000	0.000000\\
3.583703	3.811967	1.000000	0.200000	0.000000\\
3.449216	3.893423	1.000000	0.000000	0.000000\\
3.303623	3.952792	0.800000	0.000000	0.000000\\
3.150524	3.988606	0.600000	0.000000	0.000000\\
3.000000	4.000000	0.000000	0.000000	0.600000\\
2.843255	3.987639	0.000000	0.000000	0.800000\\
2.690384	3.950862	0.000000	0.000000	1.000000\\
2.545168	3.890577	0.000000	0.200000	1.000000\\
2.411197	3.808276	0.000000	0.400000	1.000000\\
2.291781	3.705993	0.000000	0.600000	1.000000\\
2.189874	3.586256	0.000000	0.800000	1.000000\\
2.107995	3.452026	0.000000	1.000000	1.000000\\
2.048168	3.306621	0.200000	1.000000	0.800000\\
2.011872	3.153636	0.400000	1.000000	0.600000\\
2.000005	2.996852	0.600000	1.000000	0.400000\\
2.012859	2.840146	0.800000	1.000000	0.200000\\
2.050118	2.687393	1.000000	1.000000	0.000000\\
2.110859	2.542367	1.000000	0.800000	0.000000\\
2.193581	2.408655	1.000000	0.600000	0.000000\\
2.296240	2.289562	1.000000	0.400000	0.000000\\
2.416297	2.188033	1.000000	0.200000	0.000000\\
2.550784	2.106577	1.000000	0.000000	0.000000\\
2.696377	2.047208	0.800000	0.000000	0.000000\\
2.849476	2.011394	0.600000	0.000000	0.000000\\
};

\node at (axis cs:4,3) [anchor=west] {$s_1$};
\node at (axis cs:2,3) [anchor=east] {$s_2$};

\addplot[
    scatter,%
    scatter/@pre marker code/.code={%
        \edef\temp{\noexpand\definecolor{mapped color}{rgb}{\pgfplotspointmeta}}%
        \temp
        \scope[draw=mapped color!80!black,fill=mapped color]%
    },%
    scatter/@post marker code/.code={%
        \endscope
    },%
    only marks,     
    mark=x,
    opacity=\estOpacity,
    mark size=\estSize,
    line width=\estLinewidth,
    point meta={TeX code symbolic={%
        \edef\pgfplotspointmeta{\thisrow{R},\thisrow{G},\thisrow{B}}%
    }},
] 
table[row sep=crcr]{%
x	y	R	G	B\\
1.2	3.7	0	0	0.508064516129032\\
2.8	1.2	0	0	0.516129032258065\\
2.8	1.2	0	0	0.524193548387097\\
2.9	2	0	0	0.532258064516129\\
3	2	0	0	0.540322580645161\\
3	2	0	0	0.548387096774194\\
3	2.1	0	0	0.556451612903226\\
3.1	2.1	0	0	0.564516129032258\\
3	2	0	0	0.57258064516129\\
3	2	0	0	0.580645161290323\\
3	2	0	0	0.588709677419355\\
3	2	0	0	0.596774193548387\\
3	2	0	0	0.604838709677419\\
3	2	0	0	0.612903225806452\\
3	2	0	0	0.620967741935484\\
3	2	0	0	0.629032258064516\\
3.1	2	0	0	0.637096774193548\\
2.9	3.9	0	0	0.645161290322581\\
2.9	3.9	0	0	0.653225806451613\\
2.9	3.9	0	0	0.661290322580645\\
2.9	3.9	0	0	0.669354838709677\\
2.9	3.9	0	0	0.67741935483871\\
2.9	3.9	0	0	0.685483870967742\\
2.9	3.9	0	0	0.693548387096774\\
2.9	3.9	0	0	0.701612903225806\\
2.9	3.9	0	0	0.709677419354839\\
2.9	3.9	0	0	0.717741935483871\\
3.1	2	0	0	0.725806451612903\\
3.2	2.1	0	0	0.733870967741935\\
3.2	2.1	0	0	0.741935483870968\\
3.2	2	0	0	0.75\\
3.2	2	0	0	0.758064516129032\\
3.2	2.1	0	0	0.766129032258065\\
3.2	2.1	0	0	0.774193548387097\\
2.9	3.9	0	0	0.782258064516129\\
2.9	3.9	0	0	0.790322580645161\\
2.9	3.9	0	0	0.798387096774194\\
2.9	3.9	0	0	0.806451612903226\\
2.9	3.9	0	0	0.814516129032258\\
2.9	3.9	0	0	0.82258064516129\\
2.9	3.9	0	0	0.830645161290323\\
2.9	3.9	0	0	0.838709677419355\\
2.9	3.9	0	0	0.846774193548387\\
2.9	1.2	0	0	0.854838709677419\\
2.9	3.9	0	0	0.862903225806452\\
2.8	4	0	0	0.870967741935484\\
2.8	4	0	0	0.879032258064516\\
2.8	4	0	0	0.887096774193548\\
2.8	4	0	0	0.895161290322581\\
2.8	3.9	0	0	0.903225806451613\\
2.8	3.9	0	0	0.911290322580645\\
2.8	3.9	0	0	0.919354838709677\\
2.8	3.9	0	0	0.92741935483871\\
2.8	3.9	0	0	0.935483870967742\\
2.8	3.9	0	0	0.943548387096774\\
2.8	3.9	0	0	0.951612903225806\\
2.8	3.9	0	0	0.959677419354839\\
2.8	3.9	0	0	0.967741935483871\\
2.7	3.9	0	0	0.975806451612903\\
2.7	3.9	0	0	0.983870967741935\\
2.7	3.9	0	0	0.991935483870968\\
2.7	3.9	0	0	1\\
2.7	3.9	0	0.00806451612903226	1\\
2.7	3.9	0	0.0161290322580645	1\\
2.7	3.9	0	0.0241935483870968	1\\
2.7	3.9	0	0.032258064516129	1\\
2.7	3.9	0	0.0403225806451613	1\\
2.7	3.9	0	0.0483870967741935	1\\
2.7	3.9	0	0.0564516129032258	1\\
2.7	3.9	0	0.0645161290322581	1\\
2.7	3.9	0	0.0725806451612903	1\\
2.7	3.9	0	0.0806451612903226	1\\
2.7	3.9	0	0.0887096774193548	1\\
2.7	3.9	0	0.0967741935483871	1\\
2.7	3.9	0	0.104838709677419	1\\
2.7	3.9	0	0.112903225806452	1\\
2.7	3.9	0	0.120967741935484	1\\
2.8	4	0	0.129032258064516	1\\
2.7	3.9	0	0.137096774193548	1\\
2.7	3.9	0	0.145161290322581	1\\
2.7	3.9	0	0.153225806451613	1\\
2.7	3.9	0	0.161290322580645	1\\
2.6	3.9	0	0.169354838709677	1\\
2.6	3.9	0	0.17741935483871	1\\
2.5	3.9	0	0.185483870967742	1\\
2.5	3.9	0	0.193548387096774	1\\
2.5	3.8	0	0.201612903225806	1\\
2.5	3.8	0	0.209677419354839	1\\
2.5	3.8	0	0.217741935483871	1\\
2.5	3.8	0	0.225806451612903	1\\
2.5	3.8	0	0.233870967741935	1\\
2.5	3.8	0	0.241935483870968	1\\
2.5	3.9	0	0.25	1\\
2.5	3.8	0	0.258064516129032	1\\
2.5	3.8	0	0.266129032258065	1\\
2.5	3.8	0	0.274193548387097	1\\
2.5	3.8	0	0.282258064516129	1\\
2.5	3.8	0	0.290322580645161	1\\
2.5	3.8	0	0.298387096774194	1\\
2.5	3.8	0	0.306451612903226	1\\
2.5	3.8	0	0.314516129032258	1\\
2.5	3.8	0	0.32258064516129	1\\
2.5	3.8	0	0.330645161290323	1\\
2.5	3.8	0	0.338709677419355	1\\
2.5	3.8	0	0.346774193548387	1\\
2.5	3.8	0	0.354838709677419	1\\
2.5	3.8	0	0.362903225806452	1\\
2.5	3.8	0	0.370967741935484	1\\
2.5	3.8	0	0.379032258064516	1\\
2.5	3.8	0	0.387096774193548	1\\
2.5	3.8	0	0.395161290322581	1\\
2.5	3.8	0	0.403225806451613	1\\
2.5	3.8	0	0.411290322580645	1\\
2.5	3.8	0	0.419354838709677	1\\
2.5	3.8	0	0.42741935483871	1\\
2.5	3.8	0	0.435483870967742	1\\
2.5	3.8	0	0.443548387096774	1\\
2.5	3.8	0	0.451612903225806	1\\
2.5	3.8	0	0.459677419354839	1\\
2.5	3.8	0	0.467741935483871	1\\
2.5	3.8	0	0.475806451612903	1\\
1.2	2.2	0	0.483870967741935	1\\
1.2	2.2	0	0.491935483870968	1\\
1.2	2.2	0	0.5	1\\
1.2	2.2	0	0.508064516129032	1\\
1.2	2.2	0	0.516129032258065	1\\
1.2	2.2	0	0.524193548387097	1\\
1.2	2.2	0	0.532258064516129	1\\
1.2	2.2	0	0.540322580645161	1\\
1.2	2.2	0	0.548387096774194	1\\
1.2	2.2	0	0.556451612903226	1\\
1.2	2.2	0	0.564516129032258	1\\
1.2	2.2	0	0.57258064516129	1\\
1.2	2.2	0	0.580645161290323	1\\
1.2	2.2	0	0.588709677419355	1\\
1.2	2.2	0	0.596774193548387	1\\
1.2	2.2	0	0.604838709677419	1\\
1.2	2.2	0	0.612903225806452	1\\
1.2	2.2	0	0.620967741935484	1\\
1.2	2.2	0	0.629032258064516	1\\
1.2	2.2	0	0.637096774193548	1\\
2.2	1.2	0	0.645161290322581	1\\
2.2	1.2	0	0.653225806451613	1\\
2.2	1.2	0	0.661290322580645	1\\
2.2	1.2	0	0.669354838709677	1\\
2.2	1.2	0	0.67741935483871	1\\
2.2	1.2	0	0.685483870967742	1\\
1.2	2.2	0	0.693548387096774	1\\
1.2	2.2	0	0.701612903225806	1\\
2.4	3.7	0	0.709677419354839	1\\
2.4	3.7	0	0.717741935483871	1\\
2.4	3.7	0	0.725806451612903	1\\
2.4	3.7	0	0.733870967741935	1\\
2.4	3.7	0	0.741935483870968	1\\
2.4	3.7	0	0.75	1\\
2.3	3.6	0	0.758064516129032	1\\
2.3	3.6	0	0.766129032258065	1\\
2.3	3.6	0	0.774193548387097	1\\
2.3	3.6	0	0.782258064516129	1\\
2.3	3.6	0	0.790322580645161	1\\
2.3	3.6	0	0.798387096774194	1\\
2.3	3.6	0	0.806451612903226	1\\
2.3	3.6	0	0.814516129032258	1\\
2.3	3.6	0	0.82258064516129	1\\
2.3	3.6	0	0.830645161290323	1\\
2.3	3.6	0	0.838709677419355	1\\
2.3	3.6	0	0.846774193548387	1\\
2.3	3.6	0	0.854838709677419	1\\
2.3	3.6	0	0.862903225806452	1\\
2.3	3.6	0	0.870967741935484	1\\
2.3	3.6	0	0.879032258064516	1\\
2.3	3.6	0	0.887096774193548	1\\
2.3	3.6	0	0.895161290322581	1\\
2.3	3.6	0	0.903225806451613	1\\
2.3	3.6	0	0.911290322580645	1\\
2.3	3.6	0	0.919354838709677	1\\
2.3	3.6	0	0.92741935483871	1\\
2.3	3.6	0	0.935483870967742	1\\
2.3	3.6	0	0.943548387096774	1\\
2.2	3.5	0	0.951612903225806	1\\
2.2	3.5	0	0.959677419354839	1\\
2.2	3.5	0	0.967741935483871	1\\
2.2	3.5	0	0.975806451612903	1\\
2.2	3.5	0	0.983870967741935	1\\
2.2	3.5	0	0.991935483870968	1\\
2.2	3.5	0	1	1\\
2.2	3.5	0.00806451612903226	1	0.991935483870968\\
2.2	3.5	0.0161290322580645	1	0.983870967741935\\
2.2	3.5	0.0241935483870968	1	0.975806451612903\\
2.2	3.5	0.032258064516129	1	0.967741935483871\\
2.2	3.5	0.0403225806451613	1	0.959677419354839\\
2.2	3.5	0.0483870967741935	1	0.951612903225806\\
2.2	3.5	0.0564516129032258	1	0.943548387096774\\
2.2	3.5	0.0645161290322581	1	0.935483870967742\\
2.2	3.5	0.0725806451612903	1	0.92741935483871\\
2.2	3.5	0.0806451612903226	1	0.919354838709677\\
2.1	3.4	0.0887096774193548	1	0.911290322580645\\
2.1	3.4	0.0967741935483871	1	0.903225806451613\\
2.1	3.4	0.104838709677419	1	0.895161290322581\\
2.1	3.4	0.112903225806452	1	0.887096774193548\\
2.1	3.4	0.120967741935484	1	0.879032258064516\\
2.1	3.4	0.129032258064516	1	0.870967741935484\\
2.1	3.4	0.137096774193548	1	0.862903225806452\\
2.1	3.4	0.145161290322581	1	0.854838709677419\\
2.1	3.4	0.153225806451613	1	0.846774193548387\\
2.1	3.4	0.161290322580645	1	0.838709677419355\\
2.1	3.4	0.169354838709677	1	0.830645161290323\\
2.1	3.4	0.17741935483871	1	0.82258064516129\\
2.1	3.4	0.185483870967742	1	0.814516129032258\\
2.1	3.4	0.193548387096774	1	0.806451612903226\\
2.2	3.4	0.201612903225806	1	0.798387096774194\\
2.2	3.4	0.209677419354839	1	0.790322580645161\\
2.2	3.4	0.217741935483871	1	0.782258064516129\\
2.2	3.4	0.225806451612903	1	0.774193548387097\\
2.2	3.4	0.233870967741935	1	0.766129032258065\\
4	2.6	0.241935483870968	1	0.758064516129032\\
4	2.6	0.25	1	0.75\\
4	2.6	0.258064516129032	1	0.741935483870968\\
4	2.6	0.266129032258065	1	0.733870967741935\\
4	2.6	0.274193548387097	1	0.725806451612903\\
4	2.6	0.282258064516129	1	0.717741935483871\\
2.1	3.3	0.290322580645161	1	0.709677419354839\\
2.1	3.3	0.298387096774194	1	0.701612903225806\\
2.1	3.3	0.306451612903226	1	0.693548387096774\\
2.1	3.3	0.314516129032258	1	0.685483870967742\\
2.1	3.3	0.32258064516129	1	0.67741935483871\\
2.1	3.3	0.330645161290323	1	0.669354838709677\\
2.1	3.3	0.338709677419355	1	0.661290322580645\\
2.1	3.3	0.346774193548387	1	0.653225806451613\\
2.1	3.3	0.354838709677419	1	0.645161290322581\\
2.1	3.3	0.362903225806452	1	0.637096774193548\\
2.1	3.3	0.370967741935484	1	0.629032258064516\\
2.1	3.3	0.379032258064516	1	0.620967741935484\\
2.1	3.3	0.387096774193548	1	0.612903225806452\\
2.1	3.3	0.395161290322581	1	0.604838709677419\\
4	2.7	0.403225806451613	1	0.596774193548387\\
4	2.7	0.411290322580645	1	0.588709677419355\\
4	2.7	0.419354838709677	1	0.580645161290323\\
4	2.7	0.42741935483871	1	0.57258064516129\\
4	2.7	0.435483870967742	1	0.564516129032258\\
4	2.7	0.443548387096774	1	0.556451612903226\\
4	2.7	0.451612903225806	1	0.548387096774194\\
4	2.7	0.459677419354839	1	0.540322580645161\\
4	2.8	0.467741935483871	1	0.532258064516129\\
4	2.8	0.475806451612903	1	0.524193548387097\\
4	2.8	0.483870967741935	1	0.516129032258065\\
4	2.8	0.491935483870968	1	0.508064516129032\\
4	2.8	0.5	1	0.5\\
4	2.8	0.508064516129032	1	0.491935483870968\\
4	2.8	0.516129032258065	1	0.483870967741935\\
4	2.8	0.524193548387097	1	0.475806451612903\\
4	2.8	0.532258064516129	1	0.467741935483871\\
4	2.8	0.540322580645161	1	0.459677419354839\\
4	2.8	0.548387096774194	1	0.451612903225806\\
4	2.8	0.556451612903226	1	0.443548387096774\\
4	2.8	0.564516129032258	1	0.435483870967742\\
3.9	2.9	0.57258064516129	1	0.42741935483871\\
3.9	2.9	0.580645161290323	1	0.419354838709677\\
3.9	2.9	0.588709677419355	1	0.411290322580645\\
3.9	2.9	0.596774193548387	1	0.403225806451613\\
3.9	2.9	0.604838709677419	1	0.395161290322581\\
3.9	2.9	0.612903225806452	1	0.387096774193548\\
3.9	2.9	0.620967741935484	1	0.379032258064516\\
3.9	2.9	0.629032258064516	1	0.370967741935484\\
3.9	2.9	0.637096774193548	1	0.362903225806452\\
3.9	2.9	0.645161290322581	1	0.354838709677419\\
3.9	2.9	0.653225806451613	1	0.346774193548387\\
3.9	2.9	0.661290322580645	1	0.338709677419355\\
3.9	2.9	0.669354838709677	1	0.330645161290323\\
3.9	2.9	0.67741935483871	1	0.32258064516129\\
3.9	2.9	0.685483870967742	1	0.314516129032258\\
3.9	2.9	0.693548387096774	1	0.306451612903226\\
3.9	2.9	0.701612903225806	1	0.298387096774194\\
3.9	2.9	0.709677419354839	1	0.290322580645161\\
3.9	2.9	0.717741935483871	1	0.282258064516129\\
2	2.9	0.725806451612903	1	0.274193548387097\\
2	2.9	0.733870967741935	1	0.266129032258065\\
2	2.9	0.741935483870968	1	0.258064516129032\\
2	2.9	0.75	1	0.25\\
2	2.9	0.758064516129032	1	0.241935483870968\\
2	2.9	0.766129032258065	1	0.233870967741935\\
2	2.8	0.774193548387097	1	0.225806451612903\\
2	2.8	0.782258064516129	1	0.217741935483871\\
2	2.8	0.790322580645161	1	0.209677419354839\\
2	2.8	0.798387096774194	1	0.201612903225806\\
2	2.8	0.806451612903226	1	0.193548387096774\\
2	2.8	0.814516129032258	1	0.185483870967742\\
2	2.8	0.82258064516129	1	0.17741935483871\\
2	2.8	0.830645161290323	1	0.169354838709677\\
2	2.8	0.838709677419355	1	0.161290322580645\\
2	2.8	0.846774193548387	1	0.153225806451613\\
2	2.8	0.854838709677419	1	0.145161290322581\\
2	2.8	0.862903225806452	1	0.137096774193548\\
1.2	3	0.870967741935484	1	0.129032258064516\\
1.2	3	0.879032258064516	1	0.120967741935484\\
1.2	3	0.887096774193548	1	0.112903225806452\\
2	2.8	0.895161290322581	1	0.104838709677419\\
2	2.8	0.903225806451613	1	0.0967741935483871\\
2	2.8	0.911290322580645	1	0.0887096774193548\\
2	2.8	0.919354838709677	1	0.0806451612903226\\
2	2.8	0.92741935483871	1	0.0725806451612903\\
2	2.8	0.935483870967742	1	0.0645161290322581\\
1.2	3	0.943548387096774	1	0.0564516129032258\\
1.2	3	0.951612903225806	1	0.0483870967741935\\
1.2	3	0.959677419354839	1	0.0403225806451613\\
2	2.8	0.967741935483871	1	0.032258064516129\\
2	2.8	0.975806451612903	1	0.0241935483870968\\
2	2.8	0.983870967741935	1	0.0161290322580645\\
2	2.8	0.991935483870968	1	0.00806451612903226\\
2.1	2.7	1	1	0\\
2.1	2.7	1	0.991935483870968	0\\
2.1	2.7	1	0.983870967741935	0\\
2.1	2.7	1	0.975806451612903	0\\
2.1	2.7	1	0.967741935483871	0\\
2.1	2.7	1	0.959677419354839	0\\
2.1	2.7	1	0.951612903225806	0\\
2.1	2.7	1	0.943548387096774	0\\
2.1	2.7	1	0.935483870967742	0\\
2.1	2.7	1	0.92741935483871	0\\
2.1	2.7	1	0.919354838709677	0\\
2.1	2.7	1	0.911290322580645	0\\
2.1	2.7	1	0.903225806451613	0\\
2.1	2.7	1	0.895161290322581	0\\
2.1	2.7	1	0.887096774193548	0\\
2.1	2.7	1	0.879032258064516	0\\
2.1	2.7	1	0.870967741935484	0\\
2.1	2.7	1	0.862903225806452	0\\
2.1	2.7	1	0.854838709677419	0\\
2.1	2.7	1	0.846774193548387	0\\
2.1	2.7	1	0.838709677419355	0\\
2.1	2.7	1	0.830645161290323	0\\
2.1	2.7	1	0.82258064516129	0\\
2.1	2.7	1	0.814516129032258	0\\
2.1	2.7	1	0.806451612903226	0\\
2.1	2.7	1	0.798387096774194	0\\
2.1	2.7	1	0.790322580645161	0\\
2.1	2.7	1	0.782258064516129	0\\
2.1	2.7	1	0.774193548387097	0\\
2.1	2.7	1	0.766129032258065	0\\
2.1	2.7	1	0.758064516129032	0\\
2.1	2.7	1	0.75	0\\
2.1	2.7	1	0.741935483870968	0\\
2.1	2.7	1	0.733870967741935	0\\
2.1	2.7	1	0.725806451612903	0\\
2.1	2.7	1	0.717741935483871	0\\
2.1	2.7	1	0.709677419354839	0\\
2.1	2.7	1	0.701612903225806	0\\
2.1	2.7	1	0.693548387096774	0\\
2.1	2.7	1	0.685483870967742	0\\
2.1	2.7	1	0.67741935483871	0\\
2.1	2.7	1	0.669354838709677	0\\
2.1	2.7	1	0.661290322580645	0\\
2.1	2.7	1	0.653225806451613	0\\
2.1	2.7	1	0.645161290322581	0\\
2.1	2.7	1	0.637096774193548	0\\
2.1	2.7	1	0.629032258064516	0\\
2.1	2.7	1	0.620967741935484	0\\
2.1	2.7	1	0.612903225806452	0\\
2.1	2.7	1	0.604838709677419	0\\
2.1	2.7	1	0.596774193548387	0\\
2.1	2.7	1	0.588709677419355	0\\
2.1	2.7	1	0.580645161290323	0\\
2.1	2.7	1	0.57258064516129	0\\
2.1	2.6	1	0.564516129032258	0\\
2.1	2.6	1	0.556451612903226	0\\
2.1	2.6	1	0.548387096774194	0\\
2.1	2.6	1	0.540322580645161	0\\
2.1	2.6	1	0.532258064516129	0\\
2.1	2.6	1	0.524193548387097	0\\
2.1	2.6	1	0.516129032258065	0\\
2.1	2.6	1	0.508064516129032	0\\
2.1	2.6	1	0.5	0\\
2.2	2.6	1	0.491935483870968	0\\
2.2	2.6	1	0.483870967741935	0\\
2.2	2.6	1	0.475806451612903	0\\
2.2	2.6	1	0.467741935483871	0\\
2.2	2.6	1	0.459677419354839	0\\
2.2	2.6	1	0.451612903225806	0\\
2.2	2.6	1	0.443548387096774	0\\
2.2	2.6	1	0.435483870967742	0\\
2.2	2.6	1	0.42741935483871	0\\
2.2	2.6	1	0.419354838709677	0\\
2.2	2.6	1	0.411290322580645	0\\
2.2	2.6	1	0.403225806451613	0\\
2.2	2.6	1	0.395161290322581	0\\
2.2	2.6	1	0.387096774193548	0\\
2.2	2.6	1	0.379032258064516	0\\
2.2	2.6	1	0.370967741935484	0\\
2.2	2.6	1	0.362903225806452	0\\
2.2	2.6	1	0.354838709677419	0\\
2.2	2.6	1	0.346774193548387	0\\
2.2	2.6	1	0.338709677419355	0\\
2.2	2.6	1	0.330645161290323	0\\
2.2	2.6	1	0.32258064516129	0\\
2.2	2.6	1	0.314516129032258	0\\
2.2	2.6	1	0.306451612903226	0\\
2.2	2.6	1	0.298387096774194	0\\
2.2	2.6	1	0.290322580645161	0\\
2.2	2.6	1	0.282258064516129	0\\
2.2	2.6	1	0.274193548387097	0\\
2.2	2.6	1	0.266129032258065	0\\
2.2	2.6	1	0.258064516129032	0\\
2.2	2.6	1	0.25	0\\
2.2	2.6	1	0.241935483870968	0\\
2.2	2.6	1	0.233870967741935	0\\
2.2	2.6	1	0.225806451612903	0\\
2.2	2.6	1	0.217741935483871	0\\
3.6	3.8	1	0.209677419354839	0\\
3.6	3.8	1	0.201612903225806	0\\
3.6	3.8	1	0.193548387096774	0\\
3.6	3.8	1	0.185483870967742	0\\
2.2	2.6	1	0.17741935483871	0\\
2.2	2.5	1	0.169354838709677	0\\
2.3	2.4	1	0.161290322580645	0\\
2.3	2.4	1	0.153225806451613	0\\
2.3	2.4	1	0.145161290322581	0\\
2.4	2.3	1	0.137096774193548	0\\
2.4	2.3	1	0.129032258064516	0\\
2.4	2.2	1	0.120967741935484	0\\
2.4	2.2	1	0.112903225806452	0\\
2.4	2.2	1	0.104838709677419	0\\
2.4	2.2	1	0.0967741935483871	0\\
2.4	2.2	1	0.0887096774193548	0\\
2.4	2.2	1	0.0806451612903226	0\\
2.4	2.2	1	0.0725806451612903	0\\
2.5	2.2	1	0.0645161290322581	0\\
2.5	2.2	1	0.0564516129032258	0\\
2.5	2.2	1	0.0483870967741935	0\\
2.5	2.2	1	0.0403225806451613	0\\
2.5	2.2	1	0.032258064516129	0\\
2.5	2.2	1	0.0241935483870968	0\\
2.5	2.2	1	0.0161290322580645	0\\
2.5	2.2	1	0.00806451612903226	0\\
2.5	2.2	1	0	0\\
2.5	2.2	0.991935483870968	0	0\\
2.5	2.2	0.983870967741935	0	0\\
2.5	2.2	0.975806451612903	0	0\\
2.5	2.2	0.967741935483871	0	0\\
2.5	2.2	0.959677419354839	0	0\\
2.5	2.2	0.951612903225806	0	0\\
2.5	2.2	0.943548387096774	0	0\\
2.5	2.2	0.935483870967742	0	0\\
2.5	2.2	0.92741935483871	0	0\\
2.5	2.2	0.919354838709677	0	0\\
2.5	2.2	0.911290322580645	0	0\\
2.5	2.2	0.903225806451613	0	0\\
2.5	2.2	0.895161290322581	0	0\\
2.5	2.2	0.887096774193548	0	0\\
2.5	2.2	0.879032258064516	0	0\\
2.5	2.2	0.870967741935484	0	0\\
2.5	2.2	0.862903225806452	0	0\\
2.5	2.1	0.854838709677419	0	0\\
2.5	2.1	0.846774193548387	0	0\\
2.5	2.1	0.838709677419355	0	0\\
2.5	2.1	0.830645161290323	0	0\\
2.5	2.1	0.82258064516129	0	0\\
2.5	2.1	0.814516129032258	0	0\\
2.5	2.1	0.806451612903226	0	0\\
2.5	2.1	0.798387096774194	0	0\\
2.5	2.1	0.790322580645161	0	0\\
2.5	2.1	0.782258064516129	0	0\\
2.5	2.1	0.774193548387097	0	0\\
2.5	2.1	0.766129032258065	0	0\\
2.5	2.1	0.758064516129032	0	0\\
2.5	2.1	0.75	0	0\\
2.5	2.1	0.741935483870968	0	0\\
2.5	2.1	0.733870967741935	0	0\\
2.5	2.1	0.725806451612903	0	0\\
2.5	2.1	0.717741935483871	0	0\\
2.5	2.1	0.709677419354839	0	0\\
2.5	2.1	0.701612903225806	0	0\\
2.5	2.1	0.693548387096774	0	0\\
2.5	2.1	0.685483870967742	0	0\\
2.5	2.1	0.67741935483871	0	0\\
2.5	2.1	0.669354838709677	0	0\\
2.5	2.1	0.661290322580645	0	0\\
2.5	2.1	0.653225806451613	0	0\\
2.5	2.1	0.645161290322581	0	0\\
2.5	2.1	0.637096774193548	0	0\\
2.5	2.1	0.629032258064516	0	0\\
2.5	2.1	0.620967741935484	0	0\\
2.5	2.1	0.612903225806452	0	0\\
2.5	2.1	0.604838709677419	0	0\\
2.5	2.1	0.596774193548387	0	0\\
2.5	2.1	0.588709677419355	0	0\\
2.5	2.1	0.580645161290323	0	0\\
2.5	2.1	0.57258064516129	0	0\\
2.5	2.1	0.564516129032258	0	0\\
2.5	2.1	0.556451612903226	0	0\\
2.5	2.1	0.548387096774194	0	0\\
2.5	2.1	0.540322580645161	0	0\\
2.5	2.1	0.532258064516129	0	0\\
2.5	2.1	0.524193548387097	0	0\\
2.5	2.1	0.516129032258065	0	0\\
2.5	2.1	0.508064516129032	0	0\\
2.5	2.1	0.5	0	0\\
3.9	1.2	0	0	0.508064516129032\\
1.2	3.7	0	0	0.516129032258065\\
3	1.9	0	0	0.524193548387097\\
2.8	1.2	0	0	0.532258064516129\\
2.9	1.4	0	0	0.540322580645161\\
3.4	2.3	0	0	0.548387096774194\\
2.9	1.5	0	0	0.556451612903226\\
2.8	1.7	0	0	0.564516129032258\\
3.3	2.4	0	0	0.57258064516129\\
3.3	2.4	0	0	0.580645161290323\\
3.3	2.4	0	0	0.588709677419355\\
3.3	2.4	0	0	0.596774193548387\\
3.3	2.4	0	0	0.604838709677419\\
3.3	2.4	0	0	0.612903225806452\\
3	3.8	0	0	0.620967741935484\\
2.9	3.8	0	0	0.629032258064516\\
2.9	3.8	0	0	0.637096774193548\\
3.1	2	0	0	0.645161290322581\\
3	2	0	0	0.653225806451613\\
3	2	0	0	0.661290322580645\\
3	2	0	0	0.669354838709677\\
3.1	2	0	0	0.67741935483871\\
3.1	2	0	0	0.685483870967742\\
3.1	2	0	0	0.693548387096774\\
3.1	2	0	0	0.701612903225806\\
3.1	2.1	0	0	0.709677419354839\\
3.1	2	0	0	0.717741935483871\\
2.9	3.9	0	0	0.725806451612903\\
2.9	3.9	0	0	0.733870967741935\\
2.9	3.8	0	0	0.741935483870968\\
2.9	3.8	0	0	0.75\\
2.9	3.8	0	0	0.758064516129032\\
3	1.5	0	0	0.766129032258065\\
2.9	3.9	0	0	0.774193548387097\\
3.2	2.1	0	0	0.782258064516129\\
3.2	2.1	0	0	0.790322580645161\\
3.2	2.1	0	0	0.798387096774194\\
3.2	2.1	0	0	0.806451612903226\\
3.1	2.1	0	0	0.814516129032258\\
3.1	2.1	0	0	0.82258064516129\\
3.1	2.1	0	0	0.830645161290323\\
3.1	2.1	0	0	0.838709677419355\\
2.9	1.2	0	0	0.846774193548387\\
2.9	3.9	0	0	0.854838709677419\\
2.9	1.2	0	0	0.862903225806452\\
2.9	1.2	0	0	0.870967741935484\\
1.2	3	0	0	0.879032258064516\\
1.2	3	0	0	0.887096774193548\\
1.2	3	0	0	0.895161290322581\\
1.2	3	0	0	0.903225806451613\\
2.8	4.5	0	0	0.911290322580645\\
2.5	3.5	0	0	0.919354838709677\\
2.5	3.5	0	0	0.92741935483871\\
2.5	3.5	0	0	0.935483870967742\\
2.5	3.5	0	0	0.943548387096774\\
3.4	2	0	0	0.951612903225806\\
3.4	2	0	0	0.959677419354839\\
3.4	2	0	0	0.967741935483871\\
3.4	2	0	0	0.975806451612903\\
3.4	2	0	0	0.983870967741935\\
3.4	2.1	0	0	0.991935483870968\\
3.4	2.1	0	0	1\\
3.4	2.1	0	0.00806451612903226	1\\
3.4	2.1	0	0.0161290322580645	1\\
3.4	2.1	0	0.0241935483870968	1\\
3.4	2	0	0.032258064516129	1\\
3.4	2	0	0.0403225806451613	1\\
3.4	2	0	0.0483870967741935	1\\
3.4	2	0	0.0564516129032258	1\\
3.4	2	0	0.0645161290322581	1\\
3.4	2	0	0.0725806451612903	1\\
3.4	2	0	0.0806451612903226	1\\
2.2	1.2	0	0.0887096774193548	1\\
2.8	1.2	0	0.0967741935483871	1\\
2.2	1.2	0	0.104838709677419	1\\
2.2	1.2	0	0.112903225806452	1\\
2.2	1.2	0	0.120967741935484	1\\
2.2	1.2	0	0.129032258064516	1\\
2.2	1.2	0	0.137096774193548	1\\
2.2	1.2	0	0.145161290322581	1\\
2.2	1.2	0	0.153225806451613	1\\
2.2	1.2	0	0.161290322580645	1\\
2.2	1.2	0	0.169354838709677	1\\
2.2	1.2	0	0.17741935483871	1\\
2.2	1.2	0	0.185483870967742	1\\
2.4	4.5	0	0.193548387096774	1\\
2.5	4.4	0	0.201612903225806	1\\
2.5	4.4	0	0.209677419354839	1\\
2.5	4.4	0	0.217741935483871	1\\
2.4	4.4	0	0.225806451612903	1\\
2.4	4.4	0	0.233870967741935	1\\
2.4	4.4	0	0.241935483870968	1\\
2.4	4.5	0	0.25	1\\
2.4	4.4	0	0.258064516129032	1\\
2.4	4.4	0	0.266129032258065	1\\
2.4	4.4	0	0.274193548387097	1\\
2.5	4.4	0	0.282258064516129	1\\
2.5	4.4	0	0.290322580645161	1\\
2.5	4.4	0	0.298387096774194	1\\
2.5	4.4	0	0.306451612903226	1\\
2.5	4.4	0	0.314516129032258	1\\
2.4	4.4	0	0.32258064516129	1\\
2.4	4.4	0	0.330645161290323	1\\
2.4	4.4	0	0.338709677419355	1\\
2.4	4.4	0	0.346774193548387	1\\
2.4	4.4	0	0.354838709677419	1\\
2.2	1.2	0	0.362903225806452	1\\
2.2	1.2	0	0.370967741935484	1\\
1.2	2.2	0	0.379032258064516	1\\
1.2	2.2	0	0.387096774193548	1\\
1.2	2.2	0	0.395161290322581	1\\
1.2	2.2	0	0.403225806451613	1\\
1.2	2.2	0	0.411290322580645	1\\
1.2	2.2	0	0.419354838709677	1\\
1.2	2.2	0	0.42741935483871	1\\
1.2	2.2	0	0.435483870967742	1\\
1.2	2.2	0	0.443548387096774	1\\
1.2	2.2	0	0.451612903225806	1\\
1.2	2.2	0	0.459677419354839	1\\
1.2	2.2	0	0.467741935483871	1\\
1.2	2.2	0	0.475806451612903	1\\
2.5	3.8	0	0.483870967741935	1\\
2.5	3.8	0	0.491935483870968	1\\
2.5	3.8	0	0.5	1\\
2.5	3.8	0	0.508064516129032	1\\
3.8	1.2	0	0.516129032258065	1\\
3.8	1.2	0	0.524193548387097	1\\
2.2	1.2	0	0.532258064516129	1\\
3.8	1.2	0	0.540322580645161	1\\
3.8	1.2	0	0.548387096774194	1\\
3.8	1.2	0	0.556451612903226	1\\
3.8	1.2	0	0.564516129032258	1\\
3.8	1.2	0	0.57258064516129	1\\
3.8	1.2	0	0.580645161290323	1\\
3.8	1.2	0	0.588709677419355	1\\
3.8	1.2	0	0.596774193548387	1\\
3.8	1.2	0	0.604838709677419	1\\
2.2	1.2	0	0.612903225806452	1\\
2.2	1.2	0	0.620967741935484	1\\
2.2	1.2	0	0.629032258064516	1\\
2.2	1.2	0	0.637096774193548	1\\
1.2	2.2	0	0.645161290322581	1\\
1.2	2.2	0	0.653225806451613	1\\
1.2	2.2	0	0.661290322580645	1\\
1.2	2.2	0	0.669354838709677	1\\
1.2	2.2	0	0.67741935483871	1\\
1.2	2.2	0	0.685483870967742	1\\
2.2	1.2	0	0.693548387096774	1\\
2.2	1.2	0	0.701612903225806	1\\
1.2	2.2	0	0.709677419354839	1\\
1.2	2.2	0	0.717741935483871	1\\
1.2	2.2	0	0.725806451612903	1\\
1.2	2.2	0	0.733870967741935	1\\
3.8	1.2	0	0.741935483870968	1\\
1.2	2.2	0	0.75	1\\
1.2	3.7	0	0.758064516129032	1\\
1.2	3.7	0	0.766129032258065	1\\
1.2	3.7	0	0.774193548387097	1\\
1.2	3.7	0	0.782258064516129	1\\
1.2	3.7	0	0.790322580645161	1\\
1.2	3.7	0	0.798387096774194	1\\
1.2	3.7	0	0.806451612903226	1\\
1.2	3.7	0	0.814516129032258	1\\
1.2	3.7	0	0.82258064516129	1\\
1.2	3.7	0	0.830645161290323	1\\
1.2	3.7	0	0.838709677419355	1\\
1.2	3.7	0	0.846774193548387	1\\
1.2	2.2	0	0.854838709677419	1\\
3.8	2.4	0	0.862903225806452	1\\
3.8	2.4	0	0.870967741935484	1\\
3.8	2.4	0	0.879032258064516	1\\
3.8	2.4	0	0.887096774193548	1\\
3.9	2.5	0	0.895161290322581	1\\
3.9	2.5	0	0.903225806451613	1\\
3.9	2.5	0	0.911290322580645	1\\
3.9	2.5	0	0.919354838709677	1\\
3.9	2.5	0	0.92741935483871	1\\
3.9	2.5	0	0.935483870967742	1\\
3.9	2.5	0	0.943548387096774	1\\
3.9	2.5	0	0.951612903225806	1\\
3.9	2.5	0	0.959677419354839	1\\
3.8	2.5	0	0.967741935483871	1\\
3.8	2.5	0	0.975806451612903	1\\
3.9	2.5	0	0.983870967741935	1\\
3.9	2.5	0	0.991935483870968	1\\
3.9	2.5	0	1	1\\
3.9	2.5	0.00806451612903226	1	0.991935483870968\\
3.9	2.5	0.0161290322580645	1	0.983870967741935\\
3.9	2.5	0.0241935483870968	1	0.975806451612903\\
3.9	2.5	0.032258064516129	1	0.967741935483871\\
3.9	2.5	0.0403225806451613	1	0.959677419354839\\
3.9	2.5	0.0483870967741935	1	0.951612903225806\\
3.9	2.5	0.0564516129032258	1	0.943548387096774\\
3.9	2.5	0.0645161290322581	1	0.935483870967742\\
3.9	2.5	0.0725806451612903	1	0.92741935483871\\
3.9	2.5	0.0806451612903226	1	0.919354838709677\\
1.2	3	0.0887096774193548	1	0.911290322580645\\
1.2	3	0.0967741935483871	1	0.903225806451613\\
1.2	3	0.104838709677419	1	0.895161290322581\\
1.2	3	0.112903225806452	1	0.887096774193548\\
1.2	3	0.120967741935484	1	0.879032258064516\\
1.2	3	0.129032258064516	1	0.870967741935484\\
1.2	3	0.137096774193548	1	0.862903225806452\\
1.2	3	0.145161290322581	1	0.854838709677419\\
1.2	3	0.153225806451613	1	0.846774193548387\\
1.2	3	0.161290322580645	1	0.838709677419355\\
1.2	3	0.169354838709677	1	0.830645161290323\\
4	2.6	0.17741935483871	1	0.82258064516129\\
4	2.6	0.185483870967742	1	0.814516129032258\\
4	2.6	0.193548387096774	1	0.806451612903226\\
4	2.6	0.201612903225806	1	0.798387096774194\\
4	2.6	0.209677419354839	1	0.790322580645161\\
4	2.6	0.217741935483871	1	0.782258064516129\\
4	2.6	0.225806451612903	1	0.774193548387097\\
4	2.6	0.233870967741935	1	0.766129032258065\\
2.2	3.4	0.241935483870968	1	0.758064516129032\\
2.2	3.4	0.25	1	0.75\\
2.2	3.4	0.258064516129032	1	0.741935483870968\\
2.2	3.4	0.266129032258065	1	0.733870967741935\\
2.2	3.4	0.274193548387097	1	0.725806451612903\\
2.2	3.4	0.282258064516129	1	0.717741935483871\\
4	2.6	0.290322580645161	1	0.709677419354839\\
4	2.6	0.298387096774194	1	0.701612903225806\\
4	2.6	0.306451612903226	1	0.693548387096774\\
4	2.6	0.314516129032258	1	0.685483870967742\\
4	2.6	0.32258064516129	1	0.67741935483871\\
4	2.6	0.330645161290323	1	0.669354838709677\\
4	2.6	0.338709677419355	1	0.661290322580645\\
4	2.6	0.346774193548387	1	0.653225806451613\\
1.2	2.3	0.354838709677419	1	0.645161290322581\\
1.2	2.3	0.362903225806452	1	0.637096774193548\\
1.2	2.3	0.370967741935484	1	0.629032258064516\\
1.2	2.3	0.379032258064516	1	0.620967741935484\\
4	2.7	0.387096774193548	1	0.612903225806452\\
4	2.7	0.395161290322581	1	0.604838709677419\\
2.1	3.3	0.403225806451613	1	0.596774193548387\\
2.1	3.3	0.411290322580645	1	0.588709677419355\\
2.1	3.3	0.419354838709677	1	0.580645161290323\\
2.1	3.3	0.42741935483871	1	0.57258064516129\\
2.1	3.3	0.435483870967742	1	0.564516129032258\\
2.1	3.3	0.443548387096774	1	0.556451612903226\\
2.1	3.3	0.451612903225806	1	0.548387096774194\\
2.1	3.3	0.459677419354839	1	0.540322580645161\\
2.1	3.3	0.467741935483871	1	0.532258064516129\\
2.1	3.3	0.475806451612903	1	0.524193548387097\\
2.1	3.3	0.483870967741935	1	0.516129032258065\\
2.1	3.3	0.491935483870968	1	0.508064516129032\\
2.1	3.3	0.5	1	0.5\\
2.2	3.3	0.508064516129032	1	0.491935483870968\\
2.2	3.3	0.516129032258065	1	0.483870967741935\\
2.2	3.3	0.524193548387097	1	0.475806451612903\\
2.2	3.3	0.532258064516129	1	0.467741935483871\\
2.2	3.3	0.540322580645161	1	0.459677419354839\\
2.2	3.3	0.548387096774194	1	0.451612903225806\\
2.2	3.3	0.556451612903226	1	0.443548387096774\\
3.6	3.1	0.564516129032258	1	0.435483870967742\\
2.2	3.3	0.57258064516129	1	0.42741935483871\\
2.2	3.3	0.580645161290323	1	0.419354838709677\\
2.2	3.3	0.588709677419355	1	0.411290322580645\\
2.2	3.3	0.596774193548387	1	0.403225806451613\\
2.2	3.3	0.604838709677419	1	0.395161290322581\\
1.2	2.3	0.612903225806452	1	0.387096774193548\\
1.2	2.3	0.620967741935484	1	0.379032258064516\\
1.2	2.3	0.629032258064516	1	0.370967741935484\\
1.2	2.3	0.637096774193548	1	0.362903225806452\\
1.2	2.3	0.645161290322581	1	0.354838709677419\\
1.2	2.3	0.653225806451613	1	0.346774193548387\\
1.2	2.3	0.661290322580645	1	0.338709677419355\\
1.2	2.3	0.669354838709677	1	0.330645161290323\\
1.2	2.3	0.67741935483871	1	0.32258064516129\\
1.2	3	0.685483870967742	1	0.314516129032258\\
2	2.9	0.693548387096774	1	0.306451612903226\\
2	2.9	0.701612903225806	1	0.298387096774194\\
2	2.9	0.709677419354839	1	0.290322580645161\\
2	2.9	0.717741935483871	1	0.282258064516129\\
3.9	2.9	0.725806451612903	1	0.274193548387097\\
3.9	2.9	0.733870967741935	1	0.266129032258065\\
3.9	2.9	0.741935483870968	1	0.258064516129032\\
3.9	2.9	0.75	1	0.25\\
3.9	2.9	0.758064516129032	1	0.241935483870968\\
3.9	2.9	0.766129032258065	1	0.233870967741935\\
1.2	3	0.774193548387097	1	0.225806451612903\\
1.2	3	0.782258064516129	1	0.217741935483871\\
1.2	3	0.790322580645161	1	0.209677419354839\\
1.2	3	0.798387096774194	1	0.201612903225806\\
1.2	3	0.806451612903226	1	0.193548387096774\\
1.2	3	0.814516129032258	1	0.185483870967742\\
1.2	3	0.82258064516129	1	0.17741935483871\\
1.2	3	0.830645161290323	1	0.169354838709677\\
1.2	3	0.838709677419355	1	0.161290322580645\\
1.2	3	0.846774193548387	1	0.153225806451613\\
1.2	3	0.854838709677419	1	0.145161290322581\\
1.2	3	0.862903225806452	1	0.137096774193548\\
2	2.8	0.870967741935484	1	0.129032258064516\\
2	2.8	0.879032258064516	1	0.120967741935484\\
2	2.8	0.887096774193548	1	0.112903225806452\\
1.2	3	0.895161290322581	1	0.104838709677419\\
1.2	3	0.903225806451613	1	0.0967741935483871\\
1.2	3	0.911290322580645	1	0.0887096774193548\\
1.2	3	0.919354838709677	1	0.0806451612903226\\
1.2	3	0.92741935483871	1	0.0725806451612903\\
1.2	3	0.935483870967742	1	0.0645161290322581\\
2	2.8	0.943548387096774	1	0.0564516129032258\\
2	2.8	0.951612903225806	1	0.0483870967741935\\
2	2.8	0.959677419354839	1	0.0403225806451613\\
1.2	3	0.967741935483871	1	0.032258064516129\\
1.2	3	0.975806451612903	1	0.0241935483870968\\
1.2	3	0.983870967741935	1	0.0161290322580645\\
1.2	3	0.991935483870968	1	0.00806451612903226\\
1.2	3	1	1	0\\
1.2	3	1	0.991935483870968	0\\
1.2	3	1	0.983870967741935	0\\
1.2	3	1	0.975806451612903	0\\
1.2	3	1	0.967741935483871	0\\
1.2	3	1	0.959677419354839	0\\
1.2	3	1	0.951612903225806	0\\
1.2	3	1	0.943548387096774	0\\
1.2	3	1	0.935483870967742	0\\
1.2	3	1	0.92741935483871	0\\
1.2	3	1	0.919354838709677	0\\
1.2	3	1	0.911290322580645	0\\
1.2	3	1	0.903225806451613	0\\
1.2	3	1	0.895161290322581	0\\
1.2	3	1	0.887096774193548	0\\
1.2	3	1	0.879032258064516	0\\
1.2	3	1	0.870967741935484	0\\
1.2	3	1	0.862903225806452	0\\
1.2	3	1	0.854838709677419	0\\
1.2	3	1	0.846774193548387	0\\
1.2	3	1	0.838709677419355	0\\
1.2	3	1	0.830645161290323	0\\
1.2	3	1	0.82258064516129	0\\
1.2	3	1	0.814516129032258	0\\
2.4	2.2	1	0.806451612903226	0\\
2.4	2.2	1	0.798387096774194	0\\
2.4	2.2	1	0.790322580645161	0\\
2.4	2.2	1	0.782258064516129	0\\
2.4	2.2	1	0.774193548387097	0\\
2.4	2.2	1	0.766129032258065	0\\
2.4	2.2	1	0.758064516129032	0\\
3.9	2.9	1	0.75	0\\
4	2.9	1	0.741935483870968	0\\
4	2.9	1	0.733870967741935	0\\
4	2.9	1	0.725806451612903	0\\
4	2.9	1	0.717741935483871	0\\
4	2.9	1	0.709677419354839	0\\
4	2.9	1	0.701612903225806	0\\
4	2.9	1	0.693548387096774	0\\
4	3.7	1	0.685483870967742	0\\
4	3.7	1	0.67741935483871	0\\
4	3.7	1	0.669354838709677	0\\
4	3.7	1	0.661290322580645	0\\
4	3.7	1	0.653225806451613	0\\
4	3.7	1	0.645161290322581	0\\
4	3.7	1	0.637096774193548	0\\
4	3.7	1	0.629032258064516	0\\
4	3.7	1	0.620967741935484	0\\
4	3.7	1	0.612903225806452	0\\
4	3.7	1	0.604838709677419	0\\
4	3.7	1	0.596774193548387	0\\
4	3.7	1	0.588709677419355	0\\
4	3.7	1	0.580645161290323	0\\
4	3.7	1	0.57258064516129	0\\
4	3.7	1	0.564516129032258	0\\
4	3.7	1	0.556451612903226	0\\
4	3.7	1	0.548387096774194	0\\
2.4	2.2	1	0.540322580645161	0\\
2.4	2.2	1	0.532258064516129	0\\
2.5	2.2	1	0.524193548387097	0\\
2.5	2.2	1	0.516129032258065	0\\
2.5	2.2	1	0.508064516129032	0\\
2.5	2.2	1	0.5	0\\
4	2.9	1	0.491935483870968	0\\
4	2.9	1	0.483870967741935	0\\
3.9	3.7	1	0.475806451612903	0\\
3.9	3.7	1	0.467741935483871	0\\
3.7	3.7	1	0.459677419354839	0\\
3.6	3.7	1	0.451612903225806	0\\
3.6	3.7	1	0.443548387096774	0\\
3.6	3.7	1	0.435483870967742	0\\
3.6	3.7	1	0.42741935483871	0\\
3.6	3.7	1	0.419354838709677	0\\
3.6	3.7	1	0.411290322580645	0\\
3.6	3.7	1	0.403225806451613	0\\
3.6	3.7	1	0.395161290322581	0\\
3.6	3.7	1	0.387096774193548	0\\
3.6	3.7	1	0.379032258064516	0\\
3.6	3.7	1	0.370967741935484	0\\
3.6	3.7	1	0.362903225806452	0\\
3.6	3.7	1	0.354838709677419	0\\
3.6	3.7	1	0.346774193548387	0\\
3.6	3.7	1	0.338709677419355	0\\
3.6	3.7	1	0.330645161290323	0\\
3.6	3.7	1	0.32258064516129	0\\
3.5	3.8	1	0.314516129032258	0\\
3.5	3.8	1	0.306451612903226	0\\
3.5	3.8	1	0.298387096774194	0\\
3.5	3.8	1	0.290322580645161	0\\
3.5	3.8	1	0.282258064516129	0\\
3.5	3.8	1	0.274193548387097	0\\
3.5	3.8	1	0.266129032258065	0\\
3.5	3.8	1	0.258064516129032	0\\
3.5	3.8	1	0.25	0\\
3.5	3.8	1	0.241935483870968	0\\
3.5	3.8	1	0.233870967741935	0\\
3.5	3.8	1	0.225806451612903	0\\
3.5	3.8	1	0.217741935483871	0\\
2.2	2.6	1	0.209677419354839	0\\
2.2	2.6	1	0.201612903225806	0\\
2.2	2.6	1	0.193548387096774	0\\
2.2	2.6	1	0.185483870967742	0\\
3.6	3.8	1	0.17741935483871	0\\
3.6	3.8	1	0.169354838709677	0\\
3.6	3.8	1	0.161290322580645	0\\
3.6	3.8	1	0.153225806451613	0\\
3.6	3.8	1	0.145161290322581	0\\
3.6	3.8	1	0.137096774193548	0\\
3.6	3.8	1	0.129032258064516	0\\
3.6	3.8	1	0.120967741935484	0\\
3.6	3.8	1	0.112903225806452	0\\
3.6	3.8	1	0.104838709677419	0\\
3.7	1.2	1	0.0967741935483871	0\\
3.7	1.2	1	0.0887096774193548	0\\
2.2	2.7	1	0.0806451612903226	0\\
2.2	2.7	1	0.0725806451612903	0\\
2.1	2.6	1	0.0645161290322581	0\\
2.1	2.6	1	0.0564516129032258	0\\
2.1	2.6	1	0.0483870967741935	0\\
2.1	2.6	1	0.0403225806451613	0\\
2.2	2.7	1	0.032258064516129	0\\
3.7	1.2	1	0.0241935483870968	0\\
2.2	2.7	1	0.0161290322580645	0\\
2.2	2.7	1	0.00806451612903226	0\\
2.2	2.7	1	0	0\\
2.2	2.7	0.991935483870968	0	0\\
2.2	2.7	0.983870967741935	0	0\\
2.2	2.7	0.975806451612903	0	0\\
3.7	1.2	0.967741935483871	0	0\\
3.7	1.2	0.959677419354839	0	0\\
3.7	1.2	0.951612903225806	0	0\\
3.7	1.2	0.943548387096774	0	0\\
3.7	1.2	0.935483870967742	0	0\\
3.7	1.2	0.92741935483871	0	0\\
3.7	1.2	0.919354838709677	0	0\\
3.7	1.2	0.911290322580645	0	0\\
3.7	1.2	0.903225806451613	0	0\\
3.7	1.2	0.895161290322581	0	0\\
3.7	1.2	0.887096774193548	0	0\\
3.7	1.2	0.879032258064516	0	0\\
3.7	1.2	0.870967741935484	0	0\\
3.7	1.2	0.862903225806452	0	0\\
3.7	1.2	0.854838709677419	0	0\\
3.7	1.2	0.846774193548387	0	0\\
3.7	1.2	0.838709677419355	0	0\\
3.6	3.8	0.830645161290323	0	0\\
3.6	3.8	0.82258064516129	0	0\\
3.6	3.8	0.814516129032258	0	0\\
3.6	3.8	0.806451612903226	0	0\\
3.6	3.8	0.798387096774194	0	0\\
3.6	3.8	0.790322580645161	0	0\\
3.6	3.8	0.782258064516129	0	0\\
3.6	3.8	0.774193548387097	0	0\\
3.5	3.9	0.766129032258065	0	0\\
3.5	3.9	0.758064516129032	0	0\\
3.4	3.9	0.75	0	0\\
3.5	3.9	0.741935483870968	0	0\\
3.5	3.9	0.733870967741935	0	0\\
3.5	3.9	0.725806451612903	0	0\\
3.5	3.9	0.717741935483871	0	0\\
3.5	3.9	0.709677419354839	0	0\\
3.5	3.9	0.701612903225806	0	0\\
3.4	3.9	0.693548387096774	0	0\\
3.4	3.9	0.685483870967742	0	0\\
3.4	3.9	0.67741935483871	0	0\\
3.4	3.9	0.669354838709677	0	0\\
3.4	3.9	0.661290322580645	0	0\\
3.4	3.9	0.653225806451613	0	0\\
3.3	4	0.645161290322581	0	0\\
3.3	4	0.637096774193548	0	0\\
3.3	4	0.629032258064516	0	0\\
3.3	4	0.620967741935484	0	0\\
3.3	4	0.612903225806452	0	0\\
3.3	4	0.604838709677419	0	0\\
3.3	4	0.596774193548387	0	0\\
3.3	4	0.588709677419355	0	0\\
3.3	4	0.580645161290323	0	0\\
3.3	4	0.57258064516129	0	0\\
3.3	4	0.564516129032258	0	0\\
3.3	4	0.556451612903226	0	0\\
3.3	4	0.548387096774194	0	0\\
3.3	4	0.540322580645161	0	0\\
3.9	1.2	0.532258064516129	0	0\\
3.9	1.2	0.524193548387097	0	0\\
3.9	1.2	0.516129032258065	0	0\\
3.9	1.2	0.508064516129032	0	0\\
3.9	1.2	0.5	0	0\\
};
\end{axis}
\end{tikzpicture}%
        \caption{Estimated Positions \glsentryshort{trem}}
	\end{subfigure}
	\begin{subfigure}{0.49\textwidth}
		 \centering
        \setlength{\figurewidth}{0.8\textwidth}
        % This file was created by matlab2tikz.
%
\definecolor{lms_red}{rgb}{0.80000,0.20780,0.21960}%
%
\begin{tikzpicture}

\begin{axis}[%
width=0.951\figurewidth,
height=\figureheight,
at={(0\figurewidth,0\figureheight)},
scale only axis,
xmin=1,
xmax=5,
xlabel style={font=\color{white!15!black}},
xlabel={$p_x^{(t)}$~[m]},
ymin=1,
ymax=5,
ylabel style={font=\color{white!15!black}},
ylabel={$p_y^{(t)}$~[m]},
axis background/.style={fill=white},
axis x line*=bottom,
axis y line*=left,
xmajorgrids,
ymajorgrids
]
\addplot[scatter, only marks, mark=o] table[row sep=crcr]{%
x	y	R	G	B\\
3	2	0	0	0.504\\
3.00629573527024	2.00001981833768	0	0	0.512\\
3.01259122099846	2.00007927256519	0	0	0.52\\
3.01888620765254	2.00017836032595	0	0	0.528\\
3.02518044572014	2.00031707769246	0	0	0.536\\
3.03147368571859	2.00049541916643	0	0	0.544\\
3.03776567820477	2.00071337767899	0	0	0.552\\
3.04405617378503	2.00097094459099	0	0	0.56\\
3.05034492312503	2.00126810969333	0	0	0.568\\
3.05663167695965	2.00160486120738	0	0	0.576\\
3.06291618610289	2.00198118578542	0	0	0.584\\
3.06919820145767	2.00239706851121	0	0	0.592\\
3.07547747402582	2.00285249290053	0	0	0.6\\
3.08175375491782	2.00334744090188	0	0	0.608\\
3.08802679536279	2.00388189289715	0	0	0.616\\
3.09429634671824	2.00445582770246	0	0	0.624\\
3.10056216048001	2.00506922256893	0	0	0.632\\
3.10682398829207	2.00572205318363	0	0	0.64\\
3.11308158195639	2.00641429367053	0	0	0.648\\
3.11933469344275	2.0071459165915	0	0	0.656\\
3.12558307489861	2.00791689294747	0	0	0.664\\
3.1318264786589	2.00872719217947	0	0	0.672\\
3.13806465725586	2.00957678216995	0	0	0.68\\
3.14429736342881	2.01046562924399	0	0	0.688\\
3.15052435013401	2.01139369817064	0	0	0.696\\
3.15674537055442	2.01236095216433	0	0	0.704\\
3.16296017810946	2.01336735288633	0	0	0.712\\
3.16916852646482	2.01441286044626	0	0	0.72\\
3.17537016954221	2.01549743340368	0	0	0.728\\
3.18156486152913	2.01662102876973	0	0	0.736\\
3.18775235688858	2.01778360200881	0	0	0.744\\
3.19393241036881	2.0189851070404	0	0	0.752\\
3.20010477701304	2.02022549624081	0	0	0.76\\
3.2062692121692	2.02150472044516	0	0	0.768\\
3.21242547149955	2.02282272894925	0	0	0.776\\
3.21857331099045	2.0241794695116	0	0	0.784\\
3.22471248696198	2.02557488835552	0	0	0.792\\
3.23084275607762	2.02700893017125	0	0	0.8\\
3.23696387535387	2.02848153811815	0	0	0.808\\
3.24307560216993	2.02999265382693	0	0	0.816\\
3.24917769427725	2.03154221740198	0	0	0.824\\
3.2552699098092	2.03313016742376	0	0	0.832\\
3.26135200729061	2.03475644095121	0	0	0.84\\
3.26742374564734	2.03642097352426	0	0	0.848\\
3.27348488421589	2.03812369916635	0	0	0.856\\
3.27953518275287	2.03986455038712	0	0	0.864\\
3.28557440144456	2.04164345818501	0	0	0.872\\
3.2916023009164	2.04346035205003	0	0	0.88\\
3.29761864224251	2.04531515996653	0	0	0.888\\
3.30362318695512	2.04720780841612	0	0	0.896\\
3.30961569705403	2.04913822238048	0	0	0.904\\
3.31559593501607	2.05110632534444	0	0	0.912\\
3.3215636638045	2.05311203929893	0	0	0.92\\
3.32751864687837	2.05515528474412	0	0	0.928\\
3.33346064820196	2.05723598069256	0	0	0.936\\
3.3393894322541	2.05935404467236	0	0	0.944\\
3.34530476403748	2.06150939273053	0	0	0.952\\
3.35120640908804	2.06370193943623	0	0	0.96\\
3.35709413348417	2.06593159788419	0	0	0.968\\
3.36296770385606	2.06819827969817	0	0	0.976\\
3.36882688739491	2.07050189503443	0	0	0.984\\
3.37467145186216	2.07284235258533	0	0	0.992\\
3.38050116559871	2.07521955958291	0	0	1\\
3.38631579753408	2.07763342180258	0	0.008	1\\
3.3921151171956	2.08008384356688	0	0.016	1\\
3.39789889471751	2.08257072774923	0	0.024	1\\
3.40366690085011	2.0850939757778	0	0.032	1\\
3.4094189069688	2.08765348763944	0	0.04	1\\
3.4151546850832	2.09024916188361	0	0.048	1\\
3.42087400784612	2.09288089562641	0	0.056	1\\
3.42657664856262	2.09554858455466	0	0.064	1\\
3.43226238119898	2.09825212293004	0	0.072	1\\
3.43793098039168	2.10099140359328	0	0.08	1\\
3.44358222145627	2.1037663179684	0	0.088	1\\
3.44921588039636	2.10657675606702	0	0.096	1\\
3.45483173391243	2.10942260649272	0	0.104	1\\
3.46042955941071	2.11230375644545	0	0.112	1\\
3.46600913501203	2.115220091726	0	0.12	1\\
3.47157023956055	2.11817149674055	0	0.128	1\\
3.47711265263257	2.12115785450519	0	0.136	1\\
3.48263615454527	2.12417904665066	0	0.144	1\\
3.4881405263654	2.12723495342692	0	0.152	1\\
3.49362554991795	2.130325453708	0	0.16	1\\
3.49909100779483	2.13345042499673	0	0.168	1\\
3.50453668336346	2.13660974342966	0	0.176	1\\
3.50996236077536	2.13980328378189	0	0.184	1\\
3.51536782497473	2.14303091947211	0	0.192	1\\
3.52075286170693	2.14629252256757	0	0.2	1\\
3.526117257527	2.14958796378916	0	0.208	1\\
3.53146079980814	2.15291711251655	0	0.216	1\\
3.53678327675009	2.15627983679336	0	0.224	1\\
3.54208447738756	2.15967600333237	0	0.232	1\\
3.54736419159858	2.16310547752083	0	0.24	1\\
3.55262221011284	2.16656812342579	0	0.248	1\\
3.55785832451995	2.17006380379948	0	0.256	1\\
3.56307232727777	2.17359238008474	0	0.264	1\\
3.56826401172054	2.17715371242055	0	0.272	1\\
3.57343317206716	2.18074765964753	0	0.28	1\\
3.5785796034293	2.18437407931356	0	0.288	1\\
3.58370310181953	2.18803282767943	0	0.296	1\\
3.5888034641594	2.19172375972451	0	0.304	1\\
3.59388048828752	2.19544672915252	0	0.312	1\\
3.59893397296752	2.19920158839734	0	0.32	1\\
3.60396371789607	2.20298818862883	0	0.328	1\\
3.60896952371081	2.20680637975874	0	0.336	1\\
3.61395119199823	2.21065601044668	0	0.344	1\\
3.61890852530157	2.21453692810608	0	0.352	1\\
3.62384132712861	2.21844897891026	0	0.36	1\\
3.62874940195949	2.22239200779854	0	0.368	1\\
3.63363255525445	2.22636585848237	0	0.376	1\\
3.63849059346152	2.23037037345151	0	0.384	1\\
3.64332332402421	2.23440539398031	0	0.392	1\\
3.64813055538916	2.23847076013396	0	0.4	1\\
3.65291209701369	2.24256631077487	0	0.408	1\\
3.6576677593734	2.24669188356903	0	0.416	1\\
3.66239735396963	2.25084731499244	0	0.424	1\\
3.66710069333698	2.25503244033762	0	0.432	1\\
3.67177759105072	2.25924709372012	0	0.44	1\\
3.67642786173418	2.26349110808509	0	0.448	1\\
3.68105132106607	2.2677643152139	0	0.456	1\\
3.68564778578784	2.27206654573084	0	0.464	1\\
3.69021707371092	2.27639762910979	0	0.472	1\\
3.69475900372392	2.280757393681	0	0.48	1\\
3.69927339579984	2.28514566663791	0	0.488	1\\
3.70376007100318	2.28956227404395	0	0.496	1\\
3.70821885149705	2.2940070408395	0	0.504	1\\
3.71264956055023	2.29847979084878	0	0.512	1\\
3.71705202254413	2.30298034678685	0	0.52	1\\
3.72142606297979	2.30750853026663	0	0.528	1\\
3.7257715084848	2.31206416180599	0	0.536	1\\
3.73008818682015	2.31664706083483	0	0.544	1\\
3.73437592688704	2.32125704570228	0	0.552	1\\
3.73863455873374	2.32589393368386	0	0.56	1\\
3.74286391356222	2.33055754098875	0	0.568	1\\
3.74706382373492	2.33524768276706	0	0.576	1\\
3.75123412278137	2.33996417311716	0	0.584	1\\
3.75537464540477	2.34470682509306	0	0.592	1\\
3.75948522748858	2.3494754507118	0	0.6	1\\
3.76356570610299	2.35426986096091	0	0.608	1\\
3.76761591951138	2.35908986580591	0	0.616	1\\
3.77163570717678	2.36393527419783	0	0.624	1\\
3.77562490976816	2.3688058940808	0	0.632	1\\
3.77958336916678	2.37370153239963	0	0.64	1\\
3.78351092847249	2.3786219951075	0	0.648	1\\
3.78740743200988	2.38356708717363	0	0.656	1\\
3.79127272533452	2.388536612591	0	0.664	1\\
3.79510665523902	2.39353037438416	0	0.672	1\\
3.79890906975915	2.39854817461698	0	0.68	1\\
3.80267981817984	2.40358981440055	0	0.688	1\\
3.80641875104116	2.40865509390102	0	0.696	1\\
3.81012572014425	2.41374381234757	0	0.704	1\\
3.81380057855716	2.4188557680403	0	0.712	1\\
3.81744318062073	2.42399075835829	0	0.72	1\\
3.82105338195433	2.42914857976759	0	0.728	1\\
3.82463103946158	2.43432902782932	0	0.736	1\\
3.82817601133602	2.43953189720773	0	0.744	1\\
3.83168815706676	2.44475698167838	0	0.752	1\\
3.83516733744402	2.4500040741363	0	0.76	1\\
3.83861341456465	2.45527296660417	0	0.768	1\\
3.84202625183762	2.46056345024063	0	0.776	1\\
3.8454057139894	2.46587531534849	0	0.784	1\\
3.84875166706935	2.47120835138308	0	0.792	1\\
3.85206397845501	2.47656234696057	0	0.8	1\\
3.85534251685737	2.48193708986639	0	0.808	1\\
3.85858715232607	2.48733236706359	0	0.816	1\\
3.86179775625454	2.49274796470133	0	0.824	1\\
3.86497420138512	2.49818366812332	0	0.832	1\\
3.86811636181409	2.50363926187634	0	0.84	1\\
3.87122411299665	2.5091145297188	0	0.848	1\\
3.87429733175188	2.51460925462929	0	0.856	1\\
3.87733589626761	2.52012321881517	0	0.864	1\\
3.88033968610523	2.52565620372124	0	0.872	1\\
3.88330858220451	2.53120799003837	0	0.88	1\\
3.88624246688827	2.53677835771221	0	0.888	1\\
3.88914122386708	2.54236708595191	0	0.896	1\\
3.89200473824386	2.54797395323885	0	0.904	1\\
3.8948328965184	2.55359873733547	0	0.912	1\\
3.89762558659192	2.55924121529401	0	0.92	1\\
3.90038269777148	2.56490116346541	0	0.928	1\\
3.90310412077434	2.57057835750815	0	0.936	1\\
3.90578974773236	2.57627257239712	0	0.944	1\\
3.9084394721962	2.58198358243258	0	0.952	1\\
3.91105318913959	2.58771116124908	0	0.96	1\\
3.91363079496349	2.59345508182444	0	0.968	1\\
3.91617218750018	2.59921511648873	0	0.976	1\\
3.91867726601729	2.60499103693334	0	0.984	1\\
3.92114593122185	2.61078261421999	0	0.992	1\\
3.92357808526418	2.6165896187898	0	1	1\\
3.92597363174177	2.62241182047242	0.008	1	0.992\\
3.92833247570313	2.62824898849513	0.016	1	0.984\\
3.93065452365153	2.634100891492	0.024	1	0.976\\
3.9329396835487	2.63996729751304	0.032	1	0.968\\
3.9351878648185	2.64584797403344	0.04	1	0.96\\
3.9373989783505	2.65174268796271	0.048	1	0.952\\
3.93957293650352	2.65765120565401	0.056	1	0.944\\
3.94170965310907	2.66357329291332	0.064	1	0.936\\
3.94380904347483	2.66950871500881	0.072	1	0.928\\
3.94587102438792	2.67545723668007	0.08	1	0.92\\
3.94789551411829	2.68141862214747	0.088	1	0.912\\
3.9498824324219	2.68739263512153	0.096	1	0.904\\
3.9518317005439	2.69337903881223	0.104	1	0.896\\
3.95374324122178	2.69937759593842	0.112	1	0.888\\
3.95561697868844	2.70538806873724	0.12	1	0.88\\
3.95745283867515	2.71141021897355	0.128	1	0.872\\
3.95925074841452	2.71744380794931	0.136	1	0.864\\
3.96101063664338	2.72348859651313	0.144	1	0.856\\
3.96273243360563	2.72954434506969	0.152	1	0.848\\
3.96441607105494	2.73561081358924	0.16	1	0.84\\
3.96606148225754	2.74168776161713	0.168	1	0.832\\
3.96766860199478	2.74777494828337	0.176	1	0.824\\
3.9692373665658	2.75387213231209	0.184	1	0.816\\
3.97076771378997	2.7599790720312	0.192	1	0.808\\
3.97225958300941	2.76609552538192	0.2	1	0.8\\
3.9737129150914	2.77222124992836	0.208	1	0.792\\
3.97512765243069	2.77835600286717	0.216	1	0.784\\
3.97650373895177	2.78449954103714	0.224	1	0.776\\
3.97784112011117	2.79065162092884	0.232	1	0.768\\
3.97913974289954	2.79681199869428	0.24	1	0.76\\
3.98039955584379	2.80298043015656	0.248	1	0.752\\
3.98162050900913	2.80915667081958	0.256	1	0.744\\
3.98280255400101	2.81534047587768	0.264	1	0.736\\
3.98394564396713	2.82153160022538	0.272	1	0.728\\
3.98504973359918	2.82772979846711	0.28	1	0.72\\
3.98611477913473	2.8339348249269	0.288	1	0.712\\
3.98714073835891	2.84014643365811	0.296	1	0.704\\
3.98812757060611	2.84636437845324	0.304	1	0.696\\
3.98907523676159	2.85258841285363	0.312	1	0.688\\
3.98998369926299	2.85881829015924	0.32	1	0.68\\
3.9908529221019	2.86505376343844	0.328	1	0.672\\
3.9916828708252	2.87129458553782	0.336	1	0.664\\
3.9924735125365	2.87754050909193	0.344	1	0.656\\
3.99322481589737	2.88379128653312	0.352	1	0.648\\
3.99393675112866	2.89004667010137	0.36	1	0.64\\
3.99460929001162	2.89630641185405	0.368	1	0.632\\
3.99524240588904	2.90257026367583	0.376	1	0.624\\
3.99583607366631	2.90883797728843	0.384	1	0.616\\
3.99639026981242	2.91510930426054	0.392	1	0.608\\
3.99690497236088	2.92138399601759	0.4	1	0.6\\
3.99738016091058	2.92766180385167	0.408	1	0.592\\
3.99781581662663	2.93394247893134	0.416	1	0.584\\
3.99821192224109	2.94022577231154	0.424	1	0.576\\
3.99856846205365	2.94651143494339	0.432	1	0.568\\
3.99888542193225	2.95279921768413	0.44	1	0.56\\
3.99916278931366	2.95908887130696	0.448	1	0.552\\
3.99940055320397	2.96538014651091	0.456	1	0.544\\
3.99959870417899	2.97167279393076	0.464	1	0.536\\
3.99975723438468	2.97796656414687	0.472	1	0.528\\
3.99987613753744	2.98426120769513	0.48	1	0.52\\
3.99995540892433	2.9905564750768	0.488	1	0.512\\
3.99999504540331	2.99685211676839	0.496	1	0.504\\
3.99999504540331	3.00314788323161	0.504	1	0.496\\
3.99995540892433	3.0094435249232	0.512	1	0.488\\
3.99987613753744	3.01573879230487	0.52	1	0.48\\
3.99975723438468	3.02203343585313	0.528	1	0.472\\
3.99959870417899	3.02832720606924	0.536	1	0.464\\
3.99940055320397	3.03461985348909	0.544	1	0.456\\
3.99916278931366	3.04091112869304	0.552	1	0.448\\
3.99888542193225	3.04720078231587	0.56	1	0.44\\
3.99856846205365	3.05348856505661	0.568	1	0.432\\
3.99821192224109	3.05977422768846	0.576	1	0.424\\
3.99781581662663	3.06605752106866	0.584	1	0.416\\
3.99738016091058	3.07233819614833	0.592	1	0.408\\
3.99690497236088	3.07861600398241	0.6	1	0.4\\
3.99639026981242	3.08489069573946	0.608	1	0.392\\
3.99583607366631	3.09116202271157	0.616	1	0.384\\
3.99524240588904	3.09742973632417	0.624	1	0.376\\
3.99460929001162	3.10369358814595	0.632	1	0.368\\
3.99393675112866	3.10995332989863	0.64	1	0.36\\
3.99322481589737	3.11620871346688	0.648	1	0.352\\
3.9924735125365	3.12245949090807	0.656	1	0.344\\
3.9916828708252	3.12870541446218	0.664	1	0.336\\
3.9908529221019	3.13494623656156	0.672	1	0.328\\
3.98998369926299	3.14118170984076	0.68	1	0.32\\
3.98907523676159	3.14741158714637	0.688	1	0.312\\
3.98812757060611	3.15363562154676	0.696	1	0.304\\
3.98714073835891	3.15985356634189	0.704	1	0.296\\
3.98611477913473	3.1660651750731	0.712	1	0.288\\
3.98504973359918	3.17227020153289	0.72	1	0.28\\
3.98394564396713	3.17846839977462	0.728	1	0.272\\
3.98280255400101	3.18465952412232	0.736	1	0.264\\
3.98162050900913	3.19084332918042	0.744	1	0.256\\
3.98039955584379	3.19701956984344	0.752	1	0.248\\
3.97913974289954	3.20318800130572	0.76	1	0.24\\
3.97784112011117	3.20934837907116	0.768	1	0.232\\
3.97650373895177	3.21550045896286	0.776	1	0.224\\
3.97512765243069	3.22164399713283	0.784	1	0.216\\
3.9737129150914	3.22777875007164	0.792	1	0.208\\
3.97225958300941	3.23390447461808	0.8	1	0.2\\
3.97076771378997	3.2400209279688	0.808	1	0.192\\
3.9692373665658	3.24612786768791	0.816	1	0.184\\
3.96766860199478	3.25222505171663	0.824	1	0.176\\
3.96606148225754	3.25831223838287	0.832	1	0.168\\
3.96441607105494	3.26438918641076	0.84	1	0.16\\
3.96273243360563	3.27045565493031	0.848	1	0.152\\
3.96101063664338	3.27651140348687	0.856	1	0.144\\
3.95925074841452	3.28255619205069	0.864	1	0.136\\
3.95745283867515	3.28858978102645	0.872	1	0.128\\
3.95561697868844	3.29461193126276	0.88	1	0.12\\
3.95374324122178	3.30062240406158	0.888	1	0.112\\
3.9518317005439	3.30662096118777	0.896	1	0.104\\
3.9498824324219	3.31260736487847	0.904	1	0.096\\
3.94789551411829	3.31858137785253	0.912	1	0.088\\
3.94587102438792	3.32454276331993	0.92	1	0.08\\
3.94380904347483	3.33049128499119	0.928	1	0.072\\
3.94170965310907	3.33642670708668	0.936	1	0.064\\
3.93957293650352	3.34234879434599	0.944	1	0.056\\
3.9373989783505	3.34825731203729	0.952	1	0.048\\
3.9351878648185	3.35415202596656	0.96	1	0.04\\
3.9329396835487	3.36003270248696	0.968	1	0.032\\
3.93065452365153	3.365899108508	0.976	1	0.024\\
3.92833247570313	3.37175101150487	0.984	1	0.016\\
3.92597363174177	3.37758817952758	0.992	1	0.008\\
3.92357808526418	3.3834103812102	1	1	0\\
3.92114593122185	3.38921738578001	1	0.992	0\\
3.91867726601729	3.39500896306666	1	0.984	0\\
3.91617218750018	3.40078488351127	1	0.976	0\\
3.91363079496349	3.40654491817556	1	0.968	0\\
3.91105318913959	3.41228883875092	1	0.96	0\\
3.9084394721962	3.41801641756742	1	0.952	0\\
3.90578974773236	3.42372742760288	1	0.944	0\\
3.90310412077434	3.42942164249185	1	0.936	0\\
3.90038269777148	3.43509883653459	1	0.928	0\\
3.89762558659192	3.44075878470599	1	0.92	0\\
3.8948328965184	3.44640126266453	1	0.912	0\\
3.89200473824386	3.45202604676115	1	0.904	0\\
3.88914122386708	3.45763291404809	1	0.896	0\\
3.88624246688827	3.46322164228779	1	0.888	0\\
3.88330858220451	3.46879200996163	1	0.88	0\\
3.88033968610523	3.47434379627876	1	0.872	0\\
3.87733589626761	3.47987678118483	1	0.864	0\\
3.87429733175188	3.48539074537071	1	0.856	0\\
3.87122411299665	3.4908854702812	1	0.848	0\\
3.86811636181409	3.49636073812366	1	0.84	0\\
3.86497420138512	3.50181633187668	1	0.832	0\\
3.86179775625454	3.50725203529867	1	0.824	0\\
3.85858715232607	3.51266763293641	1	0.816	0\\
3.85534251685737	3.51806291013361	1	0.808	0\\
3.85206397845501	3.52343765303943	1	0.8	0\\
3.84875166706935	3.52879164861692	1	0.792	0\\
3.8454057139894	3.53412468465151	1	0.784	0\\
3.84202625183762	3.53943654975937	1	0.776	0\\
3.83861341456465	3.54472703339583	1	0.768	0\\
3.83516733744402	3.5499959258637	1	0.76	0\\
3.83168815706676	3.55524301832162	1	0.752	0\\
3.82817601133602	3.56046810279227	1	0.744	0\\
3.82463103946158	3.56567097217068	1	0.736	0\\
3.82105338195433	3.57085142023241	1	0.728	0\\
3.81744318062073	3.57600924164171	1	0.72	0\\
3.81380057855716	3.5811442319597	1	0.712	0\\
3.81012572014425	3.58625618765243	1	0.704	0\\
3.80641875104116	3.59134490609898	1	0.696	0\\
3.80267981817984	3.59641018559945	1	0.688	0\\
3.79890906975915	3.60145182538302	1	0.68	0\\
3.79510665523902	3.60646962561584	1	0.672	0\\
3.79127272533452	3.611463387409	1	0.664	0\\
3.78740743200988	3.61643291282637	1	0.656	0\\
3.78351092847249	3.6213780048925	1	0.648	0\\
3.77958336916678	3.62629846760037	1	0.64	0\\
3.77562490976816	3.6311941059192	1	0.632	0\\
3.77163570717678	3.63606472580217	1	0.624	0\\
3.76761591951138	3.64091013419409	1	0.616	0\\
3.76356570610299	3.64573013903909	1	0.608	0\\
3.75948522748858	3.6505245492882	1	0.6	0\\
3.75537464540477	3.65529317490694	1	0.592	0\\
3.75123412278137	3.66003582688284	1	0.584	0\\
3.74706382373492	3.66475231723294	1	0.576	0\\
3.74286391356222	3.66944245901125	1	0.568	0\\
3.73863455873374	3.67410606631614	1	0.56	0\\
3.73437592688704	3.67874295429772	1	0.552	0\\
3.73008818682015	3.68335293916517	1	0.544	0\\
3.7257715084848	3.68793583819401	1	0.536	0\\
3.72142606297979	3.69249146973337	1	0.528	0\\
3.71705202254413	3.69701965321315	1	0.52	0\\
3.71264956055023	3.70152020915122	1	0.512	0\\
3.70821885149705	3.7059929591605	1	0.504	0\\
3.70376007100318	3.71043772595605	1	0.496	0\\
3.69927339579984	3.71485433336209	1	0.488	0\\
3.69475900372392	3.719242606319	1	0.48	0\\
3.69021707371092	3.72360237089021	1	0.472	0\\
3.68564778578784	3.72793345426916	1	0.464	0\\
3.68105132106607	3.7322356847861	1	0.456	0\\
3.67642786173418	3.73650889191491	1	0.448	0\\
3.67177759105072	3.74075290627988	1	0.44	0\\
3.66710069333698	3.74496755966238	1	0.432	0\\
3.66239735396963	3.74915268500756	1	0.424	0\\
3.65766775937339	3.75330811643097	1	0.416	0\\
3.65291209701369	3.75743368922513	1	0.408	0\\
3.64813055538916	3.76152923986604	1	0.4	0\\
3.64332332402421	3.76559460601969	1	0.392	0\\
3.63849059346152	3.76962962654849	1	0.384	0\\
3.63363255525445	3.77363414151763	1	0.376	0\\
3.62874940195949	3.77760799220146	1	0.368	0\\
3.62384132712861	3.78155102108974	1	0.36	0\\
3.61890852530157	3.78546307189393	1	0.352	0\\
3.61395119199823	3.78934398955332	1	0.344	0\\
3.60896952371081	3.79319362024126	1	0.336	0\\
3.60396371789607	3.79701181137117	1	0.328	0\\
3.59893397296752	3.80079841160266	1	0.32	0\\
3.59388048828752	3.80455327084748	1	0.312	0\\
3.5888034641594	3.80827624027549	1	0.304	0\\
3.58370310181953	3.81196717232057	1	0.296	0\\
3.5785796034293	3.81562592068644	1	0.288	0\\
3.57343317206716	3.81925234035247	1	0.28	0\\
3.56826401172054	3.82284628757945	1	0.272	0\\
3.56307232727777	3.82640761991526	1	0.264	0\\
3.55785832451995	3.82993619620052	1	0.256	0\\
3.55262221011284	3.83343187657421	1	0.248	0\\
3.54736419159858	3.83689452247917	1	0.24	0\\
3.54208447738756	3.84032399666763	1	0.232	0\\
3.53678327675009	3.84372016320664	1	0.224	0\\
3.53146079980814	3.84708288748345	1	0.216	0\\
3.526117257527	3.85041203621084	1	0.208	0\\
3.52075286170693	3.85370747743243	1	0.2	0\\
3.51536782497473	3.85696908052789	1	0.192	0\\
3.50996236077536	3.86019671621811	1	0.184	0\\
3.50453668336346	3.86339025657034	1	0.176	0\\
3.49909100779483	3.86654957500327	1	0.168	0\\
3.49362554991795	3.869674546292	1	0.16	0\\
3.4881405263654	3.87276504657308	1	0.152	0\\
3.48263615454527	3.87582095334934	1	0.144	0\\
3.47711265263257	3.87884214549481	1	0.136	0\\
3.47157023956055	3.88182850325945	1	0.128	0\\
3.46600913501203	3.884779908274	1	0.12	0\\
3.46042955941071	3.88769624355455	1	0.112	0\\
3.45483173391243	3.89057739350728	1	0.104	0\\
3.44921588039636	3.89342324393298	1	0.096	0\\
3.44358222145627	3.8962336820316	1	0.088	0\\
3.43793098039168	3.89900859640672	1	0.08	0\\
3.43226238119898	3.90174787706996	1	0.072	0\\
3.42657664856262	3.90445141544534	1	0.064	0\\
3.42087400784612	3.90711910437359	1	0.056	0\\
3.4151546850832	3.90975083811639	1	0.048	0\\
3.4094189069688	3.91234651236056	1	0.04	0\\
3.40366690085011	3.9149060242222	1	0.032	0\\
3.39789889471751	3.91742927225077	1	0.024	0\\
3.3921151171956	3.91991615643312	1	0.016	0\\
3.38631579753408	3.92236657819742	1	0.008	0\\
3.38050116559871	3.92478044041709	1	0	0\\
3.37467145186216	3.92715764741467	0.992	0	0\\
3.36882688739491	3.92949810496557	0.984	0	0\\
3.36296770385606	3.93180172030183	0.976	0	0\\
3.35709413348417	3.93406840211581	0.968	0	0\\
3.35120640908804	3.93629806056377	0.96	0	0\\
3.34530476403748	3.93849060726947	0.952	0	0\\
3.33938943225409	3.94064595532764	0.944	0	0\\
3.33346064820196	3.94276401930744	0.936	0	0\\
3.32751864687837	3.94484471525588	0.928	0	0\\
3.3215636638045	3.94688796070107	0.92	0	0\\
3.31559593501607	3.94889367465556	0.912	0	0\\
3.30961569705403	3.95086177761952	0.904	0	0\\
3.30362318695512	3.95279219158388	0.896	0	0\\
3.29761864224251	3.95468484003347	0.888	0	0\\
3.2916023009164	3.95653964794997	0.88	0	0\\
3.28557440144456	3.95835654181499	0.872	0	0\\
3.27953518275287	3.96013544961288	0.864	0	0\\
3.27348488421589	3.96187630083365	0.856	0	0\\
3.26742374564734	3.96357902647574	0.848	0	0\\
3.26135200729061	3.96524355904879	0.84	0	0\\
3.2552699098092	3.96686983257624	0.832	0	0\\
3.24917769427725	3.96845778259802	0.824	0	0\\
3.24307560216993	3.97000734617307	0.816	0	0\\
3.23696387535387	3.97151846188185	0.808	0	0\\
3.23084275607762	3.97299106982875	0.8	0	0\\
3.22471248696198	3.97442511164448	0.792	0	0\\
3.21857331099045	3.9758205304884	0.784	0	0\\
3.21242547149955	3.97717727105075	0.776	0	0\\
3.2062692121692	3.97849527955484	0.768	0	0\\
3.20010477701304	3.97977450375919	0.76	0	0\\
3.19393241036881	3.9810148929596	0.752	0	0\\
3.18775235688858	3.98221639799119	0.744	0	0\\
3.18156486152913	3.98337897123027	0.736	0	0\\
3.17537016954221	3.98450256659632	0.728	0	0\\
3.16916852646482	3.98558713955374	0.72	0	0\\
3.16296017810946	3.98663264711367	0.712	0	0\\
3.15674537055442	3.98763904783567	0.704	0	0\\
3.15052435013401	3.98860630182936	0.696	0	0\\
3.14429736342881	3.98953437075601	0.688	0	0\\
3.13806465725586	3.99042321783005	0.68	0	0\\
3.1318264786589	3.99127280782053	0.672	0	0\\
3.12558307489861	3.99208310705253	0.664	0	0\\
3.11933469344275	3.9928540834085	0.656	0	0\\
3.11308158195639	3.99358570632947	0.648	0	0\\
3.10682398829207	3.99427794681637	0.64	0	0\\
3.10056216048001	3.99493077743107	0.632	0	0\\
3.09429634671824	3.99554417229754	0.624	0	0\\
3.08802679536279	3.99611810710285	0.616	0	0\\
3.08175375491782	3.99665255909812	0.608	0	0\\
3.07547747402582	3.99714750709947	0.6	0	0\\
3.06919820145767	3.99760293148879	0.592	0	0\\
3.06291618610289	3.99801881421458	0.584	0	0\\
3.05663167695965	3.99839513879262	0.576	0	0\\
3.05034492312503	3.99873189030667	0.568	0	0\\
3.04405617378503	3.99902905540901	0.56	0	0\\
3.03776567820477	3.99928662232101	0.552	0	0\\
3.03147368571859	3.99950458083357	0.544	0	0\\
3.02518044572014	3.99968292230754	0.536	0	0\\
3.01888620765254	3.99982163967405	0.528	0	0\\
3.01259122099846	3.99992072743481	0.52	0	0\\
3.00629573527024	3.99998018166232	0.512	0	0\\
3	4	0.504	0	0\\
};
\addplot[scatter, only marks, mark=o] table[row sep=crcr]{%
x	y	R	G	B\\
3	4	0	0	0.504\\
2.99370426472976	3.99998018166232	0	0	0.512\\
2.98740877900154	3.99992072743481	0	0	0.52\\
2.98111379234746	3.99982163967405	0	0	0.528\\
2.97481955427986	3.99968292230754	0	0	0.536\\
2.96852631428141	3.99950458083357	0	0	0.544\\
2.96223432179523	3.99928662232101	0	0	0.552\\
2.95594382621497	3.99902905540901	0	0	0.56\\
2.94965507687497	3.99873189030667	0	0	0.568\\
2.94336832304035	3.99839513879262	0	0	0.576\\
2.93708381389711	3.99801881421458	0	0	0.584\\
2.93080179854233	3.99760293148879	0	0	0.592\\
2.92452252597418	3.99714750709947	0	0	0.6\\
2.91824624508218	3.99665255909812	0	0	0.608\\
2.91197320463721	3.99611810710285	0	0	0.616\\
2.90570365328176	3.99554417229754	0	0	0.624\\
2.89943783951999	3.99493077743107	0	0	0.632\\
2.89317601170793	3.99427794681637	0	0	0.64\\
2.88691841804361	3.99358570632947	0	0	0.648\\
2.88066530655725	3.9928540834085	0	0	0.656\\
2.87441692510139	3.99208310705253	0	0	0.664\\
2.8681735213411	3.99127280782053	0	0	0.672\\
2.86193534274414	3.99042321783005	0	0	0.68\\
2.85570263657119	3.98953437075601	0	0	0.688\\
2.84947564986599	3.98860630182936	0	0	0.696\\
2.84325462944558	3.98763904783567	0	0	0.704\\
2.83703982189054	3.98663264711367	0	0	0.712\\
2.83083147353518	3.98558713955374	0	0	0.72\\
2.82462983045779	3.98450256659632	0	0	0.728\\
2.81843513847087	3.98337897123027	0	0	0.736\\
2.81224764311142	3.98221639799119	0	0	0.744\\
2.80606758963119	3.9810148929596	0	0	0.752\\
2.79989522298696	3.97977450375919	0	0	0.76\\
2.7937307878308	3.97849527955484	0	0	0.768\\
2.78757452850045	3.97717727105075	0	0	0.776\\
2.78142668900955	3.9758205304884	0	0	0.784\\
2.77528751303802	3.97442511164448	0	0	0.792\\
2.76915724392238	3.97299106982875	0	0	0.8\\
2.76303612464613	3.97151846188185	0	0	0.808\\
2.75692439783007	3.97000734617307	0	0	0.816\\
2.75082230572275	3.96845778259802	0	0	0.824\\
2.7447300901908	3.96686983257624	0	0	0.832\\
2.73864799270939	3.96524355904879	0	0	0.84\\
2.73257625435266	3.96357902647574	0	0	0.848\\
2.72651511578411	3.96187630083365	0	0	0.856\\
2.72046481724713	3.96013544961288	0	0	0.864\\
2.71442559855544	3.95835654181499	0	0	0.872\\
2.7083976990836	3.95653964794997	0	0	0.88\\
2.70238135775749	3.95468484003347	0	0	0.888\\
2.69637681304488	3.95279219158388	0	0	0.896\\
2.69038430294597	3.95086177761952	0	0	0.904\\
2.68440406498393	3.94889367465556	0	0	0.912\\
2.6784363361955	3.94688796070107	0	0	0.92\\
2.67248135312163	3.94484471525588	0	0	0.928\\
2.66653935179804	3.94276401930744	0	0	0.936\\
2.66061056774591	3.94064595532764	0	0	0.944\\
2.65469523596252	3.93849060726947	0	0	0.952\\
2.64879359091196	3.93629806056377	0	0	0.96\\
2.64290586651583	3.93406840211581	0	0	0.968\\
2.63703229614394	3.93180172030183	0	0	0.976\\
2.63117311260509	3.92949810496557	0	0	0.984\\
2.62532854813784	3.92715764741467	0	0	0.992\\
2.61949883440129	3.92478044041709	0	0	1\\
2.61368420246592	3.92236657819742	0	0.008	1\\
2.6078848828044	3.91991615643312	0	0.016	1\\
2.60210110528249	3.91742927225077	0	0.024	1\\
2.59633309914989	3.9149060242222	0	0.032	1\\
2.5905810930312	3.91234651236056	0	0.04	1\\
2.5848453149168	3.90975083811639	0	0.048	1\\
2.57912599215388	3.90711910437359	0	0.056	1\\
2.57342335143738	3.90445141544534	0	0.064	1\\
2.56773761880102	3.90174787706996	0	0.072	1\\
2.56206901960832	3.89900859640672	0	0.08	1\\
2.55641777854373	3.8962336820316	0	0.088	1\\
2.55078411960364	3.89342324393298	0	0.096	1\\
2.54516826608757	3.89057739350728	0	0.104	1\\
2.53957044058929	3.88769624355455	0	0.112	1\\
2.53399086498797	3.884779908274	0	0.12	1\\
2.52842976043945	3.88182850325945	0	0.128	1\\
2.52288734736743	3.87884214549481	0	0.136	1\\
2.51736384545473	3.87582095334934	0	0.144	1\\
2.5118594736346	3.87276504657308	0	0.152	1\\
2.50637445008205	3.869674546292	0	0.16	1\\
2.50090899220517	3.86654957500327	0	0.168	1\\
2.49546331663654	3.86339025657034	0	0.176	1\\
2.49003763922463	3.86019671621811	0	0.184	1\\
2.48463217502527	3.85696908052789	0	0.192	1\\
2.47924713829307	3.85370747743243	0	0.2	1\\
2.473882742473	3.85041203621084	0	0.208	1\\
2.46853920019186	3.84708288748345	0	0.216	1\\
2.46321672324991	3.84372016320664	0	0.224	1\\
2.45791552261244	3.84032399666763	0	0.232	1\\
2.45263580840142	3.83689452247917	0	0.24	1\\
2.44737778988716	3.83343187657421	0	0.248	1\\
2.44214167548005	3.82993619620052	0	0.256	1\\
2.43692767272223	3.82640761991526	0	0.264	1\\
2.43173598827946	3.82284628757945	0	0.272	1\\
2.42656682793284	3.81925234035247	0	0.28	1\\
2.4214203965707	3.81562592068644	0	0.288	1\\
2.41629689818047	3.81196717232057	0	0.296	1\\
2.4111965358406	3.80827624027549	0	0.304	1\\
2.40611951171248	3.80455327084748	0	0.312	1\\
2.40106602703248	3.80079841160266	0	0.32	1\\
2.39603628210393	3.79701181137117	0	0.328	1\\
2.39103047628919	3.79319362024126	0	0.336	1\\
2.38604880800177	3.78934398955332	0	0.344	1\\
2.38109147469843	3.78546307189392	0	0.352	1\\
2.37615867287139	3.78155102108974	0	0.36	1\\
2.37125059804051	3.77760799220146	0	0.368	1\\
2.36636744474555	3.77363414151763	0	0.376	1\\
2.36150940653848	3.76962962654849	0	0.384	1\\
2.35667667597579	3.76559460601969	0	0.392	1\\
2.35186944461084	3.76152923986604	0	0.4	1\\
2.34708790298631	3.75743368922513	0	0.408	1\\
2.3423322406266	3.75330811643097	0	0.416	1\\
2.33760264603037	3.74915268500756	0	0.424	1\\
2.33289930666302	3.74496755966238	0	0.432	1\\
2.32822240894928	3.74075290627988	0	0.44	1\\
2.32357213826582	3.73650889191491	0	0.448	1\\
2.31894867893393	3.7322356847861	0	0.456	1\\
2.31435221421216	3.72793345426916	0	0.464	1\\
2.30978292628908	3.72360237089021	0	0.472	1\\
2.30524099627608	3.719242606319	0	0.48	1\\
2.30072660420016	3.71485433336209	0	0.488	1\\
2.29623992899682	3.71043772595605	0	0.496	1\\
2.29178114850295	3.7059929591605	0	0.504	1\\
2.28735043944977	3.70152020915122	0	0.512	1\\
2.28294797745587	3.69701965321315	0	0.52	1\\
2.27857393702021	3.69249146973337	0	0.528	1\\
2.2742284915152	3.68793583819401	0	0.536	1\\
2.26991181317985	3.68335293916517	0	0.544	1\\
2.26562407311296	3.67874295429772	0	0.552	1\\
2.26136544126626	3.67410606631614	0	0.56	1\\
2.25713608643778	3.66944245901125	0	0.568	1\\
2.25293617626508	3.66475231723294	0	0.576	1\\
2.24876587721863	3.66003582688284	0	0.584	1\\
2.24462535459523	3.65529317490694	0	0.592	1\\
2.24051477251142	3.6505245492882	0	0.6	1\\
2.23643429389701	3.64573013903909	0	0.608	1\\
2.23238408048862	3.64091013419409	0	0.616	1\\
2.22836429282322	3.63606472580217	0	0.624	1\\
2.22437509023184	3.6311941059192	0	0.632	1\\
2.22041663083322	3.62629846760037	0	0.64	1\\
2.21648907152751	3.6213780048925	0	0.648	1\\
2.21259256799012	3.61643291282637	0	0.656	1\\
2.20872727466548	3.611463387409	0	0.664	1\\
2.20489334476098	3.60646962561584	0	0.672	1\\
2.20109093024085	3.60145182538302	0	0.68	1\\
2.19732018182016	3.59641018559945	0	0.688	1\\
2.19358124895884	3.59134490609898	0	0.696	1\\
2.18987427985575	3.58625618765243	0	0.704	1\\
2.18619942144284	3.5811442319597	0	0.712	1\\
2.18255681937927	3.57600924164171	0	0.72	1\\
2.17894661804567	3.57085142023241	0	0.728	1\\
2.17536896053842	3.56567097217068	0	0.736	1\\
2.17182398866398	3.56046810279227	0	0.744	1\\
2.16831184293324	3.55524301832162	0	0.752	1\\
2.16483266255598	3.5499959258637	0	0.76	1\\
2.16138658543535	3.54472703339583	0	0.768	1\\
2.15797374816238	3.53943654975937	0	0.776	1\\
2.1545942860106	3.53412468465151	0	0.784	1\\
2.15124833293065	3.52879164861692	0	0.792	1\\
2.14793602154499	3.52343765303943	0	0.8	1\\
2.14465748314263	3.51806291013361	0	0.808	1\\
2.14141284767393	3.51266763293641	0	0.816	1\\
2.13820224374546	3.50725203529867	0	0.824	1\\
2.13502579861488	3.50181633187668	0	0.832	1\\
2.13188363818591	3.49636073812366	0	0.84	1\\
2.12877588700335	3.4908854702812	0	0.848	1\\
2.12570266824812	3.48539074537071	0	0.856	1\\
2.12266410373239	3.47987678118483	0	0.864	1\\
2.11966031389477	3.47434379627876	0	0.872	1\\
2.11669141779549	3.46879200996163	0	0.88	1\\
2.11375753311173	3.46322164228779	0	0.888	1\\
2.11085877613292	3.45763291404809	0	0.896	1\\
2.10799526175614	3.45202604676115	0	0.904	1\\
2.1051671034816	3.44640126266453	0	0.912	1\\
2.10237441340808	3.44075878470599	0	0.92	1\\
2.09961730222852	3.43509883653459	0	0.928	1\\
2.09689587922566	3.42942164249185	0	0.936	1\\
2.09421025226764	3.42372742760288	0	0.944	1\\
2.0915605278038	3.41801641756742	0	0.952	1\\
2.08894681086041	3.41228883875092	0	0.96	1\\
2.08636920503651	3.40654491817556	0	0.968	1\\
2.08382781249982	3.40078488351127	0	0.976	1\\
2.08132273398271	3.39500896306666	0	0.984	1\\
2.07885406877815	3.38921738578001	0	0.992	1\\
2.07642191473582	3.3834103812102	0	1	1\\
2.07402636825823	3.37758817952758	0.008	1	0.992\\
2.07166752429687	3.37175101150487	0.016	1	0.984\\
2.06934547634847	3.365899108508	0.024	1	0.976\\
2.0670603164513	3.36003270248696	0.032	1	0.968\\
2.0648121351815	3.35415202596656	0.04	1	0.96\\
2.0626010216495	3.34825731203729	0.048	1	0.952\\
2.06042706349648	3.34234879434599	0.056	1	0.944\\
2.05829034689093	3.33642670708668	0.064	1	0.936\\
2.05619095652517	3.33049128499119	0.072	1	0.928\\
2.05412897561208	3.32454276331993	0.08	1	0.92\\
2.05210448588171	3.31858137785253	0.088	1	0.912\\
2.0501175675781	3.31260736487847	0.096	1	0.904\\
2.0481682994561	3.30662096118777	0.104	1	0.896\\
2.04625675877822	3.30062240406158	0.112	1	0.888\\
2.04438302131156	3.29461193126276	0.12	1	0.88\\
2.04254716132485	3.28858978102645	0.128	1	0.872\\
2.04074925158548	3.28255619205069	0.136	1	0.864\\
2.03898936335662	3.27651140348687	0.144	1	0.856\\
2.03726756639437	3.27045565493031	0.152	1	0.848\\
2.03558392894506	3.26438918641076	0.16	1	0.84\\
2.03393851774246	3.25831223838287	0.168	1	0.832\\
2.03233139800522	3.25222505171663	0.176	1	0.824\\
2.0307626334342	3.24612786768791	0.184	1	0.816\\
2.02923228621003	3.2400209279688	0.192	1	0.808\\
2.02774041699059	3.23390447461808	0.2	1	0.8\\
2.0262870849086	3.22777875007164	0.208	1	0.792\\
2.02487234756931	3.22164399713283	0.216	1	0.784\\
2.02349626104823	3.21550045896286	0.224	1	0.776\\
2.02215887988882	3.20934837907116	0.232	1	0.768\\
2.02086025710046	3.20318800130572	0.24	1	0.76\\
2.01960044415621	3.19701956984344	0.248	1	0.752\\
2.01837949099087	3.19084332918042	0.256	1	0.744\\
2.01719744599899	3.18465952412232	0.264	1	0.736\\
2.01605435603287	3.17846839977462	0.272	1	0.728\\
2.01495026640082	3.17227020153289	0.28	1	0.72\\
2.01388522086527	3.1660651750731	0.288	1	0.712\\
2.01285926164109	3.15985356634189	0.296	1	0.704\\
2.01187242939389	3.15363562154676	0.304	1	0.696\\
2.01092476323841	3.14741158714637	0.312	1	0.688\\
2.01001630073701	3.14118170984076	0.32	1	0.68\\
2.0091470778981	3.13494623656156	0.328	1	0.672\\
2.0083171291748	3.12870541446218	0.336	1	0.664\\
2.0075264874635	3.12245949090807	0.344	1	0.656\\
2.00677518410263	3.11620871346688	0.352	1	0.648\\
2.00606324887134	3.10995332989863	0.36	1	0.64\\
2.00539070998838	3.10369358814595	0.368	1	0.632\\
2.00475759411096	3.09742973632417	0.376	1	0.624\\
2.00416392633369	3.09116202271157	0.384	1	0.616\\
2.00360973018758	3.08489069573946	0.392	1	0.608\\
2.00309502763912	3.07861600398241	0.4	1	0.6\\
2.00261983908942	3.07233819614833	0.408	1	0.592\\
2.00218418337337	3.06605752106866	0.416	1	0.584\\
2.00178807775891	3.05977422768846	0.424	1	0.576\\
2.00143153794635	3.05348856505661	0.432	1	0.568\\
2.00111457806775	3.04720078231587	0.44	1	0.56\\
2.00083721068634	3.04091112869304	0.448	1	0.552\\
2.00059944679603	3.03461985348909	0.456	1	0.544\\
2.00040129582101	3.02832720606924	0.464	1	0.536\\
2.00024276561532	3.02203343585313	0.472	1	0.528\\
2.00012386246256	3.01573879230487	0.48	1	0.52\\
2.00004459107567	3.0094435249232	0.488	1	0.512\\
2.00000495459669	3.00314788323161	0.496	1	0.504\\
2.00000495459669	2.99685211676839	0.504	1	0.496\\
2.00004459107567	2.9905564750768	0.512	1	0.488\\
2.00012386246256	2.98426120769513	0.52	1	0.48\\
2.00024276561532	2.97796656414687	0.528	1	0.472\\
2.00040129582101	2.97167279393076	0.536	1	0.464\\
2.00059944679603	2.96538014651091	0.544	1	0.456\\
2.00083721068634	2.95908887130696	0.552	1	0.448\\
2.00111457806775	2.95279921768413	0.56	1	0.44\\
2.00143153794635	2.94651143494339	0.568	1	0.432\\
2.00178807775891	2.94022577231154	0.576	1	0.424\\
2.00218418337337	2.93394247893134	0.584	1	0.416\\
2.00261983908942	2.92766180385167	0.592	1	0.408\\
2.00309502763912	2.92138399601759	0.6	1	0.4\\
2.00360973018758	2.91510930426054	0.608	1	0.392\\
2.00416392633369	2.90883797728843	0.616	1	0.384\\
2.00475759411096	2.90257026367583	0.624	1	0.376\\
2.00539070998838	2.89630641185405	0.632	1	0.368\\
2.00606324887134	2.89004667010137	0.64	1	0.36\\
2.00677518410263	2.88379128653312	0.648	1	0.352\\
2.0075264874635	2.87754050909193	0.656	1	0.344\\
2.0083171291748	2.87129458553782	0.664	1	0.336\\
2.0091470778981	2.86505376343844	0.672	1	0.328\\
2.01001630073701	2.85881829015924	0.68	1	0.32\\
2.01092476323841	2.85258841285363	0.688	1	0.312\\
2.01187242939389	2.84636437845324	0.696	1	0.304\\
2.01285926164109	2.84014643365811	0.704	1	0.296\\
2.01388522086527	2.8339348249269	0.712	1	0.288\\
2.01495026640082	2.82772979846711	0.72	1	0.28\\
2.01605435603287	2.82153160022538	0.728	1	0.272\\
2.01719744599899	2.81534047587768	0.736	1	0.264\\
2.01837949099087	2.80915667081958	0.744	1	0.256\\
2.01960044415621	2.80298043015656	0.752	1	0.248\\
2.02086025710046	2.79681199869428	0.76	1	0.24\\
2.02215887988883	2.79065162092884	0.768	1	0.232\\
2.02349626104823	2.78449954103714	0.776	1	0.224\\
2.02487234756931	2.77835600286717	0.784	1	0.216\\
2.0262870849086	2.77222124992836	0.792	1	0.208\\
2.02774041699059	2.76609552538192	0.8	1	0.2\\
2.02923228621003	2.7599790720312	0.808	1	0.192\\
2.0307626334342	2.75387213231209	0.816	1	0.184\\
2.03233139800522	2.74777494828337	0.824	1	0.176\\
2.03393851774246	2.74168776161713	0.832	1	0.168\\
2.03558392894506	2.73561081358924	0.84	1	0.16\\
2.03726756639437	2.72954434506969	0.848	1	0.152\\
2.03898936335662	2.72348859651313	0.856	1	0.144\\
2.04074925158548	2.71744380794931	0.864	1	0.136\\
2.04254716132485	2.71141021897355	0.872	1	0.128\\
2.04438302131156	2.70538806873724	0.88	1	0.12\\
2.04625675877822	2.69937759593842	0.888	1	0.112\\
2.0481682994561	2.69337903881223	0.896	1	0.104\\
2.0501175675781	2.68739263512153	0.904	1	0.096\\
2.05210448588171	2.68141862214747	0.912	1	0.088\\
2.05412897561208	2.67545723668007	0.92	1	0.08\\
2.05619095652517	2.66950871500881	0.928	1	0.072\\
2.05829034689093	2.66357329291332	0.936	1	0.064\\
2.06042706349648	2.65765120565401	0.944	1	0.056\\
2.0626010216495	2.65174268796271	0.952	1	0.048\\
2.0648121351815	2.64584797403344	0.96	1	0.04\\
2.0670603164513	2.63996729751304	0.968	1	0.032\\
2.06934547634847	2.634100891492	0.976	1	0.024\\
2.07166752429687	2.62824898849513	0.984	1	0.016\\
2.07402636825823	2.62241182047242	0.992	1	0.008\\
2.07642191473582	2.6165896187898	1	1	0\\
2.07885406877815	2.61078261421999	1	0.992	0\\
2.08132273398271	2.60499103693334	1	0.984	0\\
2.08382781249982	2.59921511648873	1	0.976	0\\
2.08636920503651	2.59345508182444	1	0.968	0\\
2.08894681086041	2.58771116124908	1	0.96	0\\
2.0915605278038	2.58198358243258	1	0.952	0\\
2.09421025226764	2.57627257239712	1	0.944	0\\
2.09689587922566	2.57057835750815	1	0.936	0\\
2.09961730222852	2.56490116346541	1	0.928	0\\
2.10237441340808	2.55924121529401	1	0.92	0\\
2.1051671034816	2.55359873733547	1	0.912	0\\
2.10799526175614	2.54797395323885	1	0.904	0\\
2.11085877613292	2.54236708595191	1	0.896	0\\
2.11375753311173	2.53677835771221	1	0.888	0\\
2.11669141779549	2.53120799003837	1	0.88	0\\
2.11966031389477	2.52565620372124	1	0.872	0\\
2.12266410373239	2.52012321881517	1	0.864	0\\
2.12570266824812	2.51460925462929	1	0.856	0\\
2.12877588700335	2.5091145297188	1	0.848	0\\
2.13188363818591	2.50363926187634	1	0.84	0\\
2.13502579861488	2.49818366812332	1	0.832	0\\
2.13820224374546	2.49274796470133	1	0.824	0\\
2.14141284767393	2.48733236706359	1	0.816	0\\
2.14465748314263	2.48193708986639	1	0.808	0\\
2.14793602154499	2.47656234696057	1	0.8	0\\
2.15124833293065	2.47120835138308	1	0.792	0\\
2.1545942860106	2.46587531534849	1	0.784	0\\
2.15797374816238	2.46056345024063	1	0.776	0\\
2.16138658543535	2.45527296660417	1	0.768	0\\
2.16483266255598	2.4500040741363	1	0.76	0\\
2.16831184293324	2.44475698167838	1	0.752	0\\
2.17182398866398	2.43953189720773	1	0.744	0\\
2.17536896053842	2.43432902782932	1	0.736	0\\
2.17894661804567	2.42914857976759	1	0.728	0\\
2.18255681937927	2.42399075835829	1	0.72	0\\
2.18619942144284	2.4188557680403	1	0.712	0\\
2.18987427985575	2.41374381234757	1	0.704	0\\
2.19358124895884	2.40865509390102	1	0.696	0\\
2.19732018182016	2.40358981440055	1	0.688	0\\
2.20109093024085	2.39854817461698	1	0.68	0\\
2.20489334476098	2.39353037438416	1	0.672	0\\
2.20872727466548	2.388536612591	1	0.664	0\\
2.21259256799012	2.38356708717363	1	0.656	0\\
2.21648907152751	2.3786219951075	1	0.648	0\\
2.22041663083322	2.37370153239963	1	0.64	0\\
2.22437509023184	2.3688058940808	1	0.632	0\\
2.22836429282322	2.36393527419783	1	0.624	0\\
2.23238408048862	2.35908986580591	1	0.616	0\\
2.23643429389701	2.35426986096091	1	0.608	0\\
2.24051477251142	2.3494754507118	1	0.6	0\\
2.24462535459523	2.34470682509306	1	0.592	0\\
2.24876587721863	2.33996417311716	1	0.584	0\\
2.25293617626508	2.33524768276706	1	0.576	0\\
2.25713608643778	2.33055754098875	1	0.568	0\\
2.26136544126626	2.32589393368386	1	0.56	0\\
2.26562407311296	2.32125704570228	1	0.552	0\\
2.26991181317985	2.31664706083483	1	0.544	0\\
2.2742284915152	2.31206416180599	1	0.536	0\\
2.27857393702021	2.30750853026663	1	0.528	0\\
2.28294797745587	2.30298034678685	1	0.52	0\\
2.28735043944977	2.29847979084878	1	0.512	0\\
2.29178114850295	2.2940070408395	1	0.504	0\\
2.29623992899682	2.28956227404395	1	0.496	0\\
2.30072660420016	2.28514566663791	1	0.488	0\\
2.30524099627608	2.280757393681	1	0.48	0\\
2.30978292628908	2.27639762910979	1	0.472	0\\
2.31435221421216	2.27206654573084	1	0.464	0\\
2.31894867893393	2.2677643152139	1	0.456	0\\
2.32357213826582	2.26349110808509	1	0.448	0\\
2.32822240894928	2.25924709372012	1	0.44	0\\
2.33289930666302	2.25503244033762	1	0.432	0\\
2.33760264603037	2.25084731499244	1	0.424	0\\
2.34233224062661	2.24669188356903	1	0.416	0\\
2.34708790298631	2.24256631077487	1	0.408	0\\
2.35186944461084	2.23847076013396	1	0.4	0\\
2.35667667597579	2.23440539398031	1	0.392	0\\
2.36150940653849	2.23037037345151	1	0.384	0\\
2.36636744474555	2.22636585848237	1	0.376	0\\
2.37125059804051	2.22239200779854	1	0.368	0\\
2.37615867287139	2.21844897891026	1	0.36	0\\
2.38109147469843	2.21453692810607	1	0.352	0\\
2.38604880800177	2.21065601044668	1	0.344	0\\
2.39103047628919	2.20680637975874	1	0.336	0\\
2.39603628210393	2.20298818862883	1	0.328	0\\
2.40106602703248	2.19920158839734	1	0.32	0\\
2.40611951171248	2.19544672915252	1	0.312	0\\
2.4111965358406	2.19172375972451	1	0.304	0\\
2.41629689818047	2.18803282767943	1	0.296	0\\
2.4214203965707	2.18437407931356	1	0.288	0\\
2.42656682793284	2.18074765964753	1	0.28	0\\
2.43173598827946	2.17715371242055	1	0.272	0\\
2.43692767272224	2.17359238008474	1	0.264	0\\
2.44214167548005	2.17006380379948	1	0.256	0\\
2.44737778988716	2.16656812342579	1	0.248	0\\
2.45263580840142	2.16310547752083	1	0.24	0\\
2.45791552261244	2.15967600333237	1	0.232	0\\
2.46321672324991	2.15627983679336	1	0.224	0\\
2.46853920019186	2.15291711251655	1	0.216	0\\
2.473882742473	2.14958796378916	1	0.208	0\\
2.47924713829307	2.14629252256757	1	0.2	0\\
2.48463217502527	2.14303091947211	1	0.192	0\\
2.49003763922464	2.13980328378189	1	0.184	0\\
2.49546331663654	2.13660974342966	1	0.176	0\\
2.50090899220517	2.13345042499673	1	0.168	0\\
2.50637445008205	2.130325453708	1	0.16	0\\
2.5118594736346	2.12723495342692	1	0.152	0\\
2.51736384545473	2.12417904665066	1	0.144	0\\
2.52288734736743	2.12115785450519	1	0.136	0\\
2.52842976043945	2.11817149674055	1	0.128	0\\
2.53399086498797	2.115220091726	1	0.12	0\\
2.53957044058929	2.11230375644545	1	0.112	0\\
2.54516826608757	2.10942260649272	1	0.104	0\\
2.55078411960364	2.10657675606702	1	0.096	0\\
2.55641777854373	2.1037663179684	1	0.088	0\\
2.56206901960832	2.10099140359328	1	0.08	0\\
2.56773761880102	2.09825212293004	1	0.072	0\\
2.57342335143738	2.09554858455466	1	0.064	0\\
2.57912599215388	2.09288089562641	1	0.056	0\\
2.5848453149168	2.09024916188361	1	0.048	0\\
2.5905810930312	2.08765348763944	1	0.04	0\\
2.59633309914989	2.0850939757778	1	0.032	0\\
2.60210110528249	2.08257072774923	1	0.024	0\\
2.6078848828044	2.08008384356688	1	0.016	0\\
2.61368420246592	2.07763342180258	1	0.008	0\\
2.61949883440129	2.07521955958291	1	0	0\\
2.62532854813784	2.07284235258533	0.992	0	0\\
2.63117311260509	2.07050189503443	0.984	0	0\\
2.63703229614394	2.06819827969817	0.976	0	0\\
2.64290586651583	2.06593159788419	0.968	0	0\\
2.64879359091196	2.06370193943623	0.96	0	0\\
2.65469523596252	2.06150939273053	0.952	0	0\\
2.66061056774591	2.05935404467236	0.944	0	0\\
2.66653935179804	2.05723598069256	0.936	0	0\\
2.67248135312163	2.05515528474412	0.928	0	0\\
2.6784363361955	2.05311203929893	0.92	0	0\\
2.68440406498393	2.05110632534444	0.912	0	0\\
2.69038430294597	2.04913822238048	0.904	0	0\\
2.69637681304488	2.04720780841612	0.896	0	0\\
2.70238135775749	2.04531515996653	0.888	0	0\\
2.7083976990836	2.04346035205003	0.88	0	0\\
2.71442559855544	2.04164345818501	0.872	0	0\\
2.72046481724713	2.03986455038712	0.864	0	0\\
2.72651511578411	2.03812369916635	0.856	0	0\\
2.73257625435266	2.03642097352426	0.848	0	0\\
2.73864799270939	2.03475644095121	0.84	0	0\\
2.7447300901908	2.03313016742376	0.832	0	0\\
2.75082230572275	2.03154221740198	0.824	0	0\\
2.75692439783007	2.02999265382693	0.816	0	0\\
2.76303612464613	2.02848153811815	0.808	0	0\\
2.76915724392238	2.02700893017125	0.8	0	0\\
2.77528751303802	2.02557488835552	0.792	0	0\\
2.78142668900955	2.0241794695116	0.784	0	0\\
2.78757452850045	2.02282272894925	0.776	0	0\\
2.7937307878308	2.02150472044516	0.768	0	0\\
2.79989522298696	2.02022549624081	0.76	0	0\\
2.80606758963119	2.0189851070404	0.752	0	0\\
2.81224764311142	2.01778360200881	0.744	0	0\\
2.81843513847087	2.01662102876973	0.736	0	0\\
2.82462983045779	2.01549743340368	0.728	0	0\\
2.83083147353518	2.01441286044626	0.72	0	0\\
2.83703982189054	2.01336735288633	0.712	0	0\\
2.84325462944558	2.01236095216433	0.704	0	0\\
2.84947564986599	2.01139369817064	0.696	0	0\\
2.85570263657119	2.01046562924399	0.688	0	0\\
2.86193534274415	2.00957678216995	0.68	0	0\\
2.8681735213411	2.00872719217947	0.672	0	0\\
2.87441692510139	2.00791689294747	0.664	0	0\\
2.88066530655725	2.0071459165915	0.656	0	0\\
2.88691841804361	2.00641429367053	0.648	0	0\\
2.89317601170793	2.00572205318363	0.64	0	0\\
2.89943783951999	2.00506922256893	0.632	0	0\\
2.90570365328176	2.00445582770246	0.624	0	0\\
2.91197320463721	2.00388189289715	0.616	0	0\\
2.91824624508218	2.00334744090188	0.608	0	0\\
2.92452252597418	2.00285249290053	0.6	0	0\\
2.93080179854233	2.00239706851121	0.592	0	0\\
2.93708381389711	2.00198118578542	0.584	0	0\\
2.94336832304035	2.00160486120738	0.576	0	0\\
2.94965507687497	2.00126810969333	0.568	0	0\\
2.95594382621497	2.00097094459099	0.56	0	0\\
2.96223432179523	2.00071337767899	0.552	0	0\\
2.96852631428141	2.00049541916643	0.544	0	0\\
2.97481955427986	2.00031707769246	0.536	0	0\\
2.98111379234746	2.00017836032595	0.528	0	0\\
2.98740877900154	2.00007927256519	0.52	0	0\\
2.99370426472976	2.00001981833768	0.512	0	0\\
3	2	0.504	0	0\\
};
\addplot[scatter, only marks, mark=x] table[row sep=crcr]{%
x	y	R	G	B\\
1.2	3.7	0	0	0.508064516129032\\
1.2	3.7	0	0	0.516129032258065\\
1.2	3.7	0	0	0.524193548387097\\
3	1.5	0	0	0.532258064516129\\
3	1.7	0	0	0.540322580645161\\
3	1.8	0	0	0.548387096774194\\
3	1.8	0	0	0.556451612903226\\
3	1.9	0	0	0.564516129032258\\
3	1.9	0	0	0.57258064516129\\
3	1.9	0	0	0.580645161290323\\
3	1.9	0	0	0.588709677419355\\
3	1.9	0	0	0.596774193548387\\
3	1.9	0	0	0.604838709677419\\
3	1.9	0	0	0.612903225806452\\
3	1.9	0	0	0.620967741935484\\
3	1.9	0	0	0.629032258064516\\
3	1.9	0	0	0.637096774193548\\
3	1.9	0	0	0.645161290322581\\
3	1.9	0	0	0.653225806451613\\
2.9	3.8	0	0	0.661290322580645\\
2.9	3.8	0	0	0.669354838709677\\
2.9	3.8	0	0	0.67741935483871\\
2.9	3.8	0	0	0.685483870967742\\
2.9	3.8	0	0	0.693548387096774\\
2.9	3.8	0	0	0.701612903225806\\
3	1.9	0	0	0.709677419354839\\
3	1.9	0	0	0.717741935483871\\
3.1	2	0	0	0.725806451612903\\
3.1	2	0	0	0.733870967741935\\
3.1	2	0	0	0.741935483870968\\
3.1	2	0	0	0.75\\
3.1	2	0	0	0.758064516129032\\
3.1	2	0	0	0.766129032258065\\
3.1	2	0	0	0.774193548387097\\
3.1	2	0	0	0.782258064516129\\
3.1	2	0	0	0.790322580645161\\
3.1	2	0	0	0.798387096774194\\
3.1	2	0	0	0.806451612903226\\
3.1	2	0	0	0.814516129032258\\
3.1	2	0	0	0.82258064516129\\
3.1	2	0	0	0.830645161290323\\
3.1	2	0	0	0.838709677419355\\
3.1	2	0	0	0.846774193548387\\
2.9	3.9	0	0	0.854838709677419\\
2.9	3.9	0	0	0.862903225806452\\
2.9	3.9	0	0	0.870967741935484\\
2.9	3.9	0	0	0.879032258064516\\
2.9	3.9	0	0	0.887096774193548\\
2.9	3.9	0	0	0.895161290322581\\
2.9	3.9	0	0	0.903225806451613\\
2.9	3.9	0	0	0.911290322580645\\
2.9	3.9	0	0	0.919354838709677\\
2.9	3.9	0	0	0.92741935483871\\
2.9	3.9	0	0	0.935483870967742\\
2.9	3.9	0	0	0.943548387096774\\
2.9	3.9	0	0	0.951612903225806\\
2.9	3.9	0	0	0.959677419354839\\
2.9	3.9	0	0	0.967741935483871\\
2.9	3.9	0	0	0.975806451612903\\
2.9	3.9	0	0	0.983870967741935\\
2.8	3.9	0	0	0.991935483870968\\
2.8	3.9	0	0	1\\
2.8	3.9	0	0.00806451612903226	1\\
2.8	3.9	0	0.0161290322580645	1\\
2.8	3.9	0	0.0241935483870968	1\\
2.8	3.9	0	0.032258064516129	1\\
2.8	3.9	0	0.0403225806451613	1\\
2.8	3.9	0	0.0483870967741935	1\\
2.8	3.9	0	0.0564516129032258	1\\
2.8	3.9	0	0.0645161290322581	1\\
2.8	3.9	0	0.0725806451612903	1\\
2.8	3.9	0	0.0806451612903226	1\\
2.8	3.9	0	0.0887096774193548	1\\
2.8	3.9	0	0.0967741935483871	1\\
2.8	3.9	0	0.104838709677419	1\\
2.8	3.9	0	0.112903225806452	1\\
2.8	3.9	0	0.120967741935484	1\\
2.8	3.9	0	0.129032258064516	1\\
2.8	3.9	0	0.137096774193548	1\\
2.8	3.9	0	0.145161290322581	1\\
2.8	3.9	0	0.153225806451613	1\\
2.8	3.9	0	0.161290322580645	1\\
2.8	3.9	0	0.169354838709677	1\\
2.8	3.9	0	0.17741935483871	1\\
2.8	3.9	0	0.185483870967742	1\\
2.8	3.9	0	0.193548387096774	1\\
2.8	3.9	0	0.201612903225806	1\\
2.7	3.9	0	0.209677419354839	1\\
2.7	3.9	0	0.217741935483871	1\\
2.7	3.9	0	0.225806451612903	1\\
2.7	3.9	0	0.233870967741935	1\\
2.7	3.9	0	0.241935483870968	1\\
2.7	3.9	0	0.25	1\\
2.7	3.9	0	0.258064516129032	1\\
2.7	3.9	0	0.266129032258065	1\\
2.7	3.9	0	0.274193548387097	1\\
2.7	3.9	0	0.282258064516129	1\\
2.7	3.9	0	0.290322580645161	1\\
2.7	3.9	0	0.298387096774194	1\\
2.7	3.9	0	0.306451612903226	1\\
2.7	3.9	0	0.314516129032258	1\\
2.7	3.9	0	0.32258064516129	1\\
2.7	3.9	0	0.330645161290323	1\\
2.7	3.9	0	0.338709677419355	1\\
2.7	3.9	0	0.346774193548387	1\\
2.7	3.9	0	0.354838709677419	1\\
2.7	3.9	0	0.362903225806452	1\\
2.7	3.9	0	0.370967741935484	1\\
2.7	3.9	0	0.379032258064516	1\\
2.7	3.8	0	0.387096774193548	1\\
2.7	3.8	0	0.395161290322581	1\\
2.7	3.8	0	0.403225806451613	1\\
2.7	3.8	0	0.411290322580645	1\\
2.7	3.9	0	0.419354838709677	1\\
2.7	3.9	0	0.42741935483871	1\\
2.7	3.9	0	0.435483870967742	1\\
2.7	3.9	0	0.443548387096774	1\\
2.7	3.9	0	0.451612903225806	1\\
1.2	2.3	0	0.459677419354839	1\\
1.2	2.3	0	0.467741935483871	1\\
1.2	2.3	0	0.475806451612903	1\\
1.2	2.3	0	0.483870967741935	1\\
1.2	2.3	0	0.491935483870968	1\\
1.2	2.3	0	0.5	1\\
1.2	2.3	0	0.508064516129032	1\\
1.2	2.3	0	0.516129032258065	1\\
1.2	2.3	0	0.524193548387097	1\\
1.2	2.3	0	0.532258064516129	1\\
1.2	2.3	0	0.540322580645161	1\\
1.2	2.3	0	0.548387096774194	1\\
1.2	2.3	0	0.556451612903226	1\\
1.2	2.3	0	0.564516129032258	1\\
1.2	2.3	0	0.57258064516129	1\\
1.2	2.3	0	0.580645161290323	1\\
1.2	2.3	0	0.588709677419355	1\\
1.2	2.3	0	0.596774193548387	1\\
1.2	2.3	0	0.604838709677419	1\\
1.2	2.3	0	0.612903225806452	1\\
1.2	2.3	0	0.620967741935484	1\\
1.2	2.3	0	0.629032258064516	1\\
2.2	1.2	0	0.637096774193548	1\\
2.2	1.2	0	0.645161290322581	1\\
2.2	1.2	0	0.653225806451613	1\\
2.2	1.2	0	0.661290322580645	1\\
2.2	1.2	0	0.669354838709677	1\\
2.2	1.2	0	0.67741935483871	1\\
2.2	1.2	0	0.685483870967742	1\\
2.2	1.2	0	0.693548387096774	1\\
2.2	1.2	0	0.701612903225806	1\\
2.4	3.7	0	0.709677419354839	1\\
2.4	3.7	0	0.717741935483871	1\\
2.4	3.7	0	0.725806451612903	1\\
2.4	3.7	0	0.733870967741935	1\\
2.4	3.7	0	0.741935483870968	1\\
2.3	3.6	0	0.75	1\\
2.3	3.6	0	0.758064516129032	1\\
2.3	3.6	0	0.766129032258065	1\\
2.3	3.6	0	0.774193548387097	1\\
2.3	3.6	0	0.782258064516129	1\\
2.3	3.6	0	0.790322580645161	1\\
2.3	3.6	0	0.798387096774194	1\\
2.3	3.6	0	0.806451612903226	1\\
2.3	3.6	0	0.814516129032258	1\\
2.3	3.6	0	0.82258064516129	1\\
2.3	3.6	0	0.830645161290323	1\\
2.3	3.6	0	0.838709677419355	1\\
2.3	3.6	0	0.846774193548387	1\\
2.3	3.6	0	0.854838709677419	1\\
2.3	3.6	0	0.862903225806452	1\\
2.3	3.6	0	0.870967741935484	1\\
2.3	3.6	0	0.879032258064516	1\\
2.2	3.5	0	0.887096774193548	1\\
2.2	3.5	0	0.895161290322581	1\\
2.2	3.5	0	0.903225806451613	1\\
2.2	3.5	0	0.911290322580645	1\\
2.2	3.5	0	0.919354838709677	1\\
2.2	3.5	0	0.92741935483871	1\\
2.2	3.5	0	0.935483870967742	1\\
2.2	3.5	0	0.943548387096774	1\\
2.2	3.5	0	0.951612903225806	1\\
2.2	3.5	0	0.959677419354839	1\\
2.2	3.5	0	0.967741935483871	1\\
2.2	3.5	0	0.975806451612903	1\\
2.2	3.5	0	0.983870967741935	1\\
2.2	3.5	0	0.991935483870968	1\\
2.2	3.5	0	1	1\\
2.2	3.5	0.00806451612903226	1	0.991935483870968\\
2.2	3.5	0.0161290322580645	1	0.983870967741935\\
2.2	3.5	0.0241935483870968	1	0.975806451612903\\
2.2	3.5	0.032258064516129	1	0.967741935483871\\
2.2	3.5	0.0403225806451613	1	0.959677419354839\\
2.2	3.5	0.0483870967741935	1	0.951612903225806\\
2.1	3.4	0.0564516129032258	1	0.943548387096774\\
2.1	3.4	0.0645161290322581	1	0.935483870967742\\
2.1	3.4	0.0725806451612903	1	0.92741935483871\\
2.1	3.4	0.0806451612903226	1	0.919354838709677\\
2.1	3.4	0.0887096774193548	1	0.911290322580645\\
2.1	3.4	0.0967741935483871	1	0.903225806451613\\
2.1	3.4	0.104838709677419	1	0.895161290322581\\
2.1	3.4	0.112903225806452	1	0.887096774193548\\
2.1	3.4	0.120967741935484	1	0.879032258064516\\
2.1	3.4	0.129032258064516	1	0.870967741935484\\
2.1	3.4	0.137096774193548	1	0.862903225806452\\
2.1	3.4	0.145161290322581	1	0.854838709677419\\
2.1	3.4	0.153225806451613	1	0.846774193548387\\
2.1	3.4	0.161290322580645	1	0.838709677419355\\
2.1	3.4	0.169354838709677	1	0.830645161290323\\
2.1	3.4	0.17741935483871	1	0.82258064516129\\
2.1	3.4	0.185483870967742	1	0.814516129032258\\
2.1	3.4	0.193548387096774	1	0.806451612903226\\
2.1	3.4	0.201612903225806	1	0.798387096774194\\
2.1	3.4	0.209677419354839	1	0.790322580645161\\
2.1	3.4	0.217741935483871	1	0.782258064516129\\
2.1	3.4	0.225806451612903	1	0.774193548387097\\
2.1	3.4	0.233870967741935	1	0.766129032258065\\
4	2.6	0.241935483870968	1	0.758064516129032\\
4	2.6	0.25	1	0.75\\
4	2.6	0.258064516129032	1	0.741935483870968\\
4	2.6	0.266129032258065	1	0.733870967741935\\
4	2.6	0.274193548387097	1	0.725806451612903\\
2.1	3.4	0.282258064516129	1	0.717741935483871\\
2.1	3.3	0.290322580645161	1	0.709677419354839\\
2.1	3.3	0.298387096774194	1	0.701612903225806\\
2.1	3.3	0.306451612903226	1	0.693548387096774\\
2.1	3.3	0.314516129032258	1	0.685483870967742\\
2.1	3.3	0.32258064516129	1	0.67741935483871\\
2.1	3.3	0.330645161290323	1	0.669354838709677\\
2.1	3.3	0.338709677419355	1	0.661290322580645\\
2.1	3.3	0.346774193548387	1	0.653225806451613\\
2.1	3.3	0.354838709677419	1	0.645161290322581\\
2.1	3.3	0.362903225806452	1	0.637096774193548\\
2.1	3.3	0.370967741935484	1	0.629032258064516\\
2.1	3.3	0.379032258064516	1	0.620967741935484\\
2.1	3.3	0.387096774193548	1	0.612903225806452\\
2.1	3.3	0.395161290322581	1	0.604838709677419\\
2.1	3.3	0.403225806451613	1	0.596774193548387\\
4	2.7	0.411290322580645	1	0.588709677419355\\
4	2.7	0.419354838709677	1	0.580645161290323\\
4	2.7	0.42741935483871	1	0.57258064516129\\
4	2.7	0.435483870967742	1	0.564516129032258\\
4	2.7	0.443548387096774	1	0.556451612903226\\
4	2.7	0.451612903225806	1	0.548387096774194\\
4	2.7	0.459677419354839	1	0.540322580645161\\
4	2.8	0.467741935483871	1	0.532258064516129\\
4	2.8	0.475806451612903	1	0.524193548387097\\
4	2.8	0.483870967741935	1	0.516129032258065\\
4	2.8	0.491935483870968	1	0.508064516129032\\
4	2.8	0.5	1	0.5\\
4	2.8	0.508064516129032	1	0.491935483870968\\
4	2.8	0.516129032258065	1	0.483870967741935\\
4	2.8	0.524193548387097	1	0.475806451612903\\
4	2.8	0.532258064516129	1	0.467741935483871\\
4	2.8	0.540322580645161	1	0.459677419354839\\
4	2.8	0.548387096774194	1	0.451612903225806\\
4	2.8	0.556451612903226	1	0.443548387096774\\
4	2.8	0.564516129032258	1	0.435483870967742\\
4	2.8	0.57258064516129	1	0.42741935483871\\
3.9	2.9	0.580645161290323	1	0.419354838709677\\
3.9	2.9	0.588709677419355	1	0.411290322580645\\
3.9	2.9	0.596774193548387	1	0.403225806451613\\
3.9	2.9	0.604838709677419	1	0.395161290322581\\
3.9	2.9	0.612903225806452	1	0.387096774193548\\
3.9	2.9	0.620967741935484	1	0.379032258064516\\
3.9	2.9	0.629032258064516	1	0.370967741935484\\
4	2.9	0.637096774193548	1	0.362903225806452\\
4	2.9	0.645161290322581	1	0.354838709677419\\
4	2.8	0.653225806451613	1	0.346774193548387\\
4	2.8	0.661290322580645	1	0.338709677419355\\
4	2.9	0.669354838709677	1	0.330645161290323\\
4	2.9	0.67741935483871	1	0.32258064516129\\
4	2.9	0.685483870967742	1	0.314516129032258\\
4	2.9	0.693548387096774	1	0.306451612903226\\
4	2.9	0.701612903225806	1	0.298387096774194\\
4	2.9	0.709677419354839	1	0.290322580645161\\
4	2.9	0.717741935483871	1	0.282258064516129\\
4	2.9	0.725806451612903	1	0.274193548387097\\
2.1	2.9	0.733870967741935	1	0.266129032258065\\
2.1	2.8	0.741935483870968	1	0.258064516129032\\
2.1	2.8	0.75	1	0.25\\
2.1	2.8	0.758064516129032	1	0.241935483870968\\
2.1	2.8	0.766129032258065	1	0.233870967741935\\
2.1	2.8	0.774193548387097	1	0.225806451612903\\
2.1	2.8	0.782258064516129	1	0.217741935483871\\
2.1	2.8	0.790322580645161	1	0.209677419354839\\
2.1	2.8	0.798387096774194	1	0.201612903225806\\
2.1	2.8	0.806451612903226	1	0.193548387096774\\
2.1	2.8	0.814516129032258	1	0.185483870967742\\
2.1	2.8	0.82258064516129	1	0.17741935483871\\
2.1	2.8	0.830645161290323	1	0.169354838709677\\
2.1	2.8	0.838709677419355	1	0.161290322580645\\
2.1	2.8	0.846774193548387	1	0.153225806451613\\
2.1	2.8	0.854838709677419	1	0.145161290322581\\
2.1	2.8	0.862903225806452	1	0.137096774193548\\
2.1	2.8	0.870967741935484	1	0.129032258064516\\
2.1	2.8	0.879032258064516	1	0.120967741935484\\
2.1	2.8	0.887096774193548	1	0.112903225806452\\
2.1	2.8	0.895161290322581	1	0.104838709677419\\
2.1	2.8	0.903225806451613	1	0.0967741935483871\\
2.1	2.8	0.911290322580645	1	0.0887096774193548\\
2.1	2.8	0.919354838709677	1	0.0806451612903226\\
2.1	2.8	0.92741935483871	1	0.0725806451612903\\
2.1	2.8	0.935483870967742	1	0.0645161290322581\\
2.1	2.8	0.943548387096774	1	0.0564516129032258\\
2.1	2.8	0.951612903225806	1	0.0483870967741935\\
2.1	2.8	0.959677419354839	1	0.0403225806451613\\
2.1	2.8	0.967741935483871	1	0.032258064516129\\
2	2.7	0.975806451612903	1	0.0241935483870968\\
2.1	2.7	0.983870967741935	1	0.0161290322580645\\
2.1	2.7	0.991935483870968	1	0.00806451612903226\\
2.1	2.7	1	1	0\\
2.1	2.7	1	0.991935483870968	0\\
2.1	2.7	1	0.983870967741935	0\\
2.1	2.7	1	0.975806451612903	0\\
2.1	2.7	1	0.967741935483871	0\\
2.1	2.7	1	0.959677419354839	0\\
2.1	2.7	1	0.951612903225806	0\\
2.1	2.7	1	0.943548387096774	0\\
2.1	2.7	1	0.935483870967742	0\\
2.1	2.7	1	0.92741935483871	0\\
2.1	2.7	1	0.919354838709677	0\\
2.1	2.7	1	0.911290322580645	0\\
2.1	2.7	1	0.903225806451613	0\\
2.1	2.7	1	0.895161290322581	0\\
2.1	2.7	1	0.887096774193548	0\\
2.1	2.7	1	0.879032258064516	0\\
2.1	2.7	1	0.870967741935484	0\\
2.1	2.7	1	0.862903225806452	0\\
2.1	2.7	1	0.854838709677419	0\\
2.1	2.7	1	0.846774193548387	0\\
2.1	2.7	1	0.838709677419355	0\\
2.1	2.7	1	0.830645161290323	0\\
2.1	2.7	1	0.82258064516129	0\\
2.1	2.7	1	0.814516129032258	0\\
2.1	2.7	1	0.806451612903226	0\\
2.1	2.7	1	0.798387096774194	0\\
2.1	2.7	1	0.790322580645161	0\\
2.1	2.7	1	0.782258064516129	0\\
2.1	2.7	1	0.774193548387097	0\\
2.1	2.7	1	0.766129032258065	0\\
2.1	2.7	1	0.758064516129032	0\\
2.1	2.7	1	0.75	0\\
2.1	2.7	1	0.741935483870968	0\\
2.1	2.7	1	0.733870967741935	0\\
2.1	2.7	1	0.725806451612903	0\\
2.1	2.7	1	0.717741935483871	0\\
2.1	2.7	1	0.709677419354839	0\\
2.1	2.7	1	0.701612903225806	0\\
2.1	2.7	1	0.693548387096774	0\\
2.1	2.7	1	0.685483870967742	0\\
2.1	2.7	1	0.67741935483871	0\\
2.1	2.7	1	0.669354838709677	0\\
2.1	2.7	1	0.661290322580645	0\\
2.1	2.7	1	0.653225806451613	0\\
2.1	2.7	1	0.645161290322581	0\\
2.1	2.7	1	0.637096774193548	0\\
2.1	2.6	1	0.629032258064516	0\\
2.1	2.6	1	0.620967741935484	0\\
2.1	2.6	1	0.612903225806452	0\\
2.1	2.6	1	0.604838709677419	0\\
2.1	2.6	1	0.596774193548387	0\\
2.1	2.6	1	0.588709677419355	0\\
2.1	2.6	1	0.580645161290323	0\\
2.1	2.6	1	0.57258064516129	0\\
2.1	2.6	1	0.564516129032258	0\\
2.1	2.6	1	0.556451612903226	0\\
2.1	2.6	1	0.548387096774194	0\\
2.1	2.6	1	0.540322580645161	0\\
2.1	2.6	1	0.532258064516129	0\\
2.1	2.6	1	0.524193548387097	0\\
2.1	2.6	1	0.516129032258065	0\\
2.1	2.6	1	0.508064516129032	0\\
2.1	2.6	1	0.5	0\\
2.1	2.6	1	0.491935483870968	0\\
2.1	2.6	1	0.483870967741935	0\\
2.1	2.6	1	0.475806451612903	0\\
2.1	2.6	1	0.467741935483871	0\\
2.1	2.6	1	0.459677419354839	0\\
2.1	2.6	1	0.451612903225806	0\\
2.1	2.6	1	0.443548387096774	0\\
2.1	2.6	1	0.435483870967742	0\\
2.1	2.6	1	0.42741935483871	0\\
2.1	2.6	1	0.419354838709677	0\\
2.1	2.6	1	0.411290322580645	0\\
2.1	2.6	1	0.403225806451613	0\\
2.1	2.6	1	0.395161290322581	0\\
2.1	2.6	1	0.387096774193548	0\\
2.1	2.6	1	0.379032258064516	0\\
2.1	2.6	1	0.370967741935484	0\\
2.1	2.6	1	0.362903225806452	0\\
2.1	2.6	1	0.354838709677419	0\\
2.1	2.6	1	0.346774193548387	0\\
2.1	2.6	1	0.338709677419355	0\\
2.1	2.6	1	0.330645161290323	0\\
2.1	2.6	1	0.32258064516129	0\\
2.1	2.6	1	0.314516129032258	0\\
2.1	2.6	1	0.306451612903226	0\\
2.1	2.6	1	0.298387096774194	0\\
2.1	2.6	1	0.290322580645161	0\\
2.1	2.6	1	0.282258064516129	0\\
2.1	2.6	1	0.274193548387097	0\\
2.1	2.6	1	0.266129032258065	0\\
2.1	2.6	1	0.258064516129032	0\\
2.1	2.6	1	0.25	0\\
2.1	2.6	1	0.241935483870968	0\\
2.1	2.6	1	0.233870967741935	0\\
2.1	2.6	1	0.225806451612903	0\\
2.1	2.6	1	0.217741935483871	0\\
3.7	3.8	1	0.209677419354839	0\\
3.7	3.8	1	0.201612903225806	0\\
3.7	3.8	1	0.193548387096774	0\\
3.7	3.8	1	0.185483870967742	0\\
3.7	3.8	1	0.17741935483871	0\\
2.2	2.6	1	0.169354838709677	0\\
2.2	2.5	1	0.161290322580645	0\\
2.2	2.5	1	0.153225806451613	0\\
2.2	2.5	1	0.145161290322581	0\\
2.4	2.2	1	0.137096774193548	0\\
2.4	2.2	1	0.129032258064516	0\\
2.4	2.2	1	0.120967741935484	0\\
2.4	2.2	1	0.112903225806452	0\\
2.4	2.2	1	0.104838709677419	0\\
2.4	2.2	1	0.0967741935483871	0\\
2.5	2.1	1	0.0887096774193548	0\\
2.5	2.1	1	0.0806451612903226	0\\
2.5	2.1	1	0.0725806451612903	0\\
2.5	2.1	1	0.0645161290322581	0\\
2.5	2.1	1	0.0564516129032258	0\\
2.5	2.1	1	0.0483870967741935	0\\
2.5	2.1	1	0.0403225806451613	0\\
2.5	2.1	1	0.032258064516129	0\\
2.5	2.1	1	0.0241935483870968	0\\
2.5	2.1	1	0.0161290322580645	0\\
2.5	2.1	1	0.00806451612903226	0\\
2.5	2.1	1	0	0\\
2.5	2.1	0.991935483870968	0	0\\
2.5	2.1	0.983870967741935	0	0\\
2.5	2.1	0.975806451612903	0	0\\
2.5	2.1	0.967741935483871	0	0\\
2.5	2.1	0.959677419354839	0	0\\
2.5	2.1	0.951612903225806	0	0\\
2.5	2.1	0.943548387096774	0	0\\
2.5	2.1	0.935483870967742	0	0\\
2.5	2.1	0.92741935483871	0	0\\
2.5	2.1	0.919354838709677	0	0\\
2.5	2.1	0.911290322580645	0	0\\
2.5	2.1	0.903225806451613	0	0\\
2.5	2.1	0.895161290322581	0	0\\
2.5	2.1	0.887096774193548	0	0\\
2.5	2.1	0.879032258064516	0	0\\
2.5	2.1	0.870967741935484	0	0\\
2.5	2.1	0.862903225806452	0	0\\
2.5	2.1	0.854838709677419	0	0\\
2.5	2.1	0.846774193548387	0	0\\
2.5	2.1	0.838709677419355	0	0\\
2.5	2.1	0.830645161290323	0	0\\
2.5	2.1	0.82258064516129	0	0\\
2.5	2.1	0.814516129032258	0	0\\
2.5	2.1	0.806451612903226	0	0\\
2.5	2.1	0.798387096774194	0	0\\
2.5	2.1	0.790322580645161	0	0\\
2.5	2.1	0.782258064516129	0	0\\
2.5	2.1	0.774193548387097	0	0\\
2.5	2.1	0.766129032258065	0	0\\
2.5	2.1	0.758064516129032	0	0\\
2.5	2.1	0.75	0	0\\
2.5	2.1	0.741935483870968	0	0\\
2.5	2.1	0.733870967741935	0	0\\
2.5	2.1	0.725806451612903	0	0\\
2.5	2.1	0.717741935483871	0	0\\
2.5	2.1	0.709677419354839	0	0\\
2.5	2.1	0.701612903225806	0	0\\
2.5	2.1	0.693548387096774	0	0\\
2.5	2.1	0.685483870967742	0	0\\
2.5	2.1	0.67741935483871	0	0\\
2.5	2.1	0.669354838709677	0	0\\
2.5	2.1	0.661290322580645	0	0\\
2.5	2.1	0.653225806451613	0	0\\
2.5	2.1	0.645161290322581	0	0\\
2.5	2.1	0.637096774193548	0	0\\
2.5	2.1	0.629032258064516	0	0\\
2.5	2.1	0.620967741935484	0	0\\
2.5	2.1	0.612903225806452	0	0\\
2.5	2.1	0.604838709677419	0	0\\
2.5	2.1	0.596774193548387	0	0\\
2.5	2.1	0.588709677419355	0	0\\
2.5	2.1	0.580645161290323	0	0\\
2.5	2.1	0.57258064516129	0	0\\
2.5	2.1	0.564516129032258	0	0\\
2.5	2.1	0.556451612903226	0	0\\
2.5	2.1	0.548387096774194	0	0\\
2.5	2.1	0.540322580645161	0	0\\
2.5	2.1	0.532258064516129	0	0\\
2.5	2.1	0.524193548387097	0	0\\
2.5	2.1	0.516129032258065	0	0\\
2.5	2.1	0.508064516129032	0	0\\
2.5	2.1	0.5	0	0\\
};
\addplot[scatter, only marks, mark=x] table[row sep=crcr]{%
x	y	R	G	B\\
3.9	1.2	0	0	0.508064516129032\\
3.9	1.2	0	0	0.516129032258065\\
3.9	1.2	0	0	0.524193548387097\\
1.2	3.7	0	0	0.532258064516129\\
1.2	3.7	0	0	0.540322580645161\\
1.2	3.7	0	0	0.548387096774194\\
1.2	3.7	0	0	0.556451612903226\\
2.9	1.3	0	0	0.564516129032258\\
2.9	1.3	0	0	0.57258064516129\\
2.9	1.3	0	0	0.580645161290323\\
2.9	1.3	0	0	0.588709677419355\\
2.9	1.3	0	0	0.596774193548387\\
2.9	1.3	0	0	0.604838709677419\\
2.9	1.3	0	0	0.612903225806452\\
3	3.8	0	0	0.620967741935484\\
3	3.8	0	0	0.629032258064516\\
3	3.8	0	0	0.637096774193548\\
3	3.8	0	0	0.645161290322581\\
2.9	3.8	0	0	0.653225806451613\\
3	1.9	0	0	0.661290322580645\\
3	1.9	0	0	0.669354838709677\\
3	1.9	0	0	0.67741935483871\\
3	1.9	0	0	0.685483870967742\\
3	1.9	0	0	0.693548387096774\\
3	1.9	0	0	0.701612903225806\\
2.9	3.8	0	0	0.709677419354839\\
2.9	3.8	0	0	0.717741935483871\\
2.9	3.8	0	0	0.725806451612903\\
2.9	3.8	0	0	0.733870967741935\\
2.9	3.8	0	0	0.741935483870968\\
2.9	3.8	0	0	0.75\\
2.9	3.8	0	0	0.758064516129032\\
2.9	3.8	0	0	0.766129032258065\\
2.9	3.8	0	0	0.774193548387097\\
2.9	3.8	0	0	0.782258064516129\\
2.9	3.8	0	0	0.790322580645161\\
2.9	3.9	0	0	0.798387096774194\\
2.9	3.9	0	0	0.806451612903226\\
2.9	3.9	0	0	0.814516129032258\\
2.9	3.9	0	0	0.82258064516129\\
2.9	3.9	0	0	0.830645161290323\\
2.9	3.9	0	0	0.838709677419355\\
2.9	3.9	0	0	0.846774193548387\\
3.1	2	0	0	0.854838709677419\\
3.1	2	0	0	0.862903225806452\\
3.1	2	0	0	0.870967741935484\\
3.1	2	0	0	0.879032258064516\\
3.1	2	0	0	0.887096774193548\\
3.1	2	0	0	0.895161290322581\\
3.1	2	0	0	0.903225806451613\\
3.1	2	0	0	0.911290322580645\\
3.1	2	0	0	0.919354838709677\\
3.1	2	0	0	0.92741935483871\\
3.1	2	0	0	0.935483870967742\\
3.1	2	0	0	0.943548387096774\\
3.1	2	0	0	0.951612903225806\\
3.1	2	0	0	0.959677419354839\\
3.1	2	0	0	0.967741935483871\\
3.1	2	0	0	0.975806451612903\\
3.1	2	0	0	0.983870967741935\\
3.1	2	0	0	0.991935483870968\\
3.1	2	0	0	1\\
3.1	2	0	0.00806451612903226	1\\
3.1	2	0	0.0161290322580645	1\\
3.1	2	0	0.0241935483870968	1\\
3.1	2	0	0.032258064516129	1\\
3.1	2	0	0.0403225806451613	1\\
3.1	2	0	0.0483870967741935	1\\
3.1	2	0	0.0564516129032258	1\\
3.1	2	0	0.0645161290322581	1\\
3.1	2	0	0.0725806451612903	1\\
3.1	2	0	0.0806451612903226	1\\
2.2	1.2	0	0.0887096774193548	1\\
3.1	2	0	0.0967741935483871	1\\
1.2	2.3	0	0.104838709677419	1\\
1.2	2.3	0	0.112903225806452	1\\
1.2	2.3	0	0.120967741935484	1\\
1.2	2.3	0	0.129032258064516	1\\
2.2	1.2	0	0.137096774193548	1\\
2.2	1.2	0	0.145161290322581	1\\
2.2	1.2	0	0.153225806451613	1\\
2.2	1.2	0	0.161290322580645	1\\
2.2	1.2	0	0.169354838709677	1\\
2.2	1.2	0	0.17741935483871	1\\
2.2	1.2	0	0.185483870967742	1\\
2.2	1.2	0	0.193548387096774	1\\
2.2	1.2	0	0.201612903225806	1\\
2.2	1.2	0	0.209677419354839	1\\
2.2	1.2	0	0.217741935483871	1\\
2.2	1.2	0	0.225806451612903	1\\
2.2	1.2	0	0.233870967741935	1\\
2.2	1.2	0	0.241935483870968	1\\
2.2	1.2	0	0.25	1\\
2.2	1.2	0	0.258064516129032	1\\
2.2	1.2	0	0.266129032258065	1\\
2.2	1.2	0	0.274193548387097	1\\
2.2	1.2	0	0.282258064516129	1\\
2.2	1.2	0	0.290322580645161	1\\
1.2	2.3	0	0.298387096774194	1\\
1.2	2.3	0	0.306451612903226	1\\
1.2	2.3	0	0.314516129032258	1\\
1.2	2.3	0	0.32258064516129	1\\
1.2	2.3	0	0.330645161290323	1\\
1.2	2.3	0	0.338709677419355	1\\
1.2	2.3	0	0.346774193548387	1\\
1.2	2.3	0	0.354838709677419	1\\
1.2	2.3	0	0.362903225806452	1\\
1.2	2.3	0	0.370967741935484	1\\
1.2	2.3	0	0.379032258064516	1\\
1.2	2.3	0	0.387096774193548	1\\
1.2	2.3	0	0.395161290322581	1\\
1.2	2.3	0	0.403225806451613	1\\
2.2	1.2	0	0.411290322580645	1\\
1.2	2.3	0	0.419354838709677	1\\
2.2	1.2	0	0.42741935483871	1\\
1.2	2.3	0	0.435483870967742	1\\
1.2	2.3	0	0.443548387096774	1\\
1.2	2.3	0	0.451612903225806	1\\
2.7	3.9	0	0.459677419354839	1\\
2.7	3.9	0	0.467741935483871	1\\
2.7	3.9	0	0.475806451612903	1\\
2.7	3.9	0	0.483870967741935	1\\
2.2	1.2	0	0.491935483870968	1\\
2.2	1.2	0	0.5	1\\
2.2	1.2	0	0.508064516129032	1\\
2.2	1.2	0	0.516129032258065	1\\
2.2	1.2	0	0.524193548387097	1\\
2.2	1.2	0	0.532258064516129	1\\
2.2	1.2	0	0.540322580645161	1\\
2.2	1.2	0	0.548387096774194	1\\
2.2	1.2	0	0.556451612903226	1\\
2.2	1.2	0	0.564516129032258	1\\
3.9	1.2	0	0.57258064516129	1\\
2.2	1.2	0	0.580645161290323	1\\
2.2	1.2	0	0.588709677419355	1\\
2.2	1.2	0	0.596774193548387	1\\
2.2	1.2	0	0.604838709677419	1\\
2.2	1.2	0	0.612903225806452	1\\
2.2	1.2	0	0.620967741935484	1\\
2.2	1.2	0	0.629032258064516	1\\
1.2	2.3	0	0.637096774193548	1\\
1.2	2.3	0	0.645161290322581	1\\
1.2	2.3	0	0.653225806451613	1\\
1.2	2.3	0	0.661290322580645	1\\
1.2	2.3	0	0.669354838709677	1\\
1.2	2.3	0	0.67741935483871	1\\
1.2	2.3	0	0.685483870967742	1\\
1.2	2.3	0	0.693548387096774	1\\
1.2	2.3	0	0.701612903225806	1\\
1.2	2.3	0	0.709677419354839	1\\
1.2	2.3	0	0.717741935483871	1\\
1.2	2.3	0	0.725806451612903	1\\
1.2	2.3	0	0.733870967741935	1\\
1.2	2.3	0	0.741935483870968	1\\
1.2	2.3	0	0.75	1\\
1.2	2.3	0	0.758064516129032	1\\
1.2	2.3	0	0.766129032258065	1\\
1.2	2.3	0	0.774193548387097	1\\
1.2	2.3	0	0.782258064516129	1\\
1.2	3.7	0	0.790322580645161	1\\
1.2	3.7	0	0.798387096774194	1\\
1.2	3.7	0	0.806451612903226	1\\
1.2	3.7	0	0.814516129032258	1\\
1.2	3.7	0	0.82258064516129	1\\
1.2	2.3	0	0.830645161290323	1\\
3.9	2.4	0	0.838709677419355	1\\
3.9	2.4	0	0.846774193548387	1\\
3.9	2.4	0	0.854838709677419	1\\
3.8	2.4	0	0.862903225806452	1\\
3.8	2.4	0	0.870967741935484	1\\
3.9	2.4	0	0.879032258064516	1\\
3.9	2.4	0	0.887096774193548	1\\
3.9	2.4	0	0.895161290322581	1\\
3.9	2.5	0	0.903225806451613	1\\
3.9	2.5	0	0.911290322580645	1\\
3.9	2.5	0	0.919354838709677	1\\
3.9	2.5	0	0.92741935483871	1\\
3.9	2.5	0	0.935483870967742	1\\
3.9	2.5	0	0.943548387096774	1\\
3.9	2.5	0	0.951612903225806	1\\
3.9	2.5	0	0.959677419354839	1\\
3.9	2.5	0	0.967741935483871	1\\
3.9	2.5	0	0.975806451612903	1\\
3.9	2.5	0	0.983870967741935	1\\
3.9	2.5	0	0.991935483870968	1\\
3.9	2.5	0	1	1\\
3.9	2.5	0.00806451612903226	1	0.991935483870968\\
3.9	2.5	0.0161290322580645	1	0.983870967741935\\
3.9	2.5	0.0241935483870968	1	0.975806451612903\\
3.9	2.5	0.032258064516129	1	0.967741935483871\\
1.2	2.3	0.0403225806451613	1	0.959677419354839\\
3.9	2.5	0.0483870967741935	1	0.951612903225806\\
3.9	2.5	0.0564516129032258	1	0.943548387096774\\
3.9	2.5	0.0645161290322581	1	0.935483870967742\\
3.9	2.5	0.0725806451612903	1	0.92741935483871\\
1.2	2.3	0.0806451612903226	1	0.919354838709677\\
1.2	2.3	0.0887096774193548	1	0.911290322580645\\
1.2	2.3	0.0967741935483871	1	0.903225806451613\\
1.2	2.3	0.104838709677419	1	0.895161290322581\\
1.2	2.3	0.112903225806452	1	0.887096774193548\\
1.2	2.3	0.120967741935484	1	0.879032258064516\\
1.2	2.3	0.129032258064516	1	0.870967741935484\\
1.2	2.3	0.137096774193548	1	0.862903225806452\\
1.2	2.3	0.145161290322581	1	0.854838709677419\\
1.2	2.3	0.153225806451613	1	0.846774193548387\\
1.2	2.3	0.161290322580645	1	0.838709677419355\\
1.2	2.3	0.169354838709677	1	0.830645161290323\\
4	2.5	0.17741935483871	1	0.82258064516129\\
4	2.5	0.185483870967742	1	0.814516129032258\\
4	2.6	0.193548387096774	1	0.806451612903226\\
4	2.6	0.201612903225806	1	0.798387096774194\\
4	2.6	0.209677419354839	1	0.790322580645161\\
4	2.6	0.217741935483871	1	0.782258064516129\\
4	2.6	0.225806451612903	1	0.774193548387097\\
4	2.6	0.233870967741935	1	0.766129032258065\\
2.1	3.4	0.241935483870968	1	0.758064516129032\\
2.1	3.4	0.25	1	0.75\\
2.1	3.4	0.258064516129032	1	0.741935483870968\\
2.1	3.4	0.266129032258065	1	0.733870967741935\\
2.1	3.4	0.274193548387097	1	0.725806451612903\\
4	2.6	0.282258064516129	1	0.717741935483871\\
4	2.6	0.290322580645161	1	0.709677419354839\\
4	2.6	0.298387096774194	1	0.701612903225806\\
4	2.6	0.306451612903226	1	0.693548387096774\\
4	2.6	0.314516129032258	1	0.685483870967742\\
4	2.6	0.32258064516129	1	0.67741935483871\\
1.2	2.3	0.330645161290323	1	0.669354838709677\\
1.2	2.3	0.338709677419355	1	0.661290322580645\\
1.2	2.3	0.346774193548387	1	0.653225806451613\\
1.2	2.3	0.354838709677419	1	0.645161290322581\\
1.2	2.3	0.362903225806452	1	0.637096774193548\\
1.2	2.3	0.370967741935484	1	0.629032258064516\\
1.2	2.3	0.379032258064516	1	0.620967741935484\\
1.2	2.3	0.387096774193548	1	0.612903225806452\\
1.2	2.3	0.395161290322581	1	0.604838709677419\\
4	2.7	0.403225806451613	1	0.596774193548387\\
2.1	3.3	0.411290322580645	1	0.588709677419355\\
2.1	3.3	0.419354838709677	1	0.580645161290323\\
2.1	3.3	0.42741935483871	1	0.57258064516129\\
2.1	3.3	0.435483870967742	1	0.564516129032258\\
2.1	3.3	0.443548387096774	1	0.556451612903226\\
2.1	3.3	0.451612903225806	1	0.548387096774194\\
2.1	3.3	0.459677419354839	1	0.540322580645161\\
2.1	3.3	0.467741935483871	1	0.532258064516129\\
2.1	3.3	0.475806451612903	1	0.524193548387097\\
2.1	3.3	0.483870967741935	1	0.516129032258065\\
2.1	3.3	0.491935483870968	1	0.508064516129032\\
2.1	3.3	0.5	1	0.5\\
2.1	3.3	0.508064516129032	1	0.491935483870968\\
2.1	3.3	0.516129032258065	1	0.483870967741935\\
2.1	3.3	0.524193548387097	1	0.475806451612903\\
2.1	3.3	0.532258064516129	1	0.467741935483871\\
2.1	3.3	0.540322580645161	1	0.459677419354839\\
2.1	3.3	0.548387096774194	1	0.451612903225806\\
2.1	3.3	0.556451612903226	1	0.443548387096774\\
2.1	3.3	0.564516129032258	1	0.435483870967742\\
1.2	2.3	0.57258064516129	1	0.42741935483871\\
1.2	2.3	0.580645161290323	1	0.419354838709677\\
1.2	2.3	0.588709677419355	1	0.411290322580645\\
1.2	2.3	0.596774193548387	1	0.403225806451613\\
1.2	2.3	0.604838709677419	1	0.395161290322581\\
1.2	2.3	0.612903225806452	1	0.387096774193548\\
1.2	2.3	0.620967741935484	1	0.379032258064516\\
1.2	2.3	0.629032258064516	1	0.370967741935484\\
1.2	2.3	0.637096774193548	1	0.362903225806452\\
1.2	2.3	0.645161290322581	1	0.354838709677419\\
1.2	2.3	0.653225806451613	1	0.346774193548387\\
1.2	2.3	0.661290322580645	1	0.338709677419355\\
1.2	2.3	0.669354838709677	1	0.330645161290323\\
1.2	2.3	0.67741935483871	1	0.32258064516129\\
1.2	2.3	0.685483870967742	1	0.314516129032258\\
2.1	3.2	0.693548387096774	1	0.306451612903226\\
2.1	2.9	0.701612903225806	1	0.298387096774194\\
2.1	2.9	0.709677419354839	1	0.290322580645161\\
2.1	2.9	0.717741935483871	1	0.282258064516129\\
2.1	2.9	0.725806451612903	1	0.274193548387097\\
4	2.9	0.733870967741935	1	0.266129032258065\\
4	2.9	0.741935483870968	1	0.258064516129032\\
4	2.9	0.75	1	0.25\\
4	2.9	0.758064516129032	1	0.241935483870968\\
3.9	2.9	0.766129032258065	1	0.233870967741935\\
3.9	2.9	0.774193548387097	1	0.225806451612903\\
3.9	2.9	0.782258064516129	1	0.217741935483871\\
3.9	2.9	0.790322580645161	1	0.209677419354839\\
3.9	2.9	0.798387096774194	1	0.201612903225806\\
3.9	2.9	0.806451612903226	1	0.193548387096774\\
3.9	2.9	0.814516129032258	1	0.185483870967742\\
3.9	2.9	0.82258064516129	1	0.17741935483871\\
3.9	2.9	0.830645161290323	1	0.169354838709677\\
1.2	2.3	0.838709677419355	1	0.161290322580645\\
3.9	2.9	0.846774193548387	1	0.153225806451613\\
3.9	2.9	0.854838709677419	1	0.145161290322581\\
3.9	2.9	0.862903225806452	1	0.137096774193548\\
3.9	2.9	0.870967741935484	1	0.129032258064516\\
3.9	2.9	0.879032258064516	1	0.120967741935484\\
1.2	3	0.887096774193548	1	0.112903225806452\\
1.2	2.3	0.895161290322581	1	0.104838709677419\\
1.2	2.3	0.903225806451613	1	0.0967741935483871\\
1.2	2.3	0.911290322580645	1	0.0887096774193548\\
1.2	2.3	0.919354838709677	1	0.0806451612903226\\
1.2	2.3	0.92741935483871	1	0.0725806451612903\\
1.2	2.3	0.935483870967742	1	0.0645161290322581\\
1.2	2.3	0.943548387096774	1	0.0564516129032258\\
1.2	2.3	0.951612903225806	1	0.0483870967741935\\
1.2	2.3	0.959677419354839	1	0.0403225806451613\\
1.2	2.3	0.967741935483871	1	0.032258064516129\\
1.2	2.3	0.975806451612903	1	0.0241935483870968\\
1.2	2.3	0.983870967741935	1	0.0161290322580645\\
1.2	2.3	0.991935483870968	1	0.00806451612903226\\
1.2	2.3	1	1	0\\
1.2	2.3	1	0.991935483870968	0\\
1.2	2.3	1	0.983870967741935	0\\
1.2	2.3	1	0.975806451612903	0\\
1.2	2.3	1	0.967741935483871	0\\
1.2	2.3	1	0.959677419354839	0\\
1.2	2.3	1	0.951612903225806	0\\
1.2	2.3	1	0.943548387096774	0\\
1.2	2.3	1	0.935483870967742	0\\
1.2	2.3	1	0.92741935483871	0\\
1.2	2.3	1	0.919354838709677	0\\
1.2	2.3	1	0.911290322580645	0\\
1.2	2.3	1	0.903225806451613	0\\
1.2	2.3	1	0.895161290322581	0\\
1.2	2.3	1	0.887096774193548	0\\
1.2	2.3	1	0.879032258064516	0\\
1.2	2.3	1	0.870967741935484	0\\
1.2	2.3	1	0.862903225806452	0\\
1.2	2.3	1	0.854838709677419	0\\
1.2	2.3	1	0.846774193548387	0\\
1.2	2.3	1	0.838709677419355	0\\
1.2	2.3	1	0.830645161290323	0\\
1.2	2.3	1	0.82258064516129	0\\
1.2	2.3	1	0.814516129032258	0\\
2.4	2.1	1	0.806451612903226	0\\
2.4	2.1	1	0.798387096774194	0\\
2.3	1.2	1	0.790322580645161	0\\
3.9	2.9	1	0.782258064516129	0\\
3.9	2.9	1	0.774193548387097	0\\
3.9	2.9	1	0.766129032258065	0\\
3.9	2.9	1	0.758064516129032	0\\
4	2.9	1	0.75	0\\
4	2.9	1	0.741935483870968	0\\
4	2.9	1	0.733870967741935	0\\
4	2.9	1	0.725806451612903	0\\
3.9	3.7	1	0.717741935483871	0\\
3.9	3.7	1	0.709677419354839	0\\
3.9	3.7	1	0.701612903225806	0\\
3.9	3.7	1	0.693548387096774	0\\
3.9	3.7	1	0.685483870967742	0\\
3.9	3.7	1	0.67741935483871	0\\
3.9	3.7	1	0.669354838709677	0\\
3.9	3.7	1	0.661290322580645	0\\
3.9	3.7	1	0.653225806451613	0\\
3.9	3.7	1	0.645161290322581	0\\
3.9	3.7	1	0.637096774193548	0\\
3.9	3.7	1	0.629032258064516	0\\
3.9	3.7	1	0.620967741935484	0\\
3.9	3.7	1	0.612903225806452	0\\
3.9	3.7	1	0.604838709677419	0\\
3.9	3.7	1	0.596774193548387	0\\
3.9	3.7	1	0.588709677419355	0\\
3.9	3.7	1	0.580645161290323	0\\
3.9	3.7	1	0.57258064516129	0\\
3.9	3.7	1	0.564516129032258	0\\
3.9	3.7	1	0.556451612903226	0\\
3.9	3.7	1	0.548387096774194	0\\
3.9	3.7	1	0.540322580645161	0\\
3.9	3.7	1	0.532258064516129	0\\
3.9	3.7	1	0.524193548387097	0\\
3.9	3.7	1	0.516129032258065	0\\
3.9	3.7	1	0.508064516129032	0\\
3.9	3.7	1	0.5	0\\
4	2.9	1	0.491935483870968	0\\
3.9	3.7	1	0.483870967741935	0\\
3.8	3.7	1	0.475806451612903	0\\
3.8	3.7	1	0.467741935483871	0\\
3.8	3.7	1	0.459677419354839	0\\
3.8	3.7	1	0.451612903225806	0\\
3.7	3.7	1	0.443548387096774	0\\
3.7	3.7	1	0.435483870967742	0\\
3.7	3.7	1	0.42741935483871	0\\
3.7	3.7	1	0.419354838709677	0\\
3.7	3.7	1	0.411290322580645	0\\
3.7	3.7	1	0.403225806451613	0\\
3.7	3.7	1	0.395161290322581	0\\
3.7	3.7	1	0.387096774193548	0\\
3.7	3.7	1	0.379032258064516	0\\
3.7	3.7	1	0.370967741935484	0\\
3.7	3.7	1	0.362903225806452	0\\
3.7	3.7	1	0.354838709677419	0\\
3.7	3.7	1	0.346774193548387	0\\
3.7	3.7	1	0.338709677419355	0\\
3.7	3.7	1	0.330645161290323	0\\
3.7	3.7	1	0.32258064516129	0\\
3.7	3.7	1	0.314516129032258	0\\
3.7	3.7	1	0.306451612903226	0\\
3.7	3.7	1	0.298387096774194	0\\
3.7	3.7	1	0.290322580645161	0\\
3.7	3.7	1	0.282258064516129	0\\
3.7	3.7	1	0.274193548387097	0\\
3.7	3.7	1	0.266129032258065	0\\
3.7	3.7	1	0.258064516129032	0\\
3.7	3.7	1	0.25	0\\
3.7	3.7	1	0.241935483870968	0\\
3.7	3.8	1	0.233870967741935	0\\
3.7	3.8	1	0.225806451612903	0\\
3.7	3.8	1	0.217741935483871	0\\
2.1	2.6	1	0.209677419354839	0\\
2.1	2.6	1	0.201612903225806	0\\
2.1	2.6	1	0.193548387096774	0\\
2.1	2.6	1	0.185483870967742	0\\
2.1	2.6	1	0.17741935483871	0\\
3.7	3.8	1	0.169354838709677	0\\
3.7	3.8	1	0.161290322580645	0\\
3.7	3.8	1	0.153225806451613	0\\
2.4	2	1	0.145161290322581	0\\
2.2	2.7	1	0.137096774193548	0\\
3.7	3.8	1	0.129032258064516	0\\
3.7	3.8	1	0.120967741935484	0\\
3.8	3.8	1	0.112903225806452	0\\
3.8	3.8	1	0.104838709677419	0\\
3.7	3.8	1	0.0967741935483871	0\\
2.2	2.7	1	0.0887096774193548	0\\
2.2	2.7	1	0.0806451612903226	0\\
2.2	2.7	1	0.0725806451612903	0\\
2.2	2.7	1	0.0645161290322581	0\\
2.1	2.6	1	0.0564516129032258	0\\
2.1	2.6	1	0.0483870967741935	0\\
2.1	2.6	1	0.0403225806451613	0\\
2.1	2.6	1	0.032258064516129	0\\
2.1	2.6	1	0.0241935483870968	0\\
2.1	2.6	1	0.0161290322580645	0\\
2.1	2.6	1	0.00806451612903226	0\\
2.1	2.6	1	0	0\\
2.1	2.6	0.991935483870968	0	0\\
2.1	2.6	0.983870967741935	0	0\\
2.1	2.6	0.975806451612903	0	0\\
2.1	2.6	0.967741935483871	0	0\\
2.1	2.6	0.959677419354839	0	0\\
2.1	2.6	0.951612903225806	0	0\\
2.1	2.6	0.943548387096774	0	0\\
3.7	3.8	0.935483870967742	0	0\\
3.7	3.8	0.92741935483871	0	0\\
3.7	3.8	0.919354838709677	0	0\\
3.7	3.8	0.911290322580645	0	0\\
3.7	3.8	0.903225806451613	0	0\\
3.7	3.8	0.895161290322581	0	0\\
2.2	2.6	0.887096774193548	0	0\\
2.2	2.6	0.879032258064516	0	0\\
2.2	2.6	0.870967741935484	0	0\\
2.1	2.6	0.862903225806452	0	0\\
3.7	3.8	0.854838709677419	0	0\\
2.2	2.6	0.846774193548387	0	0\\
2.2	2.6	0.838709677419355	0	0\\
2.2	2.6	0.830645161290323	0	0\\
2.2	2.6	0.82258064516129	0	0\\
2.2	2.6	0.814516129032258	0	0\\
3.7	3.8	0.806451612903226	0	0\\
3.7	3.8	0.798387096774194	0	0\\
3.7	3.8	0.790322580645161	0	0\\
3.7	3.8	0.782258064516129	0	0\\
3.6	3.8	0.774193548387097	0	0\\
3.6	3.8	0.766129032258065	0	0\\
3.6	3.8	0.758064516129032	0	0\\
3.5	3.9	0.75	0	0\\
3.6	3.8	0.741935483870968	0	0\\
3.6	3.8	0.733870967741935	0	0\\
3.6	3.8	0.725806451612903	0	0\\
3.6	3.8	0.717741935483871	0	0\\
3.6	3.8	0.709677419354839	0	0\\
3.6	3.8	0.701612903225806	0	0\\
3.6	3.8	0.693548387096774	0	0\\
3.6	3.8	0.685483870967742	0	0\\
3.6	3.8	0.67741935483871	0	0\\
3.6	3.8	0.669354838709677	0	0\\
3.6	3.8	0.661290322580645	0	0\\
3.9	1.2	0.653225806451613	0	0\\
3.4	3.9	0.645161290322581	0	0\\
3.4	3.9	0.637096774193548	0	0\\
3.3	4	0.629032258064516	0	0\\
3.3	4	0.620967741935484	0	0\\
3.3	4	0.612903225806452	0	0\\
2.9	4	0.604838709677419	0	0\\
2.9	4	0.596774193548387	0	0\\
2.9	4	0.588709677419355	0	0\\
2.9	4	0.580645161290323	0	0\\
2.9	4	0.57258064516129	0	0\\
3.4	3.9	0.564516129032258	0	0\\
3.9	1.2	0.556451612903226	0	0\\
3.9	1.2	0.548387096774194	0	0\\
3.9	1.2	0.540322580645161	0	0\\
3.9	1.2	0.532258064516129	0	0\\
3.9	1.2	0.524193548387097	0	0\\
3.9	1.2	0.516129032258065	0	0\\
3.9	1.2	0.508064516129032	0	0\\
3.9	1.2	0.5	0	0\\
};
\end{axis}
\end{tikzpicture}%
		\caption{Estimated Positions \glsentryshort{crem}}
	\end{subfigure}
}
	\caption[Estimated Position for CREM and TREM (Arc)]{Estimated Position for CREM and TREM (Arc, \Tsixty$=0.4$~s).}
	\label{fig:trackingArcRoom}
\end{figure}


\FloatBarrier
\toggletrue{quick}

