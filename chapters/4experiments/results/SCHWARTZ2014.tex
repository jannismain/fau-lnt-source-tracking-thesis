\subsubsection{Replication of Experiments in \citeauthor{Schwartz2014} \citeyearpar{Schwartz2014}}
As already mentioned in Chapter \ref{chap:algorithms}, the algorithm developed in \cite{Schwartz2014} included the calculation of multiple Gaussian component weight vectors $\psi_{\bm p, s}\in \{\psi_{\bm p, 1}, \dots, \psi_{\bm, S}\}$, one for each source $s$. In their evaluation of static localisation of two sources, they initialised $\psi_{\bm p, 1}$ to have a uniform distribution across the left side of the room and $\psi_{\bm p, 2}$ to have a uniform distribution across the right side of the room. The variance has been initialised to $\sigma^2_s=1\ \forall\ s$. The complete parameter set, that was used in \cite{Schwartz2014}, is shown in Table \ref{table:parameters-schwartz14}. The position of the sensor nodes was not stated explicitely, but could be inferred from a diagram that displayed the setup, similar to Figure \ref{fig:setup}.				

\begin{table}[htb]
	\centering
	\begin{tabular}{lcccc}
		\toprule
		Parameter                      & Symbol              & Unit & Value                    \\%& Origin      \\
		\midrule
		Room                           &                     & m    & 6 x 6 x 6.1              \\%& environment \\
		Number of sources              & $S$                 &      & 2                        \\%& environment \\
		Distance between microphones   & $d_m$               & m    & 0.2                      \\%& environment \\
		Microphones distance from wall & wd$_m$              & m    & 1                        \\%& environment \\
		Receiver Positions             & $\bm p_m^i$         & m    & $\bm p_1^1$= [1.8, 1, 1] \\%& environment \\
		                               &                     &      & $\bm p_1^2=$ [2.0, 1, 1] \\%&             \\
		                               &                     &      & $\bm p_2^1=$ [2.4, 1, 1] \\%&             \\
		                               &                     &      & $\bm p_2^2=$ [2.6, 1, 1] \\%&             \\
		                               &                     &      & $\bm p_3^1=$ [3.6, 1, 1] \\%&             \\
		                               &                     &      & $\bm p_3^2=$ [3.8, 1, 1] \\%&             \\
		                               &                     &      & \vdots                   \\%&             \\
		Source Positions               & $\bm p_s$           & m    & $s=1$: [2.6, 2.3, 1]     \\%& environment \\
		                               &                     &      & $s=2$: [3.4, 2.3, 1]     \\%&             \\
		Reverberation Time             & \Tsixty             & s    & 0.4, 0.7                 \\%& environment \\
		Signal-to-Noise Ratio          & SNR                 & dB   & 30                       \\%& environment \\
		Sampling frequency             & $f_s$               & Hz   & 16000                    \\%& simulation  \\
		\glsentryshort{stft} Frequency Bins      & $K$                 &      & 1024                     \\%& simulation  \\
		Selected Frequency Bins        & $k$                 &      & 32 : 96                  \\%& simulation  \\
		Selected Frequency Band        &                     & Hz   & 500 - 1500               \\%& simulation  \\
		EM iterations                  & $L$                 &      & 10                       \\%& algorithm   \\
		Initial variance               & $\sigma^{2, (0)}_s$ &      & 1 $\forall\ s$           \\%& algorithm   \\
		Fixed variance                 &                     &      & false                    \\%& algorithm   \\
		\bottomrule
	\end{tabular}
	\label{table:parameters-schwartz14}
	\caption[Parameter Set used in \cite{Schwartz2014}]{Parameter Set used in \cite{Schwartz2014}: \itshape }
\end{table}


\subsubsection*{Replica}
\begin{figure}[!htb]
 \centering
 \begin{subfigure}{\textwidth}
    \includegraphics[width=\textwidth]{data/plots/schwartz2014/s=2-sloc=schwartz2014-T60=0.4-prior=schwartz2014-results}
    \caption{\Tsixty$=0.4$}
 \end{subfigure}\\
 \begin{subfigure}{\textwidth}
    \includegraphics[width=\textwidth]{data/plots/schwartz2014/s=2-sloc=schwartz2014-T60=0.7-prior=schwartz2014-results}
    \caption{\Tsixty$=0.7$}
 \end{subfigure}
 \caption{Results of replicated static evaluation experiment in \cite{Schwartz2014}}
\end{figure}