\subsubsection{Replication of Experiments in \citeauthor{Schwartz2014} \citeyearpar{Schwartz2014}}
As already mentioned in Chapter \ref{chap:algorithms}, the algorithm developed in \cite{Schwartz2014} included the calculation of multiple Gaussian component weight vectors $\psi_{\bm p, s}\in \{\psi_{\bm p, 1}, \dots, \psi_{\bm, S}\}$, one for each source $s$. In their evaluation of static localisation of two sources, they initialised $\psi_{\bm p, 1}$ to have a uniform distribution across the left side of the room and $\psi_{\bm p, 2}$ to have a uniform distribution across the right side of the room. The variance has been initialised to $\sigma^2_s=1\ \forall\ s$. The complete parameter set, that was used in \cite{Schwartz2014}, is shown in Table \ref{table:parameters-schwartz14}. The position of the sensor nodes was not stated explicitely, but could be inferred from a diagram that displayed the setup, similar to Figure \ref{fig:setup}.				

\begin{table}[!htbp]
	\begin{tabular}{lccc}
		\toprule
		Parameter                                     & Unit & Symbol                 & Value                                       \\
		\midrule
		Number of sources                             &      & $S$                    & 2                                           \\
		Source Positions                              & m    & $\bm p_s$              & $s=1$: [2.6, 2.3, 1]                        \\
		                                              &      &                        & $s=2$: [3.4, 2.3, 1]                        \\
		Microphone Positions                          & m    & $\bm p_m^i$            & see \autoref{fig:resultsReplication} \\
		Reverberation Time                            & s    & \Tsixty                & 0.4, 0.7                                    \\
		Signal-to-Noise Ratio                         & dB   & SNR                    & 30                                          \\
		Number of \glsentryshort{stft} Frequency Bins &      & $K$                    & 65 (bins 32-96)                             \\
		Selected Frequency Band                       & Hz   &                        & 500 - 1500                                  \\
		EM iterations                                 &      & $L$                    & 10                                          \\
		Initial variance                              &      & $\sigma^{2, (0)}_s$ & 1                                           \\
		Fixed variance                                &      &                        & true                                       \\
		\bottomrule
	\end{tabular}
	\caption[Adjusted Parameters for Replication of \cite{Schwartz2014}]{Adjusted Parameter Set for Replication of \cite{Schwartz2014}.}
	\label{table:parametersSchwartz14}
\end{table}


\subsubsection*{Replica}
\begin{figure}[!htb]
 \centering
 \begin{subfigure}{\textwidth}
    \includegraphics[width=\textwidth]{data/plots/schwartz2014/s=2-sloc=schwartz2014-T60=0.4-prior=schwartz2014-results}
    \caption{\Tsixty$=0.4$}
 \end{subfigure}\\
 \begin{subfigure}{\textwidth}
    \includegraphics[width=\textwidth]{data/plots/schwartz2014/s=2-sloc=schwartz2014-T60=0.7-prior=schwartz2014-results}
    \caption{\Tsixty$=0.7$}
 \end{subfigure}
 \caption{Results of replicated static evaluation experiment in \cite{Schwartz2014}}
\end{figure}