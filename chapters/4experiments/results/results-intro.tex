% Configuration, so that floats take up less space!
\setcounter{topnumber}{2}
\setcounter{bottomnumber}{2}
\setcounter{totalnumber}{4}
\renewcommand{\topfraction}{0.85}
\renewcommand{\bottomfraction}{0.85}
\renewcommand{\textfraction}{0.15}
\renewcommand{\floatpagefraction}{0.8}
\renewcommand{\textfraction}{0.1}
\setlength{\floatsep}{5pt plus 2pt minus 2pt}
\setlength{\textfloatsep}{5pt plus 2pt minus 2pt}
\setlength{\intextsep}{5pt plus 2pt minus 2pt}
\tikzset{every picture/.style={scale=1.0}}

\setlength\figureheight{7cm}
\setlength\figurewidth{\textwidth}

\section{Results}
\label{chap:results}

For all boxplots below, the x-axis represents the number of sources $S$, whereas the y-axis represents the mean localisation error in metre.

\newcommand{\boxplotDescription}{Mean Localisation Error across Number of Sources $S$\ }

%% all-in-one boxplot
\begin{figure}[H]
    \setlength\figurewidth{5cm}
	\centering
	% This file was created by matplotlib2tikz v0.6.14.
\begin{tikzpicture}

\definecolor{color0}{rgb}{0.8,0.207843137254902,0.219607843137255}

\begin{axis}[
xlabel={number of sources},
ylabel={mean localisation error},
xmin=0.5, xmax=1.5,
ymin=0, ymax=2.5,
width=\figurewidth,
height=\figureheight,
xtick={1},
xticklabels={2},
ytick={0,0.25,0.5,0.75,1,1.25,1.5,1.75},
minor xtick={},
minor ytick={},
tick align=outside,
tick pos=left,
x grid style={white!69.019607843137251!black},
ymajorgrids,
y grid style={white!69.019607843137251!black}
]
\addplot [line width=1.0pt, color0, opacity=1, forget plot]
table {%
0.96 0.570710678118655
1.04 0.570710678118655
1.04 0.976485552705722
0.96 0.976485552705722
0.96 0.570710678118655
};
\addplot [line width=1.0pt, color0, opacity=1, forget plot]
table {%
1 0.570710678118655
1 0.5
};
\addplot [line width=1.0pt, color0, opacity=1, forget plot]
table {%
1 0.976485552705722
1 1.53501585870364
};
\addplot [line width=1.0pt, color0, forget plot]
table {%
0.98 0.5
1.02 0.5
};
\addplot [line width=1.0pt, color0, forget plot]
table {%
0.98 1.53501585870364
1.02 1.53501585870364
};
\addplot [line width=1.0pt, color0, opacity=0.2, mark=*, mark size=1, mark options={solid}, only marks, forget plot]
table {%
1 2.50659639261368
1 1.69447616957539
1 1.86579701927343
1 2.08182987023963
1 2.43149008471149
1 2.47540987087482
1 2.13783879652589
1 1.601387818866
1 1.60415458863651
1 1.92360679774998
1 1.74118066919062
1 4.60613878386295
1 1.61803398874989
1 3.76589889011722
1 2.11235406713628
1 1.9555239148283
1 2.3669903431165
1 1.75933866224478
1 1.65113577727726
1 2.33153476147244
1 2.45880074906351
1 2.38698863246246
1 1.77800248377821
1 1.83911514474839
1 2.41567489908994
1 1.6116085680851
1 1.64060910007205
1 2.04373562519443
1 1.72576506721313
1 1.74118066919062
1 1.72071016610744
1 1.62957210080317
1 2.02656876555683
1 3.06097930393636
};
\addplot [line width=1.0pt, color0, opacity=1, forget plot]
table {%
0.96 0.707134582144651
1.04 0.707134582144651
};
\addplot [line width=1.0pt, color0, dashed, mark=x, mark size=3, mark options={solid}, forget plot]
table {%
1 0.929542795255198
};
\end{axis}

\node at ({$(current bounding box.south west)!0.5!(current bounding box.south east)$}|-{$(current bounding box.south west)!0.98!(current bounding box.north west)$})[
  anchor=north,
  text=black,
  rotate=0.0
]{ };
\end{tikzpicture}
	\caption[Box plot reference]{Box plot reference: \itshape A common box plot consists of the following elements: the median, shown as a horizontal line, represents the 50\% percentile (also: second quantile) or the "middle" of the dataset by datapoints ordered from low to high values. The box around the median indicates the 25\% and 75\% percentile (also called first and third quantile). The mean, or "middle value" of the data by datapoint values, is represented by a cross. The vertical lines originating from the box are called whiskers and reach up to the 97.5 percentile or 2.5 percentile respectively.}
	\label{fig:boxplot-reference}
\end{figure}


