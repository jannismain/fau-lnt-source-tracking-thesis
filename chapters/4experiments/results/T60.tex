\subsubsection*{Reverberation Time}
%% all-in-one boxplot
\begin{figure}[H]
\iftoggle{quick}{%
    \includegraphics[width=\textwidth]{plots/boxplots/boxplot-joined-T60}
}{%
    \begin{tikzpicture}
	    % This file was created by matplotlib2tikz v0.6.14.
\definecolor{color0}{rgb}{0.8,0.207843137254902,0.219607843137255}
\definecolor{color1}{rgb}{1,0.647058823529412,0}
\definecolor{color2}{rgb}{0.0235294117647059,0.603921568627451,0.952941176470588}

\begin{axis}[
xlabel={$S$},
ylabel={MAE},
xmin=0.5, xmax=6.5,
ymin=0, ymax=2.5,
width=\figurewidth,
height=\figureheight,
xtick={1,2,3,4,5,6},
xticklabels={2,3,4,5,6,7},
ytick={0,0.5,1,1.5,2,2.5},
minor xtick={},
minor ytick={},
tick align=outside,
tick pos=left,
x grid style={white!69.019607843137251!black},
ymajorgrids,
y grid style={white!69.019607843137251!black}
]
\addplot [line width=1.0pt, black, opacity=1, forget plot]
table {%
0.76 0
0.84 0
0.84 0
0.76 0
0.76 0
};
\addplot [line width=1.0pt, black, opacity=1, forget plot]
table {%
0.8 0
0.8 0
};
\addplot [line width=1.0pt, black, opacity=1, forget plot]
table {%
0.8 0
0.8 0
};
\addplot [line width=1.0pt, black, forget plot]
table {%
0.78 0
0.82 0
};
\addplot [line width=1.0pt, black, forget plot]
table {%
0.78 0
0.82 0
};
\addplot [line width=0.5pt, black, opacity=0.2, mark=*, mark size=1, mark options={solid}, only marks, forget plot]
table {%
0.8 0.0500000000000003
0.8 1.31529464379659
0.8 1.33416640641263
0.8 0.0500000000000003
0.8 0.743639324992696
0.8 0.05
0.8 0.0500000000000003
0.8 0.11180339887499
0.8 0.282842712474619
0.8 0.0707106781186549
0.8 0.0707106781186546
0.8 0.0500000000000003
0.8 0.182514076993644
0.8 0.360555127546399
0.8 0.0499999999999999
0.8 0.0707106781186546
0.8 0.744974746830583
};
\addplot [line width=1.0pt, black, opacity=1, forget plot]
table {%
1.76 0
1.84 0
1.84 0
1.76 0
1.76 0
};
\addplot [line width=1.0pt, black, opacity=1, forget plot]
table {%
1.8 0
1.8 0
};
\addplot [line width=1.0pt, black, opacity=1, forget plot]
table {%
1.8 0
1.8 0
};
\addplot [line width=1.0pt, black, forget plot]
table {%
1.78 0
1.82 0
};
\addplot [line width=1.0pt, black, forget plot]
table {%
1.78 0
1.82 0
};
\addplot [line width=0.5pt, black, opacity=0.2, mark=*, mark size=1, mark options={solid}, only marks, forget plot]
table {%
1.8 0.0471404520791032
1.8 0.0471404520791032
1.8 0.0471404520791031
1.8 0.0333333333333333
1.8 0.284800124843918
1.8 0.307318148576429
1.8 0.235702260395516
1.8 0.0333333333333333
1.8 0.357378311724612
1.8 0.706320670013903
1.8 0.543790283299492
1.8 0.0471404520791031
1.8 0.107868932583326
1.8 0.141421356237309
1.8 0.0745355992499929
1.8 0.074535599249993
1.8 0.5
1.8 0.0471404520791034
1.8 0.074535599249993
1.8 0.0333333333333335
1.8 0.0333333333333333
1.8 0.402768199119819
1.8 1.09188929027728
1.8 0.380058475033046
1.8 0.401386485959743
1.8 1.48024022074497
1.8 0.401478645836925
1.8 0.0471404520791033
1.8 0.0471404520791033
1.8 0.0471404520791032
1.8 0.0471404520791034
1.8 0.0942809041582064
1.8 0.074535599249993
1.8 0.120185042515466
1.8 0.517472489875334
1.8 0.0471404520791031
1.8 0.286744175568088
1.8 0.0471404520791032
1.8 0.166666666666667
1.8 0.0471404520791034
1.8 0.767390962214756
1.8 0.4
1.8 0.316227766016838
1.8 0.807987932779992
1.8 0.767390962214756
1.8 0.282842712474619
1.8 0.0333333333333332
1.8 0.0471404520791033
1.8 0.210818510677892
1.8 0.121676051329096
1.8 0.402768199119819
1.8 0.360555127546399
};
\addplot [line width=1.0pt, black, opacity=1, forget plot]
table {%
2.76 0
2.84 0
2.84 0.110369053450738
2.76 0.110369053450738
2.76 0
};
\addplot [line width=1.0pt, black, opacity=1, forget plot]
table {%
2.8 0
2.8 0
};
\addplot [line width=1.0pt, black, opacity=1, forget plot]
table {%
2.8 0.110369053450738
2.8 0.251246890528022
};
\addplot [line width=1.0pt, black, forget plot]
table {%
2.78 0
2.82 0
};
\addplot [line width=1.0pt, black, forget plot]
table {%
2.78 0.251246890528022
2.82 0.251246890528022
};
\addplot [line width=0.5pt, black, opacity=0.2, mark=*, mark size=1, mark options={solid}, only marks, forget plot]
table {%
2.8 0.325
2.8 0.650480591562884
2.8 0.920007046982968
2.8 0.364005494464026
2.8 0.372210140291786
2.8 1.3540667357575
2.8 0.313249102153542
2.8 0.600520607473216
2.8 0.813941029804985
2.8 0.357945526581909
2.8 0.832165848854662
2.8 0.48541219597369
2.8 0.286343580951181
2.8 0.522015325445528
2.8 0.326039864469807
2.8 0.301039864469807
2.8 0.657647321898295
2.8 0.926350365682445
2.8 0.726821173356294
2.8 0.492442890089805
2.8 0.522015325445528
2.8 0.410766175205522
2.8 0.617454451761424
2.8 0.626996810199223
2.8 0.81891696770796
2.8 0.575
2.8 0.336340601176843
2.8 0.687275579579592
2.8 0.427200187265877
2.8 0.608276253029822
2.8 0.776208734813001
2.8 0.756637297521078
2.8 0.291547594742265
2.8 0.40551155093097
2.8 0.459619407771256
2.8 0.290753645318366
};
\addplot [line width=1.0pt, black, opacity=1, forget plot]
table {%
3.76 0
3.84 0
3.84 0.413867623582886
3.76 0.413867623582886
3.76 0
};
\addplot [line width=1.0pt, black, opacity=1, forget plot]
table {%
3.8 0
3.8 0
};
\addplot [line width=1.0pt, black, opacity=1, forget plot]
table {%
3.8 0.413867623582886
3.8 1.012861434392
};
\addplot [line width=1.0pt, black, forget plot]
table {%
3.78 0
3.82 0
};
\addplot [line width=1.0pt, black, forget plot]
table {%
3.78 1.012861434392
3.82 1.012861434392
};
\addplot [line width=0.5pt, black, opacity=0.2, mark=*, mark size=1, mark options={solid}, only marks, forget plot]
table {%
3.8 1.45080397107786
3.8 1.55367329815267
3.8 1.2520361407306
3.8 1.31615575305292
};
\addplot [line width=1.0pt, black, opacity=1, forget plot]
table {%
4.76 0.0166666666666667
4.84 0.0166666666666667
4.84 0.444531129851734
4.76 0.444531129851734
4.76 0.0166666666666667
};
\addplot [line width=1.0pt, black, opacity=1, forget plot]
table {%
4.8 0.0166666666666667
4.8 0
};
\addplot [line width=1.0pt, black, opacity=1, forget plot]
table {%
4.8 0.444531129851734
4.8 1.03502336381997
};
\addplot [line width=1.0pt, black, forget plot]
table {%
4.78 0
4.82 0
};
\addplot [line width=1.0pt, black, forget plot]
table {%
4.78 1.03502336381997
4.82 1.03502336381997
};
\addplot [line width=0.5pt, black, opacity=0.2, mark=*, mark size=1, mark options={solid}, only marks, forget plot]
table {%
4.8 1.34281901752144
4.8 1.2291338117043
};
\addplot [line width=1.0pt, black, opacity=1, forget plot]
table {%
5.76 0.0235702260395517
5.84 0.0235702260395517
5.84 0.533328789774181
5.76 0.533328789774181
5.76 0.0235702260395517
};
\addplot [line width=1.0pt, black, opacity=1, forget plot]
table {%
5.8 0.0235702260395517
5.8 0
};
\addplot [line width=1.0pt, black, opacity=1, forget plot]
table {%
5.8 0.533328789774181
5.8 1.09180081331247
};
\addplot [line width=1.0pt, black, forget plot]
table {%
5.78 0
5.82 0
};
\addplot [line width=1.0pt, black, forget plot]
table {%
5.78 1.09180081331247
5.82 1.09180081331247
};
\addplot [line width=1.0pt, color0, opacity=1, forget plot]
table {%
0.893333333333333 0
0.973333333333333 0
0.973333333333333 0.175659022548647
0.893333333333333 0.175659022548647
0.893333333333333 0
};
\addplot [line width=1.0pt, color0, opacity=1, forget plot]
table {%
0.933333333333333 0
0.933333333333333 0
};
\addplot [line width=1.0pt, color0, opacity=1, forget plot]
table {%
0.933333333333333 0.175659022548647
0.933333333333333 0.431959610746632
};
\addplot [line width=1.0pt, color0, forget plot]
table {%
0.913333333333333 0
0.953333333333333 0
};
\addplot [line width=1.0pt, color0, forget plot]
table {%
0.913333333333333 0.431959610746632
0.953333333333333 0.431959610746632
};
\addplot [line width=0.5pt, color0, opacity=0.2, mark=*, mark size=1, mark options={solid}, only marks, forget plot]
table {%
0.933333333333333 0.462132034355964
0.933333333333333 0.47169905660283
0.933333333333333 0.484415690288111
0.933333333333333 0.515154392492244
0.933333333333333 0.541547594742265
0.933333333333333 0.51478150704935
0.933333333333333 1.30862523283024
0.933333333333333 1.66883192682786
0.933333333333333 1.48490890352285
0.933333333333333 1.30096118312577
0.933333333333333 0.5
0.933333333333333 0.496505329790037
0.933333333333333 0.923212459828649
0.933333333333333 0.5
};
\addplot [line width=1.0pt, color0, opacity=1, forget plot]
table {%
1.89333333333333 0.0333333333333332
1.97333333333333 0.0333333333333332
1.97333333333333 0.201528400484592
1.89333333333333 0.201528400484592
1.89333333333333 0.0333333333333332
};
\addplot [line width=1.0pt, color0, opacity=1, forget plot]
table {%
1.93333333333333 0.0333333333333332
1.93333333333333 0
};
\addplot [line width=1.0pt, color0, opacity=1, forget plot]
table {%
1.93333333333333 0.201528400484592
1.93333333333333 0.438742588672279
};
\addplot [line width=1.0pt, color0, forget plot]
table {%
1.91333333333333 0
1.95333333333333 0
};
\addplot [line width=1.0pt, color0, forget plot]
table {%
1.91333333333333 0.438742588672279
1.95333333333333 0.438742588672279
};
\addplot [line width=0.5pt, color0, opacity=0.2, mark=*, mark size=1, mark options={solid}, only marks, forget plot]
table {%
1.93333333333333 0.839311887467611
1.93333333333333 0.687764031317035
1.93333333333333 1.59345357754673
1.93333333333333 1.16428327977153
1.93333333333333 1.69367769532606
1.93333333333333 0.996636367889542
1.93333333333333 0.7879026390477
1.93333333333333 0.767591879243998
1.93333333333333 1.481210271108
1.93333333333333 0.781465175094483
1.93333333333333 1.09020667062973
1.93333333333333 0.933851653835582
1.93333333333333 0.989949493661167
1.93333333333333 0.849904011650012
1.93333333333333 0.547684194144059
1.93333333333333 0.542959659379718
1.93333333333333 1.30586716604234
1.93333333333333 1.22465908830831
1.93333333333333 1.53479138744366
1.93333333333333 0.480740170061866
1.93333333333333 1.21737727279302
1.93333333333333 0.94695242113682
1.93333333333333 0.846481634062347
1.93333333333333 1.6126429699796
1.93333333333333 0.673554357532564
1.93333333333333 0.481132704401887
1.93333333333333 0.47377913044465
1.93333333333333 1.14114951790825
1.93333333333333 0.827453779460412
1.93333333333333 0.690010572469092
};
\addplot [line width=1.0pt, color0, opacity=1, forget plot]
table {%
2.89333333333333 0.0500000000000001
2.97333333333333 0.0500000000000001
2.97333333333333 0.474952895312947
2.89333333333333 0.474952895312947
2.89333333333333 0.0500000000000001
};
\addplot [line width=1.0pt, color0, opacity=1, forget plot]
table {%
2.93333333333333 0.0500000000000001
2.93333333333333 0
};
\addplot [line width=1.0pt, color0, opacity=1, forget plot]
table {%
2.93333333333333 0.474952895312947
2.93333333333333 1.06941725889323
};
\addplot [line width=1.0pt, color0, forget plot]
table {%
2.91333333333333 0
2.95333333333333 0
};
\addplot [line width=1.0pt, color0, forget plot]
table {%
2.91333333333333 1.06941725889323
2.95333333333333 1.06941725889323
};
\addplot [line width=0.5pt, color0, opacity=0.2, mark=*, mark size=1, mark options={solid}, only marks, forget plot]
table {%
2.93333333333333 1.87024478774715
2.93333333333333 1.32167139813433
2.93333333333333 1.12249248548055
2.93333333333333 1.26034531628728
2.93333333333333 1.25432286885853
2.93333333333333 1.59808433135283
2.93333333333333 1.19129328274686
2.93333333333333 1.13178881988409
2.93333333333333 1.27787591146511
2.93333333333333 1.27807764064044
2.93333333333333 1.39940126607038
2.93333333333333 1.13334788368656
2.93333333333333 1.35636856474861
};
\addplot [line width=1.0pt, color0, opacity=1, forget plot]
table {%
3.89333333333333 0.104721359549996
3.97333333333333 0.104721359549996
3.97333333333333 0.615739268316702
3.89333333333333 0.615739268316702
3.89333333333333 0.104721359549996
};
\addplot [line width=1.0pt, color0, opacity=1, forget plot]
table {%
3.93333333333333 0.104721359549996
3.93333333333333 0
};
\addplot [line width=1.0pt, color0, opacity=1, forget plot]
table {%
3.93333333333333 0.615739268316702
3.93333333333333 1.36690159696263
};
\addplot [line width=1.0pt, color0, forget plot]
table {%
3.91333333333333 0
3.95333333333333 0
};
\addplot [line width=1.0pt, color0, forget plot]
table {%
3.91333333333333 1.36690159696263
3.95333333333333 1.36690159696263
};
\addplot [line width=0.5pt, color0, opacity=0.2, mark=*, mark size=1, mark options={solid}, only marks, forget plot]
table {%
3.93333333333333 1.40674650581674
3.93333333333333 1.38750402094249
3.93333333333333 1.43625593313066
3.93333333333333 1.42399785904219
3.93333333333333 1.50581116099217
};
\addplot [line width=1.0pt, color0, opacity=1, forget plot]
table {%
4.89333333333333 0.158671500983403
4.97333333333333 0.158671500983403
4.97333333333333 0.671273841223798
4.89333333333333 0.671273841223798
4.89333333333333 0.158671500983403
};
\addplot [line width=1.0pt, color0, opacity=1, forget plot]
table {%
4.93333333333333 0.158671500983403
4.93333333333333 0
};
\addplot [line width=1.0pt, color0, opacity=1, forget plot]
table {%
4.93333333333333 0.671273841223798
4.93333333333333 1.4266188444742
};
\addplot [line width=1.0pt, color0, forget plot]
table {%
4.91333333333333 0
4.95333333333333 0
};
\addplot [line width=1.0pt, color0, forget plot]
table {%
4.91333333333333 1.4266188444742
4.95333333333333 1.4266188444742
};
\addplot [line width=1.0pt, color0, opacity=1, forget plot]
table {%
5.89333333333333 0.199562806159567
5.97333333333333 0.199562806159567
5.97333333333333 0.634250475951949
5.89333333333333 0.634250475951949
5.89333333333333 0.199562806159567
};
\addplot [line width=1.0pt, color0, opacity=1, forget plot]
table {%
5.93333333333333 0.199562806159567
5.93333333333333 0.0166666666666668
};
\addplot [line width=1.0pt, color0, opacity=1, forget plot]
table {%
5.93333333333333 0.634250475951949
5.93333333333333 1.20849569100204
};
\addplot [line width=1.0pt, color0, forget plot]
table {%
5.91333333333333 0.0166666666666668
5.95333333333333 0.0166666666666668
};
\addplot [line width=1.0pt, color0, forget plot]
table {%
5.91333333333333 1.20849569100204
5.95333333333333 1.20849569100204
};
\addplot [line width=0.5pt, color0, opacity=0.2, mark=*, mark size=1, mark options={solid}, only marks, forget plot]
table {%
5.93333333333333 1.53183445293587
5.93333333333333 1.36989854186536
};
\addplot [line width=1.0pt, color1, opacity=1, forget plot]
table {%
1.02666666666667 0.0499999999999998
1.10666666666667 0.0499999999999998
1.10666666666667 0.289371374175354
1.02666666666667 0.289371374175354
1.02666666666667 0.0499999999999998
};
\addplot [line width=1.0pt, color1, opacity=1, forget plot]
table {%
1.06666666666667 0.0499999999999998
1.06666666666667 0
};
\addplot [line width=1.0pt, color1, opacity=1, forget plot]
table {%
1.06666666666667 0.289371374175354
1.06666666666667 0.633864246327115
};
\addplot [line width=1.0pt, color1, forget plot]
table {%
1.04666666666667 0
1.08666666666667 0
};
\addplot [line width=1.0pt, color1, forget plot]
table {%
1.04666666666667 0.633864246327115
1.08666666666667 0.633864246327115
};
\addplot [line width=0.5pt, color1, opacity=0.2, mark=*, mark size=1, mark options={solid}, only marks, forget plot]
table {%
1.06666666666667 0.850326482914885
1.06666666666667 0.654619212798457
1.06666666666667 0.936002257333468
1.06666666666667 0.70178344238091
1.06666666666667 0.715891053163818
1.06666666666667 1.8580904176062
1.06666666666667 1.17219776264172
1.06666666666667 0.886002257333468
1.06666666666667 1.00958320130474
1.06666666666667 1.43442970499026
1.06666666666667 1.48416538025861
1.06666666666667 1.04403065089106
1.06666666666667 1.10453610171873
1.06666666666667 1.06117675050949
1.06666666666667 1.55724115023974
1.06666666666667 1.7464249196573
1.06666666666667 1.21711959646805
1.06666666666667 0.764400085215328
1.06666666666667 1.17686022959398
1.06666666666667 0.919238815542512
1.06666666666667 0.680073525436772
1.06666666666667 0.659345629855922
1.06666666666667 0.913133825081604
};
\addplot [line width=1.0pt, color1, opacity=1, forget plot]
table {%
2.02666666666667 0.122042341259485
2.10666666666667 0.122042341259485
2.10666666666667 0.809701604496446
2.02666666666667 0.809701604496446
2.02666666666667 0.122042341259485
};
\addplot [line width=1.0pt, color1, opacity=1, forget plot]
table {%
2.06666666666667 0.122042341259485
2.06666666666667 0
};
\addplot [line width=1.0pt, color1, opacity=1, forget plot]
table {%
2.06666666666667 0.809701604496446
2.06666666666667 1.68830421959243
};
\addplot [line width=1.0pt, color1, forget plot]
table {%
2.04666666666667 0
2.08666666666667 0
};
\addplot [line width=1.0pt, color1, forget plot]
table {%
2.04666666666667 1.68830421959243
2.08666666666667 1.68830421959243
};
\addplot [line width=0.5pt, color1, opacity=0.2, mark=*, mark size=1, mark options={solid}, only marks, forget plot]
table {%
2.06666666666667 2.02243462933147
2.06666666666667 1.8661198814742
};
\addplot [line width=1.0pt, color1, opacity=1, forget plot]
table {%
3.02666666666667 0.375775357910832
3.10666666666667 0.375775357910832
3.10666666666667 0.945431576323539
3.02666666666667 0.945431576323539
3.02666666666667 0.375775357910832
};
\addplot [line width=1.0pt, color1, opacity=1, forget plot]
table {%
3.06666666666667 0.375775357910832
3.06666666666667 0.0249999999999999
};
\addplot [line width=1.0pt, color1, opacity=1, forget plot]
table {%
3.06666666666667 0.945431576323539
3.06666666666667 1.62248217855221
};
\addplot [line width=1.0pt, color1, forget plot]
table {%
3.04666666666667 0.0249999999999999
3.08666666666667 0.0249999999999999
};
\addplot [line width=1.0pt, color1, forget plot]
table {%
3.04666666666667 1.62248217855221
3.08666666666667 1.62248217855221
};
\addplot [line width=0.5pt, color1, opacity=0.2, mark=*, mark size=1, mark options={solid}, only marks, forget plot]
table {%
3.06666666666667 1.81360890138675
3.06666666666667 1.86140944084869
};
\addplot [line width=1.0pt, color1, opacity=1, forget plot]
table {%
4.02666666666667 0.542071073365796
4.10666666666667 0.542071073365796
4.10666666666667 1.01712155858353
4.02666666666667 1.01712155858353
4.02666666666667 0.542071073365796
};
\addplot [line width=1.0pt, color1, opacity=1, forget plot]
table {%
4.06666666666667 0.542071073365796
4.06666666666667 0.0199999999999999
};
\addplot [line width=1.0pt, color1, opacity=1, forget plot]
table {%
4.06666666666667 1.01712155858353
4.06666666666667 1.71110036512443
};
\addplot [line width=1.0pt, color1, forget plot]
table {%
4.04666666666667 0.0199999999999999
4.08666666666667 0.0199999999999999
};
\addplot [line width=1.0pt, color1, forget plot]
table {%
4.04666666666667 1.71110036512443
4.08666666666667 1.71110036512443
};
\addplot [line width=0.5pt, color1, opacity=0.2, mark=*, mark size=1, mark options={solid}, only marks, forget plot]
table {%
4.06666666666667 1.82286107980356
4.06666666666667 1.98143947743723
4.06666666666667 2.04041207093658
4.06666666666667 1.79107337662464
};
\addplot [line width=1.0pt, color1, opacity=1, forget plot]
table {%
5.02666666666667 0.61984011648905
5.10666666666667 0.61984011648905
5.10666666666667 1.05573193013855
5.02666666666667 1.05573193013855
5.02666666666667 0.61984011648905
};
\addplot [line width=1.0pt, color1, opacity=1, forget plot]
table {%
5.06666666666667 0.61984011648905
5.06666666666667 0.0971404520791031
};
\addplot [line width=1.0pt, color1, opacity=1, forget plot]
table {%
5.06666666666667 1.05573193013855
5.06666666666667 1.64523704409378
};
\addplot [line width=1.0pt, color1, forget plot]
table {%
5.04666666666667 0.0971404520791031
5.08666666666667 0.0971404520791031
};
\addplot [line width=1.0pt, color1, forget plot]
table {%
5.04666666666667 1.64523704409378
5.08666666666667 1.64523704409378
};
\addplot [line width=1.0pt, color1, opacity=1, forget plot]
table {%
6.02666666666667 0.503874764029806
6.10666666666667 0.503874764029806
6.10666666666667 0.923071316528485
6.02666666666667 0.923071316528485
6.02666666666667 0.503874764029806
};
\addplot [line width=1.0pt, color1, opacity=1, forget plot]
table {%
6.06666666666667 0.503874764029806
6.06666666666667 0.202381174390913
};
\addplot [line width=1.0pt, color1, opacity=1, forget plot]
table {%
6.06666666666667 0.923071316528485
6.06666666666667 1.54003411482351
};
\addplot [line width=1.0pt, color1, forget plot]
table {%
6.04666666666667 0.202381174390913
6.08666666666667 0.202381174390913
};
\addplot [line width=1.0pt, color1, forget plot]
table {%
6.04666666666667 1.54003411482351
6.08666666666667 1.54003411482351
};
\addplot [line width=0.5pt, color1, opacity=0.2, mark=*, mark size=1, mark options={solid}, only marks, forget plot]
table {%
6.06666666666667 1.68265205518671
6.06666666666667 1.62119423156085
6.06666666666667 1.65383790608507
6.06666666666667 1.61588952099776
6.06666666666667 1.64878156429865
};
\addplot [line width=1.0pt, color2, opacity=1, forget plot]
table {%
1.16 0.0707106781186548
1.24 0.0707106781186548
1.24 0.997474977573215
1.16 0.997474977573215
1.16 0.0707106781186548
};
\addplot [line width=1.0pt, color2, opacity=1, forget plot]
table {%
1.2 0.0707106781186548
1.2 0
};
\addplot [line width=1.0pt, color2, opacity=1, forget plot]
table {%
1.2 0.997474977573215
1.2 2.09396073613187
};
\addplot [line width=1.0pt, color2, forget plot]
table {%
1.18 0
1.22 0
};
\addplot [line width=1.0pt, color2, forget plot]
table {%
1.18 2.09396073613187
1.22 2.09396073613187
};
\addplot [line width=0.5pt, color2, opacity=0.2, mark=*, mark size=1, mark options={solid}, only marks, forget plot]
table {%
1.2 2.45622577482986
};
\addplot [line width=1.0pt, color2, opacity=1, forget plot]
table {%
2.16 0.341743611104116
2.24 0.341743611104116
2.24 1.07895573035458
2.16 1.07895573035458
2.16 0.341743611104116
};
\addplot [line width=1.0pt, color2, opacity=1, forget plot]
table {%
2.2 0.341743611104116
2.2 0
};
\addplot [line width=1.0pt, color2, opacity=1, forget plot]
table {%
2.2 1.07895573035458
2.2 2.08240764379692
};
\addplot [line width=1.0pt, color2, forget plot]
table {%
2.18 0
2.22 0
};
\addplot [line width=1.0pt, color2, forget plot]
table {%
2.18 2.08240764379692
2.22 2.08240764379692
};
\addplot [line width=0.5pt, color2, opacity=0.2, mark=*, mark size=1, mark options={solid}, only marks, forget plot]
table {%
2.2 2.51640868839341
};
\addplot [line width=1.0pt, color2, opacity=1, forget plot]
table {%
3.16 0.676707863920588
3.24 0.676707863920588
3.24 1.16047502208024
3.16 1.16047502208024
3.16 0.676707863920588
};
\addplot [line width=1.0pt, color2, opacity=1, forget plot]
table {%
3.2 0.676707863920588
3.2 0.0790569415042094
};
\addplot [line width=1.0pt, color2, opacity=1, forget plot]
table {%
3.2 1.16047502208024
3.2 1.80197640778147
};
\addplot [line width=1.0pt, color2, forget plot]
table {%
3.18 0.0790569415042094
3.22 0.0790569415042094
};
\addplot [line width=1.0pt, color2, forget plot]
table {%
3.18 1.80197640778147
3.22 1.80197640778147
};
\addplot [line width=0.5pt, color2, opacity=0.2, mark=*, mark size=1, mark options={solid}, only marks, forget plot]
table {%
3.2 2.01848192155636
3.2 2.23432729682269
3.2 1.94774816429623
};
\addplot [line width=1.0pt, color2, opacity=1, forget plot]
table {%
4.16 0.65629167352773
4.24 0.65629167352773
4.24 1.17735741204651
4.16 1.17735741204651
4.16 0.65629167352773
};
\addplot [line width=1.0pt, color2, opacity=1, forget plot]
table {%
4.2 0.65629167352773
4.2 0.172688272303359
};
\addplot [line width=1.0pt, color2, opacity=1, forget plot]
table {%
4.2 1.17735741204651
4.2 1.93517729388425
};
\addplot [line width=1.0pt, color2, forget plot]
table {%
4.18 0.172688272303359
4.22 0.172688272303359
};
\addplot [line width=1.0pt, color2, forget plot]
table {%
4.18 1.93517729388425
4.22 1.93517729388425
};
\addplot [line width=1.0pt, color2, opacity=1, forget plot]
table {%
5.16 0.7796379225433
5.24 0.7796379225433
5.24 1.17229207367006
5.16 1.17229207367006
5.16 0.7796379225433
};
\addplot [line width=1.0pt, color2, opacity=1, forget plot]
table {%
5.2 0.7796379225433
5.2 0.366176045807952
};
\addplot [line width=1.0pt, color2, opacity=1, forget plot]
table {%
5.2 1.17229207367006
5.2 1.71763537783638
};
\addplot [line width=1.0pt, color2, forget plot]
table {%
5.18 0.366176045807952
5.22 0.366176045807952
};
\addplot [line width=1.0pt, color2, forget plot]
table {%
5.18 1.71763537783638
5.22 1.71763537783638
};
\addplot [line width=0.5pt, color2, opacity=0.2, mark=*, mark size=1, mark options={solid}, only marks, forget plot]
table {%
5.2 1.94753146454734
5.2 2.18015116334651
};
\addplot [line width=1.0pt, color2, opacity=1, forget plot]
table {%
6.16 0.666698478672069
6.24 0.666698478672069
6.24 1.06668639345309
6.16 1.06668639345309
6.16 0.666698478672069
};
\addplot [line width=1.0pt, color2, opacity=1, forget plot]
table {%
6.2 0.666698478672069
6.2 0.203779734334972
};
\addplot [line width=1.0pt, color2, opacity=1, forget plot]
table {%
6.2 1.06668639345309
6.2 1.54145147212391
};
\addplot [line width=1.0pt, color2, forget plot]
table {%
6.18 0.203779734334972
6.22 0.203779734334972
};
\addplot [line width=1.0pt, color2, forget plot]
table {%
6.18 1.54145147212391
6.22 1.54145147212391
};
\addplot [line width=0.5pt, color2, opacity=0.2, mark=*, mark size=1, mark options={solid}, only marks, forget plot]
table {%
6.2 1.69880258280329
};
\addplot [line width=1.0pt, black, opacity=1, forget plot]
table {%
0.76 0
0.84 0
};
\addplot [line width=1.0pt, black, dashed, mark=x, mark size=3, mark options={solid}, forget plot]
table {%
0.8 0.0223516898891125
};
\addplot [line width=1.0pt, black, opacity=1, forget plot]
table {%
1.76 0
1.84 0
};
\addplot [line width=1.0pt, black, dashed, mark=x, mark size=3, mark options={solid}, forget plot]
table {%
1.8 0.0561722553939968
};
\addplot [line width=1.0pt, black, opacity=1, forget plot]
table {%
2.76 0
2.84 0
};
\addplot [line width=1.0pt, black, dashed, mark=x, mark size=3, mark options={solid}, forget plot]
table {%
2.8 0.107833609526278
};
\addplot [line width=1.0pt, black, opacity=1, forget plot]
table {%
3.76 0.046502815398729
3.84 0.046502815398729
};
\addplot [line width=1.0pt, black, dashed, mark=x, mark size=3, mark options={solid}, forget plot]
table {%
3.8 0.218414884868778
};
\addplot [line width=1.0pt, black, opacity=1, forget plot]
table {%
4.76 0.153588725068507
4.84 0.153588725068507
};
\addplot [line width=1.0pt, black, dashed, mark=x, mark size=3, mark options={solid}, forget plot]
table {%
4.8 0.263209118180098
};
\addplot [line width=1.0pt, black, opacity=1, forget plot]
table {%
5.76 0.205934173713596
5.84 0.205934173713596
};
\addplot [line width=1.0pt, black, dashed, mark=x, mark size=3, mark options={solid}, forget plot]
table {%
5.8 0.307063779501924
};
\addplot [line width=1.0pt, color0, opacity=1, forget plot]
table {%
0.893333333333333 0.0499999999999998
0.973333333333333 0.0499999999999998
};
\addplot [line width=1.0pt, color0, dashed, mark=x, mark size=3, mark options={solid}, forget plot]
table {%
0.933333333333333 0.120691092505005
};
\addplot [line width=1.0pt, color0, opacity=1, forget plot]
table {%
1.89333333333333 0.0942809041582064
1.97333333333333 0.0942809041582064
};
\addplot [line width=1.0pt, color0, dashed, mark=x, mark size=3, mark options={solid}, forget plot]
table {%
1.93333333333333 0.205277469288374
};
\addplot [line width=1.0pt, color0, opacity=1, forget plot]
table {%
2.89333333333333 0.149883809811432
2.97333333333333 0.149883809811432
};
\addplot [line width=1.0pt, color0, dashed, mark=x, mark size=3, mark options={solid}, forget plot]
table {%
2.93333333333333 0.312545969320306
};
\addplot [line width=1.0pt, color0, opacity=1, forget plot]
table {%
3.89333333333333 0.249349766727654
3.97333333333333 0.249349766727654
};
\addplot [line width=1.0pt, color0, dashed, mark=x, mark size=3, mark options={solid}, forget plot]
table {%
3.93333333333333 0.386302443125616
};
\addplot [line width=1.0pt, color0, opacity=1, forget plot]
table {%
4.89333333333333 0.439042823286947
4.97333333333333 0.439042823286947
};
\addplot [line width=1.0pt, color0, dashed, mark=x, mark size=3, mark options={solid}, forget plot]
table {%
4.93333333333333 0.466998227119248
};
\addplot [line width=1.0pt, color0, opacity=1, forget plot]
table {%
5.89333333333333 0.366629559524816
5.97333333333333 0.366629559524816
};
\addplot [line width=1.0pt, color0, dashed, mark=x, mark size=3, mark options={solid}, forget plot]
table {%
5.93333333333333 0.440001075481877
};
\addplot [line width=1.0pt, color1, opacity=1, forget plot]
table {%
1.02666666666667 0.0999999999999999
1.10666666666667 0.0999999999999999
};
\addplot [line width=1.0pt, color1, dashed, mark=x, mark size=3, mark options={solid}, forget plot]
table {%
1.06666666666667 0.225727889446119
};
\addplot [line width=1.0pt, color1, opacity=1, forget plot]
table {%
2.02666666666667 0.37251692131874
2.10666666666667 0.37251692131874
};
\addplot [line width=1.0pt, color1, dashed, mark=x, mark size=3, mark options={solid}, forget plot]
table {%
2.06666666666667 0.502256968321234
};
\addplot [line width=1.0pt, color1, opacity=1, forget plot]
table {%
3.02666666666667 0.689861349828052
3.10666666666667 0.689861349828052
};
\addplot [line width=1.0pt, color1, dashed, mark=x, mark size=3, mark options={solid}, forget plot]
table {%
3.06666666666667 0.696704903208663
};
\addplot [line width=1.0pt, color1, opacity=1, forget plot]
table {%
4.02666666666667 0.764085739293198
4.10666666666667 0.764085739293198
};
\addplot [line width=1.0pt, color1, dashed, mark=x, mark size=3, mark options={solid}, forget plot]
table {%
4.06666666666667 0.79931943370046
};
\addplot [line width=1.0pt, color1, opacity=1, forget plot]
table {%
5.02666666666667 0.831027441912347
5.10666666666667 0.831027441912347
};
\addplot [line width=1.0pt, color1, dashed, mark=x, mark size=3, mark options={solid}, forget plot]
table {%
5.06666666666667 0.832043477671639
};
\addplot [line width=1.0pt, color1, opacity=1, forget plot]
table {%
6.02666666666667 0.704091591508899
6.10666666666667 0.704091591508899
};
\addplot [line width=1.0pt, color1, dashed, mark=x, mark size=3, mark options={solid}, forget plot]
table {%
6.06666666666667 0.745490017784198
};
\addplot [line width=1.0pt, color2, opacity=1, forget plot]
table {%
1.16 0.312132034355964
1.24 0.312132034355964
};
\addplot [line width=1.0pt, color2, dashed, mark=x, mark size=3, mark options={solid}, forget plot]
table {%
1.2 0.552864765094307
};
\addplot [line width=1.0pt, color2, opacity=1, forget plot]
table {%
2.16 0.668886351001172
2.24 0.668886351001172
};
\addplot [line width=1.0pt, color2, dashed, mark=x, mark size=3, mark options={solid}, forget plot]
table {%
2.2 0.735349214960208
};
\addplot [line width=1.0pt, color2, opacity=1, forget plot]
table {%
3.16 0.93288821833292
3.24 0.93288821833292
};
\addplot [line width=1.0pt, color2, dashed, mark=x, mark size=3, mark options={solid}, forget plot]
table {%
3.2 0.930865441327228
};
\addplot [line width=1.0pt, color2, opacity=1, forget plot]
table {%
4.16 0.900202565934821
4.24 0.900202565934821
};
\addplot [line width=1.0pt, color2, dashed, mark=x, mark size=3, mark options={solid}, forget plot]
table {%
4.2 0.932432172225408
};
\addplot [line width=1.0pt, color2, opacity=1, forget plot]
table {%
5.16 0.969675028416425
5.24 0.969675028416425
};
\addplot [line width=1.0pt, color2, dashed, mark=x, mark size=3, mark options={solid}, forget plot]
table {%
5.2 0.986962741565931
};
\addplot [line width=1.0pt, color2, opacity=1, forget plot]
table {%
6.16 0.860709994053905
6.24 0.860709994053905
};
\addplot [line width=1.0pt, color2, dashed, mark=x, mark size=3, mark options={solid}, forget plot]
table {%
6.2 0.876017361410472
};
\end{axis}

\node at ({$(current bounding box.south west)!0.5!(current bounding box.south east)$}|-{$(current bounding box.south west)!0.98!(current bounding box.north west)$})[
  anchor=north,
  text=black,
  rotate=0.0
]{ };

	    \begin{customlegend}[
legend entries={T$_{60}=0.0$~s,T$_{60}=0.3$~s,T$_{60}=0.6$~s,T$_{60}=0.9$~s},
legend cell align=left,
legend style={at={(13.87,5.37)}, anchor=north east, draw=white!80.0!black, font=\footnotesize,fill opacity=0.5, draw opacity=1,text opacity=1}]
    \addlegendimage{area legend,black,fill=black, fill opacity=1}
    \addlegendimage{area legend,color0,fill=color0, fill opacity=1}
    \addlegendimage{area legend,color1,fill=color1, fill opacity=1}
    \addlegendimage{area legend,color2,fill=color2, fill opacity=1}
\end{customlegend}
	\end{tikzpicture}

}%
	\caption[Evaluation Results for Varying \Tsixty]{Evaluation Results for Varying \Tsixty ($n=250$, $r=$~max)}
	\label{fig:trialT60}
\end{figure}

For this trial, the maximum reflection order $r$ has been adjusted from its base value of $r=3$ to maximum, in order to allow for full simulation of all reverberations for \nolinebreak{\Tsixty$=0.9$~s} and not cut off the \gls{rir} prematurely due to a low $r$. The general effect to be observed is that a longer reverberation time strongly increases the localisation error. This effect is especially accentuated for fewer sources, as for these, the localisation error usually is quite low. \Tsixty$=0.9$~s has a significant negative effect on the localisation performance for two sources, as it increases the median localisation error to around $0.4$m, the mean to over $0.5$m and the upper whisker reaches up to $2$m. Localisation in these conditions is much less reliable and the variance of the localisation error is increased strongly. Interestingly, the upper whisker of \Tsixty$=0.9$~s is generally lower the more sources are added to the simulation and the mean \gls{mae} for both \Tsixty$=0.6$~s and \Tsixty$=0.9$~s when adding a seventh source. This reflects the effect additional sources had on the trials in \autoref{fig:trialCases}, where location estimates have been guessed. This does not mean that the algorithm performs better the more sources there are, but rather indicates that there is less chance for the location estimates to be wrong.