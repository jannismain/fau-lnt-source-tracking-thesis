\subsubsection{Minimum Distance of Sources}

%% all-in-one boxplot
\begin{figure}[H]
	\begin{tikzpicture}
	    % This file was created by matplotlib2tikz v0.6.14.
\definecolor{color0}{rgb}{0.8,0.207843137254902,0.219607843137255}
\definecolor{color1}{rgb}{1,0.647058823529412,0}
\definecolor{color2}{rgb}{0.0235294117647059,0.603921568627451,0.952941176470588}

\begin{axis}[
xlabel={number of sources ($S$)},
ylabel={mean localisation error},
xmin=0.5, xmax=6.5,
ymin=0, ymax=2.5,
width=\figurewidth,
height=\figureheight,
xtick={1,2,3,4,5,6},
xticklabels={2,3,4,5,6,7},
ytick={0,0.5,1,1.5,2,2.5},
minor xtick={},
minor ytick={},
tick align=outside,
tick pos=left,
x grid style={white!69.019607843137251!black},
ymajorgrids,
y grid style={white!69.019607843137251!black}
]
\addplot [line width=1.0pt, black, opacity=1, forget plot]
table {%
0.76 0
0.84 0
0.84 0.307160536390393
0.76 0.307160536390393
0.76 0
};
\addplot [line width=1.0pt, black, opacity=1, forget plot]
table {%
0.8 0
0.8 0
};
\addplot [line width=1.0pt, black, opacity=1, forget plot]
table {%
0.8 0.307160536390393
0.8 0.710668695666975
};
\addplot [line width=1.0pt, black, forget plot]
table {%
0.78 0
0.82 0
};
\addplot [line width=1.0pt, black, forget plot]
table {%
0.78 0.710668695666975
0.82 0.710668695666975
};
\addplot [line width=1.0pt, black, opacity=0.2, mark=*, mark size=1, mark options={solid}, only marks, forget plot]
table {%
0.8 1.2
0.8 0.925744428679975
0.8 1.05475115548645
0.8 1.25104121494643
0.8 1.15
0.8 0.863133825081603
0.8 1.31284461382142
0.8 1.11018016555873
0.8 0.813941029804986
0.8 1.07360679774998
0.8 1.64575306308277
0.8 1.07569824023074
0.8 1.4642135623731
0.8 1.30104121494643
0.8 1.01369301145003
0.8 1.37455115915918
0.8 1.10708175202871
0.8 0.9
0.8 1.21655250605964
};
\addplot [line width=1.0pt, black, opacity=1, forget plot]
table {%
1.76 0.0333333333333334
1.84 0.0333333333333334
1.84 0.566110630551756
1.76 0.566110630551756
1.76 0.0333333333333334
};
\addplot [line width=1.0pt, black, opacity=1, forget plot]
table {%
1.8 0.0333333333333334
1.8 0
};
\addplot [line width=1.0pt, black, opacity=1, forget plot]
table {%
1.8 0.566110630551756
1.8 1.25717065004572
};
\addplot [line width=1.0pt, black, forget plot]
table {%
1.78 0
1.82 0
};
\addplot [line width=1.0pt, black, forget plot]
table {%
1.78 1.25717065004572
1.82 1.25717065004572
};
\addplot [line width=1.0pt, black, opacity=0.2, mark=*, mark size=1, mark options={solid}, only marks, forget plot]
table {%
1.8 1.75580805472615
1.8 1.44006318973395
1.8 1.62600881046829
};
\addplot [line width=1.0pt, black, opacity=1, forget plot]
table {%
2.76 0.0666666666666666
2.84 0.0666666666666666
2.84 0.635871674969106
2.76 0.635871674969106
2.76 0.0666666666666666
};
\addplot [line width=1.0pt, black, opacity=1, forget plot]
table {%
2.8 0.0666666666666666
2.8 0
};
\addplot [line width=1.0pt, black, opacity=1, forget plot]
table {%
2.8 0.635871674969106
2.8 1.4183197166192
};
\addplot [line width=1.0pt, black, forget plot]
table {%
2.78 0
2.82 0
};
\addplot [line width=1.0pt, black, forget plot]
table {%
2.78 1.4183197166192
2.82 1.4183197166192
};
\addplot [line width=1.0pt, black, opacity=0.2, mark=*, mark size=1, mark options={solid}, only marks, forget plot]
table {%
2.8 1.74470646980482
2.8 1.58307323535459
2.8 1.72442384655223
2.8 1.64065148190976
};
\addplot [line width=1.0pt, black, opacity=1, forget plot]
table {%
3.76 0.0804737854124367
3.84 0.0804737854124367
3.84 0.78314702923408
3.76 0.78314702923408
3.76 0.0804737854124367
};
\addplot [line width=1.0pt, black, opacity=1, forget plot]
table {%
3.8 0.0804737854124367
3.8 0
};
\addplot [line width=1.0pt, black, opacity=1, forget plot]
table {%
3.8 0.78314702923408
3.8 1.74032632632892
};
\addplot [line width=1.0pt, black, forget plot]
table {%
3.78 0
3.82 0
};
\addplot [line width=1.0pt, black, forget plot]
table {%
3.78 1.74032632632892
3.82 1.74032632632892
};
\addplot [line width=1.0pt, black, opacity=1, forget plot]
table {%
4.76 0.0804737854124367
4.84 0.0804737854124367
4.84 0.551797633144103
4.76 0.551797633144103
4.76 0.0804737854124367
};
\addplot [line width=1.0pt, black, opacity=1, forget plot]
table {%
4.8 0.0804737854124367
4.8 0
};
\addplot [line width=1.0pt, black, opacity=1, forget plot]
table {%
4.8 0.551797633144103
4.8 1.25129990995833
};
\addplot [line width=1.0pt, black, forget plot]
table {%
4.78 0
4.82 0
};
\addplot [line width=1.0pt, black, forget plot]
table {%
4.78 1.25129990995833
4.82 1.25129990995833
};
\addplot [line width=1.0pt, black, opacity=0.2, mark=*, mark size=1, mark options={solid}, only marks, forget plot]
table {%
4.8 1.45460037368852
4.8 1.41177457471527
4.8 1.59034363800389
4.8 1.3522808044587
4.8 1.44793967826153
};
\addplot [line width=1.0pt, black, opacity=1, forget plot]
table {%
5.76 0.0666666666666667
5.84 0.0666666666666667
5.84 0.555014858366742
5.76 0.555014858366742
5.76 0.0666666666666667
};
\addplot [line width=1.0pt, black, opacity=1, forget plot]
table {%
5.8 0.0666666666666667
5.8 0
};
\addplot [line width=1.0pt, black, opacity=1, forget plot]
table {%
5.8 0.555014858366742
5.8 1.24668498154675
};
\addplot [line width=1.0pt, black, forget plot]
table {%
5.78 0
5.82 0
};
\addplot [line width=1.0pt, black, forget plot]
table {%
5.78 1.24668498154675
5.82 1.24668498154675
};
\addplot [line width=1.0pt, black, opacity=0.2, mark=*, mark size=1, mark options={solid}, only marks, forget plot]
table {%
5.8 1.36996771280752
5.8 1.91852783066005
5.8 1.36169321238935
5.8 1.67536129097861
5.8 1.49313661348425
};
\addplot [line width=1.0pt, color0, opacity=1, forget plot]
table {%
0.893333333333333 0
0.973333333333333 0
0.973333333333333 0.206644931712767
0.893333333333333 0.206644931712767
0.893333333333333 0
};
\addplot [line width=1.0pt, color0, opacity=1, forget plot]
table {%
0.933333333333333 0
0.933333333333333 0
};
\addplot [line width=1.0pt, color0, opacity=1, forget plot]
table {%
0.933333333333333 0.206644931712767
0.933333333333333 0.515154392492244
};
\addplot [line width=1.0pt, color0, forget plot]
table {%
0.913333333333333 0
0.953333333333333 0
};
\addplot [line width=1.0pt, color0, forget plot]
table {%
0.913333333333333 0.515154392492244
0.953333333333333 0.515154392492244
};
\addplot [line width=1.0pt, color0, opacity=0.2, mark=*, mark size=1, mark options={solid}, only marks, forget plot]
table {%
0.933333333333333 1.09201648339208
0.933333333333333 1.25399362039845
0.933333333333333 0.65031987958844
0.933333333333333 0.806637297521078
0.933333333333333 1.25099960031968
0.933333333333333 1.1015950031863
0.933333333333333 0.629727672493603
0.933333333333333 0.756637297521078
0.933333333333333 0.689176521961304
0.933333333333333 1.12805141726785
0.933333333333333 1.11803398874989
0.933333333333333 0.707106781186547
0.933333333333333 1.06066017177982
0.933333333333333 0.9
0.933333333333333 1.35724172460302
0.933333333333333 1.55086181285469
0.933333333333333 0.560977222864645
0.933333333333333 0.8
0.933333333333333 0.670820393249937
0.933333333333333 0.863941029804985
0.933333333333333 1.20933866224478
0.933333333333333 1.40890028036054
0.933333333333333 1.1987620953865
0.933333333333333 0.806225774829855
0.933333333333333 0.570710678118655
0.933333333333333 0.715891053163818
0.933333333333333 0.855862138431185
};
\addplot [line width=1.0pt, color0, opacity=1, forget plot]
table {%
1.89333333333333 0.0333333333333334
1.97333333333333 0.0333333333333334
1.97333333333333 0.64410668228231
1.89333333333333 0.64410668228231
1.89333333333333 0.0333333333333334
};
\addplot [line width=1.0pt, color0, opacity=1, forget plot]
table {%
1.93333333333333 0.0333333333333334
1.93333333333333 0
};
\addplot [line width=1.0pt, color0, opacity=1, forget plot]
table {%
1.93333333333333 0.64410668228231
1.93333333333333 1.44417460774894
};
\addplot [line width=1.0pt, color0, forget plot]
table {%
1.91333333333333 0
1.95333333333333 0
};
\addplot [line width=1.0pt, color0, forget plot]
table {%
1.91333333333333 1.44417460774894
1.95333333333333 1.44417460774894
};
\addplot [line width=1.0pt, color0, opacity=0.2, mark=*, mark size=1, mark options={solid}, only marks, forget plot]
table {%
1.93333333333333 2.0564452189282
1.93333333333333 1.78449720325188
1.93333333333333 1.73267028590275
};
\addplot [line width=1.0pt, color0, opacity=1, forget plot]
table {%
2.89333333333333 0.0333333333333334
2.97333333333333 0.0333333333333334
2.97333333333333 0.600231303144333
2.89333333333333 0.600231303144333
2.89333333333333 0.0333333333333334
};
\addplot [line width=1.0pt, color0, opacity=1, forget plot]
table {%
2.93333333333333 0.0333333333333334
2.93333333333333 0
};
\addplot [line width=1.0pt, color0, opacity=1, forget plot]
table {%
2.93333333333333 0.600231303144333
2.93333333333333 1.43460265278412
};
\addplot [line width=1.0pt, color0, forget plot]
table {%
2.91333333333333 0
2.95333333333333 0
};
\addplot [line width=1.0pt, color0, forget plot]
table {%
2.91333333333333 1.43460265278412
2.95333333333333 1.43460265278412
};
\addplot [line width=1.0pt, color0, opacity=0.2, mark=*, mark size=1, mark options={solid}, only marks, forget plot]
table {%
2.93333333333333 1.74709994639449
2.93333333333333 2.47575626877018
2.93333333333333 1.49535240357725
2.93333333333333 1.51192026478969
2.93333333333333 2.1820380635314
2.93333333333333 1.5014814803996
};
\addplot [line width=1.0pt, color0, opacity=1, forget plot]
table {%
3.89333333333333 0.0666666666666664
3.97333333333333 0.0666666666666664
3.97333333333333 0.647802967711067
3.89333333333333 0.647802967711067
3.89333333333333 0.0666666666666664
};
\addplot [line width=1.0pt, color0, opacity=1, forget plot]
table {%
3.93333333333333 0.0666666666666664
3.93333333333333 0
};
\addplot [line width=1.0pt, color0, opacity=1, forget plot]
table {%
3.93333333333333 0.647802967711067
3.93333333333333 1.24869430065516
};
\addplot [line width=1.0pt, color0, forget plot]
table {%
3.91333333333333 0
3.95333333333333 0
};
\addplot [line width=1.0pt, color0, forget plot]
table {%
3.91333333333333 1.24869430065516
3.95333333333333 1.24869430065516
};
\addplot [line width=1.0pt, color0, opacity=0.2, mark=*, mark size=1, mark options={solid}, only marks, forget plot]
table {%
3.93333333333333 1.66321753562839
3.93333333333333 1.89990731053802
3.93333333333333 1.77833881797741
3.93333333333333 1.57001224368817
3.93333333333333 1.55769961924298
};
\addplot [line width=1.0pt, color0, opacity=1, forget plot]
table {%
4.89333333333333 0.047140452079103
4.97333333333333 0.047140452079103
4.97333333333333 0.619188662777825
4.89333333333333 0.619188662777825
4.89333333333333 0.047140452079103
};
\addplot [line width=1.0pt, color0, opacity=1, forget plot]
table {%
4.93333333333333 0.047140452079103
4.93333333333333 0
};
\addplot [line width=1.0pt, color0, opacity=1, forget plot]
table {%
4.93333333333333 0.619188662777825
4.93333333333333 1.30384048104053
};
\addplot [line width=1.0pt, color0, forget plot]
table {%
4.91333333333333 0
4.95333333333333 0
};
\addplot [line width=1.0pt, color0, forget plot]
table {%
4.91333333333333 1.30384048104053
4.95333333333333 1.30384048104053
};
\addplot [line width=1.0pt, color0, opacity=0.2, mark=*, mark size=1, mark options={solid}, only marks, forget plot]
table {%
4.93333333333333 1.49215438855105
4.93333333333333 2.53407601421461
4.93333333333333 1.72397992317918
4.93333333333333 1.84613652354293
};
\addplot [line width=1.0pt, color0, opacity=1, forget plot]
table {%
5.89333333333333 0.0804737854124365
5.97333333333333 0.0804737854124365
5.97333333333333 0.525999504139544
5.89333333333333 0.525999504139544
5.89333333333333 0.0804737854124365
};
\addplot [line width=1.0pt, color0, opacity=1, forget plot]
table {%
5.93333333333333 0.0804737854124365
5.93333333333333 0
};
\addplot [line width=1.0pt, color0, opacity=1, forget plot]
table {%
5.93333333333333 0.525999504139544
5.93333333333333 1.12986599739896
};
\addplot [line width=1.0pt, color0, forget plot]
table {%
5.91333333333333 0
5.95333333333333 0
};
\addplot [line width=1.0pt, color0, forget plot]
table {%
5.91333333333333 1.12986599739896
5.95333333333333 1.12986599739896
};
\addplot [line width=1.0pt, color0, opacity=0.2, mark=*, mark size=1, mark options={solid}, only marks, forget plot]
table {%
5.93333333333333 1.28295670322136
};
\addplot [line width=1.0pt, color1, opacity=1, forget plot]
table {%
1.02666666666667 0
1.10666666666667 0
1.10666666666667 0.11180339887499
1.02666666666667 0.11180339887499
1.02666666666667 0
};
\addplot [line width=1.0pt, color1, opacity=1, forget plot]
table {%
1.06666666666667 0
1.06666666666667 0
};
\addplot [line width=1.0pt, color1, opacity=1, forget plot]
table {%
1.06666666666667 0.11180339887499
1.06666666666667 0.273606797749979
};
\addplot [line width=1.0pt, color1, forget plot]
table {%
1.04666666666667 0
1.08666666666667 0
};
\addplot [line width=1.0pt, color1, forget plot]
table {%
1.04666666666667 0.273606797749979
1.08666666666667 0.273606797749979
};
\addplot [line width=1.0pt, color1, opacity=0.2, mark=*, mark size=1, mark options={solid}, only marks, forget plot]
table {%
1.06666666666667 0.619091105873063
1.06666666666667 0.292080962648189
1.06666666666667 0.353553390593274
1.06666666666667 0.695965324171367
1.06666666666667 1.45715488160056
1.06666666666667 0.457649122254148
1.06666666666667 1.4229530217746
1.06666666666667 0.972098496984652
1.06666666666667 0.3
1.06666666666667 0.643397840021018
1.06666666666667 1.59259401181246
1.06666666666667 1.67705098312484
1.06666666666667 1.0295630140987
1.06666666666667 0.511667273601693
1.06666666666667 0.282842712474619
1.06666666666667 0.316227766016838
1.06666666666667 0.292080962648189
1.06666666666667 0.806637297521078
1.06666666666667 0.682060078537598
};
\addplot [line width=1.0pt, color1, opacity=1, forget plot]
table {%
2.02666666666667 0.0333333333333332
2.10666666666667 0.0333333333333332
2.10666666666667 0.211473251653693
2.02666666666667 0.211473251653693
2.02666666666667 0.0333333333333332
};
\addplot [line width=1.0pt, color1, opacity=1, forget plot]
table {%
2.06666666666667 0.0333333333333332
2.06666666666667 0
};
\addplot [line width=1.0pt, color1, opacity=1, forget plot]
table {%
2.06666666666667 0.211473251653693
2.06666666666667 0.474523236273288
};
\addplot [line width=1.0pt, color1, forget plot]
table {%
2.04666666666667 0
2.08666666666667 0
};
\addplot [line width=1.0pt, color1, forget plot]
table {%
2.04666666666667 0.474523236273288
2.08666666666667 0.474523236273288
};
\addplot [line width=1.0pt, color1, opacity=0.2, mark=*, mark size=1, mark options={solid}, only marks, forget plot]
table {%
2.06666666666667 1.00199760574278
2.06666666666667 1.0968308212069
2.06666666666667 0.8977800099763
2.06666666666667 0.708676387515643
2.06666666666667 0.501512033906242
2.06666666666667 1.79339563726885
2.06666666666667 0.751764845241562
2.06666666666667 1.32706861582629
2.06666666666667 0.688319920801481
2.06666666666667 0.502245917853894
2.06666666666667 0.547383969627615
2.06666666666667 0.616740701766939
2.06666666666667 0.964117635926944
2.06666666666667 0.527221794213066
2.06666666666667 0.886452339546715
2.06666666666667 0.821204934288811
2.06666666666667 0.780313327381308
2.06666666666667 1.01638167279776
2.06666666666667 1.51163367918949
2.06666666666667 0.620836079765516
};
\addplot [line width=1.0pt, color1, opacity=1, forget plot]
table {%
3.02666666666667 0.0333333333333334
3.10666666666667 0.0333333333333334
3.10666666666667 0.213437474581095
3.02666666666667 0.213437474581095
3.02666666666667 0.0333333333333334
};
\addplot [line width=1.0pt, color1, opacity=1, forget plot]
table {%
3.06666666666667 0.0333333333333334
3.06666666666667 0
};
\addplot [line width=1.0pt, color1, opacity=1, forget plot]
table {%
3.06666666666667 0.213437474581095
3.06666666666667 0.480473785412437
};
\addplot [line width=1.0pt, color1, forget plot]
table {%
3.04666666666667 0
3.08666666666667 0
};
\addplot [line width=1.0pt, color1, forget plot]
table {%
3.04666666666667 0.480473785412437
3.08666666666667 0.480473785412437
};
\addplot [line width=1.0pt, color1, opacity=0.2, mark=*, mark size=1, mark options={solid}, only marks, forget plot]
table {%
3.06666666666667 1.84908742835726
3.06666666666667 0.771198120705955
3.06666666666667 1.34920206374247
3.06666666666667 1.48599340992024
3.06666666666667 0.514073503395199
3.06666666666667 1.09567259493967
3.06666666666667 0.52829526005987
3.06666666666667 1.06372256453737
3.06666666666667 0.67016577248476
3.06666666666667 1.57114481521515
3.06666666666667 0.983945301102334
3.06666666666667 1.57971729894359
3.06666666666667 0.86965962870843
3.06666666666667 0.517719817900712
3.06666666666667 0.865541035015396
3.06666666666667 0.900640788750512
3.06666666666667 0.630086993874745
3.06666666666667 0.994897976565293
3.06666666666667 0.615474973219099
3.06666666666667 1.23163926309731
3.06666666666667 0.942869163977371
3.06666666666667 1.52788620362458
3.06666666666667 0.869226987360353
3.06666666666667 0.571488679794335
};
\addplot [line width=1.0pt, color1, opacity=1, forget plot]
table {%
4.02666666666667 0.0666666666666667
4.10666666666667 0.0666666666666667
4.10666666666667 0.235702260395516
4.02666666666667 0.235702260395516
4.02666666666667 0.0666666666666667
};
\addplot [line width=1.0pt, color1, opacity=1, forget plot]
table {%
4.06666666666667 0.0666666666666667
4.06666666666667 0
};
\addplot [line width=1.0pt, color1, opacity=1, forget plot]
table {%
4.06666666666667 0.235702260395516
4.06666666666667 0.430034884762608
};
\addplot [line width=1.0pt, color1, forget plot]
table {%
4.04666666666667 0
4.08666666666667 0
};
\addplot [line width=1.0pt, color1, forget plot]
table {%
4.04666666666667 0.430034884762608
4.08666666666667 0.430034884762608
};
\addplot [line width=1.0pt, color1, opacity=0.2, mark=*, mark size=1, mark options={solid}, only marks, forget plot]
table {%
4.06666666666667 0.999686618766331
4.06666666666667 0.929134493033891
4.06666666666667 0.696284793999944
4.06666666666667 0.63748384988657
4.06666666666667 1.26851709182133
4.06666666666667 0.707520062107167
4.06666666666667 2.20031280051393
4.06666666666667 0.645155061244849
4.06666666666667 0.554798536195643
4.06666666666667 0.504424865014052
4.06666666666667 1.01914897476769
4.06666666666667 0.965127699211297
4.06666666666667 0.694280904158207
4.06666666666667 0.781642380436541
4.06666666666667 1.66065703098726
4.06666666666667 0.654777232569775
4.06666666666667 1.09768099505956
4.06666666666667 0.723354010949847
4.06666666666667 1.23626503703403
4.06666666666667 1.10148245896462
4.06666666666667 1.02394721018265
};
\addplot [line width=1.0pt, color2, opacity=1, forget plot]
table {%
1.16 0
1.24 0
1.24 0.158113883008419
1.16 0.158113883008419
1.16 0
};
\addplot [line width=1.0pt, color2, opacity=1, forget plot]
table {%
1.2 0
1.2 0
};
\addplot [line width=1.0pt, color2, opacity=1, forget plot]
table {%
1.2 0.158113883008419
1.2 0.370245917364383
};
\addplot [line width=1.0pt, color2, forget plot]
table {%
1.18 0
1.22 0
};
\addplot [line width=1.0pt, color2, forget plot]
table {%
1.18 0.370245917364383
1.22 0.370245917364383
};
\addplot [line width=1.0pt, color2, opacity=0.2, mark=*, mark size=1, mark options={solid}, only marks, forget plot]
table {%
1.2 1.48492424049175
1.2 1.65451457371904
1.2 1.57912404307306
1.2 0.763668014961327
1.2 0.403112887414928
1.2 1.15974135047432
1.2 0.585492185168005
1.2 0.750365276322768
};
\addplot [line width=1.0pt, color2, opacity=1, forget plot]
table {%
2.16 0.0333333333333335
2.24 0.0333333333333335
2.24 0.35591779580415
2.16 0.35591779580415
2.16 0.0333333333333335
};
\addplot [line width=1.0pt, color2, opacity=1, forget plot]
table {%
2.2 0.0333333333333335
2.2 0
};
\addplot [line width=1.0pt, color2, opacity=1, forget plot]
table {%
2.2 0.35591779580415
2.2 0.839559108163189
};
\addplot [line width=1.0pt, color2, forget plot]
table {%
2.18 0
2.22 0
};
\addplot [line width=1.0pt, color2, forget plot]
table {%
2.18 0.839559108163189
2.22 0.839559108163189
};
\addplot [line width=1.0pt, color2, opacity=0.2, mark=*, mark size=1, mark options={solid}, only marks, forget plot]
table {%
2.2 1.23385403957214
2.2 0.880025252162939
2.2 0.87842496576069
2.2 0.93392838174146
2.2 1.38267896266787
2.2 1.10840642720138
2.2 0.92306373667385
2.2 1.10540925533895
2.2 1.73727066750637
2.2 1.19629708605797
2.2 1.91960549168352
2.2 0.940826579945508
2.2 0.894427190999916
2.2 1.08074492999827
2.2 0.92828063770466
2.2 1.36179055468288
2.2 1.09215810074824
};
\addplot [line width=1.0pt, color2, opacity=1, forget plot]
table {%
3.16 0.105409255338946
3.24 0.105409255338946
3.24 0.472049594404048
3.16 0.472049594404048
3.16 0.105409255338946
};
\addplot [line width=1.0pt, color2, opacity=1, forget plot]
table {%
3.2 0.105409255338946
3.2 0
};
\addplot [line width=1.0pt, color2, opacity=1, forget plot]
table {%
3.2 0.472049594404048
3.2 1.00796196427493
};
\addplot [line width=1.0pt, color2, forget plot]
table {%
3.18 0
3.22 0
};
\addplot [line width=1.0pt, color2, forget plot]
table {%
3.18 1.00796196427493
3.22 1.00796196427493
};
\addplot [line width=1.0pt, color2, opacity=0.2, mark=*, mark size=1, mark options={solid}, only marks, forget plot]
table {%
3.2 1.29922438527237
3.2 1.71422364875789
3.2 1.15528052924827
3.2 1.02392397824913
3.2 1.61801852489771
3.2 1.20543974892211
3.2 1.73533005221966
3.2 1.49879340584686
3.2 1.1508451001088
3.2 1.52344861560492
3.2 1.55609823986766
3.2 1.11005505368978
3.2 1.05934990547138
3.2 1.54179654112946
3.2 1.3457651338709
3.2 1.26067262207245
3.2 1.16108344262371
};
\addplot [line width=1.0pt, color2, opacity=1, forget plot]
table {%
4.16 0.1
4.24 0.1
4.24 0.857334334152742
4.16 0.857334334152742
4.16 0.1
};
\addplot [line width=1.0pt, color2, opacity=1, forget plot]
table {%
4.2 0.1
4.2 0
};
\addplot [line width=1.0pt, color2, opacity=1, forget plot]
table {%
4.2 0.857334334152742
4.2 1.79137227169487
};
\addplot [line width=1.0pt, color2, forget plot]
table {%
4.18 0
4.22 0
};
\addplot [line width=1.0pt, color2, forget plot]
table {%
4.18 1.79137227169487
4.22 1.79137227169487
};
\addplot [line width=1.0pt, color2, opacity=0.2, mark=*, mark size=1, mark options={solid}, only marks, forget plot]
table {%
4.2 2.47684692417253
};
\addplot [line width=1.0pt, color2, opacity=1, forget plot]
table {%
5.16 0.166666666666667
5.24 0.166666666666667
5.24 0.582556043645021
5.16 0.582556043645021
5.16 0.166666666666667
};
\addplot [line width=1.0pt, color2, opacity=1, forget plot]
table {%
5.2 0.166666666666667
5.2 0
};
\addplot [line width=1.0pt, color2, opacity=1, forget plot]
table {%
5.2 0.582556043645021
5.2 1.0978867132219
};
\addplot [line width=1.0pt, color2, forget plot]
table {%
5.18 0
5.22 0
};
\addplot [line width=1.0pt, color2, forget plot]
table {%
5.18 1.0978867132219
5.22 1.0978867132219
};
\addplot [line width=1.0pt, color2, opacity=0.2, mark=*, mark size=1, mark options={solid}, only marks, forget plot]
table {%
5.2 1.84880359350149
5.2 1.4815535094672
5.2 1.73327772934153
5.2 1.74290383897792
};
\addplot [line width=1.0pt, color2, opacity=1, forget plot]
table {%
6.16 0.169306861488462
6.24 0.169306861488462
6.24 0.573189725667183
6.16 0.573189725667183
6.16 0.169306861488462
};
\addplot [line width=1.0pt, color2, opacity=1, forget plot]
table {%
6.2 0.169306861488462
6.2 0
};
\addplot [line width=1.0pt, color2, opacity=1, forget plot]
table {%
6.2 0.573189725667183
6.2 1.14818135632661
};
\addplot [line width=1.0pt, color2, forget plot]
table {%
6.18 0
6.22 0
};
\addplot [line width=1.0pt, color2, forget plot]
table {%
6.18 1.14818135632661
6.22 1.14818135632661
};
\addplot [line width=1.0pt, color2, opacity=0.2, mark=*, mark size=1, mark options={solid}, only marks, forget plot]
table {%
6.2 1.53675012534779
6.2 1.64455996865898
6.2 1.22356086593249
6.2 1.20639738900664
6.2 1.396411253246
6.2 1.21641408341569
};
\addplot [line width=1.0pt, black, opacity=1, forget plot]
table {%
0.76 0.1
0.84 0.1
};
\addplot [line width=1.0pt, black, dashed, mark=x, mark size=3, mark options={solid}, forget plot]
table {%
0.8 0.239570474733876
};
\addplot [line width=1.0pt, black, opacity=1, forget plot]
table {%
1.76 0.181174693211129
1.84 0.181174693211129
};
\addplot [line width=1.0pt, black, dashed, mark=x, mark size=3, mark options={solid}, forget plot]
table {%
1.8 0.339453407621877
};
\addplot [line width=1.0pt, black, opacity=1, forget plot]
table {%
2.76 0.240937800135773
2.84 0.240937800135773
};
\addplot [line width=1.0pt, black, dashed, mark=x, mark size=3, mark options={solid}, forget plot]
table {%
2.8 0.392653442168253
};
\addplot [line width=1.0pt, black, opacity=1, forget plot]
table {%
3.76 0.28826744107375
3.84 0.28826744107375
};
\addplot [line width=1.0pt, black, dashed, mark=x, mark size=3, mark options={solid}, forget plot]
table {%
3.8 0.488350578705671
};
\addplot [line width=1.0pt, black, opacity=1, forget plot]
table {%
4.76 0.232623622153969
4.84 0.232623622153969
};
\addplot [line width=1.0pt, black, dashed, mark=x, mark size=3, mark options={solid}, forget plot]
table {%
4.8 0.365424479598446
};
\addplot [line width=1.0pt, black, opacity=1, forget plot]
table {%
5.76 0.223941719505155
5.84 0.223941719505155
};
\addplot [line width=1.0pt, black, dashed, mark=x, mark size=3, mark options={solid}, forget plot]
table {%
5.8 0.356400540129317
};
\addplot [line width=1.0pt, color0, opacity=1, forget plot]
table {%
0.893333333333333 0.05
0.973333333333333 0.05
};
\addplot [line width=1.0pt, color0, dashed, mark=x, mark size=3, mark options={solid}, forget plot]
table {%
0.933333333333333 0.197967238937681
};
\addplot [line width=1.0pt, color0, opacity=1, forget plot]
table {%
1.89333333333333 0.19436506316151
1.97333333333333 0.19436506316151
};
\addplot [line width=1.0pt, color0, dashed, mark=x, mark size=3, mark options={solid}, forget plot]
table {%
1.93333333333333 0.375088506798603
};
\addplot [line width=1.0pt, color0, opacity=1, forget plot]
table {%
2.89333333333333 0.205409255338946
2.97333333333333 0.205409255338946
};
\addplot [line width=1.0pt, color0, dashed, mark=x, mark size=3, mark options={solid}, forget plot]
table {%
2.93333333333333 0.37221469556599
};
\addplot [line width=1.0pt, color0, opacity=1, forget plot]
table {%
3.89333333333333 0.213058507413378
3.97333333333333 0.213058507413378
};
\addplot [line width=1.0pt, color0, dashed, mark=x, mark size=3, mark options={solid}, forget plot]
table {%
3.93333333333333 0.374724693070953
};
\addplot [line width=1.0pt, color0, opacity=1, forget plot]
table {%
4.89333333333333 0.267349663966481
4.97333333333333 0.267349663966481
};
\addplot [line width=1.0pt, color0, dashed, mark=x, mark size=3, mark options={solid}, forget plot]
table {%
4.93333333333333 0.412248702556569
};
\addplot [line width=1.0pt, color0, opacity=1, forget plot]
table {%
5.89333333333333 0.254543789255938
5.97333333333333 0.254543789255938
};
\addplot [line width=1.0pt, color0, dashed, mark=x, mark size=3, mark options={solid}, forget plot]
table {%
5.93333333333333 0.351609909522366
};
\addplot [line width=1.0pt, color1, opacity=1, forget plot]
table {%
1.02666666666667 0.0499999999999998
1.10666666666667 0.0499999999999998
};
\addplot [line width=1.0pt, color1, dashed, mark=x, mark size=3, mark options={solid}, forget plot]
table {%
1.06666666666667 0.123564454059492
};
\addplot [line width=1.0pt, color1, opacity=1, forget plot]
table {%
2.02666666666667 0.0804737854124363
2.10666666666667 0.0804737854124363
};
\addplot [line width=1.0pt, color1, dashed, mark=x, mark size=3, mark options={solid}, forget plot]
table {%
2.06666666666667 0.180898282702756
};
\addplot [line width=1.0pt, color1, opacity=1, forget plot]
table {%
3.02666666666667 0.0942809041582063
3.10666666666667 0.0942809041582063
};
\addplot [line width=1.0pt, color1, dashed, mark=x, mark size=3, mark options={solid}, forget plot]
table {%
3.06666666666667 0.20844855898249
};
\addplot [line width=1.0pt, color1, opacity=1, forget plot]
table {%
4.02666666666667 0.120185042515466
4.10666666666667 0.120185042515466
};
\addplot [line width=1.0pt, color1, dashed, mark=x, mark size=3, mark options={solid}, forget plot]
table {%
4.06666666666667 0.240513329684954
};
\addplot [line width=1.0pt, color2, opacity=1, forget plot]
table {%
1.16 0.0499999999999998
1.24 0.0499999999999998
};
\addplot [line width=1.0pt, color2, dashed, mark=x, mark size=3, mark options={solid}, forget plot]
table {%
1.2 0.113985633215155
};
\addplot [line width=1.0pt, color2, opacity=1, forget plot]
table {%
2.16 0.129556452758176
2.24 0.129556452758176
};
\addplot [line width=1.0pt, color2, dashed, mark=x, mark size=3, mark options={solid}, forget plot]
table {%
2.2 0.282343031273619
};
\addplot [line width=1.0pt, color2, opacity=1, forget plot]
table {%
3.16 0.194322983659858
3.24 0.194322983659858
};
\addplot [line width=1.0pt, color2, dashed, mark=x, mark size=3, mark options={solid}, forget plot]
table {%
3.2 0.355064348866971
};
\addplot [line width=1.0pt, color2, opacity=1, forget plot]
table {%
4.16 0.311648478254428
4.24 0.311648478254428
};
\addplot [line width=1.0pt, color2, dashed, mark=x, mark size=3, mark options={solid}, forget plot]
table {%
4.2 0.480771691093811
};
\addplot [line width=1.0pt, color2, opacity=1, forget plot]
table {%
5.16 0.289652156714555
5.24 0.289652156714555
};
\addplot [line width=1.0pt, color2, dashed, mark=x, mark size=3, mark options={solid}, forget plot]
table {%
5.2 0.428705141878362
};
\addplot [line width=1.0pt, color2, opacity=1, forget plot]
table {%
6.16 0.312314613927134
6.24 0.312314613927134
};
\addplot [line width=1.0pt, color2, dashed, mark=x, mark size=3, mark options={solid}, forget plot]
table {%
6.2 0.424150798055071
};
\end{axis}

\node at ({$(current bounding box.south west)!0.5!(current bounding box.south east)$}|-{$(current bounding box.south west)!0.98!(current bounding box.north west)$})[
  anchor=north,
  text=black,
  rotate=0.0
]{ };

	    \begin{customlegend}[
legend entries={md\ $=0.1$m,md\ $=0.3$m,md\ $=0.5$m,md\ $=1.0$m},
legend cell align=left,
legend style={at={(0.05,5.37)}, anchor=north west, draw=white!80.0!black, font=\footnotesize,fill opacity=0.5, draw opacity=1,text opacity=1}]
    \addlegendimage{area legend,black,fill=black, fill opacity=1}
    \addlegendimage{area legend,color0,fill=color0, fill opacity=1}
    \addlegendimage{area legend,color1,fill=color1, fill opacity=1}
    \addlegendimage{area legend,color2,fill=color2, fill opacity=1}
\end{customlegend}
	\end{tikzpicture}
	\caption[Evaluation results for varying \glsentryshort{em} iterations]{Evaluation results for varying \glsentryshort{em} iterations: }
	\label{fig:trial1}
\end{figure}

The mean localisation error for a minimum required source distance seems to be identical for 0.1m - 0.5m. With a required distance of 1.0m although, localisation performance deteriorates significantly for $S>3$. This is surprising, as intuitively, the more space there is between the sources, the easier it should be for the algorithm to estimate the location for each individual source more precisely. 

\begin{itemize}
    \item increasing md does not yield better performance overall
    \item for 0.1-0.5m, effect seems to be negligible
    \item md=1.0m has severe negative effects on performance
    \item this might be due to more sources being placed close to the wall (to follow the minimum distance requirement)
\end{itemize}