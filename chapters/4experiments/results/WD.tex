\subsubsection*{Wall Distance}

\begin{figure}[H]
\iftoggle{quick}{%
    \includegraphics[width=\textwidth]{plots/boxplots/boxplot-joined-wd}
}{%
    	\begin{tikzpicture}
	    % This file was created by matplotlib2tikz v0.6.14.
\definecolor{color0}{rgb}{0.8,0.207843137254902,0.219607843137255}
\definecolor{color1}{rgb}{1,0.647058823529412,0}

\begin{axis}[
xlabel={$S$},
ylabel={MAE},
xmin=0.5, xmax=6.5,
ymin=0, ymax=2.5,
width=\figurewidth,
height=\figureheight,
xtick={1,2,3,4,5,6},
xticklabels={2,3,4,5,6,7},
ytick={0,0.5,1,1.5,2,2.5},
minor xtick={},
minor ytick={},
tick align=outside,
tick pos=left,
x grid style={white!69.019607843137251!black},
ymajorgrids,
y grid style={white!69.019607843137251!black}
]
\addplot [line width=1.0pt, black, opacity=1, forget plot]
table {%
0.81 0
0.89 0
0.89 0.120710678118655
0.81 0.120710678118655
0.81 0
};
\addplot [line width=1.0pt, black, opacity=1, forget plot]
table {%
0.85 0
0.85 0
};
\addplot [line width=1.0pt, black, opacity=1, forget plot]
table {%
0.85 0.120710678118655
0.85 0.3
};
\addplot [line width=1.0pt, black, forget plot]
table {%
0.83 0
0.87 0
};
\addplot [line width=1.0pt, black, forget plot]
table {%
0.83 0.3
0.87 0.3
};
\addplot [line width=0.5pt, black, opacity=0.2, mark=*, mark size=1, mark options={solid}, only marks, forget plot]
table {%
0.85 0.353553390593274
0.85 0.531507290636732
0.85 0.403112887414927
0.85 0.559016994374947
0.85 0.396372331916949
0.85 0.320156211871642
0.85 0.365028153987288
0.85 0.370156211871642
0.85 0.629812939611249
0.85 0.570087712549569
0.85 0.319258240356725
0.85 0.821532350411043
0.85 0.370156211871642
0.85 0.615244921770892
0.85 1.31189134730927
0.85 0.540832691319598
0.85 1.57397506541282
0.85 0.370156211871642
0.85 0.553112887414927
0.85 1.32938296506358
0.85 0.461223161913988
0.85 1.22225742387742
0.85 1.75019117647915
};
\addplot [line width=1.0pt, black, opacity=1, forget plot]
table {%
1.81 0.0333333333333332
1.89 0.0333333333333332
1.89 0.241970265751807
1.81 0.241970265751807
1.81 0.0333333333333332
};
\addplot [line width=1.0pt, black, opacity=1, forget plot]
table {%
1.85 0.0333333333333332
1.85 0
};
\addplot [line width=1.0pt, black, opacity=1, forget plot]
table {%
1.85 0.241970265751807
1.85 0.49397153130637
};
\addplot [line width=1.0pt, black, forget plot]
table {%
1.83 0
1.87 0
};
\addplot [line width=1.0pt, black, forget plot]
table {%
1.83 0.49397153130637
1.87 0.49397153130637
};
\addplot [line width=0.5pt, black, opacity=0.2, mark=*, mark size=1, mark options={solid}, only marks, forget plot]
table {%
1.85 1.22371473299859
1.85 1.26183288204985
1.85 1.20695471545204
1.85 0.674535599249993
1.85 1.0029367231947
1.85 0.912553036929503
1.85 0.928447050641566
1.85 0.6
1.85 0.982016631383847
1.85 0.740440114519881
1.85 1.00808802290398
1.85 1.0420384286444
1.85 0.960947570824873
1.85 0.943762586610346
1.85 1.37308866317009
1.85 0.76761154691623
1.85 0.651898485986552
1.85 0.695224942816886
1.85 1.02272147197706
1.85 0.91053134618229
};
\addplot [line width=1.0pt, black, opacity=1, forget plot]
table {%
2.81 0.0592419291538691
2.89 0.0592419291538691
2.89 0.300547933335521
2.81 0.300547933335521
2.81 0.0592419291538691
};
\addplot [line width=1.0pt, black, opacity=1, forget plot]
table {%
2.85 0.0592419291538691
2.85 0
};
\addplot [line width=1.0pt, black, opacity=1, forget plot]
table {%
2.85 0.300547933335521
2.85 0.642454451761424
};
\addplot [line width=1.0pt, black, forget plot]
table {%
2.83 0
2.87 0
};
\addplot [line width=1.0pt, black, forget plot]
table {%
2.83 0.642454451761424
2.87 0.642454451761424
};
\addplot [line width=0.5pt, black, opacity=0.2, mark=*, mark size=1, mark options={solid}, only marks, forget plot]
table {%
2.85 0.723974124267613
2.85 0.777956079402704
2.85 0.910161069436712
2.85 1.38536739464369
2.85 1.01035183961776
2.85 0.771040481324095
2.85 0.749516545721839
2.85 0.963308622881019
2.85 0.922062731607737
2.85 0.669251310792162
2.85 0.677375577743225
2.85 0.937297345880458
2.85 0.961791461512756
2.85 1.3972971081941
2.85 1.65767642266885
2.85 0.74344345637708
2.85 0.685812765083981
2.85 0.776208734813001
2.85 0.778540444945756
2.85 0.700194593719363
2.85 1.02516777484405
2.85 0.815569415042095
2.85 0.805982736611292
2.85 0.777079299987345
2.85 1.04860679774998
2.85 0.870284369629842
};
\addplot [line width=1.0pt, black, opacity=1, forget plot]
table {%
3.81 0.0682842712474618
3.89 0.0682842712474618
3.89 0.612006614560583
3.81 0.612006614560583
3.81 0.0682842712474618
};
\addplot [line width=1.0pt, black, opacity=1, forget plot]
table {%
3.85 0.0682842712474618
3.85 0
};
\addplot [line width=1.0pt, black, opacity=1, forget plot]
table {%
3.85 0.612006614560583
3.85 1.40859871624658
};
\addplot [line width=1.0pt, black, forget plot]
table {%
3.83 0
3.87 0
};
\addplot [line width=1.0pt, black, forget plot]
table {%
3.83 1.40859871624658
3.87 1.40859871624658
};
\addplot [line width=0.5pt, black, opacity=0.2, mark=*, mark size=1, mark options={solid}, only marks, forget plot]
table {%
3.85 1.65155969442983
};
\addplot [line width=1.0pt, black, opacity=1, forget plot]
table {%
4.81 0.13563053367813
4.89 0.13563053367813
4.89 0.676100669953884
4.81 0.676100669953884
4.81 0.13563053367813
};
\addplot [line width=1.0pt, black, opacity=1, forget plot]
table {%
4.85 0.13563053367813
4.85 0.0166666666666666
};
\addplot [line width=1.0pt, black, opacity=1, forget plot]
table {%
4.85 0.676100669953884
4.85 1.35725600520391
};
\addplot [line width=1.0pt, black, forget plot]
table {%
4.83 0.0166666666666666
4.87 0.0166666666666666
};
\addplot [line width=1.0pt, black, forget plot]
table {%
4.83 1.35725600520391
4.87 1.35725600520391
};
\addplot [line width=1.0pt, black, opacity=1, forget plot]
table {%
5.81 0.165913049623586
5.89 0.165913049623586
5.89 0.646471586519708
5.81 0.646471586519708
5.81 0.165913049623586
};
\addplot [line width=1.0pt, black, opacity=1, forget plot]
table {%
5.85 0.165913049623586
5.85 0.0166666666666667
};
\addplot [line width=1.0pt, black, opacity=1, forget plot]
table {%
5.85 0.646471586519708
5.85 1.20574268613061
};
\addplot [line width=1.0pt, black, forget plot]
table {%
5.83 0.0166666666666667
5.87 0.0166666666666667
};
\addplot [line width=1.0pt, black, forget plot]
table {%
5.83 1.20574268613061
5.87 1.20574268613061
};
\addplot [line width=0.5pt, black, opacity=0.2, mark=*, mark size=1, mark options={solid}, only marks, forget plot]
table {%
5.85 1.41776141898511
};
\addplot [line width=1.0pt, color0, opacity=1, forget plot]
table {%
0.96 0
1.04 0
1.04 0.1
0.96 0.1
0.96 0
};
\addplot [line width=1.0pt, color0, opacity=1, forget plot]
table {%
1 0
1 0
};
\addplot [line width=1.0pt, color0, opacity=1, forget plot]
table {%
1 0.1
1 0.25
};
\addplot [line width=1.0pt, color0, forget plot]
table {%
0.98 0
1.02 0
};
\addplot [line width=1.0pt, color0, forget plot]
table {%
0.98 0.25
1.02 0.25
};
\addplot [line width=0.5pt, color0, opacity=0.2, mark=*, mark size=1, mark options={solid}, only marks, forget plot]
table {%
1 0.95156097709407
1 0.353553390593274
1 0.62658490592434
1 0.602644518151297
1 0.320156211871642
1 0.269258240356725
1 0.320156211871642
1 0.25
1 0.25
1 0.282842712474619
1 0.320710678118655
1 0.262132034355964
1 0.25
1 0.320156211871642
1 0.291547594742265
1 0.256155281280883
1 0.390512483795333
1 0.403112887414928
1 0.25
1 0.353553390593274
1 0.370156211871642
1 1.10679718105893
};
\addplot [line width=1.0pt, color0, opacity=1, forget plot]
table {%
1.96 0.0333333333333332
2.04 0.0333333333333332
2.04 0.208515936929035
1.96 0.208515936929035
1.96 0.0333333333333332
};
\addplot [line width=1.0pt, color0, opacity=1, forget plot]
table {%
2 0.0333333333333332
2 0
};
\addplot [line width=1.0pt, color0, opacity=1, forget plot]
table {%
2 0.208515936929035
2 0.415124774601576
};
\addplot [line width=1.0pt, color0, forget plot]
table {%
1.98 0
2.02 0
};
\addplot [line width=1.0pt, color0, forget plot]
table {%
1.98 0.415124774601576
2.02 0.415124774601576
};
\addplot [line width=0.5pt, color0, opacity=0.2, mark=*, mark size=1, mark options={solid}, only marks, forget plot]
table {%
2 0.734971706792116
2 1.43519802968925
2 1.28153754992212
2 0.788514229886163
2 0.547059991677656
2 0.802969009066879
2 0.780357813049499
2 0.90575015522016
2 1.07544844885062
2 0.883246757331923
2 0.597195922471373
2 1.66673724566923
2 1.235183758488
2 1.284468322287
2 1.51213911173956
2 1.28452119829714
2 0.658257878000665
2 1.54624885400211
2 1.18709021713524
2 0.929755045398757
2 1.47346907368669
2 1.2207692058278
2 0.807645389459296
2 0.823551306075313
2 1.47348638559418
2 1.20462073330618
};
\addplot [line width=1.0pt, color0, opacity=1, forget plot]
table {%
2.96 0.0353553390593274
3.04 0.0353553390593274
3.04 0.391471614885363
2.96 0.391471614885363
2.96 0.0353553390593274
};
\addplot [line width=1.0pt, color0, opacity=1, forget plot]
table {%
3 0.0353553390593274
3 0
};
\addplot [line width=1.0pt, color0, opacity=1, forget plot]
table {%
3 0.391471614885363
3 0.909728412174481
};
\addplot [line width=1.0pt, color0, forget plot]
table {%
2.98 0
3.02 0
};
\addplot [line width=1.0pt, color0, forget plot]
table {%
2.98 0.909728412174481
3.02 0.909728412174481
};
\addplot [line width=0.5pt, color0, opacity=0.2, mark=*, mark size=1, mark options={solid}, only marks, forget plot]
table {%
3 0.981228448019593
3 1.71882377336844
3 1.22976164084571
3 1.40144859971549
3 0.926570605213951
3 1.13781258586612
3 1.51527064494447
};
\addplot [line width=1.0pt, color0, opacity=1, forget plot]
table {%
3.96 0.079142135623731
4.04 0.079142135623731
4.04 0.510360380155207
3.96 0.510360380155207
3.96 0.079142135623731
};
\addplot [line width=1.0pt, color0, opacity=1, forget plot]
table {%
4 0.079142135623731
4 0
};
\addplot [line width=1.0pt, color0, opacity=1, forget plot]
table {%
4 0.510360380155207
4 1.15635610590157
};
\addplot [line width=1.0pt, color0, forget plot]
table {%
3.98 0
4.02 0
};
\addplot [line width=1.0pt, color0, forget plot]
table {%
3.98 1.15635610590157
4.02 1.15635610590157
};
\addplot [line width=0.5pt, color0, opacity=0.2, mark=*, mark size=1, mark options={solid}, only marks, forget plot]
table {%
4 1.43624273464524
4 1.39564832059948
4 1.27573351385039
4 1.2169991928458
};
\addplot [line width=1.0pt, color0, opacity=1, forget plot]
table {%
4.96 0.146383532914452
5.04 0.146383532914452
5.04 0.614779989423924
4.96 0.614779989423924
4.96 0.146383532914452
};
\addplot [line width=1.0pt, color0, opacity=1, forget plot]
table {%
5 0.146383532914452
5 0
};
\addplot [line width=1.0pt, color0, opacity=1, forget plot]
table {%
5 0.614779989423924
5 1.21403644848723
};
\addplot [line width=1.0pt, color0, forget plot]
table {%
4.98 0
5.02 0
};
\addplot [line width=1.0pt, color0, forget plot]
table {%
4.98 1.21403644848723
5.02 1.21403644848723
};
\addplot [line width=0.5pt, color0, opacity=0.2, mark=*, mark size=1, mark options={solid}, only marks, forget plot]
table {%
5 1.72005095676293
};
\addplot [line width=1.0pt, color0, opacity=1, forget plot]
table {%
5.96 0.149344344309341
6.04 0.149344344309341
6.04 0.53600869611076
5.96 0.53600869611076
5.96 0.149344344309341
};
\addplot [line width=1.0pt, color0, opacity=1, forget plot]
table {%
6 0.149344344309341
6 0.0166666666666666
};
\addplot [line width=1.0pt, color0, opacity=1, forget plot]
table {%
6 0.53600869611076
6 1.04842494302803
};
\addplot [line width=1.0pt, color0, forget plot]
table {%
5.98 0.0166666666666666
6.02 0.0166666666666666
};
\addplot [line width=1.0pt, color0, forget plot]
table {%
5.98 1.04842494302803
6.02 1.04842494302803
};
\addplot [line width=0.5pt, color0, opacity=0.2, mark=*, mark size=1, mark options={solid}, only marks, forget plot]
table {%
6 1.26005865639591
6 1.16172936633607
6 1.16673333870788
6 1.14048937029867
};
\addplot [line width=1.0pt, color1, opacity=1, forget plot]
table {%
1.11 0
1.19 0
1.19 0.05
1.11 0.05
1.11 0
};
\addplot [line width=1.0pt, color1, opacity=1, forget plot]
table {%
1.15 0
1.15 0
};
\addplot [line width=1.0pt, color1, opacity=1, forget plot]
table {%
1.15 0.05
1.15 0.11180339887499
};
\addplot [line width=1.0pt, color1, forget plot]
table {%
1.13 0
1.17 0
};
\addplot [line width=1.0pt, color1, forget plot]
table {%
1.13 0.11180339887499
1.17 0.11180339887499
};
\addplot [line width=0.5pt, color1, opacity=0.2, mark=*, mark size=1, mark options={solid}, only marks, forget plot]
table {%
1.15 0.158113883008419
1.15 0.141421356237309
1.15 0.515154392492244
1.15 0.806637297521078
1.15 0.982735008131951
1.15 0.403553390593274
1.15 0.320156211871642
1.15 1.29828930810756
1.15 0.353553390593274
1.15 1.66207701385947
1.15 0.223606797749979
1.15 0.180277563773199
1.15 1.17046999107196
1.15 0.291547594742265
1.15 0.158113883008419
1.15 0.158113883008419
1.15 0.282842712474619
1.15 1.47648230602334
1.15 1.75
1.15 0.158113883008419
1.15 0.25
1.15 0.320156211871642
1.15 0.440956595483038
1.15 0.320156211871642
1.15 0.25
1.15 0.320156211871642
1.15 0.223606797749979
1.15 0.341547594742265
1.15 0.158113883008419
};
\addplot [line width=1.0pt, color1, opacity=1, forget plot]
table {%
2.11 0
2.19 0
2.19 0.105046260628866
2.11 0.105046260628866
2.11 0
};
\addplot [line width=1.0pt, color1, opacity=1, forget plot]
table {%
2.15 0
2.15 0
};
\addplot [line width=1.0pt, color1, opacity=1, forget plot]
table {%
2.15 0.105046260628866
2.15 0.260341655863555
};
\addplot [line width=1.0pt, color1, forget plot]
table {%
2.13 0
2.17 0
};
\addplot [line width=1.0pt, color1, forget plot]
table {%
2.13 0.260341655863555
2.17 0.260341655863555
};
\addplot [line width=0.5pt, color1, opacity=0.2, mark=*, mark size=1, mark options={solid}, only marks, forget plot]
table {%
2.15 0.53437398472938
2.15 0.634258545910665
2.15 1.24224366598401
2.15 0.273703418364266
2.15 0.269035593728849
2.15 1.00813543754376
2.15 0.290955230036055
2.15 0.843932593411478
2.15 0.885513396586707
2.15 0.803385768857251
2.15 0.582141639885766
2.15 0.776506703047276
2.15 0.844368337373096
2.15 0.924361641590803
2.15 0.7
2.15 0.795645854342791
2.15 0.52549410200778
2.15 0.421637021355784
2.15 0.747406836728532
2.15 0.293674989196888
2.15 0.568068481525776
2.15 0.690668171555645
2.15 0.582141639885766
2.15 1.17106987772507
2.15 1.05413513778284
2.15 0.374151180282231
2.15 0.284914748910096
2.15 0.405183573839402
2.15 0.376769594994829
};
\addplot [line width=1.0pt, color1, opacity=1, forget plot]
table {%
3.11 0.0249999999999999
3.19 0.0249999999999999
3.19 0.209011080610204
3.11 0.209011080610204
3.11 0.0249999999999999
};
\addplot [line width=1.0pt, color1, opacity=1, forget plot]
table {%
3.15 0.0249999999999999
3.15 0
};
\addplot [line width=1.0pt, color1, opacity=1, forget plot]
table {%
3.15 0.209011080610204
3.15 0.466922251600129
};
\addplot [line width=1.0pt, color1, forget plot]
table {%
3.13 0
3.17 0
};
\addplot [line width=1.0pt, color1, forget plot]
table {%
3.13 0.466922251600129
3.17 0.466922251600129
};
\addplot [line width=0.5pt, color1, opacity=0.2, mark=*, mark size=1, mark options={solid}, only marks, forget plot]
table {%
3.15 0.685078105935821
3.15 0.762045055083761
3.15 0.716465834296967
3.15 0.95247349701973
3.15 0.819793511800304
3.15 0.499341649025257
3.15 0.676444720213127
3.15 0.945668656263582
3.15 0.714795153876199
3.15 0.673608525231095
3.15 0.785078105935821
3.15 1.36638495891758
3.15 1.03142230691071
3.15 0.510318296899795
3.15 1.01091066031266
3.15 0.623292958428326
3.15 0.825847322807905
3.15 0.704152273039862
3.15 0.696511439016672
3.15 1.27241905290658
3.15 0.488901766884969
3.15 0.685855139219168
3.15 0.548978772520395
3.15 1.03527112271824
3.15 0.66069008366105
};
\addplot [line width=1.0pt, color1, opacity=1, forget plot]
table {%
4.11 0.06
4.19 0.06
4.19 0.575296840784572
4.11 0.575296840784572
4.11 0.06
};
\addplot [line width=1.0pt, color1, opacity=1, forget plot]
table {%
4.15 0.06
4.15 0
};
\addplot [line width=1.0pt, color1, opacity=1, forget plot]
table {%
4.15 0.575296840784572
4.15 1.28951110756165
};
\addplot [line width=1.0pt, color1, forget plot]
table {%
4.13 0
4.17 0
};
\addplot [line width=1.0pt, color1, forget plot]
table {%
4.13 1.28951110756165
4.17 1.28951110756165
};
\addplot [line width=0.5pt, color1, opacity=0.2, mark=*, mark size=1, mark options={solid}, only marks, forget plot]
table {%
4.15 1.56941607803541
4.15 1.50099238558455
};
\addplot [line width=1.0pt, color1, opacity=1, forget plot]
table {%
5.11 0.0999999999999999
5.19 0.0999999999999999
5.19 0.530946800273526
5.11 0.530946800273526
5.11 0.0999999999999999
};
\addplot [line width=1.0pt, color1, opacity=1, forget plot]
table {%
5.15 0.0999999999999999
5.15 0
};
\addplot [line width=1.0pt, color1, opacity=1, forget plot]
table {%
5.15 0.530946800273526
5.15 1.05094678869984
};
\addplot [line width=1.0pt, color1, forget plot]
table {%
5.13 0
5.17 0
};
\addplot [line width=1.0pt, color1, forget plot]
table {%
5.13 1.05094678869984
5.17 1.05094678869984
};
\addplot [line width=1.0pt, color1, opacity=1, forget plot]
table {%
6.11 0.107627589280716
6.19 0.107627589280716
6.19 0.556103409045645
6.11 0.556103409045645
6.11 0.107627589280716
};
\addplot [line width=1.0pt, color1, opacity=1, forget plot]
table {%
6.15 0.107627589280716
6.15 0
};
\addplot [line width=1.0pt, color1, opacity=1, forget plot]
table {%
6.15 0.556103409045645
6.15 1.18285242444964
};
\addplot [line width=1.0pt, color1, forget plot]
table {%
6.13 0
6.17 0
};
\addplot [line width=1.0pt, color1, forget plot]
table {%
6.13 1.18285242444964
6.17 1.18285242444964
};
\addplot [line width=0.5pt, color1, opacity=0.2, mark=*, mark size=1, mark options={solid}, only marks, forget plot]
table {%
6.15 1.41079541912523
};
\addplot [line width=1.0pt, black, opacity=1, forget plot]
table {%
0.81 0.0499999999999998
0.89 0.0499999999999998
};
\addplot [line width=1.0pt, black, dashed, mark=x, mark size=3, mark options={solid}, forget plot]
table {%
0.85 0.122369494174645
};
\addplot [line width=1.0pt, black, opacity=1, forget plot]
table {%
1.81 0.099845079748576
1.89 0.099845079748576
};
\addplot [line width=1.0pt, black, dashed, mark=x, mark size=3, mark options={solid}, forget plot]
table {%
1.85 0.198970324699303
};
\addplot [line width=1.0pt, black, opacity=1, forget plot]
table {%
2.81 0.105983858307738
2.89 0.105983858307738
};
\addplot [line width=1.0pt, black, dashed, mark=x, mark size=3, mark options={solid}, forget plot]
table {%
2.85 0.246801848263101
};
\addplot [line width=1.0pt, black, opacity=1, forget plot]
table {%
3.81 0.216963839251665
3.89 0.216963839251665
};
\addplot [line width=1.0pt, black, dashed, mark=x, mark size=3, mark options={solid}, forget plot]
table {%
3.85 0.358590564635279
};
\addplot [line width=1.0pt, black, opacity=1, forget plot]
table {%
4.81 0.403234806953404
4.89 0.403234806953404
};
\addplot [line width=1.0pt, black, dashed, mark=x, mark size=3, mark options={solid}, forget plot]
table {%
4.85 0.459884468515693
};
\addplot [line width=1.0pt, black, opacity=1, forget plot]
table {%
5.81 0.398604760529198
5.89 0.398604760529198
};
\addplot [line width=1.0pt, black, dashed, mark=x, mark size=3, mark options={solid}, forget plot]
table {%
5.85 0.435271738130316
};
\addplot [line width=1.0pt, color0, opacity=1, forget plot]
table {%
0.96 0
1.04 0
};
\addplot [line width=1.0pt, color0, dashed, mark=x, mark size=3, mark options={solid}, forget plot]
table {%
1 0.0783929349193506
};
\addplot [line width=1.0pt, color0, opacity=1, forget plot]
table {%
1.96 0.0999999999999999
2.04 0.0999999999999999
};
\addplot [line width=1.0pt, color0, dashed, mark=x, mark size=3, mark options={solid}, forget plot]
table {%
2 0.224473924396022
};
\addplot [line width=1.0pt, color0, opacity=1, forget plot]
table {%
2.96 0.1
3.04 0.1
};
\addplot [line width=1.0pt, color0, dashed, mark=x, mark size=3, mark options={solid}, forget plot]
table {%
3 0.258895252821338
};
\addplot [line width=1.0pt, color0, opacity=1, forget plot]
table {%
3.96 0.180710678118655
4.04 0.180710678118655
};
\addplot [line width=1.0pt, color0, dashed, mark=x, mark size=3, mark options={solid}, forget plot]
table {%
4 0.324251712827345
};
\addplot [line width=1.0pt, color0, opacity=1, forget plot]
table {%
4.96 0.370929958942574
5.04 0.370929958942574
};
\addplot [line width=1.0pt, color0, dashed, mark=x, mark size=3, mark options={solid}, forget plot]
table {%
5 0.416326629464858
};
\addplot [line width=1.0pt, color0, opacity=1, forget plot]
table {%
5.96 0.299669164688408
6.04 0.299669164688408
};
\addplot [line width=1.0pt, color0, dashed, mark=x, mark size=3, mark options={solid}, forget plot]
table {%
6 0.365185678184156
};
\addplot [line width=1.0pt, color1, opacity=1, forget plot]
table {%
1.11 0
1.19 0
};
\addplot [line width=1.0pt, color1, dashed, mark=x, mark size=3, mark options={solid}, forget plot]
table {%
1.15 0.0879494440355321
};
\addplot [line width=1.0pt, color1, opacity=1, forget plot]
table {%
2.11 0.0333333333333334
2.19 0.0333333333333334
};
\addplot [line width=1.0pt, color1, dashed, mark=x, mark size=3, mark options={solid}, forget plot]
table {%
2.15 0.132231189458433
};
\addplot [line width=1.0pt, color1, opacity=1, forget plot]
table {%
3.11 0.0603553390593274
3.19 0.0603553390593274
};
\addplot [line width=1.0pt, color1, dashed, mark=x, mark size=3, mark options={solid}, forget plot]
table {%
3.15 0.179443655214603
};
\addplot [line width=1.0pt, color1, opacity=1, forget plot]
table {%
4.11 0.174585093755703
4.19 0.174585093755703
};
\addplot [line width=1.0pt, color1, dashed, mark=x, mark size=3, mark options={solid}, forget plot]
table {%
4.15 0.323093320165174
};
\addplot [line width=1.0pt, color1, opacity=1, forget plot]
table {%
5.11 0.28186761362165
5.19 0.28186761362165
};
\addplot [line width=1.0pt, color1, dashed, mark=x, mark size=3, mark options={solid}, forget plot]
table {%
5.15 0.337629529498116
};
\addplot [line width=1.0pt, color1, opacity=1, forget plot]
table {%
6.11 0.307525330587592
6.19 0.307525330587592
};
\addplot [line width=1.0pt, color1, dashed, mark=x, mark size=3, mark options={solid}, forget plot]
table {%
6.15 0.354192832256851
};
\end{axis}

\node at ({$(current bounding box.south west)!0.5!(current bounding box.south east)$}|-{$(current bounding box.south west)!0.98!(current bounding box.north west)$})[
  anchor=north,
  text=black,
  rotate=0.0
]{ };

	    \begin{customlegend}[
legend entries={$d^w_{s,\text{min}}=1.2$m,$d^w_{s,\text{min}}=1.3$m,$d^w_{s,\text{min}}=1.5$m},
legend cell align=left,
legend style={at={(0.05,5.37)}, anchor=north west, draw=white!80.0!black, font=\footnotesize,fill opacity=0.5, draw opacity=1,text opacity=1}]
    \addlegendimage{area legend,black,fill=black, fill opacity=1}
    \addlegendimage{area legend,color0,fill=color0, fill opacity=1}
    \addlegendimage{area legend,color1,fill=color1, fill opacity=1}
\end{customlegend}
	\end{tikzpicture}
}   
	\caption[Evaluation Results for $d^w_{s,\text{min}}$]{Evaluation Results for $d^w_{s,\text{min}}$ ($n=200$).}
	\label{fig:trialWD}
\end{figure}

\autoref{fig:trialWD} shows, that restricting the source positions by requiring a minimum wall distances only has a small effect on the localisation error. For $S=3$, 4 and 6 the median, mean and upper whisker of the \gls{mae} for $d_{s,\text{min}}^w=1.5$~m are slightly below the other two parameter values. Therefore, requiring the sources to be located towards the middle of the room achieves a slight performance improvement. However, this requirement is rather strict considering the otherwise contrained setup. $d_{s,\text{min}}^w=1.5$~m would only leave a ${3\text{~m}\times 3\text{~m}}$ area for possible source locations, which is only half the width and height of the simulated room.