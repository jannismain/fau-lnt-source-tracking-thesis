\subsubsection*{Fixed Variance}

% BOXPLOT
\begin{figure}[H]
\iftoggle{quick}{
    \includegraphics[width=\textwidth]{plots/boxplots/boxplot-joined-var-val}
}{%
    	\begin{tikzpicture}
	    % This file was created by matplotlib2tikz v0.6.14.
\definecolor{color0}{rgb}{0.8,0.207843137254902,0.219607843137255}
\definecolor{color1}{rgb}{1,0.647058823529412,0}
\definecolor{color2}{rgb}{0.0235294117647059,0.603921568627451,0.952941176470588}
\definecolor{color3}{rgb}{0.219607843137255,0.00784313725490196,0.509803921568627}
\definecolor{color4}{rgb}{0.76078431372549,0,0.470588235294118}

\begin{axis}[
xlabel={$S$},
ylabel={MAE},
xmin=0.5, xmax=6.5,
ymin=0, ymax=2.5,
width=\figurewidth,
height=\figureheight,
xtick={1,2,3,4,5,6},
xticklabels={2,3,4,5,6,7},
ytick={0,0.5,1,1.5,2,2.5},
minor xtick={},
minor ytick={},
tick align=outside,
tick pos=left,
x grid style={white!69.019607843137251!black},
ymajorgrids,
y grid style={white!69.019607843137251!black}
]
\addplot [line width=1.0pt, black, opacity=1, forget plot]
table {%
0.61 0
0.69 0
0.69 0.111803398874989
0.61 0.111803398874989
0.61 0
};
\addplot [line width=1.0pt, black, opacity=1, forget plot]
table {%
0.65 0
0.65 0
};
\addplot [line width=1.0pt, black, opacity=1, forget plot]
table {%
0.65 0.111803398874989
0.65 0.250988241891854
};
\addplot [line width=1.0pt, black, forget plot]
table {%
0.63 0
0.67 0
};
\addplot [line width=1.0pt, black, forget plot]
table {%
0.63 0.250988241891854
0.67 0.250988241891854
};
\addplot [line width=0.5pt, black, opacity=0.2, mark=*, mark size=1, mark options={solid}, only marks, forget plot]
table {%
0.65 0.604246288964795
0.65 1.46612565475441
0.65 1.37321245982865
0.65 0.536067467586918
0.65 0.460977222864644
0.65 0.440512483795333
0.65 0.440512483795333
0.65 0.33284271247462
0.65 1.0358644463914
0.65 1.68153960890869
0.65 0.641254786637844
0.65 0.490512483795333
0.65 0.502315882670322
0.65 1.82905554375099
0.65 1.38311640469197
0.65 0.570087712549569
0.65 1.78032428943307
0.65 0.56478150704935
0.65 0.33996891847538
0.65 0.452769256906871
0.65 0.746987972600482
0.65 0.364005494464026
0.65 0.76478150704935
0.65 0.36180339887499
0.65 1.32990778863778
};
\addplot [line width=1.0pt, black, opacity=1, forget plot]
table {%
1.61 0.0499999999999998
1.69 0.0499999999999998
1.69 0.471730582080989
1.61 0.471730582080989
1.61 0.0499999999999998
};
\addplot [line width=1.0pt, black, opacity=1, forget plot]
table {%
1.65 0.0499999999999998
1.65 0
};
\addplot [line width=1.0pt, black, opacity=1, forget plot]
table {%
1.65 0.471730582080989
1.65 1.10113577727726
};
\addplot [line width=1.0pt, black, forget plot]
table {%
1.63 0
1.67 0
};
\addplot [line width=1.0pt, black, forget plot]
table {%
1.63 1.10113577727726
1.67 1.10113577727726
};
\addplot [line width=0.5pt, black, opacity=0.2, mark=*, mark size=1, mark options={solid}, only marks, forget plot]
table {%
1.65 1.56204993518133
1.65 1.45344418537486
1.65 1.7060219778561
1.65 1.90607111932706
1.65 1.41598022585063
1.65 1.2
1.65 1.52073277865068
1.65 1.14142135623731
1.65 1.97003171740728
1.65 2.18002293565916
1.65 1.38949663279305
1.65 2.21020361053003
1.65 1.95428176415132
1.65 1.36244047484067
1.65 1.23490890352285
1.65 1.34629120178363
1.65 1.22065556157337
1.65 1.56159634249283
1.65 1.25438320415439
1.65 1.31244047484067
1.65 1.16018016555873
1.65 2.20227155455452
1.65 1.13372525939212
};
\addplot [line width=1.0pt, black, opacity=1, forget plot]
table {%
2.61 0.0499999999999999
2.69 0.0499999999999999
2.69 0.395594487930087
2.61 0.395594487930087
2.61 0.0499999999999999
};
\addplot [line width=1.0pt, black, opacity=1, forget plot]
table {%
2.65 0.0499999999999999
2.65 0
};
\addplot [line width=1.0pt, black, opacity=1, forget plot]
table {%
2.65 0.395594487930087
2.65 0.9
};
\addplot [line width=1.0pt, black, forget plot]
table {%
2.63 0
2.67 0
};
\addplot [line width=1.0pt, black, forget plot]
table {%
2.63 0.9
2.67 0.9
};
\addplot [line width=0.5pt, black, opacity=0.2, mark=*, mark size=1, mark options={solid}, only marks, forget plot]
table {%
2.65 1.10118117946231
2.65 1.51023252670426
2.65 1.31361911887903
2.65 1.89463256809814
2.65 1.35
2.65 1.2369316876853
2.65 1.53085978172694
2.65 2.18002293565916
2.65 1.32576506721313
2.65 1.06242283656583
2.65 1.01012538952145
2.65 1.14267057596235
2.65 1.76294149169848
2.65 1
2.65 2.00969018209801
2.65 1.88636104423578
2.65 1.36473440639562
2.65 1.48192443257046
2.65 1.10118980208143
2.65 1.17071067811866
2.65 1.21853633739162
2.65 0.940224690738243
2.65 0.95524865872714
2.65 2.57955023541887
2.65 1.05492034799999
2.65 1.50399362039845
2.65 1.01242283656583
2.65 1.03362691603005
};
\addplot [line width=1.0pt, black, opacity=1, forget plot]
table {%
3.61 0.05
3.69 0.05
3.69 0.405990911782825
3.61 0.405990911782825
3.61 0.05
};
\addplot [line width=1.0pt, black, opacity=1, forget plot]
table {%
3.65 0.05
3.65 0
};
\addplot [line width=1.0pt, black, opacity=1, forget plot]
table {%
3.65 0.405990911782825
3.65 0.832287988705046
};
\addplot [line width=1.0pt, black, forget plot]
table {%
3.63 0
3.67 0
};
\addplot [line width=1.0pt, black, forget plot]
table {%
3.63 0.832287988705046
3.67 0.832287988705046
};
\addplot [line width=0.5pt, black, opacity=0.2, mark=*, mark size=1, mark options={solid}, only marks, forget plot]
table {%
3.65 1.53397620084129
3.65 2.40488843914952
3.65 1.12722237058771
3.65 1.99137375071635
3.65 1.19230942744668
3.65 1.49197771847384
3.65 2.03062484748657
3.65 1.1295630140987
3.65 1.06925347589447
3.65 1.60332894866465
3.65 2.05060966544099
3.65 1.1629703349613
3.65 1.18357840487546
3.65 1.287941840314
3.65 1.5305227865014
3.65 1.0816653826392
3.65 1.1629703349613
3.65 1.14620911896943
};
\addplot [line width=1.0pt, black, opacity=1, forget plot]
table {%
4.61 0.0999999999999996
4.69 0.0999999999999996
4.69 0.678794445724762
4.61 0.678794445724762
4.61 0.0999999999999996
};
\addplot [line width=1.0pt, black, opacity=1, forget plot]
table {%
4.65 0.0999999999999996
4.65 0
};
\addplot [line width=1.0pt, black, opacity=1, forget plot]
table {%
4.65 0.678794445724762
4.65 1.50346754780561
};
\addplot [line width=1.0pt, black, forget plot]
table {%
4.63 0
4.67 0
};
\addplot [line width=1.0pt, black, forget plot]
table {%
4.63 1.50346754780561
4.67 1.50346754780561
};
\addplot [line width=0.5pt, black, opacity=0.2, mark=*, mark size=1, mark options={solid}, only marks, forget plot]
table {%
4.65 2.04274932307477
4.65 1.68787979456661
};
\addplot [line width=1.0pt, black, opacity=1, forget plot]
table {%
5.61 0.1
5.69 0.1
5.69 0.497076473283515
5.61 0.497076473283515
5.61 0.1
};
\addplot [line width=1.0pt, black, opacity=1, forget plot]
table {%
5.65 0.1
5.65 0
};
\addplot [line width=1.0pt, black, opacity=1, forget plot]
table {%
5.65 0.497076473283515
5.65 1.06486786442091
};
\addplot [line width=1.0pt, black, forget plot]
table {%
5.63 0
5.67 0
};
\addplot [line width=1.0pt, black, forget plot]
table {%
5.63 1.06486786442091
5.67 1.06486786442091
};
\addplot [line width=0.5pt, black, opacity=0.2, mark=*, mark size=1, mark options={solid}, only marks, forget plot]
table {%
5.65 1.73171426390382
5.65 1.5233628202243
5.65 1.99729705776437
5.65 1.52104781099153
5.65 1.32598171673552
5.65 1.20479715820391
5.65 1.09310370639165
5.65 1.21213203435596
5.65 1.17613789538997
5.65 1.41547547554639
5.65 1.19330343736593
5.65 2.32597967423951
5.65 1.20933866224478
5.65 1.25494120945927
};
\addplot [line width=1.0pt, color0, opacity=1, forget plot]
table {%
0.75 0
0.83 0
0.83 0.111803398874989
0.75 0.111803398874989
0.75 0
};
\addplot [line width=1.0pt, color0, opacity=1, forget plot]
table {%
0.79 0
0.79 0
};
\addplot [line width=1.0pt, color0, opacity=1, forget plot]
table {%
0.79 0.111803398874989
0.79 0.254950975679639
};
\addplot [line width=1.0pt, color0, forget plot]
table {%
0.77 0
0.81 0
};
\addplot [line width=1.0pt, color0, forget plot]
table {%
0.77 0.254950975679639
0.81 0.254950975679639
};
\addplot [line width=0.5pt, color0, opacity=0.2, mark=*, mark size=1, mark options={solid}, only marks, forget plot]
table {%
0.79 0.319258240356725
0.79 1.63545085566823
0.79 0.474341649025257
0.79 0.430116263352131
0.79 0.360555127546399
0.79 0.75
0.79 0.403350993617254
0.79 1.26242283656583
0.79 0.611543369438253
0.79 0.570087712549569
0.79 0.282842712474619
0.79 0.477200187265876
0.79 0.320156211871642
0.79 1.31529464379659
0.79 0.291547594742265
0.79 0.341547594742265
0.79 0.320156211871642
0.79 0.320156211871642
0.79 1.51444397018583
0.79 0.390512483795333
0.79 0.320156211871642
0.79 0.542442890089805
0.79 0.410679596594035
0.79 1.35830777072061
0.79 1.25603763723162
0.79 2.2106716141979
0.79 0.338391446781618
};
\addplot [line width=1.0pt, color0, opacity=1, forget plot]
table {%
1.75 0
1.83 0
1.83 0.170710678118655
1.75 0.170710678118655
1.75 0
};
\addplot [line width=1.0pt, color0, opacity=1, forget plot]
table {%
1.79 0
1.79 0
};
\addplot [line width=1.0pt, color0, opacity=1, forget plot]
table {%
1.79 0.170710678118655
1.79 0.410555127546399
};
\addplot [line width=1.0pt, color0, forget plot]
table {%
1.77 0
1.81 0
};
\addplot [line width=1.0pt, color0, forget plot]
table {%
1.77 0.410555127546399
1.81 0.410555127546399
};
\addplot [line width=0.5pt, color0, opacity=0.2, mark=*, mark size=1, mark options={solid}, only marks, forget plot]
table {%
1.79 1.50896783120142
1.79 0.474341649025257
1.79 0.99630129145361
1.79 2.07030384869225
1.79 1.04695987005105
1.79 0.99247166206396
1.79 1.20208152801713
1.79 1.41598022585063
1.79 1.20415945787923
1.79 0.540832691319598
1.79 1.39541497663575
1.79 0.690832691319598
1.79 0.500826941470786
1.79 1.91049731745428
1.79 2.15803349771236
1.79 1.40093312534657
1.79 0.51478150704935
1.79 1.03174688206952
1.79 2.11009478460092
1.79 0.75
1.79 1.95352771649493
1.79 2.19796977215954
};
\addplot [line width=1.0pt, color0, opacity=1, forget plot]
table {%
2.75 0
2.83 0
2.83 0.320156211871642
2.75 0.320156211871642
2.75 0
};
\addplot [line width=1.0pt, color0, opacity=1, forget plot]
table {%
2.79 0
2.79 0
};
\addplot [line width=1.0pt, color0, opacity=1, forget plot]
table {%
2.79 0.320156211871642
2.79 0.707106781186548
};
\addplot [line width=1.0pt, color0, forget plot]
table {%
2.77 0
2.81 0
};
\addplot [line width=1.0pt, color0, forget plot]
table {%
2.77 0.707106781186548
2.81 0.707106781186548
};
\addplot [line width=0.5pt, color0, opacity=0.2, mark=*, mark size=1, mark options={solid}, only marks, forget plot]
table {%
2.79 1.79120780944556
2.79 1.65538890407761
2.79 1.25896783120142
2.79 1.31959587195412
2.79 1.97353840616713
2.79 1.37382022948737
2.79 2.18138894624022
2.79 1.08816764489533
2.79 0.820060973342836
2.79 1.15108644332213
2.79 1.01176920308357
2.79 2.18849769065341
2.79 0.905862138431184
2.79 0.85146931829632
2.79 0.85
2.79 1.70725202766136
2.79 0.982398275111335
2.79 1.19268604418766
2.79 2.22036033111745
2.79 2.08086520466848
2.79 1.0434113888548
2.79 1.53120317238383
};
\addplot [line width=1.0pt, color0, opacity=1, forget plot]
table {%
3.75 0.0499999999999998
3.83 0.0499999999999998
3.83 0.336944546154293
3.75 0.336944546154293
3.75 0.0499999999999998
};
\addplot [line width=1.0pt, color0, opacity=1, forget plot]
table {%
3.79 0.0499999999999998
3.79 0
};
\addplot [line width=1.0pt, color0, opacity=1, forget plot]
table {%
3.79 0.336944546154293
3.79 0.763723639395254
};
\addplot [line width=1.0pt, color0, forget plot]
table {%
3.77 0
3.81 0
};
\addplot [line width=1.0pt, color0, forget plot]
table {%
3.77 0.763723639395254
3.81 0.763723639395254
};
\addplot [line width=0.5pt, color0, opacity=0.2, mark=*, mark size=1, mark options={solid}, only marks, forget plot]
table {%
3.79 0.97082439194738
3.79 1.18185012682662
3.79 1.86285747541548
3.79 1.3
3.79 1.59646183135365
3.79 0.951314879522022
3.79 1.12015621187164
3.79 2.38092315927882
3.79 1.91419812993321
3.79 0.813941029804985
3.79 1.62213527277196
3.79 1.25099960031968
3.79 0.85173564570932
3.79 0.95524865872714
3.79 0.874642784226795
3.79 1.20838900884805
3.79 0.781024967590666
3.79 0.85146931829632
3.79 1.19346878151419
3.79 1.25933866224478
3.79 1.38924439894498
3.79 0.944427190999916
3.79 1.46155942134931
};
\addplot [line width=1.0pt, color0, opacity=1, forget plot]
table {%
4.75 0.0499999999999998
4.83 0.0499999999999998
4.83 0.474341649025257
4.75 0.474341649025257
4.75 0.0499999999999998
};
\addplot [line width=1.0pt, color0, opacity=1, forget plot]
table {%
4.79 0.0499999999999998
4.79 0
};
\addplot [line width=1.0pt, color0, opacity=1, forget plot]
table {%
4.79 0.474341649025257
4.79 1.05559645829827
};
\addplot [line width=1.0pt, color0, forget plot]
table {%
4.77 0
4.81 0
};
\addplot [line width=1.0pt, color0, forget plot]
table {%
4.77 1.05559645829827
4.81 1.05559645829827
};
\addplot [line width=0.5pt, color0, opacity=0.2, mark=*, mark size=1, mark options={solid}, only marks, forget plot]
table {%
4.79 1.12896086182371
4.79 2.46496950516668
4.79 2.40242222084056
4.79 2.2297922231488
4.79 1.18687951492129
4.79 1.16726175299288
4.79 1.59556108247042
4.79 1.30096118312577
4.79 1.25
4.79 1.23102744344029
};
\addplot [line width=1.0pt, color0, opacity=1, forget plot]
table {%
5.75 0.05
5.83 0.05
5.83 0.447825566062639
5.75 0.447825566062639
5.75 0.05
};
\addplot [line width=1.0pt, color0, opacity=1, forget plot]
table {%
5.79 0.05
5.79 0
};
\addplot [line width=1.0pt, color0, opacity=1, forget plot]
table {%
5.79 0.447825566062639
5.79 1.00234484144908
};
\addplot [line width=1.0pt, color0, forget plot]
table {%
5.77 0
5.81 0
};
\addplot [line width=1.0pt, color0, forget plot]
table {%
5.77 1.00234484144908
5.81 1.00234484144908
};
\addplot [line width=0.5pt, color0, opacity=0.2, mark=*, mark size=1, mark options={solid}, only marks, forget plot]
table {%
5.79 1.36244047484067
5.79 1.1
5.79 1.10113577727726
5.79 1.75316617226175
5.79 1.2747548783982
5.79 1.16211610464047
5.79 2.00284584913089
5.79 1.07470475477312
5.79 1.30862523283024
5.79 1.07354552767919
5.79 1.30862523283024
5.79 1.04929332443957
5.79 1.26288984219712
5.79 1.96176920308357
};
\addplot [line width=1.0pt, color1, opacity=1, forget plot]
table {%
0.89 0
0.97 0
0.97 0.1
0.89 0.1
0.89 0
};
\addplot [line width=1.0pt, color1, opacity=1, forget plot]
table {%
0.93 0
0.93 0
};
\addplot [line width=1.0pt, color1, opacity=1, forget plot]
table {%
0.93 0.1
0.93 0.25
};
\addplot [line width=1.0pt, color1, forget plot]
table {%
0.91 0
0.95 0
};
\addplot [line width=1.0pt, color1, forget plot]
table {%
0.91 0.25
0.95 0.25
};
\addplot [line width=0.5pt, color1, opacity=0.2, mark=*, mark size=1, mark options={solid}, only marks, forget plot]
table {%
0.93 0.269917281883409
0.93 0.424264068711929
0.93 0.449535804129925
0.93 0.461106256960522
0.93 1.36579701927343
0.93 1.69222759816439
0.93 0.365028153987288
0.93 0.269258240356725
0.93 1.25399362039845
0.93 0.472358526421388
};
\addplot [line width=1.0pt, color1, opacity=1, forget plot]
table {%
1.89 0
1.97 0
1.97 0.11180339887499
1.89 0.11180339887499
1.89 0
};
\addplot [line width=1.0pt, color1, opacity=1, forget plot]
table {%
1.93 0
1.93 0
};
\addplot [line width=1.0pt, color1, opacity=1, forget plot]
table {%
1.93 0.11180339887499
1.93 0.269917281883408
};
\addplot [line width=1.0pt, color1, forget plot]
table {%
1.91 0
1.95 0
};
\addplot [line width=1.0pt, color1, forget plot]
table {%
1.91 0.269917281883408
1.95 0.269917281883408
};
\addplot [line width=0.5pt, color1, opacity=0.2, mark=*, mark size=1, mark options={solid}, only marks, forget plot]
table {%
1.93 1.28004934036344
1.93 0.570156211871642
1.93 1.96026313764871
1.93 2.33452745028706
1.93 0.292080962648189
1.93 0.3
1.93 0.366754374554629
1.93 0.403350993617255
1.93 0.320156211871642
1.93 1.41564387991552
1.93 0.697208882425738
1.93 0.88465170792364
1.93 0.35
1.93 0.358113883008419
};
\addplot [line width=1.0pt, color1, opacity=1, forget plot]
table {%
2.89 0
2.97 0
2.97 0.15
2.89 0.15
2.89 0
};
\addplot [line width=1.0pt, color1, opacity=1, forget plot]
table {%
2.93 0
2.93 0
};
\addplot [line width=1.0pt, color1, opacity=1, forget plot]
table {%
2.93 0.15
2.93 0.335410196624969
};
\addplot [line width=1.0pt, color1, forget plot]
table {%
2.91 0
2.95 0
};
\addplot [line width=1.0pt, color1, forget plot]
table {%
2.91 0.335410196624969
2.95 0.335410196624969
};
\addplot [line width=0.5pt, color1, opacity=0.2, mark=*, mark size=1, mark options={solid}, only marks, forget plot]
table {%
2.93 0.460977222864644
2.93 0.542442890089805
2.93 2.03494332412792
2.93 1.94497474683058
2.93 0.728010988928052
2.93 0.660977222864645
2.93 1.25399362039845
2.93 2.11009478460092
2.93 0.390512483795333
2.93 2.09403915913719
2.93 1.56630247314046
2.93 1.63783393541592
2.93 1.28172413572063
2.93 1.40830777072061
2.93 0.432968950979575
2.93 0.406155281280883
2.93 1.56107030473543
2.93 0.90146931829632
};
\addplot [line width=1.0pt, color1, opacity=1, forget plot]
table {%
3.89 0.0499999999999998
3.97 0.0499999999999998
3.97 0.183063511889227
3.89 0.183063511889227
3.89 0.0499999999999998
};
\addplot [line width=1.0pt, color1, opacity=1, forget plot]
table {%
3.93 0.0499999999999998
3.93 0
};
\addplot [line width=1.0pt, color1, opacity=1, forget plot]
table {%
3.93 0.183063511889227
3.93 0.370156211871642
};
\addplot [line width=1.0pt, color1, forget plot]
table {%
3.91 0
3.95 0
};
\addplot [line width=1.0pt, color1, forget plot]
table {%
3.91 0.370156211871642
3.95 0.370156211871642
};
\addplot [line width=0.5pt, color1, opacity=0.2, mark=*, mark size=1, mark options={solid}, only marks, forget plot]
table {%
3.93 1.22505031618017
3.93 2.05780929661063
3.93 0.386432845054083
3.93 1.18004237212059
3.93 2.02914876652595
3.93 0.760633520177594
3.93 1.2872631841783
3.93 0.581507290636733
3.93 0.559016994374948
3.93 0.686396103067893
3.93 0.565685424949238
3.93 0.431265805665053
3.93 1.61657664520178
3.93 1.0688779163216
3.93 1.0310422144175
3.93 1.40038918743873
3.93 0.873212459828649
3.93 0.56478150704935
3.93 1.88942097318386
3.93 1.31284461382142
3.93 0.764852927038918
3.93 1.12422623544082
};
\addplot [line width=1.0pt, color1, opacity=1, forget plot]
table {%
4.89 0.05
4.97 0.05
4.97 0.271160170117428
4.89 0.271160170117428
4.89 0.05
};
\addplot [line width=1.0pt, color1, opacity=1, forget plot]
table {%
4.93 0.05
4.93 0
};
\addplot [line width=1.0pt, color1, opacity=1, forget plot]
table {%
4.93 0.271160170117428
4.93 0.602268050859363
};
\addplot [line width=1.0pt, color1, forget plot]
table {%
4.91 0
4.95 0
};
\addplot [line width=1.0pt, color1, forget plot]
table {%
4.91 0.602268050859363
4.95 0.602268050859363
};
\addplot [line width=0.5pt, color1, opacity=0.2, mark=*, mark size=1, mark options={solid}, only marks, forget plot]
table {%
4.93 1.25104121494643
4.93 0.951387818865997
4.93 0.610555127546399
4.93 1.35369863706809
4.93 0.715955697554718
4.93 0.901417320374344
4.93 2.73413358219682
4.93 2.68631429974751
4.93 0.814852927038918
4.93 1.21105397017761
4.93 1.3652427692048
4.93 1.90918830920368
4.93 2.02169993097866
4.93 1.25830459735946
4.93 2.15866204103302
4.93 1.36236504173064
4.93 1.48492424049175
4.93 0.828934276258356
4.93 1.52366577423399
4.93 0.75
4.93 0.764852927038918
4.93 0.618465843842649
};
\addplot [line width=1.0pt, color1, opacity=1, forget plot]
table {%
5.89 0.05
5.97 0.05
5.97 0.282842712474619
5.89 0.282842712474619
5.89 0.05
};
\addplot [line width=1.0pt, color1, opacity=1, forget plot]
table {%
5.93 0.05
5.93 0
};
\addplot [line width=1.0pt, color1, opacity=1, forget plot]
table {%
5.93 0.282842712474619
5.93 0.565685424949238
};
\addplot [line width=1.0pt, color1, forget plot]
table {%
5.91 0
5.95 0
};
\addplot [line width=1.0pt, color1, forget plot]
table {%
5.91 0.565685424949238
5.95 0.565685424949238
};
\addplot [line width=0.5pt, color1, opacity=0.2, mark=*, mark size=1, mark options={solid}, only marks, forget plot]
table {%
5.93 0.936002257333468
5.93 1.10475115548645
5.93 0.882165848854662
5.93 1.09043260233424
5.93 1.65511754334894
5.93 0.680116263352131
5.93 0.827347975639733
5.93 0.85
5.93 1.03459030064771
5.93 1.36244047484067
5.93 1.02082439194738
5.93 1.42424387371566
5.93 1.28077640640442
5.93 0.743303437365925
5.93 0.760193611948034
5.93 0.782842712474619
5.93 1.14425198991093
5.93 1.78863157137883
5.93 0.760633520177595
};
\addplot [line width=1.0pt, color2, opacity=1, forget plot]
table {%
1.03 0
1.11 0
1.11 0.0999999999999996
1.03 0.0999999999999996
1.03 0
};
\addplot [line width=1.0pt, color2, opacity=1, forget plot]
table {%
1.07 0
1.07 0
};
\addplot [line width=1.0pt, color2, opacity=1, forget plot]
table {%
1.07 0.0999999999999996
1.07 0.228824561127074
};
\addplot [line width=1.0pt, color2, forget plot]
table {%
1.05 0
1.09 0
};
\addplot [line width=1.0pt, color2, forget plot]
table {%
1.05 0.228824561127074
1.09 0.228824561127074
};
\addplot [line width=0.5pt, color2, opacity=0.2, mark=*, mark size=1, mark options={solid}, only marks, forget plot]
table {%
1.07 1.16726175299288
1.07 0.403350993617254
1.07 0.292080962648189
1.07 0.250988241891854
1.07 0.292080962648189
1.07 0.36180339887499
1.07 0.292080962648189
1.07 0.323935433230954
1.07 0.353553390593274
1.07 0.347576637518192
};
\addplot [line width=1.0pt, color2, opacity=1, forget plot]
table {%
2.03 0.0499999999999998
2.11 0.0499999999999998
2.11 0.120710678118655
2.03 0.120710678118655
2.03 0.0499999999999998
};
\addplot [line width=1.0pt, color2, opacity=1, forget plot]
table {%
2.07 0.0499999999999998
2.07 0
};
\addplot [line width=1.0pt, color2, opacity=1, forget plot]
table {%
2.07 0.120710678118655
2.07 0.212132034355965
};
\addplot [line width=1.0pt, color2, forget plot]
table {%
2.05 0
2.09 0
};
\addplot [line width=1.0pt, color2, forget plot]
table {%
2.05 0.212132034355965
2.09 0.212132034355965
};
\addplot [line width=0.5pt, color2, opacity=0.2, mark=*, mark size=1, mark options={solid}, only marks, forget plot]
table {%
2.07 1.45773797371132
2.07 0.2302775637732
2.07 0.2302775637732
2.07 0.410679596594035
2.07 0.254950975679639
2.07 0.884590300647707
2.07 0.228824561127074
2.07 1.43293166859393
2.07 0.608113883008419
2.07 0.2302775637732
2.07 0.230277563773199
2.07 0.269917281883409
2.07 0.320156211871642
2.07 1.05622577482986
2.07 1.90402446592483
2.07 0.270710678118655
};
\addplot [line width=1.0pt, color2, opacity=1, forget plot]
table {%
3.03 0.0499999999999998
3.11 0.0499999999999998
3.11 0.14142135623731
3.03 0.14142135623731
3.03 0.0499999999999998
};
\addplot [line width=1.0pt, color2, opacity=1, forget plot]
table {%
3.07 0.0499999999999998
3.07 0
};
\addplot [line width=1.0pt, color2, opacity=1, forget plot]
table {%
3.07 0.14142135623731
3.07 0.253224755112299
};
\addplot [line width=1.0pt, color2, forget plot]
table {%
3.05 0
3.09 0
};
\addplot [line width=1.0pt, color2, forget plot]
table {%
3.05 0.253224755112299
3.09 0.253224755112299
};
\addplot [line width=0.5pt, color2, opacity=0.2, mark=*, mark size=1, mark options={solid}, only marks, forget plot]
table {%
3.07 0.570087712549569
3.07 1.01410879132432
3.07 0.35
3.07 0.652079728939615
3.07 0.320156211871642
3.07 0.316227766016838
3.07 0.542409734721485
3.07 0.33996891847538
3.07 0.366754374554629
3.07 0.320156211871643
3.07 0.782357292712814
3.07 1.65604951616792
3.07 0.461223161913988
};
\addplot [line width=1.0pt, color2, opacity=1, forget plot]
table {%
4.03 0.0499999999999999
4.11 0.0499999999999999
4.11 0.161803398874989
4.03 0.161803398874989
4.03 0.0499999999999999
};
\addplot [line width=1.0pt, color2, opacity=1, forget plot]
table {%
4.07 0.0499999999999999
4.07 0
};
\addplot [line width=1.0pt, color2, opacity=1, forget plot]
table {%
4.07 0.161803398874989
4.07 0.316227766016838
};
\addplot [line width=1.0pt, color2, forget plot]
table {%
4.05 0
4.09 0
};
\addplot [line width=1.0pt, color2, forget plot]
table {%
4.05 0.316227766016838
4.09 0.316227766016838
};
\addplot [line width=0.5pt, color2, opacity=0.2, mark=*, mark size=1, mark options={solid}, only marks, forget plot]
table {%
4.07 0.356155281280883
4.07 0.381720680758398
4.07 2.47086699069569
4.07 1.23848643240047
4.07 1.09043260233424
4.07 2.2410934831015
4.07 0.341547594742265
4.07 1.3865424623862
4.07 1.32071067811865
4.07 1.18137084989848
4.07 0.39142135623731
4.07 0.353553390593273
4.07 0.901387818865997
4.07 0.332842712474619
4.07 0.961769203083567
4.07 1.35777472107018
};
\addplot [line width=1.0pt, color2, opacity=1, forget plot]
table {%
5.03 0.05
5.11 0.05
5.11 0.182514076993644
5.03 0.182514076993644
5.03 0.05
};
\addplot [line width=1.0pt, color2, opacity=1, forget plot]
table {%
5.07 0.05
5.07 0
};
\addplot [line width=1.0pt, color2, opacity=1, forget plot]
table {%
5.07 0.182514076993644
5.07 0.361803398874989
};
\addplot [line width=1.0pt, color2, forget plot]
table {%
5.05 0
5.09 0
};
\addplot [line width=1.0pt, color2, forget plot]
table {%
5.05 0.361803398874989
5.09 0.361803398874989
};
\addplot [line width=0.5pt, color2, opacity=0.2, mark=*, mark size=1, mark options={solid}, only marks, forget plot]
table {%
5.07 1.26757956618859
5.07 1.25
5.07 0.884724793603235
5.07 0.403553390593273
5.07 0.502769256906871
5.07 0.626584905924339
5.07 0.500433775644842
5.07 0.612310562561766
5.07 0.435738832105943
5.07 1.20901699437495
5.07 0.471825158515465
5.07 0.45
5.07 0.572687161902363
5.07 1.17046999107196
};
\addplot [line width=1.0pt, color2, opacity=1, forget plot]
table {%
6.03 0.0707106781186549
6.11 0.0707106781186549
6.11 0.215000725204468
6.03 0.215000725204468
6.03 0.0707106781186549
};
\addplot [line width=1.0pt, color2, opacity=1, forget plot]
table {%
6.07 0.0707106781186549
6.07 0
};
\addplot [line width=1.0pt, color2, opacity=1, forget plot]
table {%
6.07 0.215000725204468
6.07 0.408113883008419
};
\addplot [line width=1.0pt, color2, forget plot]
table {%
6.05 0
6.09 0
};
\addplot [line width=1.0pt, color2, forget plot]
table {%
6.05 0.408113883008419
6.09 0.408113883008419
};
\addplot [line width=0.5pt, color2, opacity=0.2, mark=*, mark size=1, mark options={solid}, only marks, forget plot]
table {%
6.07 0.500433775644842
6.07 0.472358526421389
6.07 1.03077640640441
6.07 1.01980390271856
6.07 1.46272900748562
6.07 0.492592054168185
6.07 1.30475115548645
6.07 1.55759759087282
6.07 0.770820393249937
6.07 0.510555127546399
6.07 0.636396103067893
6.07 0.814852927038918
6.07 1.14426802917422
6.07 0.45149933341185
6.07 1.54609151776881
6.07 1.1
6.07 0.477200187265876
6.07 0.73309518948453
6.07 0.480116263352131
6.07 0.680073525436772
};
\addplot [line width=1.0pt, color3, opacity=1, forget plot]
table {%
1.17 0
1.25 0
1.25 0.120710678118655
1.17 0.120710678118655
1.17 0
};
\addplot [line width=1.0pt, color3, opacity=1, forget plot]
table {%
1.21 0
1.21 0
};
\addplot [line width=1.0pt, color3, opacity=1, forget plot]
table {%
1.21 0.120710678118655
1.21 0.3
};
\addplot [line width=1.0pt, color3, forget plot]
table {%
1.19 0
1.23 0
};
\addplot [line width=1.0pt, color3, forget plot]
table {%
1.19 0.3
1.23 0.3
};
\addplot [line width=0.5pt, color3, opacity=0.2, mark=*, mark size=1, mark options={solid}, only marks, forget plot]
table {%
1.21 0.403884361523178
1.21 0.515154392492244
1.21 1.40987889744971
1.21 0.304138126514911
1.21 0.540832691319598
1.21 0.427200187265876
1.21 0.317958680155873
1.21 0.488997993755502
1.21 0.449535804129925
};
\addplot [line width=1.0pt, color3, opacity=1, forget plot]
table {%
2.17 0.0499999999999998
2.25 0.0499999999999998
2.25 0.15
2.17 0.15
2.17 0.0499999999999998
};
\addplot [line width=1.0pt, color3, opacity=1, forget plot]
table {%
2.21 0.0499999999999998
2.21 0
};
\addplot [line width=1.0pt, color3, opacity=1, forget plot]
table {%
2.21 0.15
2.21 0.282842712474619
};
\addplot [line width=1.0pt, color3, forget plot]
table {%
2.19 0
2.23 0
};
\addplot [line width=1.0pt, color3, forget plot]
table {%
2.19 0.282842712474619
2.23 0.282842712474619
};
\addplot [line width=0.5pt, color3, opacity=0.2, mark=*, mark size=1, mark options={solid}, only marks, forget plot]
table {%
2.21 1.19772492000501
2.21 1.52656876555683
2.21 0.341547594742265
2.21 0.410555127546399
2.21 0.403112887414927
2.21 0.30495097567964
2.21 1.87416648139913
2.21 0.934590300647707
2.21 0.381720680758398
2.21 0.304138126514911
2.21 1.35092560861063
};
\addplot [line width=1.0pt, color3, opacity=1, forget plot]
table {%
3.17 0.05
3.25 0.05
3.25 0.2
3.17 0.2
3.17 0.05
};
\addplot [line width=1.0pt, color3, opacity=1, forget plot]
table {%
3.21 0.05
3.21 0
};
\addplot [line width=1.0pt, color3, opacity=1, forget plot]
table {%
3.21 0.2
3.21 0.410555127546399
};
\addplot [line width=1.0pt, color3, forget plot]
table {%
3.19 0
3.23 0
};
\addplot [line width=1.0pt, color3, forget plot]
table {%
3.19 0.410555127546399
3.23 0.410555127546399
};
\addplot [line width=0.5pt, color3, opacity=0.2, mark=*, mark size=1, mark options={solid}, only marks, forget plot]
table {%
3.21 0.52169905660283
3.21 0.962414379544733
3.21 2.44470080598439
3.21 1.1600313217485
3.21 0.56180339887499
3.21 0.432968950979574
3.21 1.56524758424985
3.21 1.21023252670426
3.21 0.652636090194588
3.21 1.40801278403287
3.21 0.680073525436772
3.21 1.82662725123941
3.21 0.70178344238091
3.21 0.473606797749979
3.21 0.632455532033676
3.21 0.859897325389329
3.21 0.70316621015233
3.21 1.62184199983886
};
\addplot [line width=1.0pt, color3, opacity=1, forget plot]
table {%
4.17 0.0707106781186548
4.25 0.0707106781186548
4.25 0.223606797749979
4.17 0.223606797749979
4.17 0.0707106781186548
};
\addplot [line width=1.0pt, color3, opacity=1, forget plot]
table {%
4.21 0.0707106781186548
4.21 0
};
\addplot [line width=1.0pt, color3, opacity=1, forget plot]
table {%
4.21 0.223606797749979
4.21 0.381720680758398
};
\addplot [line width=1.0pt, color3, forget plot]
table {%
4.19 0
4.23 0
};
\addplot [line width=1.0pt, color3, forget plot]
table {%
4.19 0.381720680758398
4.23 0.381720680758398
};
\addplot [line width=0.5pt, color3, opacity=0.2, mark=*, mark size=1, mark options={solid}, only marks, forget plot]
table {%
4.21 1.29807549857472
4.21 1.35092560861063
4.21 0.502315882670322
4.21 0.75
4.21 1.16726175299288
4.21 0.631507290636732
4.21 0.604246288964795
4.21 0.75
4.21 1.69056873925968
4.21 0.474341649025257
4.21 1.17524677983738
4.21 1.97191092264882
4.21 1.58829034780607
4.21 0.477200187265876
4.21 0.680073525436772
4.21 0.559016994374947
4.21 0.749697609267132
4.21 1.17361025271221
4.21 0.641547594742265
4.21 1.36031993589379
4.21 0.7673698073716
4.21 1.20104121494643
4.21 0.453112887414927
};
\addplot [line width=1.0pt, color3, opacity=1, forget plot]
table {%
5.17 0.0999999999999996
5.25 0.0999999999999996
5.25 0.25756731067941
5.17 0.25756731067941
5.17 0.0999999999999996
};
\addplot [line width=1.0pt, color3, opacity=1, forget plot]
table {%
5.21 0.0999999999999996
5.21 0
};
\addplot [line width=1.0pt, color3, opacity=1, forget plot]
table {%
5.21 0.25756731067941
5.21 0.492442890089805
};
\addplot [line width=1.0pt, color3, forget plot]
table {%
5.19 0
5.23 0
};
\addplot [line width=1.0pt, color3, forget plot]
table {%
5.19 0.492442890089805
5.23 0.492442890089805
};
\addplot [line width=0.5pt, color3, opacity=0.2, mark=*, mark size=1, mark options={solid}, only marks, forget plot]
table {%
5.21 0.665685424949238
5.21 0.502769256906871
5.21 0.912414379544733
5.21 1.63492870508414
5.21 0.635234995535981
5.21 0.702292889218727
5.21 1.05118980208143
5.21 0.778886602081306
5.21 1.36579701927343
5.21 0.656155281280883
5.21 0.514916286289917
5.21 0.527200187265877
5.21 1.81478150704935
5.21 0.572780621739635
5.21 0.58309518948453
5.21 0.506449510224598
5.21 0.559016994374947
5.21 0.840569415042095
};
\addplot [line width=1.0pt, color3, opacity=1, forget plot]
table {%
6.17 0.1
6.25 0.1
6.25 0.30910613099963
6.17 0.30910613099963
6.17 0.1
};
\addplot [line width=1.0pt, color3, opacity=1, forget plot]
table {%
6.21 0.1
6.21 0
};
\addplot [line width=1.0pt, color3, opacity=1, forget plot]
table {%
6.21 0.30910613099963
6.21 0.604152298679729
};
\addplot [line width=1.0pt, color3, forget plot]
table {%
6.19 0
6.23 0
};
\addplot [line width=1.0pt, color3, forget plot]
table {%
6.19 0.604152298679729
6.23 0.604152298679729
};
\addplot [line width=0.5pt, color3, opacity=0.2, mark=*, mark size=1, mark options={solid}, only marks, forget plot]
table {%
6.21 1.40099960031968
6.21 0.818910180061537
6.21 0.7
6.21 1.31735024386788
6.21 1.21018016555873
6.21 1.13766429849587
6.21 0.874004936172236
6.21 0.777817459305202
6.21 0.652268050859363
6.21 0.658276253029822
6.21 0.840569415042095
6.21 0.676498204307083
6.21 1.55479288608424
};
\addplot [line width=1.0pt, color4, opacity=1, forget plot]
table {%
1.31 0.0500000000000003
1.39 0.0500000000000003
1.39 0.2352081728299
1.31 0.2352081728299
1.31 0.0500000000000003
};
\addplot [line width=1.0pt, color4, opacity=1, forget plot]
table {%
1.35 0.0500000000000003
1.35 0
};
\addplot [line width=1.0pt, color4, opacity=1, forget plot]
table {%
1.35 0.2352081728299
1.35 0.506449510224598
};
\addplot [line width=1.0pt, color4, forget plot]
table {%
1.33 0
1.37 0
};
\addplot [line width=1.0pt, color4, forget plot]
table {%
1.33 0.506449510224598
1.37 0.506449510224598
};
\addplot [line width=0.5pt, color4, opacity=0.2, mark=*, mark size=1, mark options={solid}, only marks, forget plot]
table {%
1.35 1.37568164921976
1.35 0.912414379544733
1.35 0.905862138431184
1.35 0.901308556317268
1.35 0.782623792124926
1.35 0.961769203083567
1.35 1.1500623353648
1.35 0.707106781186548
1.35 0.920087712549569
1.35 0.667083203206317
1.35 0.58309518948453
1.35 1.02195444572929
1.35 1.40099960031968
1.35 0.651920240520265
1.35 0.651920240520265
1.35 0.643310689511722
1.35 1.2747548783982
};
\addplot [line width=1.0pt, color4, opacity=1, forget plot]
table {%
2.31 0.1
2.39 0.1
2.39 0.45756731067941
2.31 0.45756731067941
2.31 0.1
};
\addplot [line width=1.0pt, color4, opacity=1, forget plot]
table {%
2.35 0.1
2.35 0
};
\addplot [line width=1.0pt, color4, opacity=1, forget plot]
table {%
2.35 0.45756731067941
2.35 0.939450666737332
};
\addplot [line width=1.0pt, color4, forget plot]
table {%
2.33 0
2.37 0
};
\addplot [line width=1.0pt, color4, forget plot]
table {%
2.33 0.939450666737332
2.37 0.939450666737332
};
\addplot [line width=0.5pt, color4, opacity=0.2, mark=*, mark size=1, mark options={solid}, only marks, forget plot]
table {%
2.35 1.25399362039845
2.35 1.10453610171873
2.35 1.76294149169848
2.35 1.93215302247187
2.35 1.90918830920368
2.35 1.8769498649375
2.35 1.54623943478093
2.35 1.95788961227228
2.35 1.38655827727319
2.35 1.16619037896906
2.35 1.10475115548645
2.35 1.28004934036344
2.35 1.57876534038469
2.35 1.62950777996463
2.35 1.24197423483742
2.35 1.11301458127347
2.35 1.57003184681076
2.35 1.7805227865014
2.35 1.07935165724615
2.35 1.42163628672928
};
\addplot [line width=1.0pt, color4, opacity=1, forget plot]
table {%
3.31 0.120710678118655
3.39 0.120710678118655
3.39 0.469160032367445
3.31 0.469160032367445
3.31 0.120710678118655
};
\addplot [line width=1.0pt, color4, opacity=1, forget plot]
table {%
3.35 0.120710678118655
3.35 0
};
\addplot [line width=1.0pt, color4, opacity=1, forget plot]
table {%
3.35 0.469160032367445
3.35 0.946419413859206
};
\addplot [line width=1.0pt, color4, forget plot]
table {%
3.33 0
3.37 0
};
\addplot [line width=1.0pt, color4, forget plot]
table {%
3.33 0.946419413859206
3.37 0.946419413859206
};
\addplot [line width=0.5pt, color4, opacity=0.2, mark=*, mark size=1, mark options={solid}, only marks, forget plot]
table {%
3.35 1.01046863561493
3.35 1.0105551275464
3.35 1.64624294225856
3.35 1.17046999107196
3.35 1.13137084989848
3.35 1.16272716151073
3.35 1.31470656399984
3.35 1.47302494707577
3.35 1.71714027288886
3.35 1.0246427842268
3.35 1.07935165724615
3.35 1.25830459735946
3.35 2.38269980717826
3.35 1.36014705087354
3.35 1.40732024635453
3.35 1.10124921972504
3.35 1.35830777072061
3.35 1.79073574112941
3.35 1.12004672795164
3.35 1
3.35 1.12071067811865
3.35 1.20948100502085
};
\addplot [line width=1.0pt, color4, opacity=1, forget plot]
table {%
4.31 0.15
4.39 0.15
4.39 0.527102337974296
4.31 0.527102337974296
4.31 0.15
};
\addplot [line width=1.0pt, color4, opacity=1, forget plot]
table {%
4.35 0.15
4.35 0
};
\addplot [line width=1.0pt, color4, opacity=1, forget plot]
table {%
4.35 0.527102337974296
4.35 1.0424327243156
};
\addplot [line width=1.0pt, color4, forget plot]
table {%
4.33 0
4.37 0
};
\addplot [line width=1.0pt, color4, forget plot]
table {%
4.33 1.0424327243156
4.37 1.0424327243156
};
\addplot [line width=0.5pt, color4, opacity=0.2, mark=*, mark size=1, mark options={solid}, only marks, forget plot]
table {%
4.35 1.86698281927997
4.35 1.74220241559552
4.35 1.25896783120142
4.35 1.27906515186786
4.35 1.23004237212059
4.35 1.48475642064959
4.35 1.25099960031968
4.35 1.12004672795163
4.35 1.46550203594308
4.35 1.17098437033601
};
\addplot [line width=1.0pt, color4, opacity=1, forget plot]
table {%
5.31 0.16180339887499
5.39 0.16180339887499
5.39 0.526007446323391
5.31 0.526007446323391
5.31 0.16180339887499
};
\addplot [line width=1.0pt, color4, opacity=1, forget plot]
table {%
5.35 0.16180339887499
5.35 0
};
\addplot [line width=1.0pt, color4, opacity=1, forget plot]
table {%
5.35 0.526007446323391
5.35 1.02660969988462
};
\addplot [line width=1.0pt, color4, forget plot]
table {%
5.33 0
5.37 0
};
\addplot [line width=1.0pt, color4, forget plot]
table {%
5.33 1.02660969988462
5.37 1.02660969988462
};
\addplot [line width=0.5pt, color4, opacity=0.2, mark=*, mark size=1, mark options={solid}, only marks, forget plot]
table {%
5.35 1.5
5.35 1.30388781648541
5.35 1.16655455436144
5.35 1.33772084733507
5.35 1.12367567697886
5.35 1.44023025310385
5.35 1.25679718105893
5.35 1.17246357065481
5.35 1.55081978839096
5.35 1.13381592116126
5.35 1.42050701097049
5.35 1.80623918681884
};
\addplot [line width=1.0pt, color4, opacity=1, forget plot]
table {%
6.31 0.168483858307738
6.39 0.168483858307738
6.39 0.464271974889235
6.31 0.464271974889235
6.31 0.168483858307738
};
\addplot [line width=1.0pt, color4, opacity=1, forget plot]
table {%
6.35 0.168483858307738
6.35 0
};
\addplot [line width=1.0pt, color4, opacity=1, forget plot]
table {%
6.35 0.464271974889235
6.35 0.9029986668237
};
\addplot [line width=1.0pt, color4, forget plot]
table {%
6.33 0
6.37 0
};
\addplot [line width=1.0pt, color4, forget plot]
table {%
6.33 0.9029986668237
6.37 0.9029986668237
};
\addplot [line width=0.5pt, color4, opacity=0.2, mark=*, mark size=1, mark options={solid}, only marks, forget plot]
table {%
6.35 1.24798058159319
6.35 1.1
6.35 1.84714171354369
6.35 0.920060973342836
6.35 1.7465172611105
6.35 0.982735008131951
6.35 1.41473440639562
6.35 1.49975466950829
6.35 1.14702285137894
6.35 1.18402810944258
6.35 1.00663729752108
6.35 1.16066017177982
6.35 1.1483936148715
6.35 1.34350159560362
6.35 1.04817559006576
};
\addplot [line width=1.0pt, black, opacity=1, forget plot]
table {%
0.61 0
0.69 0
};
\addplot [line width=1.0pt, black, dashed, mark=x, mark size=3, mark options={solid}, forget plot]
table {%
0.65 0.136813230999457
};
\addplot [line width=1.0pt, black, opacity=1, forget plot]
table {%
1.61 0.11180339887499
1.69 0.11180339887499
};
\addplot [line width=1.0pt, black, dashed, mark=x, mark size=3, mark options={solid}, forget plot]
table {%
1.65 0.350632436084626
};
\addplot [line width=1.0pt, black, opacity=1, forget plot]
table {%
2.61 0.111803398874989
2.69 0.111803398874989
};
\addplot [line width=1.0pt, black, dashed, mark=x, mark size=3, mark options={solid}, forget plot]
table {%
2.65 0.338664372853771
};
\addplot [line width=1.0pt, black, opacity=1, forget plot]
table {%
3.61 0.145710678118655
3.69 0.145710678118655
};
\addplot [line width=1.0pt, black, dashed, mark=x, mark size=3, mark options={solid}, forget plot]
table {%
3.65 0.313051881331764
};
\addplot [line width=1.0pt, black, opacity=1, forget plot]
table {%
4.61 0.271432519053352
4.69 0.271432519053352
};
\addplot [line width=1.0pt, black, dashed, mark=x, mark size=3, mark options={solid}, forget plot]
table {%
4.65 0.418207915675873
};
\addplot [line width=1.0pt, black, opacity=1, forget plot]
table {%
5.61 0.217869416052971
5.69 0.217869416052971
};
\addplot [line width=1.0pt, black, dashed, mark=x, mark size=3, mark options={solid}, forget plot]
table {%
5.65 0.37014834297979
};
\addplot [line width=1.0pt, color0, opacity=1, forget plot]
table {%
0.75 0.0499999999999998
0.83 0.0499999999999998
};
\addplot [line width=1.0pt, color0, dashed, mark=x, mark size=3, mark options={solid}, forget plot]
table {%
0.79 0.131709192092931
};
\addplot [line width=1.0pt, color0, opacity=1, forget plot]
table {%
1.75 0.0499999999999999
1.83 0.0499999999999999
};
\addplot [line width=1.0pt, color0, dashed, mark=x, mark size=3, mark options={solid}, forget plot]
table {%
1.79 0.205549296454691
};
\addplot [line width=1.0pt, color0, opacity=1, forget plot]
table {%
2.75 0.0707106781186551
2.83 0.0707106781186551
};
\addplot [line width=1.0pt, color0, dashed, mark=x, mark size=3, mark options={solid}, forget plot]
table {%
2.79 0.278650694841483
};
\addplot [line width=1.0pt, color0, opacity=1, forget plot]
table {%
3.75 0.111803398874989
3.83 0.111803398874989
};
\addplot [line width=1.0pt, color0, dashed, mark=x, mark size=3, mark options={solid}, forget plot]
table {%
3.79 0.278015485851331
};
\addplot [line width=1.0pt, color0, opacity=1, forget plot]
table {%
4.75 0.116257038496822
4.83 0.116257038496822
};
\addplot [line width=1.0pt, color0, dashed, mark=x, mark size=3, mark options={solid}, forget plot]
table {%
4.79 0.313595053074284
};
\addplot [line width=1.0pt, color0, opacity=1, forget plot]
table {%
5.75 0.158113883008419
5.83 0.158113883008419
};
\addplot [line width=1.0pt, color0, dashed, mark=x, mark size=3, mark options={solid}, forget plot]
table {%
5.79 0.302914087942178
};
\addplot [line width=1.0pt, color1, opacity=1, forget plot]
table {%
0.89 0.0499999999999998
0.97 0.0499999999999998
};
\addplot [line width=1.0pt, color1, dashed, mark=x, mark size=3, mark options={solid}, forget plot]
table {%
0.93 0.0818713762112009
};
\addplot [line width=1.0pt, color1, opacity=1, forget plot]
table {%
1.89 0.05
1.97 0.05
};
\addplot [line width=1.0pt, color1, dashed, mark=x, mark size=3, mark options={solid}, forget plot]
table {%
1.93 0.109323700175608
};
\addplot [line width=1.0pt, color1, opacity=1, forget plot]
table {%
2.89 0.0707106781186546
2.97 0.0707106781186546
};
\addplot [line width=1.0pt, color1, dashed, mark=x, mark size=3, mark options={solid}, forget plot]
table {%
2.93 0.17581757048337
};
\addplot [line width=1.0pt, color1, opacity=1, forget plot]
table {%
3.89 0.0999999999999998
3.97 0.0999999999999998
};
\addplot [line width=1.0pt, color1, dashed, mark=x, mark size=3, mark options={solid}, forget plot]
table {%
3.93 0.20140683292156
};
\addplot [line width=1.0pt, color1, opacity=1, forget plot]
table {%
4.89 0.111803398874989
4.97 0.111803398874989
};
\addplot [line width=1.0pt, color1, dashed, mark=x, mark size=3, mark options={solid}, forget plot]
table {%
4.93 0.265870583476451
};
\addplot [line width=1.0pt, color1, opacity=1, forget plot]
table {%
5.89 0.120710678118655
5.97 0.120710678118655
};
\addplot [line width=1.0pt, color1, dashed, mark=x, mark size=3, mark options={solid}, forget plot]
table {%
5.93 0.235293111960185
};
\addplot [line width=1.0pt, color2, opacity=1, forget plot]
table {%
1.03 0.05
1.11 0.05
};
\addplot [line width=1.0pt, color2, dashed, mark=x, mark size=3, mark options={solid}, forget plot]
table {%
1.07 0.0664085809561382
};
\addplot [line width=1.0pt, color2, opacity=1, forget plot]
table {%
2.03 0.0500000000000003
2.11 0.0500000000000003
};
\addplot [line width=1.0pt, color2, dashed, mark=x, mark size=3, mark options={solid}, forget plot]
table {%
2.07 0.114764804758535
};
\addplot [line width=1.0pt, color2, opacity=1, forget plot]
table {%
3.03 0.0707106781186548
3.11 0.0707106781186548
};
\addplot [line width=1.0pt, color2, dashed, mark=x, mark size=3, mark options={solid}, forget plot]
table {%
3.07 0.115048474030805
};
\addplot [line width=1.0pt, color2, opacity=1, forget plot]
table {%
4.03 0.111803398874989
4.11 0.111803398874989
};
\addplot [line width=1.0pt, color2, dashed, mark=x, mark size=3, mark options={solid}, forget plot]
table {%
4.07 0.17506385950821
};
\addplot [line width=1.0pt, color2, opacity=1, forget plot]
table {%
5.03 0.11180339887499
5.11 0.11180339887499
};
\addplot [line width=1.0pt, color2, dashed, mark=x, mark size=3, mark options={solid}, forget plot]
table {%
5.07 0.163395086331725
};
\addplot [line width=1.0pt, color2, opacity=1, forget plot]
table {%
6.03 0.131066017177982
6.11 0.131066017177982
};
\addplot [line width=1.0pt, color2, dashed, mark=x, mark size=3, mark options={solid}, forget plot]
table {%
6.07 0.211091156105211
};
\addplot [line width=1.0pt, color3, opacity=1, forget plot]
table {%
1.17 0.0500000000000003
1.25 0.0500000000000003
};
\addplot [line width=1.0pt, color3, dashed, mark=x, mark size=3, mark options={solid}, forget plot]
table {%
1.21 0.0966097059919734
};
\addplot [line width=1.0pt, color3, opacity=1, forget plot]
table {%
2.17 0.070710678118655
2.25 0.070710678118655
};
\addplot [line width=1.0pt, color3, dashed, mark=x, mark size=3, mark options={solid}, forget plot]
table {%
2.21 0.126025287993838
};
\addplot [line width=1.0pt, color3, opacity=1, forget plot]
table {%
3.17 0.120710678118655
3.25 0.120710678118655
};
\addplot [line width=1.0pt, color3, dashed, mark=x, mark size=3, mark options={solid}, forget plot]
table {%
3.21 0.204486023081343
};
\addplot [line width=1.0pt, color3, opacity=1, forget plot]
table {%
4.17 0.14142135623731
4.25 0.14142135623731
};
\addplot [line width=1.0pt, color3, dashed, mark=x, mark size=3, mark options={solid}, forget plot]
table {%
4.21 0.230412166226
};
\addplot [line width=1.0pt, color3, opacity=1, forget plot]
table {%
5.17 0.159958640941704
5.25 0.159958640941704
};
\addplot [line width=1.0pt, color3, dashed, mark=x, mark size=3, mark options={solid}, forget plot]
table {%
5.21 0.220722569444487
};
\addplot [line width=1.0pt, color3, opacity=1, forget plot]
table {%
6.17 0.16180339887499
6.25 0.16180339887499
};
\addplot [line width=1.0pt, color3, dashed, mark=x, mark size=3, mark options={solid}, forget plot]
table {%
6.21 0.245882268441939
};
\addplot [line width=1.0pt, color4, opacity=1, forget plot]
table {%
1.31 0.120710678118654
1.39 0.120710678118654
};
\addplot [line width=1.0pt, color4, dashed, mark=x, mark size=3, mark options={solid}, forget plot]
table {%
1.35 0.201204525895123
};
\addplot [line width=1.0pt, color4, opacity=1, forget plot]
table {%
2.31 0.186967716615477
2.39 0.186967716615477
};
\addplot [line width=1.0pt, color4, dashed, mark=x, mark size=3, mark options={solid}, forget plot]
table {%
2.35 0.367751347007149
};
\addplot [line width=1.0pt, color4, opacity=1, forget plot]
table {%
3.31 0.211967716615477
3.39 0.211967716615477
};
\addplot [line width=1.0pt, color4, dashed, mark=x, mark size=3, mark options={solid}, forget plot]
table {%
3.35 0.374748812542264
};
\addplot [line width=1.0pt, color4, opacity=1, forget plot]
table {%
4.31 0.284473461023863
4.39 0.284473461023863
};
\addplot [line width=1.0pt, color4, dashed, mark=x, mark size=3, mark options={solid}, forget plot]
table {%
4.35 0.39180432999565
};
\addplot [line width=1.0pt, color4, opacity=1, forget plot]
table {%
5.31 0.259958640941704
5.39 0.259958640941704
};
\addplot [line width=1.0pt, color4, dashed, mark=x, mark size=3, mark options={solid}, forget plot]
table {%
5.35 0.397686205770339
};
\addplot [line width=1.0pt, color4, opacity=1, forget plot]
table {%
6.31 0.278571761586369
6.39 0.278571761586369
};
\addplot [line width=1.0pt, color4, dashed, mark=x, mark size=3, mark options={solid}, forget plot]
table {%
6.35 0.378112955641413
};
\end{axis}

\node at ({$(current bounding box.south west)!0.5!(current bounding box.south east)$}|-{$(current bounding box.south west)!0.98!(current bounding box.north west)$})[
  anchor=north,
  text=black,
  rotate=0.0
]{ };

	    \begin{customlegend}[
legend entries={$\sigma^2_{\text{fixed}}=0.1$,$\sigma^2_{\text{fixed}}=0.5$,$\sigma^2_{\text{fixed}}=1.0$,$\sigma^2_{\text{fixed}}=2.0$,$\sigma^2_{\text{fixed}}=3.0$,$\sigma^2_{\text{fixed}}=5.0$},
legend cell align=left,
legend style={at={(0.05,5.37)}, anchor=north west, draw=white!80.0!black, font=\footnotesize,fill opacity=0.5, draw opacity=1,text opacity=1}]
% the following are the "images" and numbers in the legend
    \addlegendimage{area legend,black,fill=black, fill opacity=1}
    \addlegendimage{area legend,color0,fill=color0, fill opacity=1}
    \addlegendimage{area legend,color1,fill=color1, fill opacity=1}
    \addlegendimage{area legend,color2,fill=color2, fill opacity=1}
    \addlegendimage{area legend,color3,fill=color3, fill opacity=1}
    \addlegendimage{area legend,color4,fill=color4, fill opacity=1}
\end{customlegend}
	\end{tikzpicture}
}
	\caption[Evaluation Results for $\sigma^2_{\text{fixed}}$]{Evaluation Results for $\sigma^2_{\text{fixed}}$ ($L=10$, $n=200$).}
	\label{fig:trialVarianceFixed}
\end{figure}

The results in \autoref{fig:trialVarianceFixed} show, that the extreme values $\sigma^2_{\text{fixed}}=0.1$ and $\sigma^2_{\text{fixed}}=5.0$ yield significantly higher localisation errors across all $S$ compared to the other initial variances. Overall, $\sigma^2_{\text{fixed}}=2$ performs best for all but $S=7$, where $\sigma^2_{\text{fixed}}=1$ results in a slightly lower mean and median \gls{mae}. Comparing these results with the evaluation above, where the variance has not been fixed, does reveal that there is no performance advantage in fixing the variance. However, estimating the variance uses about 30\% of the computation time per iteration. Therefore, not estimating the variance reduces the runtime of the \gls{em} algorithm by 30\%, which amounts to about $1$ second per iteration (on the hardware described in \autoref{sec:computationalComplexity}). This improvement, however, comes with the challenge of choosing a good initial value for $\sigma^2_{\text{fixed}}$. As we have seen in \autoref{fig:trialVarianceFixed}, a bad fixed variance can lead to a significant worse localisation performance, at least for $S>2$. Therefore, according to these results, estimating the variance is the better option when trying to optimise the localisation algorithm for performance.