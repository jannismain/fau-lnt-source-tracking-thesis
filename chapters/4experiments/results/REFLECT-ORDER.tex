\subsubsection*{Reflection Order}
%% Combined Box Plot
\begin{figure}[H]
\iftoggle{quick}{
    \includegraphics[width=\textwidth]{plots/boxplots/boxplot-joined-reflect-order}
}{%
    \begin{tikzpicture}
	    % This file was created by matplotlib2tikz v0.6.14.
\definecolor{color0}{rgb}{0.8,0.207843137254902,0.219607843137255}
\definecolor{color1}{rgb}{1,0.647058823529412,0}

\begin{axis}[
xlabel={$S$},
ylabel={MAE},
xmin=0.5, xmax=6.5,
ymin=0, ymax=2.5,
width=\figurewidth,
height=\figureheight,
xtick={1,2,3,4,5,6},
xticklabels={2,3,4,5,6,7},
ytick={0,0.5,1,1.5,2,2.5},
minor xtick={},
minor ytick={},
tick align=outside,
tick pos=left,
x grid style={white!69.019607843137251!black},
ymajorgrids,
y grid style={white!69.019607843137251!black}
]
\addplot [line width=1.0pt, black, opacity=1, forget plot]
table {%
0.81 0
0.89 0
0.89 0.118483858307738
0.81 0.118483858307738
0.81 0
};
\addplot [line width=1.0pt, black, opacity=1, forget plot]
table {%
0.85 0
0.85 0
};
\addplot [line width=1.0pt, black, opacity=1, forget plot]
table {%
0.85 0.118483858307738
0.85 0.291547594742265
};
\addplot [line width=1.0pt, black, forget plot]
table {%
0.83 0
0.87 0
};
\addplot [line width=1.0pt, black, forget plot]
table {%
0.83 0.291547594742265
0.87 0.291547594742265
};
\addplot [line width=0.5pt, black, opacity=0.2, mark=*, mark size=1, mark options={solid}, only marks, forget plot]
table {%
0.85 1.81879497631796
0.85 0.570087712549569
0.85 0.300000000000001
0.85 0.360555127546399
0.85 0.320710678118655
0.85 1.10113577727726
0.85 0.738388304799329
0.85 1.25099960031968
0.85 0.656155281280883
0.85 1.32114206111977
0.85 0.424264068711929
0.85 0.3
0.85 0.536067467586918
0.85 0.494974746830583
0.85 0.308113883008419
0.85 0.476831552862278
0.85 0.613120412840139
0.85 0.370156211871643
0.85 0.476831552862278
0.85 0.473823565533582
0.85 0.320156211871642
0.85 0.394646111349609
0.85 0.332842712474619
0.85 0.353553390593274
0.85 0.320156211871642
};
\addplot [line width=1.0pt, black, opacity=1, forget plot]
table {%
1.81 0.0333333333333333
1.89 0.0333333333333333
1.89 0.141421356237309
1.81 0.141421356237309
1.81 0.0333333333333333
};
\addplot [line width=1.0pt, black, opacity=1, forget plot]
table {%
1.85 0.0333333333333333
1.85 0
};
\addplot [line width=1.0pt, black, opacity=1, forget plot]
table {%
1.85 0.141421356237309
1.85 0.302075258276618
};
\addplot [line width=1.0pt, black, forget plot]
table {%
1.83 0
1.87 0
};
\addplot [line width=1.0pt, black, forget plot]
table {%
1.83 0.302075258276618
1.87 0.302075258276618
};
\addplot [line width=0.5pt, black, opacity=0.2, mark=*, mark size=1, mark options={solid}, only marks, forget plot]
table {%
1.85 0.708281891043887
1.85 1.52175515565615
1.85 0.568665345090803
1.85 0.803692342542495
1.85 0.417665469538055
1.85 0.391320559326353
1.85 1.4893609622647
1.85 0.882546819658248
1.85 0.320077508901421
1.85 0.978001253268187
1.85 1.3294439112595
1.85 0.454970354689117
1.85 2.02739246130394
1.85 0.780031151967757
1.85 1.1726952694386
1.85 0.65659052011974
1.85 1.02885099581353
1.85 1.06349895936529
1.85 0.872645220800945
1.85 1.54215784193288
1.85 0.527562521893063
1.85 0.556823330357321
1.85 0.448903464179968
1.85 0.389085704926969
1.85 0.337633006816399
1.85 1.25500685267195
1.85 1.02278152800765
};
\addplot [line width=1.0pt, black, opacity=1, forget plot]
table {%
2.81 0.0749999999999998
2.89 0.0749999999999998
2.89 0.238860355975559
2.81 0.238860355975559
2.81 0.0749999999999998
};
\addplot [line width=1.0pt, black, opacity=1, forget plot]
table {%
2.85 0.0749999999999998
2.85 0
};
\addplot [line width=1.0pt, black, opacity=1, forget plot]
table {%
2.85 0.238860355975559
2.85 0.472565245406059
};
\addplot [line width=1.0pt, black, forget plot]
table {%
2.83 0
2.87 0
};
\addplot [line width=1.0pt, black, forget plot]
table {%
2.83 0.472565245406059
2.87 0.472565245406059
};
\addplot [line width=0.5pt, black, opacity=0.2, mark=*, mark size=1, mark options={solid}, only marks, forget plot]
table {%
2.85 1.01007810593582
2.85 0.606590689152015
2.85 1.12630905108414
2.85 0.762510938276544
2.85 1.10335580722402
2.85 0.651996810199223
2.85 0.924103661702208
2.85 1.03023497468821
2.85 1.3741491299342
2.85 0.878683041630124
2.85 0.746831849946627
2.85 0.704508497187474
2.85 0.916294516764129
2.85 0.48590314984643
2.85 0.554064157519522
2.85 1.06309557861405
2.85 0.714667955474447
2.85 0.75756155624115
2.85 0.958407735129724
2.85 0.516237558240206
2.85 1.48238695529207
2.85 0.97696010459073
2.85 0.796046262224709
2.85 0.732578923727238
2.85 0.980668169854663
2.85 0.899428203968916
2.85 1.12059401806375
2.85 0.76297185210878
2.85 1.12468849316917
2.85 0.515275401030502
2.85 0.811055792763006
2.85 0.726040764008565
2.85 1.34977325807868
2.85 1.22658790489272
2.85 1.23765565339855
2.85 0.68762338991869
2.85 0.503321010216138
2.85 0.487635808334312
2.85 0.654669331122391
2.85 0.535977222864644
2.85 0.654598555259563
};
\addplot [line width=1.0pt, black, opacity=1, forget plot]
table {%
3.81 0.0847213595499959
3.89 0.0847213595499959
3.89 0.53767858169252
3.81 0.53767858169252
3.81 0.0847213595499959
};
\addplot [line width=1.0pt, black, opacity=1, forget plot]
table {%
3.85 0.0847213595499959
3.85 0.0199999999999999
};
\addplot [line width=1.0pt, black, opacity=1, forget plot]
table {%
3.85 0.53767858169252
3.85 1.19735457198578
};
\addplot [line width=1.0pt, black, forget plot]
table {%
3.83 0.0199999999999999
3.87 0.0199999999999999
};
\addplot [line width=1.0pt, black, forget plot]
table {%
3.83 1.19735457198578
3.87 1.19735457198578
};
\addplot [line width=0.5pt, black, opacity=0.2, mark=*, mark size=1, mark options={solid}, only marks, forget plot]
table {%
3.85 1.45122199539078
3.85 1.75891976483729
3.85 1.23984204336903
3.85 1.86770298853507
};
\addplot [line width=1.0pt, black, opacity=1, forget plot]
table {%
4.81 0.140203056008638
4.89 0.140203056008638
4.89 0.669967457958557
4.81 0.669967457958557
4.81 0.140203056008638
};
\addplot [line width=1.0pt, black, opacity=1, forget plot]
table {%
4.85 0.140203056008638
4.85 0
};
\addplot [line width=1.0pt, black, opacity=1, forget plot]
table {%
4.85 0.669967457958557
4.85 1.43930858321675
};
\addplot [line width=1.0pt, black, forget plot]
table {%
4.83 0
4.87 0
};
\addplot [line width=1.0pt, black, forget plot]
table {%
4.83 1.43930858321675
4.87 1.43930858321675
};
\addplot [line width=1.0pt, black, opacity=1, forget plot]
table {%
5.81 0.151751075288135
5.89 0.151751075288135
5.89 0.613420547379396
5.81 0.613420547379396
5.81 0.151751075288135
};
\addplot [line width=1.0pt, black, opacity=1, forget plot]
table {%
5.85 0.151751075288135
5.85 0
};
\addplot [line width=1.0pt, black, opacity=1, forget plot]
table {%
5.85 0.613420547379396
5.85 1.26144907427581
};
\addplot [line width=1.0pt, black, forget plot]
table {%
5.83 0
5.87 0
};
\addplot [line width=1.0pt, black, forget plot]
table {%
5.83 1.26144907427581
5.87 1.26144907427581
};
\addplot [line width=0.5pt, black, opacity=0.2, mark=*, mark size=1, mark options={solid}, only marks, forget plot]
table {%
5.85 1.46953226774191
5.85 1.4128528026599
5.85 1.37451371685978
};
\addplot [line width=1.0pt, color0, opacity=1, forget plot]
table {%
0.96 0
1.04 0
1.04 0.181954948688533
0.96 0.181954948688533
0.96 0
};
\addplot [line width=1.0pt, color0, opacity=1, forget plot]
table {%
1 0
1 0
};
\addplot [line width=1.0pt, color0, opacity=1, forget plot]
table {%
1 0.181954948688533
1 0.453112887414927
};
\addplot [line width=1.0pt, color0, forget plot]
table {%
0.98 0
1.02 0
};
\addplot [line width=1.0pt, color0, forget plot]
table {%
0.98 0.453112887414927
1.02 0.453112887414927
};
\addplot [line width=0.5pt, color0, opacity=0.2, mark=*, mark size=1, mark options={solid}, only marks, forget plot]
table {%
1 1.02009631108587
1 1.56923548264752
1 1.05663729752108
1 1.12805141726785
1 0.570156211871642
1 0.956552314939536
1 0.540832691319599
1 0.734254389867042
1 0.51478150704935
1 0.494974746830583
1 0.559016994374947
1 0.541547594742265
1 1.72923503981303
1 0.502315882670322
1 1.03191727680481
1 0.515154392492244
1 0.541547594742265
1 0.51478150704935
1 0.56478150704935
1 0.554138126514911
};
\addplot [line width=1.0pt, color0, opacity=1, forget plot]
table {%
1.96 0.0471404520791032
2.04 0.0471404520791032
2.04 0.317657035137148
1.96 0.317657035137148
1.96 0.0471404520791032
};
\addplot [line width=1.0pt, color0, opacity=1, forget plot]
table {%
2 0.0471404520791032
2 0
};
\addplot [line width=1.0pt, color0, opacity=1, forget plot]
table {%
2 0.317657035137148
2 0.690010572469092
};
\addplot [line width=1.0pt, color0, forget plot]
table {%
1.98 0
2.02 0
};
\addplot [line width=1.0pt, color0, forget plot]
table {%
1.98 0.690010572469092
2.02 0.690010572469092
};
\addplot [line width=0.5pt, color0, opacity=0.2, mark=*, mark size=1, mark options={solid}, only marks, forget plot]
table {%
2 1.33773092840912
2 0.796317446383593
2 0.91053134618229
2 1.05455324682642
2 1.28766005647433
2 1.31693991391655
2 1.09189253101726
2 0.734258545910665
2 0.736103059091004
2 0.949071198499986
2 0.827453779460412
2 2.73742585590864
2 0.74535599249993
2 2.17726592674481
2 1.07490271142419
2 0.970349276552203
2 1.24466022874433
2 0.839559108163188
2 0.767423851518175
2 1.22102317431496
2 1.30586716604234
2 1.25799242164164
2 0.852001655852138
2 0.892743429832263
2 1.27711659939329
2 0.741072378723401
2 0.760406066714697
2 1.02915708346295
2 0.817176711475418
2 0.827453779460412
2 0.81380677740496
2 0.753048067202114
2 1.01771251037122
2 1.70731953595757
};
\addplot [line width=1.0pt, color0, opacity=1, forget plot]
table {%
2.96 0.112455619297126
3.04 0.112455619297126
3.04 0.59504750370721
2.96 0.59504750370721
2.96 0.112455619297126
};
\addplot [line width=1.0pt, color0, opacity=1, forget plot]
table {%
3 0.112455619297126
3 0
};
\addplot [line width=1.0pt, color0, opacity=1, forget plot]
table {%
3 0.59504750370721
3 1.27410415417437
};
\addplot [line width=1.0pt, color0, forget plot]
table {%
2.98 0
3.02 0
};
\addplot [line width=1.0pt, color0, forget plot]
table {%
2.98 1.27410415417437
3.02 1.27410415417437
};
\addplot [line width=0.5pt, color0, opacity=0.2, mark=*, mark size=1, mark options={solid}, only marks, forget plot]
table {%
3 1.65143213723643
3 1.45065733071255
3 1.34005622624516
3 1.50914185259818
3 1.51787766977529
3 1.58956125413922
};
\addplot [line width=1.0pt, color0, opacity=1, forget plot]
table {%
3.96 0.165824424940892
4.04 0.165824424940892
4.04 0.731764919939242
3.96 0.731764919939242
3.96 0.165824424940892
};
\addplot [line width=1.0pt, color0, opacity=1, forget plot]
table {%
4 0.165824424940892
4 0.0199999999999999
};
\addplot [line width=1.0pt, color0, opacity=1, forget plot]
table {%
4 0.731764919939242
4 1.35587645886755
};
\addplot [line width=1.0pt, color0, forget plot]
table {%
3.98 0.0199999999999999
4.02 0.0199999999999999
};
\addplot [line width=1.0pt, color0, forget plot]
table {%
3.98 1.35587645886755
4.02 1.35587645886755
};
\addplot [line width=0.5pt, color0, opacity=0.2, mark=*, mark size=1, mark options={solid}, only marks, forget plot]
table {%
4 1.5888770921879
4 1.62003019814932
};
\addplot [line width=1.0pt, color0, opacity=1, forget plot]
table {%
4.96 0.344043814351852
5.04 0.344043814351852
5.04 0.841259160161866
4.96 0.841259160161866
4.96 0.344043814351852
};
\addplot [line width=1.0pt, color0, opacity=1, forget plot]
table {%
5 0.344043814351852
5 0.0333333333333333
};
\addplot [line width=1.0pt, color0, opacity=1, forget plot]
table {%
5 0.841259160161866
5 1.42418950732103
};
\addplot [line width=1.0pt, color0, forget plot]
table {%
4.98 0.0333333333333333
5.02 0.0333333333333333
};
\addplot [line width=1.0pt, color0, forget plot]
table {%
4.98 1.42418950732103
5.02 1.42418950732103
};
\addplot [line width=1.0pt, color0, opacity=1, forget plot]
table {%
5.96 0.34607418312141
6.04 0.34607418312141
6.04 0.721356838246698
5.96 0.721356838246698
5.96 0.34607418312141
};
\addplot [line width=1.0pt, color0, opacity=1, forget plot]
table {%
6 0.34607418312141
6 0.0693712943361397
};
\addplot [line width=1.0pt, color0, opacity=1, forget plot]
table {%
6 0.721356838246698
6 1.25767671358356
};
\addplot [line width=1.0pt, color0, forget plot]
table {%
5.98 0.0693712943361397
6.02 0.0693712943361397
};
\addplot [line width=1.0pt, color0, forget plot]
table {%
5.98 1.25767671358356
6.02 1.25767671358356
};
\addplot [line width=0.5pt, color0, opacity=0.2, mark=*, mark size=1, mark options={solid}, only marks, forget plot]
table {%
6 1.29011029517326
6 1.33377233279389
};
\addplot [line width=1.0pt, color1, opacity=1, forget plot]
table {%
1.11 0.0499999999999998
1.19 0.0499999999999998
1.19 0.289371374175354
1.11 0.289371374175354
1.11 0.0499999999999998
};
\addplot [line width=1.0pt, color1, opacity=1, forget plot]
table {%
1.15 0.0499999999999998
1.15 0
};
\addplot [line width=1.0pt, color1, opacity=1, forget plot]
table {%
1.15 0.289371374175354
1.15 0.633864246327115
};
\addplot [line width=1.0pt, color1, forget plot]
table {%
1.13 0
1.17 0
};
\addplot [line width=1.0pt, color1, forget plot]
table {%
1.13 0.633864246327115
1.17 0.633864246327115
};
\addplot [line width=0.5pt, color1, opacity=0.2, mark=*, mark size=1, mark options={solid}, only marks, forget plot]
table {%
1.15 0.850326482914885
1.15 0.654619212798457
1.15 0.936002257333468
1.15 0.70178344238091
1.15 0.715891053163818
1.15 1.8580904176062
1.15 1.17219776264172
1.15 0.886002257333468
1.15 1.00958320130474
1.15 1.43442970499026
1.15 1.48416538025861
1.15 1.04403065089106
1.15 1.10453610171873
1.15 1.06117675050949
1.15 1.55724115023974
1.15 1.7464249196573
1.15 1.21711959646805
1.15 0.764400085215328
1.15 1.17686022959398
1.15 0.919238815542512
1.15 0.680073525436772
1.15 0.659345629855922
1.15 0.913133825081604
};
\addplot [line width=1.0pt, color1, opacity=1, forget plot]
table {%
2.11 0.122042341259485
2.19 0.122042341259485
2.19 0.809701604496446
2.11 0.809701604496446
2.11 0.122042341259485
};
\addplot [line width=1.0pt, color1, opacity=1, forget plot]
table {%
2.15 0.122042341259485
2.15 0
};
\addplot [line width=1.0pt, color1, opacity=1, forget plot]
table {%
2.15 0.809701604496446
2.15 1.68830421959243
};
\addplot [line width=1.0pt, color1, forget plot]
table {%
2.13 0
2.17 0
};
\addplot [line width=1.0pt, color1, forget plot]
table {%
2.13 1.68830421959243
2.17 1.68830421959243
};
\addplot [line width=0.5pt, color1, opacity=0.2, mark=*, mark size=1, mark options={solid}, only marks, forget plot]
table {%
2.15 2.02243462933147
2.15 1.8661198814742
};
\addplot [line width=1.0pt, color1, opacity=1, forget plot]
table {%
3.11 0.375775357910832
3.19 0.375775357910832
3.19 0.945431576323539
3.11 0.945431576323539
3.11 0.375775357910832
};
\addplot [line width=1.0pt, color1, opacity=1, forget plot]
table {%
3.15 0.375775357910832
3.15 0.0249999999999999
};
\addplot [line width=1.0pt, color1, opacity=1, forget plot]
table {%
3.15 0.945431576323539
3.15 1.62248217855221
};
\addplot [line width=1.0pt, color1, forget plot]
table {%
3.13 0.0249999999999999
3.17 0.0249999999999999
};
\addplot [line width=1.0pt, color1, forget plot]
table {%
3.13 1.62248217855221
3.17 1.62248217855221
};
\addplot [line width=0.5pt, color1, opacity=0.2, mark=*, mark size=1, mark options={solid}, only marks, forget plot]
table {%
3.15 1.81360890138675
3.15 1.86140944084869
};
\addplot [line width=1.0pt, color1, opacity=1, forget plot]
table {%
4.11 0.542071073365796
4.19 0.542071073365796
4.19 1.01712155858353
4.11 1.01712155858353
4.11 0.542071073365796
};
\addplot [line width=1.0pt, color1, opacity=1, forget plot]
table {%
4.15 0.542071073365796
4.15 0.0199999999999999
};
\addplot [line width=1.0pt, color1, opacity=1, forget plot]
table {%
4.15 1.01712155858353
4.15 1.71110036512443
};
\addplot [line width=1.0pt, color1, forget plot]
table {%
4.13 0.0199999999999999
4.17 0.0199999999999999
};
\addplot [line width=1.0pt, color1, forget plot]
table {%
4.13 1.71110036512443
4.17 1.71110036512443
};
\addplot [line width=0.5pt, color1, opacity=0.2, mark=*, mark size=1, mark options={solid}, only marks, forget plot]
table {%
4.15 1.82286107980356
4.15 1.98143947743723
4.15 2.04041207093658
4.15 1.79107337662464
};
\addplot [line width=1.0pt, color1, opacity=1, forget plot]
table {%
5.11 0.61984011648905
5.19 0.61984011648905
5.19 1.05573193013855
5.11 1.05573193013855
5.11 0.61984011648905
};
\addplot [line width=1.0pt, color1, opacity=1, forget plot]
table {%
5.15 0.61984011648905
5.15 0.0971404520791031
};
\addplot [line width=1.0pt, color1, opacity=1, forget plot]
table {%
5.15 1.05573193013855
5.15 1.64523704409378
};
\addplot [line width=1.0pt, color1, forget plot]
table {%
5.13 0.0971404520791031
5.17 0.0971404520791031
};
\addplot [line width=1.0pt, color1, forget plot]
table {%
5.13 1.64523704409378
5.17 1.64523704409378
};
\addplot [line width=1.0pt, color1, opacity=1, forget plot]
table {%
6.11 0.503874764029806
6.19 0.503874764029806
6.19 0.923071316528485
6.11 0.923071316528485
6.11 0.503874764029806
};
\addplot [line width=1.0pt, color1, opacity=1, forget plot]
table {%
6.15 0.503874764029806
6.15 0.202381174390913
};
\addplot [line width=1.0pt, color1, opacity=1, forget plot]
table {%
6.15 0.923071316528485
6.15 1.54003411482351
};
\addplot [line width=1.0pt, color1, forget plot]
table {%
6.13 0.202381174390913
6.17 0.202381174390913
};
\addplot [line width=1.0pt, color1, forget plot]
table {%
6.13 1.54003411482351
6.17 1.54003411482351
};
\addplot [line width=0.5pt, color1, opacity=0.2, mark=*, mark size=1, mark options={solid}, only marks, forget plot]
table {%
6.15 1.68265205518671
6.15 1.62119423156085
6.15 1.65383790608507
6.15 1.61588952099776
6.15 1.64878156429865
};
\addplot [line width=1.0pt, black, opacity=1, forget plot]
table {%
0.81 0.05
0.89 0.05
};
\addplot [line width=1.0pt, black, dashed, mark=x, mark size=3, mark options={solid}, forget plot]
table {%
0.85 0.116032098799097
};
\addplot [line width=1.0pt, black, opacity=1, forget plot]
table {%
1.81 0.074535599249993
1.89 0.074535599249993
};
\addplot [line width=1.0pt, black, dashed, mark=x, mark size=3, mark options={solid}, forget plot]
table {%
1.85 0.165531391278566
};
\addplot [line width=1.0pt, black, opacity=1, forget plot]
table {%
2.81 0.125247060472964
2.89 0.125247060472964
};
\addplot [line width=1.0pt, black, dashed, mark=x, mark size=3, mark options={solid}, forget plot]
table {%
2.85 0.246580555853142
};
\addplot [line width=1.0pt, black, opacity=1, forget plot]
table {%
3.81 0.195477888107424
3.89 0.195477888107424
};
\addplot [line width=1.0pt, black, dashed, mark=x, mark size=3, mark options={solid}, forget plot]
table {%
3.85 0.348143251050348
};
\addplot [line width=1.0pt, black, opacity=1, forget plot]
table {%
4.81 0.347412204609502
4.89 0.347412204609502
};
\addplot [line width=1.0pt, black, dashed, mark=x, mark size=3, mark options={solid}, forget plot]
table {%
4.85 0.431065992034396
};
\addplot [line width=1.0pt, black, opacity=1, forget plot]
table {%
5.81 0.386106754799573
5.89 0.386106754799573
};
\addplot [line width=1.0pt, black, dashed, mark=x, mark size=3, mark options={solid}, forget plot]
table {%
5.85 0.431414634056223
};
\addplot [line width=1.0pt, color0, opacity=1, forget plot]
table {%
0.96 0.05
1.04 0.05
};
\addplot [line width=1.0pt, color0, dashed, mark=x, mark size=3, mark options={solid}, forget plot]
table {%
1 0.150865604905422
};
\addplot [line width=1.0pt, color0, opacity=1, forget plot]
table {%
1.96 0.138089721429767
2.04 0.138089721429767
};
\addplot [line width=1.0pt, color0, dashed, mark=x, mark size=3, mark options={solid}, forget plot]
table {%
2 0.286361863503465
};
\addplot [line width=1.0pt, color0, opacity=1, forget plot]
table {%
2.96 0.249303398874989
3.04 0.249303398874989
};
\addplot [line width=1.0pt, color0, dashed, mark=x, mark size=3, mark options={solid}, forget plot]
table {%
3 0.399925417717139
};
\addplot [line width=1.0pt, color0, opacity=1, forget plot]
table {%
3.96 0.509163801712671
4.04 0.509163801712671
};
\addplot [line width=1.0pt, color0, dashed, mark=x, mark size=3, mark options={solid}, forget plot]
table {%
4 0.505204259063092
};
\addplot [line width=1.0pt, color0, opacity=1, forget plot]
table {%
4.96 0.589301691045092
5.04 0.589301691045092
};
\addplot [line width=1.0pt, color0, dashed, mark=x, mark size=3, mark options={solid}, forget plot]
table {%
5 0.596819835538899
};
\addplot [line width=1.0pt, color0, opacity=1, forget plot]
table {%
5.96 0.532369723552282
6.04 0.532369723552282
};
\addplot [line width=1.0pt, color0, dashed, mark=x, mark size=3, mark options={solid}, forget plot]
table {%
6 0.556488346997599
};
\addplot [line width=1.0pt, color1, opacity=1, forget plot]
table {%
1.11 0.0999999999999999
1.19 0.0999999999999999
};
\addplot [line width=1.0pt, color1, dashed, mark=x, mark size=3, mark options={solid}, forget plot]
table {%
1.15 0.225727889446119
};
\addplot [line width=1.0pt, color1, opacity=1, forget plot]
table {%
2.11 0.37251692131874
2.19 0.37251692131874
};
\addplot [line width=1.0pt, color1, dashed, mark=x, mark size=3, mark options={solid}, forget plot]
table {%
2.15 0.502256968321234
};
\addplot [line width=1.0pt, color1, opacity=1, forget plot]
table {%
3.11 0.689861349828052
3.19 0.689861349828052
};
\addplot [line width=1.0pt, color1, dashed, mark=x, mark size=3, mark options={solid}, forget plot]
table {%
3.15 0.696704903208663
};
\addplot [line width=1.0pt, color1, opacity=1, forget plot]
table {%
4.11 0.764085739293198
4.19 0.764085739293198
};
\addplot [line width=1.0pt, color1, dashed, mark=x, mark size=3, mark options={solid}, forget plot]
table {%
4.15 0.79931943370046
};
\addplot [line width=1.0pt, color1, opacity=1, forget plot]
table {%
5.11 0.831027441912347
5.19 0.831027441912347
};
\addplot [line width=1.0pt, color1, dashed, mark=x, mark size=3, mark options={solid}, forget plot]
table {%
5.15 0.832043477671639
};
\addplot [line width=1.0pt, color1, opacity=1, forget plot]
table {%
6.11 0.704091591508899
6.19 0.704091591508899
};
\addplot [line width=1.0pt, color1, dashed, mark=x, mark size=3, mark options={solid}, forget plot]
table {%
6.15 0.745490017784198
};
\end{axis}

\node at ({$(current bounding box.south west)!0.5!(current bounding box.south east)$}|-{$(current bounding box.south west)!0.98!(current bounding box.north west)$})[
  anchor=north,
  text=black,
  rotate=0.0
]{ };

	    \begin{customlegend}[
legend entries={$r=0$,$r=3$,$r=$\ max},
legend cell align=left,
legend style={at={(0.05,5.37)}, anchor=north west, draw=white!80.0!black, font=\footnotesize,fill opacity=0.5, draw opacity=1,text opacity=1}]
    \addlegendimage{area legend,black,fill=black, fill opacity=1}
    \addlegendimage{area legend,color0,fill=color0, fill opacity=1}
    \addlegendimage{area legend,color1,fill=color1, fill opacity=1}
\end{customlegend}
\end{tikzpicture}
	
}
	\caption[Evaluation Results for $r$]{Evaluation Results for $r$ ($n=250$, \Tsixty$=0.6$~s).}
	\label{fig:trialR}
\end{figure}

For this evaluation, the amount of reverberation has been increased to T$_{60}=0.6$~s. As expected, the reflection order $r$ has a significant negative effect on localisation performance. In addition, $r=3$ seems to strike a good balance between a realistic simulation environment and managable computational complexity. The \gls{mae} is right in between of either having no reverberation ($r=0$) or fully simulating all reverberations ($r=$~max). In general, this validates the choice of $r=3$ as default parameter for most other evaluations. However, it has to be noted that the effect other parameters might exhibit to have on \gls{mae}, might be diminished by not fully simulating all parts of the reverberation. Therefore, for trials where reverberation is the explicit target of the evaluation (e.g., for the evaluation of T$_{60}$), the additional computational complexity is accepted in order to achieve more accurate results. For all other trials, where $r=3$ is choosen to achieve a larger sample size, the additional negative effect of a more realistic simulation with $r=$~max has to be kept in mind when interpreting the results.

