\subsubsection{Static Scenario Evaluation}
\label{sec:ScenarioEvalStatic}

To see, what localisation performance is achievable by the source localisation algorithm in this setup, a first evaluation of different scenarios compares an environment considered ideal for source localisation (no noise, no reverberation) to more adverse conditions. The parameters used in this evaluation are summarised in \autoref{table:parametersBestCase}.

%% Parameter sets
\begin{table}[!hbt]
	\begin{tabular}{rccccc}
		\toprule
		Parameter          & Unit & Symbol   & Best Case & Base     & Worst Case \\
		\midrule
		Reverberation Time & s    & T$_{60}$ & 0.0       & 0.3      & 0.9        \\
		SNR                & dB   &          & no noise  & 30       & 5          \\
		Reflection Order   &      & $r$      & 0         & 3        & max        \\
		\bottomrule
	\end{tabular}
	\caption[Parameter Set for Static Scenario Evaluation]{Parameter Set for Static Scenario Evaluation.}
	\label{table:parametersBestCase}
\end{table}

% BOXPLOT
\begin{figure}[H]
	\iftoggle{quick}{%
		\includegraphics[width=\textwidth]{plots/boxplots/boxplot-joined-best-case}
	}{%
		\begin{tikzpicture}
		% This file was created by matplotlib2tikz v0.6.14.
\definecolor{color0}{rgb}{0.8,0.207843137254902,0.219607843137255}
\definecolor{color1}{rgb}{1,0.647058823529412,0}
\definecolor{color2}{rgb}{0.0235294117647059,0.603921568627451,0.952941176470588}
\begin{axis}[
xlabel={$S$},
ylabel={MAE},
xmin=0.5, xmax=6.5,
ymin=0, ymax=3.51,
width=\figurewidth,
height=\figureheight,
xtick={1,2,3,4,5,6},
xticklabels={2,3,4,5,6,7},
ytick={0,0.5,1,1.5,2,2.5,3,3.5},
minor xtick={},
minor ytick={},
tick align=outside,
tick pos=left,
x grid style={white!69.019607843137251!black},
ymajorgrids,
y grid style={white!69.019607843137251!black}
]
\addplot [line width=1.0pt, black, opacity=1, forget plot]
table {%
0.76 0
0.84 0
0.84 0
0.76 0
0.76 0
};
\addplot [line width=1.0pt, black, opacity=1, forget plot]
table {%
0.8 0
0.8 0
};
\addplot [line width=1.0pt, black, opacity=1, forget plot]
table {%
0.8 0
0.8 0
};
\addplot [line width=1.0pt, black, forget plot]
table {%
0.78 0
0.82 0
};
\addplot [line width=1.0pt, black, forget plot]
table {%
0.78 0
0.82 0
};
\addplot [line width=0.5pt, black, opacity=0.2, mark=*, mark size=1, mark options={solid}, only marks, forget plot]
table {%
0.8 0.223606797749979
};
\addplot [line width=1.0pt, black, opacity=1, forget plot]
table {%
1.76 0
1.84 0
1.84 0
1.76 0
1.76 0
};
\addplot [line width=1.0pt, black, opacity=1, forget plot]
table {%
1.8 0
1.8 0
};
\addplot [line width=1.0pt, black, opacity=1, forget plot]
table {%
1.8 0
1.8 0
};
\addplot [line width=1.0pt, black, forget plot]
table {%
1.78 0
1.82 0
};
\addplot [line width=1.0pt, black, forget plot]
table {%
1.78 0
1.82 0
};
\addplot [line width=0.5pt, black, opacity=0.2, mark=*, mark size=1, mark options={solid}, only marks, forget plot]
table {%
1.8 0.333333333333333
1.8 0.687184270936277
1.8 0.298142396999972
1.8 0.213437474581095
1.8 0.887568463712957
1.8 0.166666666666667
1.8 0.687184270936277
1.8 0.188561808316413
1.8 0.120185042515466
1.8 0.260341655863555
1.8 0.780313327381308
1.8 0.307318148576429
1.8 0.52704627669473
1.8 0.333333333333333
1.8 1.22955321715015
1.8 0.169967317119759
1.8 0.817176711475417
1.8 0.169967317119759
1.8 0.477260702109212
1.8 0.120185042515466
1.8 0.495535624910617
1.8 0.554016645060443
1.8 0.900617072407087
1.8 0.74535599249993
};
\addplot [line width=1.0pt, black, opacity=1, forget plot]
table {%
2.76 0
2.84 0
2.84 0
2.76 0
2.76 0
};
\addplot [line width=1.0pt, black, opacity=1, forget plot]
table {%
2.8 0
2.8 0
};
\addplot [line width=1.0pt, black, opacity=1, forget plot]
table {%
2.8 0
2.8 0
};
\addplot [line width=1.0pt, black, forget plot]
table {%
2.78 0
2.82 0
};
\addplot [line width=1.0pt, black, forget plot]
table {%
2.78 0
2.82 0
};
\addplot [line width=0.5pt, black, opacity=0.2, mark=*, mark size=1, mark options={solid}, only marks, forget plot]
table {%
2.8 0.549758189657856
2.8 0.5
2.8 0.206155281280883
2.8 0.450693909432999
2.8 0.835568665640951
2.8 0.458938993767146
2.8 0.11180339887499
2.8 0.195256241897666
2.8 0.450693909432999
2.8 0.201556443707464
2.8 0.447213595499958
2.8 0.922385025721586
2.8 0.391311896062463
2.8 0.305277563773199
2.8 0.704278769020671
2.8 0.425
2.8 0.20405694150421
2.8 0.6
2.8 0.195256241897666
2.8 0.213600093632938
2.8 0.815272599161946
2.8 0.5
2.8 0.828402076289044
2.8 0.625
2.8 0.160078105935821
2.8 1.05504739230046
2.8 0.125
2.8 0.141421356237309
2.8 0.3
2.8 0.728868986855663
2.8 0.390512483795333
2.8 0.643465843842649
2.8 0.160078105935821
2.8 0.883883476483184
2.8 0.474341649025257
2.8 0.206155281280883
2.8 0.910470982217074
2.8 0.348209706929603
2.8 0.608789783094296
2.8 0.0790569415042094
2.8 0.452827171712309
2.8 0.503115294937453
2.8 0.807000619578449
2.8 0.325960120260132
2.8 0.664266512779321
};
\addplot [line width=1.0pt, black, opacity=1, forget plot]
table {%
3.76 0
3.84 0
3.84 0.244676870129118
3.76 0.244676870129118
3.76 0
};
\addplot [line width=1.0pt, black, opacity=1, forget plot]
table {%
3.8 0
3.8 0
};
\addplot [line width=1.0pt, black, opacity=1, forget plot]
table {%
3.8 0.244676870129118
3.8 0.608276253029822
};
\addplot [line width=1.0pt, black, forget plot]
table {%
3.78 0
3.82 0
};
\addplot [line width=1.0pt, black, forget plot]
table {%
3.78 0.608276253029822
3.82 0.608276253029822
};
\addplot [line width=0.5pt, black, opacity=0.2, mark=*, mark size=1, mark options={solid}, only marks, forget plot]
table {%
3.8 0.647312336830356
3.8 1.21426219830839
3.8 0.773666562565053
3.8 0.693397432934389
3.8 1.00916061841246
3.8 0.886952072420197
3.8 0.627694193059009
3.8 0.653038674684924
3.8 0.860716733341669
3.8 0.642806347199528
3.8 1.15792249378939
3.8 0.722495674727538
};
\addplot [line width=1.0pt, black, opacity=1, forget plot]
table {%
4.76 0
4.84 0
4.84 0.352959866627896
4.76 0.352959866627896
4.76 0
};
\addplot [line width=1.0pt, black, opacity=1, forget plot]
table {%
4.8 0
4.8 0
};
\addplot [line width=1.0pt, black, opacity=1, forget plot]
table {%
4.8 0.352959866627896
4.8 0.870445625247571
};
\addplot [line width=1.0pt, black, forget plot]
table {%
4.78 0
4.82 0
};
\addplot [line width=1.0pt, black, forget plot]
table {%
4.78 0.870445625247571
4.82 0.870445625247571
};
\addplot [line width=0.5pt, black, opacity=0.2, mark=*, mark size=1, mark options={solid}, only marks, forget plot]
table {%
4.8 0.92518880942054
4.8 0.942310269929699
};
\addplot [line width=1.0pt, black, opacity=1, forget plot]
table {%
5.76 0
5.84 0
5.84 0.410540987180671
5.76 0.410540987180671
5.76 0
};
\addplot [line width=1.0pt, black, opacity=1, forget plot]
table {%
5.8 0
5.8 0
};
\addplot [line width=1.0pt, black, opacity=1, forget plot]
table {%
5.8 0.410540987180671
5.8 0.957766495294535
};
\addplot [line width=1.0pt, black, forget plot]
table {%
5.78 0
5.82 0
};
\addplot [line width=1.0pt, black, forget plot]
table {%
5.78 0.957766495294535
5.82 0.957766495294535
};
\addplot [line width=1.0pt, color0, opacity=1, forget plot]
table {%
0.893333333333333 0
0.973333333333333 0
0.973333333333333 0.14142135623731
0.893333333333333 0.14142135623731
0.893333333333333 0
};
\addplot [line width=1.0pt, color0, opacity=1, forget plot]
table {%
0.933333333333333 0
0.933333333333333 0
};
\addplot [line width=1.0pt, color0, opacity=1, forget plot]
table {%
0.933333333333333 0.14142135623731
0.933333333333333 0.341547594742265
};
\addplot [line width=1.0pt, color0, forget plot]
table {%
0.913333333333333 0
0.953333333333333 0
};
\addplot [line width=1.0pt, color0, forget plot]
table {%
0.913333333333333 0.341547594742265
0.953333333333333 0.341547594742265
};
\addplot [line width=0.5pt, color0, opacity=0.2, mark=*, mark size=1, mark options={solid}, only marks, forget plot]
table {%
0.933333333333333 1.30384048104053
0.933333333333333 0.694660482157192
0.933333333333333 0.518669010554817
0.933333333333333 0.381720680758398
0.933333333333333 0.403884361523178
0.933333333333333 1.14236596587959
0.933333333333333 0.629727672493603
0.933333333333333 0.403112887414928
0.933333333333333 0.700944867164838
0.933333333333333 1.41014705087354
0.933333333333333 0.497702876023148
};
\addplot [line width=1.0pt, color0, opacity=1, forget plot]
table {%
1.89333333333333 0.0333333333333332
1.97333333333333 0.0333333333333332
1.97333333333333 0.179615333825597
1.89333333333333 0.179615333825597
1.89333333333333 0.0333333333333332
};
\addplot [line width=1.0pt, color0, opacity=1, forget plot]
table {%
1.93333333333333 0.0333333333333332
1.93333333333333 0
};
\addplot [line width=1.0pt, color0, opacity=1, forget plot]
table {%
1.93333333333333 0.179615333825597
1.93333333333333 0.389001636985213
};
\addplot [line width=1.0pt, color0, forget plot]
table {%
1.91333333333333 0
1.95333333333333 0
};
\addplot [line width=1.0pt, color0, forget plot]
table {%
1.91333333333333 0.389001636985213
1.95333333333333 0.389001636985213
};
\addplot [line width=0.5pt, color0, opacity=0.2, mark=*, mark size=1, mark options={solid}, only marks, forget plot]
table {%
1.93333333333333 1.67116980736225
1.93333333333333 0.936670798525062
1.93333333333333 0.847834595376724
1.93333333333333 0.869226987360353
1.93333333333333 1.71617263394592
1.93333333333333 0.883583757334196
1.93333333333333 1.52691489891608
1.93333333333333 0.985815641434352
1.93333333333333 0.622826878618383
1.93333333333333 0.763450444278588
1.93333333333333 0.97016577248476
1.93333333333333 1.01623534988749
1.93333333333333 0.739532895847734
1.93333333333333 1.30649864429524
1.93333333333333 0.738054501352472
1.93333333333333 0.633625567074838
1.93333333333333 1.89467765559472
1.93333333333333 0.424264068711929
1.93333333333333 0.724493748554388
1.93333333333333 0.659609807601865
1.93333333333333 1.0295630140987
1.93333333333333 1.36829171678492
1.93333333333333 1.33333333333333
};
\addplot [line width=1.0pt, color0, opacity=1, forget plot]
table {%
2.89333333333333 0.0707106781186548
2.97333333333333 0.0707106781186548
2.97333333333333 0.485121287580844
2.89333333333333 0.485121287580844
2.89333333333333 0.0707106781186548
};
\addplot [line width=1.0pt, color0, opacity=1, forget plot]
table {%
2.93333333333333 0.0707106781186548
2.93333333333333 0
};
\addplot [line width=1.0pt, color0, opacity=1, forget plot]
table {%
2.93333333333333 0.485121287580844
2.93333333333333 1.07648945844305
};
\addplot [line width=1.0pt, color0, forget plot]
table {%
2.91333333333333 0
2.95333333333333 0
};
\addplot [line width=1.0pt, color0, forget plot]
table {%
2.91333333333333 1.07648945844305
2.95333333333333 1.07648945844305
};
\addplot [line width=0.5pt, color0, opacity=0.2, mark=*, mark size=1, mark options={solid}, only marks, forget plot]
table {%
2.93333333333333 1.12541655099018
2.93333333333333 1.21604587685333
2.93333333333333 1.35873330119747
2.93333333333333 1.16018016555873
2.93333333333333 1.3348020272152
2.93333333333333 1.35814081650878
2.93333333333333 1.19171893311731
};
\addplot [line width=1.0pt, color0, opacity=1, forget plot]
table {%
3.89333333333333 0.0809016994374947
3.97333333333333 0.0809016994374947
3.97333333333333 0.487994923610762
3.89333333333333 0.487994923610762
3.89333333333333 0.0809016994374947
};
\addplot [line width=1.0pt, color0, opacity=1, forget plot]
table {%
3.93333333333333 0.0809016994374947
3.93333333333333 0
};
\addplot [line width=1.0pt, color0, opacity=1, forget plot]
table {%
3.93333333333333 0.487994923610762
3.93333333333333 1.05643623705788
};
\addplot [line width=1.0pt, color0, forget plot]
table {%
3.91333333333333 0
3.95333333333333 0
};
\addplot [line width=1.0pt, color0, forget plot]
table {%
3.91333333333333 1.05643623705788
3.95333333333333 1.05643623705788
};
\addplot [line width=0.5pt, color0, opacity=0.2, mark=*, mark size=1, mark options={solid}, only marks, forget plot]
table {%
3.93333333333333 1.23478896316371
3.93333333333333 1.42608938005497
3.93333333333333 1.15619235781782
3.93333333333333 1.28898363856257
3.93333333333333 1.14717412576052
3.93333333333333 1.73148433964139
3.93333333333333 1.50173443984357
3.93333333333333 1.17663204947805
3.93333333333333 1.13035227821223
};
\addplot [line width=1.0pt, color0, opacity=1, forget plot]
table {%
4.89333333333333 0.0912570384968223
4.97333333333333 0.0912570384968223
4.97333333333333 0.557056691605868
4.89333333333333 0.557056691605868
4.89333333333333 0.0912570384968223
};
\addplot [line width=1.0pt, color0, opacity=1, forget plot]
table {%
4.93333333333333 0.0912570384968223
4.93333333333333 0
};
\addplot [line width=1.0pt, color0, opacity=1, forget plot]
table {%
4.93333333333333 0.557056691605868
4.93333333333333 1.24765875723692
};
\addplot [line width=1.0pt, color0, forget plot]
table {%
4.91333333333333 0
4.95333333333333 0
};
\addplot [line width=1.0pt, color0, forget plot]
table {%
4.91333333333333 1.24765875723692
4.95333333333333 1.24765875723692
};
\addplot [line width=0.5pt, color0, opacity=0.2, mark=*, mark size=1, mark options={solid}, only marks, forget plot]
table {%
4.93333333333333 2.24801531865128
4.93333333333333 1.29508771254957
4.93333333333333 1.40582988604975
4.93333333333333 1.3524962799298
};
\addplot [line width=1.0pt, color0, opacity=1, forget plot]
table {%
5.89333333333333 0.109241929153869
5.97333333333333 0.109241929153869
5.97333333333333 0.42630019628071
5.89333333333333 0.42630019628071
5.89333333333333 0.109241929153869
};
\addplot [line width=1.0pt, color0, opacity=1, forget plot]
table {%
5.93333333333333 0.109241929153869
5.93333333333333 0
};
\addplot [line width=1.0pt, color0, opacity=1, forget plot]
table {%
5.93333333333333 0.42630019628071
5.93333333333333 0.885828603018025
};
\addplot [line width=1.0pt, color0, forget plot]
table {%
5.91333333333333 0
5.95333333333333 0
};
\addplot [line width=1.0pt, color0, forget plot]
table {%
5.91333333333333 0.885828603018025
5.95333333333333 0.885828603018025
};
\addplot [line width=0.5pt, color0, opacity=0.2, mark=*, mark size=1, mark options={solid}, only marks, forget plot]
table {%
5.93333333333333 1.01493333816588
5.93333333333333 0.979358822164515
5.93333333333333 1.34168132653353
5.93333333333333 1.73133773184488
5.93333333333333 1.14451604405345
5.93333333333333 0.967452923129838
5.93333333333333 1.03305840397518
5.93333333333333 1.92043358301823
5.93333333333333 1.39156207689245
5.93333333333333 1.03160494507209
};
\addplot [line width=1.0pt, color1, opacity=1, forget plot]
table {%
1.02666666666667 0.269752521501738
1.10666666666667 0.269752521501738
1.10666666666667 1.10849272735239
1.02666666666667 1.10849272735239
1.02666666666667 0.269752521501738
};
\addplot [line width=1.0pt, color1, opacity=1, forget plot]
table {%
1.06666666666667 0.269752521501738
1.06666666666667 0
};
\addplot [line width=1.0pt, color1, opacity=1, forget plot]
table {%
1.06666666666667 1.10849272735239
1.06666666666667 2.17190951739229
};
\addplot [line width=1.0pt, color1, forget plot]
table {%
1.04666666666667 0
1.08666666666667 0
};
\addplot [line width=1.0pt, color1, forget plot]
table {%
1.04666666666667 2.17190951739229
1.08666666666667 2.17190951739229
};
\addplot [line width=0.5pt, color1, opacity=0.2, mark=*, mark size=1, mark options={solid}, only marks, forget plot]
table {%
1.06666666666667 2.81333907633335
1.06666666666667 2.46024966777491
};
\addplot [line width=1.0pt, color1, opacity=1, forget plot]
table {%
2.02666666666667 0.70851489180413
2.10666666666667 0.70851489180413
2.10666666666667 1.31554873765619
2.02666666666667 1.31554873765619
2.02666666666667 0.70851489180413
};
\addplot [line width=1.0pt, color1, opacity=1, forget plot]
table {%
2.06666666666667 0.70851489180413
2.06666666666667 0.138742588672279
};
\addplot [line width=1.0pt, color1, opacity=1, forget plot]
table {%
2.06666666666667 1.31554873765619
2.06666666666667 2.12433699127745
};
\addplot [line width=1.0pt, color1, forget plot]
table {%
2.04666666666667 0.138742588672279
2.08666666666667 0.138742588672279
};
\addplot [line width=1.0pt, color1, forget plot]
table {%
2.04666666666667 2.12433699127745
2.08666666666667 2.12433699127745
};
\addplot [line width=0.5pt, color1, opacity=0.2, mark=*, mark size=1, mark options={solid}, only marks, forget plot]
table {%
2.06666666666667 2.37605220565632
};
\addplot [line width=1.0pt, color1, opacity=1, forget plot]
table {%
3.02666666666667 0.615284594501342
3.10666666666667 0.615284594501342
3.10666666666667 1.17610144966066
3.02666666666667 1.17610144966066
3.02666666666667 0.615284594501342
};
\addplot [line width=1.0pt, color1, opacity=1, forget plot]
table {%
3.06666666666667 0.615284594501342
3.06666666666667 0.0471404520791032
};
\addplot [line width=1.0pt, color1, opacity=1, forget plot]
table {%
3.06666666666667 1.17610144966066
3.06666666666667 2.01621893663773
};
\addplot [line width=1.0pt, color1, forget plot]
table {%
3.04666666666667 0.0471404520791032
3.08666666666667 0.0471404520791032
};
\addplot [line width=1.0pt, color1, forget plot]
table {%
3.04666666666667 2.01621893663773
3.08666666666667 2.01621893663773
};
\addplot [line width=0.5pt, color1, opacity=0.2, mark=*, mark size=1, mark options={solid}, only marks, forget plot]
table {%
3.06666666666667 2.0541125003336
3.06666666666667 2.166119795119
3.06666666666667 2.20895934204196
};
\addplot [line width=1.0pt, color1, opacity=1, forget plot]
table {%
4.02666666666667 0.484074096802648
4.10666666666667 0.484074096802648
4.10666666666667 1.06206155631352
4.02666666666667 1.06206155631352
4.02666666666667 0.484074096802648
};
\addplot [line width=1.0pt, color1, opacity=1, forget plot]
table {%
4.06666666666667 0.484074096802648
4.06666666666667 0.0804737854124364
};
\addplot [line width=1.0pt, color1, opacity=1, forget plot]
table {%
4.06666666666667 1.06206155631352
4.06666666666667 1.85162136003413
};
\addplot [line width=1.0pt, color1, forget plot]
table {%
4.04666666666667 0.0804737854124364
4.08666666666667 0.0804737854124364
};
\addplot [line width=1.0pt, color1, forget plot]
table {%
4.04666666666667 1.85162136003413
4.08666666666667 1.85162136003413
};
\addplot [line width=0.5pt, color1, opacity=0.2, mark=*, mark size=1, mark options={solid}, only marks, forget plot]
table {%
4.06666666666667 2.08203757116228
};
\addplot [line width=1.0pt, color1, opacity=1, forget plot]
table {%
5.02666666666667 0.529822389532531
5.10666666666667 0.529822389532531
5.10666666666667 0.983160994151946
5.02666666666667 0.983160994151946
5.02666666666667 0.529822389532531
};
\addplot [line width=1.0pt, color1, opacity=1, forget plot]
table {%
5.06666666666667 0.529822389532531
5.06666666666667 0.1
};
\addplot [line width=1.0pt, color1, opacity=1, forget plot]
table {%
5.06666666666667 0.983160994151946
5.06666666666667 1.52075922005613
};
\addplot [line width=1.0pt, color1, forget plot]
table {%
5.04666666666667 0.1
5.08666666666667 0.1
};
\addplot [line width=1.0pt, color1, forget plot]
table {%
5.04666666666667 1.52075922005613
5.08666666666667 1.52075922005613
};
\addplot [line width=0.5pt, color1, opacity=0.2, mark=*, mark size=1, mark options={solid}, only marks, forget plot]
table {%
5.06666666666667 1.76616204194763
5.06666666666667 1.8913536276932
5.06666666666667 1.69085452718737
5.06666666666667 1.68625692679761
};
\addplot [line width=1.0pt, color1, opacity=1, forget plot]
table {%
6.02666666666667 0.488224907002558
6.10666666666667 0.488224907002558
6.10666666666667 0.86172080709631
6.02666666666667 0.86172080709631
6.02666666666667 0.488224907002558
};
\addplot [line width=1.0pt, color1, opacity=1, forget plot]
table {%
6.06666666666667 0.488224907002558
6.06666666666667 0.0333333333333334
};
\addplot [line width=1.0pt, color1, opacity=1, forget plot]
table {%
6.06666666666667 0.86172080709631
6.06666666666667 1.28557760732131
};
\addplot [line width=1.0pt, color1, forget plot]
table {%
6.04666666666667 0.0333333333333334
6.08666666666667 0.0333333333333334
};
\addplot [line width=1.0pt, color1, forget plot]
table {%
6.04666666666667 1.28557760732131
6.08666666666667 1.28557760732131
};
\addplot [line width=0.5pt, color1, opacity=0.2, mark=*, mark size=1, mark options={solid}, only marks, forget plot]
table {%
6.06666666666667 1.60008643947993
6.06666666666667 1.45972278930027
6.06666666666667 1.5608368927356
6.06666666666667 1.65666440402801
6.06666666666667 1.57971180711216
};
\addplot [line width=1.0pt, color2, opacity=1, forget plot]
table {%
1.16 1.13008142844494
1.24 1.13008142844494
1.24 2.09463239369265
1.16 2.09463239369265
1.16 1.13008142844494
};
\addplot [line width=1.0pt, color2, opacity=1, forget plot]
table {%
1.2 1.13008142844494
1.2 0.408113883008419
};
\addplot [line width=1.0pt, color2, opacity=1, forget plot]
table {%
1.2 2.09463239369265
1.2 3.24020102389543
};
\addplot [line width=1.0pt, color2, forget plot]
table {%
1.18 0.408113883008419
1.22 0.408113883008419
};
\addplot [line width=1.0pt, color2, forget plot]
table {%
1.18 3.24020102389543
1.22 3.24020102389543
};
\addplot [line width=1.0pt, color2, opacity=1, forget plot]
table {%
2.16 1.15693161841874
2.24 1.15693161841874
2.24 1.86519676981159
2.16 1.86519676981159
2.16 1.15693161841874
};
\addplot [line width=1.0pt, color2, opacity=1, forget plot]
table {%
2.2 1.15693161841874
2.2 0.378639120103915
};
\addplot [line width=1.0pt, color2, opacity=1, forget plot]
table {%
2.2 1.86519676981159
2.2 2.77856881039722
};
\addplot [line width=1.0pt, color2, forget plot]
table {%
2.18 0.378639120103915
2.22 0.378639120103915
};
\addplot [line width=1.0pt, color2, forget plot]
table {%
2.18 2.77856881039722
2.22 2.77856881039722
};
\addplot [line width=1.0pt, color2, opacity=1, forget plot]
table {%
3.16 1.11839043104366
3.24 1.11839043104366
3.24 1.65925726722682
3.16 1.65925726722682
3.16 1.11839043104366
};
\addplot [line width=1.0pt, color2, opacity=1, forget plot]
table {%
3.2 1.11839043104366
3.2 0.569582826011188
};
\addplot [line width=1.0pt, color2, opacity=1, forget plot]
table {%
3.2 1.65925726722682
3.2 2.45741247230659
};
\addplot [line width=1.0pt, color2, forget plot]
table {%
3.18 0.569582826011188
3.22 0.569582826011188
};
\addplot [line width=1.0pt, color2, forget plot]
table {%
3.18 2.45741247230659
3.22 2.45741247230659
};
\addplot [line width=0.5pt, color2, opacity=0.2, mark=*, mark size=1, mark options={solid}, only marks, forget plot]
table {%
3.2 2.47337473759678
3.2 2.50705826626135
};
\addplot [line width=1.0pt, color2, opacity=1, forget plot]
table {%
4.16 0.912041557534763
4.24 0.912041557534763
4.24 1.35689484537257
4.16 1.35689484537257
4.16 0.912041557534763
};
\addplot [line width=1.0pt, color2, opacity=1, forget plot]
table {%
4.2 0.912041557534763
4.2 0.460410196624968
};
\addplot [line width=1.0pt, color2, opacity=1, forget plot]
table {%
4.2 1.35689484537257
4.2 2.01630196641426
};
\addplot [line width=1.0pt, color2, forget plot]
table {%
4.18 0.460410196624968
4.22 0.460410196624968
};
\addplot [line width=1.0pt, color2, forget plot]
table {%
4.18 2.01630196641426
4.22 2.01630196641426
};
\addplot [line width=0.5pt, color2, opacity=0.2, mark=*, mark size=1, mark options={solid}, only marks, forget plot]
table {%
4.2 2.10320692879199
4.2 2.16985916094142
4.2 2.04227880249105
4.2 2.22773068388131
};
\addplot [line width=1.0pt, color2, opacity=1, forget plot]
table {%
5.16 0.765165117671063
5.24 0.765165117671063
5.24 1.20937929926448
5.16 1.20937929926448
5.16 0.765165117671063
};
\addplot [line width=1.0pt, color2, opacity=1, forget plot]
table {%
5.2 0.765165117671063
5.2 0.365860340379199
};
\addplot [line width=1.0pt, color2, opacity=1, forget plot]
table {%
5.2 1.20937929926448
5.2 1.78668149210949
};
\addplot [line width=1.0pt, color2, forget plot]
table {%
5.18 0.365860340379199
5.22 0.365860340379199
};
\addplot [line width=1.0pt, color2, forget plot]
table {%
5.18 1.78668149210949
5.22 1.78668149210949
};
\addplot [line width=0.5pt, color2, opacity=0.2, mark=*, mark size=1, mark options={solid}, only marks, forget plot]
table {%
5.2 2.07079228732163
5.2 2.15218314694814
5.2 1.95926477508579
5.2 1.9876038791113
};
\addplot [line width=1.0pt, color2, opacity=1, forget plot]
table {%
6.16 0.69977663539778
6.24 0.69977663539778
6.24 1.06855018197035
6.16 1.06855018197035
6.16 0.69977663539778
};
\addplot [line width=1.0pt, color2, opacity=1, forget plot]
table {%
6.2 0.69977663539778
6.2 0.297273364031251
};
\addplot [line width=1.0pt, color2, opacity=1, forget plot]
table {%
6.2 1.06855018197035
6.2 1.58557656142173
};
\addplot [line width=1.0pt, color2, forget plot]
table {%
6.18 0.297273364031251
6.22 0.297273364031251
};
\addplot [line width=1.0pt, color2, forget plot]
table {%
6.18 1.58557656142173
6.22 1.58557656142173
};
\addplot [line width=0.5pt, color2, opacity=0.2, mark=*, mark size=1, mark options={solid}, only marks, forget plot]
table {%
6.2 1.72586106600712
};
\addplot [line width=1.0pt, black, opacity=1, forget plot]
table {%
0.76 0
0.84 0
};
\addplot [line width=1.0pt, black, dashed, mark=x, mark size=3, mark options={solid}, forget plot]
table {%
0.8 0.0011180339887499
};
\addplot [line width=1.0pt, black, opacity=1, forget plot]
table {%
1.76 0
1.84 0
};
\addplot [line width=1.0pt, black, dashed, mark=x, mark size=3, mark options={solid}, forget plot]
table {%
1.8 0.0573512605610783
};
\addplot [line width=1.0pt, black, opacity=1, forget plot]
table {%
2.76 0
2.84 0
};
\addplot [line width=1.0pt, black, dashed, mark=x, mark size=3, mark options={solid}, forget plot]
table {%
2.8 0.105530148254809
};
\addplot [line width=1.0pt, black, opacity=1, forget plot]
table {%
3.76 0
3.84 0
};
\addplot [line width=1.0pt, black, dashed, mark=x, mark size=3, mark options={solid}, forget plot]
table {%
3.8 0.153448583719417
};
\addplot [line width=1.0pt, black, opacity=1, forget plot]
table {%
4.76 0.106718737290548
4.84 0.106718737290548
};
\addplot [line width=1.0pt, black, dashed, mark=x, mark size=3, mark options={solid}, forget plot]
table {%
4.8 0.201532353164405
};
\addplot [line width=1.0pt, black, opacity=1, forget plot]
table {%
5.76 0.19452934323983
5.84 0.19452934323983
};
\addplot [line width=1.0pt, black, dashed, mark=x, mark size=3, mark options={solid}, forget plot]
table {%
5.8 0.250813553219818
};
\addplot [line width=1.0pt, color0, opacity=1, forget plot]
table {%
0.893333333333333 0.0499999999999998
0.973333333333333 0.0499999999999998
};
\addplot [line width=1.0pt, color0, dashed, mark=x, mark size=3, mark options={solid}, forget plot]
table {%
0.933333333333333 0.103201918911421
};
\addplot [line width=1.0pt, color0, opacity=1, forget plot]
table {%
1.89333333333333 0.0666666666666667
1.97333333333333 0.0666666666666667
};
\addplot [line width=1.0pt, color0, dashed, mark=x, mark size=3, mark options={solid}, forget plot]
table {%
1.93333333333333 0.192914525605671
};
\addplot [line width=1.0pt, color0, opacity=1, forget plot]
table {%
2.89333333333333 0.154304001977173
2.97333333333333 0.154304001977173
};
\addplot [line width=1.0pt, color0, dashed, mark=x, mark size=3, mark options={solid}, forget plot]
table {%
2.93333333333333 0.303630724385165
};
\addplot [line width=1.0pt, color0, opacity=1, forget plot]
table {%
3.89333333333333 0.193427400616229
3.97333333333333 0.193427400616229
};
\addplot [line width=1.0pt, color0, dashed, mark=x, mark size=3, mark options={solid}, forget plot]
table {%
3.93333333333333 0.327065521730923
};
\addplot [line width=1.0pt, color0, opacity=1, forget plot]
table {%
4.89333333333333 0.199613900595886
4.97333333333333 0.199613900595886
};
\addplot [line width=1.0pt, color0, dashed, mark=x, mark size=3, mark options={solid}, forget plot]
table {%
4.93333333333333 0.355194351174054
};
\addplot [line width=1.0pt, color0, opacity=1, forget plot]
table {%
5.89333333333333 0.193087160190115
5.97333333333333 0.193087160190115
};
\addplot [line width=1.0pt, color0, dashed, mark=x, mark size=3, mark options={solid}, forget plot]
table {%
5.93333333333333 0.308500525658535
};
\addplot [line width=1.0pt, color1, opacity=1, forget plot]
table {%
1.02666666666667 0.607877002034449
1.10666666666667 0.607877002034449
};
\addplot [line width=1.0pt, color1, dashed, mark=x, mark size=3, mark options={solid}, forget plot]
table {%
1.06666666666667 0.721290629670846
};
\addplot [line width=1.0pt, color1, opacity=1, forget plot]
table {%
2.02666666666667 1.03084531641706
2.10666666666667 1.03084531641706
};
\addplot [line width=1.0pt, color1, dashed, mark=x, mark size=3, mark options={solid}, forget plot]
table {%
2.06666666666667 1.03172321181079
};
\addplot [line width=1.0pt, color1, opacity=1, forget plot]
table {%
3.02666666666667 0.863414852226937
3.10666666666667 0.863414852226937
};
\addplot [line width=1.0pt, color1, dashed, mark=x, mark size=3, mark options={solid}, forget plot]
table {%
3.06666666666667 0.93104796123572
};
\addplot [line width=1.0pt, color1, opacity=1, forget plot]
table {%
4.02666666666667 0.795058300939824
4.10666666666667 0.795058300939824
};
\addplot [line width=1.0pt, color1, dashed, mark=x, mark size=3, mark options={solid}, forget plot]
table {%
4.06666666666667 0.817100830316506
};
\addplot [line width=1.0pt, color1, opacity=1, forget plot]
table {%
5.02666666666667 0.717854239335685
5.10666666666667 0.717854239335685
};
\addplot [line width=1.0pt, color1, dashed, mark=x, mark size=3, mark options={solid}, forget plot]
table {%
5.06666666666667 0.774625914265088
};
\addplot [line width=1.0pt, color1, opacity=1, forget plot]
table {%
6.02666666666667 0.679166041520125
6.10666666666667 0.679166041520125
};
\addplot [line width=1.0pt, color1, dashed, mark=x, mark size=3, mark options={solid}, forget plot]
table {%
6.06666666666667 0.693449862062485
};
\addplot [line width=1.0pt, color2, opacity=1, forget plot]
table {%
1.16 1.51906550709892
1.24 1.51906550709892
};
\addplot [line width=1.0pt, color2, dashed, mark=x, mark size=3, mark options={solid}, forget plot]
table {%
1.2 1.61624241294533
};
\addplot [line width=1.0pt, color2, opacity=1, forget plot]
table {%
2.16 1.58315095323648
2.24 1.58315095323648
};
\addplot [line width=1.0pt, color2, dashed, mark=x, mark size=3, mark options={solid}, forget plot]
table {%
2.2 1.5245203293067
};
\addplot [line width=1.0pt, color2, opacity=1, forget plot]
table {%
3.16 1.44480012162431
3.24 1.44480012162431
};
\addplot [line width=1.0pt, color2, dashed, mark=x, mark size=3, mark options={solid}, forget plot]
table {%
3.2 1.43037673679783
};
\addplot [line width=1.0pt, color2, opacity=1, forget plot]
table {%
4.16 1.10933203367222
4.24 1.10933203367222
};
\addplot [line width=1.0pt, color2, dashed, mark=x, mark size=3, mark options={solid}, forget plot]
table {%
4.2 1.15283550445182
};
\addplot [line width=1.0pt, color2, opacity=1, forget plot]
table {%
5.16 0.955722530896501
5.24 0.955722530896501
};
\addplot [line width=1.0pt, color2, dashed, mark=x, mark size=3, mark options={solid}, forget plot]
table {%
5.2 1.00385877914995
};
\addplot [line width=1.0pt, color2, opacity=1, forget plot]
table {%
6.16 0.864883564441296
6.24 0.864883564441296
};
\addplot [line width=1.0pt, color2, dashed, mark=x, mark size=3, mark options={solid}, forget plot]
table {%
6.2 0.893914101868282
};
\end{axis}

\node at ({$(current bounding box.south west)!0.5!(current bounding box.south east)$}|-{$(current bounding box.south west)!0.98!(current bounding box.north west)$})[
  anchor=north,
  text=black,
  rotate=0.0
]{ };
		\begin{customlegend}[
legend entries={best case,base,worst case,guessing},
legend cell align=left,
legend style={at={(13.87,5.37)}, anchor=north east, draw=white!80.0!black, font=\footnotesize,fill opacity=0.5, draw opacity=1,text opacity=1}]
    \addlegendimage{area legend,black,fill=black, fill opacity=1}
    \addlegendimage{area legend,color0,fill=color0, fill opacity=1}
    \addlegendimage{area legend,color1,fill=color1, fill opacity=1}
    \addlegendimage{area legend,color2,fill=color2, fill opacity=1}
\end{customlegend}
\end{tikzpicture}
	}
	\caption[Evaluation Results for Different Scenarios]{Evaluation Results for Different Scenarios ($n=200$).}
	\label{fig:trialCases}
\end{figure}

In ideal conditions, the algorithm is able to successfully locate two sources in virtually every trial. However, with an increasing number of sources $S$, the \gls{mae} still increases, despite the lack of any disturbances other than different sources. Also, only in ideal conditions does the localisation error continuously increase with every new source that is being added. For the base and worst case scenario, at some point the localisation decreases when adding more sources. This effect is best demonstrated in the case, where location estimates were only guessed. When guessing location estimates, more sources lead to a lower localisation error, as each estimate has a better chance of being close to one of the original source locations. This is the reason, why the \gls{mae} decreases for the worst case when adding additional sources for $S>4$ and for the base case, when adding a seventh source. Further, the results for the base case seem to validate the assessment of a rather conservative choice of default parameters, as the \gls{mae} of the base case is closer to the best case than the worst case for any $S$. Last, the performance of the worst case scenario converges with the results of the guessing trial with increasing $S$. For $S=7$, for example, the difference in error is only about $0.1$~m. However, this could be due to the relatively small room of 6~m $\times$ 6~m, as the guessing estimates trial resulted in a mean absolute error of below $1$~m. For larger rooms, the gap between the worst case scenario and guessing estimates could be wieder than it is the case in these results.