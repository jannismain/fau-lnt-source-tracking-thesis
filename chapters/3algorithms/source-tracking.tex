\section{Source Tracking}
\label{sec:algSrcTrack}

To integrate data from prior steps into each \gls{em} iteration, recursive variants of the \gls{em} algorithm have been developed. These will be called \gls{rem} algorithms in the upcoming sections.

In \cite{Schwartz2014}, two different \gls{rem} algorithms for speaker tracking are proposed. One is based on \citeauthor{Titterington1984}'s \gls{rem} algorithm\ and will be referenced as \acrshort{trem}, whereas the other one is based on \citeauthor{Cappe2009}'s \gls{rem} algorithm and will be referenced as \acrshort{crem}.
As the derivation of these algorithms is too involved to be repeated here, only the resulting algorithm for the specific application of source tracking shall be presented below.
The main difference from the static location estimation is, that where the results have been summed up over all time-bins before, now each time-bin is analysed seperately and the estimation results of the prior time-bin will be used as prior probability of the next iteration. Another difference is, that the E- and M-Step are combined into one single recursion update, which means there are no more iterations over the E- and M-Step but rather recursions over the time-bins now. Therefore the iteration index $(l)$ is replaced by the time index $(t)$.

% Left out Calculation of FIM here

Recursive estimation of the Gaussian component weights is the same for both algorithms. First, reusing \eqref{eq:responsibility} for recursive updates over time $(t)$ instead of \gls{em} iterations $(l)$ yields
\begin{equation}
    \muRlast \triangleq \frac{\pdffuncR}{\sum_{\vect{p}}\pdffuncR}
\end{equation}

Similar to \eqref{eq:m_step_psi}, a recursive estimate of $\psip$ is given by
\begin{equation}
    \psi_{\bm p}^{(t)}\triangleq\sum_k \frac{\muRlast}{K}
\end{equation}

Finally, using the result of the algorithm derivation based on the \gls{fim} in \cite{Schwartz2014}, the recursive update step is given by

\begin{equation}
\label{eq:crem-psi}
    \psiRnow = \psiRlast +\gamma_t(\bm\psi^{(t)} -\psiRlast ).
\end{equation}
\renewcommand{\*}{\cdot}

The two algorithms, however, do differ in the recursive update of the variance

\paragraph{\glsentryshort{trem}}
\begin{align}
\label{eq:trem-var}
    \sigma_R^{2,(t)} &= \sigma_R^{2,(t-1)}+\gamma_t\frac{1}{K\*\sum_{\bm p}\psiRPlast} \sum_{k,\bm p}\mulast\*\frac{1}{M}\sum_{m}|\phi_m(t,k)-\tilde\phi_m^k(\bm p)|^2-\sigma_{m,R}^{2,(t-1)}\\
                     &= \sigma_R^{2,(t-1)}+\gamma_t\frac{\sum_{k,\bm p}\mulast}{K\*\sum_{\bm p}\psiRPlast} \*\Bigg ( \frac{1}{M}\sum_{m}|\phi_m(t,k)-\tilde\phi_m^k(\bm p)|^2-\sigma_{m,R}^{2,(t-1)}\Bigg ).
\end{align}

\paragraph{\glsentryshort{crem}}
\begin{align}
\label{eq:crem-var}
    \sigma_R^{2,(t)} = \sigma_R^{2,(t-1)}\cdot\frac{\sum_{\bm p}\psiRPlast}{\sum_{\bm p}\psiRPnow}&+\gamma_t\Bigg ( \frac{\sum_{k,\bm p,m}\mulast|\phi_m(t,k)-\tilde\phi_m^k(\bm p)|^2}{K \* M \* \sum_{\bm p}\psiRPnow}\\
    &-\sigma_R^{2,(t-1)}\frac{\sum_{\bm p}\psiRPlast}{\sum_{\bm p}\psiRPnow} \Bigg ).
\end{align}

\begin{algorithm}[H]\label{alg:trem}
\caption{\acrshort{trem} algorithm for source tracking}
\begin{algorithmic}
\State \textbf{set} $\tilde\phi^k_m(\vect{p})$ using \eqref{eq:phi_tilde}
\For{$t=1$ \textbf{to} $T$}
\State obtain $z_m^1(t,k), z_m^2(t,k)\ \forall\ k,m$
\State calculate $\phi_m(t,k)$ using \eqref{eq:phi_gmm}
\State calculate $\psi^{(t)}_{s\vect{p},R}$ using \eqref{eq:m_step_psi}
\State calculate $\sigma^{2,(t)}_{s,R}$ using \eqref{eq:trem-var}
\EndFor
\end{algorithmic}
\end{algorithm}

\begin{algorithm}[H]
\label{alg:crem}
\caption{\acrshort{crem} algorithm for source tracking}
\begin{algorithmic}
\State \textbf{set} $\tilde\phi^k_m(\vect{p})$ using \eqref{eq:phi_tilde}
\State \textbf{initialize} $\psi_{\bm p, R}^{(0)}$ and $\sigma_{s,R}^{2,(0)}$
\For{$t=1$ \textbf{to} $T$}
\State obtain $z_m^1(t,k), z_m^2(t,k)\ \forall\ k,m$
\State calculate $\phi_m(t,k)$ using \eqref{eq:phi_gmm}
\State calculate $\psi^{(t)}_{s\vect{p},R}$ using \eqref{eq:m_step_psi}
\State calculate $\sigma^{2,(t)}_{s,R}$ using \eqref{eq:crem-var}
\EndFor
\end{algorithmic}
\end{algorithm}

