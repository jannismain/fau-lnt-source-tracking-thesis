\bigskip

In this chapter, the probabilistic model of the environment has been presented. A \gls{gmm} has been used to model the \gls{prp} readings for each grid point as a Gaussian component and aggregate all components into a mixture. The parameters of this mixture are then estimated using the \gls{em} algorithm, which has been derived to estimate the most likely source locations given the \gls{prp} observations. The result of the \gls{em} algorithm is a vector of Gaussian component weights $\hat{\bm\psi}=\bm\psi^{(L)}$, which contains a weight for each grid point $\bm p$ and source $s$ that indicates, how likely position $\bm p$ is the original location of source $s$. Finding the $S$ highest values of $\hat{\bm\psi}$ will yield the estimated source positions $\hat{\bm p}_s\ \forall\ s$. A deviation from the algorithm in \cite{Schwartz2014} has been proposed, that reduces the dimension of the estimation parameters and removes the requirement of a prior knowledge of $S$ during the \gls{em} iterations. In the end, it has briefly been laid out, how the basis of the source localisation algorithm can be used to recursively estimate multiple source positions in a dynamic environment and the parameter updates for two recursive adaptions of the \gls{em} algorithm, namely \gls{trem} and \gls{crem}, have been shown.