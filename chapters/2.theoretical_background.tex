\chapter{Theoretical Background}
\label{chap:theory}

\newacronym{gmm}{GMM}{Gaussian Mixture Model}
\newacronym{em}{EM}{Expectation-Maximisation} 

In this section, the main theoretical concepts needed for audio source localisation and tracking, as it is implemented here, are laid out and put into context. First, the signal model is defined and it is shown, which features of the signal are exploited to estimate the location of it's origin. As we are simulating the received signal, instead of using recordings, the approach used for simulating a received signal, that is influenced by room acoustics and noise, is shown and it's limitations are discussed. Next, the \gls{gmm} is introduced as a mathematical representation of the possible locations of the audio sources, that are to be estimated. Last, the \gls{em} algorithm is explained and it is shown, how it can be used to estimate and optimise the parameters of the \gls{gmm} to yield the most likely source locations.

\section{Signal Model}
\label{sec:signal}
Received signal:
\begin{equation}
	z_m^i(t,k)=\sum_{s=1}^{S}a_{sm}^i(t,k)\cdot v_s(t,k)+n_m^i(t,k)
\end{equation}
Acoustic transfer function:
\begin{equation}
	a_{sm}^i(t,k)\approx\frac{1}{\|\mathbf{p}_s-\vect{p}_m^i\|}\cdot\exp{\left(-j\frac{2\pi k}{K}\frac{\tau^i_{sm}}{T_s}\right)}
\end{equation}
Signal travel time:
\begin{equation}
	\tau^i_{sm}=\frac{1}{c}\cdot\left(\|\vect{p}_s-\vect{p}_m^i\|\right)
\end{equation}

\section{Simulation of Room Acoustics}
\label{sec:simulation}
To simulate the setup described in \ref{sec:setup}, the image method for simulating small-room acoustics is used \cite{Allen1979}.

\section{\acrfull{gmm}}
\label{sec:gmm}
\begin{equation}
	\phi(t,k)\sim\sum_{s,\vect{p}}\psi_{s\vect{p}}\cdot\big(\phi(t,k);\tilde\phi^k(\vect{p}),\Sigma_s\big)
\end{equation}

\section{\acrfull{em}}
\label{sec:em}
\chapter{Theoretical Background}
\label{chap:2theory}

\chapter{Theoretical Background}
\label{chap:2theory}

In this section, the main theoretical concepts needed for speech source localisation and tracking, as it is implemented here, are laid out and put into context. First, the signal model is defined and it is shown, which features of the signal are exploited to estimate the locations of their origin. As we are simulating the received signal, the image-source model for simulating a received signal influenced by room acoustics and noise, is shown and it's limitations are discussed. Next, the \gls{gmm} is introduced as a probabilistic model of the possible locations of the audio sources, that are to be estimated. Last, the \gls{em} algorithm is explained and it is shown, how it can be used to estimate the parameters of a \gls{gmm} to yield the most likely source locations, with respect to the underlying probabilistic model.
\section{Signal Model}
\label{sec:signal}

\begin{figure}[!b]
\centering
    \includegraphics[scale=0.7]{data/figures/signal2}
    \caption{Direct Propagation Paths for Source $s$ and Microphone Pair $m$}
    \label{fig:signal}
\end{figure}

The basic configuration, as depicted in Figure \ref{fig:signal}, consists of two microphones that constitute one microphone pair. The received signal $x^i_m$ at microphone $i$ of pair $m$ can be described as a linear combination of the acoustic transfer function $a^i_{sm}$, multiplied bin-wise with the source signal $v_s$, and additive white Gaussian noise $n^i_m$ in the \acrfull{stft} domain
\begin{equation}
	x_m^i(t,k)=\sum_{s=1}^{S}a_{sm}^i(t,k)\cdot v_s(t,k)+n_m^i(t,k),
	\label{eq:x}
\end{equation}

where $t\in\{1,\dots,T\}$ denotes the time-index or time-bin, $k\in\{1,\dots,K\}$ denotes the frequency-index or frequency-bin, $(m,i)$ describe microphone $i\in\{1,2\}$ of microphone pair $m\in\{1,\dots,M\}$ and $s\in\{1,\dots,S\}$ denotes the source index. For the experimental part in \autoref{chap:experiments}, the \gls{rir} is used as the time-domain equivalent to the acoustic transfer function. The direct path of the transfer function, that part corresponding to the line-of-sight, can be described as
\begin{equation}
	a_{sm}^i(t,k)\approx\frac{1}{\|\bm p_s-\bm p_m^i\|}\cdot\exp{\left(-\iota\frac{2\pi k}{K}\frac{\tau^i_{sm}}{T_s}\right)},
	\label{eq:acoustic_transfer_function}
\end{equation}

where $T_s$ is the sampling period of source signal $v_s$, $\ps$ is the position of source $s$, $\bm p^i_m$ is the position of microphone $(m,i)$, $\|\bm p_s-\bm p_m^i\|$ is the Euclidean distance between the source and the microphone position and $j$ is the imaginary unit. The signal travel time $\tau^i_{sm}$ between source position $\bm p_s$ and microphone position $\bm p^i_m$ is determined by the sound velocity $c$
\begin{equation}
	\tau^i_{sm}=\frac{\left(\|\bm p_s-\bm p_m^i\|\right)}{c},
\end{equation}

which is determined by both the type and properties of the medium the signal as travelling in as well as the temperature. To estimate the location of a source, we solve for the \gls{tdoa} $\Delta\tau_{sm}$, which we define as
\begin{equation}
    \Delta\tau_{sm}=\tau^2_{sm}-\tau^1_{sm}=\frac{\|\bm p_s-\bm p_m^2\|-\|\bm p_s-\bm p_m^1\|}{c},
\end{equation}

assuming planar wave fronts impinging on the microphones according to the far-field approximation of sound signals. From the \gls{tdoa}, the \gls{doa} can then be inferred with
\begin{equation}
    \text{DOA}=\arccos\left (\frac{c\cdot \Delta\tau_{sm}}{d_m}\right ),
\end{equation}

where arccos is the inverse of the cosine-function and $d_m$ is the Euclidean distance of the position of the two microphones $\bm p_m^1$ and $\bm p_m^2$ of microphone pair $m$
\begin{equation}
    d_m=\| \bm p_m^1-\bm p_m^2\|.
\end{equation}

The direct path of the transfer function multiplied with the source signal contains the original phase information that is needed to estimate the position of the source. The remaining part of the received signal (the reverberant tail of the transfer function multiplied with the source signal) distorts this information and will decrease the location estimation performance. 

%TODO: Klären, wann index s notwendig und wann nicht
Following \cite{Schwartz2014}, the \gls{prp} $\phi_{m}(t,k)$ of the two signals $x_{m}^1$ and $x_{m}^2$ received at each microphone pair $m$ will be used as the feature to be used for the purpose of localisation
\begin{equation}
    \phi_{m}(t,k)=\frac{x^2_{m}(t,k)}{x^1_{m}(t,k)}\cdot \left |\frac{x^1_{m}(t,k)}{x^2_{m}(t,k)}\right |.
\label{eq:prp}
\end{equation}

To understand the link of \gls{prp} and \gls{doa}, let's solve the equation for two received signals in a noiseless environment (i.e., $n^i_{m}(t,k)=0$) by inserting the definition of the received signal \eqref{eq:x} into \eqref{eq:prp} and reducing the resulting equation
%\begin{equation}
%    \frac{x^2_{sm}}{x^1_{sm}}=\frac{v_{sm}\cdot a^2_{sm}}{v_{sm}\cdot a^1_{sm}}=\frac{\|\bm p_s-\bm p_m^1\|\cdot\exp{\left(-j\frac{2\pi k}{K}\frac{\tau^2_{sm}}{T_s}\right)}}{\|\bm p_s-\bm p_m^2\|\cdot\exp{\left(-j\frac{2\pi k}{K}\frac{\tau^1_{sm}}{T_s}\right)}}=\exp{\left ( j\frac{2\pi k}{K}\frac{(\tau^1-\tau^2)}{T_s}\right )}
%\end{equation}

\begin{equation}
    \phi_{m}(t,k)=\exp{\left ( -j\frac{2\pi k}{K}\frac{\tau_{m}^1-\tau_{m}^2}{T_s}\right )}.
\end{equation}

The amplitude has been eliminated by multiplication with $|x^1(t,k)|\ /\ |x^2(t,k)|$. The \gls{prp} feature can therefore be interpreted as the effect the difference in signal travel time of different source signals has on the phase difference that can be calculated at each microphone pair.

\paragraph{W-disjoint Orthogonality}
The sources are assumed to exhibit W-disjoint orthogonality in the \gls{stft} domain \cite[p.~393]{Schwartz2014}, \cite{Rickard2006}. This means that the source signals do not overlap in the \gls{stft} domain. For two sources $s\in\{1,2\}$ and their respective source signals $v_s(t,k)$, this assumption can be formalised as
\begin{equation}
    v_1(t,k)\cdot v_2(t,k)=0~\forall~t,k.
\end{equation}

In reality, this assumption does not fully hold true but is approximated by speech signals, as these are considered to be sparse in the time-frequency domain, meaning most parts of the signal are equal to zero. For the mixture of multiple source signals it is then further assumed, that each time-frequency bin is dominated by only one source, which allows for spatial seperation of the received signal's components and subsequent source localisation. We will see that with increased reverberation and number of sources simultaneously simulated, this assumption is challenged, as the received signal will be less and less sparse.
\section{Gaussian Mixture Model (GMM)}
\label{sec:gmm}
% univariate Gaussian
In it's univariate form, a Gaussian or normal distribution is defined by
\begin{equation}
	\gaussian{x\vert\mu,\sigma}=\frac{1}{(2\pi\sigma^2)^{1/2}}\exp\left\{-\frac{1}{2\sigma^2}(x-\mu)^2\right\},
\end{equation}
where $\mu$ is called its \textit{mean} and $\sigma^2$ is called its \textit{variance}.
%This distribuion satisfies the requirements $\int_{-\infty}^{\infty}\gaussian{x\vert\mu,\sigma^2}\text{d}x=1$ and $\gaussian{x\vert\mu,\sigma^2}>0\ \forall\ x\in\mathbb{R}$, which qualifies the Gaussian distribution as a probability distribution.
The multivariate Gaussian distribution for an input vector $\bm x\in\mathbb{R}^D$ is defined by
\begin{equation}
	\gaussian{\bm x\vert\bm\mu,\bm\Sigma}=\frac{1}{(2\pi)^{D/2}}\frac{1}{\bm\Sigma^{1/2}}\exp\left\{-\frac{1}{2}(\bm x-\bm\mu)^{\text{T}} \bm\Sigma^{-1}(\bm x-\bm\mu)\right\}.
\end{equation}



% Complexity of multivariate Gaussians
Compared to the univariate form with only two free parameters, this \gls{pdf} is completely defined by its covariance matrix $\bm\Sigma\in\mathbb{R}^{D\times D}$ and its mean vector $\bm\mu\in\mathbb{R}^D$. As the covariance matrix is symmetric, is has $D(D+1)/2$ independent parameters (assuming it is positive definite). Adding the $D$ independent parameters of $\bm\mu$ results in $D(D+3)/2$ independent parameters that define a multivariate Gaussian distribution. 

%TODO: Introduce complex Gaussian notation!
The complex variant of the Gaussian distribution is defined by
\begin{equation}
    \mathcal{N}^c(\bm z,\Gamma)=\frac{1}{\pi^N\cdot\text{det}\ \Gamma}\exp\{ -\bm z^*\Gamma^{-1}\bm z \},
\end{equation}
where $\bm z\in\mathbb{C}^N$, $\Gamma$ is the complex covariance matrix, $\text{det}\ \Gamma$ is the determinant of $\Gamma$, $\bm z^*$ denotes the complex conjugate of $\bm z$ and $\Gamma^{-1}$ the inverse of $\Gamma$. As shown in \cite{Gallager2008}, for the density of $m$ independent circular-symmetric Gaussian random variables, this can be written using the eigenvalues $\lambda_j$ and eigenfunction $\bm q_j$ of the covariance matrix
\begin{equation}
\label{eg:complexGaussianDef}
    \mathcal{N}^c(\bm z,\lambda_j)=\prod_{j=1}^n\frac{1}{\pi\lambda_j}\exp(-|z,q_j|^2\lambda_j^{-1}).
\end{equation}
This formulation will be used when modelling the \gls{prp} using a \gls{gmm}.\\

% from single gaussian to mixture model
Despite possessing this many degrees of freedom, the Gaussian distribution is limited in a sense, that it has only one maximum (\textit{unimodality}) and therefore cannot adequately model multimodally distrbuted data. To overcome this limitation, multiple Gaussians can be combined into what is called a \acrlong{gmm}

% the GMM
\begin{equation}
	p(\bm x)=\sum^J_{j=1}\psi_j\cdot\gaussian{\bm x\vert\bm\mu_j,\bm\Sigma_j},
\end{equation}

where $\psi_j$ is the weighting factor or \textit{mixing coefficient} of each gaussian component $\gaussian{\bm x\vert\bm\mu_j,\bm\Sigma_j}$ of the \gls{gmm} with $0\leq\psi_j\leq 1$ and $\sum_{j=1}^J \psi_j=1$. So in addition to $\bm\mu=\{\bm\mu_1,\dots,\bm\mu_J\}$ and $\bm\Sigma=\{\bm\Sigma_1,\dots,\bm\Sigma_J\}$, each a concatination of the parameters of the multivariate Gaussian for each component $j$, the \gls{gmm} has a third parameter $\bm\psi=\{\psi_1,\dots,\psi_J\}$. These can be calculated under the Maximum-Likelihood criterion as follows:

\begin{equation}
	\text{ln}\ p(\bm X\vert\bm\psi,\bm\mu,\bm\Sigma)=\sum_{n=1}^N \text{ln}\left\{\sum_{j=1}^J \psi_k\cdot\gaussian{\bm x_n\vert\bm\mu_j,\bm\Sigma_j} \right\},
\end{equation}

with $\bm X=\{\bm x_1,\dots,\bm x_N\}$. As a result of the presence of the sum inside the logarithm, the \gls{ml} solution for these parameters does not have a closed form solution \cite[p.113]{Bishop2006}. One way to solve these type of \gls{ml} problems is the \gls{em} algorithm presented in the following section.

\begin{itemize}
    \item inwiefern? -> latente zufallsvariablen
    \item evtl. Beispiel GMM
\end{itemize}


\section{Expectation-Maximisation (EM)}
\label{sec:em}

\paragraph{History}
Although it has been used for parameter estimation of mixture models as early as \citeyear{Newcomb1886} \cite{Newcomb1886}, the \gls{em} algorithm was formally described in a generalised form by \citeauthor{Dempster1977} in \citeyear{Dempster1977}, who also coined the name of the two-step procedure \cite{Dempster1977}. From then on, it has been used in various applications (many of them are described in \cite{McLachlan2008}). In the signal processing community, it has been successfully applied to tasks as diverse as emission tomography image reconstruction \cite{Shepp1982}, active noise cancelling with a single microphone \cite{Feder1989} and parameter estimation of hidden markov models \cite{Moon1996}. Outside of signal processing, the algorithm is also utilised in many machine learning applications, where it is often introduced as an extended form of the $k$-means algorithm for clustering \cite{Bishop2006}.

% Initialisation
The name \glsentrylong{em} goes back to the algorithm's distinct two steps that both rely on the output of the respective other step. Therefore, prior to the iteration over the E- and M-Step, the estimated parameter set $\bm \theta$ has to be initialised. This initialisation is done randomly in most cases, although one can also include prior knowledge of the estimated parameters by choosing certain initial values for $\theta$ accordingly.

\paragraph{Description} In general, the algorithm allows to determine \gls{ml} estimates or \gls{map} estimates, where only incomplete data is available \cite[p.1]{Dempster1977}. Incomplete data means that there is a \textit{hidden} or \textit{latent} variable $Z$, that cannot be observed directly but might be inferred by some observable variable $X$. The \textit{complete data} is given by combining the (discrete) \textit{hidden variable} and the observations.
%TODO: erklärender satz, warum latente variable diskret ist, auf diskreten fall beschränken...


\paragraph{Derivation}
The cost function of the maximisation problem is the following likelihood function
\begin{equation}
    p(\vect{X}\given{\theta})=\sum_\vect{z} p(\vect{X},\vect{Z}\given{\theta}).
\end{equation}

The assumption is, that maximising $p(\vect{X}\given{\theta})$ is difficult, but optimisation of the complete data likelihood function $p(\vect{X},\vect{Z}\given{\theta})$ is significantly easier. Next, a distribution $q(\vect{Z})$ over the latent variables is defined, which allows, following \cite[p.450]{Bishop2006}, for the decomposition

\begin{equation}\label{eq:decomposition}
    \ln p(\vect{X}\given{\theta})=\mathcal{L}(q,\theta)+\text{KL}(q\|p),
\end{equation}

where the summands are defined as
\begin{align}
    \mathcal{L}(q,\theta)&=\sum_\vect{Z}q(\vect{Z})\ln\left\{\frac{p(\vect{X},\vect{Z}\given{\theta})}{q(\vect{Z})}\right\}\label{eq:L},\\
    \text{KL}(q\|p)&=-\sum_{\vect{Z}}q(\vect{Z})\ln\left\{\frac{p(\vect{Z}\given{\vect{X},\theta})}{q(\vect{Z})} \right\}\label{eq:KL}.
\end{align}

This decomposition can be verified by using the multiplication formula for probabilities, which directly results from the definition of conditional probabilities
\begin{align}
\label{eq:defProbCond}
    p(A\given{B})&=\frac{p(A,B)}{p(B)},\\
\label{eq:defProbProductRule}
    p(A,B)&=p(A\given{B})\cdot p(B).
\end{align}

Applying \eqref{eq:defProbProductRule} to $p(\vect{X},\vect{Z})$ in \eqref{eq:L} and using the product rule for logarithms $\ln(a\cdot b)=\ln(a)+\ln(b)$ yields

\begin{align}
    \mathcal{L}(q,\theta)&=\sum_{\vect{Z}}q(\vect{Z})\ln\left\{\frac{p(\vect{Z}\vert\vect{X},\theta)\cdot p(\vect{X}\given{\theta})}{q(\vect{Z})}\right\}\\
\label{eq:Lexpanded}
    &=\sum_{\vect{Z}}q(\vect{Z})\ln\left\{\frac{p(\vect{Z}\given{\vect{X},\theta})}{q(\vect{Z})} \right\}+\sum_{\vect{Z}}q(\vect{Z})\ln\{p(\vect{X}\given{\theta})\}.
\end{align}

The first summand is equal to $-\text{KL}(q\|p)$, therefore it cancels out when inserting \eqref{eq:Lexpanded} back into \eqref{eq:decomposition}. The second summand can be simplified making use of the fact, that $q(\vect{Z})$ is a probability distribution and therefore $\sum_{\vect{Z}}q(\vect{Z})=1$ holds true
%TODO: restructure this section!
\begin{align}
    \mathcal{L}(q,\theta)+\text{KL}(q\|p)&=\sum_{\vect{Z}}q(\vect{Z})\ln p(\vect{X}\given{\theta})-\text{KL}(q\|p)+\text{KL}(q\|p)\\
    &=\ln p(\vect{X}\given{\theta}).
\end{align}

From \eqref{eq:decomposition} it is apparent, that $\mathcal{L}(q,\theta)$ is a lower bound of $\ln p(\vect{X}\given{\theta})$, because from $q(\vect{Z})>0\ \forall\ \vect{Z}$ follows KL$(q\|p)\geq0$. In order to maximise the lower bound for a given $\theta^{(l-1)}$, we therefore need to set KL$(q\|p)=0$, so that $\mathcal{L}(q,\theta)=\ln p(X\given{\theta})$. In other words, to compute a lower bound, that is tight with the original likelihood function at $\theta^{(l-1)}$ (i.e. "touches" $\ln p(\vect{X}\given{\theta})$ in $\theta^{(l-1)}$), we set $\mathcal{L}(q,\theta)=\ln p(\vect{X}\given{\theta})$, which necessitates
\begin{align}
\label{eq:thatOtherFormulaAbove}
    \text{KL}(q\|p)&=0,\\
    -\sum_{\vect{Z}}q(\vect{Z})\ln\left\{\frac{p(\vect{Z}\given{\vect{X},\theta})}{q(\vect{Z})} \right\}&=0.
\label{eq:thatFormulaAbove}
\end{align}

Choosing
\begin{equation}
    q(\vect{Z})=p(\vect{Z}\given{\vect{X}},\theta)
\label{eq:q}
\end{equation}
satisfies \eqref{eq:thatFormulaAbove}. Inserting \eqref{eq:q} into \eqref{eq:L} and using $\ln(\frac{x}{y})=\ln{x}-\ln{y}$, the lower bound is given by
\begin{align}
    \mathcal{L}(q, \theta)&=\sum_\vect{Z}p(\vect{Z}\given{\vect{X}, \theta^{(l-1)}})\ln p(\vect{X},\vect{Z}\given{\theta})-\sum_\vect{Z}p(\vect{Z}\given{\vect{X}, \theta^{(l-1)}})\ln p(\vect{Z}\given{\vect{X},\theta^{(l-1)}})\\
    &=\Q+\text{const},
\end{align}

with
\begin{equation}
\label{eq:e-step}
    \Q=\sum_\vect{Z}p(\vect{Z}\given{\vect{X}, \theta^{(l-1)}})\ln p(\vect{X},\vect{Z}\given{\theta}).
\end{equation}
    
Computing this lower bound constitutes the E-Step. The M-Step is to find a new value $\theta^{(l)}$, that maximises this lower bound
\begin{equation}
\label{eq:m-step}
    \theta^{(l)} = \argmax_{\theta}\Q.
\end{equation}

\begin{figure}[!hb]
\label{fig:em}
    \centering
    \includegraphics[scale=1]{data/figures/em-Q4}
    \caption[Short Title for EM-Algorithm]{EM-Algorithm as iterative lower-bound optimisation}
    \medskip
    \small
    Here in a new line a long description about the figure, in a smaller text
\end{figure}

To summarise, the goal of the E-Step is to find a lower bound for the original, incomplete data likelihood function $p(\vect{X}\vert\theta)$. This lower bound $\mathcal{L}(q,\bm\theta)$ is choosen to be equal to the original likelihood function in $\theta^{(l-1)}$, which is done by selecting $q(\vect{Z})$ to be the posterior probability of the latent variable $\vect{Z}$ given the observation $\vect{X}$. This gives rise to the $\mathcal{Q}$-function, which, in the M-Step, is maximised with respects to the estimated parameter set $\bm\theta$. The result is a new value $\bm\theta^{(l)}$, for which a new lower-bound can be found. This procedure is repeated iteratively either until some fixed number of iterations are reached or some convergence criterion is met. For a full account of the convergence of \gls{em}, see \cite{Wu1983}.\alt{Convergence criteria for the \gls{em} algorithm are shown in \cite{Wu1983}.} It is interesting to note, that the M-Step only states \textit{that} the result of the E-Step is to be maximised, but not \textit{how} this is to be done. Rather than a specific solution for a certain type of problems, the \gls{em} algorithm is therefore more of a template, that can be applied to a variety of different problems, each of them required to individually derive the concrete steps necessary to obtain a lower-bound for the incomplete data log likelihood function and find it's maximum.

\begin{SCfigure}
\centering
\includegraphics{data/figures/em-flowchart}
\caption{Steps of the \glsentryshort{em} algorithm}
\end{SCfigure}

\paragraph{Limitations}
One existing limitation of the \gls{em} algorithm is the possibility of a slow convergence rate, which increases the computational complexity as more iterations have to be carried out before convergence is reached. In this case, the convergence threshold can be increased or a fixed number of iterations can be carried out, although this means that the solution is no longer optimal in the \gls{mmse} sense. Another limitation is, that the algorithm is susceptible to local optima for more than one target parameter, meaning that the determined solution is not necessarily the best solution. This can be alleviated by running the algorithm multiple times with a different, random initialisations and choose the best outcome, a procedure known as \textit{random-restart hill climb}. Another method to circumvent getting stuck in local optima is \textit{simulated annealing}, which is incorporated into an \gls{em} template in \cite{Guo2007}.
%TODO: define "best outcome"
%TODO: Describe simulated annealing on a high-level (1 sentence)
%TODO: "more than one target parameter" ist unpräzise.. umformulieren!
%TODO: define "convergence threshold" mathematically


TODO: Zusammenfassung von Kapitel 2

%As KL$(q\|p)$ is the Kullback-Leibler divergence between $q(\vect{Z})$ and $p(\vect{Z}\given{\vect{X},\theta})$, it also satisfies KL$(p\|q)\geq 0$ with equality only if $q(\vect{Z})=p(\vect{Z}\given{\vect{X},\theta})$. This means, that, using \eqref{eq:decomposition}, we can state, that $\mathcal{L}(q,\theta)$ is a lower bound for the original likelihood function $\ln p(\vect{X}\given{\theta})$. As the 

%%%%%%%%%%%% OLD STUFF %%%%%%%%%%%%%%%


%\paragraph{Derivation by Andrew Ng, CS229}
%The goal is to maximise the log-likelihood function that is given by
%\begin{equation}\label{eq:likelihood}
%    l(\theta)=\sum_i\log p(x_i;\theta)=\sum_i\log\sum_{z_i} p(x_i,z_i;\theta).
%\end{equation}
%
%Now we assume, that maximising the probability of the observed variable $X$ with respect to the parameter $\theta$ is difficult, but doing this for the probability of the complete data $p(X,Z;theta)$, given by the joint distribution of observed and hidden variables, is easier.
%
%\begin{align}
%    \sum_i \log p(x_i;\theta)&=\sum_i \log \sum_{z_i} p(x_i, z_i;\theta)\\
%          &=\sum_i \log \sum_{z_i} Q_i(z_i)\frac{p(x_i, z_i;\theta)}{Q_i(z_i)}\label{eq:introduce-Q}\\
%          &\geq \sum_i\sum_{z_i}Q_i(z_i)\log\frac{p(x_i,z_i;\theta)}{Q_i(z_i)}\label{eq:lower-bound}
%\end{align}
%
%In \eqref{eq:introduce-Q} we introduced a probability function of the \textit{hidden variable} $Q_i(z_i)$ with the properties $Q_i(z_i)\geq 0\ \forall\ z_i$ and $\sum_{z_i}Q_i(z_i)=1$. This allowed us to use the definition for expectation
%\begin{equation}
%    \sum_{z_i} Q_i(z_i)\frac{p(x_i, z_i;\theta)}{Q_i(z_i)}=E\left\{\frac{p(x_i, z_i;\theta)}{Q_i(z_i)}\right\}
%\end{equation}
%
%and finally use Jensen's inequality
%\begin{align}
%    \text{when\ }f(x) \text{\ is concave:\ }& f(E\{x\}) \geq E\{f(x)\},\label{eq:jensen-concave}\\
%    \text{when\ }f(x) \text{\ is convex:\ }& f(E\{x\}) \leq E\{f(x)\},\label{eq:jensen-convex}
%\end{align}
%
%given that the logarithm is a concave function, which can be shown by
%\begin{equation}
%    \frac{\partial}{\partial x}\log(x)=-\frac{1}{x^2} < 0 \ \forall\ x\in\mathbb{R}^+.
%\end{equation}
%
%Now we have a lower bound \eqref{eq:lower-bound} for the likelihood function \eqref{eq:likelihood}. To define Q we set the constraint, that the lower bound should be as close to the original likelihood function as possible. Therefore, we set the lower bound equal to the original likelihood function for the given parameter $\theta$, which means, that Jensen's inequality needs to be true for the equal case. Combining \eqref{eq:jensen-concave} and \eqref{eq:jensen-convex} shows, that $f(E[x]) = E[f(x)]$ is only true when $f(x)$ is neither concave nor convex, which is only the case for a constant value $f(x) = c\ \forall\ x$. That means
%\begin{equation}
%    \frac{p(x_i,z_i;\theta)}{Q_i(z_i)}=c,
%\end{equation}
%
%which is satisfied, when
%\begin{equation}
%    Q_i(z_i)\propto p(x_i,z_i;\theta).
%\end{equation}
%
%Using the probability distribution property $\sum_{z_i}Q_i(z_i)=1$, we can write
%\begin{align}
%    Q_i(z_i)&=\frac{p(x_i,z_i;\theta)}{\sum_z p(x_i,z;\theta)}\\
%            &=frac{p(x_i,z_i;\theta)}{p(x_i;\theta)}\\
%            &=p(z_i\mid x_i;\theta),
%\end{align}
%
%which is the posterior probability of the hidden variable $z$ given $x$.
%
%In a general form, the two steps of the \gls{em} algorithm are defined as
%
%\subsubsection*{E-Step}
%\begin{equation}\label{eq:e_step_general}
%	Q(\theta,\hat\theta^{(l)})=E_{\hat\theta^{(l)}}\left\{ \log{f_{Y}(y;\theta)\mid z} \right\},
%\end{equation}
%
%\subsubsection*{M-Step}
%\begin{equation}\label{eq:m_step_general}
%	\hat\theta^{(l+1)}=\arg \max_\theta Q\left ( \theta,\hat\theta^{(l)}\right ).
%\end{equation}
%
%%The $Q$-function used in the E-Step can be interpreted as a lower bound of the log-likelihood function to be maximised. To calculate this lower bound, \textit{Jensen's inequality} is used, which states for a concave function $f(x)$ that
%
%The expectation operator is defined as
%\begin{equation}\label{eg:expectation-definition}
%    E\{f(x)\}=\sum_x f(x)P(x),
%\end{equation}
%
%Bringing this all together, we are now able to determine a lower bound for the log likelihood function
%\begin{equation}
%    \log\left(\sum_x f(x)P(x)\right) \leq \sum_x \log f(x)P(x),
%\end{equation}
\bigskip

In this chapter, the signal model has been defined and the \gls{prp} has been identified as a measure of the \gls{tdoa} in the frequency domain. Next, the Gaussian distribution has been introduced and definitions have been presented for its real, its complex and its multivariate form. The Gaussian was mixture model was motivated by the limitation of a single Gaussian to only model unimodally distributed data and has been defined as a weighted linear combination of Gaussian components. The weighting factor of each components together with the variance is the parameter set to be estimated using the \gls{em} algorithm, which is able to determine the \gls{ml} estimates for these parameters despite incomplete data by introducing a latent variable that models the missing information. It has been shown, how the \gls{em} algorithm can be understood as a iterative lower-bound optimisation of the incomplete data log-likelihood function, where the lower-bound is considerably easier to maximise for distributions of the exponential family, than the original likelihood function. In the end, limitations of the \gls{em} algorithm and strategies to alleviate these limitations have been discussed.