\chapter{Introduction}
TODO: Put an interesting introduction here!
\section{Application Scenarios}
Examples for possible applications of audio source localisation and tracking are:
\begin{itemize}
\setlength\itemsep{0cm}
	\item teleconferencing
	\item automated multimedia capture
	\item smart meeting rooms
	\item lecture theatres \cite{Lehmann2007}
\end{itemize}

\section{Overview of Existing Literature}
A preliminary exploration of the acoustic source localisation and tracking literature shall provide an understanding of the progress, the current state-of-the-art and the most promising ways forward in this diverse research field. 

% No subsections here will provide more flexibility in structuring this chapter
%\subsection{Acoustic source localisation}
%\subsection{Acoustic source tracking}

\section{Research Problem}
\section{Setting Constraints}
This thesis is approaching the problem of source localisation and tracking using the \gls{em} algorithm to estimate the parameters of a \gls{gmm}. Other methods, like particle filters or Kalman filters, that are often used in this context and provide promising results (see \cite{Lehmann2007} for source tracking using a particle filter and \cite{Gannot2012} for source tracking using a Kalman filter), are not subject of this thesis.
\begin{itemize}
	\item Number of speakers a-priori known
	\item ?
\end{itemize}
\section{Structure of Thesis}
	In the beginning, chapter \ref{chap:theory} will introduce the theoretical concepts, that the algorithms for location estimation and tracking reviewed in this thesis are based upon. Chapter \ref{chap:algorithms} then will formally define these algorithms. In chapter \ref{chap:implementation} the implementation of these algorithms in \matlab is shown and test scenarios to evaluate these implementations are defined (all relevant code is included in \ref{sec:appCode}). The results of these tests are presented and discussed in chapter \ref{chap:results}. Finally, the findings of this thesis are summarized in chapter \ref{chap:concl} and a conclusion towards the applicability of these findings to the general performance of the reviewed algorithms is drawn.
\section{Notation}
\newcommand{\vect}[1]{\mathbf{#1}}
For easier understanding of the mathematical terms included within this thesis, the norms of mathematical notation are followed. A bold font indicates vectors ($\vect{p}$, $\vect{v}$, \dots).

\begin{itemize}
	\item vectors
	\item matrices
	\item ?
\end{itemize}
