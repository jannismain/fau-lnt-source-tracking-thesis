\chapter{Introduction}
Here comes the introduction...
\section{Application scenarios}
Examples for possible applications of audio source localisation and tracking are teleconferencing, automated multimedia capture, smart meeting rooms, lecture theatres \cite{Lehmann2007}
\section{Overview of Existing Literature}
\section{Research Problem}
\section{Setting Constraints}
\begin{itemize}
	\item Number of speakers a-priori known
	\item ?
\end{itemize}
\section{Structure of Thesis}
	In the beginning, chapter \ref{chap:theory} will introduce the theoretical concepts, that the algorithms for location estimation and tracking reviewed in this thesis are based upon. Chapter \ref{chap:algorithms} then will formally define these algorithms. In chapter \ref{chap:implementation} the implementation of these algorithms in \matlab is shown and test scenarios to evaluate these implementations are defined. All relevant code is included in \ref{sec:appCode}. The results of these tests are presented and discussed in chapter \ref{chap:results}. Finally, the findings of this thesis are summarized in chapter \ref{chap:concl} and a conclusion towards the applicability of these findings to the general performance of the reviewed algorithms is drawn.
\section{Notation}
\newcommand{\vect}[1]{\mathbf{#1}}
For easier understanding of the mathematical terms included within this thesis, the norms of mathematical notation are followed. A bold font indicates vectors ($\vect{p}$, $\vect{v}$, \dots).

\begin{itemize}
	\item vectors
	\item matrices
	\item ?
\end{itemize}
