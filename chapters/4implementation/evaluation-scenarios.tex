\section{Evaluation Scenarios}
\subsection{Location Estimation}
There are many parameters that each could have an effect on the performance of the localisation. The ones most often cited in the literature are the amount of reverberation, given as the time it takes for the reverberation to decay by 60dB (\Tsixty), as well as the amount of noise added to the received signals (\emph{SNR}). Other factors include the number of simultaneously simulated sources ($S$), which has an impact on the sparsity assumption of the received signal. The number of \gls{em} iterations $L$ presumably has a positive effect on localisation accuracy, whereas it is also the main driver of computational complexity, as the iterations cannot be executed in parallel due to the data dependency within and across iterations. Further, the individual speech samples might have an effect on the localisation performance, as different samples exhibit different frequency spectra and speech activity over time. After confirming that the order of speech samples used in the evaluation indeed had an effect on the mean error of the location estimates, the order of speech samples used for each trial was randomised for all further trials. The set of speech samples itself consists of 7 anechoic recordings, that can be accessed by visiting the website referenced in \cite{Mainczyk2017}.
\begin{table}[H]
	\begin{tabular}{rccc}
		\toprule
		Parameter                      & Unit & Symbol           & Tested Values                  \\
		\midrule
		Room dimensions                & m    &                  & 6 x 6 x 6.1                    \\
		Evaluated grid points           &      & $\pall$          & $41\cdot41=1,681$                    \\
		Number of microphone pairs     &      & $M$              & 12                             \\
		Distance between microphones   & m    & $d_m$            & 0.2                            \\
		Microphone's distance from wall & m    & $d^w_m$           & 1                            \\
		Microphone positions           & m    & $\bm p_m^i$      & see \autoref{fig:setup}        \\
		Minimum distance between sources & m    & $d_{s,\text{min}}$            & 0.5             \\
		Minimum source distance from wall & m    & $d^w_{s,\text{min}}$           & 1.2           \\
		Number of sources              &      & $S$              & \default{2 - 7}                \\
		Source positions               & m    & $\bm p_s$        & random ($\bm p_s\in \pall$)  \\
		Sampling frequency             & Hz   & $f_s$            & 16,000                         \\
		Number of \glsentryshort{stft} frequency bins &      & $K$                    & 65 (bins 32-96)                             \\
		Selected frequency band                       & Hz   &                        & 500 - 1500                                  \\
		Reflection order               &      & $r$              & 1, \default{3}, max            \\
		Reverberation time             & s    & \Tsixty          & 0.0, \default{0.3}, 0.6, 0.9   \\
		SNR                            & dB   &                  & \default{no noise}, 5, 10, 15, 30     \\
		EM iterations                  &      & $L$              & 1, 2, 3, \default{5}, 10, 20   \\
		Initial variance               &      & $\sigma^{2,(0)}$ & \default{0.1}, $0.5$, 1, 2, 5  \\
		Fixed variance                 &      &                  & \default{false}, true          \\
		Step size of recursion         &      & $\gamma_t$                 & \default{false}, true          \\
		\bottomrule
	\end{tabular}
	\caption[Parameter Set for Evaluation]{Parameter Set for Evaluation: \itshape The \textbf{default values} are set in bold. They provide a simulation environment that includes some reverberation, while keeping the computational complexity low enough to allow for a range of different evaluations and a large sample size per trial. Note, that both \Tsixty and the reflection order are chosen conservatively compared to real-world scenarios, where \Tsixty$\in [0.2, 0.8]$ for a domestic or office environment \cite[p.695]{Gannot2017}.}
	\label{table:parameterset}
\end{table}


\subsection{Source Tracking}
\begin{itemize}
    \item 3 different evaluation scenarios
    \begin{itemize}
        \item Linear Moving Sources, side-by-side
        \item Linear Moving Sources, crossing
        \item Circular Moving Sources
    \end{itemize}
\end{itemize}
