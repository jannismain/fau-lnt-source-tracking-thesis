\subsubsection{Fixed Variance}

Now, fixed variance:

%% Combined Box Plot
\begin{figure}[H]
    \setlength\figureheight{7cm}
    \small
    \setlength\figurewidth{\textwidth}
	\centering
	\begin{tikzpicture}
	    \footnotesize
	    % This file was created by matplotlib2tikz v0.6.14.
\definecolor{color0}{rgb}{0.8,0.207843137254902,0.219607843137255}
\definecolor{color1}{rgb}{1,0.647058823529412,0}
\definecolor{color2}{rgb}{0.0235294117647059,0.603921568627451,0.952941176470588}
\definecolor{color3}{rgb}{0.219607843137255,0.00784313725490196,0.509803921568627}

\begin{axis}[
xlabel={number of sources ($S$)},
ylabel={mean localisation error},
xmin=0.5, xmax=6.5,
ymin=0, ymax=2.5,
width=\figurewidth,
height=\figureheight,
xtick={1,2,3,4,5,6},
xticklabels={2,3,4,5,6,7},
ytick={0,0.5,1,1.5,2,2.5},
minor xtick={},
minor ytick={},
tick align=outside,
tick pos=left,
x grid style={white!69.019607843137251!black},
ymajorgrids,
y grid style={white!69.019607843137251!black}
]
\addplot [line width=1.0pt, black, opacity=1, forget plot]
table {%
0.71 0
0.79 0
0.79 0.111803398874989
0.71 0.111803398874989
0.71 0
};
\addplot [line width=1.0pt, black, opacity=1, forget plot]
table {%
0.75 0
0.75 0
};
\addplot [line width=1.0pt, black, opacity=1, forget plot]
table {%
0.75 0.111803398874989
0.75 0.250988241891854
};
\addplot [line width=1.0pt, black, forget plot]
table {%
0.73 0
0.77 0
};
\addplot [line width=1.0pt, black, forget plot]
table {%
0.73 0.250988241891854
0.77 0.250988241891854
};
\addplot [line width=1.0pt, black, opacity=0.2, mark=*, mark size=1, mark options={solid}, only marks, forget plot]
table {%
0.75 0.604246288964795
0.75 1.46612565475441
0.75 1.37321245982865
0.75 0.536067467586918
0.75 0.460977222864644
0.75 0.440512483795333
0.75 0.440512483795333
0.75 0.33284271247462
0.75 1.0358644463914
0.75 1.68153960890869
0.75 0.641254786637844
0.75 0.490512483795333
0.75 0.502315882670322
0.75 1.82905554375099
0.75 1.38311640469197
0.75 0.570087712549569
0.75 1.78032428943307
0.75 0.56478150704935
0.75 0.33996891847538
0.75 0.452769256906871
0.75 0.746987972600482
0.75 0.364005494464026
0.75 0.76478150704935
0.75 0.36180339887499
0.75 1.32990778863778
};
\addplot [line width=1.0pt, black, opacity=1, forget plot]
table {%
1.71 0.0333333333333334
1.79 0.0333333333333334
1.79 0.559380412509883
1.71 0.559380412509883
1.71 0.0333333333333334
};
\addplot [line width=1.0pt, black, opacity=1, forget plot]
table {%
1.75 0.0333333333333334
1.75 0
};
\addplot [line width=1.0pt, black, opacity=1, forget plot]
table {%
1.75 0.559380412509883
1.75 1.3453733031935
};
\addplot [line width=1.0pt, black, forget plot]
table {%
1.73 0
1.77 0
};
\addplot [line width=1.0pt, black, forget plot]
table {%
1.73 1.3453733031935
1.77 1.3453733031935
};
\addplot [line width=1.0pt, black, opacity=0.2, mark=*, mark size=1, mark options={solid}, only marks, forget plot]
table {%
1.75 1.4759801171344
1.75 1.55372456330245
1.75 1.66678609835387
1.75 2.00040886095362
1.75 1.592439966989
1.75 1.55618861891248
1.75 1.76021324925478
1.75 1.3773901086842
1.75 1.34959353814459
1.75 2.32517837851518
1.75 1.81935391516433
1.75 1.54762952793098
};
\addplot [line width=1.0pt, black, opacity=1, forget plot]
table {%
2.71 0.0603553390593273
2.79 0.0603553390593273
2.79 0.659768566669274
2.71 0.659768566669274
2.71 0.0603553390593273
};
\addplot [line width=1.0pt, black, opacity=1, forget plot]
table {%
2.75 0.0603553390593273
2.75 0
};
\addplot [line width=1.0pt, black, opacity=1, forget plot]
table {%
2.75 0.659768566669274
2.75 1.52213456355617
};
\addplot [line width=1.0pt, black, forget plot]
table {%
2.73 0
2.77 0
};
\addplot [line width=1.0pt, black, forget plot]
table {%
2.73 1.52213456355617
2.77 1.52213456355617
};
\addplot [line width=1.0pt, black, opacity=0.2, mark=*, mark size=1, mark options={solid}, only marks, forget plot]
table {%
2.75 1.63313556476789
2.75 1.57548070157467
2.75 1.69914594760131
};
\addplot [line width=1.0pt, black, opacity=1, forget plot]
table {%
3.71 0.192858911629629
3.79 0.192858911629629
3.79 0.775228073173712
3.71 0.775228073173712
3.71 0.192858911629629
};
\addplot [line width=1.0pt, black, opacity=1, forget plot]
table {%
3.75 0.192858911629629
3.75 0.0199999999999999
};
\addplot [line width=1.0pt, black, opacity=1, forget plot]
table {%
3.75 0.775228073173712
3.75 1.49949126384085
};
\addplot [line width=1.0pt, black, forget plot]
table {%
3.73 0.0199999999999999
3.77 0.0199999999999999
};
\addplot [line width=1.0pt, black, forget plot]
table {%
3.73 1.49949126384085
3.77 1.49949126384085
};
\addplot [line width=1.0pt, black, opacity=1, forget plot]
table {%
4.71 0.358906172421913
4.79 0.358906172421913
4.79 0.840543452914473
4.71 0.840543452914473
4.71 0.358906172421913
};
\addplot [line width=1.0pt, black, opacity=1, forget plot]
table {%
4.75 0.358906172421913
4.75 0.0402368927062183
};
\addplot [line width=1.0pt, black, opacity=1, forget plot]
table {%
4.75 0.840543452914473
4.75 1.52483095949077
};
\addplot [line width=1.0pt, black, forget plot]
table {%
4.73 0.0402368927062183
4.77 0.0402368927062183
};
\addplot [line width=1.0pt, black, forget plot]
table {%
4.73 1.52483095949077
4.77 1.52483095949077
};
\addplot [line width=1.0pt, black, opacity=0.2, mark=*, mark size=1, mark options={solid}, only marks, forget plot]
table {%
4.75 1.83142804880783
};
\addplot [line width=1.0pt, black, opacity=1, forget plot]
table {%
5.71 0.312146579198834
5.79 0.312146579198834
5.79 0.770049848858206
5.71 0.770049848858206
5.71 0.312146579198834
};
\addplot [line width=1.0pt, black, opacity=1, forget plot]
table {%
5.75 0.312146579198834
5.75 0.056903559372885
};
\addplot [line width=1.0pt, black, opacity=1, forget plot]
table {%
5.75 0.770049848858206
5.75 1.34958077235239
};
\addplot [line width=1.0pt, black, forget plot]
table {%
5.73 0.056903559372885
5.77 0.056903559372885
};
\addplot [line width=1.0pt, black, forget plot]
table {%
5.73 1.34958077235239
5.77 1.34958077235239
};
\addplot [line width=1.0pt, black, opacity=0.2, mark=*, mark size=1, mark options={solid}, only marks, forget plot]
table {%
5.75 1.55946578375792
};
\addplot [line width=1.0pt, color0, opacity=1, forget plot]
table {%
0.835 0
0.915 0
0.915 0.111803398874989
0.835 0.111803398874989
0.835 0
};
\addplot [line width=1.0pt, color0, opacity=1, forget plot]
table {%
0.875 0
0.875 0
};
\addplot [line width=1.0pt, color0, opacity=1, forget plot]
table {%
0.875 0.111803398874989
0.875 0.254950975679639
};
\addplot [line width=1.0pt, color0, forget plot]
table {%
0.855 0
0.895 0
};
\addplot [line width=1.0pt, color0, forget plot]
table {%
0.855 0.254950975679639
0.895 0.254950975679639
};
\addplot [line width=1.0pt, color0, opacity=0.2, mark=*, mark size=1, mark options={solid}, only marks, forget plot]
table {%
0.875 0.319258240356725
0.875 1.63545085566823
0.875 0.474341649025257
0.875 0.430116263352131
0.875 0.360555127546399
0.875 0.75
0.875 0.403350993617254
0.875 1.26242283656583
0.875 0.611543369438253
0.875 0.570087712549569
0.875 0.282842712474619
0.875 0.477200187265876
0.875 0.320156211871642
0.875 1.31529464379659
0.875 0.291547594742265
0.875 0.341547594742265
0.875 0.320156211871642
0.875 0.320156211871642
0.875 1.51444397018583
0.875 0.390512483795333
0.875 0.320156211871642
0.875 0.542442890089805
0.875 0.410679596594035
0.875 1.35830777072061
0.875 1.25603763723162
0.875 2.2106716141979
0.875 0.338391446781618
};
\addplot [line width=1.0pt, color0, opacity=1, forget plot]
table {%
1.835 0.0333333333333332
1.915 0.0333333333333332
1.915 0.194616407142096
1.835 0.194616407142096
1.835 0.0333333333333332
};
\addplot [line width=1.0pt, color0, opacity=1, forget plot]
table {%
1.875 0.0333333333333332
1.875 0
};
\addplot [line width=1.0pt, color0, opacity=1, forget plot]
table {%
1.875 0.194616407142096
1.875 0.433333333333333
};
\addplot [line width=1.0pt, color0, forget plot]
table {%
1.855 0
1.895 0
};
\addplot [line width=1.0pt, color0, forget plot]
table {%
1.855 0.433333333333333
1.895 0.433333333333333
};
\addplot [line width=1.0pt, color0, opacity=0.2, mark=*, mark size=1, mark options={solid}, only marks, forget plot]
table {%
1.875 1.00597855413428
1.875 0.57007122416061
1.875 0.568349878404607
1.875 1.16832852296419
1.875 1.18368539363765
1.875 0.81327205946906
1.875 0.836112909836384
1.875 1.41353589912817
1.875 1.01420101271754
1.875 1.04170624974235
1.875 0.86932908026563
1.875 0.942809041582064
1.875 1.32111043386658
1.875 1.57233273062717
1.875 1.61310807608346
1.875 2.23709637001801
1.875 0.460555127546399
1.875 0.439293882986137
1.875 0.64218181144754
1.875 1.54853858667736
1.875 1.43868899847491
1.875 0.933955416897713
1.875 0.687831254713013
1.875 0.673416926347635
1.875 1.44006318973395
1.875 1.28810201498125
1.875 0.882546819658249
1.875 0.5
1.875 1.30235181099662
1.875 1.58549822395516
1.875 0.547688233607362
1.875 0.906131407012751
};
\addplot [line width=1.0pt, color0, opacity=1, forget plot]
table {%
2.835 0.0592419291538692
2.915 0.0592419291538692
2.915 0.553559521197136
2.835 0.553559521197136
2.835 0.0592419291538692
};
\addplot [line width=1.0pt, color0, opacity=1, forget plot]
table {%
2.875 0.0592419291538692
2.875 0
};
\addplot [line width=1.0pt, color0, opacity=1, forget plot]
table {%
2.875 0.553559521197136
2.875 1.28950508722437
};
\addplot [line width=1.0pt, color0, forget plot]
table {%
2.855 0
2.895 0
};
\addplot [line width=1.0pt, color0, forget plot]
table {%
2.855 1.28950508722437
2.895 1.28950508722437
};
\addplot [line width=1.0pt, color0, opacity=0.2, mark=*, mark size=1, mark options={solid}, only marks, forget plot]
table {%
2.875 1.88178821936625
2.875 1.77163824686015
2.875 1.3260966605874
2.875 1.67422238542854
2.875 1.45174918504839
2.875 1.40143140165969
2.875 1.36284090787237
};
\addplot [line width=1.0pt, color0, opacity=1, forget plot]
table {%
3.835 0.0982514076993645
3.915 0.0982514076993645
3.915 0.636468944123383
3.835 0.636468944123383
3.835 0.0982514076993645
};
\addplot [line width=1.0pt, color0, opacity=1, forget plot]
table {%
3.875 0.0982514076993645
3.875 0
};
\addplot [line width=1.0pt, color0, opacity=1, forget plot]
table {%
3.875 0.636468944123383
3.875 1.36930180868083
};
\addplot [line width=1.0pt, color0, forget plot]
table {%
3.855 0
3.895 0
};
\addplot [line width=1.0pt, color0, forget plot]
table {%
3.855 1.36930180868083
3.895 1.36930180868083
};
\addplot [line width=1.0pt, color0, opacity=1, forget plot]
table {%
4.835 0.199864899763432
4.915 0.199864899763432
4.915 0.726236885220065
4.835 0.726236885220065
4.835 0.199864899763432
};
\addplot [line width=1.0pt, color0, opacity=1, forget plot]
table {%
4.875 0.199864899763432
4.875 0
};
\addplot [line width=1.0pt, color0, opacity=1, forget plot]
table {%
4.875 0.726236885220065
4.875 1.49392278062728
};
\addplot [line width=1.0pt, color0, forget plot]
table {%
4.855 0
4.895 0
};
\addplot [line width=1.0pt, color0, forget plot]
table {%
4.855 1.49392278062728
4.895 1.49392278062728
};
\addplot [line width=1.0pt, color0, opacity=1, forget plot]
table {%
5.835 0.189895710237726
5.915 0.189895710237726
5.915 0.641368472073213
5.835 0.641368472073213
5.835 0.189895710237726
};
\addplot [line width=1.0pt, color0, opacity=1, forget plot]
table {%
5.875 0.189895710237726
5.875 0.0166666666666666
};
\addplot [line width=1.0pt, color0, opacity=1, forget plot]
table {%
5.875 0.641368472073213
5.875 1.3054901765262
};
\addplot [line width=1.0pt, color0, forget plot]
table {%
5.855 0.0166666666666666
5.895 0.0166666666666666
};
\addplot [line width=1.0pt, color0, forget plot]
table {%
5.855 1.3054901765262
5.895 1.3054901765262
};
\addplot [line width=1.0pt, color0, opacity=0.2, mark=*, mark size=1, mark options={solid}, only marks, forget plot]
table {%
5.875 1.423951025341
};
\addplot [line width=1.0pt, color1, opacity=1, forget plot]
table {%
0.96 0
1.04 0
1.04 0.1
0.96 0.1
0.96 0
};
\addplot [line width=1.0pt, color1, opacity=1, forget plot]
table {%
1 0
1 0
};
\addplot [line width=1.0pt, color1, opacity=1, forget plot]
table {%
1 0.1
1 0.25
};
\addplot [line width=1.0pt, color1, forget plot]
table {%
0.98 0
1.02 0
};
\addplot [line width=1.0pt, color1, forget plot]
table {%
0.98 0.25
1.02 0.25
};
\addplot [line width=1.0pt, color1, opacity=0.2, mark=*, mark size=1, mark options={solid}, only marks, forget plot]
table {%
1 0.606778145705573
1 0.353553390593273
1 1.40092560861063
1 0.269917281883409
1 0.424264068711929
1 0.449535804129925
1 0.461106256960522
1 1.36579701927343
1 1.69222759816439
1 0.365028153987288
1 0.269258240356725
1 1.25399362039845
1 0.472358526421388
1 0.360555127546399
1 0.320156211871642
1 0.250988241891854
1 0.254950975679639
1 0.250988241891854
1 0.320710678118654
1 0.253224755112299
1 0.282842712474619
1 0.366754374554629
};
\addplot [line width=1.0pt, color1, opacity=1, forget plot]
table {%
1.96 0.0333333333333332
2.04 0.0333333333333332
2.04 0.107868932583326
1.96 0.107868932583326
1.96 0.0333333333333332
};
\addplot [line width=1.0pt, color1, opacity=1, forget plot]
table {%
2 0.0333333333333332
2 0
};
\addplot [line width=1.0pt, color1, opacity=1, forget plot]
table {%
2 0.107868932583326
2 0.21380711874577
};
\addplot [line width=1.0pt, color1, forget plot]
table {%
1.98 0
2.02 0
};
\addplot [line width=1.0pt, color1, forget plot]
table {%
1.98 0.21380711874577
2.02 0.21380711874577
};
\addplot [line width=1.0pt, color1, opacity=0.2, mark=*, mark size=1, mark options={solid}, only marks, forget plot]
table {%
2 0.900506678988061
2 0.938213497569684
2 1.18287711316725
2 0.223606797749979
2 1.3068420917658
2 1.55635163352471
2 0.244502916369753
2 0.268900662411503
2 0.754404867729031
2 0.221895141649746
2 0.624283015999014
2 0.333622517096561
2 0.653700534707335
2 0.833835366826166
2 0.990903038689449
2 0.392665329170335
2 0.908535332796447
2 0.938213497569684
2 0.235702260395516
2 0.52068331172711
2 0.435815374068073
2 1.14939597663778
2 0.448526938038846
2 0.313437474581095
2 0.542144584082458
2 1.6752203942783
2 0.61478666427665
2 1.02440458315457
2 0.484554368574852
2 1.88074525764191
2 1.37523228249377
2 0.86932908026563
2 1.08744675759026
2 0.856997342145496
2 0.843932593411478
2 0.904758528911454
2 0.464805921617159
2 0.709952847797893
2 0.50914909293017
2 0.233333333333333
2 0.344151844011225
};
\addplot [line width=1.0pt, color1, opacity=1, forget plot]
table {%
2.96 0.0499999999999999
3.04 0.0499999999999999
3.04 0.18480341447135
2.96 0.18480341447135
2.96 0.0499999999999999
};
\addplot [line width=1.0pt, color1, opacity=1, forget plot]
table {%
3 0.0499999999999999
3 0
};
\addplot [line width=1.0pt, color1, opacity=1, forget plot]
table {%
3 0.18480341447135
3 0.383565438691684
};
\addplot [line width=1.0pt, color1, forget plot]
table {%
2.98 0
3.02 0
};
\addplot [line width=1.0pt, color1, forget plot]
table {%
2.98 0.383565438691684
3.02 0.383565438691684
};
\addplot [line width=1.0pt, color1, opacity=0.2, mark=*, mark size=1, mark options={solid}, only marks, forget plot]
table {%
3 0.770146090037468
3 0.728228538205365
3 0.67527403287177
3 0.54521630116712
3 0.565388203202208
3 1.21902810577142
3 1.02838907285279
3 0.405374986601503
3 1.42811959340942
3 0.527393881366062
3 0.545546743108388
3 0.434716172582681
3 1.36635854127499
3 1.12575807041911
3 0.77124482778118
3 1.0470195795686
3 0.787093482917222
3 0.897222814543478
3 0.661910114743686
3 1.48715762858667
3 1.19056868639092
3 0.612525158090087
3 0.69676376729781
3 0.808979580753958
3 0.718002660957623
3 0.510977222864644
3 1.02235193689425
3 0.440672099148125
3 1.05158857479567
3 0.732647321898295
3 0.455116263352132
3 0.815890491427044
3 0.72835800001695
3 0.423209706929603
3 1.50742267167998
3 0.725446286306095
3 0.73557067732435
3 0.883883476483184
3 0.897566153550812
3 1.12000337944352
3 0.676996810199223
3 0.712887348773797
3 0.560950240100043
3 0.962634856486014
3 0.85633653816301
3 0.647978728978018
};
\addplot [line width=1.0pt, color1, opacity=1, forget plot]
table {%
3.96 0.0647213595499958
4.04 0.0647213595499958
4.04 0.448009517272384
3.96 0.448009517272384
3.96 0.0647213595499958
};
\addplot [line width=1.0pt, color1, opacity=1, forget plot]
table {%
4 0.0647213595499958
4 0
};
\addplot [line width=1.0pt, color1, opacity=1, forget plot]
table {%
4 0.448009517272384
4 1.0205922873964
};
\addplot [line width=1.0pt, color1, forget plot]
table {%
3.98 0
4.02 0
};
\addplot [line width=1.0pt, color1, forget plot]
table {%
3.98 1.0205922873964
4.02 1.0205922873964
};
\addplot [line width=1.0pt, color1, opacity=0.2, mark=*, mark size=1, mark options={solid}, only marks, forget plot]
table {%
4 1.30974472106811
4 1.21473558660811
4 1.19396805034985
4 1.02503164109268
4 1.27062176287997
4 1.14176592588134
4 1.36279691488886
4 1.1831791933872
4 1.14498540544819
4 1.25003254942188
4 1.59486987161193
};
\addplot [line width=1.0pt, color1, opacity=1, forget plot]
table {%
4.96 0.104637861952051
5.04 0.104637861952051
5.04 0.568464974711117
4.96 0.568464974711117
4.96 0.104637861952051
};
\addplot [line width=1.0pt, color1, opacity=1, forget plot]
table {%
5 0.104637861952051
5 0
};
\addplot [line width=1.0pt, color1, opacity=1, forget plot]
table {%
5 0.568464974711117
5 1.16131283975716
};
\addplot [line width=1.0pt, color1, forget plot]
table {%
4.98 0
5.02 0
};
\addplot [line width=1.0pt, color1, forget plot]
table {%
4.98 1.16131283975716
5.02 1.16131283975716
};
\addplot [line width=1.0pt, color1, opacity=0.2, mark=*, mark size=1, mark options={solid}, only marks, forget plot]
table {%
5 1.41897954046316
5 1.35165473400069
5 1.41400887028387
5 1.61460344346465
};
\addplot [line width=1.0pt, color1, opacity=1, forget plot]
table {%
5.96 0.109729270596202
6.04 0.109729270596202
6.04 0.524390491181517
5.96 0.524390491181517
5.96 0.109729270596202
};
\addplot [line width=1.0pt, color1, opacity=1, forget plot]
table {%
6 0.109729270596202
6 0
};
\addplot [line width=1.0pt, color1, opacity=1, forget plot]
table {%
6 0.524390491181517
6 1.13597210447398
};
\addplot [line width=1.0pt, color1, forget plot]
table {%
5.98 0
6.02 0
};
\addplot [line width=1.0pt, color1, forget plot]
table {%
5.98 1.13597210447398
6.02 1.13597210447398
};
\addplot [line width=1.0pt, color1, opacity=0.2, mark=*, mark size=1, mark options={solid}, only marks, forget plot]
table {%
6 1.16091855980316
6 1.17786346287388
6 1.30362531542731
6 1.2877400609331
};
\addplot [line width=1.0pt, color2, opacity=1, forget plot]
table {%
1.085 0
1.165 0
1.165 0.070710678118655
1.085 0.070710678118655
1.085 0
};
\addplot [line width=1.0pt, color2, opacity=1, forget plot]
table {%
1.125 0
1.125 0
};
\addplot [line width=1.0pt, color2, opacity=1, forget plot]
table {%
1.125 0.070710678118655
1.125 0.15
};
\addplot [line width=1.0pt, color2, forget plot]
table {%
1.105 0
1.145 0
};
\addplot [line width=1.0pt, color2, forget plot]
table {%
1.105 0.15
1.145 0.15
};
\addplot [line width=1.0pt, color2, opacity=0.2, mark=*, mark size=1, mark options={solid}, only marks, forget plot]
table {%
1.125 0.182514076993644
1.125 0.2
1.125 1.16726175299288
1.125 0.228824561127074
1.125 0.191421356237309
1.125 0.19142135623731
1.125 0.403350993617254
1.125 0.292080962648189
1.125 0.250988241891854
1.125 0.292080962648189
1.125 0.36180339887499
1.125 0.211803398874989
};
\addplot [line width=1.0pt, color2, opacity=1, forget plot]
table {%
2.085 0.0333333333333335
2.165 0.0333333333333335
2.165 0.1535183758488
2.085 0.1535183758488
2.085 0.0333333333333335
};
\addplot [line width=1.0pt, color2, opacity=1, forget plot]
table {%
2.125 0.0333333333333335
2.125 0
};
\addplot [line width=1.0pt, color2, opacity=1, forget plot]
table {%
2.125 0.1535183758488
2.125 0.329983164553722
};
\addplot [line width=1.0pt, color2, forget plot]
table {%
2.105 0
2.145 0
};
\addplot [line width=1.0pt, color2, forget plot]
table {%
2.105 0.329983164553722
2.145 0.329983164553722
};
\addplot [line width=1.0pt, color2, opacity=0.2, mark=*, mark size=1, mark options={solid}, only marks, forget plot]
table {%
2.125 0.667591879243998
2.125 1.09201035832302
2.125 0.913635325914892
2.125 1.39430215199413
2.125 0.636867319177574
2.125 0.366666666666667
2.125 1.17044462559952
2.125 0.405409255338946
2.125 0.406601300906186
2.125 0.891712310725411
2.125 0.930241603138247
2.125 0.73748384988657
2.125 1.41077100018719
};
\addplot [line width=1.0pt, color2, opacity=1, forget plot]
table {%
3.085 0.0559016994374947
3.165 0.0559016994374947
3.165 0.201332521472478
3.085 0.201332521472478
3.085 0.0559016994374947
};
\addplot [line width=1.0pt, color2, opacity=1, forget plot]
table {%
3.125 0.0559016994374947
3.125 0
};
\addplot [line width=1.0pt, color2, opacity=1, forget plot]
table {%
3.125 0.201332521472478
3.125 0.403112887414928
};
\addplot [line width=1.0pt, color2, forget plot]
table {%
3.105 0
3.145 0
};
\addplot [line width=1.0pt, color2, forget plot]
table {%
3.105 0.403112887414928
3.145 0.403112887414928
};
\addplot [line width=1.0pt, color2, opacity=0.2, mark=*, mark size=1, mark options={solid}, only marks, forget plot]
table {%
3.125 0.478887360535088
3.125 1.07548059156288
3.125 0.512872214339987
3.125 1.32819965235567
3.125 0.653821182901976
3.125 1.07024538901291
3.125 0.988392013728357
3.125 0.424099750971441
3.125 0.881066017177982
3.125 0.588190993350749
3.125 0.862969629869756
3.125 0.6945023633429
3.125 0.5
3.125 0.722955399649978
3.125 0.532169828919137
3.125 0.965114502586868
3.125 0.83689029230998
3.125 0.677809790820751
3.125 0.808377187277314
3.125 0.788433220375791
3.125 0.539844502878319
3.125 0.986761764868409
3.125 1.06705205760213
3.125 0.83222315165654
};
\addplot [line width=1.0pt, color2, opacity=1, forget plot]
table {%
4.085 0.0806155281280882
4.165 0.0806155281280882
4.165 0.282791012775226
4.085 0.282791012775226
4.085 0.0806155281280882
};
\addplot [line width=1.0pt, color2, opacity=1, forget plot]
table {%
4.125 0.0806155281280882
4.125 0
};
\addplot [line width=1.0pt, color2, opacity=1, forget plot]
table {%
4.125 0.282791012775226
4.125 0.576205832334885
};
\addplot [line width=1.0pt, color2, forget plot]
table {%
4.105 0
4.145 0
};
\addplot [line width=1.0pt, color2, forget plot]
table {%
4.105 0.576205832334885
4.145 0.576205832334885
};
\addplot [line width=1.0pt, color2, opacity=0.2, mark=*, mark size=1, mark options={solid}, only marks, forget plot]
table {%
4.125 0.58642923313041
4.125 0.885147447614599
4.125 0.764794715562175
4.125 1.06045782178756
4.125 0.691139750084629
4.125 0.599758087413705
4.125 0.59234309159688
4.125 0.594178033805709
4.125 0.616568542494924
4.125 0.720770138224265
4.125 1.01867737288815
4.125 0.684844795524746
4.125 0.848764369059527
4.125 0.620298257090466
4.125 0.665278090035422
4.125 1.04999760228894
4.125 0.587831247978066
};
\addplot [line width=1.0pt, color2, opacity=1, forget plot]
table {%
5.085 0.12337187698074
5.165 0.12337187698074
5.165 0.524056447637707
5.085 0.524056447637707
5.085 0.12337187698074
};
\addplot [line width=1.0pt, color2, opacity=1, forget plot]
table {%
5.125 0.12337187698074
5.125 0.0166666666666666
};
\addplot [line width=1.0pt, color2, opacity=1, forget plot]
table {%
5.125 0.524056447637707
5.125 1.01134400526428
};
\addplot [line width=1.0pt, color2, forget plot]
table {%
5.105 0.0166666666666666
5.145 0.0166666666666666
};
\addplot [line width=1.0pt, color2, forget plot]
table {%
5.105 1.01134400526428
5.145 1.01134400526428
};
\addplot [line width=1.0pt, color2, opacity=0.2, mark=*, mark size=1, mark options={solid}, only marks, forget plot]
table {%
5.125 1.36987596430755
};
\addplot [line width=1.0pt, color2, opacity=1, forget plot]
table {%
6.085 0.155335754046722
6.165 0.155335754046722
6.165 0.556713057350806
6.085 0.556713057350806
6.085 0.155335754046722
};
\addplot [line width=1.0pt, color2, opacity=1, forget plot]
table {%
6.125 0.155335754046722
6.125 0.0235702260395516
};
\addplot [line width=1.0pt, color2, opacity=1, forget plot]
table {%
6.125 0.556713057350806
6.125 1.12901050663133
};
\addplot [line width=1.0pt, color2, forget plot]
table {%
6.105 0.0235702260395516
6.145 0.0235702260395516
};
\addplot [line width=1.0pt, color2, forget plot]
table {%
6.105 1.12901050663133
6.145 1.12901050663133
};
\addplot [line width=1.0pt, color2, opacity=0.2, mark=*, mark size=1, mark options={solid}, only marks, forget plot]
table {%
6.125 1.19194827230855
6.125 1.61409861110605
};
\addplot [line width=1.0pt, color3, opacity=1, forget plot]
table {%
1.21 0.0500000000000001
1.29 0.0500000000000001
1.29 0.2275201202828
1.21 0.2275201202828
1.21 0.0500000000000001
};
\addplot [line width=1.0pt, color3, opacity=1, forget plot]
table {%
1.25 0.0500000000000001
1.25 0
};
\addplot [line width=1.0pt, color3, opacity=1, forget plot]
table {%
1.25 0.2275201202828
1.25 0.493524079633387
};
\addplot [line width=1.0pt, color3, forget plot]
table {%
1.23 0
1.27 0
};
\addplot [line width=1.0pt, color3, forget plot]
table {%
1.23 0.493524079633387
1.27 0.493524079633387
};
\addplot [line width=1.0pt, color3, opacity=0.2, mark=*, mark size=1, mark options={solid}, only marks, forget plot]
table {%
1.25 1.37568164921976
1.25 0.912414379544733
1.25 0.905862138431184
1.25 0.901308556317268
1.25 0.782623792124926
1.25 0.961769203083567
1.25 1.1500623353648
1.25 0.707106781186548
1.25 0.920087712549569
1.25 0.667083203206317
1.25 0.58309518948453
1.25 1.02195444572929
1.25 1.40099960031968
1.25 0.651920240520265
1.25 0.506449510224598
1.25 0.651920240520265
1.25 0.643310689511722
1.25 1.2747548783982
1.25 0.782623792124926
1.25 0.961769203083567
1.25 1.1500623353648
1.25 0.707106781186548
1.25 0.920087712549569
1.25 0.667083203206317
};
\addplot [line width=1.0pt, color3, opacity=1, forget plot]
table {%
2.21 0.15500938466243
2.29 0.15500938466243
2.29 0.596641663881264
2.21 0.596641663881264
2.21 0.15500938466243
};
\addplot [line width=1.0pt, color3, opacity=1, forget plot]
table {%
2.25 0.15500938466243
2.25 0
};
\addplot [line width=1.0pt, color3, opacity=1, forget plot]
table {%
2.25 0.596641663881264
2.25 1.2536818576676
};
\addplot [line width=1.0pt, color3, forget plot]
table {%
2.23 0
2.27 0
};
\addplot [line width=1.0pt, color3, forget plot]
table {%
2.23 1.2536818576676
2.27 1.2536818576676
};
\addplot [line width=1.0pt, color3, opacity=0.2, mark=*, mark size=1, mark options={solid}, only marks, forget plot]
table {%
2.25 1.30261275289467
2.25 1.27960093463241
2.25 1.47378107613193
2.25 1.31876381987945
2.25 1.4160122960012
2.25 1.29547937031445
2.25 1.31493646573158
2.25 1.45476868164791
2.25 1.31993265821489
2.25 1.39272126619564
2.25 1.35240019359396
2.25 1.29702629147208
};
\addplot [line width=1.0pt, color3, opacity=1, forget plot]
table {%
3.21 0.242705098312484
3.29 0.242705098312484
3.29 0.706041066925718
3.21 0.706041066925718
3.21 0.242705098312484
};
\addplot [line width=1.0pt, color3, opacity=1, forget plot]
table {%
3.25 0.242705098312484
3.25 0.0250000000000001
};
\addplot [line width=1.0pt, color3, opacity=1, forget plot]
table {%
3.25 0.706041066925718
3.25 1.35062297614975
};
\addplot [line width=1.0pt, color3, forget plot]
table {%
3.23 0.0250000000000001
3.27 0.0250000000000001
};
\addplot [line width=1.0pt, color3, forget plot]
table {%
3.23 1.35062297614975
3.27 1.35062297614975
};
\addplot [line width=1.0pt, color3, opacity=0.2, mark=*, mark size=1, mark options={solid}, only marks, forget plot]
table {%
3.25 1.4710617799977
};
\addplot [line width=1.0pt, color3, opacity=1, forget plot]
table {%
4.21 0.315499668300181
4.29 0.315499668300181
4.29 0.74227223487545
4.21 0.74227223487545
4.21 0.315499668300181
};
\addplot [line width=1.0pt, color3, opacity=1, forget plot]
table {%
4.25 0.315499668300181
4.25 0.0682842712474618
};
\addplot [line width=1.0pt, color3, opacity=1, forget plot]
table {%
4.25 0.74227223487545
4.25 1.31251513626844
};
\addplot [line width=1.0pt, color3, forget plot]
table {%
4.23 0.0682842712474618
4.27 0.0682842712474618
};
\addplot [line width=1.0pt, color3, forget plot]
table {%
4.23 1.31251513626844
4.27 1.31251513626844
};
\addplot [line width=1.0pt, color3, opacity=0.2, mark=*, mark size=1, mark options={solid}, only marks, forget plot]
table {%
4.25 1.44137527275966
4.25 1.67954551139348
4.25 1.46890036076962
4.25 1.92237393666179
};
\addplot [line width=1.0pt, color3, opacity=1, forget plot]
table {%
5.21 0.453275011809849
5.29 0.453275011809849
5.29 0.848196233226909
5.21 0.848196233226909
5.21 0.453275011809849
};
\addplot [line width=1.0pt, color3, opacity=1, forget plot]
table {%
5.25 0.453275011809849
5.25 0.131439158622878
};
\addplot [line width=1.0pt, color3, opacity=1, forget plot]
table {%
5.25 0.848196233226909
5.25 1.43659200522907
};
\addplot [line width=1.0pt, color3, forget plot]
table {%
5.23 0.131439158622878
5.27 0.131439158622878
};
\addplot [line width=1.0pt, color3, forget plot]
table {%
5.23 1.43659200522907
5.27 1.43659200522907
};
\addplot [line width=1.0pt, color3, opacity=0.2, mark=*, mark size=1, mark options={solid}, only marks, forget plot]
table {%
5.25 1.46384423779094
5.25 1.51534155178417
5.25 1.46449461839321
5.25 1.50729264413117
5.25 1.67438328003837
};
\addplot [line width=1.0pt, color3, opacity=1, forget plot]
table {%
6.21 0.399045370860738
6.29 0.399045370860738
6.29 0.703645154472796
6.21 0.703645154472796
6.21 0.399045370860738
};
\addplot [line width=1.0pt, color3, opacity=1, forget plot]
table {%
6.25 0.399045370860738
6.25 0.123570226039552
};
\addplot [line width=1.0pt, color3, opacity=1, forget plot]
table {%
6.25 0.703645154472796
6.25 1.13114861423054
};
\addplot [line width=1.0pt, color3, forget plot]
table {%
6.23 0.123570226039552
6.27 0.123570226039552
};
\addplot [line width=1.0pt, color3, forget plot]
table {%
6.23 1.13114861423054
6.27 1.13114861423054
};
\addplot [line width=1.0pt, color3, opacity=0.2, mark=*, mark size=1, mark options={solid}, only marks, forget plot]
table {%
6.25 1.537977535711
6.25 1.30075907060088
6.25 1.59058162409458
6.25 1.26027675423441
6.25 1.25176813344263
6.25 1.21213958207844
};
\addplot [line width=1.0pt, black, opacity=1, forget plot]
table {%
0.71 0
0.79 0
};
\addplot [line width=1.0pt, black, dashed, mark=x, mark size=3, mark options={solid}, forget plot]
table {%
0.75 0.136813230999457
};
\addplot [line width=1.0pt, black, opacity=1, forget plot]
table {%
1.71 0.127504692331215
1.79 0.127504692331215
};
\addplot [line width=1.0pt, black, dashed, mark=x, mark size=3, mark options={solid}, forget plot]
table {%
1.75 0.356385341153677
};
\addplot [line width=1.0pt, black, opacity=1, forget plot]
table {%
2.71 0.215920072876257
2.79 0.215920072876257
};
\addplot [line width=1.0pt, black, dashed, mark=x, mark size=3, mark options={solid}, forget plot]
table {%
2.75 0.402717829942746
};
\addplot [line width=1.0pt, black, opacity=1, forget plot]
table {%
3.71 0.515123747484223
3.79 0.515123747484223
};
\addplot [line width=1.0pt, black, dashed, mark=x, mark size=3, mark options={solid}, forget plot]
table {%
3.75 0.511390041008845
};
\addplot [line width=1.0pt, black, opacity=1, forget plot]
table {%
4.71 0.639655986044734
4.79 0.639655986044734
};
\addplot [line width=1.0pt, black, dashed, mark=x, mark size=3, mark options={solid}, forget plot]
table {%
4.75 0.629411942591584
};
\addplot [line width=1.0pt, black, opacity=1, forget plot]
table {%
5.71 0.533946219123222
5.79 0.533946219123222
};
\addplot [line width=1.0pt, black, dashed, mark=x, mark size=3, mark options={solid}, forget plot]
table {%
5.75 0.560600095137806
};
\addplot [line width=1.0pt, color0, opacity=1, forget plot]
table {%
0.835 0.0499999999999998
0.915 0.0499999999999998
};
\addplot [line width=1.0pt, color0, dashed, mark=x, mark size=3, mark options={solid}, forget plot]
table {%
0.875 0.131709192092931
};
\addplot [line width=1.0pt, color0, opacity=1, forget plot]
table {%
1.835 0.0666666666666667
1.915 0.0666666666666667
};
\addplot [line width=1.0pt, color0, dashed, mark=x, mark size=3, mark options={solid}, forget plot]
table {%
1.875 0.233789658983868
};
\addplot [line width=1.0pt, color0, opacity=1, forget plot]
table {%
2.835 0.177427167968539
2.915 0.177427167968539
};
\addplot [line width=1.0pt, color0, dashed, mark=x, mark size=3, mark options={solid}, forget plot]
table {%
2.875 0.359230618732277
};
\addplot [line width=1.0pt, color0, opacity=1, forget plot]
table {%
3.835 0.356346180953454
3.915 0.356346180953454
};
\addplot [line width=1.0pt, color0, dashed, mark=x, mark size=3, mark options={solid}, forget plot]
table {%
3.875 0.396280324254095
};
\addplot [line width=1.0pt, color0, opacity=1, forget plot]
table {%
4.835 0.519570060337422
4.915 0.519570060337422
};
\addplot [line width=1.0pt, color0, dashed, mark=x, mark size=3, mark options={solid}, forget plot]
table {%
4.875 0.506573900605875
};
\addplot [line width=1.0pt, color0, opacity=1, forget plot]
table {%
5.835 0.433636555283121
5.915 0.433636555283121
};
\addplot [line width=1.0pt, color0, dashed, mark=x, mark size=3, mark options={solid}, forget plot]
table {%
5.875 0.45410213158402
};
\addplot [line width=1.0pt, color1, opacity=1, forget plot]
table {%
0.96 0.0499999999999998
1.04 0.0499999999999998
};
\addplot [line width=1.0pt, color1, dashed, mark=x, mark size=3, mark options={solid}, forget plot]
table {%
1 0.0788577237106476
};
\addplot [line width=1.0pt, color1, opacity=1, forget plot]
table {%
1.96 0.0471404520791031
2.04 0.0471404520791031
};
\addplot [line width=1.0pt, color1, dashed, mark=x, mark size=3, mark options={solid}, forget plot]
table {%
2 0.135776292452477
};
\addplot [line width=1.0pt, color1, opacity=1, forget plot]
table {%
2.96 0.0853553390593273
3.04 0.0853553390593273
};
\addplot [line width=1.0pt, color1, dashed, mark=x, mark size=3, mark options={solid}, forget plot]
table {%
3 0.189889423286644
};
\addplot [line width=1.0pt, color1, opacity=1, forget plot]
table {%
3.96 0.150160278526098
4.04 0.150160278526098
};
\addplot [line width=1.0pt, color1, dashed, mark=x, mark size=3, mark options={solid}, forget plot]
table {%
4 0.289673829046843
};
\addplot [line width=1.0pt, color1, opacity=1, forget plot]
table {%
4.96 0.281635215721924
5.04 0.281635215721924
};
\addplot [line width=1.0pt, color1, dashed, mark=x, mark size=3, mark options={solid}, forget plot]
table {%
5 0.366924981768927
};
\addplot [line width=1.0pt, color1, opacity=1, forget plot]
table {%
5.96 0.241323245417691
6.04 0.241323245417691
};
\addplot [line width=1.0pt, color1, dashed, mark=x, mark size=3, mark options={solid}, forget plot]
table {%
6 0.344960881628136
};
\addplot [line width=1.0pt, color2, opacity=1, forget plot]
table {%
1.085 0.05
1.165 0.05
};
\addplot [line width=1.0pt, color2, dashed, mark=x, mark size=3, mark options={solid}, forget plot]
table {%
1.125 0.0640379259875692
};
\addplot [line width=1.0pt, color2, opacity=1, forget plot]
table {%
2.085 0.0804737854124365
2.165 0.0804737854124365
};
\addplot [line width=1.0pt, color2, dashed, mark=x, mark size=3, mark options={solid}, forget plot]
table {%
2.125 0.157574128827886
};
\addplot [line width=1.0pt, color2, opacity=1, forget plot]
table {%
3.085 0.0912570384968221
3.165 0.0912570384968221
};
\addplot [line width=1.0pt, color2, dashed, mark=x, mark size=3, mark options={solid}, forget plot]
table {%
3.125 0.206990594393249
};
\addplot [line width=1.0pt, color2, opacity=1, forget plot]
table {%
4.085 0.128704815926677
4.165 0.128704815926677
};
\addplot [line width=1.0pt, color2, dashed, mark=x, mark size=3, mark options={solid}, forget plot]
table {%
4.125 0.232951808453975
};
\addplot [line width=1.0pt, color2, opacity=1, forget plot]
table {%
5.085 0.212443940947679
5.165 0.212443940947679
};
\addplot [line width=1.0pt, color2, dashed, mark=x, mark size=3, mark options={solid}, forget plot]
table {%
5.125 0.317093805714458
};
\addplot [line width=1.0pt, color2, opacity=1, forget plot]
table {%
6.085 0.304668498154675
6.165 0.304668498154675
};
\addplot [line width=1.0pt, color2, dashed, mark=x, mark size=3, mark options={solid}, forget plot]
table {%
6.125 0.377992570535079
};
\addplot [line width=1.0pt, color3, opacity=1, forget plot]
table {%
1.21 0.120710678118654
1.29 0.120710678118654
};
\addplot [line width=1.0pt, color3, dashed, mark=x, mark size=3, mark options={solid}, forget plot]
table {%
1.25 0.201020829761283
};
\addplot [line width=1.0pt, color3, opacity=1, forget plot]
table {%
2.21 0.286679880934687
2.29 0.286679880934687
};
\addplot [line width=1.0pt, color3, dashed, mark=x, mark size=3, mark options={solid}, forget plot]
table {%
2.25 0.423029634544486
};
\addplot [line width=1.0pt, color3, opacity=1, forget plot]
table {%
3.21 0.458304021584257
3.29 0.458304021584257
};
\addplot [line width=1.0pt, color3, dashed, mark=x, mark size=3, mark options={solid}, forget plot]
table {%
3.25 0.491587157968608
};
\addplot [line width=1.0pt, color3, opacity=1, forget plot]
table {%
4.21 0.502188696901362
4.29 0.502188696901362
};
\addplot [line width=1.0pt, color3, dashed, mark=x, mark size=3, mark options={solid}, forget plot]
table {%
4.25 0.554255736211583
};
\addplot [line width=1.0pt, color3, opacity=1, forget plot]
table {%
5.21 0.623081577715303
5.29 0.623081577715303
};
\addplot [line width=1.0pt, color3, dashed, mark=x, mark size=3, mark options={solid}, forget plot]
table {%
5.25 0.666752535612518
};
\addplot [line width=1.0pt, color3, opacity=1, forget plot]
table {%
6.21 0.542245318744278
6.29 0.542245318744278
};
\addplot [line width=1.0pt, color3, dashed, mark=x, mark size=3, mark options={solid}, forget plot]
table {%
6.25 0.58088761472104
};
\end{axis}

\node at ({$(current bounding box.south west)!0.5!(current bounding box.south east)$}|-{$(current bounding box.south west)!0.98!(current bounding box.north west)$})[
  anchor=north,
  text=black,
  rotate=0.0
]{ };

	    \begin{customlegend}[
legend entries={$\sigma^2=0.1$,$\sigma^2=0.5$,$\sigma^2=1.0$,$\sigma^2=2.0$,$\sigma^2=3.0$,$\sigma^2=5.0$},
legend cell align=left,
legend style={at={(0.05,5.37)}, anchor=north west, draw=white!80.0!black, font=\footnotesize}] % <= to define position and font legend
% the following are the "images" and numbers in the legend
    \addlegendimage{area legend,gray,fill=gray}
    \addlegendimage{area legend,black,fill=black}
    \addlegendimage{area legend,color0,fill=color0}
    \addlegendimage{area legend,color1,fill=color1}
    \addlegendimage{area legend,color2,fill=color2}
    \addlegendimage{area legend,white!82.745098039215677!black,fill=white!82.745098039215677!black}
\end{customlegend}
	\end{tikzpicture}
	
	\caption{Evaluation results for fixed and estimated variance}
	\label{fig:boxplotVariance}
\end{figure}

\begin{itemize}
    \item Extreme values for fixed variance yield significant worse performance than more moderate values
    \item Optimal variance seems to be around 2
    \item 0.1 and 0.5 were outperformed across all number of sources
    \item 1, 2 and 3 performed almost equally well, although 3 was slightly worse for 4-6 sources
\end{itemize}

For comparison, estimated variance with same initial values for variance as fixed variance evaluation:
%% Combined Box Plot
\begin{figure}[H]
    \setlength\figureheight{7cm}
    \small
    \setlength\figurewidth{\textwidth}
	\centering
	\begin{tikzpicture}
	    \footnotesize
	    % This file was created by matplotlib2tikz v0.6.14.
\definecolor{color0}{rgb}{0.8,0.207843137254902,0.219607843137255}
\definecolor{color1}{rgb}{0.67843137254902,0.847058823529412,0.901960784313726}
\definecolor{color2}{rgb}{1,0.647058823529412,0}

\begin{axis}[
xmin=0.5, xmax=6.5,
ymin=0, ymax=1.75,
width=\figurewidth,
height=\figureheight,
xtick={1,2,3,4,5,6},
xticklabels={2,3,4,5,6,7},
ytick={0,0.25,0.5,0.75,1,1.25,1.5,1.75},
minor xtick={},
minor ytick={},
tick align=outside,
tick pos=left,
x grid style={white!69.019607843137251!black},
ymajorgrids,
y grid style={white!69.019607843137251!black}
]
\addplot [line width=0.5pt, white!66.274509803921561!black, opacity=1, forget plot]
table {%
0.66 0
0.74 0
0.74 0.111803398874989
0.66 0.111803398874989
0.66 0
};
\addplot [line width=0.5pt, white!66.274509803921561!black, opacity=1, forget plot]
table {%
0.7 0
0.7 0
};
\addplot [line width=0.5pt, white!66.274509803921561!black, opacity=1, forget plot]
table {%
0.7 0.111803398874989
0.7 0.269258240356725
};
\addplot [line width=0.5pt, white!66.274509803921561!black, forget plot]
table {%
0.68 0
0.72 0
};
\addplot [line width=0.5pt, white!66.274509803921561!black, forget plot]
table {%
0.68 0.269258240356725
0.72 0.269258240356725
};
\addplot [line width=0.5pt, white!66.274509803921561!black, opacity=1, forget plot]
table {%
1.66 0.0333333333333332
1.74 0.0333333333333332
1.74 0.170770187520589
1.66 0.170770187520589
1.66 0.0333333333333332
};
\addplot [line width=0.5pt, white!66.274509803921561!black, opacity=1, forget plot]
table {%
1.7 0.0333333333333332
1.7 0
};
\addplot [line width=0.5pt, white!66.274509803921561!black, opacity=1, forget plot]
table {%
1.7 0.170770187520589
1.7 0.372677996249965
};
\addplot [line width=0.5pt, white!66.274509803921561!black, forget plot]
table {%
1.68 0
1.72 0
};
\addplot [line width=0.5pt, white!66.274509803921561!black, forget plot]
table {%
1.68 0.372677996249965
1.72 0.372677996249965
};
\addplot [line width=0.5pt, white!66.274509803921561!black, opacity=1, forget plot]
table {%
2.66 0.0499999999999999
2.74 0.0499999999999999
2.74 0.324876959514958
2.66 0.324876959514958
2.66 0.0499999999999999
};
\addplot [line width=0.5pt, white!66.274509803921561!black, opacity=1, forget plot]
table {%
2.7 0.0499999999999999
2.7 0
};
\addplot [line width=0.5pt, white!66.274509803921561!black, opacity=1, forget plot]
table {%
2.7 0.324876959514958
2.7 0.708655496763772
};
\addplot [line width=0.5pt, white!66.274509803921561!black, forget plot]
table {%
2.68 0
2.72 0
};
\addplot [line width=0.5pt, white!66.274509803921561!black, forget plot]
table {%
2.68 0.708655496763772
2.72 0.708655496763772
};
\addplot [line width=0.5pt, white!66.274509803921561!black, opacity=1, forget plot]
table {%
3.66 0.0707106781186548
3.74 0.0707106781186548
3.74 0.502651194474483
3.66 0.502651194474483
3.66 0.0707106781186548
};
\addplot [line width=0.5pt, white!66.274509803921561!black, opacity=1, forget plot]
table {%
3.7 0.0707106781186548
3.7 0
};
\addplot [line width=0.5pt, white!66.274509803921561!black, opacity=1, forget plot]
table {%
3.7 0.502651194474483
3.7 1.13572721333167
};
\addplot [line width=0.5pt, white!66.274509803921561!black, forget plot]
table {%
3.68 0
3.72 0
};
\addplot [line width=0.5pt, white!66.274509803921561!black, forget plot]
table {%
3.68 1.13572721333167
3.72 1.13572721333167
};
\addplot [line width=0.5pt, white!66.274509803921561!black, opacity=1, forget plot]
table {%
4.66 0.0992956328951887
4.74 0.0992956328951887
4.74 0.559063307498281
4.66 0.559063307498281
4.66 0.0992956328951887
};
\addplot [line width=0.5pt, white!66.274509803921561!black, opacity=1, forget plot]
table {%
4.7 0.0992956328951887
4.7 0
};
\addplot [line width=0.5pt, white!66.274509803921561!black, opacity=1, forget plot]
table {%
4.7 0.559063307498281
4.7 1.23389496228055
};
\addplot [line width=0.5pt, white!66.274509803921561!black, forget plot]
table {%
4.68 0
4.72 0
};
\addplot [line width=0.5pt, white!66.274509803921561!black, forget plot]
table {%
4.68 1.23389496228055
4.72 1.23389496228055
};
\addplot [line width=0.5pt, white!66.274509803921561!black, opacity=1, forget plot]
table {%
5.66 0.113306188778075
5.74 0.113306188778075
5.74 0.590264014068069
5.66 0.590264014068069
5.66 0.113306188778075
};
\addplot [line width=0.5pt, white!66.274509803921561!black, opacity=1, forget plot]
table {%
5.7 0.113306188778075
5.7 0
};
\addplot [line width=0.5pt, white!66.274509803921561!black, opacity=1, forget plot]
table {%
5.7 0.590264014068069
5.7 1.18547406329423
};
\addplot [line width=0.5pt, white!66.274509803921561!black, forget plot]
table {%
5.68 0
5.72 0
};
\addplot [line width=0.5pt, white!66.274509803921561!black, forget plot]
table {%
5.68 1.18547406329423
5.72 1.18547406329423
};
\addplot [line width=0.5pt, black, opacity=1, forget plot]
table {%
0.78 0
0.86 0
0.86 0.101530218685906
0.78 0.101530218685906
0.78 0
};
\addplot [line width=0.5pt, black, opacity=1, forget plot]
table {%
0.82 0
0.82 0
};
\addplot [line width=0.5pt, black, opacity=1, forget plot]
table {%
0.82 0.101530218685906
0.82 0.1802775637732
};
\addplot [line width=0.5pt, black, forget plot]
table {%
0.8 0
0.84 0
};
\addplot [line width=0.5pt, black, forget plot]
table {%
0.8 0.1802775637732
0.84 0.1802775637732
};
\addplot [line width=0.5pt, black, opacity=1, forget plot]
table {%
1.78 0.0333333333333332
1.86 0.0333333333333332
1.86 0.107254013272231
1.78 0.107254013272231
1.78 0.0333333333333332
};
\addplot [line width=0.5pt, black, opacity=1, forget plot]
table {%
1.82 0.0333333333333332
1.82 0
};
\addplot [line width=0.5pt, black, opacity=1, forget plot]
table {%
1.82 0.107254013272231
1.82 0.166666666666667
};
\addplot [line width=0.5pt, black, forget plot]
table {%
1.8 0
1.84 0
};
\addplot [line width=0.5pt, black, forget plot]
table {%
1.8 0.166666666666667
1.84 0.166666666666667
};
\addplot [line width=0.5pt, black, opacity=1, forget plot]
table {%
2.78 0.03125
2.86 0.03125
2.86 0.24454755485092
2.78 0.24454755485092
2.78 0.03125
};
\addplot [line width=0.5pt, black, opacity=1, forget plot]
table {%
2.82 0.03125
2.82 0
};
\addplot [line width=0.5pt, black, opacity=1, forget plot]
table {%
2.82 0.24454755485092
2.82 0.375891721190455
};
\addplot [line width=0.5pt, black, forget plot]
table {%
2.8 0
2.84 0
};
\addplot [line width=0.5pt, black, forget plot]
table {%
2.8 0.375891721190455
2.84 0.375891721190455
};
\addplot [line width=0.5pt, black, opacity=1, forget plot]
table {%
3.78 0.0512132034355965
3.86 0.0512132034355965
3.86 0.437231523799283
3.78 0.437231523799283
3.78 0.0512132034355965
};
\addplot [line width=0.5pt, black, opacity=1, forget plot]
table {%
3.82 0.0512132034355965
3.82 0
};
\addplot [line width=0.5pt, black, opacity=1, forget plot]
table {%
3.82 0.437231523799283
3.82 0.886880372161175
};
\addplot [line width=0.5pt, black, forget plot]
table {%
3.8 0
3.84 0
};
\addplot [line width=0.5pt, black, forget plot]
table {%
3.8 0.886880372161175
3.84 0.886880372161175
};
\addplot [line width=0.5pt, black, opacity=1, forget plot]
table {%
4.78 0.121959847582213
4.86 0.121959847582213
4.86 0.617371128370128
4.78 0.617371128370128
4.78 0.121959847582213
};
\addplot [line width=0.5pt, black, opacity=1, forget plot]
table {%
4.82 0.121959847582213
4.82 0.0235702260395517
};
\addplot [line width=0.5pt, black, opacity=1, forget plot]
table {%
4.82 0.617371128370128
4.82 1.15149743469306
};
\addplot [line width=0.5pt, black, forget plot]
table {%
4.8 0.0235702260395517
4.84 0.0235702260395517
};
\addplot [line width=0.5pt, black, forget plot]
table {%
4.8 1.15149743469306
4.84 1.15149743469306
};
\addplot [line width=0.5pt, black, opacity=1, forget plot]
table {%
5.78 0.190468892461702
5.86 0.190468892461702
5.86 0.691472576678585
5.78 0.691472576678585
5.78 0.190468892461702
};
\addplot [line width=0.5pt, black, opacity=1, forget plot]
table {%
5.82 0.190468892461702
5.82 0.0166666666666667
};
\addplot [line width=0.5pt, black, opacity=1, forget plot]
table {%
5.82 0.691472576678585
5.82 1.20947870980611
};
\addplot [line width=0.5pt, black, forget plot]
table {%
5.8 0.0166666666666667
5.84 0.0166666666666667
};
\addplot [line width=0.5pt, black, forget plot]
table {%
5.8 1.20947870980611
5.84 1.20947870980611
};
\addplot [line width=0.5pt, color0, opacity=1, forget plot]
table {%
0.9 0
0.98 0
0.98 0.141421356237309
0.9 0.141421356237309
0.9 0
};
\addplot [line width=0.5pt, color0, opacity=1, forget plot]
table {%
0.94 0
0.94 0
};
\addplot [line width=0.5pt, color0, opacity=1, forget plot]
table {%
0.94 0.141421356237309
0.94 0.320156211871642
};
\addplot [line width=0.5pt, color0, forget plot]
table {%
0.92 0
0.96 0
};
\addplot [line width=0.5pt, color0, forget plot]
table {%
0.92 0.320156211871642
0.96 0.320156211871642
};
\addplot [line width=0.5pt, color0, opacity=1, forget plot]
table {%
1.9 0.0249999999999999
1.98 0.0249999999999999
1.98 0.137763287808511
1.9 0.137763287808511
1.9 0.0249999999999999
};
\addplot [line width=0.5pt, color0, opacity=1, forget plot]
table {%
1.94 0.0249999999999999
1.94 0
};
\addplot [line width=0.5pt, color0, opacity=1, forget plot]
table {%
1.94 0.137763287808511
1.94 0.225594297854413
};
\addplot [line width=0.5pt, color0, forget plot]
table {%
1.92 0
1.96 0
};
\addplot [line width=0.5pt, color0, forget plot]
table {%
1.92 0.225594297854413
1.96 0.225594297854413
};
\addplot [line width=0.5pt, color0, opacity=1, forget plot]
table {%
2.9 0.0353553390593274
2.98 0.0353553390593274
2.98 0.23345386545294
2.9 0.23345386545294
2.9 0.0353553390593274
};
\addplot [line width=0.5pt, color0, opacity=1, forget plot]
table {%
2.94 0.0353553390593274
2.94 0
};
\addplot [line width=0.5pt, color0, opacity=1, forget plot]
table {%
2.94 0.23345386545294
2.94 0.509619407771256
};
\addplot [line width=0.5pt, color0, forget plot]
table {%
2.92 0
2.96 0
};
\addplot [line width=0.5pt, color0, forget plot]
table {%
2.92 0.509619407771256
2.96 0.509619407771256
};
\addplot [line width=0.5pt, color0, opacity=1, forget plot]
table {%
3.9 0.0600000000000001
3.98 0.0600000000000001
3.98 0.444930810684426
3.9 0.444930810684426
3.9 0.0600000000000001
};
\addplot [line width=0.5pt, color0, opacity=1, forget plot]
table {%
3.94 0.0600000000000001
3.94 0
};
\addplot [line width=0.5pt, color0, opacity=1, forget plot]
table {%
3.94 0.444930810684426
3.94 0.952028943712642
};
\addplot [line width=0.5pt, color0, forget plot]
table {%
3.92 0
3.96 0
};
\addplot [line width=0.5pt, color0, forget plot]
table {%
3.92 0.952028943712642
3.96 0.952028943712642
};
\addplot [line width=0.5pt, color0, opacity=1, forget plot]
table {%
4.9 0.128470065541656
4.98 0.128470065541656
4.98 0.55354560761742
4.9 0.55354560761742
4.9 0.128470065541656
};
\addplot [line width=0.5pt, color0, opacity=1, forget plot]
table {%
4.94 0.128470065541656
4.94 0
};
\addplot [line width=0.5pt, color0, opacity=1, forget plot]
table {%
4.94 0.55354560761742
4.94 1.09879744244614
};
\addplot [line width=0.5pt, color0, forget plot]
table {%
4.92 0
4.96 0
};
\addplot [line width=0.5pt, color0, forget plot]
table {%
4.92 1.09879744244614
4.96 1.09879744244614
};
\addplot [line width=0.5pt, color0, opacity=1, forget plot]
table {%
5.9 0.138064845475464
5.98 0.138064845475464
5.98 0.67712038495387
5.9 0.67712038495387
5.9 0.138064845475464
};
\addplot [line width=0.5pt, color0, opacity=1, forget plot]
table {%
5.94 0.138064845475464
5.94 0
};
\addplot [line width=0.5pt, color0, opacity=1, forget plot]
table {%
5.94 0.67712038495387
5.94 1.41007283072559
};
\addplot [line width=0.5pt, color0, forget plot]
table {%
5.92 0
5.96 0
};
\addplot [line width=0.5pt, color0, forget plot]
table {%
5.92 1.41007283072559
5.96 1.41007283072559
};
\addplot [line width=0.5pt, color1, opacity=1, forget plot]
table {%
1.02 0
1.1 0
1.1 0.070710678118655
1.02 0.070710678118655
1.02 0
};
\addplot [line width=0.5pt, color1, opacity=1, forget plot]
table {%
1.06 0
1.06 0
};
\addplot [line width=0.5pt, color1, opacity=1, forget plot]
table {%
1.06 0.070710678118655
1.06 0.16180339887499
};
\addplot [line width=0.5pt, color1, forget plot]
table {%
1.04 0
1.08 0
};
\addplot [line width=0.5pt, color1, forget plot]
table {%
1.04 0.16180339887499
1.08 0.16180339887499
};
\addplot [line width=0.5pt, color1, opacity=1, forget plot]
table {%
2.02 0
2.1 0
2.1 0.120185042515466
2.02 0.120185042515466
2.02 0
};
\addplot [line width=0.5pt, color1, opacity=1, forget plot]
table {%
2.06 0
2.06 0
};
\addplot [line width=0.5pt, color1, opacity=1, forget plot]
table {%
2.06 0.120185042515466
2.06 0.254041092821143
};
\addplot [line width=0.5pt, color1, forget plot]
table {%
2.04 0
2.08 0
};
\addplot [line width=0.5pt, color1, forget plot]
table {%
2.04 0.254041092821143
2.08 0.254041092821143
};
\addplot [line width=0.5pt, color1, opacity=1, forget plot]
table {%
3.02 0.0499999999999999
3.1 0.0499999999999999
3.1 0.200502988632573
3.02 0.200502988632573
3.02 0.0499999999999999
};
\addplot [line width=0.5pt, color1, opacity=1, forget plot]
table {%
3.06 0.0499999999999999
3.06 0
};
\addplot [line width=0.5pt, color1, opacity=1, forget plot]
table {%
3.06 0.200502988632573
3.06 0.363388067097552
};
\addplot [line width=0.5pt, color1, forget plot]
table {%
3.04 0
3.08 0
};
\addplot [line width=0.5pt, color1, forget plot]
table {%
3.04 0.363388067097552
3.08 0.363388067097552
};
\addplot [line width=0.5pt, color1, opacity=1, forget plot]
table {%
4.02 0.0643524079633387
4.1 0.0643524079633387
4.1 0.496862260542846
4.02 0.496862260542846
4.02 0.0643524079633387
};
\addplot [line width=0.5pt, color1, opacity=1, forget plot]
table {%
4.06 0.0643524079633387
4.06 0
};
\addplot [line width=0.5pt, color1, opacity=1, forget plot]
table {%
4.06 0.496862260542846
4.06 1.11576707013848
};
\addplot [line width=0.5pt, color1, forget plot]
table {%
4.04 0
4.08 0
};
\addplot [line width=0.5pt, color1, forget plot]
table {%
4.04 1.11576707013848
4.08 1.11576707013848
};
\addplot [line width=0.5pt, color1, opacity=1, forget plot]
table {%
5.02 0.108683722995716
5.1 0.108683722995716
5.1 0.572546565452337
5.02 0.572546565452337
5.02 0.108683722995716
};
\addplot [line width=0.5pt, color1, opacity=1, forget plot]
table {%
5.06 0.108683722995716
5.06 0.0166666666666666
};
\addplot [line width=0.5pt, color1, opacity=1, forget plot]
table {%
5.06 0.572546565452337
5.06 1.0482772437926
};
\addplot [line width=0.5pt, color1, forget plot]
table {%
5.04 0.0166666666666666
5.08 0.0166666666666666
};
\addplot [line width=0.5pt, color1, forget plot]
table {%
5.04 1.0482772437926
5.08 1.0482772437926
};
\addplot [line width=0.5pt, color1, opacity=1, forget plot]
table {%
6.02 0.145187344125388
6.1 0.145187344125388
6.1 0.575463007395674
6.02 0.575463007395674
6.02 0.145187344125388
};
\addplot [line width=0.5pt, color1, opacity=1, forget plot]
table {%
6.06 0.145187344125388
6.06 0.0333333333333334
};
\addplot [line width=0.5pt, color1, opacity=1, forget plot]
table {%
6.06 0.575463007395674
6.06 1.07274917181732
};
\addplot [line width=0.5pt, color1, forget plot]
table {%
6.04 0.0333333333333334
6.08 0.0333333333333334
};
\addplot [line width=0.5pt, color1, forget plot]
table {%
6.04 1.07274917181732
6.08 1.07274917181732
};
\addplot [line width=0.5pt, color2, opacity=1, forget plot]
table {%
1.14 0
1.22 0
1.22 0.0500000000000003
1.14 0.0500000000000003
1.14 0
};
\addplot [line width=0.5pt, color2, opacity=1, forget plot]
table {%
1.18 0
1.18 0
};
\addplot [line width=0.5pt, color2, opacity=1, forget plot]
table {%
1.18 0.0500000000000003
1.18 0.11180339887499
};
\addplot [line width=0.5pt, color2, forget plot]
table {%
1.16 0
1.2 0
};
\addplot [line width=0.5pt, color2, forget plot]
table {%
1.16 0.11180339887499
1.2 0.11180339887499
};
\addplot [line width=0.5pt, color2, opacity=1, forget plot]
table {%
2.14 0
2.22 0
2.22 0.148588511894765
2.14 0.148588511894765
2.14 0
};
\addplot [line width=0.5pt, color2, opacity=1, forget plot]
table {%
2.18 0
2.18 0
};
\addplot [line width=0.5pt, color2, opacity=1, forget plot]
table {%
2.18 0.148588511894765
2.18 0.288342717995763
};
\addplot [line width=0.5pt, color2, forget plot]
table {%
2.16 0
2.2 0
};
\addplot [line width=0.5pt, color2, forget plot]
table {%
2.16 0.288342717995763
2.2 0.288342717995763
};
\addplot [line width=0.5pt, color2, opacity=1, forget plot]
table {%
3.14 0.0570151093429529
3.22 0.0570151093429529
3.22 0.533143425204337
3.14 0.533143425204337
3.14 0.0570151093429529
};
\addplot [line width=0.5pt, color2, opacity=1, forget plot]
table {%
3.18 0.0570151093429529
3.18 0
};
\addplot [line width=0.5pt, color2, opacity=1, forget plot]
table {%
3.18 0.533143425204337
3.18 1.08770179336765
};
\addplot [line width=0.5pt, color2, forget plot]
table {%
3.16 0
3.2 0
};
\addplot [line width=0.5pt, color2, forget plot]
table {%
3.16 1.08770179336765
3.2 1.08770179336765
};
\addplot [line width=0.5pt, color2, opacity=1, forget plot]
table {%
4.14 0.0656120874743622
4.22 0.0656120874743622
4.22 0.446612954754479
4.14 0.446612954754479
4.14 0.0656120874743622
};
\addplot [line width=0.5pt, color2, opacity=1, forget plot]
table {%
4.18 0.0656120874743622
4.18 0
};
\addplot [line width=0.5pt, color2, opacity=1, forget plot]
table {%
4.18 0.446612954754479
4.18 0.951155006329504
};
\addplot [line width=0.5pt, color2, forget plot]
table {%
4.16 0
4.2 0
};
\addplot [line width=0.5pt, color2, forget plot]
table {%
4.16 0.951155006329504
4.2 0.951155006329504
};
\addplot [line width=0.5pt, color2, opacity=1, forget plot]
table {%
5.14 0.107462360261657
5.22 0.107462360261657
5.22 0.737032410328739
5.14 0.737032410328739
5.14 0.107462360261657
};
\addplot [line width=0.5pt, color2, opacity=1, forget plot]
table {%
5.18 0.107462360261657
5.18 0
};
\addplot [line width=0.5pt, color2, opacity=1, forget plot]
table {%
5.18 0.737032410328739
5.18 1.14490321324996
};
\addplot [line width=0.5pt, color2, forget plot]
table {%
5.16 0
5.2 0
};
\addplot [line width=0.5pt, color2, forget plot]
table {%
5.16 1.14490321324996
5.2 1.14490321324996
};
\addplot [line width=0.5pt, color2, opacity=1, forget plot]
table {%
6.14 0.156388027159492
6.22 0.156388027159492
6.22 0.480443884574725
6.14 0.480443884574725
6.14 0.156388027159492
};
\addplot [line width=0.5pt, color2, opacity=1, forget plot]
table {%
6.18 0.156388027159492
6.18 0.0166666666666666
};
\addplot [line width=0.5pt, color2, opacity=1, forget plot]
table {%
6.18 0.480443884574725
6.18 0.960318215690069
};
\addplot [line width=0.5pt, color2, forget plot]
table {%
6.16 0.0166666666666666
6.2 0.0166666666666666
};
\addplot [line width=0.5pt, color2, forget plot]
table {%
6.16 0.960318215690069
6.2 0.960318215690069
};
\addplot [line width=0.5pt, white!82.745098039215677!black, opacity=1, forget plot]
table {%
1.26 0
1.34 0
1.34 0.0999999999999999
1.26 0.0999999999999999
1.26 0
};
\addplot [line width=0.5pt, white!82.745098039215677!black, opacity=1, forget plot]
table {%
1.3 0
1.3 0
};
\addplot [line width=0.5pt, white!82.745098039215677!black, opacity=1, forget plot]
table {%
1.3 0.0999999999999999
1.3 0.206155281280883
};
\addplot [line width=0.5pt, white!82.745098039215677!black, forget plot]
table {%
1.28 0
1.32 0
};
\addplot [line width=0.5pt, white!82.745098039215677!black, forget plot]
table {%
1.28 0.206155281280883
1.32 0.206155281280883
};
\addplot [line width=0.5pt, white!82.745098039215677!black, opacity=1, forget plot]
table {%
2.26 0.0333333333333332
2.34 0.0333333333333332
2.34 0.110947960066361
2.26 0.110947960066361
2.26 0.0333333333333332
};
\addplot [line width=0.5pt, white!82.745098039215677!black, opacity=1, forget plot]
table {%
2.3 0.0333333333333332
2.3 0
};
\addplot [line width=0.5pt, white!82.745098039215677!black, opacity=1, forget plot]
table {%
2.3 0.110947960066361
2.3 0.215956955487302
};
\addplot [line width=0.5pt, white!82.745098039215677!black, forget plot]
table {%
2.28 0
2.32 0
};
\addplot [line width=0.5pt, white!82.745098039215677!black, forget plot]
table {%
2.28 0.215956955487302
2.32 0.215956955487302
};
\addplot [line width=0.5pt, white!82.745098039215677!black, opacity=1, forget plot]
table {%
3.26 0.0463388347648318
3.34 0.0463388347648318
3.34 0.176357998824763
3.26 0.176357998824763
3.26 0.0463388347648318
};
\addplot [line width=0.5pt, white!82.745098039215677!black, opacity=1, forget plot]
table {%
3.3 0.0463388347648318
3.3 0
};
\addplot [line width=0.5pt, white!82.745098039215677!black, opacity=1, forget plot]
table {%
3.3 0.176357998824763
3.3 0.318198051533946
};
\addplot [line width=0.5pt, white!82.745098039215677!black, forget plot]
table {%
3.28 0
3.32 0
};
\addplot [line width=0.5pt, white!82.745098039215677!black, forget plot]
table {%
3.28 0.318198051533946
3.32 0.318198051533946
};
\addplot [line width=0.5pt, white!82.745098039215677!black, opacity=1, forget plot]
table {%
4.26 0.06
4.34 0.06
4.34 0.337278689602681
4.26 0.337278689602681
4.26 0.06
};
\addplot [line width=0.5pt, white!82.745098039215677!black, opacity=1, forget plot]
table {%
4.3 0.06
4.3 0
};
\addplot [line width=0.5pt, white!82.745098039215677!black, opacity=1, forget plot]
table {%
4.3 0.337278689602681
4.3 0.751053592633738
};
\addplot [line width=0.5pt, white!82.745098039215677!black, forget plot]
table {%
4.28 0
4.32 0
};
\addplot [line width=0.5pt, white!82.745098039215677!black, forget plot]
table {%
4.28 0.751053592633738
4.32 0.751053592633738
};
\addplot [line width=0.5pt, white!82.745098039215677!black, opacity=1, forget plot]
table {%
5.26 0.0924454691546602
5.34 0.0924454691546602
5.34 0.56342973571202
5.26 0.56342973571202
5.26 0.0924454691546602
};
\addplot [line width=0.5pt, white!82.745098039215677!black, opacity=1, forget plot]
table {%
5.3 0.0924454691546602
5.3 0
};
\addplot [line width=0.5pt, white!82.745098039215677!black, opacity=1, forget plot]
table {%
5.3 0.56342973571202
5.3 1.18411347530493
};
\addplot [line width=0.5pt, white!82.745098039215677!black, forget plot]
table {%
5.28 0
5.32 0
};
\addplot [line width=0.5pt, white!82.745098039215677!black, forget plot]
table {%
5.28 1.18411347530493
5.32 1.18411347530493
};
\addplot [line width=0.5pt, white!82.745098039215677!black, opacity=1, forget plot]
table {%
6.26 0.104044011451988
6.34 0.104044011451988
6.34 0.525525655933607
6.26 0.525525655933607
6.26 0.104044011451988
};
\addplot [line width=0.5pt, white!82.745098039215677!black, opacity=1, forget plot]
table {%
6.3 0.104044011451988
6.3 0
};
\addplot [line width=0.5pt, white!82.745098039215677!black, opacity=1, forget plot]
table {%
6.3 0.525525655933607
6.3 1.02057141610677
};
\addplot [line width=0.5pt, white!82.745098039215677!black, forget plot]
table {%
6.28 0
6.32 0
};
\addplot [line width=0.5pt, white!82.745098039215677!black, forget plot]
table {%
6.28 1.02057141610677
6.32 1.02057141610677
};
\addplot [line width=0.5pt, white!66.274509803921561!black, opacity=1, forget plot]
table {%
0.66 0.0499999999999998
0.74 0.0499999999999998
};
\addplot [line width=0.5pt, white!66.274509803921561!black, dashed, mark=x, mark size=3, mark options={solid}, forget plot]
table {%
0.7 0.122695978407104
};
\addplot [line width=0.5pt, white!66.274509803921561!black, opacity=1, forget plot]
table {%
1.66 0.0666666666666667
1.74 0.0666666666666667
};
\addplot [line width=0.5pt, white!66.274509803921561!black, dashed, mark=x, mark size=3, mark options={solid}, forget plot]
table {%
1.7 0.190526332137489
};
\addplot [line width=0.5pt, white!66.274509803921561!black, opacity=1, forget plot]
table {%
2.66 0.120710678118655
2.74 0.120710678118655
};
\addplot [line width=0.5pt, white!66.274509803921561!black, dashed, mark=x, mark size=3, mark options={solid}, forget plot]
table {%
2.7 0.253785193942022
};
\addplot [line width=0.5pt, white!66.274509803921561!black, opacity=1, forget plot]
table {%
3.66 0.186409863247875
3.74 0.186409863247875
};
\addplot [line width=0.5pt, white!66.274509803921561!black, dashed, mark=x, mark size=3, mark options={solid}, forget plot]
table {%
3.7 0.310250409120604
};
\addplot [line width=0.5pt, white!66.274509803921561!black, opacity=1, forget plot]
table {%
4.66 0.269561458835578
4.74 0.269561458835578
};
\addplot [line width=0.5pt, white!66.274509803921561!black, dashed, mark=x, mark size=3, mark options={solid}, forget plot]
table {%
4.7 0.371796126901667
};
\addplot [line width=0.5pt, white!66.274509803921561!black, opacity=1, forget plot]
table {%
5.66 0.268706213145209
5.74 0.268706213145209
};
\addplot [line width=0.5pt, white!66.274509803921561!black, dashed, mark=x, mark size=3, mark options={solid}, forget plot]
table {%
5.7 0.376976654515578
};
\addplot [line width=0.5pt, black, opacity=1, forget plot]
table {%
0.78 0
0.86 0
};
\addplot [line width=0.5pt, black, dashed, mark=x, mark size=3, mark options={solid}, forget plot]
table {%
0.82 0.0719004525445583
};
\addplot [line width=0.5pt, black, opacity=1, forget plot]
table {%
1.78 0.0471404520791031
1.86 0.0471404520791031
};
\addplot [line width=0.5pt, black, dashed, mark=x, mark size=3, mark options={solid}, forget plot]
table {%
1.82 0.0885007885950777
};
\addplot [line width=0.5pt, black, opacity=1, forget plot]
table {%
2.78 0.117924730002627
2.86 0.117924730002627
};
\addplot [line width=0.5pt, black, dashed, mark=x, mark size=3, mark options={solid}, forget plot]
table {%
2.82 0.242017362996996
};
\addplot [line width=0.5pt, black, opacity=1, forget plot]
table {%
3.78 0.160284403507339
3.86 0.160284403507339
};
\addplot [line width=0.5pt, black, dashed, mark=x, mark size=3, mark options={solid}, forget plot]
table {%
3.82 0.289761061277818
};
\addplot [line width=0.5pt, black, opacity=1, forget plot]
table {%
4.78 0.343932255102082
4.86 0.343932255102082
};
\addplot [line width=0.5pt, black, dashed, mark=x, mark size=3, mark options={solid}, forget plot]
table {%
4.82 0.420065351073176
};
\addplot [line width=0.5pt, black, opacity=1, forget plot]
table {%
5.78 0.384121047967837
5.86 0.384121047967837
};
\addplot [line width=0.5pt, black, dashed, mark=x, mark size=3, mark options={solid}, forget plot]
table {%
5.82 0.467410469404451
};
\addplot [line width=0.5pt, color0, opacity=1, forget plot]
table {%
0.9 0.0499999999999998
0.98 0.0499999999999998
};
\addplot [line width=0.5pt, color0, dashed, mark=x, mark size=3, mark options={solid}, forget plot]
table {%
0.94 0.111913047175908
};
\addplot [line width=0.5pt, color0, opacity=1, forget plot]
table {%
1.9 0.0666666666666667
1.98 0.0666666666666667
};
\addplot [line width=0.5pt, color0, dashed, mark=x, mark size=3, mark options={solid}, forget plot]
table {%
1.94 0.131096196325793
};
\addplot [line width=0.5pt, color0, opacity=1, forget plot]
table {%
2.9 0.132440519757716
2.98 0.132440519757716
};
\addplot [line width=0.5pt, color0, dashed, mark=x, mark size=3, mark options={solid}, forget plot]
table {%
2.94 0.246265481284227
};
\addplot [line width=0.5pt, color0, opacity=1, forget plot]
table {%
3.9 0.130776872304636
3.98 0.130776872304636
};
\addplot [line width=0.5pt, color0, dashed, mark=x, mark size=3, mark options={solid}, forget plot]
table {%
3.94 0.27693533407298
};
\addplot [line width=0.5pt, color0, opacity=1, forget plot]
table {%
4.9 0.318295295019906
4.98 0.318295295019906
};
\addplot [line width=0.5pt, color0, dashed, mark=x, mark size=3, mark options={solid}, forget plot]
table {%
4.94 0.381716497692815
};
\addplot [line width=0.5pt, color0, opacity=1, forget plot]
table {%
5.9 0.389947600955219
5.98 0.389947600955219
};
\addplot [line width=0.5pt, color0, dashed, mark=x, mark size=3, mark options={solid}, forget plot]
table {%
5.94 0.428940196613944
};
\addplot [line width=0.5pt, color1, opacity=1, forget plot]
table {%
1.02 0.0499999999999998
1.1 0.0499999999999998
};
\addplot [line width=0.5pt, color1, dashed, mark=x, mark size=3, mark options={solid}, forget plot]
table {%
1.06 0.0602732091731273
};
\addplot [line width=0.5pt, color1, opacity=1, forget plot]
table {%
2.02 0.0471404520791031
2.1 0.0471404520791031
};
\addplot [line width=0.5pt, color1, dashed, mark=x, mark size=3, mark options={solid}, forget plot]
table {%
2.06 0.122888336897905
};
\addplot [line width=0.5pt, color1, opacity=1, forget plot]
table {%
3.02 0.0901387818865997
3.1 0.0901387818865997
};
\addplot [line width=0.5pt, color1, dashed, mark=x, mark size=3, mark options={solid}, forget plot]
table {%
3.06 0.220359566279767
};
\addplot [line width=0.5pt, color1, opacity=1, forget plot]
table {%
4.02 0.17365746312737
4.1 0.17365746312737
};
\addplot [line width=0.5pt, color1, dashed, mark=x, mark size=3, mark options={solid}, forget plot]
table {%
4.06 0.306598946420165
};
\addplot [line width=0.5pt, color1, opacity=1, forget plot]
table {%
5.02 0.296098611126347
5.1 0.296098611126347
};
\addplot [line width=0.5pt, color1, dashed, mark=x, mark size=3, mark options={solid}, forget plot]
table {%
5.06 0.361365226108834
};
\addplot [line width=0.5pt, color1, opacity=1, forget plot]
table {%
6.02 0.338307562119915
6.1 0.338307562119915
};
\addplot [line width=0.5pt, color1, dashed, mark=x, mark size=3, mark options={solid}, forget plot]
table {%
6.06 0.394374687630961
};
\addplot [line width=0.5pt, color2, opacity=1, forget plot]
table {%
1.14 0
1.22 0
};
\addplot [line width=0.5pt, color2, dashed, mark=x, mark size=3, mark options={solid}, forget plot]
table {%
1.18 0.0457732178469096
};
\addplot [line width=0.5pt, color2, opacity=1, forget plot]
table {%
2.14 0.0402368927062183
2.22 0.0402368927062183
};
\addplot [line width=0.5pt, color2, dashed, mark=x, mark size=3, mark options={solid}, forget plot]
table {%
2.18 0.147899180857366
};
\addplot [line width=0.5pt, color2, opacity=1, forget plot]
table {%
3.14 0.147458640941704
3.22 0.147458640941704
};
\addplot [line width=0.5pt, color2, dashed, mark=x, mark size=3, mark options={solid}, forget plot]
table {%
3.18 0.313126677847469
};
\addplot [line width=0.5pt, color2, opacity=1, forget plot]
table {%
4.14 0.138507167326824
4.22 0.138507167326824
};
\addplot [line width=0.5pt, color2, dashed, mark=x, mark size=3, mark options={solid}, forget plot]
table {%
4.18 0.276455373330559
};
\addplot [line width=0.5pt, color2, opacity=1, forget plot]
table {%
5.14 0.218234447106176
5.22 0.218234447106176
};
\addplot [line width=0.5pt, color2, dashed, mark=x, mark size=3, mark options={solid}, forget plot]
table {%
5.18 0.391850798472762
};
\addplot [line width=0.5pt, color2, opacity=1, forget plot]
table {%
6.14 0.277112198280526
6.22 0.277112198280526
};
\addplot [line width=0.5pt, color2, dashed, mark=x, mark size=3, mark options={solid}, forget plot]
table {%
6.18 0.363832789504058
};
\addplot [line width=0.5pt, white!82.745098039215677!black, opacity=1, forget plot]
table {%
1.26 0.0249999999999999
1.34 0.0249999999999999
};
\addplot [line width=0.5pt, white!82.745098039215677!black, dashed, mark=x, mark size=3, mark options={solid}, forget plot]
table {%
1.3 0.112476274146538
};
\addplot [line width=0.5pt, white!82.745098039215677!black, opacity=1, forget plot]
table {%
2.26 0.0471404520791032
2.34 0.0471404520791032
};
\addplot [line width=0.5pt, white!82.745098039215677!black, dashed, mark=x, mark size=3, mark options={solid}, forget plot]
table {%
2.3 0.135582541559664
};
\addplot [line width=0.5pt, white!82.745098039215677!black, opacity=1, forget plot]
table {%
3.26 0.0883061887780748
3.34 0.0883061887780748
};
\addplot [line width=0.5pt, white!82.745098039215677!black, dashed, mark=x, mark size=3, mark options={solid}, forget plot]
table {%
3.3 0.181999863101947
};
\addplot [line width=0.5pt, white!82.745098039215677!black, opacity=1, forget plot]
table {%
4.26 0.0925583281533688
4.34 0.0925583281533688
};
\addplot [line width=0.5pt, white!82.745098039215677!black, dashed, mark=x, mark size=3, mark options={solid}, forget plot]
table {%
4.3 0.222701026280471
};
\addplot [line width=0.5pt, white!82.745098039215677!black, opacity=1, forget plot]
table {%
5.26 0.31257754329947
5.34 0.31257754329947
};
\addplot [line width=0.5pt, white!82.745098039215677!black, dashed, mark=x, mark size=3, mark options={solid}, forget plot]
table {%
5.3 0.371087207816928
};
\addplot [line width=0.5pt, white!82.745098039215677!black, opacity=1, forget plot]
table {%
6.26 0.302311719870146
6.34 0.302311719870146
};
\addplot [line width=0.5pt, white!82.745098039215677!black, dashed, mark=x, mark size=3, mark options={solid}, forget plot]
table {%
6.3 0.356657188371931
};
\end{axis}

\node at ({$(current bounding box.south west)!0.5!(current bounding box.south east)$}|-{$(current bounding box.south west)!0.98!(current bounding box.north west)$})[
  anchor=north,
  text=black,
  rotate=0.0
]{ };

	    \input{helpers/legends/variance-est-legend}
	    
	\end{tikzpicture}
	
	\caption{Evaluation results for estimated variance with different $\sigma^2_0$}
	\label{fig:boxplotVariance}
\end{figure}

\begin{itemize}
    \item Localisation performance basically identical across different $sigma^2_0$
    \item Looks like variance converges to a value quickly, so different initialisations have no effect
    \item Now of interest, which value the variance converges to when estimated
    \item Fixed variance trials allow for the assumption, that estimated variance converges to a value between 1 and 3
\end{itemize}
