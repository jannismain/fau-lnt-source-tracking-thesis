\subsubsection{SNR}
%\begin{figure}[H]
%    \centering
%    \begin{subfigure}{0.49\textwidth}
%          \centering
%	       \includestandalone[width=\textwidth]{data/plots/plot_snr_err-mean}
%%            % This file was created by matplotlib2tikz v0.6.13.
\begin{tikzpicture}

\definecolor{color0}{rgb}{0.8,0.207843137254902,0.219607843137255}
\definecolor{color1}{rgb}{1,0.647058823529412,0}
\definecolor{color2}{rgb}{1,0,1}

\begin{axis}[
title={err-mean},
xlabel={number of sources},
xmin=2, xmax=7,
ymin=0, ymax=1,
width=\figurewidth,
height=\figureheight,
tick align=outside,
tick pos=left,
x grid style={lightgray!92.026143790849673!black},
ymajorgrids,
y grid style={lightgray!92.026143790849673!black},
legend entries={{1.0},{2.0},{3.0},{5.0},{10.0},{20.0}},
legend style={at={(0.97,0.03)}, anchor=south east, draw=white!80.0!black},
legend cell align={left}
]
\addlegendimage{no markers, color0}
\addlegendimage{no markers, color1}
\addlegendimage{no markers, black}
\addlegendimage{no markers, lightgray!88.366013071895424!black}
\addlegendimage{no markers, blue}
\addlegendimage{no markers, color2}
\addplot [semithick, color0, mark=*, mark size=3, mark options={solid}]
table {%
2 0.290719260974231
3 0.479301880051738
4 0.644171288750938
5 0.744992267692781
6 0.815194102350153
7 0.80761917101706
};
\addplot [semithick, color1, mark=*, mark size=3, mark options={solid}]
table {%
2 0.245694149597362
3 0.421649843327725
4 0.500562120786198
5 0.655687681534857
6 0.633687481891525
7 0.793106398964979
};
\addplot [semithick, black, mark=*, mark size=3, mark options={solid}]
table {%
2 0.244192685094324
3 0.318939976510832
4 0.436516755504563
5 0.540639551860509
6 0.618377832825832
7 0.708533647793283
};
\addplot [semithick, lightgray!88.366013071895424!black, mark=*, mark size=3, mark options={solid}]
table {%
2 0.151357304870224
3 0.298658572932984
4 0.409077232118482
5 0.502185647164528
6 0.601037546098582
7 0.635845669023758
};
\addplot [semithick, blue, mark=*, mark size=3, mark options={solid}]
table {%
2 0.156145785073103
3 0.262900994349871
4 0.444291990749065
5 0.502497456459568
6 0.536465373800886
7 0.635142635271785
};
\addplot [semithick, color2, mark=*, mark size=3, mark options={solid}]
table {%
2 0.158641822221236
3 0.292719119608
4 0.407112096408886
5 0.458557818271804
6 0.545724599411615
7 0.603303151192354
};
\end{axis}

\end{tikzpicture}
%	\end{subfigure}
%    \begin{subfigure}{0.49\textwidth}
%          \centering
%	       \includestandalone[width=\textwidth]{data/plots/plot_snr_percent-matched}
%%            % This file was created by matplotlib2tikz v0.6.13.
\begin{tikzpicture}

\definecolor{color0}{rgb}{0.8,0.207843137254902,0.219607843137255}
\definecolor{color1}{rgb}{1,0.647058823529412,0}
\definecolor{color2}{rgb}{1,0,1}

\begin{axis}[
title={err-mean},
xlabel={number of sources},
xmin=2, xmax=7,
ymin=0, ymax=1,
width=\figurewidth,
height=\figureheight,
tick align=outside,
tick pos=left,
x grid style={lightgray!92.026143790849673!black},
ymajorgrids,
y grid style={lightgray!92.026143790849673!black},
legend entries={{1.0},{2.0},{3.0},{5.0},{10.0},{20.0}},
legend style={at={(0.97,0.03)}, anchor=south east, draw=white!80.0!black},
legend cell align={left}
]
\addlegendimage{no markers, color0}
\addlegendimage{no markers, color1}
\addlegendimage{no markers, black}
\addlegendimage{no markers, lightgray!88.366013071895424!black}
\addlegendimage{no markers, blue}
\addlegendimage{no markers, color2}
\addplot [semithick, color0, mark=*, mark size=3, mark options={solid}]
table {%
2 0.290719260974231
3 0.479301880051738
4 0.644171288750938
5 0.744992267692781
6 0.815194102350153
7 0.80761917101706
};
\addplot [semithick, color1, mark=*, mark size=3, mark options={solid}]
table {%
2 0.245694149597362
3 0.421649843327725
4 0.500562120786198
5 0.655687681534857
6 0.633687481891525
7 0.793106398964979
};
\addplot [semithick, black, mark=*, mark size=3, mark options={solid}]
table {%
2 0.244192685094324
3 0.318939976510832
4 0.436516755504563
5 0.540639551860509
6 0.618377832825832
7 0.708533647793283
};
\addplot [semithick, lightgray!88.366013071895424!black, mark=*, mark size=3, mark options={solid}]
table {%
2 0.151357304870224
3 0.298658572932984
4 0.409077232118482
5 0.502185647164528
6 0.601037546098582
7 0.635845669023758
};
\addplot [semithick, blue, mark=*, mark size=3, mark options={solid}]
table {%
2 0.156145785073103
3 0.262900994349871
4 0.444291990749065
5 0.502497456459568
6 0.536465373800886
7 0.635142635271785
};
\addplot [semithick, color2, mark=*, mark size=3, mark options={solid}]
table {%
2 0.158641822221236
3 0.292719119608
4 0.407112096408886
5 0.458557818271804
6 0.545724599411615
7 0.603303151192354
};
\end{axis}

\end{tikzpicture}
%	\end{subfigure}
%\caption{Evaluation of SNR (EM-Iterations=5, T60=0.3, refl-ord=3, n=200)}
%\end{figure}

%% all-in-one boxplot
\begin{figure}[H]
    \setlength\figureheight{7cm}
    \small
    \setlength\figurewidth{\textwidth}
	\centering
	\begin{tikzpicture}
	    \footnotesize
	    % This file was created by matplotlib2tikz v0.6.14.
\definecolor{color0}{rgb}{0.8,0.207843137254902,0.219607843137255}
\definecolor{color1}{rgb}{0.67843137254902,0.847058823529412,0.901960784313726}
\definecolor{color2}{rgb}{1,0.647058823529412,0}

\begin{axis}[
xmin=0.5, xmax=6.5,
ymin=0, ymax=1.75,
width=\figurewidth,
height=\figureheight,
xtick={1,2,3,4,5,6},
xticklabels={2,3,4,5,6,7},
ytick={0,0.25,0.5,0.75,1,1.25,1.5,1.75},
minor xtick={},
minor ytick={},
tick align=outside,
tick pos=left,
x grid style={white!69.019607843137251!black},
ymajorgrids,
y grid style={white!69.019607843137251!black}
]
\addplot [line width=0.5pt, white!66.274509803921561!black, opacity=1, forget plot]
table {%
0.71 0
0.79 0
0.79 0.14142135623731
0.71 0.14142135623731
0.71 0
};
\addplot [line width=0.5pt, white!66.274509803921561!black, opacity=1, forget plot]
table {%
0.75 0
0.75 0
};
\addplot [line width=0.5pt, white!66.274509803921561!black, opacity=1, forget plot]
table {%
0.75 0.14142135623731
0.75 0.341547594742265
};
\addplot [line width=0.5pt, white!66.274509803921561!black, forget plot]
table {%
0.73 0
0.77 0
};
\addplot [line width=0.5pt, white!66.274509803921561!black, forget plot]
table {%
0.73 0.341547594742265
0.77 0.341547594742265
};
\addplot [line width=0.5pt, white!66.274509803921561!black, opacity=1, forget plot]
table {%
1.71 0.0333333333333332
1.79 0.0333333333333332
1.79 0.179615333825597
1.71 0.179615333825597
1.71 0.0333333333333332
};
\addplot [line width=0.5pt, white!66.274509803921561!black, opacity=1, forget plot]
table {%
1.75 0.0333333333333332
1.75 0
};
\addplot [line width=0.5pt, white!66.274509803921561!black, opacity=1, forget plot]
table {%
1.75 0.179615333825597
1.75 0.389001636985213
};
\addplot [line width=0.5pt, white!66.274509803921561!black, forget plot]
table {%
1.73 0
1.77 0
};
\addplot [line width=0.5pt, white!66.274509803921561!black, forget plot]
table {%
1.73 0.389001636985213
1.77 0.389001636985213
};
\addplot [line width=0.5pt, white!66.274509803921561!black, opacity=1, forget plot]
table {%
2.71 0.0707106781186548
2.79 0.0707106781186548
2.79 0.485121287580844
2.71 0.485121287580844
2.71 0.0707106781186548
};
\addplot [line width=0.5pt, white!66.274509803921561!black, opacity=1, forget plot]
table {%
2.75 0.0707106781186548
2.75 0
};
\addplot [line width=0.5pt, white!66.274509803921561!black, opacity=1, forget plot]
table {%
2.75 0.485121287580844
2.75 1.07648945844305
};
\addplot [line width=0.5pt, white!66.274509803921561!black, forget plot]
table {%
2.73 0
2.77 0
};
\addplot [line width=0.5pt, white!66.274509803921561!black, forget plot]
table {%
2.73 1.07648945844305
2.77 1.07648945844305
};
\addplot [line width=0.5pt, white!66.274509803921561!black, opacity=1, forget plot]
table {%
3.71 0.0809016994374947
3.79 0.0809016994374947
3.79 0.487994923610762
3.71 0.487994923610762
3.71 0.0809016994374947
};
\addplot [line width=0.5pt, white!66.274509803921561!black, opacity=1, forget plot]
table {%
3.75 0.0809016994374947
3.75 0
};
\addplot [line width=0.5pt, white!66.274509803921561!black, opacity=1, forget plot]
table {%
3.75 0.487994923610762
3.75 1.05643623705788
};
\addplot [line width=0.5pt, white!66.274509803921561!black, forget plot]
table {%
3.73 0
3.77 0
};
\addplot [line width=0.5pt, white!66.274509803921561!black, forget plot]
table {%
3.73 1.05643623705788
3.77 1.05643623705788
};
\addplot [line width=0.5pt, white!66.274509803921561!black, opacity=1, forget plot]
table {%
4.71 0.0912570384968223
4.79 0.0912570384968223
4.79 0.557056691605868
4.71 0.557056691605868
4.71 0.0912570384968223
};
\addplot [line width=0.5pt, white!66.274509803921561!black, opacity=1, forget plot]
table {%
4.75 0.0912570384968223
4.75 0
};
\addplot [line width=0.5pt, white!66.274509803921561!black, opacity=1, forget plot]
table {%
4.75 0.557056691605868
4.75 1.24765875723692
};
\addplot [line width=0.5pt, white!66.274509803921561!black, forget plot]
table {%
4.73 0
4.77 0
};
\addplot [line width=0.5pt, white!66.274509803921561!black, forget plot]
table {%
4.73 1.24765875723692
4.77 1.24765875723692
};
\addplot [line width=0.5pt, white!66.274509803921561!black, opacity=1, forget plot]
table {%
5.71 0.109241929153869
5.79 0.109241929153869
5.79 0.42630019628071
5.71 0.42630019628071
5.71 0.109241929153869
};
\addplot [line width=0.5pt, white!66.274509803921561!black, opacity=1, forget plot]
table {%
5.75 0.109241929153869
5.75 0
};
\addplot [line width=0.5pt, white!66.274509803921561!black, opacity=1, forget plot]
table {%
5.75 0.42630019628071
5.75 0.885828603018025
};
\addplot [line width=0.5pt, white!66.274509803921561!black, forget plot]
table {%
5.73 0
5.77 0
};
\addplot [line width=0.5pt, white!66.274509803921561!black, forget plot]
table {%
5.73 0.885828603018025
5.77 0.885828603018025
};
\addplot [line width=0.5pt, black, opacity=1, forget plot]
table {%
0.835 0
0.915 0
0.915 0.152028470752105
0.835 0.152028470752105
0.835 0
};
\addplot [line width=0.5pt, black, opacity=1, forget plot]
table {%
0.875 0
0.875 0
};
\addplot [line width=0.5pt, black, opacity=1, forget plot]
table {%
0.875 0.152028470752105
0.875 0.35
};
\addplot [line width=0.5pt, black, forget plot]
table {%
0.855 0
0.895 0
};
\addplot [line width=0.5pt, black, forget plot]
table {%
0.855 0.35
0.895 0.35
};
\addplot [line width=0.5pt, black, opacity=1, forget plot]
table {%
1.835 0.0333333333333332
1.915 0.0333333333333332
1.915 0.173933314896997
1.835 0.173933314896997
1.835 0.0333333333333332
};
\addplot [line width=0.5pt, black, opacity=1, forget plot]
table {%
1.875 0.0333333333333332
1.875 0
};
\addplot [line width=0.5pt, black, opacity=1, forget plot]
table {%
1.875 0.173933314896997
1.875 0.360555127546399
};
\addplot [line width=0.5pt, black, forget plot]
table {%
1.855 0
1.895 0
};
\addplot [line width=0.5pt, black, forget plot]
table {%
1.855 0.360555127546399
1.895 0.360555127546399
};
\addplot [line width=0.5pt, black, opacity=1, forget plot]
table {%
2.835 0.05
2.915 0.05
2.915 0.238703965836182
2.835 0.238703965836182
2.835 0.05
};
\addplot [line width=0.5pt, black, opacity=1, forget plot]
table {%
2.875 0.05
2.875 0
};
\addplot [line width=0.5pt, black, opacity=1, forget plot]
table {%
2.875 0.238703965836182
2.875 0.521404587551653
};
\addplot [line width=0.5pt, black, forget plot]
table {%
2.855 0
2.895 0
};
\addplot [line width=0.5pt, black, forget plot]
table {%
2.855 0.521404587551653
2.895 0.521404587551653
};
\addplot [line width=0.5pt, black, opacity=1, forget plot]
table {%
3.835 0.0500000000000002
3.915 0.0500000000000002
3.915 0.506133256749339
3.835 0.506133256749339
3.835 0.0500000000000002
};
\addplot [line width=0.5pt, black, opacity=1, forget plot]
table {%
3.875 0.0500000000000002
3.875 0
};
\addplot [line width=0.5pt, black, opacity=1, forget plot]
table {%
3.875 0.506133256749339
3.875 1.18819127855732
};
\addplot [line width=0.5pt, black, forget plot]
table {%
3.855 0
3.895 0
};
\addplot [line width=0.5pt, black, forget plot]
table {%
3.855 1.18819127855732
3.895 1.18819127855732
};
\addplot [line width=0.5pt, black, opacity=1, forget plot]
table {%
4.835 0.0901387818865997
4.915 0.0901387818865997
4.915 0.552739482038997
4.835 0.552739482038997
4.835 0.0901387818865997
};
\addplot [line width=0.5pt, black, opacity=1, forget plot]
table {%
4.875 0.0901387818865997
4.875 0
};
\addplot [line width=0.5pt, black, opacity=1, forget plot]
table {%
4.875 0.552739482038997
4.875 1.22736113858953
};
\addplot [line width=0.5pt, black, forget plot]
table {%
4.855 0
4.895 0
};
\addplot [line width=0.5pt, black, forget plot]
table {%
4.855 1.22736113858953
4.895 1.22736113858953
};
\addplot [line width=0.5pt, black, opacity=1, forget plot]
table {%
5.835 0.102308230480331
5.915 0.102308230480331
5.915 0.415552387633531
5.835 0.415552387633531
5.835 0.102308230480331
};
\addplot [line width=0.5pt, black, opacity=1, forget plot]
table {%
5.875 0.102308230480331
5.875 0
};
\addplot [line width=0.5pt, black, opacity=1, forget plot]
table {%
5.875 0.415552387633531
5.875 0.882082567546926
};
\addplot [line width=0.5pt, black, forget plot]
table {%
5.855 0
5.895 0
};
\addplot [line width=0.5pt, black, forget plot]
table {%
5.855 0.882082567546926
5.895 0.882082567546926
};
\addplot [line width=0.5pt, color0, opacity=1, forget plot]
table {%
0.96 0
1.04 0
1.04 0.1802775637732
0.96 0.1802775637732
0.96 0
};
\addplot [line width=0.5pt, color0, opacity=1, forget plot]
table {%
1 0
1 0
};
\addplot [line width=0.5pt, color0, opacity=1, forget plot]
table {%
1 0.1802775637732
1 0.45
};
\addplot [line width=0.5pt, color0, forget plot]
table {%
0.98 0
1.02 0
};
\addplot [line width=0.5pt, color0, forget plot]
table {%
0.98 0.45
1.02 0.45
};
\addplot [line width=0.5pt, color0, opacity=1, forget plot]
table {%
1.96 0.0333333333333334
2.04 0.0333333333333334
2.04 0.299125612319133
1.96 0.299125612319133
1.96 0.0333333333333334
};
\addplot [line width=0.5pt, color0, opacity=1, forget plot]
table {%
2 0.0333333333333334
2 0
};
\addplot [line width=0.5pt, color0, opacity=1, forget plot]
table {%
2 0.299125612319133
2 0.697474812267084
};
\addplot [line width=0.5pt, color0, forget plot]
table {%
1.98 0
2.02 0
};
\addplot [line width=0.5pt, color0, forget plot]
table {%
1.98 0.697474812267084
2.02 0.697474812267084
};
\addplot [line width=0.5pt, color0, opacity=1, forget plot]
table {%
2.96 0.103812116288267
3.04 0.103812116288267
3.04 0.540975730697202
2.96 0.540975730697202
2.96 0.103812116288267
};
\addplot [line width=0.5pt, color0, opacity=1, forget plot]
table {%
3 0.103812116288267
3 0
};
\addplot [line width=0.5pt, color0, opacity=1, forget plot]
table {%
3 0.540975730697202
3 1.18296402795114
};
\addplot [line width=0.5pt, color0, forget plot]
table {%
2.98 0
3.02 0
};
\addplot [line width=0.5pt, color0, forget plot]
table {%
2.98 1.18296402795114
3.02 1.18296402795114
};
\addplot [line width=0.5pt, color0, opacity=1, forget plot]
table {%
3.96 0.133738344428268
4.04 0.133738344428268
4.04 0.625510655535147
3.96 0.625510655535147
3.96 0.133738344428268
};
\addplot [line width=0.5pt, color0, opacity=1, forget plot]
table {%
4 0.133738344428268
4 0
};
\addplot [line width=0.5pt, color0, opacity=1, forget plot]
table {%
4 0.625510655535147
4 1.28045346311218
};
\addplot [line width=0.5pt, color0, forget plot]
table {%
3.98 0
4.02 0
};
\addplot [line width=0.5pt, color0, forget plot]
table {%
3.98 1.28045346311218
4.02 1.28045346311218
};
\addplot [line width=0.5pt, color0, opacity=1, forget plot]
table {%
4.96 0.141257038496822
5.04 0.141257038496822
5.04 0.678890446697227
4.96 0.678890446697227
4.96 0.141257038496822
};
\addplot [line width=0.5pt, color0, opacity=1, forget plot]
table {%
5 0.141257038496822
5 0
};
\addplot [line width=0.5pt, color0, opacity=1, forget plot]
table {%
5 0.678890446697227
5 1.46513692947726
};
\addplot [line width=0.5pt, color0, forget plot]
table {%
4.98 0
5.02 0
};
\addplot [line width=0.5pt, color0, forget plot]
table {%
4.98 1.46513692947726
5.02 1.46513692947726
};
\addplot [line width=0.5pt, color0, opacity=1, forget plot]
table {%
5.96 0.175685110880123
6.04 0.175685110880123
6.04 0.58341542912783
5.96 0.58341542912783
5.96 0.175685110880123
};
\addplot [line width=0.5pt, color0, opacity=1, forget plot]
table {%
6 0.175685110880123
6 0
};
\addplot [line width=0.5pt, color0, opacity=1, forget plot]
table {%
6 0.58341542912783
6 1.18942960039246
};
\addplot [line width=0.5pt, color0, forget plot]
table {%
5.98 0
6.02 0
};
\addplot [line width=0.5pt, color0, forget plot]
table {%
5.98 1.18942960039246
6.02 1.18942960039246
};
\addplot [line width=0.5pt, color1, opacity=1, forget plot]
table {%
1.085 0.0499999999999998
1.165 0.0499999999999998
1.165 0.208113883008418
1.085 0.208113883008418
1.085 0.0499999999999998
};
\addplot [line width=0.5pt, color1, opacity=1, forget plot]
table {%
1.125 0.0499999999999998
1.125 0
};
\addplot [line width=0.5pt, color1, opacity=1, forget plot]
table {%
1.125 0.208113883008418
1.125 0.4159415253899
};
\addplot [line width=0.5pt, color1, forget plot]
table {%
1.105 0
1.145 0
};
\addplot [line width=0.5pt, color1, forget plot]
table {%
1.105 0.4159415253899
1.145 0.4159415253899
};
\addplot [line width=0.5pt, color1, opacity=1, forget plot]
table {%
2.085 0.0666666666666664
2.165 0.0666666666666664
2.165 0.503334722269288
2.085 0.503334722269288
2.085 0.0666666666666664
};
\addplot [line width=0.5pt, color1, opacity=1, forget plot]
table {%
2.125 0.0666666666666664
2.125 0
};
\addplot [line width=0.5pt, color1, opacity=1, forget plot]
table {%
2.125 0.503334722269288
2.125 1.1488949080401
};
\addplot [line width=0.5pt, color1, forget plot]
table {%
2.105 0
2.145 0
};
\addplot [line width=0.5pt, color1, forget plot]
table {%
2.105 1.1488949080401
2.145 1.1488949080401
};
\addplot [line width=0.5pt, color1, opacity=1, forget plot]
table {%
3.085 0.0999999999999999
3.165 0.0999999999999999
3.165 0.725234818582267
3.085 0.725234818582267
3.085 0.0999999999999999
};
\addplot [line width=0.5pt, color1, opacity=1, forget plot]
table {%
3.125 0.0999999999999999
3.125 0
};
\addplot [line width=0.5pt, color1, opacity=1, forget plot]
table {%
3.125 0.725234818582267
3.125 1.65754649990307
};
\addplot [line width=0.5pt, color1, forget plot]
table {%
3.105 0
3.145 0
};
\addplot [line width=0.5pt, color1, forget plot]
table {%
3.105 1.65754649990307
3.145 1.65754649990307
};
\addplot [line width=0.5pt, color1, opacity=1, forget plot]
table {%
4.085 0.123927669529664
4.165 0.123927669529664
4.165 0.720963745926263
4.085 0.720963745926263
4.085 0.123927669529664
};
\addplot [line width=0.5pt, color1, opacity=1, forget plot]
table {%
4.125 0.123927669529664
4.125 0
};
\addplot [line width=0.5pt, color1, opacity=1, forget plot]
table {%
4.125 0.720963745926263
4.125 1.6011083310896
};
\addplot [line width=0.5pt, color1, forget plot]
table {%
4.105 0
4.145 0
};
\addplot [line width=0.5pt, color1, forget plot]
table {%
4.105 1.6011083310896
4.145 1.6011083310896
};
\addplot [line width=0.5pt, color1, opacity=1, forget plot]
table {%
5.085 0.163398045187485
5.165 0.163398045187485
5.165 0.667925470398605
5.085 0.667925470398605
5.085 0.163398045187485
};
\addplot [line width=0.5pt, color1, opacity=1, forget plot]
table {%
5.125 0.163398045187485
5.125 0.025
};
\addplot [line width=0.5pt, color1, opacity=1, forget plot]
table {%
5.125 0.667925470398605
5.125 1.36193166608039
};
\addplot [line width=0.5pt, color1, forget plot]
table {%
5.105 0.025
5.145 0.025
};
\addplot [line width=0.5pt, color1, forget plot]
table {%
5.105 1.36193166608039
5.145 1.36193166608039
};
\addplot [line width=0.5pt, color1, opacity=1, forget plot]
table {%
6.085 0.195124014594064
6.165 0.195124014594064
6.165 0.641042224133932
6.085 0.641042224133932
6.085 0.195124014594064
};
\addplot [line width=0.5pt, color1, opacity=1, forget plot]
table {%
6.125 0.195124014594064
6.125 0.0249999999999999
};
\addplot [line width=0.5pt, color1, opacity=1, forget plot]
table {%
6.125 0.641042224133932
6.125 1.30581750186973
};
\addplot [line width=0.5pt, color1, forget plot]
table {%
6.105 0.0249999999999999
6.145 0.0249999999999999
};
\addplot [line width=0.5pt, color1, forget plot]
table {%
6.105 1.30581750186973
6.145 1.30581750186973
};
\addplot [line width=0.5pt, color2, opacity=1, forget plot]
table {%
1.21 0.0499999999999998
1.29 0.0499999999999998
1.29 0.210518160398157
1.21 0.210518160398157
1.21 0.0499999999999998
};
\addplot [line width=0.5pt, color2, opacity=1, forget plot]
table {%
1.25 0.0499999999999998
1.25 0
};
\addplot [line width=0.5pt, color2, opacity=1, forget plot]
table {%
1.25 0.210518160398157
1.25 0.420156211871642
};
\addplot [line width=0.5pt, color2, forget plot]
table {%
1.23 0
1.27 0
};
\addplot [line width=0.5pt, color2, forget plot]
table {%
1.23 0.420156211871642
1.27 0.420156211871642
};
\addplot [line width=0.5pt, color2, opacity=1, forget plot]
table {%
2.21 0.0789892388718256
2.29 0.0789892388718256
2.29 0.685012267895355
2.21 0.685012267895355
2.21 0.0789892388718256
};
\addplot [line width=0.5pt, color2, opacity=1, forget plot]
table {%
2.25 0.0789892388718256
2.25 0
};
\addplot [line width=0.5pt, color2, opacity=1, forget plot]
table {%
2.25 0.685012267895355
2.25 1.45144805324697
};
\addplot [line width=0.5pt, color2, forget plot]
table {%
2.23 0
2.27 0
};
\addplot [line width=0.5pt, color2, forget plot]
table {%
2.23 1.45144805324697
2.27 1.45144805324697
};
\addplot [line width=0.5pt, color2, opacity=1, forget plot]
table {%
3.21 0.160327974015615
3.29 0.160327974015615
3.29 0.849752444704236
3.21 0.849752444704236
3.21 0.160327974015615
};
\addplot [line width=0.5pt, color2, opacity=1, forget plot]
table {%
3.25 0.160327974015615
3.25 0.0353553390593273
};
\addplot [line width=0.5pt, color2, opacity=1, forget plot]
table {%
3.25 0.849752444704236
3.25 1.72872914930157
};
\addplot [line width=0.5pt, color2, forget plot]
table {%
3.23 0.0353553390593273
3.27 0.0353553390593273
};
\addplot [line width=0.5pt, color2, forget plot]
table {%
3.23 1.72872914930157
3.27 1.72872914930157
};
\addplot [line width=0.5pt, color2, opacity=1, forget plot]
table {%
4.21 0.264866253099674
4.29 0.264866253099674
4.29 0.881246361164786
4.21 0.881246361164786
4.21 0.264866253099674
};
\addplot [line width=0.5pt, color2, opacity=1, forget plot]
table {%
4.25 0.264866253099674
4.25 0.0249999999999999
};
\addplot [line width=0.5pt, color2, opacity=1, forget plot]
table {%
4.25 0.881246361164786
4.25 1.53072476590464
};
\addplot [line width=0.5pt, color2, forget plot]
table {%
4.23 0.0249999999999999
4.27 0.0249999999999999
};
\addplot [line width=0.5pt, color2, forget plot]
table {%
4.23 1.53072476590464
4.27 1.53072476590464
};
\addplot [line width=0.5pt, color2, opacity=1, forget plot]
table {%
5.21 0.237296737670565
5.29 0.237296737670565
5.29 0.826151369172303
5.21 0.826151369172303
5.21 0.237296737670565
};
\addplot [line width=0.5pt, color2, opacity=1, forget plot]
table {%
5.25 0.237296737670565
5.25 0.025
};
\addplot [line width=0.5pt, color2, opacity=1, forget plot]
table {%
5.25 0.826151369172303
5.25 1.60709495971588
};
\addplot [line width=0.5pt, color2, forget plot]
table {%
5.23 0.025
5.27 0.025
};
\addplot [line width=0.5pt, color2, forget plot]
table {%
5.23 1.60709495971588
5.27 1.60709495971588
};
\addplot [line width=0.5pt, color2, opacity=1, forget plot]
table {%
6.21 0.261628800597555
6.29 0.261628800597555
6.29 0.711026448033977
6.21 0.711026448033977
6.21 0.261628800597555
};
\addplot [line width=0.5pt, color2, opacity=1, forget plot]
table {%
6.25 0.261628800597555
6.25 0.025
};
\addplot [line width=0.5pt, color2, opacity=1, forget plot]
table {%
6.25 0.711026448033977
6.25 1.344602727156
};
\addplot [line width=0.5pt, color2, forget plot]
table {%
6.23 0.025
6.27 0.025
};
\addplot [line width=0.5pt, color2, forget plot]
table {%
6.23 1.344602727156
6.27 1.344602727156
};
\addplot [line width=0.5pt, white!66.274509803921561!black, opacity=1, forget plot]
table {%
0.71 0.0499999999999998
0.79 0.0499999999999998
};
\addplot [line width=0.5pt, white!66.274509803921561!black, dashed, mark=x, mark size=3, mark options={solid}, forget plot]
table {%
0.75 0.103201918911421
};
\addplot [line width=0.5pt, white!66.274509803921561!black, opacity=1, forget plot]
table {%
1.71 0.0666666666666667
1.79 0.0666666666666667
};
\addplot [line width=0.5pt, white!66.274509803921561!black, dashed, mark=x, mark size=3, mark options={solid}, forget plot]
table {%
1.75 0.192914525605671
};
\addplot [line width=0.5pt, white!66.274509803921561!black, opacity=1, forget plot]
table {%
2.71 0.154304001977173
2.79 0.154304001977173
};
\addplot [line width=0.5pt, white!66.274509803921561!black, dashed, mark=x, mark size=3, mark options={solid}, forget plot]
table {%
2.75 0.303630724385165
};
\addplot [line width=0.5pt, white!66.274509803921561!black, opacity=1, forget plot]
table {%
3.71 0.193427400616229
3.79 0.193427400616229
};
\addplot [line width=0.5pt, white!66.274509803921561!black, dashed, mark=x, mark size=3, mark options={solid}, forget plot]
table {%
3.75 0.327065521730923
};
\addplot [line width=0.5pt, white!66.274509803921561!black, opacity=1, forget plot]
table {%
4.71 0.199613900595886
4.79 0.199613900595886
};
\addplot [line width=0.5pt, white!66.274509803921561!black, dashed, mark=x, mark size=3, mark options={solid}, forget plot]
table {%
4.75 0.355194351174054
};
\addplot [line width=0.5pt, white!66.274509803921561!black, opacity=1, forget plot]
table {%
5.71 0.193087160190115
5.79 0.193087160190115
};
\addplot [line width=0.5pt, white!66.274509803921561!black, dashed, mark=x, mark size=3, mark options={solid}, forget plot]
table {%
5.75 0.308500525658535
};
\addplot [line width=0.5pt, black, opacity=1, forget plot]
table {%
0.835 0.0499999999999998
0.915 0.0499999999999998
};
\addplot [line width=0.5pt, black, dashed, mark=x, mark size=3, mark options={solid}, forget plot]
table {%
0.875 0.125017110619512
};
\addplot [line width=0.5pt, black, opacity=1, forget plot]
table {%
1.835 0.0666666666666667
1.915 0.0666666666666667
};
\addplot [line width=0.5pt, black, dashed, mark=x, mark size=3, mark options={solid}, forget plot]
table {%
1.875 0.155881722628191
};
\addplot [line width=0.5pt, black, opacity=1, forget plot]
table {%
2.835 0.103077640640442
2.915 0.103077640640442
};
\addplot [line width=0.5pt, black, dashed, mark=x, mark size=3, mark options={solid}, forget plot]
table {%
2.875 0.24038527794314
};
\addplot [line width=0.5pt, black, opacity=1, forget plot]
table {%
3.835 0.144953200754732
3.915 0.144953200754732
};
\addplot [line width=0.5pt, black, dashed, mark=x, mark size=3, mark options={solid}, forget plot]
table {%
3.875 0.320915049808084
};
\addplot [line width=0.5pt, black, opacity=1, forget plot]
table {%
4.835 0.189626091568228
4.915 0.189626091568228
};
\addplot [line width=0.5pt, black, dashed, mark=x, mark size=3, mark options={solid}, forget plot]
table {%
4.875 0.358746074362444
};
\addplot [line width=0.5pt, black, opacity=1, forget plot]
table {%
5.835 0.212501999098811
5.915 0.212501999098811
};
\addplot [line width=0.5pt, black, dashed, mark=x, mark size=3, mark options={solid}, forget plot]
table {%
5.875 0.317157174802269
};
\addplot [line width=0.5pt, color0, opacity=1, forget plot]
table {%
0.96 0.0500000000000002
1.04 0.0500000000000002
};
\addplot [line width=0.5pt, color0, dashed, mark=x, mark size=3, mark options={solid}, forget plot]
table {%
1 0.162540128816541
};
\addplot [line width=0.5pt, color0, opacity=1, forget plot]
table {%
1.96 0.106639093961136
2.04 0.106639093961136
};
\addplot [line width=0.5pt, color0, dashed, mark=x, mark size=3, mark options={solid}, forget plot]
table {%
2 0.254182143160617
};
\addplot [line width=0.5pt, color0, opacity=1, forget plot]
table {%
2.96 0.225266902835433
3.04 0.225266902835433
};
\addplot [line width=0.5pt, color0, dashed, mark=x, mark size=3, mark options={solid}, forget plot]
table {%
3 0.362879051910614
};
\addplot [line width=0.5pt, color0, opacity=1, forget plot]
table {%
3.96 0.344423225217581
4.04 0.344423225217581
};
\addplot [line width=0.5pt, color0, dashed, mark=x, mark size=3, mark options={solid}, forget plot]
table {%
4 0.436274313295025
};
\addplot [line width=0.5pt, color0, opacity=1, forget plot]
table {%
4.96 0.332337044714244
5.04 0.332337044714244
};
\addplot [line width=0.5pt, color0, dashed, mark=x, mark size=3, mark options={solid}, forget plot]
table {%
5 0.458997366046748
};
\addplot [line width=0.5pt, color0, opacity=1, forget plot]
table {%
5.96 0.348925998193767
6.04 0.348925998193767
};
\addplot [line width=0.5pt, color0, dashed, mark=x, mark size=3, mark options={solid}, forget plot]
table {%
6 0.419435475198472
};
\addplot [line width=0.5pt, color1, opacity=1, forget plot]
table {%
1.085 0.0707106781186546
1.165 0.0707106781186546
};
\addplot [line width=0.5pt, color1, dashed, mark=x, mark size=3, mark options={solid}, forget plot]
table {%
1.125 0.16797653541589
};
\addplot [line width=0.5pt, color1, opacity=1, forget plot]
table {%
2.085 0.144280904158206
2.165 0.144280904158206
};
\addplot [line width=0.5pt, color1, dashed, mark=x, mark size=3, mark options={solid}, forget plot]
table {%
2.125 0.342733292505696
};
\addplot [line width=0.5pt, color1, opacity=1, forget plot]
table {%
3.085 0.275984957221428
3.165 0.275984957221428
};
\addplot [line width=0.5pt, color1, dashed, mark=x, mark size=3, mark options={solid}, forget plot]
table {%
3.125 0.450364275040593
};
\addplot [line width=0.5pt, color1, opacity=1, forget plot]
table {%
4.085 0.323808161121566
4.165 0.323808161121566
};
\addplot [line width=0.5pt, color1, dashed, mark=x, mark size=3, mark options={solid}, forget plot]
table {%
4.125 0.451371108769039
};
\addplot [line width=0.5pt, color1, opacity=1, forget plot]
table {%
5.085 0.364405052038868
5.165 0.364405052038868
};
\addplot [line width=0.5pt, color1, dashed, mark=x, mark size=3, mark options={solid}, forget plot]
table {%
5.125 0.467331612164226
};
\addplot [line width=0.5pt, color1, opacity=1, forget plot]
table {%
6.085 0.388973714660886
6.165 0.388973714660886
};
\addplot [line width=0.5pt, color1, dashed, mark=x, mark size=3, mark options={solid}, forget plot]
table {%
6.125 0.457655304538824
};
\addplot [line width=0.5pt, color2, opacity=1, forget plot]
table {%
1.21 0.070710678118655
1.29 0.070710678118655
};
\addplot [line width=0.5pt, color2, dashed, mark=x, mark size=3, mark options={solid}, forget plot]
table {%
1.25 0.16686116069469
};
\addplot [line width=0.5pt, color2, opacity=1, forget plot]
table {%
2.21 0.225652597122411
2.29 0.225652597122411
};
\addplot [line width=0.5pt, color2, dashed, mark=x, mark size=3, mark options={solid}, forget plot]
table {%
2.25 0.422213097133527
};
\addplot [line width=0.5pt, color2, opacity=1, forget plot]
table {%
3.21 0.480199650173389
3.29 0.480199650173389
};
\addplot [line width=0.5pt, color2, dashed, mark=x, mark size=3, mark options={solid}, forget plot]
table {%
3.25 0.563753100309146
};
\addplot [line width=0.5pt, color2, opacity=1, forget plot]
table {%
4.21 0.505708131133393
4.29 0.505708131133393
};
\addplot [line width=0.5pt, color2, dashed, mark=x, mark size=3, mark options={solid}, forget plot]
table {%
4.25 0.584617428377441
};
\addplot [line width=0.5pt, color2, opacity=1, forget plot]
table {%
5.21 0.459287943192255
5.29 0.459287943192255
};
\addplot [line width=0.5pt, color2, dashed, mark=x, mark size=3, mark options={solid}, forget plot]
table {%
5.25 0.557208047586385
};
\addplot [line width=0.5pt, color2, opacity=1, forget plot]
table {%
6.21 0.44726555164435
6.29 0.44726555164435
};
\addplot [line width=0.5pt, color2, dashed, mark=x, mark size=3, mark options={solid}, forget plot]
table {%
6.25 0.518532712134289
};
\end{axis}

\node at ({$(current bounding box.south west)!0.5!(current bounding box.south east)$}|-{$(current bounding box.south west)!0.98!(current bounding box.north west)$})[
  anchor=north,
  text=black,
  rotate=0.0
]{ };

	    \begin{customlegend}[
legend entries={no noise,SNR\ $=30$,SNR\ $=15$, SNR\ $=10$, SNR\ $=5$},
legend cell align=left,
legend style={at={(0.05,5.37)}, anchor=north west, draw=white!80.0!black, font=\footnotesize}]
    \addlegendimage{area legend,gray,fill=gray}
    \addlegendimage{area legend,black,fill=black}
    \addlegendimage{area legend,color0,fill=color0}
    \addlegendimage{area legend,color1,fill=color1}
    \addlegendimage{area legend,color1,fill=color2}
\end{customlegend}
	\end{tikzpicture}
	
	\caption[Evaluation results for varying \glsentryshort{em} iterations]{Evaluation results for varying \glsentryshort{em} iterations: }
	\label{fig:trial1}
\end{figure}

\begin{itemize}
    \item not surprisingly, more noise (lower SNR) means worse location estimates
    \item Worst estimates for lowest SNR
    \item With increasing SNR, estimates then get better
    \item Consistent over all trials of different number of sources
    \item But, SNR=5dB decreased mean-err and size of 95\% percentile for 3-7 sources compared to no noise.
    \item Results seem to indicate, that SNR=5 improves location estimation performance, while more than 5dB decreases performance compared to noiseless environment
\end{itemize}