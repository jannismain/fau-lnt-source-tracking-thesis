\subsubsection{Best Case vs. Worst Case Simulation Scenario}

The best and worst case scenarios have to be understood in terms of the constraints, that have been applied to the parameter set.
\newcommand{\bestc}[1]{\textcolor{OliveGreen}{#1}}
\newcommand{\worstc}[1]{\textcolor{Mahogany}{#1}}
%% Parameter sets
\begin{table}[H]
	\centering
	\begin{tabular}{rcc}
	   \toprule
		\textbf{Parameter}                  & \textbf{Unit} & \textbf{Tested Values}                        \\
		\midrule
		T$_{60}$                            & [s]           & \bestc{0.0}, \default{0.3}, 0.6, \worstc{0.9} \\
		SNR                                 & [dB]          & \bestc{\default{0}}, 30, 15, 10, \worstc{5}   \\
		\glsentryshort{em} iterations ($L$) &               & 1, 2, 3, \default{5}, 10, 20                  \\
		reflect order                       &               & \bestc{1}, \default{3}, \worstc{max}          \\
		\bottomrule
	\end{tabular}
	\label{table:parameterset}
	\caption[Parameter Set for Different Scenarios]{Parameter Set for Different Scenarios: \itshape Parameters of the base scenarios are indicated by a bold font, parameters of the best and worst case parameter set are indicated by \bestc{green} and \worstc{red} color respectively. A colored bold value means, that the base scenario value was equal to the scenario indicated by the color, whereas missing of colored values in a row means that the value was hold constant for all scenarios.}
\end{table}

%% all-in-one boxplot
\begin{figure}[H]
	\setlength\figureheight{7cm}
	\small
	\setlength\figurewidth{\textwidth}
	\centering
	\begin{tikzpicture}
		\footnotesize
		% This file was created by matplotlib2tikz v0.6.14.
\definecolor{color0}{rgb}{0.8,0.207843137254902,0.219607843137255}
\definecolor{color1}{rgb}{1,0.647058823529412,0}
\definecolor{color2}{rgb}{0.0235294117647059,0.603921568627451,0.952941176470588}

\begin{axis}[
xlabel={number of sources ($S$)},
ylabel={mean localisation error},
xmin=0.5, xmax=6.5,
ymin=0, ymax=3.51,
width=\figurewidth,
height=\figureheight,
xtick={1,2,3,4,5,6},
xticklabels={2,3,4,5,6,7},
ytick={0,0.5,1,1.5,2,2.5,3,3.5},
minor xtick={},
minor ytick={},
tick align=outside,
tick pos=left,
x grid style={white!69.019607843137251!black},
ymajorgrids,
y grid style={white!69.019607843137251!black}
]
\addplot [line width=1.0pt, black, opacity=1, forget plot]
table {%
0.76 0
0.84 0
0.84 0
0.76 0
0.76 0
};
\addplot [line width=1.0pt, black, opacity=1, forget plot]
table {%
0.8 0
0.8 0
};
\addplot [line width=1.0pt, black, opacity=1, forget plot]
table {%
0.8 0
0.8 0
};
\addplot [line width=1.0pt, black, forget plot]
table {%
0.78 0
0.82 0
};
\addplot [line width=1.0pt, black, forget plot]
table {%
0.78 0
0.82 0
};
\addplot [line width=1.0pt, black, opacity=0.2, mark=*, mark size=1, mark options={solid}, only marks, forget plot]
table {%
0.8 0.223606797749979
};
\addplot [line width=1.0pt, black, opacity=1, forget plot]
table {%
1.76 0
1.84 0
1.84 0
1.76 0
1.76 0
};
\addplot [line width=1.0pt, black, opacity=1, forget plot]
table {%
1.8 0
1.8 0
};
\addplot [line width=1.0pt, black, opacity=1, forget plot]
table {%
1.8 0
1.8 0
};
\addplot [line width=1.0pt, black, forget plot]
table {%
1.78 0
1.82 0
};
\addplot [line width=1.0pt, black, forget plot]
table {%
1.78 0
1.82 0
};
\addplot [line width=1.0pt, black, opacity=0.2, mark=*, mark size=1, mark options={solid}, only marks, forget plot]
table {%
1.8 0.333333333333333
1.8 0.687184270936277
1.8 0.298142396999972
1.8 0.213437474581095
1.8 0.887568463712957
1.8 0.166666666666667
1.8 0.687184270936277
1.8 0.188561808316413
1.8 0.120185042515466
1.8 0.260341655863555
1.8 0.780313327381308
1.8 0.307318148576429
1.8 0.52704627669473
1.8 0.333333333333333
1.8 1.22955321715015
1.8 0.169967317119759
1.8 0.817176711475417
1.8 0.169967317119759
1.8 0.477260702109212
1.8 0.120185042515466
1.8 0.495535624910617
1.8 0.554016645060443
1.8 0.900617072407087
1.8 0.74535599249993
};
\addplot [line width=1.0pt, black, opacity=1, forget plot]
table {%
2.76 0
2.84 0
2.84 0
2.76 0
2.76 0
};
\addplot [line width=1.0pt, black, opacity=1, forget plot]
table {%
2.8 0
2.8 0
};
\addplot [line width=1.0pt, black, opacity=1, forget plot]
table {%
2.8 0
2.8 0
};
\addplot [line width=1.0pt, black, forget plot]
table {%
2.78 0
2.82 0
};
\addplot [line width=1.0pt, black, forget plot]
table {%
2.78 0
2.82 0
};
\addplot [line width=1.0pt, black, opacity=0.2, mark=*, mark size=1, mark options={solid}, only marks, forget plot]
table {%
2.8 0.549758189657856
2.8 0.5
2.8 0.206155281280883
2.8 0.450693909432999
2.8 0.835568665640951
2.8 0.458938993767146
2.8 0.11180339887499
2.8 0.195256241897666
2.8 0.450693909432999
2.8 0.201556443707464
2.8 0.447213595499958
2.8 0.922385025721586
2.8 0.391311896062463
2.8 0.305277563773199
2.8 0.704278769020671
2.8 0.425
2.8 0.20405694150421
2.8 0.6
2.8 0.195256241897666
2.8 0.213600093632938
2.8 0.815272599161946
2.8 0.5
2.8 0.828402076289044
2.8 0.625
2.8 0.160078105935821
2.8 1.05504739230046
2.8 0.125
2.8 0.141421356237309
2.8 0.3
2.8 0.728868986855663
2.8 0.390512483795333
2.8 0.643465843842649
2.8 0.160078105935821
2.8 0.883883476483184
2.8 0.474341649025257
2.8 0.206155281280883
2.8 0.910470982217074
2.8 0.348209706929603
2.8 0.608789783094296
2.8 0.0790569415042094
2.8 0.452827171712309
2.8 0.503115294937453
2.8 0.807000619578449
2.8 0.325960120260132
2.8 0.664266512779321
};
\addplot [line width=1.0pt, black, opacity=1, forget plot]
table {%
3.76 0
3.84 0
3.84 0.244676870129118
3.76 0.244676870129118
3.76 0
};
\addplot [line width=1.0pt, black, opacity=1, forget plot]
table {%
3.8 0
3.8 0
};
\addplot [line width=1.0pt, black, opacity=1, forget plot]
table {%
3.8 0.244676870129118
3.8 0.608276253029822
};
\addplot [line width=1.0pt, black, forget plot]
table {%
3.78 0
3.82 0
};
\addplot [line width=1.0pt, black, forget plot]
table {%
3.78 0.608276253029822
3.82 0.608276253029822
};
\addplot [line width=1.0pt, black, opacity=0.2, mark=*, mark size=1, mark options={solid}, only marks, forget plot]
table {%
3.8 0.647312336830356
3.8 1.21426219830839
3.8 0.773666562565053
3.8 0.693397432934389
3.8 1.00916061841246
3.8 0.886952072420197
3.8 0.627694193059009
3.8 0.653038674684924
3.8 0.860716733341669
3.8 0.642806347199528
3.8 1.15792249378939
3.8 0.722495674727538
};
\addplot [line width=1.0pt, black, opacity=1, forget plot]
table {%
4.76 0
4.84 0
4.84 0.352959866627896
4.76 0.352959866627896
4.76 0
};
\addplot [line width=1.0pt, black, opacity=1, forget plot]
table {%
4.8 0
4.8 0
};
\addplot [line width=1.0pt, black, opacity=1, forget plot]
table {%
4.8 0.352959866627896
4.8 0.870445625247571
};
\addplot [line width=1.0pt, black, forget plot]
table {%
4.78 0
4.82 0
};
\addplot [line width=1.0pt, black, forget plot]
table {%
4.78 0.870445625247571
4.82 0.870445625247571
};
\addplot [line width=1.0pt, black, opacity=0.2, mark=*, mark size=1, mark options={solid}, only marks, forget plot]
table {%
4.8 0.92518880942054
4.8 0.942310269929699
};
\addplot [line width=1.0pt, black, opacity=1, forget plot]
table {%
5.76 0
5.84 0
5.84 0.410540987180671
5.76 0.410540987180671
5.76 0
};
\addplot [line width=1.0pt, black, opacity=1, forget plot]
table {%
5.8 0
5.8 0
};
\addplot [line width=1.0pt, black, opacity=1, forget plot]
table {%
5.8 0.410540987180671
5.8 0.957766495294535
};
\addplot [line width=1.0pt, black, forget plot]
table {%
5.78 0
5.82 0
};
\addplot [line width=1.0pt, black, forget plot]
table {%
5.78 0.957766495294535
5.82 0.957766495294535
};
\addplot [line width=1.0pt, color0, opacity=1, forget plot]
table {%
0.893333333333333 0
0.973333333333333 0
0.973333333333333 0.159036261975062
0.893333333333333 0.159036261975062
0.893333333333333 0
};
\addplot [line width=1.0pt, color0, opacity=1, forget plot]
table {%
0.933333333333333 0
0.933333333333333 0
};
\addplot [line width=1.0pt, color0, opacity=1, forget plot]
table {%
0.933333333333333 0.159036261975062
0.933333333333333 0.390512483795333
};
\addplot [line width=1.0pt, color0, forget plot]
table {%
0.913333333333333 0
0.953333333333333 0
};
\addplot [line width=1.0pt, color0, forget plot]
table {%
0.913333333333333 0.390512483795333
0.953333333333333 0.390512483795333
};
\addplot [line width=1.0pt, color0, opacity=0.2, mark=*, mark size=1, mark options={solid}, only marks, forget plot]
table {%
0.933333333333333 0.462132034355964
0.933333333333333 0.430116263352131
0.933333333333333 0.424264068711928
0.933333333333333 0.461223161913987
0.933333333333333 0.431959610746632
0.933333333333333 1.03191727680481
0.933333333333333 0.403112887414928
0.933333333333333 0.515154392492244
0.933333333333333 0.541547594742265
0.933333333333333 0.51478150704935
0.933333333333333 1.49473506746892
0.933333333333333 1.2747548783982
0.933333333333333 0.430277563773199
0.933333333333333 1.45004658906755
0.933333333333333 1.26491106406735
};
\addplot [line width=1.0pt, color0, opacity=1, forget plot]
table {%
1.89333333333333 0
1.97333333333333 0
1.97333333333333 0.166666666666667
1.89333333333333 0.166666666666667
1.89333333333333 0
};
\addplot [line width=1.0pt, color0, opacity=1, forget plot]
table {%
1.93333333333333 0
1.93333333333333 0
};
\addplot [line width=1.0pt, color0, opacity=1, forget plot]
table {%
1.93333333333333 0.166666666666667
1.93333333333333 0.405517502019881
};
\addplot [line width=1.0pt, color0, forget plot]
table {%
1.91333333333333 0
1.95333333333333 0
};
\addplot [line width=1.0pt, color0, forget plot]
table {%
1.91333333333333 0.405517502019881
1.95333333333333 0.405517502019881
};
\addplot [line width=1.0pt, color0, opacity=0.2, mark=*, mark size=1, mark options={solid}, only marks, forget plot]
table {%
1.93333333333333 0.723124466372173
1.93333333333333 0.811530858727814
1.93333333333333 0.929001922836294
1.93333333333333 0.69591054289104
1.93333333333333 1.13065877872932
1.93333333333333 0.533333333333333
1.93333333333333 1.06099682060485
1.93333333333333 0.46971661629718
1.93333333333333 1.22465908830831
1.93333333333333 0.480740170061866
1.93333333333333 1.21737727279302
1.93333333333333 0.94695242113682
1.93333333333333 0.741072378723401
1.93333333333333 0.745630279538177
1.93333333333333 0.481132704401887
1.93333333333333 0.52068331172711
1.93333333333333 1.17426099692057
1.93333333333333 0.827453779460412
1.93333333333333 0.690010572469092
1.93333333333333 0.968962790249909
};
\addplot [line width=1.0pt, color0, opacity=1, forget plot]
table {%
2.89333333333333 0.05
2.97333333333333 0.05
2.97333333333333 0.463957703183814
2.89333333333333 0.463957703183814
2.89333333333333 0.05
};
\addplot [line width=1.0pt, color0, opacity=1, forget plot]
table {%
2.93333333333333 0.05
2.93333333333333 0
};
\addplot [line width=1.0pt, color0, opacity=1, forget plot]
table {%
2.93333333333333 0.463957703183814
2.93333333333333 1.07036741628355
};
\addplot [line width=1.0pt, color0, forget plot]
table {%
2.91333333333333 0
2.95333333333333 0
};
\addplot [line width=1.0pt, color0, forget plot]
table {%
2.91333333333333 1.07036741628355
2.95333333333333 1.07036741628355
};
\addplot [line width=1.0pt, color0, opacity=0.2, mark=*, mark size=1, mark options={solid}, only marks, forget plot]
table {%
2.93333333333333 1.21037568281582
2.93333333333333 1.79181713103589
2.93333333333333 1.87024478774715
2.93333333333333 1.09501105835888
2.93333333333333 1.32167139813433
2.93333333333333 1.12249248548055
2.93333333333333 1.26034531628728
2.93333333333333 1.23906771731623
2.93333333333333 1.5740636322166
2.93333333333333 1.18829582466077
2.93333333333333 1.13178881988409
2.93333333333333 1.22484968569455
2.93333333333333 1.15813925005129
};
\addplot [line width=1.0pt, color0, opacity=1, forget plot]
table {%
3.89333333333333 0.075
3.97333333333333 0.075
3.97333333333333 0.480891996016141
3.89333333333333 0.480891996016141
3.89333333333333 0.075
};
\addplot [line width=1.0pt, color0, opacity=1, forget plot]
table {%
3.93333333333333 0.075
3.93333333333333 0
};
\addplot [line width=1.0pt, color0, opacity=1, forget plot]
table {%
3.93333333333333 0.480891996016141
3.93333333333333 1.06441261772754
};
\addplot [line width=1.0pt, color0, forget plot]
table {%
3.91333333333333 0
3.95333333333333 0
};
\addplot [line width=1.0pt, color0, forget plot]
table {%
3.91333333333333 1.06441261772754
3.95333333333333 1.06441261772754
};
\addplot [line width=1.0pt, color0, opacity=0.2, mark=*, mark size=1, mark options={solid}, only marks, forget plot]
table {%
3.93333333333333 1.62114706811532
3.93333333333333 1.150693909433
3.93333333333333 1.21020374003025
3.93333333333333 1.45499581403728
3.93333333333333 1.15689762096067
3.93333333333333 1.78608570870901
3.93333333333333 1.44811730292805
3.93333333333333 1.16296430777624
3.93333333333333 1.56446217884382
};
\addplot [line width=1.0pt, color0, opacity=1, forget plot]
table {%
4.89333333333333 0.0853553390593275
4.97333333333333 0.0853553390593275
4.97333333333333 0.552909544689231
4.89333333333333 0.552909544689231
4.89333333333333 0.0853553390593275
};
\addplot [line width=1.0pt, color0, opacity=1, forget plot]
table {%
4.93333333333333 0.0853553390593275
4.93333333333333 0
};
\addplot [line width=1.0pt, color0, opacity=1, forget plot]
table {%
4.93333333333333 0.552909544689231
4.93333333333333 1.20488091421801
};
\addplot [line width=1.0pt, color0, forget plot]
table {%
4.91333333333333 0
4.95333333333333 0
};
\addplot [line width=1.0pt, color0, forget plot]
table {%
4.91333333333333 1.20488091421801
4.95333333333333 1.20488091421801
};
\addplot [line width=1.0pt, color0, opacity=0.2, mark=*, mark size=1, mark options={solid}, only marks, forget plot]
table {%
4.93333333333333 1.25698702769536
4.93333333333333 1.68507514496448
4.93333333333333 1.33917282448251
4.93333333333333 1.53828359960461
4.93333333333333 1.49267229786383
4.93333333333333 1.35097360200167
};
\addplot [line width=1.0pt, color0, opacity=1, forget plot]
table {%
5.89333333333333 0.0957106781186549
5.97333333333333 0.0957106781186549
5.97333333333333 0.481364998336788
5.89333333333333 0.481364998336788
5.89333333333333 0.0957106781186549
};
\addplot [line width=1.0pt, color0, opacity=1, forget plot]
table {%
5.93333333333333 0.0957106781186549
5.93333333333333 0
};
\addplot [line width=1.0pt, color0, opacity=1, forget plot]
table {%
5.93333333333333 0.481364998336788
5.93333333333333 1.04603276397469
};
\addplot [line width=1.0pt, color0, forget plot]
table {%
5.91333333333333 0
5.95333333333333 0
};
\addplot [line width=1.0pt, color0, forget plot]
table {%
5.91333333333333 1.04603276397469
5.95333333333333 1.04603276397469
};
\addplot [line width=1.0pt, color0, opacity=0.2, mark=*, mark size=1, mark options={solid}, only marks, forget plot]
table {%
5.93333333333333 1.75666687682469
5.93333333333333 1.1445928843406
5.93333333333333 1.17814399732646
5.93333333333333 1.16219317952013
5.93333333333333 1.27310368254746
5.93333333333333 1.29746136145466
};
\addplot [line width=1.0pt, color1, opacity=1, forget plot]
table {%
1.02666666666667 0.269752521501738
1.10666666666667 0.269752521501738
1.10666666666667 1.15487970695088
1.02666666666667 1.15487970695088
1.02666666666667 0.269752521501738
};
\addplot [line width=1.0pt, color1, opacity=1, forget plot]
table {%
1.06666666666667 0.269752521501738
1.06666666666667 0
};
\addplot [line width=1.0pt, color1, opacity=1, forget plot]
table {%
1.06666666666667 1.15487970695088
1.06666666666667 2.46024966777491
};
\addplot [line width=1.0pt, color1, forget plot]
table {%
1.04666666666667 0
1.08666666666667 0
};
\addplot [line width=1.0pt, color1, forget plot]
table {%
1.04666666666667 2.46024966777491
1.08666666666667 2.46024966777491
};
\addplot [line width=1.0pt, color1, opacity=1, forget plot]
table {%
2.02666666666667 0.70851489180413
2.10666666666667 0.70851489180413
2.10666666666667 1.31554873765619
2.02666666666667 1.31554873765619
2.02666666666667 0.70851489180413
};
\addplot [line width=1.0pt, color1, opacity=1, forget plot]
table {%
2.06666666666667 0.70851489180413
2.06666666666667 0.138742588672279
};
\addplot [line width=1.0pt, color1, opacity=1, forget plot]
table {%
2.06666666666667 1.31554873765619
2.06666666666667 2.12433699127745
};
\addplot [line width=1.0pt, color1, forget plot]
table {%
2.04666666666667 0.138742588672279
2.08666666666667 0.138742588672279
};
\addplot [line width=1.0pt, color1, forget plot]
table {%
2.04666666666667 2.12433699127745
2.08666666666667 2.12433699127745
};
\addplot [line width=1.0pt, color1, opacity=0.2, mark=*, mark size=1, mark options={solid}, only marks, forget plot]
table {%
2.06666666666667 2.37605220565632
};
\addplot [line width=1.0pt, color1, opacity=1, forget plot]
table {%
3.02666666666667 0.615284594501342
3.10666666666667 0.615284594501342
3.10666666666667 1.17610144966066
3.02666666666667 1.17610144966066
3.02666666666667 0.615284594501342
};
\addplot [line width=1.0pt, color1, opacity=1, forget plot]
table {%
3.06666666666667 0.615284594501342
3.06666666666667 0.0471404520791032
};
\addplot [line width=1.0pt, color1, opacity=1, forget plot]
table {%
3.06666666666667 1.17610144966066
3.06666666666667 2.01621893663773
};
\addplot [line width=1.0pt, color1, forget plot]
table {%
3.04666666666667 0.0471404520791032
3.08666666666667 0.0471404520791032
};
\addplot [line width=1.0pt, color1, forget plot]
table {%
3.04666666666667 2.01621893663773
3.08666666666667 2.01621893663773
};
\addplot [line width=1.0pt, color1, opacity=0.2, mark=*, mark size=1, mark options={solid}, only marks, forget plot]
table {%
3.06666666666667 2.0541125003336
3.06666666666667 2.166119795119
3.06666666666667 2.20895934204196
};
\addplot [line width=1.0pt, color1, opacity=1, forget plot]
table {%
4.02666666666667 0.484074096802648
4.10666666666667 0.484074096802648
4.10666666666667 1.06206155631352
4.02666666666667 1.06206155631352
4.02666666666667 0.484074096802648
};
\addplot [line width=1.0pt, color1, opacity=1, forget plot]
table {%
4.06666666666667 0.484074096802648
4.06666666666667 0.0804737854124364
};
\addplot [line width=1.0pt, color1, opacity=1, forget plot]
table {%
4.06666666666667 1.06206155631352
4.06666666666667 1.85162136003413
};
\addplot [line width=1.0pt, color1, forget plot]
table {%
4.04666666666667 0.0804737854124364
4.08666666666667 0.0804737854124364
};
\addplot [line width=1.0pt, color1, forget plot]
table {%
4.04666666666667 1.85162136003413
4.08666666666667 1.85162136003413
};
\addplot [line width=1.0pt, color1, opacity=0.2, mark=*, mark size=1, mark options={solid}, only marks, forget plot]
table {%
4.06666666666667 2.08203757116228
};
\addplot [line width=1.0pt, color1, opacity=1, forget plot]
table {%
5.02666666666667 0.529822389532531
5.10666666666667 0.529822389532531
5.10666666666667 0.983160994151946
5.02666666666667 0.983160994151946
5.02666666666667 0.529822389532531
};
\addplot [line width=1.0pt, color1, opacity=1, forget plot]
table {%
5.06666666666667 0.529822389532531
5.06666666666667 0.1
};
\addplot [line width=1.0pt, color1, opacity=1, forget plot]
table {%
5.06666666666667 0.983160994151946
5.06666666666667 1.52075922005613
};
\addplot [line width=1.0pt, color1, forget plot]
table {%
5.04666666666667 0.1
5.08666666666667 0.1
};
\addplot [line width=1.0pt, color1, forget plot]
table {%
5.04666666666667 1.52075922005613
5.08666666666667 1.52075922005613
};
\addplot [line width=1.0pt, color1, opacity=0.2, mark=*, mark size=1, mark options={solid}, only marks, forget plot]
table {%
5.06666666666667 1.76616204194763
5.06666666666667 1.8913536276932
5.06666666666667 1.69085452718737
5.06666666666667 1.68625692679761
};
\addplot [line width=1.0pt, color1, opacity=1, forget plot]
table {%
6.02666666666667 0.488224907002558
6.10666666666667 0.488224907002558
6.10666666666667 0.86172080709631
6.02666666666667 0.86172080709631
6.02666666666667 0.488224907002558
};
\addplot [line width=1.0pt, color1, opacity=1, forget plot]
table {%
6.06666666666667 0.488224907002558
6.06666666666667 0.0333333333333334
};
\addplot [line width=1.0pt, color1, opacity=1, forget plot]
table {%
6.06666666666667 0.86172080709631
6.06666666666667 1.28557760732131
};
\addplot [line width=1.0pt, color1, forget plot]
table {%
6.04666666666667 0.0333333333333334
6.08666666666667 0.0333333333333334
};
\addplot [line width=1.0pt, color1, forget plot]
table {%
6.04666666666667 1.28557760732131
6.08666666666667 1.28557760732131
};
\addplot [line width=1.0pt, color1, opacity=0.2, mark=*, mark size=1, mark options={solid}, only marks, forget plot]
table {%
6.06666666666667 1.60008643947993
6.06666666666667 1.45972278930027
6.06666666666667 1.5608368927356
6.06666666666667 1.65666440402801
6.06666666666667 1.57971180711216
};
\addplot [line width=1.0pt, color2, opacity=1, forget plot]
table {%
1.16 1.19657928299416
1.24 1.19657928299416
1.24 2.11535396914255
1.16 2.11535396914255
1.16 1.19657928299416
};
\addplot [line width=1.0pt, color2, opacity=1, forget plot]
table {%
1.2 1.19657928299416
1.2 0.161803398874989
};
\addplot [line width=1.0pt, color2, opacity=1, forget plot]
table {%
1.2 2.11535396914255
1.2 3.44411068783207
};
\addplot [line width=1.0pt, color2, forget plot]
table {%
1.18 0.161803398874989
1.22 0.161803398874989
};
\addplot [line width=1.0pt, color2, forget plot]
table {%
1.18 3.44411068783207
1.22 3.44411068783207
};
\addplot [line width=1.0pt, color2, opacity=0.2, mark=*, mark size=1, mark options={solid}, only marks, forget plot]
table {%
1.2 3.56166754346607
};
\addplot [line width=1.0pt, color2, opacity=1, forget plot]
table {%
2.16 1.18911140450059
2.24 1.18911140450059
2.24 1.84412589274571
2.16 1.84412589274571
2.16 1.18911140450059
};
\addplot [line width=1.0pt, color2, opacity=1, forget plot]
table {%
2.2 1.18911140450059
2.2 0.320525770195464
};
\addplot [line width=1.0pt, color2, opacity=1, forget plot]
table {%
2.2 1.84412589274571
2.2 2.77721155745514
};
\addplot [line width=1.0pt, color2, forget plot]
table {%
2.18 0.320525770195464
2.22 0.320525770195464
};
\addplot [line width=1.0pt, color2, forget plot]
table {%
2.18 2.77721155745514
2.22 2.77721155745514
};
\addplot [line width=1.0pt, color2, opacity=0.2, mark=*, mark size=1, mark options={solid}, only marks, forget plot]
table {%
2.2 2.85004013022048
2.2 2.90753379895794
2.2 2.9311489007798
2.2 2.88407777210695
2.2 2.92759068379257
};
\addplot [line width=1.0pt, color2, opacity=1, forget plot]
table {%
3.16 1.13738933995924
3.24 1.13738933995924
3.24 1.70444888428155
3.16 1.70444888428155
3.16 1.13738933995924
};
\addplot [line width=1.0pt, color2, opacity=1, forget plot]
table {%
3.2 1.13738933995924
3.2 0.468469222067746
};
\addplot [line width=1.0pt, color2, opacity=1, forget plot]
table {%
3.2 1.70444888428155
3.2 2.51900899510985
};
\addplot [line width=1.0pt, color2, forget plot]
table {%
3.18 0.468469222067746
3.22 0.468469222067746
};
\addplot [line width=1.0pt, color2, forget plot]
table {%
3.18 2.51900899510985
3.22 2.51900899510985
};
\addplot [line width=1.0pt, color2, opacity=0.2, mark=*, mark size=1, mark options={solid}, only marks, forget plot]
table {%
3.2 2.63840511187134
};
\addplot [line width=1.0pt, color2, opacity=1, forget plot]
table {%
4.16 1.1094854472023
4.24 1.1094854472023
4.24 1.60201211276075
4.16 1.60201211276075
4.16 1.1094854472023
};
\addplot [line width=1.0pt, color2, opacity=1, forget plot]
table {%
4.2 1.1094854472023
4.2 0.447118964357664
};
\addplot [line width=1.0pt, color2, opacity=1, forget plot]
table {%
4.2 1.60201211276075
4.2 2.22487279032102
};
\addplot [line width=1.0pt, color2, forget plot]
table {%
4.18 0.447118964357664
4.22 0.447118964357664
};
\addplot [line width=1.0pt, color2, forget plot]
table {%
4.18 2.22487279032102
4.22 2.22487279032102
};
\addplot [line width=1.0pt, color2, opacity=0.2, mark=*, mark size=1, mark options={solid}, only marks, forget plot]
table {%
4.2 2.478817977441
4.2 2.41025834300307
4.2 2.36051762372747
};
\addplot [line width=1.0pt, color2, opacity=1, forget plot]
table {%
5.16 1.0170554930883
5.24 1.0170554930883
5.24 1.49793776650355
5.16 1.49793776650355
5.16 1.0170554930883
};
\addplot [line width=1.0pt, color2, opacity=1, forget plot]
table {%
5.2 1.0170554930883
5.2 0.440782478359975
};
\addplot [line width=1.0pt, color2, opacity=1, forget plot]
table {%
5.2 1.49793776650355
5.2 2.11768183484837
};
\addplot [line width=1.0pt, color2, forget plot]
table {%
5.18 0.440782478359975
5.22 0.440782478359975
};
\addplot [line width=1.0pt, color2, forget plot]
table {%
5.18 2.11768183484837
5.22 2.11768183484837
};
\addplot [line width=1.0pt, color2, opacity=0.2, mark=*, mark size=1, mark options={solid}, only marks, forget plot]
table {%
5.2 2.36280558656778
5.2 2.39411891982652
};
\addplot [line width=1.0pt, color2, opacity=1, forget plot]
table {%
6.16 0.866414189383766
6.24 0.866414189383766
6.24 1.2786879794083
6.16 1.2786879794083
6.16 0.866414189383766
};
\addplot [line width=1.0pt, color2, opacity=1, forget plot]
table {%
6.2 0.866414189383766
6.2 0.46235404656751
};
\addplot [line width=1.0pt, color2, opacity=1, forget plot]
table {%
6.2 1.2786879794083
6.2 1.84764853227329
};
\addplot [line width=1.0pt, color2, forget plot]
table {%
6.18 0.46235404656751
6.22 0.46235404656751
};
\addplot [line width=1.0pt, color2, forget plot]
table {%
6.18 1.84764853227329
6.22 1.84764853227329
};
\addplot [line width=1.0pt, color2, opacity=0.2, mark=*, mark size=1, mark options={solid}, only marks, forget plot]
table {%
6.2 2.08932864482867
6.2 2.07356162744851
6.2 2.18319171903564
6.2 1.9459233897936
};
\addplot [line width=1.0pt, black, opacity=1, forget plot]
table {%
0.76 0
0.84 0
};
\addplot [line width=1.0pt, black, dashed, mark=x, mark size=3, mark options={solid}, forget plot]
table {%
0.8 0.0011180339887499
};
\addplot [line width=1.0pt, black, opacity=1, forget plot]
table {%
1.76 0
1.84 0
};
\addplot [line width=1.0pt, black, dashed, mark=x, mark size=3, mark options={solid}, forget plot]
table {%
1.8 0.0573512605610783
};
\addplot [line width=1.0pt, black, opacity=1, forget plot]
table {%
2.76 0
2.84 0
};
\addplot [line width=1.0pt, black, dashed, mark=x, mark size=3, mark options={solid}, forget plot]
table {%
2.8 0.105530148254809
};
\addplot [line width=1.0pt, black, opacity=1, forget plot]
table {%
3.76 0
3.84 0
};
\addplot [line width=1.0pt, black, dashed, mark=x, mark size=3, mark options={solid}, forget plot]
table {%
3.8 0.153448583719417
};
\addplot [line width=1.0pt, black, opacity=1, forget plot]
table {%
4.76 0.106718737290548
4.84 0.106718737290548
};
\addplot [line width=1.0pt, black, dashed, mark=x, mark size=3, mark options={solid}, forget plot]
table {%
4.8 0.201532353164405
};
\addplot [line width=1.0pt, black, opacity=1, forget plot]
table {%
5.76 0.19452934323983
5.84 0.19452934323983
};
\addplot [line width=1.0pt, black, dashed, mark=x, mark size=3, mark options={solid}, forget plot]
table {%
5.8 0.250813553219818
};
\addplot [line width=1.0pt, color0, opacity=1, forget plot]
table {%
0.893333333333333 0.0499999999999998
0.973333333333333 0.0499999999999998
};
\addplot [line width=1.0pt, color0, dashed, mark=x, mark size=3, mark options={solid}, forget plot]
table {%
0.933333333333333 0.120473200244573
};
\addplot [line width=1.0pt, color0, opacity=1, forget plot]
table {%
1.89333333333333 0.0666666666666666
1.97333333333333 0.0666666666666666
};
\addplot [line width=1.0pt, color0, dashed, mark=x, mark size=3, mark options={solid}, forget plot]
table {%
1.93333333333333 0.151335163450336
};
\addplot [line width=1.0pt, color0, opacity=1, forget plot]
table {%
2.89333333333333 0.145773797371133
2.97333333333333 0.145773797371133
};
\addplot [line width=1.0pt, color0, dashed, mark=x, mark size=3, mark options={solid}, forget plot]
table {%
2.93333333333333 0.310475687463452
};
\addplot [line width=1.0pt, color0, opacity=1, forget plot]
table {%
3.89333333333333 0.159304230128149
3.97333333333333 0.159304230128149
};
\addplot [line width=1.0pt, color0, dashed, mark=x, mark size=3, mark options={solid}, forget plot]
table {%
3.93333333333333 0.318908245570073
};
\addplot [line width=1.0pt, color0, opacity=1, forget plot]
table {%
4.89333333333333 0.19805568721695
4.97333333333333 0.19805568721695
};
\addplot [line width=1.0pt, color0, dashed, mark=x, mark size=3, mark options={solid}, forget plot]
table {%
4.93333333333333 0.345423936350837
};
\addplot [line width=1.0pt, color0, opacity=1, forget plot]
table {%
5.89333333333333 0.202522591698032
5.97333333333333 0.202522591698032
};
\addplot [line width=1.0pt, color0, dashed, mark=x, mark size=3, mark options={solid}, forget plot]
table {%
5.93333333333333 0.326665919052159
};
\addplot [line width=1.0pt, color1, opacity=1, forget plot]
table {%
1.02666666666667 0.604861465775776
1.10666666666667 0.604861465775776
};
\addplot [line width=1.0pt, color1, dashed, mark=x, mark size=3, mark options={solid}, forget plot]
table {%
1.06666666666667 0.73760320776886
};
\addplot [line width=1.0pt, color1, opacity=1, forget plot]
table {%
2.02666666666667 1.03084531641706
2.10666666666667 1.03084531641706
};
\addplot [line width=1.0pt, color1, dashed, mark=x, mark size=3, mark options={solid}, forget plot]
table {%
2.06666666666667 1.03172321181079
};
\addplot [line width=1.0pt, color1, opacity=1, forget plot]
table {%
3.02666666666667 0.863414852226937
3.10666666666667 0.863414852226937
};
\addplot [line width=1.0pt, color1, dashed, mark=x, mark size=3, mark options={solid}, forget plot]
table {%
3.06666666666667 0.93104796123572
};
\addplot [line width=1.0pt, color1, opacity=1, forget plot]
table {%
4.02666666666667 0.795058300939824
4.10666666666667 0.795058300939824
};
\addplot [line width=1.0pt, color1, dashed, mark=x, mark size=3, mark options={solid}, forget plot]
table {%
4.06666666666667 0.817100830316506
};
\addplot [line width=1.0pt, color1, opacity=1, forget plot]
table {%
5.02666666666667 0.717854239335685
5.10666666666667 0.717854239335685
};
\addplot [line width=1.0pt, color1, dashed, mark=x, mark size=3, mark options={solid}, forget plot]
table {%
5.06666666666667 0.774625914265088
};
\addplot [line width=1.0pt, color1, opacity=1, forget plot]
table {%
6.02666666666667 0.679166041520125
6.10666666666667 0.679166041520125
};
\addplot [line width=1.0pt, color1, dashed, mark=x, mark size=3, mark options={solid}, forget plot]
table {%
6.06666666666667 0.693449862062485
};
\addplot [line width=1.0pt, color2, opacity=1, forget plot]
table {%
1.16 1.65356158124355
1.24 1.65356158124355
};
\addplot [line width=1.0pt, color2, dashed, mark=x, mark size=3, mark options={solid}, forget plot]
table {%
1.2 1.67616562997466
};
\addplot [line width=1.0pt, color2, opacity=1, forget plot]
table {%
2.16 1.48558333682021
2.24 1.48558333682021
};
\addplot [line width=1.0pt, color2, dashed, mark=x, mark size=3, mark options={solid}, forget plot]
table {%
2.2 1.53888843902515
};
\addplot [line width=1.0pt, color2, opacity=1, forget plot]
table {%
3.16 1.41451667743675
3.24 1.41451667743675
};
\addplot [line width=1.0pt, color2, dashed, mark=x, mark size=3, mark options={solid}, forget plot]
table {%
3.2 1.42360947660203
};
\addplot [line width=1.0pt, color2, opacity=1, forget plot]
table {%
4.16 1.37055188871675
4.24 1.37055188871675
};
\addplot [line width=1.0pt, color2, dashed, mark=x, mark size=3, mark options={solid}, forget plot]
table {%
4.2 1.35910161188734
};
\addplot [line width=1.0pt, color2, opacity=1, forget plot]
table {%
5.16 1.23788582083866
5.24 1.23788582083866
};
\addplot [line width=1.0pt, color2, dashed, mark=x, mark size=3, mark options={solid}, forget plot]
table {%
5.2 1.26016195805405
};
\addplot [line width=1.0pt, color2, opacity=1, forget plot]
table {%
6.16 1.05519901075963
6.24 1.05519901075963
};
\addplot [line width=1.0pt, color2, dashed, mark=x, mark size=3, mark options={solid}, forget plot]
table {%
6.2 1.08055919506246
};
\end{axis}

\node at ({$(current bounding box.south west)!0.5!(current bounding box.south east)$}|-{$(current bounding box.south west)!0.98!(current bounding box.north west)$})[
  anchor=north,
  text=black,
  rotate=0.0
]{ };

		\begin{customlegend}[
legend entries={best case,base,worst case,guessing},
legend cell align=left,
legend style={at={(10.5,5.37)}, anchor=north east, draw=white!80.0!black, font=\footnotesize}]
    \addlegendimage{area legend,gray,fill=gray}
    \addlegendimage{area legend,black,fill=black}
    \addlegendimage{area legend,color0,fill=color0}
    \addlegendimage{area legend,color1,fill=color1}
\end{customlegend}
	\end{tikzpicture}

	\caption[Evaluation results for varying \glsentryshort{em} iterations]{Evaluation results for varying \glsentryshort{em} iterations: }
	\label{fig:trial1}
\end{figure}

\begin{itemize}
	\item best case has almost optimal localisation for 2-4 sources (third quantile of datapoints still 0, mean>0 allows for conclusion, that few but big outliers are present)
	\item guessing gets increasingly better with number of sources, as the chance is better for each guess to be close to a source
	\item consistently, 75\% of worst case results are better than 25\% of guessed results
	\item also, conservative choice of base parameter set can be seen, especially for 2-4 sources.
	\item For more sources and less favorable conditions, the effect of the simulation parameters seems to diminish and the results behave very much like guessing results, just a little bit better
\end{itemize}
