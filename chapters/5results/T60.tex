\subsubsection{T60}
%% all-in-one boxplot
\begin{figure}[H]
    \setlength\figureheight{7cm}
    \small
    \setlength\figurewidth{\textwidth}
	\centering
	\begin{tikzpicture}
	    \footnotesize
	    % This file was created by matplotlib2tikz v0.6.14.
\definecolor{color0}{rgb}{0.8,0.207843137254902,0.219607843137255}
\definecolor{color1}{rgb}{0.67843137254902,0.847058823529412,0.901960784313726}

\begin{axis}[
xmin=0.5, xmax=6.5,
ymin=0, ymax=1.75,
width=\figurewidth,
height=\figureheight,
xtick={1,2,3,4,5,6},
xticklabels={2,3,4,5,6,7},
ytick={0,0.25,0.5,0.75,1,1.25,1.5,1.75},
minor xtick={},
minor ytick={},
tick align=outside,
tick pos=left,
x grid style={white!69.019607843137251!black},
ymajorgrids,
y grid style={white!69.019607843137251!black}
]
\addplot [line width=0.5pt, white!66.274509803921561!black, opacity=1, forget plot]
table {%
0.71 0
0.79 0
0.79 0
0.71 0
0.71 0
};
\addplot [line width=0.5pt, white!66.274509803921561!black, opacity=1, forget plot]
table {%
0.75 0
0.75 0
};
\addplot [line width=0.5pt, white!66.274509803921561!black, opacity=1, forget plot]
table {%
0.75 0
0.75 0
};
\addplot [line width=0.5pt, white!66.274509803921561!black, forget plot]
table {%
0.73 0
0.77 0
};
\addplot [line width=0.5pt, white!66.274509803921561!black, forget plot]
table {%
0.73 0
0.77 0
};
\addplot [line width=0.5pt, white!66.274509803921561!black, opacity=1, forget plot]
table {%
1.71 0
1.79 0
1.79 0
1.71 0
1.71 0
};
\addplot [line width=0.5pt, white!66.274509803921561!black, opacity=1, forget plot]
table {%
1.75 0
1.75 0
};
\addplot [line width=0.5pt, white!66.274509803921561!black, opacity=1, forget plot]
table {%
1.75 0
1.75 0
};
\addplot [line width=0.5pt, white!66.274509803921561!black, forget plot]
table {%
1.73 0
1.77 0
};
\addplot [line width=0.5pt, white!66.274509803921561!black, forget plot]
table {%
1.73 0
1.77 0
};
\addplot [line width=0.5pt, white!66.274509803921561!black, opacity=1, forget plot]
table {%
2.71 0
2.79 0
2.79 0.0926040864149498
2.71 0.0926040864149498
2.71 0
};
\addplot [line width=0.5pt, white!66.274509803921561!black, opacity=1, forget plot]
table {%
2.75 0
2.75 0
};
\addplot [line width=0.5pt, white!66.274509803921561!black, opacity=1, forget plot]
table {%
2.75 0.0926040864149498
2.75 0.227698396494843
};
\addplot [line width=0.5pt, white!66.274509803921561!black, forget plot]
table {%
2.73 0
2.77 0
};
\addplot [line width=0.5pt, white!66.274509803921561!black, forget plot]
table {%
2.73 0.227698396494843
2.77 0.227698396494843
};
\addplot [line width=0.5pt, white!66.274509803921561!black, opacity=1, forget plot]
table {%
3.71 0
3.79 0
3.79 0.206110649229432
3.71 0.206110649229432
3.71 0
};
\addplot [line width=0.5pt, white!66.274509803921561!black, opacity=1, forget plot]
table {%
3.75 0
3.75 0
};
\addplot [line width=0.5pt, white!66.274509803921561!black, opacity=1, forget plot]
table {%
3.75 0.206110649229432
3.75 0.495535624910617
};
\addplot [line width=0.5pt, white!66.274509803921561!black, forget plot]
table {%
3.73 0
3.77 0
};
\addplot [line width=0.5pt, white!66.274509803921561!black, forget plot]
table {%
3.73 0.495535624910617
3.77 0.495535624910617
};
\addplot [line width=0.5pt, white!66.274509803921561!black, opacity=1, forget plot]
table {%
4.71 0
4.79 0
4.79 0.42761016062282
4.71 0.42761016062282
4.71 0
};
\addplot [line width=0.5pt, white!66.274509803921561!black, opacity=1, forget plot]
table {%
4.75 0
4.75 0
};
\addplot [line width=0.5pt, white!66.274509803921561!black, opacity=1, forget plot]
table {%
4.75 0.42761016062282
4.75 1.06718737290547
};
\addplot [line width=0.5pt, white!66.274509803921561!black, forget plot]
table {%
4.73 0
4.77 0
};
\addplot [line width=0.5pt, white!66.274509803921561!black, forget plot]
table {%
4.73 1.06718737290547
4.77 1.06718737290547
};
\addplot [line width=0.5pt, white!66.274509803921561!black, opacity=1, forget plot]
table {%
5.71 0
5.79 0
5.79 0.4873386793776
5.71 0.4873386793776
5.71 0
};
\addplot [line width=0.5pt, white!66.274509803921561!black, opacity=1, forget plot]
table {%
5.75 0
5.75 0
};
\addplot [line width=0.5pt, white!66.274509803921561!black, opacity=1, forget plot]
table {%
5.75 0.4873386793776
5.75 1.20046287369127
};
\addplot [line width=0.5pt, white!66.274509803921561!black, forget plot]
table {%
5.73 0
5.77 0
};
\addplot [line width=0.5pt, white!66.274509803921561!black, forget plot]
table {%
5.73 1.20046287369127
5.77 1.20046287369127
};
\addplot [line width=0.5pt, black, opacity=1, forget plot]
table {%
0.835 0
0.915 0
0.915 0.15
0.835 0.15
0.835 0
};
\addplot [line width=0.5pt, black, opacity=1, forget plot]
table {%
0.875 0
0.875 0
};
\addplot [line width=0.5pt, black, opacity=1, forget plot]
table {%
0.875 0.15
0.875 0.370156211871642
};
\addplot [line width=0.5pt, black, forget plot]
table {%
0.855 0
0.895 0
};
\addplot [line width=0.5pt, black, forget plot]
table {%
0.855 0.370156211871642
0.895 0.370156211871642
};
\addplot [line width=0.5pt, black, opacity=1, forget plot]
table {%
1.835 0.0333333333333332
1.915 0.0333333333333332
1.915 0.182947725441579
1.835 0.182947725441579
1.835 0.0333333333333332
};
\addplot [line width=0.5pt, black, opacity=1, forget plot]
table {%
1.875 0.0333333333333332
1.875 0
};
\addplot [line width=0.5pt, black, opacity=1, forget plot]
table {%
1.875 0.182947725441579
1.875 0.405517502019881
};
\addplot [line width=0.5pt, black, forget plot]
table {%
1.855 0
1.895 0
};
\addplot [line width=0.5pt, black, forget plot]
table {%
1.855 0.405517502019881
1.895 0.405517502019881
};
\addplot [line width=0.5pt, black, opacity=1, forget plot]
table {%
2.835 0.0559016994374948
2.915 0.0559016994374948
2.915 0.406280514572352
2.835 0.406280514572352
2.835 0.0559016994374948
};
\addplot [line width=0.5pt, black, opacity=1, forget plot]
table {%
2.875 0.0559016994374948
2.875 0
};
\addplot [line width=0.5pt, black, opacity=1, forget plot]
table {%
2.875 0.406280514572352
2.875 0.922321906469108
};
\addplot [line width=0.5pt, black, forget plot]
table {%
2.855 0
2.895 0
};
\addplot [line width=0.5pt, black, forget plot]
table {%
2.855 0.922321906469108
2.895 0.922321906469108
};
\addplot [line width=0.5pt, black, opacity=1, forget plot]
table {%
3.835 0.0749999999999999
3.915 0.0749999999999999
3.915 0.528264655414455
3.835 0.528264655414455
3.835 0.0749999999999999
};
\addplot [line width=0.5pt, black, opacity=1, forget plot]
table {%
3.875 0.0749999999999999
3.875 0
};
\addplot [line width=0.5pt, black, opacity=1, forget plot]
table {%
3.875 0.528264655414455
3.875 1.17663204947805
};
\addplot [line width=0.5pt, black, forget plot]
table {%
3.855 0
3.895 0
};
\addplot [line width=0.5pt, black, forget plot]
table {%
3.855 1.17663204947805
3.895 1.17663204947805
};
\addplot [line width=0.5pt, black, opacity=1, forget plot]
table {%
4.835 0.105901699437495
4.915 0.105901699437495
4.915 0.612376945865933
4.835 0.612376945865933
4.835 0.105901699437495
};
\addplot [line width=0.5pt, black, opacity=1, forget plot]
table {%
4.875 0.105901699437495
4.875 0
};
\addplot [line width=0.5pt, black, opacity=1, forget plot]
table {%
4.875 0.612376945865933
4.875 1.35725600520391
};
\addplot [line width=0.5pt, black, forget plot]
table {%
4.855 0
4.895 0
};
\addplot [line width=0.5pt, black, forget plot]
table {%
4.855 1.35725600520391
4.895 1.35725600520391
};
\addplot [line width=0.5pt, black, opacity=1, forget plot]
table {%
5.835 0.123817286512515
5.915 0.123817286512515
5.915 0.535724526675181
5.835 0.535724526675181
5.835 0.123817286512515
};
\addplot [line width=0.5pt, black, opacity=1, forget plot]
table {%
5.875 0.123817286512515
5.875 0
};
\addplot [line width=0.5pt, black, opacity=1, forget plot]
table {%
5.875 0.535724526675181
5.875 1.1445928843406
};
\addplot [line width=0.5pt, black, forget plot]
table {%
5.855 0
5.895 0
};
\addplot [line width=0.5pt, black, forget plot]
table {%
5.855 1.1445928843406
5.895 1.1445928843406
};
\addplot [line width=0.5pt, color0, opacity=1, forget plot]
table {%
0.96 0
1.04 0
1.04 0.181954948688533
0.96 0.181954948688533
0.96 0
};
\addplot [line width=0.5pt, color0, opacity=1, forget plot]
table {%
1 0
1 0
};
\addplot [line width=0.5pt, color0, opacity=1, forget plot]
table {%
1 0.181954948688533
1 0.453112887414927
};
\addplot [line width=0.5pt, color0, forget plot]
table {%
0.98 0
1.02 0
};
\addplot [line width=0.5pt, color0, forget plot]
table {%
0.98 0.453112887414927
1.02 0.453112887414927
};
\addplot [line width=0.5pt, color0, opacity=1, forget plot]
table {%
1.96 0.0471404520791032
2.04 0.0471404520791032
2.04 0.317657035137148
1.96 0.317657035137148
1.96 0.0471404520791032
};
\addplot [line width=0.5pt, color0, opacity=1, forget plot]
table {%
2 0.0471404520791032
2 0
};
\addplot [line width=0.5pt, color0, opacity=1, forget plot]
table {%
2 0.317657035137148
2 0.690010572469092
};
\addplot [line width=0.5pt, color0, forget plot]
table {%
1.98 0
2.02 0
};
\addplot [line width=0.5pt, color0, forget plot]
table {%
1.98 0.690010572469092
2.02 0.690010572469092
};
\addplot [line width=0.5pt, color0, opacity=1, forget plot]
table {%
2.96 0.112455619297126
3.04 0.112455619297126
3.04 0.59504750370721
2.96 0.59504750370721
2.96 0.112455619297126
};
\addplot [line width=0.5pt, color0, opacity=1, forget plot]
table {%
3 0.112455619297126
3 0
};
\addplot [line width=0.5pt, color0, opacity=1, forget plot]
table {%
3 0.59504750370721
3 1.27410415417437
};
\addplot [line width=0.5pt, color0, forget plot]
table {%
2.98 0
3.02 0
};
\addplot [line width=0.5pt, color0, forget plot]
table {%
2.98 1.27410415417437
3.02 1.27410415417437
};
\addplot [line width=0.5pt, color0, opacity=1, forget plot]
table {%
3.96 0.165824424940892
4.04 0.165824424940892
4.04 0.731764919939242
3.96 0.731764919939242
3.96 0.165824424940892
};
\addplot [line width=0.5pt, color0, opacity=1, forget plot]
table {%
4 0.165824424940892
4 0.0199999999999999
};
\addplot [line width=0.5pt, color0, opacity=1, forget plot]
table {%
4 0.731764919939242
4 1.35587645886755
};
\addplot [line width=0.5pt, color0, forget plot]
table {%
3.98 0.0199999999999999
4.02 0.0199999999999999
};
\addplot [line width=0.5pt, color0, forget plot]
table {%
3.98 1.35587645886755
4.02 1.35587645886755
};
\addplot [line width=0.5pt, color0, opacity=1, forget plot]
table {%
4.96 0.344043814351852
5.04 0.344043814351852
5.04 0.841259160161866
4.96 0.841259160161866
4.96 0.344043814351852
};
\addplot [line width=0.5pt, color0, opacity=1, forget plot]
table {%
5 0.344043814351852
5 0.0333333333333333
};
\addplot [line width=0.5pt, color0, opacity=1, forget plot]
table {%
5 0.841259160161866
5 1.42418950732103
};
\addplot [line width=0.5pt, color0, forget plot]
table {%
4.98 0.0333333333333333
5.02 0.0333333333333333
};
\addplot [line width=0.5pt, color0, forget plot]
table {%
4.98 1.42418950732103
5.02 1.42418950732103
};
\addplot [line width=0.5pt, color0, opacity=1, forget plot]
table {%
5.96 0.34607418312141
6.04 0.34607418312141
6.04 0.721356838246698
5.96 0.721356838246698
5.96 0.34607418312141
};
\addplot [line width=0.5pt, color0, opacity=1, forget plot]
table {%
6 0.34607418312141
6 0.0693712943361397
};
\addplot [line width=0.5pt, color0, opacity=1, forget plot]
table {%
6 0.721356838246698
6 1.25767671358356
};
\addplot [line width=0.5pt, color0, forget plot]
table {%
5.98 0.0693712943361397
6.02 0.0693712943361397
};
\addplot [line width=0.5pt, color0, forget plot]
table {%
5.98 1.25767671358356
6.02 1.25767671358356
};
\addplot [line width=0.5pt, color1, opacity=1, forget plot]
table {%
1.085 0
1.165 0
1.165 0.212132034355964
1.085 0.212132034355964
1.085 0
};
\addplot [line width=0.5pt, color1, opacity=1, forget plot]
table {%
1.125 0
1.125 0
};
\addplot [line width=0.5pt, color1, opacity=1, forget plot]
table {%
1.125 0.212132034355964
1.125 0.52169905660283
};
\addplot [line width=0.5pt, color1, forget plot]
table {%
1.105 0
1.145 0
};
\addplot [line width=0.5pt, color1, forget plot]
table {%
1.105 0.52169905660283
1.145 0.52169905660283
};
\addplot [line width=0.5pt, color1, opacity=1, forget plot]
table {%
2.085 0.0666666666666667
2.165 0.0666666666666667
2.165 0.384644369142159
2.085 0.384644369142159
2.085 0.0666666666666667
};
\addplot [line width=0.5pt, color1, opacity=1, forget plot]
table {%
2.125 0.0666666666666667
2.125 0
};
\addplot [line width=0.5pt, color1, opacity=1, forget plot]
table {%
2.125 0.384644369142159
2.125 0.845053044125728
};
\addplot [line width=0.5pt, color1, forget plot]
table {%
2.105 0
2.145 0
};
\addplot [line width=0.5pt, color1, forget plot]
table {%
2.105 0.845053044125728
2.145 0.845053044125728
};
\addplot [line width=0.5pt, color1, opacity=1, forget plot]
table {%
3.085 0.144638347648318
3.165 0.144638347648318
3.165 0.745196840855918
3.085 0.745196840855918
3.085 0.144638347648318
};
\addplot [line width=0.5pt, color1, opacity=1, forget plot]
table {%
3.125 0.144638347648318
3.125 0
};
\addplot [line width=0.5pt, color1, opacity=1, forget plot]
table {%
3.125 0.745196840855918
3.125 1.61420708531568
};
\addplot [line width=0.5pt, color1, forget plot]
table {%
3.105 0
3.145 0
};
\addplot [line width=0.5pt, color1, forget plot]
table {%
3.105 1.61420708531568
3.145 1.61420708531568
};
\addplot [line width=0.5pt, color1, opacity=1, forget plot]
table {%
4.085 0.183209929917826
4.165 0.183209929917826
4.165 0.651012244216397
4.085 0.651012244216397
4.085 0.183209929917826
};
\addplot [line width=0.5pt, color1, opacity=1, forget plot]
table {%
4.125 0.183209929917826
4.125 0
};
\addplot [line width=0.5pt, color1, opacity=1, forget plot]
table {%
4.125 0.651012244216397
4.125 1.34306991356718
};
\addplot [line width=0.5pt, color1, forget plot]
table {%
4.105 0
4.145 0
};
\addplot [line width=0.5pt, color1, forget plot]
table {%
4.105 1.34306991356718
4.145 1.34306991356718
};
\addplot [line width=0.5pt, color1, opacity=1, forget plot]
table {%
5.085 0.236811407842327
5.165 0.236811407842327
5.165 0.676112785311496
5.085 0.676112785311496
5.085 0.236811407842327
};
\addplot [line width=0.5pt, color1, opacity=1, forget plot]
table {%
5.125 0.236811407842327
5.125 0.0249999999999999
};
\addplot [line width=0.5pt, color1, opacity=1, forget plot]
table {%
5.125 0.676112785311496
5.125 1.23480947595384
};
\addplot [line width=0.5pt, color1, forget plot]
table {%
5.105 0.0249999999999999
5.145 0.0249999999999999
};
\addplot [line width=0.5pt, color1, forget plot]
table {%
5.105 1.23480947595384
5.145 1.23480947595384
};
\addplot [line width=0.5pt, color1, opacity=1, forget plot]
table {%
6.085 0.219654127461438
6.165 0.219654127461438
6.165 0.666298251680331
6.085 0.666298251680331
6.085 0.219654127461438
};
\addplot [line width=0.5pt, color1, opacity=1, forget plot]
table {%
6.125 0.219654127461438
6.125 0
};
\addplot [line width=0.5pt, color1, opacity=1, forget plot]
table {%
6.125 0.666298251680331
6.125 1.30704505513979
};
\addplot [line width=0.5pt, color1, forget plot]
table {%
6.105 0
6.145 0
};
\addplot [line width=0.5pt, color1, forget plot]
table {%
6.105 1.30704505513979
6.145 1.30704505513979
};
\addplot [line width=0.5pt, white!66.274509803921561!black, opacity=1, forget plot]
table {%
0.71 0
0.79 0
};
\addplot [line width=0.5pt, white!66.274509803921561!black, dashed, mark=x, mark size=3, mark options={solid}, forget plot]
table {%
0.75 0.034787834432853
};
\addplot [line width=0.5pt, white!66.274509803921561!black, opacity=1, forget plot]
table {%
1.71 0
1.79 0
};
\addplot [line width=0.5pt, white!66.274509803921561!black, dashed, mark=x, mark size=3, mark options={solid}, forget plot]
table {%
1.75 0.0786009843334433
};
\addplot [line width=0.5pt, white!66.274509803921561!black, opacity=1, forget plot]
table {%
2.71 0
2.79 0
};
\addplot [line width=0.5pt, white!66.274509803921561!black, dashed, mark=x, mark size=3, mark options={solid}, forget plot]
table {%
2.75 0.125181952915824
};
\addplot [line width=0.5pt, white!66.274509803921561!black, opacity=1, forget plot]
table {%
3.71 0.0282842712474619
3.79 0.0282842712474619
};
\addplot [line width=0.5pt, white!66.274509803921561!black, dashed, mark=x, mark size=3, mark options={solid}, forget plot]
table {%
3.75 0.181782082014763
};
\addplot [line width=0.5pt, white!66.274509803921561!black, opacity=1, forget plot]
table {%
4.71 0.0833333333333333
4.79 0.0833333333333333
};
\addplot [line width=0.5pt, white!66.274509803921561!black, dashed, mark=x, mark size=3, mark options={solid}, forget plot]
table {%
4.75 0.247795291898826
};
\addplot [line width=0.5pt, white!66.274509803921561!black, opacity=1, forget plot]
table {%
5.71 0.107277838166405
5.79 0.107277838166405
};
\addplot [line width=0.5pt, white!66.274509803921561!black, dashed, mark=x, mark size=3, mark options={solid}, forget plot]
table {%
5.75 0.302171324741248
};
\addplot [line width=0.5pt, black, opacity=1, forget plot]
table {%
0.835 0.0499999999999998
0.915 0.0499999999999998
};
\addplot [line width=0.5pt, black, dashed, mark=x, mark size=3, mark options={solid}, forget plot]
table {%
0.875 0.115348204443546
};
\addplot [line width=0.5pt, black, opacity=1, forget plot]
table {%
1.835 0.0706011329583298
1.915 0.0706011329583298
};
\addplot [line width=0.5pt, black, dashed, mark=x, mark size=3, mark options={solid}, forget plot]
table {%
1.875 0.181073337918437
};
\addplot [line width=0.5pt, black, opacity=1, forget plot]
table {%
2.835 0.135881019908347
2.915 0.135881019908347
};
\addplot [line width=0.5pt, black, dashed, mark=x, mark size=3, mark options={solid}, forget plot]
table {%
2.875 0.286969420037239
};
\addplot [line width=0.5pt, black, opacity=1, forget plot]
table {%
3.835 0.187160966652941
3.915 0.187160966652941
};
\addplot [line width=0.5pt, black, dashed, mark=x, mark size=3, mark options={solid}, forget plot]
table {%
3.875 0.334854777312092
};
\addplot [line width=0.5pt, black, opacity=1, forget plot]
table {%
4.835 0.250620510183685
4.915 0.250620510183685
};
\addplot [line width=0.5pt, black, dashed, mark=x, mark size=3, mark options={solid}, forget plot]
table {%
4.875 0.386834252013528
};
\addplot [line width=0.5pt, black, opacity=1, forget plot]
table {%
5.835 0.257596235863888
5.915 0.257596235863888
};
\addplot [line width=0.5pt, black, dashed, mark=x, mark size=3, mark options={solid}, forget plot]
table {%
5.875 0.35681272761367
};
\addplot [line width=0.5pt, color0, opacity=1, forget plot]
table {%
0.96 0.05
1.04 0.05
};
\addplot [line width=0.5pt, color0, dashed, mark=x, mark size=3, mark options={solid}, forget plot]
table {%
1 0.150865604905422
};
\addplot [line width=0.5pt, color0, opacity=1, forget plot]
table {%
1.96 0.138089721429767
2.04 0.138089721429767
};
\addplot [line width=0.5pt, color0, dashed, mark=x, mark size=3, mark options={solid}, forget plot]
table {%
2 0.286361863503465
};
\addplot [line width=0.5pt, color0, opacity=1, forget plot]
table {%
2.96 0.249303398874989
3.04 0.249303398874989
};
\addplot [line width=0.5pt, color0, dashed, mark=x, mark size=3, mark options={solid}, forget plot]
table {%
3 0.399925417717139
};
\addplot [line width=0.5pt, color0, opacity=1, forget plot]
table {%
3.96 0.509163801712671
4.04 0.509163801712671
};
\addplot [line width=0.5pt, color0, dashed, mark=x, mark size=3, mark options={solid}, forget plot]
table {%
4 0.505204259063092
};
\addplot [line width=0.5pt, color0, opacity=1, forget plot]
table {%
4.96 0.589301691045092
5.04 0.589301691045092
};
\addplot [line width=0.5pt, color0, dashed, mark=x, mark size=3, mark options={solid}, forget plot]
table {%
5 0.596819835538899
};
\addplot [line width=0.5pt, color0, opacity=1, forget plot]
table {%
5.96 0.532369723552282
6.04 0.532369723552282
};
\addplot [line width=0.5pt, color0, dashed, mark=x, mark size=3, mark options={solid}, forget plot]
table {%
6 0.556488346997599
};
\addplot [line width=0.5pt, color1, opacity=1, forget plot]
table {%
1.085 0.0500000000000003
1.165 0.0500000000000003
};
\addplot [line width=0.5pt, color1, dashed, mark=x, mark size=3, mark options={solid}, forget plot]
table {%
1.125 0.182160501287364
};
\addplot [line width=0.5pt, color1, opacity=1, forget plot]
table {%
2.085 0.1535183758488
2.165 0.1535183758488
};
\addplot [line width=0.5pt, color1, dashed, mark=x, mark size=3, mark options={solid}, forget plot]
table {%
2.125 0.304725681986841
};
\addplot [line width=0.5pt, color1, opacity=1, forget plot]
table {%
3.085 0.333454681080129
3.165 0.333454681080129
};
\addplot [line width=0.5pt, color1, dashed, mark=x, mark size=3, mark options={solid}, forget plot]
table {%
3.125 0.492687442347053
};
\addplot [line width=0.5pt, color1, opacity=1, forget plot]
table {%
4.085 0.387116909620689
4.165 0.387116909620689
};
\addplot [line width=0.5pt, color1, dashed, mark=x, mark size=3, mark options={solid}, forget plot]
table {%
4.125 0.464558999324218
};
\addplot [line width=0.5pt, color1, opacity=1, forget plot]
table {%
5.085 0.412594758702181
5.165 0.412594758702181
};
\addplot [line width=0.5pt, color1, dashed, mark=x, mark size=3, mark options={solid}, forget plot]
table {%
5.125 0.492752298520933
};
\addplot [line width=0.5pt, color1, opacity=1, forget plot]
table {%
6.085 0.419588509824339
6.165 0.419588509824339
};
\addplot [line width=0.5pt, color1, dashed, mark=x, mark size=3, mark options={solid}, forget plot]
table {%
6.125 0.483904201436663
};
\end{axis}

\node at ({$(current bounding box.south west)!0.5!(current bounding box.south east)$}|-{$(current bounding box.south west)!0.98!(current bounding box.north west)$})[
  anchor=north,
  text=black,
  rotate=0.0
]{ };

	    \begin{customlegend}[
legend entries={T$_{60}=0.0$s,T$_{60}=0.3$s,T$_{60}=0.6$s,T$_{60}=0.9$s},
legend cell align=left,
legend style={at={(0.05,5.37)}, anchor=north west, draw=white!80.0!black, font=\footnotesize,fill opacity=0.5, draw opacity=1,text opacity=1}]
    \addlegendimage{area legend,black,fill=black, fill opacity=1}
    \addlegendimage{area legend,color0,fill=color0, fill opacity=1}
    \addlegendimage{area legend,color1,fill=color1, fill opacity=1}
    \addlegendimage{area legend,color2,fill=color2, fill opacity=1}
\end{customlegend}
	    
	\end{tikzpicture}
	
	\caption{\boxplotDescription for Different \glsentryshort{em} Iterations}
	\label{fig:trial1}
\end{figure}

\begin{itemize}
    \item Longer T60 increases mean error and therefore decreased localisation performance
    \item Interestingly, T60=0.9s yielded better localisation than T60=0.6s for 5-7 sources
    \item The variance of the results increases with T60 for 2-4 sources. After that, the variance of the results is high regardless of T60, which could mean, that the sparsity assumption does not hold anymore and the effect the number of sources has on the localisation error is greater than the effect the reverberation time has.
\end{itemize}

\begin{figure}[H]
    \centering
    \begin{subfigure}{0.49\textwidth}
          \centering
	       \includestandalone[width=\textwidth]{data/plots/plot_T60_err-mean}
	\end{subfigure}
    \begin{subfigure}{0.49\textwidth}
          \centering
	       \includestandalone[width=\textwidth]{data/plots/plot_T60_percent-matched}
%            % This file was created by matplotlib2tikz v0.6.13.
\begin{tikzpicture}

\definecolor{color0}{rgb}{0.8,0.207843137254902,0.219607843137255}
\definecolor{color1}{rgb}{1,0.647058823529412,0}
\definecolor{color2}{rgb}{1,0,1}

\begin{axis}[
title={err-mean},
xlabel={number of sources},
xmin=2, xmax=7,
ymin=0, ymax=1,
width=\figurewidth,
height=\figureheight,
tick align=outside,
tick pos=left,
x grid style={lightgray!92.026143790849673!black},
ymajorgrids,
y grid style={lightgray!92.026143790849673!black},
legend entries={{1.0},{2.0},{3.0},{5.0},{10.0},{20.0}},
legend style={at={(0.97,0.03)}, anchor=south east, draw=white!80.0!black},
legend cell align={left}
]
\addlegendimage{no markers, color0}
\addlegendimage{no markers, color1}
\addlegendimage{no markers, black}
\addlegendimage{no markers, lightgray!88.366013071895424!black}
\addlegendimage{no markers, blue}
\addlegendimage{no markers, color2}
\addplot [semithick, color0, mark=*, mark size=3, mark options={solid}]
table {%
2 0.290719260974231
3 0.479301880051738
4 0.644171288750938
5 0.744992267692781
6 0.815194102350153
7 0.80761917101706
};
\addplot [semithick, color1, mark=*, mark size=3, mark options={solid}]
table {%
2 0.245694149597362
3 0.421649843327725
4 0.500562120786198
5 0.655687681534857
6 0.633687481891525
7 0.793106398964979
};
\addplot [semithick, black, mark=*, mark size=3, mark options={solid}]
table {%
2 0.244192685094324
3 0.318939976510832
4 0.436516755504563
5 0.540639551860509
6 0.618377832825832
7 0.708533647793283
};
\addplot [semithick, lightgray!88.366013071895424!black, mark=*, mark size=3, mark options={solid}]
table {%
2 0.151357304870224
3 0.298658572932984
4 0.409077232118482
5 0.502185647164528
6 0.601037546098582
7 0.635845669023758
};
\addplot [semithick, blue, mark=*, mark size=3, mark options={solid}]
table {%
2 0.156145785073103
3 0.262900994349871
4 0.444291990749065
5 0.502497456459568
6 0.536465373800886
7 0.635142635271785
};
\addplot [semithick, color2, mark=*, mark size=3, mark options={solid}]
table {%
2 0.158641822221236
3 0.292719119608
4 0.407112096408886
5 0.458557818271804
6 0.545724599411615
7 0.603303151192354
};
\end{axis}

\end{tikzpicture}
	\end{subfigure}
\caption{Evaluation of T60 (EM-Iterations=10, SNR=0, refl-ord=-1, n=100)}
\end{figure}