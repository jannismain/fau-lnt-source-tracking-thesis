The task of acoustic source localisation and tracking has both a long history in the signal processing literature as well as many modern applications. In this thesis, the \glsentrylong{em} algorithm is utilised to solve this task, estimating the parameters of a \glsentrylong{gmm} which models the spatial information available from the received signals at an acoustical sensor array. Recursive variations of the algorithm for source localisation are presented to track moving speakers over time. Both Source Localisation and Tracking are tested in a simulation setup, where the algorithm's performance is evaluated in different scenarios. These evaluations show that localisation performance strongly depends on the conditions and is not reliable for highly reverberant and noisy environments with this method. Two versions of the source tracking algorithm, \glsentryshort{crem} and \glsentryshort{trem}, have been implemented and evaluated. These perform equally well across all evaluation scenarios. In general, these algorithms were able to track sources in mildly reverberant conditions. Also, high memory requirements and computational complexity are an issue for many practical applications.