\chaptermark{List of Symbols}
\sectionmark{List of Symbols}
\chapter*{List of Symbols}
\addcontentsline{toc}{chapter}{List of Symbols}

\subsubsection*{Notational Remarks}
The following notational conventions are followed throughout this thesis: A lower case regular letter indicates scalars, a lower case bold letter indicates vectors, and a capital bold letter indicates matrices.

%Positions in the cartesian coordinate system are described by their location vector $\bm p_{\text{index}}=[x, y, z]$ and are used for receiver locations $\bm p_m^i$ and source locations $\bm p_s$. The set of all possible positions is described by $\pall$.

%Sources are addressed by their index $s=1,\dots,S$ and receivers are addressed by their receiver pair $m\in[1, \dots, M]$ and pair index $i\in[1, 2]$. A braced superscript $l$, like $\psi^{(l)}$, denotes the iteration the \gls{em} algorithm is currently in, so $\psi^{(l-1)}$ indicates the value of $\psi$ of the preceding iteration. As recursions will be iterated with respects to the time-index, superscript $(t)$ (e.g. $\psi^{(t)}$) is used respectively. 
The diag$(\cdot)$ operator describes a diagonal matrix, where the elements in braces are placed on the diagonal, whereas all other entries of the matrix are equal to 0.


%% DEFINITIONS %%

% Single Symbols
\newcommand{\prp}{\bm\phi}

\newcommand{\p}{\bm p}
\newcommand{\ps}{\p_s}
\newcommand{\pr}{\p_m^i}
\newcommand{\pall}{\mathcal{P}}
\newcommand{\pinp}{\p \in \pall}

\newcommand{\psip}{\bm\psi_{\bm p}}
\newcommand{\psiRlast}{\bm\psi_R^{(t-1)}}
\newcommand{\psiRPlast}{\psi_{\bm p,R}^{(t-1)}}
\newcommand{\psiRnow}{\bm\psi_R^{(t)}}
\newcommand{\psiRPnow}{\psi_{\bm p,R}^{(t)}}
\newcommand{\psips}{\psi_{\bm p,s}}

% Functions
\newcommand{\gaussian}[1]{\mathcal{N}\left (#1\right )}
\newcommand{\pdffunc}{\psi_{\bm p}^{(l-1)}\prod_{m}\mathcal{N}^c\left (\phi^k_m(t,k);\tilde\phi^k_m(\bm p),\sigma^{2,(l-1)}\right )}
\newcommand{\pdffuncR}{\psi_{\bm p,R}^{(t-1)}\prod_{m}\mathcal{N}^c\left (\phi^k_m(t,k);\tilde\phi^k_m(\bm p),\sigma_R^{2,(t-1)}\right )}
\newcommand{\gauss}{\mathcal{N}^c\big(\phi_m(t,k);\tilde\phi^k_m,\sigma^2\big)}
\newcommand{\Q}{\mathcal{Q}\left (\theta\vert\theta^{(l-1)}\right )}
\newcommand{\mulast}{h^{(l)}(t,k,\bm p)}
\newcommand{\muRlast}{h(t,k,\bm p)}
\newcommand{\z}{z(t,k,\bm p)}
\newcommand{\lcompl}{\prod_{t,k}\sum_{\vect{p}}\psip\cdot\z\prod_{m}\gauss}

% Other
\newcommand{\vect}[1]{\mathbf{#1}}
\newcommand{\norm}[1]{|{#1}|_2}
\newcommand{\abs}[1]{|{#1}|_1}
\newcommand{\Tsixty}{T$_{60}$\ }

%% TABLE %%

\begin{longtable*}[l]{ll}
	\multicolumn{2}{l}{\textbf{Mathematical Operators}} \\[2pt]
	$|\cdot |$    & Absolute value (when applied to scalar)\\
	$|\cdot |$    & Cardinality (when applied to set)\\
	$\|\cdot \|$  & Length of vector (when applied to vector)\\
	$\|\cdot \|$  & Eucledian distance (when applied to difference of two vectors)\\
	det$(\cdot)$  & Determinant of matrix\\
	diag$(\cdot)$ & Diagonal matrix of vector\\

	\multicolumn{2}{l}{\textbf{Indices}} \\[2pt]
	$t$         & Time-bin index \\
	$k$         & Frequency-bin index \\
	$i$         & Receiver index \\
	$m$         & Sensor node index \\
	$s$         & Source index \\
	$l$         & Algorithm iteration index \\[6pt]

	\multicolumn{2}{l}{\textbf{Scalars}} \\[2pt]
	$T$         & Number of time-bins \\
	$K$         & Number of frequency-bins \\
	$M$         & Number of sensor nodes \\
	$S$         & Number of sources \\
	$L$         & Number of Iterations \\[6pt]

    \multicolumn{2}{l}{\textbf{Vectors}} \\[2pt]
	$\p $      & Location vector of grid point\\
	$\pall $      & Set of all grid points $\p$\\
	$\ps $      & Location vector of source $s$ \\
	$\pr $      & Location vector of receiver $i$ of sensor node $m$ \\

	$\prp$      & Vector of pair-wise relative phase ratios \\
	$\prp_m$      & Pair-wise relative phase ratio at sensor node $m$ \\
	$\psi$      & Weight of Gaussian mixture component \\
	$\psip$      & Vector of weights of Gaussian mixture components \\
\end{longtable*}
